\chapter{Discussion}
\label{chap:discussionfuture}

\section{Review and possible future directions}
We point out the most favorable and biggest flaws of the design and
implementation of Slope. We also discuss possible directions for Slope
and the migratable model going forward.

\paragraph{Handling failures:}
Before dealing with failures, we need to specify what failures mean for the
    systems that will use Slope. Depending on the type of application,
    this could be a non-issue (e.g. in data pipelines, where tasks are
    usually stateless and can easily be restarted) or a deal-breaker.
    In both cases, the notion of migrating runnable objects discussed in
    \autoref{subsec:runnable} can be a starting point for approaching failures.
    For example migrating or rerunning objects can be though of as natural
    points of synchronization or taking snapshots.


\paragraph{Routing and discovery:}
It is unlikely that Slope can be used without an extra multiplexing
layer on top. In listen and serve applications, a router is required to deliver
a request to the correct partition and in the state machine model,
given the heterogeneity of the nodes, a service discovery layer is
required to help nodes find appropriate migration destinations.

Given the
performance requirements of Slope, it might make more sense to build such
a system as part of Slope than to use an external service. Not only the
performance of Slope will be bottlenecked by the external routing layer, but
also such an application may benefit from running on \emph{top} of Slope.
Unreliable broadcasts over Infiniband and an Arp-like protocol sounds suitable
if this is to be done inside Slope in a leaderless fashion.



\paragraph{Dynamic resizing of the cluster:}
This is easier to achieve than the other requirements, but is still a required
engineering effort before we are able use Slope in production environments.
The machines must coordinate to make the shared address space reusable, as we
might add or remove machines indefinitely. We must also make minor changes to
the discovery protocol and possibly extract some of it to point to point
communication between the nodes.



\paragraph{Handling updates to Slope and application code:} Another technical
debt in the design of Slope lies in defining a procedure for updating Slope and
the applications that run on top of it such that the successive versions of
application binaries can coexist in the same cluster during the update period.
At least the core
functionality of Slope has to be backwards compatible with its previous versions.
On The applications side, system designers need to come up with an internal versioning
scheme to prevent objects from migrating back to the machines running older
binaries.

A much more complicated scenario
is one in which we make updates not to the internal functionalities, but to the
data members of the migratable objects. For example one might try to start by
providing functions that can translate an object in the pre-update format to an
object
in the current format, but it is not clear how we can reference the members of the
pre-update object while we only have code defining the current version of that type.
We might be able to get around this by limiting how migratable object schemes can be
updated similar to how this is done in Google's Protocol Buffers \cite{google:protobuf}.


\paragraph{Partial migration:}
With a black box view of the objects, we miss the cases where dividing the
algorithm or data structure into self-contained parts is not natural.
Tree-based data structures where subtrees typically resemble the full data
structure are good examples. With the current model, it is not
clear how one might approach the need for distributing subtrees across the
cluster arbitrarily. This motivates a search for a more capable ownership model
from one side, and tweaking the designs of data structures to make them
migration-friendly from the other.

\paragraph{Fixed address space:}
The fixed address space requirement is currently fundamental. A possible next
step would be coming up with a framework for creating \emph{relocatable}
objects, ones that can be placed at any base address in memory. We need
to define how relocatable objects compose, because upon relocating an object,
we most likely need to recursively relocate its members too. In addition to
that, we need to enforce a limited programming model to prevent direct reliance
on virtual memory addresses, as discussed in \autoref{sec:fixedfundamental}.

Another way to mitigate this is by allowing only ``named'' accesses to
resources, similar to how objects are accessed in Python (except the code
boundary of calling into external C functions). This way, we are effectively
wrapping the application's
virtual address space in another layer of virtual addressing provided by the
language runtime.



\section{Conclusion}

We identified a family of distributed applications and hypothesized that
their limited resource access pattern might call for a specialized transport
that they can benefit from. We then came up with an object-based migration
scheme which brings locality, load balancing, and easy of programming to the
above applications all at the same time.

Based on the above requirements, we designed Slope, provided its low level
specifications, and implemented it in C++ over RDMA. We pinpointed metrics that
we deem central to the performance of such a system and designed benchmarks to
measure them in environments which resemble those of real world applications.

We pointed out strengths, weaknesses and various possible future directions for
Slope or other systems working on a similar problem. We think of Slope as one
step towards a ubiquitous solution for making high performance computing
environments where specialized computing hardware can be added and used easily.

