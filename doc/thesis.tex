%%ut-thesis.tex -- document template for graduate theses at UofT
%%
%% Copyright (c) 1998-2013 Francois Pitt <fpitt@cs.utoronto.ca>
%% last updated at 16:20 (EDT) on Wed 25 Sep 2013
%%
%% This work may be distributed and/or modified under the conditions of
%% the LaTeX Project Public License, either version 1.3c of this license
%% or (at your option) any later version.
%% The latest version of this license is in
%%     http://www.latex-project.org/lppl.txt
%% and version 1.3c or later is part of all distributions of LaTeX
%% version 2005/12/01 or later.
%%
%% This work has the LPPL maintenance status "maintained".
%%
%% The Current Maintainer of this work is
%% Francois Pitt <fpitt@cs.utoronto.ca>.
%%
%% This work consists of the files listed in the accompanying README.

%% SUMMARY OF FEATURES:
%%
%% All environments, commands, and options provided by the `ut-thesis'
%% class will be described below, at the point where they should appear
%% in the document.  See the file `ut-thesis.cls' for more details.
%%
%% To explicitly set the pagestyle of any blank page inserted with
%% \cleardoublepage, use one of \clearemptydoublepage,
%% \clearplaindoublepage, \clearthesisdoublepage, or
%% \clearstandarddoublepage (to use the style currently in effect).
%%
%% For single-spaced quotes or quotations, use the `longquote' and
%% `longquotation' environments.


%%%%%%%%%%%%         PREAMBLE         %%%%%%%%%%%%

%%  - Default settings format a final copy (single-sided, normal
%%    margins, one-and-a-half-spaced with single-spaced notes).
%%  - For a rough copy (double-sided, normal margins, double-spaced,
%%    with the word "DRAFT" printed at each corner of every page), use
%%    the `draft' option.
%%  - The default global line spacing can be changed with one of the
%%    options `singlespaced', `onehalfspaced', or `doublespaced'.
%%  - Footnotes and marginal notes are all single-spaced by default, but
%%    can be made to have the same spacing as the rest of the document
%%    by using the option `standardspacednotes'.
%%  - The size of the margins can be changed with one of the options:
%%     . `narrowmargins' (1 1/4" left, 3/4" others),
%%     . `normalmargins' (1 1/4" left, 1" others),
%%     . `widemargins' (1 1/4" all),
%%     . `extrawidemargins' (1 1/2" all).
%%  - The pagestyle of "cleared" pages (empty pages inserted in
%%    two-sided documents to put the next page on the right-hand side)
%%    can be set with one of the options `cleardoublepagestyleempty',
%%    `cleardoublepagestyleplain', or `cleardoublepagestylestandard'.
%%  - Any other standard option for the `report' document class can be
%%    used to override the default or draft settings (such as `10pt',
%%    `11pt', `12pt'), and standard LaTeX packages can be used to
%%    further customize the layout and/or formatting of the document.

%% *** Add any desired options. ***
\documentclass{ut-thesis}

\pdfinfoomitdate 1
\pdfsuppressptexinfo-1
\pdftrailerid{}


\usepackage{xcolor}
\usepackage{enumitem}
\usepackage{float}
\usepackage{pgfplots}
\usepackage{mwe}
\pgfplotsset{compat=newest}
\usepackage[multidot]{grffile}
%% *** Add \usepackage declarations here. ***
%% The standard packages `geometry' and `setspace' are already loaded by
%% `ut-thesis' -- see their documentation for details of the features
%% they provide.  In particular, you may use the \geometry command here
%% to adjust the margins if none of the ut-thesis options are suitable
%% (see the `geometry' package for details).  You may also use the
%% \setstretch command to set the line spacing to a value other than
%% single, one-and-a-half, or double spaced (see the `setspace' package
%% for details).


%%%%%%%%%%%%%%%%%%%%%%%%%%%%%%%%%%%%%%%%%%%%%%%%%%%%%%%%%%%%%%%%%%%%%%%%
%%                                                                    %%
%%                   ***   I M P O R T A N T   ***                    %%
%%                                                                    %%
%%  Fill in the following fields with the required information:       %%
%%   - \degree{...}       name of the degree obtained                 %%
%%   - \department{...}   name of the graduate department             %%
%%   - \gradyear{...}     year of graduation                          %%
%%   - \author{...}       name of the author                          %%
%%   - \title{...}        title of the thesis                         %%
%%%%%%%%%%%%%%%%%%%%%%%%%%%%%%%%%%%%%%%%%%%%%%%%%%%%%%%%%%%%%%%%%%%%%%%%

%% *** Change this example to appropriate values. ***
\degree{Master of Science}
\department{Computer Science}
\gradyear{2020}
\author{Mohammad Nasirifar}
\title{Black box migration of data structures over RDMA}

%% *** NOTE ***
%% Put here all other formatting commands that belong in the preamble.
%% In particular, you should put all of your \newcommand's,
%% \newenvironment's, \newtheorem's, etc. (in other words, all the
%% global definitions that you will need throughout your thesis) in a
%% separate file and use "\input{filename}" to input it here.
\newcommand\TODO[1]{\textcolor{red}{(#1)}}

\newcommand{\includesvg}[1]{%
    \immediate\write18{inkscape -D --export-type=pdf --export-latex -o #1.pdf #1.svg}%
    \input{#1.pdf_tex}%
}

%% *** Adjust the following settings as desired. ***

%% List only down to subsections in the table of contents;
%% 0=chapter, 1=section, 2=subsection, 3=subsubsection, etc.
\setcounter{tocdepth}{2}

%% Make each page fill up the entire page.
\flushbottom


%%%%%%%%%%%%      MAIN  DOCUMENT      %%%%%%%%%%%%

\begin{document}

%% This sets the page style and numbering for preliminary sections.
\begin{preliminary}

%% This generates the title page from the information given above.
\maketitle

%% There should be NOTHING between the title page and abstract.
%% However, if your document is two-sided and you want the abstract
%% _not_ to appear on the back of the title page, then uncomment the
%% following line.
%\cleardoublepage

%% This generates the abstract page, with the line spacing adjusted
%% according to SGS guidelines.

\begin{abstract}

%% *** Put your Abstract here. ***
    Load imbalance and data locality are important \TODO{how important,
    based on what} performance factor in software systems \TODO{more specific?}.
    Moving in-memory resources across machines can directly impact the
    above metrics, however changing \TODO{better word?} the designs of
    data~structures to make them migratable \TODO{a bit complicated} is not only
    a non-trivial task but also can incur
    performance penalties resulting from serialization and de-serialization and
    increased CPU usage.

    In this thesis we introduce Slope, a plugable solution for making already
    existing data~structures migratable. Our approach eliminates data
    serialization and can be applied to already existing data~structures. We
    use RDMA as the transport which plays naturally with the design of Slope \TODO{this says nothing.}
    Slope can be incorporated to already existing performance-critical systems
    like transaction processing systems to provide them means of load balancing
    their hot data~structures.\TODO{more concrete}
    We evaluate on blah \TODO{incomplete}.

% TODO:
\TODO{remove these before final build: pdfinfoomitdate 1 pdfsuppressptexinfo-1 pdftrailerid
}
\end{abstract}

%% Anything placed between the abstract and table of contents will
%% appear on a separate page since the abstract ends with \newpage and
%% the table of contents starts with \clearpage.  Use \cleardoublepage
%% for anything that you want to appear on a right-hand page.

%% This generates a "dedication" section, if needed -- just a paragraph
%% formatted flush right (uncomment to have it appear in the document).
%\begin{dedication}
%% *** Put your Dedication here. ***
%\end{dedication}

%% The `dedication' and `acknowledgements' sections do not create new
%% pages so if you want the two sections to appear on separate pages,
%% uncomment the following line.
%\newpage  % separate pages for dedication and acknowledgements

%% Alternatively, if you leave both on the same page, it is probably a
%% good idea to add a bit of extra vertical space in between the two --
%% for example, as follows (adjust as desired).
%\vspace{.5in}  % vertical space between dedication and acknowledgements

%% This generates an "acknowledgements" section, if needed
%% (uncomment to have it appear in the document).
\begin{acknowledgements}
    Thank you Angela for never yelling at a very yellabel me.
\end{acknowledgements}

%% This generates the Table of Contents (on a separate page).
\tableofcontents

%% This generates the List of Tables (on a separate page), if needed
%% (uncomment to have it appear in the document).
%\listoftables

%% This generates the List of Figures (on a separate page), if needed
%% (uncomment to have it appear in the document).
%\listoffigures

%% You can add commands here to generate any other material that belongs
%% in the head matter (for example, List of Plates, Index of Symbols, or
%% List of Appendices).

%% End of the preliminary sections: reset page style and numbering.
\end{preliminary}


%%%%%%%%%%%%%%%%%%%%%%%%%%%%%%%%%%%%%%%%%%%%%%%%%%%%%%%%%%%%%%%%%%%%%%%%
%%  Put your Chapters here; the easiest way to do this is to keep     %%
%%  each chapter in a separate file and `\include' all the files.     %%
%%  Each chapter file should start with "\chapter{ChapterName}".      %%
%%  Note that using `\include' instead of `\input' will make each     %%
%%  chapter start on a new page, and allow you to format only parts   %%
%%  of your thesis at a time by using `\includeonly'.                 %%
%%%%%%%%%%%%%%%%%%%%%%%%%%%%%%%%%%%%%%%%%%%%%%%%%%%%%%%%%%%%%%%%%%%%%%%%

%% *** Include chapter files here. ***

\chapter{Introduction}
\label{chap:introduction}

\section{Some section}

\chapter{Design of Slope}
\label{chap:design}

\section{Design goals}
- pluggable

\section{Platform}
- usable with current c++ data structures

\section{???}

Other rpc's are pluggable into slope.
An elaborate rpc inside won't make too much of a difference: very small number of messages exchanged upon migrations

We believe rdma must be used the way more mature resources are used, and must not require an exclusive perimeter of cpu-bound threads and lots of memory and whatnot is dedicated to it.

One migration destination per node
\TODO{For the cases similar to this one where we discussed a better solution but
I'm currently implementing a simpler solution (e.g. allocation),
is it ok to just talk about and
provide the final solution without mentioning my more trivial implementation,
even if I never implement the better version (for things that are not on
the critical path of a particular requirement and do not affect benchmarks)?}



\section{Slope internals}

instances mmap a mutual (fixed possibly at run-time) chunk of virtual address. This is called the migratable memory or slope memory.
We keep a thread local context stack which keeps track of the owner of each memory allocation. \TODO{Sequence diagram-like figure, denoting the ownership stack
next to the function activation record stack of the thread}.
- The ith machine owns the i/n'th section of the memory, assuming the machines have equal memory. Machines use a two level memory
allocation scheme, in which they ask for large chunks (e.g. 4Gb) from other nodes in the cluster to make sure we do not exhaust
the address space that each node owns. This will eliminate the need for a centralized way of keeping track of object allocations.
Applications can use their own memory allocator over the reserved regions of the shared address space that they have allocated.
Deallocations return the memory to the owner of the big segment periodically, off the critical path. \TODO{These are tricky.. revisit}

Provide definitions: source/host: current owner, destination/remote: future owner

- source acquires destination's migration lock to initiate the transfer
    - we use memcached
\TODO{Do we even need to mention this? Instead: Max K concurrent migrations to the same node are possible, K must be known at init time}

- Migration happens in two separate steps: prepare, and execute. Before execute
  is called, the owner of the data structure is always the source. Up until then
  routines at the source are allowed to read and write to the data structure.
  After a call to prepare, the source is no longer allowed to call into
  procedures which allocate memory to or deallocate memory from the data structure. Typically all const qualified methods in c++ 
  are safe to be called after a call to prepare, however not every safe procedure is necessarily const qualified (e.g. updating an element in a vector)
  At the call to execute, the source is no longer allowed to write to the data structure and yields the ownership. Its reads will also return stale values.
  We cannot effectively mprotect away all of the memory that the data structure uses. A failed call will result in a SIGSEGV with no general way of
  recovering from it.

  - It is the responsibility of the application to prevent the program from falling in one of these restricted paths.


\TODO{sequence diagram}

- Prefetch step happens in the prepare phase. source walks through the pages one by one, mprotects them to readonly, and
  sends them to the corresponding addresses over RDMA.

  - This step is optional. If the data structure is under heavy usage and most of the memory is being touched, this
  will only lengthen the migration process, during which the source cannot allocate/deallocate to the object that is being
  migrated, without much improvement towards decreasing the handoff time, during which
  none of the two machines have read or write access to the data structure.


** Idea: From the call to execute, until when the source tells the sink about the call and send the
dirty pages that it needs to pull, we are losing time. What if the source pushes a few of the pages
based on a parameter that we optimize, to leverage the bandwidth during the idle time?
 - I think regardless of the size of the data structure, this is a useful approach: If it's big,
 only a small ratio is dirtied in between the phase changes and if it's small, then well, it's already small.
 Since we can send 100Gb/s = 100Mb/ms, this is even close to the rate at which we dirty the pages, we win.

\section{Handling failures}
Migrations effectively allow us to persist a correct state of the data structure: We have a snapshot at migration time

If a node goes away, we don't lose anything because of the point to point
nature of node communications. Every node knows who has allocated its memory
and can evict those leases if it desirable. No application fault tolerance
features provided aside from snapshots at migration time.

\chapter{Evaluation}
\label{chap:evaluation}

We go over a few use cases of Slope in real-world systems and present
benchmarks for select applications. We also discuss
the metrics that the applications using Slope can measure to get a sense of
how much Slope is impacting their performance.

To measure times and calculate performance metrics globally in the cluster,
we synchronize the start time from one machine to all other machines in the
cluster, by estimating the round trip time between them, which we do by
calculating the median among multiple round trip time calculations. Round
trip times are approximately $5 {\mu}{s}$, a pessimistic upper bound for the
error in time synchronization.

To make sure our graphs are accurate, we run each configuration at least 5 times
and average out the results, and in some cases, drop the min and max values to
eliminate the outlying points. That means each point in each of our graphs is
the resulting value from the above calculation on multiple runs with the same
configuration. Even without eliminating the outlying points, the standard errors
of our measurements are negligible (i.e. multiple orders of magnitude smaller)
compared to the reported values, unless otherwise stated.

Looking at each sub-system in Slope, one can define certain metrics that
reflect if that sub-system is working efficiently. For example we measure
the elapsed time until the prefill operation completes. We also measure
other metrics which more directly impact application performance, such as
end to end migration delay or time during which the object is unusable at
either end.

\subparagraph{Migration friendliness of data structures:}
Based on how the migration process and specifically how the prefill operation
works, objects which use their internal memory in a ``fixed'' manner, that is
without doing much memory allocation/deallocation, are very good candidates
for migration, since they can function seamlessly throughout the prefill phase,
by only inducing dirty page overhead.

Examples of these objects include bloom filters, where we have a fixed
array of bits the size of which always stays the same.
Similarly hash tables which use open addressing techniques such as cuckoo
hashing scheme for their collision resolution are also good candidates for the
same reason.
Apart from these objects which make the best-case scenario for Slope, we also
discuss migrating a generic object whose allocation/deallocation patterns are not
ideal.

\section{Case study: core metrics and STL objects}
\subsection{Migrating a vector with clean pages}
\label{sec:cleanvec}

\begin{figure}[tp]
    \begin{center}
        %% Creator: Matplotlib, PGF backend
%%
%% To include the figure in your LaTeX document, write
%%   \input{<filename>.pgf}
%%
%% Make sure the required packages are loaded in your preamble
%%   \usepackage{pgf}
%%
%% and, on pdftex
%%   \usepackage[utf8]{inputenc}\DeclareUnicodeCharacter{2212}{-}
%%
%% or, on luatex and xetex
%%   \usepackage{unicode-math}
%%
%% Figures using additional raster images can only be included by \input if
%% they are in the same directory as the main LaTeX file. For loading figures
%% from other directories you can use the `import` package
%%   \usepackage{import}
%%
%% and then include the figures with
%%   \import{<path to file>}{<filename>.pgf}
%%
%% Matplotlib used the following preamble
%%
\begingroup%
\makeatletter%
\begin{pgfpicture}%
\pgfpathrectangle{\pgfpointorigin}{\pgfqpoint{6.251220in}{3.516311in}}%
\pgfusepath{use as bounding box, clip}%
\begin{pgfscope}%
\pgfsetbuttcap%
\pgfsetmiterjoin%
\definecolor{currentfill}{rgb}{1.000000,1.000000,1.000000}%
\pgfsetfillcolor{currentfill}%
\pgfsetlinewidth{0.000000pt}%
\definecolor{currentstroke}{rgb}{1.000000,1.000000,1.000000}%
\pgfsetstrokecolor{currentstroke}%
\pgfsetdash{}{0pt}%
\pgfpathmoveto{\pgfqpoint{0.000000in}{0.000000in}}%
\pgfpathlineto{\pgfqpoint{6.251220in}{0.000000in}}%
\pgfpathlineto{\pgfqpoint{6.251220in}{3.516311in}}%
\pgfpathlineto{\pgfqpoint{0.000000in}{3.516311in}}%
\pgfpathclose%
\pgfusepath{fill}%
\end{pgfscope}%
\begin{pgfscope}%
\pgfsetbuttcap%
\pgfsetmiterjoin%
\definecolor{currentfill}{rgb}{1.000000,1.000000,1.000000}%
\pgfsetfillcolor{currentfill}%
\pgfsetlinewidth{0.000000pt}%
\definecolor{currentstroke}{rgb}{0.000000,0.000000,0.000000}%
\pgfsetstrokecolor{currentstroke}%
\pgfsetstrokeopacity{0.000000}%
\pgfsetdash{}{0pt}%
\pgfpathmoveto{\pgfqpoint{0.781402in}{0.386794in}}%
\pgfpathlineto{\pgfqpoint{5.626098in}{0.386794in}}%
\pgfpathlineto{\pgfqpoint{5.626098in}{3.094354in}}%
\pgfpathlineto{\pgfqpoint{0.781402in}{3.094354in}}%
\pgfpathclose%
\pgfusepath{fill}%
\end{pgfscope}%
\begin{pgfscope}%
\pgfsetbuttcap%
\pgfsetroundjoin%
\definecolor{currentfill}{rgb}{0.000000,0.000000,0.000000}%
\pgfsetfillcolor{currentfill}%
\pgfsetlinewidth{0.803000pt}%
\definecolor{currentstroke}{rgb}{0.000000,0.000000,0.000000}%
\pgfsetstrokecolor{currentstroke}%
\pgfsetdash{}{0pt}%
\pgfsys@defobject{currentmarker}{\pgfqpoint{0.000000in}{-0.048611in}}{\pgfqpoint{0.000000in}{0.000000in}}{%
\pgfpathmoveto{\pgfqpoint{0.000000in}{0.000000in}}%
\pgfpathlineto{\pgfqpoint{0.000000in}{-0.048611in}}%
\pgfusepath{stroke,fill}%
}%
\begin{pgfscope}%
\pgfsys@transformshift{1.001616in}{0.386794in}%
\pgfsys@useobject{currentmarker}{}%
\end{pgfscope}%
\end{pgfscope}%
\begin{pgfscope}%
\definecolor{textcolor}{rgb}{0.000000,0.000000,0.000000}%
\pgfsetstrokecolor{textcolor}%
\pgfsetfillcolor{textcolor}%
\pgftext[x=1.001616in,y=0.289572in,,top]{\color{textcolor}\rmfamily\fontsize{10.000000}{12.000000}\selectfont \(\displaystyle {1}\)}%
\end{pgfscope}%
\begin{pgfscope}%
\pgfsetbuttcap%
\pgfsetroundjoin%
\definecolor{currentfill}{rgb}{0.000000,0.000000,0.000000}%
\pgfsetfillcolor{currentfill}%
\pgfsetlinewidth{0.803000pt}%
\definecolor{currentstroke}{rgb}{0.000000,0.000000,0.000000}%
\pgfsetstrokecolor{currentstroke}%
\pgfsetdash{}{0pt}%
\pgfsys@defobject{currentmarker}{\pgfqpoint{0.000000in}{-0.048611in}}{\pgfqpoint{0.000000in}{0.000000in}}{%
\pgfpathmoveto{\pgfqpoint{0.000000in}{0.000000in}}%
\pgfpathlineto{\pgfqpoint{0.000000in}{-0.048611in}}%
\pgfusepath{stroke,fill}%
}%
\begin{pgfscope}%
\pgfsys@transformshift{1.882434in}{0.386794in}%
\pgfsys@useobject{currentmarker}{}%
\end{pgfscope}%
\end{pgfscope}%
\begin{pgfscope}%
\definecolor{textcolor}{rgb}{0.000000,0.000000,0.000000}%
\pgfsetstrokecolor{textcolor}%
\pgfsetfillcolor{textcolor}%
\pgftext[x=1.882434in,y=0.289572in,,top]{\color{textcolor}\rmfamily\fontsize{10.000000}{12.000000}\selectfont \(\displaystyle {20000}\)}%
\end{pgfscope}%
\begin{pgfscope}%
\pgfsetbuttcap%
\pgfsetroundjoin%
\definecolor{currentfill}{rgb}{0.000000,0.000000,0.000000}%
\pgfsetfillcolor{currentfill}%
\pgfsetlinewidth{0.803000pt}%
\definecolor{currentstroke}{rgb}{0.000000,0.000000,0.000000}%
\pgfsetstrokecolor{currentstroke}%
\pgfsetdash{}{0pt}%
\pgfsys@defobject{currentmarker}{\pgfqpoint{0.000000in}{-0.048611in}}{\pgfqpoint{0.000000in}{0.000000in}}{%
\pgfpathmoveto{\pgfqpoint{0.000000in}{0.000000in}}%
\pgfpathlineto{\pgfqpoint{0.000000in}{-0.048611in}}%
\pgfusepath{stroke,fill}%
}%
\begin{pgfscope}%
\pgfsys@transformshift{2.763297in}{0.386794in}%
\pgfsys@useobject{currentmarker}{}%
\end{pgfscope}%
\end{pgfscope}%
\begin{pgfscope}%
\definecolor{textcolor}{rgb}{0.000000,0.000000,0.000000}%
\pgfsetstrokecolor{textcolor}%
\pgfsetfillcolor{textcolor}%
\pgftext[x=2.763297in,y=0.289572in,,top]{\color{textcolor}\rmfamily\fontsize{10.000000}{12.000000}\selectfont \(\displaystyle {40000}\)}%
\end{pgfscope}%
\begin{pgfscope}%
\pgfsetbuttcap%
\pgfsetroundjoin%
\definecolor{currentfill}{rgb}{0.000000,0.000000,0.000000}%
\pgfsetfillcolor{currentfill}%
\pgfsetlinewidth{0.803000pt}%
\definecolor{currentstroke}{rgb}{0.000000,0.000000,0.000000}%
\pgfsetstrokecolor{currentstroke}%
\pgfsetdash{}{0pt}%
\pgfsys@defobject{currentmarker}{\pgfqpoint{0.000000in}{-0.048611in}}{\pgfqpoint{0.000000in}{0.000000in}}{%
\pgfpathmoveto{\pgfqpoint{0.000000in}{0.000000in}}%
\pgfpathlineto{\pgfqpoint{0.000000in}{-0.048611in}}%
\pgfusepath{stroke,fill}%
}%
\begin{pgfscope}%
\pgfsys@transformshift{3.644159in}{0.386794in}%
\pgfsys@useobject{currentmarker}{}%
\end{pgfscope}%
\end{pgfscope}%
\begin{pgfscope}%
\definecolor{textcolor}{rgb}{0.000000,0.000000,0.000000}%
\pgfsetstrokecolor{textcolor}%
\pgfsetfillcolor{textcolor}%
\pgftext[x=3.644159in,y=0.289572in,,top]{\color{textcolor}\rmfamily\fontsize{10.000000}{12.000000}\selectfont \(\displaystyle {60000}\)}%
\end{pgfscope}%
\begin{pgfscope}%
\pgfsetbuttcap%
\pgfsetroundjoin%
\definecolor{currentfill}{rgb}{0.000000,0.000000,0.000000}%
\pgfsetfillcolor{currentfill}%
\pgfsetlinewidth{0.803000pt}%
\definecolor{currentstroke}{rgb}{0.000000,0.000000,0.000000}%
\pgfsetstrokecolor{currentstroke}%
\pgfsetdash{}{0pt}%
\pgfsys@defobject{currentmarker}{\pgfqpoint{0.000000in}{-0.048611in}}{\pgfqpoint{0.000000in}{0.000000in}}{%
\pgfpathmoveto{\pgfqpoint{0.000000in}{0.000000in}}%
\pgfpathlineto{\pgfqpoint{0.000000in}{-0.048611in}}%
\pgfusepath{stroke,fill}%
}%
\begin{pgfscope}%
\pgfsys@transformshift{4.525022in}{0.386794in}%
\pgfsys@useobject{currentmarker}{}%
\end{pgfscope}%
\end{pgfscope}%
\begin{pgfscope}%
\definecolor{textcolor}{rgb}{0.000000,0.000000,0.000000}%
\pgfsetstrokecolor{textcolor}%
\pgfsetfillcolor{textcolor}%
\pgftext[x=4.525022in,y=0.289572in,,top]{\color{textcolor}\rmfamily\fontsize{10.000000}{12.000000}\selectfont \(\displaystyle {80000}\)}%
\end{pgfscope}%
\begin{pgfscope}%
\pgfsetbuttcap%
\pgfsetroundjoin%
\definecolor{currentfill}{rgb}{0.000000,0.000000,0.000000}%
\pgfsetfillcolor{currentfill}%
\pgfsetlinewidth{0.803000pt}%
\definecolor{currentstroke}{rgb}{0.000000,0.000000,0.000000}%
\pgfsetstrokecolor{currentstroke}%
\pgfsetdash{}{0pt}%
\pgfsys@defobject{currentmarker}{\pgfqpoint{0.000000in}{-0.048611in}}{\pgfqpoint{0.000000in}{0.000000in}}{%
\pgfpathmoveto{\pgfqpoint{0.000000in}{0.000000in}}%
\pgfpathlineto{\pgfqpoint{0.000000in}{-0.048611in}}%
\pgfusepath{stroke,fill}%
}%
\begin{pgfscope}%
\pgfsys@transformshift{5.405885in}{0.386794in}%
\pgfsys@useobject{currentmarker}{}%
\end{pgfscope}%
\end{pgfscope}%
\begin{pgfscope}%
\definecolor{textcolor}{rgb}{0.000000,0.000000,0.000000}%
\pgfsetstrokecolor{textcolor}%
\pgfsetfillcolor{textcolor}%
\pgftext[x=5.405885in,y=0.289572in,,top]{\color{textcolor}\rmfamily\fontsize{10.000000}{12.000000}\selectfont \(\displaystyle {100000}\)}%
\end{pgfscope}%
\begin{pgfscope}%
\definecolor{textcolor}{rgb}{0.000000,0.000000,0.000000}%
\pgfsetstrokecolor{textcolor}%
\pgfsetfillcolor{textcolor}%
\pgftext[x=3.203750in,y=0.110560in,,top]{\color{textcolor}\rmfamily\fontsize{10.000000}{12.000000}\selectfont Number of 4KB pages}%
\end{pgfscope}%
\begin{pgfscope}%
\pgfsetbuttcap%
\pgfsetroundjoin%
\definecolor{currentfill}{rgb}{0.000000,0.000000,0.000000}%
\pgfsetfillcolor{currentfill}%
\pgfsetlinewidth{0.803000pt}%
\definecolor{currentstroke}{rgb}{0.000000,0.000000,0.000000}%
\pgfsetstrokecolor{currentstroke}%
\pgfsetdash{}{0pt}%
\pgfsys@defobject{currentmarker}{\pgfqpoint{-0.048611in}{0.000000in}}{\pgfqpoint{-0.000000in}{0.000000in}}{%
\pgfpathmoveto{\pgfqpoint{-0.000000in}{0.000000in}}%
\pgfpathlineto{\pgfqpoint{-0.048611in}{0.000000in}}%
\pgfusepath{stroke,fill}%
}%
\begin{pgfscope}%
\pgfsys@transformshift{0.781402in}{0.509865in}%
\pgfsys@useobject{currentmarker}{}%
\end{pgfscope}%
\end{pgfscope}%
\begin{pgfscope}%
\definecolor{textcolor}{rgb}{0.000000,0.000000,0.000000}%
\pgfsetstrokecolor{textcolor}%
\pgfsetfillcolor{textcolor}%
\pgftext[x=0.614736in, y=0.461639in, left, base]{\color{textcolor}\rmfamily\fontsize{10.000000}{12.000000}\selectfont \(\displaystyle {0}\)}%
\end{pgfscope}%
\begin{pgfscope}%
\pgfsetbuttcap%
\pgfsetroundjoin%
\definecolor{currentfill}{rgb}{0.000000,0.000000,0.000000}%
\pgfsetfillcolor{currentfill}%
\pgfsetlinewidth{0.803000pt}%
\definecolor{currentstroke}{rgb}{0.000000,0.000000,0.000000}%
\pgfsetstrokecolor{currentstroke}%
\pgfsetdash{}{0pt}%
\pgfsys@defobject{currentmarker}{\pgfqpoint{-0.048611in}{0.000000in}}{\pgfqpoint{-0.000000in}{0.000000in}}{%
\pgfpathmoveto{\pgfqpoint{-0.000000in}{0.000000in}}%
\pgfpathlineto{\pgfqpoint{-0.048611in}{0.000000in}}%
\pgfusepath{stroke,fill}%
}%
\begin{pgfscope}%
\pgfsys@transformshift{0.781402in}{0.831595in}%
\pgfsys@useobject{currentmarker}{}%
\end{pgfscope}%
\end{pgfscope}%
\begin{pgfscope}%
\definecolor{textcolor}{rgb}{0.000000,0.000000,0.000000}%
\pgfsetstrokecolor{textcolor}%
\pgfsetfillcolor{textcolor}%
\pgftext[x=0.475846in, y=0.783370in, left, base]{\color{textcolor}\rmfamily\fontsize{10.000000}{12.000000}\selectfont \(\displaystyle {500}\)}%
\end{pgfscope}%
\begin{pgfscope}%
\pgfsetbuttcap%
\pgfsetroundjoin%
\definecolor{currentfill}{rgb}{0.000000,0.000000,0.000000}%
\pgfsetfillcolor{currentfill}%
\pgfsetlinewidth{0.803000pt}%
\definecolor{currentstroke}{rgb}{0.000000,0.000000,0.000000}%
\pgfsetstrokecolor{currentstroke}%
\pgfsetdash{}{0pt}%
\pgfsys@defobject{currentmarker}{\pgfqpoint{-0.048611in}{0.000000in}}{\pgfqpoint{-0.000000in}{0.000000in}}{%
\pgfpathmoveto{\pgfqpoint{-0.000000in}{0.000000in}}%
\pgfpathlineto{\pgfqpoint{-0.048611in}{0.000000in}}%
\pgfusepath{stroke,fill}%
}%
\begin{pgfscope}%
\pgfsys@transformshift{0.781402in}{1.153326in}%
\pgfsys@useobject{currentmarker}{}%
\end{pgfscope}%
\end{pgfscope}%
\begin{pgfscope}%
\definecolor{textcolor}{rgb}{0.000000,0.000000,0.000000}%
\pgfsetstrokecolor{textcolor}%
\pgfsetfillcolor{textcolor}%
\pgftext[x=0.406402in, y=1.105100in, left, base]{\color{textcolor}\rmfamily\fontsize{10.000000}{12.000000}\selectfont \(\displaystyle {1000}\)}%
\end{pgfscope}%
\begin{pgfscope}%
\pgfsetbuttcap%
\pgfsetroundjoin%
\definecolor{currentfill}{rgb}{0.000000,0.000000,0.000000}%
\pgfsetfillcolor{currentfill}%
\pgfsetlinewidth{0.803000pt}%
\definecolor{currentstroke}{rgb}{0.000000,0.000000,0.000000}%
\pgfsetstrokecolor{currentstroke}%
\pgfsetdash{}{0pt}%
\pgfsys@defobject{currentmarker}{\pgfqpoint{-0.048611in}{0.000000in}}{\pgfqpoint{-0.000000in}{0.000000in}}{%
\pgfpathmoveto{\pgfqpoint{-0.000000in}{0.000000in}}%
\pgfpathlineto{\pgfqpoint{-0.048611in}{0.000000in}}%
\pgfusepath{stroke,fill}%
}%
\begin{pgfscope}%
\pgfsys@transformshift{0.781402in}{1.475056in}%
\pgfsys@useobject{currentmarker}{}%
\end{pgfscope}%
\end{pgfscope}%
\begin{pgfscope}%
\definecolor{textcolor}{rgb}{0.000000,0.000000,0.000000}%
\pgfsetstrokecolor{textcolor}%
\pgfsetfillcolor{textcolor}%
\pgftext[x=0.406402in, y=1.426831in, left, base]{\color{textcolor}\rmfamily\fontsize{10.000000}{12.000000}\selectfont \(\displaystyle {1500}\)}%
\end{pgfscope}%
\begin{pgfscope}%
\pgfsetbuttcap%
\pgfsetroundjoin%
\definecolor{currentfill}{rgb}{0.000000,0.000000,0.000000}%
\pgfsetfillcolor{currentfill}%
\pgfsetlinewidth{0.803000pt}%
\definecolor{currentstroke}{rgb}{0.000000,0.000000,0.000000}%
\pgfsetstrokecolor{currentstroke}%
\pgfsetdash{}{0pt}%
\pgfsys@defobject{currentmarker}{\pgfqpoint{-0.048611in}{0.000000in}}{\pgfqpoint{-0.000000in}{0.000000in}}{%
\pgfpathmoveto{\pgfqpoint{-0.000000in}{0.000000in}}%
\pgfpathlineto{\pgfqpoint{-0.048611in}{0.000000in}}%
\pgfusepath{stroke,fill}%
}%
\begin{pgfscope}%
\pgfsys@transformshift{0.781402in}{1.796787in}%
\pgfsys@useobject{currentmarker}{}%
\end{pgfscope}%
\end{pgfscope}%
\begin{pgfscope}%
\definecolor{textcolor}{rgb}{0.000000,0.000000,0.000000}%
\pgfsetstrokecolor{textcolor}%
\pgfsetfillcolor{textcolor}%
\pgftext[x=0.406402in, y=1.748562in, left, base]{\color{textcolor}\rmfamily\fontsize{10.000000}{12.000000}\selectfont \(\displaystyle {2000}\)}%
\end{pgfscope}%
\begin{pgfscope}%
\pgfsetbuttcap%
\pgfsetroundjoin%
\definecolor{currentfill}{rgb}{0.000000,0.000000,0.000000}%
\pgfsetfillcolor{currentfill}%
\pgfsetlinewidth{0.803000pt}%
\definecolor{currentstroke}{rgb}{0.000000,0.000000,0.000000}%
\pgfsetstrokecolor{currentstroke}%
\pgfsetdash{}{0pt}%
\pgfsys@defobject{currentmarker}{\pgfqpoint{-0.048611in}{0.000000in}}{\pgfqpoint{-0.000000in}{0.000000in}}{%
\pgfpathmoveto{\pgfqpoint{-0.000000in}{0.000000in}}%
\pgfpathlineto{\pgfqpoint{-0.048611in}{0.000000in}}%
\pgfusepath{stroke,fill}%
}%
\begin{pgfscope}%
\pgfsys@transformshift{0.781402in}{2.118517in}%
\pgfsys@useobject{currentmarker}{}%
\end{pgfscope}%
\end{pgfscope}%
\begin{pgfscope}%
\definecolor{textcolor}{rgb}{0.000000,0.000000,0.000000}%
\pgfsetstrokecolor{textcolor}%
\pgfsetfillcolor{textcolor}%
\pgftext[x=0.406402in, y=2.070292in, left, base]{\color{textcolor}\rmfamily\fontsize{10.000000}{12.000000}\selectfont \(\displaystyle {2500}\)}%
\end{pgfscope}%
\begin{pgfscope}%
\pgfsetbuttcap%
\pgfsetroundjoin%
\definecolor{currentfill}{rgb}{0.000000,0.000000,0.000000}%
\pgfsetfillcolor{currentfill}%
\pgfsetlinewidth{0.803000pt}%
\definecolor{currentstroke}{rgb}{0.000000,0.000000,0.000000}%
\pgfsetstrokecolor{currentstroke}%
\pgfsetdash{}{0pt}%
\pgfsys@defobject{currentmarker}{\pgfqpoint{-0.048611in}{0.000000in}}{\pgfqpoint{-0.000000in}{0.000000in}}{%
\pgfpathmoveto{\pgfqpoint{-0.000000in}{0.000000in}}%
\pgfpathlineto{\pgfqpoint{-0.048611in}{0.000000in}}%
\pgfusepath{stroke,fill}%
}%
\begin{pgfscope}%
\pgfsys@transformshift{0.781402in}{2.440248in}%
\pgfsys@useobject{currentmarker}{}%
\end{pgfscope}%
\end{pgfscope}%
\begin{pgfscope}%
\definecolor{textcolor}{rgb}{0.000000,0.000000,0.000000}%
\pgfsetstrokecolor{textcolor}%
\pgfsetfillcolor{textcolor}%
\pgftext[x=0.406402in, y=2.392023in, left, base]{\color{textcolor}\rmfamily\fontsize{10.000000}{12.000000}\selectfont \(\displaystyle {3000}\)}%
\end{pgfscope}%
\begin{pgfscope}%
\pgfsetbuttcap%
\pgfsetroundjoin%
\definecolor{currentfill}{rgb}{0.000000,0.000000,0.000000}%
\pgfsetfillcolor{currentfill}%
\pgfsetlinewidth{0.803000pt}%
\definecolor{currentstroke}{rgb}{0.000000,0.000000,0.000000}%
\pgfsetstrokecolor{currentstroke}%
\pgfsetdash{}{0pt}%
\pgfsys@defobject{currentmarker}{\pgfqpoint{-0.048611in}{0.000000in}}{\pgfqpoint{-0.000000in}{0.000000in}}{%
\pgfpathmoveto{\pgfqpoint{-0.000000in}{0.000000in}}%
\pgfpathlineto{\pgfqpoint{-0.048611in}{0.000000in}}%
\pgfusepath{stroke,fill}%
}%
\begin{pgfscope}%
\pgfsys@transformshift{0.781402in}{2.761979in}%
\pgfsys@useobject{currentmarker}{}%
\end{pgfscope}%
\end{pgfscope}%
\begin{pgfscope}%
\definecolor{textcolor}{rgb}{0.000000,0.000000,0.000000}%
\pgfsetstrokecolor{textcolor}%
\pgfsetfillcolor{textcolor}%
\pgftext[x=0.406402in, y=2.713753in, left, base]{\color{textcolor}\rmfamily\fontsize{10.000000}{12.000000}\selectfont \(\displaystyle {3500}\)}%
\end{pgfscope}%
\begin{pgfscope}%
\pgfsetbuttcap%
\pgfsetroundjoin%
\definecolor{currentfill}{rgb}{0.000000,0.000000,0.000000}%
\pgfsetfillcolor{currentfill}%
\pgfsetlinewidth{0.803000pt}%
\definecolor{currentstroke}{rgb}{0.000000,0.000000,0.000000}%
\pgfsetstrokecolor{currentstroke}%
\pgfsetdash{}{0pt}%
\pgfsys@defobject{currentmarker}{\pgfqpoint{-0.048611in}{0.000000in}}{\pgfqpoint{-0.000000in}{0.000000in}}{%
\pgfpathmoveto{\pgfqpoint{-0.000000in}{0.000000in}}%
\pgfpathlineto{\pgfqpoint{-0.048611in}{0.000000in}}%
\pgfusepath{stroke,fill}%
}%
\begin{pgfscope}%
\pgfsys@transformshift{0.781402in}{3.083709in}%
\pgfsys@useobject{currentmarker}{}%
\end{pgfscope}%
\end{pgfscope}%
\begin{pgfscope}%
\definecolor{textcolor}{rgb}{0.000000,0.000000,0.000000}%
\pgfsetstrokecolor{textcolor}%
\pgfsetfillcolor{textcolor}%
\pgftext[x=0.406402in, y=3.035484in, left, base]{\color{textcolor}\rmfamily\fontsize{10.000000}{12.000000}\selectfont \(\displaystyle {4000}\)}%
\end{pgfscope}%
\begin{pgfscope}%
\definecolor{textcolor}{rgb}{0.000000,0.000000,0.000000}%
\pgfsetstrokecolor{textcolor}%
\pgfsetfillcolor{textcolor}%
\pgftext[x=0.350846in,y=1.740574in,,bottom,rotate=90.000000]{\color{textcolor}\rmfamily\fontsize{10.000000}{12.000000}\selectfont Elapsed time (milliseconds)}%
\end{pgfscope}%
\begin{pgfscope}%
\pgfpathrectangle{\pgfqpoint{0.781402in}{0.386794in}}{\pgfqpoint{4.844695in}{2.707560in}}%
\pgfusepath{clip}%
\pgfsetrectcap%
\pgfsetroundjoin%
\pgfsetlinewidth{1.505625pt}%
\definecolor{currentstroke}{rgb}{0.121569,0.466667,0.705882}%
\pgfsetstrokecolor{currentstroke}%
\pgfsetdash{}{0pt}%
\pgfpathmoveto{\pgfqpoint{1.001616in}{0.511396in}}%
\pgfpathlineto{\pgfqpoint{1.442003in}{0.708084in}}%
\pgfpathlineto{\pgfqpoint{1.882434in}{0.908626in}}%
\pgfpathlineto{\pgfqpoint{2.322866in}{1.123131in}}%
\pgfpathlineto{\pgfqpoint{2.763297in}{1.320690in}}%
\pgfpathlineto{\pgfqpoint{3.203728in}{1.536318in}}%
\pgfpathlineto{\pgfqpoint{3.644159in}{1.748271in}}%
\pgfpathlineto{\pgfqpoint{4.084591in}{1.958908in}}%
\pgfpathlineto{\pgfqpoint{4.525022in}{2.163276in}}%
\pgfpathlineto{\pgfqpoint{4.965453in}{2.378237in}}%
\pgfpathlineto{\pgfqpoint{5.405885in}{2.549731in}}%
\pgfusepath{stroke}%
\end{pgfscope}%
\begin{pgfscope}%
\pgfpathrectangle{\pgfqpoint{0.781402in}{0.386794in}}{\pgfqpoint{4.844695in}{2.707560in}}%
\pgfusepath{clip}%
\pgfsetbuttcap%
\pgfsetroundjoin%
\definecolor{currentfill}{rgb}{0.121569,0.466667,0.705882}%
\pgfsetfillcolor{currentfill}%
\pgfsetlinewidth{1.003750pt}%
\definecolor{currentstroke}{rgb}{0.121569,0.466667,0.705882}%
\pgfsetstrokecolor{currentstroke}%
\pgfsetdash{}{0pt}%
\pgfsys@defobject{currentmarker}{\pgfqpoint{-0.041667in}{-0.041667in}}{\pgfqpoint{0.041667in}{0.041667in}}{%
\pgfpathmoveto{\pgfqpoint{0.000000in}{-0.041667in}}%
\pgfpathcurveto{\pgfqpoint{0.011050in}{-0.041667in}}{\pgfqpoint{0.021649in}{-0.037276in}}{\pgfqpoint{0.029463in}{-0.029463in}}%
\pgfpathcurveto{\pgfqpoint{0.037276in}{-0.021649in}}{\pgfqpoint{0.041667in}{-0.011050in}}{\pgfqpoint{0.041667in}{0.000000in}}%
\pgfpathcurveto{\pgfqpoint{0.041667in}{0.011050in}}{\pgfqpoint{0.037276in}{0.021649in}}{\pgfqpoint{0.029463in}{0.029463in}}%
\pgfpathcurveto{\pgfqpoint{0.021649in}{0.037276in}}{\pgfqpoint{0.011050in}{0.041667in}}{\pgfqpoint{0.000000in}{0.041667in}}%
\pgfpathcurveto{\pgfqpoint{-0.011050in}{0.041667in}}{\pgfqpoint{-0.021649in}{0.037276in}}{\pgfqpoint{-0.029463in}{0.029463in}}%
\pgfpathcurveto{\pgfqpoint{-0.037276in}{0.021649in}}{\pgfqpoint{-0.041667in}{0.011050in}}{\pgfqpoint{-0.041667in}{0.000000in}}%
\pgfpathcurveto{\pgfqpoint{-0.041667in}{-0.011050in}}{\pgfqpoint{-0.037276in}{-0.021649in}}{\pgfqpoint{-0.029463in}{-0.029463in}}%
\pgfpathcurveto{\pgfqpoint{-0.021649in}{-0.037276in}}{\pgfqpoint{-0.011050in}{-0.041667in}}{\pgfqpoint{0.000000in}{-0.041667in}}%
\pgfpathclose%
\pgfusepath{stroke,fill}%
}%
\begin{pgfscope}%
\pgfsys@transformshift{1.001616in}{0.511396in}%
\pgfsys@useobject{currentmarker}{}%
\end{pgfscope}%
\begin{pgfscope}%
\pgfsys@transformshift{1.442003in}{0.708084in}%
\pgfsys@useobject{currentmarker}{}%
\end{pgfscope}%
\begin{pgfscope}%
\pgfsys@transformshift{1.882434in}{0.908626in}%
\pgfsys@useobject{currentmarker}{}%
\end{pgfscope}%
\begin{pgfscope}%
\pgfsys@transformshift{2.322866in}{1.123131in}%
\pgfsys@useobject{currentmarker}{}%
\end{pgfscope}%
\begin{pgfscope}%
\pgfsys@transformshift{2.763297in}{1.320690in}%
\pgfsys@useobject{currentmarker}{}%
\end{pgfscope}%
\begin{pgfscope}%
\pgfsys@transformshift{3.203728in}{1.536318in}%
\pgfsys@useobject{currentmarker}{}%
\end{pgfscope}%
\begin{pgfscope}%
\pgfsys@transformshift{3.644159in}{1.748271in}%
\pgfsys@useobject{currentmarker}{}%
\end{pgfscope}%
\begin{pgfscope}%
\pgfsys@transformshift{4.084591in}{1.958908in}%
\pgfsys@useobject{currentmarker}{}%
\end{pgfscope}%
\begin{pgfscope}%
\pgfsys@transformshift{4.525022in}{2.163276in}%
\pgfsys@useobject{currentmarker}{}%
\end{pgfscope}%
\begin{pgfscope}%
\pgfsys@transformshift{4.965453in}{2.378237in}%
\pgfsys@useobject{currentmarker}{}%
\end{pgfscope}%
\begin{pgfscope}%
\pgfsys@transformshift{5.405885in}{2.549731in}%
\pgfsys@useobject{currentmarker}{}%
\end{pgfscope}%
\end{pgfscope}%
\begin{pgfscope}%
\pgfpathrectangle{\pgfqpoint{0.781402in}{0.386794in}}{\pgfqpoint{4.844695in}{2.707560in}}%
\pgfusepath{clip}%
\pgfsetrectcap%
\pgfsetroundjoin%
\pgfsetlinewidth{1.505625pt}%
\definecolor{currentstroke}{rgb}{1.000000,0.498039,0.054902}%
\pgfsetstrokecolor{currentstroke}%
\pgfsetdash{}{0pt}%
\pgfpathmoveto{\pgfqpoint{1.001616in}{0.509865in}}%
\pgfpathlineto{\pgfqpoint{1.442003in}{0.509866in}}%
\pgfpathlineto{\pgfqpoint{1.882434in}{0.509866in}}%
\pgfpathlineto{\pgfqpoint{2.322866in}{0.509944in}}%
\pgfpathlineto{\pgfqpoint{2.763297in}{0.509925in}}%
\pgfpathlineto{\pgfqpoint{3.203728in}{0.509935in}}%
\pgfpathlineto{\pgfqpoint{3.644159in}{0.509920in}}%
\pgfpathlineto{\pgfqpoint{4.084591in}{0.509966in}}%
\pgfpathlineto{\pgfqpoint{4.525022in}{0.510025in}}%
\pgfpathlineto{\pgfqpoint{4.965453in}{0.509958in}}%
\pgfpathlineto{\pgfqpoint{5.405885in}{0.510074in}}%
\pgfusepath{stroke}%
\end{pgfscope}%
\begin{pgfscope}%
\pgfpathrectangle{\pgfqpoint{0.781402in}{0.386794in}}{\pgfqpoint{4.844695in}{2.707560in}}%
\pgfusepath{clip}%
\pgfsetbuttcap%
\pgfsetroundjoin%
\definecolor{currentfill}{rgb}{1.000000,0.498039,0.054902}%
\pgfsetfillcolor{currentfill}%
\pgfsetlinewidth{1.003750pt}%
\definecolor{currentstroke}{rgb}{1.000000,0.498039,0.054902}%
\pgfsetstrokecolor{currentstroke}%
\pgfsetdash{}{0pt}%
\pgfsys@defobject{currentmarker}{\pgfqpoint{-0.041667in}{-0.041667in}}{\pgfqpoint{0.041667in}{0.041667in}}{%
\pgfpathmoveto{\pgfqpoint{0.000000in}{-0.041667in}}%
\pgfpathcurveto{\pgfqpoint{0.011050in}{-0.041667in}}{\pgfqpoint{0.021649in}{-0.037276in}}{\pgfqpoint{0.029463in}{-0.029463in}}%
\pgfpathcurveto{\pgfqpoint{0.037276in}{-0.021649in}}{\pgfqpoint{0.041667in}{-0.011050in}}{\pgfqpoint{0.041667in}{0.000000in}}%
\pgfpathcurveto{\pgfqpoint{0.041667in}{0.011050in}}{\pgfqpoint{0.037276in}{0.021649in}}{\pgfqpoint{0.029463in}{0.029463in}}%
\pgfpathcurveto{\pgfqpoint{0.021649in}{0.037276in}}{\pgfqpoint{0.011050in}{0.041667in}}{\pgfqpoint{0.000000in}{0.041667in}}%
\pgfpathcurveto{\pgfqpoint{-0.011050in}{0.041667in}}{\pgfqpoint{-0.021649in}{0.037276in}}{\pgfqpoint{-0.029463in}{0.029463in}}%
\pgfpathcurveto{\pgfqpoint{-0.037276in}{0.021649in}}{\pgfqpoint{-0.041667in}{0.011050in}}{\pgfqpoint{-0.041667in}{0.000000in}}%
\pgfpathcurveto{\pgfqpoint{-0.041667in}{-0.011050in}}{\pgfqpoint{-0.037276in}{-0.021649in}}{\pgfqpoint{-0.029463in}{-0.029463in}}%
\pgfpathcurveto{\pgfqpoint{-0.021649in}{-0.037276in}}{\pgfqpoint{-0.011050in}{-0.041667in}}{\pgfqpoint{0.000000in}{-0.041667in}}%
\pgfpathclose%
\pgfusepath{stroke,fill}%
}%
\begin{pgfscope}%
\pgfsys@transformshift{1.001616in}{0.509865in}%
\pgfsys@useobject{currentmarker}{}%
\end{pgfscope}%
\begin{pgfscope}%
\pgfsys@transformshift{1.442003in}{0.509866in}%
\pgfsys@useobject{currentmarker}{}%
\end{pgfscope}%
\begin{pgfscope}%
\pgfsys@transformshift{1.882434in}{0.509866in}%
\pgfsys@useobject{currentmarker}{}%
\end{pgfscope}%
\begin{pgfscope}%
\pgfsys@transformshift{2.322866in}{0.509944in}%
\pgfsys@useobject{currentmarker}{}%
\end{pgfscope}%
\begin{pgfscope}%
\pgfsys@transformshift{2.763297in}{0.509925in}%
\pgfsys@useobject{currentmarker}{}%
\end{pgfscope}%
\begin{pgfscope}%
\pgfsys@transformshift{3.203728in}{0.509935in}%
\pgfsys@useobject{currentmarker}{}%
\end{pgfscope}%
\begin{pgfscope}%
\pgfsys@transformshift{3.644159in}{0.509920in}%
\pgfsys@useobject{currentmarker}{}%
\end{pgfscope}%
\begin{pgfscope}%
\pgfsys@transformshift{4.084591in}{0.509966in}%
\pgfsys@useobject{currentmarker}{}%
\end{pgfscope}%
\begin{pgfscope}%
\pgfsys@transformshift{4.525022in}{0.510025in}%
\pgfsys@useobject{currentmarker}{}%
\end{pgfscope}%
\begin{pgfscope}%
\pgfsys@transformshift{4.965453in}{0.509958in}%
\pgfsys@useobject{currentmarker}{}%
\end{pgfscope}%
\begin{pgfscope}%
\pgfsys@transformshift{5.405885in}{0.510074in}%
\pgfsys@useobject{currentmarker}{}%
\end{pgfscope}%
\end{pgfscope}%
\begin{pgfscope}%
\pgfpathrectangle{\pgfqpoint{0.781402in}{0.386794in}}{\pgfqpoint{4.844695in}{2.707560in}}%
\pgfusepath{clip}%
\pgfsetrectcap%
\pgfsetroundjoin%
\pgfsetlinewidth{1.505625pt}%
\definecolor{currentstroke}{rgb}{0.172549,0.627451,0.172549}%
\pgfsetstrokecolor{currentstroke}%
\pgfsetdash{}{0pt}%
\pgfpathmoveto{\pgfqpoint{1.001616in}{0.511428in}}%
\pgfpathlineto{\pgfqpoint{1.442003in}{0.748177in}}%
\pgfpathlineto{\pgfqpoint{1.882434in}{0.989661in}}%
\pgfpathlineto{\pgfqpoint{2.322866in}{1.245169in}}%
\pgfpathlineto{\pgfqpoint{2.763297in}{1.482760in}}%
\pgfpathlineto{\pgfqpoint{3.203728in}{1.741947in}}%
\pgfpathlineto{\pgfqpoint{3.644159in}{1.998971in}}%
\pgfpathlineto{\pgfqpoint{4.084591in}{2.250919in}}%
\pgfpathlineto{\pgfqpoint{4.525022in}{2.501474in}}%
\pgfpathlineto{\pgfqpoint{4.965453in}{2.747795in}}%
\pgfpathlineto{\pgfqpoint{5.405885in}{2.971283in}}%
\pgfusepath{stroke}%
\end{pgfscope}%
\begin{pgfscope}%
\pgfpathrectangle{\pgfqpoint{0.781402in}{0.386794in}}{\pgfqpoint{4.844695in}{2.707560in}}%
\pgfusepath{clip}%
\pgfsetbuttcap%
\pgfsetroundjoin%
\definecolor{currentfill}{rgb}{0.172549,0.627451,0.172549}%
\pgfsetfillcolor{currentfill}%
\pgfsetlinewidth{1.003750pt}%
\definecolor{currentstroke}{rgb}{0.172549,0.627451,0.172549}%
\pgfsetstrokecolor{currentstroke}%
\pgfsetdash{}{0pt}%
\pgfsys@defobject{currentmarker}{\pgfqpoint{-0.041667in}{-0.041667in}}{\pgfqpoint{0.041667in}{0.041667in}}{%
\pgfpathmoveto{\pgfqpoint{0.000000in}{-0.041667in}}%
\pgfpathcurveto{\pgfqpoint{0.011050in}{-0.041667in}}{\pgfqpoint{0.021649in}{-0.037276in}}{\pgfqpoint{0.029463in}{-0.029463in}}%
\pgfpathcurveto{\pgfqpoint{0.037276in}{-0.021649in}}{\pgfqpoint{0.041667in}{-0.011050in}}{\pgfqpoint{0.041667in}{0.000000in}}%
\pgfpathcurveto{\pgfqpoint{0.041667in}{0.011050in}}{\pgfqpoint{0.037276in}{0.021649in}}{\pgfqpoint{0.029463in}{0.029463in}}%
\pgfpathcurveto{\pgfqpoint{0.021649in}{0.037276in}}{\pgfqpoint{0.011050in}{0.041667in}}{\pgfqpoint{0.000000in}{0.041667in}}%
\pgfpathcurveto{\pgfqpoint{-0.011050in}{0.041667in}}{\pgfqpoint{-0.021649in}{0.037276in}}{\pgfqpoint{-0.029463in}{0.029463in}}%
\pgfpathcurveto{\pgfqpoint{-0.037276in}{0.021649in}}{\pgfqpoint{-0.041667in}{0.011050in}}{\pgfqpoint{-0.041667in}{0.000000in}}%
\pgfpathcurveto{\pgfqpoint{-0.041667in}{-0.011050in}}{\pgfqpoint{-0.037276in}{-0.021649in}}{\pgfqpoint{-0.029463in}{-0.029463in}}%
\pgfpathcurveto{\pgfqpoint{-0.021649in}{-0.037276in}}{\pgfqpoint{-0.011050in}{-0.041667in}}{\pgfqpoint{0.000000in}{-0.041667in}}%
\pgfpathclose%
\pgfusepath{stroke,fill}%
}%
\begin{pgfscope}%
\pgfsys@transformshift{1.001616in}{0.511428in}%
\pgfsys@useobject{currentmarker}{}%
\end{pgfscope}%
\begin{pgfscope}%
\pgfsys@transformshift{1.442003in}{0.748177in}%
\pgfsys@useobject{currentmarker}{}%
\end{pgfscope}%
\begin{pgfscope}%
\pgfsys@transformshift{1.882434in}{0.989661in}%
\pgfsys@useobject{currentmarker}{}%
\end{pgfscope}%
\begin{pgfscope}%
\pgfsys@transformshift{2.322866in}{1.245169in}%
\pgfsys@useobject{currentmarker}{}%
\end{pgfscope}%
\begin{pgfscope}%
\pgfsys@transformshift{2.763297in}{1.482760in}%
\pgfsys@useobject{currentmarker}{}%
\end{pgfscope}%
\begin{pgfscope}%
\pgfsys@transformshift{3.203728in}{1.741947in}%
\pgfsys@useobject{currentmarker}{}%
\end{pgfscope}%
\begin{pgfscope}%
\pgfsys@transformshift{3.644159in}{1.998971in}%
\pgfsys@useobject{currentmarker}{}%
\end{pgfscope}%
\begin{pgfscope}%
\pgfsys@transformshift{4.084591in}{2.250919in}%
\pgfsys@useobject{currentmarker}{}%
\end{pgfscope}%
\begin{pgfscope}%
\pgfsys@transformshift{4.525022in}{2.501474in}%
\pgfsys@useobject{currentmarker}{}%
\end{pgfscope}%
\begin{pgfscope}%
\pgfsys@transformshift{4.965453in}{2.747795in}%
\pgfsys@useobject{currentmarker}{}%
\end{pgfscope}%
\begin{pgfscope}%
\pgfsys@transformshift{5.405885in}{2.971283in}%
\pgfsys@useobject{currentmarker}{}%
\end{pgfscope}%
\end{pgfscope}%
\begin{pgfscope}%
\pgfsetrectcap%
\pgfsetmiterjoin%
\pgfsetlinewidth{0.803000pt}%
\definecolor{currentstroke}{rgb}{0.000000,0.000000,0.000000}%
\pgfsetstrokecolor{currentstroke}%
\pgfsetdash{}{0pt}%
\pgfpathmoveto{\pgfqpoint{0.781402in}{0.386794in}}%
\pgfpathlineto{\pgfqpoint{0.781402in}{3.094354in}}%
\pgfusepath{stroke}%
\end{pgfscope}%
\begin{pgfscope}%
\pgfsetrectcap%
\pgfsetmiterjoin%
\pgfsetlinewidth{0.803000pt}%
\definecolor{currentstroke}{rgb}{0.000000,0.000000,0.000000}%
\pgfsetstrokecolor{currentstroke}%
\pgfsetdash{}{0pt}%
\pgfpathmoveto{\pgfqpoint{5.626098in}{0.386794in}}%
\pgfpathlineto{\pgfqpoint{5.626098in}{3.094354in}}%
\pgfusepath{stroke}%
\end{pgfscope}%
\begin{pgfscope}%
\pgfsetrectcap%
\pgfsetmiterjoin%
\pgfsetlinewidth{0.803000pt}%
\definecolor{currentstroke}{rgb}{0.000000,0.000000,0.000000}%
\pgfsetstrokecolor{currentstroke}%
\pgfsetdash{}{0pt}%
\pgfpathmoveto{\pgfqpoint{0.781402in}{0.386794in}}%
\pgfpathlineto{\pgfqpoint{5.626098in}{0.386794in}}%
\pgfusepath{stroke}%
\end{pgfscope}%
\begin{pgfscope}%
\pgfsetrectcap%
\pgfsetmiterjoin%
\pgfsetlinewidth{0.803000pt}%
\definecolor{currentstroke}{rgb}{0.000000,0.000000,0.000000}%
\pgfsetstrokecolor{currentstroke}%
\pgfsetdash{}{0pt}%
\pgfpathmoveto{\pgfqpoint{0.781402in}{3.094354in}}%
\pgfpathlineto{\pgfqpoint{5.626098in}{3.094354in}}%
\pgfusepath{stroke}%
\end{pgfscope}%
\begin{pgfscope}%
\pgfsetbuttcap%
\pgfsetmiterjoin%
\definecolor{currentfill}{rgb}{1.000000,1.000000,1.000000}%
\pgfsetfillcolor{currentfill}%
\pgfsetfillopacity{0.800000}%
\pgfsetlinewidth{1.003750pt}%
\definecolor{currentstroke}{rgb}{0.800000,0.800000,0.800000}%
\pgfsetstrokecolor{currentstroke}%
\pgfsetstrokeopacity{0.800000}%
\pgfsetdash{}{0pt}%
\pgfpathmoveto{\pgfqpoint{0.878625in}{2.402224in}}%
\pgfpathlineto{\pgfqpoint{2.791822in}{2.402224in}}%
\pgfpathquadraticcurveto{\pgfqpoint{2.819600in}{2.402224in}}{\pgfqpoint{2.819600in}{2.430002in}}%
\pgfpathlineto{\pgfqpoint{2.819600in}{2.997132in}}%
\pgfpathquadraticcurveto{\pgfqpoint{2.819600in}{3.024909in}}{\pgfqpoint{2.791822in}{3.024909in}}%
\pgfpathlineto{\pgfqpoint{0.878625in}{3.024909in}}%
\pgfpathquadraticcurveto{\pgfqpoint{0.850847in}{3.024909in}}{\pgfqpoint{0.850847in}{2.997132in}}%
\pgfpathlineto{\pgfqpoint{0.850847in}{2.430002in}}%
\pgfpathquadraticcurveto{\pgfqpoint{0.850847in}{2.402224in}}{\pgfqpoint{0.878625in}{2.402224in}}%
\pgfpathclose%
\pgfusepath{stroke,fill}%
\end{pgfscope}%
\begin{pgfscope}%
\pgfsetrectcap%
\pgfsetroundjoin%
\pgfsetlinewidth{1.505625pt}%
\definecolor{currentstroke}{rgb}{0.121569,0.466667,0.705882}%
\pgfsetstrokecolor{currentstroke}%
\pgfsetdash{}{0pt}%
\pgfpathmoveto{\pgfqpoint{0.906402in}{2.920743in}}%
\pgfpathlineto{\pgfqpoint{1.184180in}{2.920743in}}%
\pgfusepath{stroke}%
\end{pgfscope}%
\begin{pgfscope}%
\pgfsetbuttcap%
\pgfsetroundjoin%
\definecolor{currentfill}{rgb}{0.121569,0.466667,0.705882}%
\pgfsetfillcolor{currentfill}%
\pgfsetlinewidth{1.003750pt}%
\definecolor{currentstroke}{rgb}{0.121569,0.466667,0.705882}%
\pgfsetstrokecolor{currentstroke}%
\pgfsetdash{}{0pt}%
\pgfsys@defobject{currentmarker}{\pgfqpoint{-0.041667in}{-0.041667in}}{\pgfqpoint{0.041667in}{0.041667in}}{%
\pgfpathmoveto{\pgfqpoint{0.000000in}{-0.041667in}}%
\pgfpathcurveto{\pgfqpoint{0.011050in}{-0.041667in}}{\pgfqpoint{0.021649in}{-0.037276in}}{\pgfqpoint{0.029463in}{-0.029463in}}%
\pgfpathcurveto{\pgfqpoint{0.037276in}{-0.021649in}}{\pgfqpoint{0.041667in}{-0.011050in}}{\pgfqpoint{0.041667in}{0.000000in}}%
\pgfpathcurveto{\pgfqpoint{0.041667in}{0.011050in}}{\pgfqpoint{0.037276in}{0.021649in}}{\pgfqpoint{0.029463in}{0.029463in}}%
\pgfpathcurveto{\pgfqpoint{0.021649in}{0.037276in}}{\pgfqpoint{0.011050in}{0.041667in}}{\pgfqpoint{0.000000in}{0.041667in}}%
\pgfpathcurveto{\pgfqpoint{-0.011050in}{0.041667in}}{\pgfqpoint{-0.021649in}{0.037276in}}{\pgfqpoint{-0.029463in}{0.029463in}}%
\pgfpathcurveto{\pgfqpoint{-0.037276in}{0.021649in}}{\pgfqpoint{-0.041667in}{0.011050in}}{\pgfqpoint{-0.041667in}{0.000000in}}%
\pgfpathcurveto{\pgfqpoint{-0.041667in}{-0.011050in}}{\pgfqpoint{-0.037276in}{-0.021649in}}{\pgfqpoint{-0.029463in}{-0.029463in}}%
\pgfpathcurveto{\pgfqpoint{-0.021649in}{-0.037276in}}{\pgfqpoint{-0.011050in}{-0.041667in}}{\pgfqpoint{0.000000in}{-0.041667in}}%
\pgfpathclose%
\pgfusepath{stroke,fill}%
}%
\begin{pgfscope}%
\pgfsys@transformshift{1.045291in}{2.920743in}%
\pgfsys@useobject{currentmarker}{}%
\end{pgfscope}%
\end{pgfscope}%
\begin{pgfscope}%
\definecolor{textcolor}{rgb}{0.000000,0.000000,0.000000}%
\pgfsetstrokecolor{textcolor}%
\pgfsetfillcolor{textcolor}%
\pgftext[x=1.295291in,y=2.872132in,left,base]{\color{textcolor}\rmfamily\fontsize{10.000000}{12.000000}\selectfont Prefill duration}%
\end{pgfscope}%
\begin{pgfscope}%
\pgfsetrectcap%
\pgfsetroundjoin%
\pgfsetlinewidth{1.505625pt}%
\definecolor{currentstroke}{rgb}{1.000000,0.498039,0.054902}%
\pgfsetstrokecolor{currentstroke}%
\pgfsetdash{}{0pt}%
\pgfpathmoveto{\pgfqpoint{0.906402in}{2.727070in}}%
\pgfpathlineto{\pgfqpoint{1.184180in}{2.727070in}}%
\pgfusepath{stroke}%
\end{pgfscope}%
\begin{pgfscope}%
\pgfsetbuttcap%
\pgfsetroundjoin%
\definecolor{currentfill}{rgb}{1.000000,0.498039,0.054902}%
\pgfsetfillcolor{currentfill}%
\pgfsetlinewidth{1.003750pt}%
\definecolor{currentstroke}{rgb}{1.000000,0.498039,0.054902}%
\pgfsetstrokecolor{currentstroke}%
\pgfsetdash{}{0pt}%
\pgfsys@defobject{currentmarker}{\pgfqpoint{-0.041667in}{-0.041667in}}{\pgfqpoint{0.041667in}{0.041667in}}{%
\pgfpathmoveto{\pgfqpoint{0.000000in}{-0.041667in}}%
\pgfpathcurveto{\pgfqpoint{0.011050in}{-0.041667in}}{\pgfqpoint{0.021649in}{-0.037276in}}{\pgfqpoint{0.029463in}{-0.029463in}}%
\pgfpathcurveto{\pgfqpoint{0.037276in}{-0.021649in}}{\pgfqpoint{0.041667in}{-0.011050in}}{\pgfqpoint{0.041667in}{0.000000in}}%
\pgfpathcurveto{\pgfqpoint{0.041667in}{0.011050in}}{\pgfqpoint{0.037276in}{0.021649in}}{\pgfqpoint{0.029463in}{0.029463in}}%
\pgfpathcurveto{\pgfqpoint{0.021649in}{0.037276in}}{\pgfqpoint{0.011050in}{0.041667in}}{\pgfqpoint{0.000000in}{0.041667in}}%
\pgfpathcurveto{\pgfqpoint{-0.011050in}{0.041667in}}{\pgfqpoint{-0.021649in}{0.037276in}}{\pgfqpoint{-0.029463in}{0.029463in}}%
\pgfpathcurveto{\pgfqpoint{-0.037276in}{0.021649in}}{\pgfqpoint{-0.041667in}{0.011050in}}{\pgfqpoint{-0.041667in}{0.000000in}}%
\pgfpathcurveto{\pgfqpoint{-0.041667in}{-0.011050in}}{\pgfqpoint{-0.037276in}{-0.021649in}}{\pgfqpoint{-0.029463in}{-0.029463in}}%
\pgfpathcurveto{\pgfqpoint{-0.021649in}{-0.037276in}}{\pgfqpoint{-0.011050in}{-0.041667in}}{\pgfqpoint{0.000000in}{-0.041667in}}%
\pgfpathclose%
\pgfusepath{stroke,fill}%
}%
\begin{pgfscope}%
\pgfsys@transformshift{1.045291in}{2.727070in}%
\pgfsys@useobject{currentmarker}{}%
\end{pgfscope}%
\end{pgfscope}%
\begin{pgfscope}%
\definecolor{textcolor}{rgb}{0.000000,0.000000,0.000000}%
\pgfsetstrokecolor{textcolor}%
\pgfsetfillcolor{textcolor}%
\pgftext[x=1.295291in,y=2.678459in,left,base]{\color{textcolor}\rmfamily\fontsize{10.000000}{12.000000}\selectfont Duration without owner}%
\end{pgfscope}%
\begin{pgfscope}%
\pgfsetrectcap%
\pgfsetroundjoin%
\pgfsetlinewidth{1.505625pt}%
\definecolor{currentstroke}{rgb}{0.172549,0.627451,0.172549}%
\pgfsetstrokecolor{currentstroke}%
\pgfsetdash{}{0pt}%
\pgfpathmoveto{\pgfqpoint{0.906402in}{2.533397in}}%
\pgfpathlineto{\pgfqpoint{1.184180in}{2.533397in}}%
\pgfusepath{stroke}%
\end{pgfscope}%
\begin{pgfscope}%
\pgfsetbuttcap%
\pgfsetroundjoin%
\definecolor{currentfill}{rgb}{0.172549,0.627451,0.172549}%
\pgfsetfillcolor{currentfill}%
\pgfsetlinewidth{1.003750pt}%
\definecolor{currentstroke}{rgb}{0.172549,0.627451,0.172549}%
\pgfsetstrokecolor{currentstroke}%
\pgfsetdash{}{0pt}%
\pgfsys@defobject{currentmarker}{\pgfqpoint{-0.041667in}{-0.041667in}}{\pgfqpoint{0.041667in}{0.041667in}}{%
\pgfpathmoveto{\pgfqpoint{0.000000in}{-0.041667in}}%
\pgfpathcurveto{\pgfqpoint{0.011050in}{-0.041667in}}{\pgfqpoint{0.021649in}{-0.037276in}}{\pgfqpoint{0.029463in}{-0.029463in}}%
\pgfpathcurveto{\pgfqpoint{0.037276in}{-0.021649in}}{\pgfqpoint{0.041667in}{-0.011050in}}{\pgfqpoint{0.041667in}{0.000000in}}%
\pgfpathcurveto{\pgfqpoint{0.041667in}{0.011050in}}{\pgfqpoint{0.037276in}{0.021649in}}{\pgfqpoint{0.029463in}{0.029463in}}%
\pgfpathcurveto{\pgfqpoint{0.021649in}{0.037276in}}{\pgfqpoint{0.011050in}{0.041667in}}{\pgfqpoint{0.000000in}{0.041667in}}%
\pgfpathcurveto{\pgfqpoint{-0.011050in}{0.041667in}}{\pgfqpoint{-0.021649in}{0.037276in}}{\pgfqpoint{-0.029463in}{0.029463in}}%
\pgfpathcurveto{\pgfqpoint{-0.037276in}{0.021649in}}{\pgfqpoint{-0.041667in}{0.011050in}}{\pgfqpoint{-0.041667in}{0.000000in}}%
\pgfpathcurveto{\pgfqpoint{-0.041667in}{-0.011050in}}{\pgfqpoint{-0.037276in}{-0.021649in}}{\pgfqpoint{-0.029463in}{-0.029463in}}%
\pgfpathcurveto{\pgfqpoint{-0.021649in}{-0.037276in}}{\pgfqpoint{-0.011050in}{-0.041667in}}{\pgfqpoint{0.000000in}{-0.041667in}}%
\pgfpathclose%
\pgfusepath{stroke,fill}%
}%
\begin{pgfscope}%
\pgfsys@transformshift{1.045291in}{2.533397in}%
\pgfsys@useobject{currentmarker}{}%
\end{pgfscope}%
\end{pgfscope}%
\begin{pgfscope}%
\definecolor{textcolor}{rgb}{0.000000,0.000000,0.000000}%
\pgfsetstrokecolor{textcolor}%
\pgfsetfillcolor{textcolor}%
\pgftext[x=1.295291in,y=2.484786in,left,base]{\color{textcolor}\rmfamily\fontsize{10.000000}{12.000000}\selectfont End to end latency}%
\end{pgfscope}%
\end{pgfpicture}%
\makeatother%
\endgroup%

    \end{center}
    \caption{Migration statistics of a clean vector (4KB pages)}
    \label{fig:vectorreadonly}
\end{figure}

\begin{figure}[tp]
    \begin{center}
        %% Creator: Matplotlib, PGF backend
%%
%% To include the figure in your LaTeX document, write
%%   \input{<filename>.pgf}
%%
%% Make sure the required packages are loaded in your preamble
%%   \usepackage{pgf}
%%
%% and, on pdftex
%%   \usepackage[utf8]{inputenc}\DeclareUnicodeCharacter{2212}{-}
%%
%% or, on luatex and xetex
%%   \usepackage{unicode-math}
%%
%% Figures using additional raster images can only be included by \input if
%% they are in the same directory as the main LaTeX file. For loading figures
%% from other directories you can use the `import` package
%%   \usepackage{import}
%%
%% and then include the figures with
%%   \import{<path to file>}{<filename>.pgf}
%%
%% Matplotlib used the following preamble
%%
\begingroup%
\makeatletter%
\begin{pgfpicture}%
\pgfpathrectangle{\pgfpointorigin}{\pgfqpoint{6.251220in}{3.516311in}}%
\pgfusepath{use as bounding box, clip}%
\begin{pgfscope}%
\pgfsetbuttcap%
\pgfsetmiterjoin%
\definecolor{currentfill}{rgb}{1.000000,1.000000,1.000000}%
\pgfsetfillcolor{currentfill}%
\pgfsetlinewidth{0.000000pt}%
\definecolor{currentstroke}{rgb}{1.000000,1.000000,1.000000}%
\pgfsetstrokecolor{currentstroke}%
\pgfsetdash{}{0pt}%
\pgfpathmoveto{\pgfqpoint{0.000000in}{0.000000in}}%
\pgfpathlineto{\pgfqpoint{6.251220in}{0.000000in}}%
\pgfpathlineto{\pgfqpoint{6.251220in}{3.516311in}}%
\pgfpathlineto{\pgfqpoint{0.000000in}{3.516311in}}%
\pgfpathclose%
\pgfusepath{fill}%
\end{pgfscope}%
\begin{pgfscope}%
\pgfsetbuttcap%
\pgfsetmiterjoin%
\definecolor{currentfill}{rgb}{1.000000,1.000000,1.000000}%
\pgfsetfillcolor{currentfill}%
\pgfsetlinewidth{0.000000pt}%
\definecolor{currentstroke}{rgb}{0.000000,0.000000,0.000000}%
\pgfsetstrokecolor{currentstroke}%
\pgfsetstrokeopacity{0.000000}%
\pgfsetdash{}{0pt}%
\pgfpathmoveto{\pgfqpoint{0.781402in}{0.386794in}}%
\pgfpathlineto{\pgfqpoint{5.626098in}{0.386794in}}%
\pgfpathlineto{\pgfqpoint{5.626098in}{3.094354in}}%
\pgfpathlineto{\pgfqpoint{0.781402in}{3.094354in}}%
\pgfpathclose%
\pgfusepath{fill}%
\end{pgfscope}%
\begin{pgfscope}%
\pgfsetbuttcap%
\pgfsetroundjoin%
\definecolor{currentfill}{rgb}{0.000000,0.000000,0.000000}%
\pgfsetfillcolor{currentfill}%
\pgfsetlinewidth{0.803000pt}%
\definecolor{currentstroke}{rgb}{0.000000,0.000000,0.000000}%
\pgfsetstrokecolor{currentstroke}%
\pgfsetdash{}{0pt}%
\pgfsys@defobject{currentmarker}{\pgfqpoint{0.000000in}{-0.048611in}}{\pgfqpoint{0.000000in}{0.000000in}}{%
\pgfpathmoveto{\pgfqpoint{0.000000in}{0.000000in}}%
\pgfpathlineto{\pgfqpoint{0.000000in}{-0.048611in}}%
\pgfusepath{stroke,fill}%
}%
\begin{pgfscope}%
\pgfsys@transformshift{1.001616in}{0.386794in}%
\pgfsys@useobject{currentmarker}{}%
\end{pgfscope}%
\end{pgfscope}%
\begin{pgfscope}%
\definecolor{textcolor}{rgb}{0.000000,0.000000,0.000000}%
\pgfsetstrokecolor{textcolor}%
\pgfsetfillcolor{textcolor}%
\pgftext[x=1.001616in,y=0.289572in,,top]{\color{textcolor}\rmfamily\fontsize{10.000000}{12.000000}\selectfont \(\displaystyle {1}\)}%
\end{pgfscope}%
\begin{pgfscope}%
\pgfsetbuttcap%
\pgfsetroundjoin%
\definecolor{currentfill}{rgb}{0.000000,0.000000,0.000000}%
\pgfsetfillcolor{currentfill}%
\pgfsetlinewidth{0.803000pt}%
\definecolor{currentstroke}{rgb}{0.000000,0.000000,0.000000}%
\pgfsetstrokecolor{currentstroke}%
\pgfsetdash{}{0pt}%
\pgfsys@defobject{currentmarker}{\pgfqpoint{0.000000in}{-0.048611in}}{\pgfqpoint{0.000000in}{0.000000in}}{%
\pgfpathmoveto{\pgfqpoint{0.000000in}{0.000000in}}%
\pgfpathlineto{\pgfqpoint{0.000000in}{-0.048611in}}%
\pgfusepath{stroke,fill}%
}%
\begin{pgfscope}%
\pgfsys@transformshift{1.532784in}{0.386794in}%
\pgfsys@useobject{currentmarker}{}%
\end{pgfscope}%
\end{pgfscope}%
\begin{pgfscope}%
\definecolor{textcolor}{rgb}{0.000000,0.000000,0.000000}%
\pgfsetstrokecolor{textcolor}%
\pgfsetfillcolor{textcolor}%
\pgftext[x=1.532784in,y=0.289572in,,top]{\color{textcolor}\rmfamily\fontsize{10.000000}{12.000000}\selectfont \(\displaystyle {25}\)}%
\end{pgfscope}%
\begin{pgfscope}%
\pgfsetbuttcap%
\pgfsetroundjoin%
\definecolor{currentfill}{rgb}{0.000000,0.000000,0.000000}%
\pgfsetfillcolor{currentfill}%
\pgfsetlinewidth{0.803000pt}%
\definecolor{currentstroke}{rgb}{0.000000,0.000000,0.000000}%
\pgfsetstrokecolor{currentstroke}%
\pgfsetdash{}{0pt}%
\pgfsys@defobject{currentmarker}{\pgfqpoint{0.000000in}{-0.048611in}}{\pgfqpoint{0.000000in}{0.000000in}}{%
\pgfpathmoveto{\pgfqpoint{0.000000in}{0.000000in}}%
\pgfpathlineto{\pgfqpoint{0.000000in}{-0.048611in}}%
\pgfusepath{stroke,fill}%
}%
\begin{pgfscope}%
\pgfsys@transformshift{2.086084in}{0.386794in}%
\pgfsys@useobject{currentmarker}{}%
\end{pgfscope}%
\end{pgfscope}%
\begin{pgfscope}%
\definecolor{textcolor}{rgb}{0.000000,0.000000,0.000000}%
\pgfsetstrokecolor{textcolor}%
\pgfsetfillcolor{textcolor}%
\pgftext[x=2.086084in,y=0.289572in,,top]{\color{textcolor}\rmfamily\fontsize{10.000000}{12.000000}\selectfont \(\displaystyle {50}\)}%
\end{pgfscope}%
\begin{pgfscope}%
\pgfsetbuttcap%
\pgfsetroundjoin%
\definecolor{currentfill}{rgb}{0.000000,0.000000,0.000000}%
\pgfsetfillcolor{currentfill}%
\pgfsetlinewidth{0.803000pt}%
\definecolor{currentstroke}{rgb}{0.000000,0.000000,0.000000}%
\pgfsetstrokecolor{currentstroke}%
\pgfsetdash{}{0pt}%
\pgfsys@defobject{currentmarker}{\pgfqpoint{0.000000in}{-0.048611in}}{\pgfqpoint{0.000000in}{0.000000in}}{%
\pgfpathmoveto{\pgfqpoint{0.000000in}{0.000000in}}%
\pgfpathlineto{\pgfqpoint{0.000000in}{-0.048611in}}%
\pgfusepath{stroke,fill}%
}%
\begin{pgfscope}%
\pgfsys@transformshift{2.639384in}{0.386794in}%
\pgfsys@useobject{currentmarker}{}%
\end{pgfscope}%
\end{pgfscope}%
\begin{pgfscope}%
\definecolor{textcolor}{rgb}{0.000000,0.000000,0.000000}%
\pgfsetstrokecolor{textcolor}%
\pgfsetfillcolor{textcolor}%
\pgftext[x=2.639384in,y=0.289572in,,top]{\color{textcolor}\rmfamily\fontsize{10.000000}{12.000000}\selectfont \(\displaystyle {75}\)}%
\end{pgfscope}%
\begin{pgfscope}%
\pgfsetbuttcap%
\pgfsetroundjoin%
\definecolor{currentfill}{rgb}{0.000000,0.000000,0.000000}%
\pgfsetfillcolor{currentfill}%
\pgfsetlinewidth{0.803000pt}%
\definecolor{currentstroke}{rgb}{0.000000,0.000000,0.000000}%
\pgfsetstrokecolor{currentstroke}%
\pgfsetdash{}{0pt}%
\pgfsys@defobject{currentmarker}{\pgfqpoint{0.000000in}{-0.048611in}}{\pgfqpoint{0.000000in}{0.000000in}}{%
\pgfpathmoveto{\pgfqpoint{0.000000in}{0.000000in}}%
\pgfpathlineto{\pgfqpoint{0.000000in}{-0.048611in}}%
\pgfusepath{stroke,fill}%
}%
\begin{pgfscope}%
\pgfsys@transformshift{3.192684in}{0.386794in}%
\pgfsys@useobject{currentmarker}{}%
\end{pgfscope}%
\end{pgfscope}%
\begin{pgfscope}%
\definecolor{textcolor}{rgb}{0.000000,0.000000,0.000000}%
\pgfsetstrokecolor{textcolor}%
\pgfsetfillcolor{textcolor}%
\pgftext[x=3.192684in,y=0.289572in,,top]{\color{textcolor}\rmfamily\fontsize{10.000000}{12.000000}\selectfont \(\displaystyle {100}\)}%
\end{pgfscope}%
\begin{pgfscope}%
\pgfsetbuttcap%
\pgfsetroundjoin%
\definecolor{currentfill}{rgb}{0.000000,0.000000,0.000000}%
\pgfsetfillcolor{currentfill}%
\pgfsetlinewidth{0.803000pt}%
\definecolor{currentstroke}{rgb}{0.000000,0.000000,0.000000}%
\pgfsetstrokecolor{currentstroke}%
\pgfsetdash{}{0pt}%
\pgfsys@defobject{currentmarker}{\pgfqpoint{0.000000in}{-0.048611in}}{\pgfqpoint{0.000000in}{0.000000in}}{%
\pgfpathmoveto{\pgfqpoint{0.000000in}{0.000000in}}%
\pgfpathlineto{\pgfqpoint{0.000000in}{-0.048611in}}%
\pgfusepath{stroke,fill}%
}%
\begin{pgfscope}%
\pgfsys@transformshift{3.745984in}{0.386794in}%
\pgfsys@useobject{currentmarker}{}%
\end{pgfscope}%
\end{pgfscope}%
\begin{pgfscope}%
\definecolor{textcolor}{rgb}{0.000000,0.000000,0.000000}%
\pgfsetstrokecolor{textcolor}%
\pgfsetfillcolor{textcolor}%
\pgftext[x=3.745984in,y=0.289572in,,top]{\color{textcolor}\rmfamily\fontsize{10.000000}{12.000000}\selectfont \(\displaystyle {125}\)}%
\end{pgfscope}%
\begin{pgfscope}%
\pgfsetbuttcap%
\pgfsetroundjoin%
\definecolor{currentfill}{rgb}{0.000000,0.000000,0.000000}%
\pgfsetfillcolor{currentfill}%
\pgfsetlinewidth{0.803000pt}%
\definecolor{currentstroke}{rgb}{0.000000,0.000000,0.000000}%
\pgfsetstrokecolor{currentstroke}%
\pgfsetdash{}{0pt}%
\pgfsys@defobject{currentmarker}{\pgfqpoint{0.000000in}{-0.048611in}}{\pgfqpoint{0.000000in}{0.000000in}}{%
\pgfpathmoveto{\pgfqpoint{0.000000in}{0.000000in}}%
\pgfpathlineto{\pgfqpoint{0.000000in}{-0.048611in}}%
\pgfusepath{stroke,fill}%
}%
\begin{pgfscope}%
\pgfsys@transformshift{4.299284in}{0.386794in}%
\pgfsys@useobject{currentmarker}{}%
\end{pgfscope}%
\end{pgfscope}%
\begin{pgfscope}%
\definecolor{textcolor}{rgb}{0.000000,0.000000,0.000000}%
\pgfsetstrokecolor{textcolor}%
\pgfsetfillcolor{textcolor}%
\pgftext[x=4.299284in,y=0.289572in,,top]{\color{textcolor}\rmfamily\fontsize{10.000000}{12.000000}\selectfont \(\displaystyle {150}\)}%
\end{pgfscope}%
\begin{pgfscope}%
\pgfsetbuttcap%
\pgfsetroundjoin%
\definecolor{currentfill}{rgb}{0.000000,0.000000,0.000000}%
\pgfsetfillcolor{currentfill}%
\pgfsetlinewidth{0.803000pt}%
\definecolor{currentstroke}{rgb}{0.000000,0.000000,0.000000}%
\pgfsetstrokecolor{currentstroke}%
\pgfsetdash{}{0pt}%
\pgfsys@defobject{currentmarker}{\pgfqpoint{0.000000in}{-0.048611in}}{\pgfqpoint{0.000000in}{0.000000in}}{%
\pgfpathmoveto{\pgfqpoint{0.000000in}{0.000000in}}%
\pgfpathlineto{\pgfqpoint{0.000000in}{-0.048611in}}%
\pgfusepath{stroke,fill}%
}%
\begin{pgfscope}%
\pgfsys@transformshift{4.852584in}{0.386794in}%
\pgfsys@useobject{currentmarker}{}%
\end{pgfscope}%
\end{pgfscope}%
\begin{pgfscope}%
\definecolor{textcolor}{rgb}{0.000000,0.000000,0.000000}%
\pgfsetstrokecolor{textcolor}%
\pgfsetfillcolor{textcolor}%
\pgftext[x=4.852584in,y=0.289572in,,top]{\color{textcolor}\rmfamily\fontsize{10.000000}{12.000000}\selectfont \(\displaystyle {175}\)}%
\end{pgfscope}%
\begin{pgfscope}%
\pgfsetbuttcap%
\pgfsetroundjoin%
\definecolor{currentfill}{rgb}{0.000000,0.000000,0.000000}%
\pgfsetfillcolor{currentfill}%
\pgfsetlinewidth{0.803000pt}%
\definecolor{currentstroke}{rgb}{0.000000,0.000000,0.000000}%
\pgfsetstrokecolor{currentstroke}%
\pgfsetdash{}{0pt}%
\pgfsys@defobject{currentmarker}{\pgfqpoint{0.000000in}{-0.048611in}}{\pgfqpoint{0.000000in}{0.000000in}}{%
\pgfpathmoveto{\pgfqpoint{0.000000in}{0.000000in}}%
\pgfpathlineto{\pgfqpoint{0.000000in}{-0.048611in}}%
\pgfusepath{stroke,fill}%
}%
\begin{pgfscope}%
\pgfsys@transformshift{5.405885in}{0.386794in}%
\pgfsys@useobject{currentmarker}{}%
\end{pgfscope}%
\end{pgfscope}%
\begin{pgfscope}%
\definecolor{textcolor}{rgb}{0.000000,0.000000,0.000000}%
\pgfsetstrokecolor{textcolor}%
\pgfsetfillcolor{textcolor}%
\pgftext[x=5.405885in,y=0.289572in,,top]{\color{textcolor}\rmfamily\fontsize{10.000000}{12.000000}\selectfont \(\displaystyle {200}\)}%
\end{pgfscope}%
\begin{pgfscope}%
\definecolor{textcolor}{rgb}{0.000000,0.000000,0.000000}%
\pgfsetstrokecolor{textcolor}%
\pgfsetfillcolor{textcolor}%
\pgftext[x=3.203750in,y=0.110560in,,top]{\color{textcolor}\rmfamily\fontsize{10.000000}{12.000000}\selectfont Number of 2MB pages}%
\end{pgfscope}%
\begin{pgfscope}%
\pgfsetbuttcap%
\pgfsetroundjoin%
\definecolor{currentfill}{rgb}{0.000000,0.000000,0.000000}%
\pgfsetfillcolor{currentfill}%
\pgfsetlinewidth{0.803000pt}%
\definecolor{currentstroke}{rgb}{0.000000,0.000000,0.000000}%
\pgfsetstrokecolor{currentstroke}%
\pgfsetdash{}{0pt}%
\pgfsys@defobject{currentmarker}{\pgfqpoint{-0.048611in}{0.000000in}}{\pgfqpoint{-0.000000in}{0.000000in}}{%
\pgfpathmoveto{\pgfqpoint{-0.000000in}{0.000000in}}%
\pgfpathlineto{\pgfqpoint{-0.048611in}{0.000000in}}%
\pgfusepath{stroke,fill}%
}%
\begin{pgfscope}%
\pgfsys@transformshift{0.781402in}{0.509850in}%
\pgfsys@useobject{currentmarker}{}%
\end{pgfscope}%
\end{pgfscope}%
\begin{pgfscope}%
\definecolor{textcolor}{rgb}{0.000000,0.000000,0.000000}%
\pgfsetstrokecolor{textcolor}%
\pgfsetfillcolor{textcolor}%
\pgftext[x=0.614736in, y=0.461625in, left, base]{\color{textcolor}\rmfamily\fontsize{10.000000}{12.000000}\selectfont \(\displaystyle {0}\)}%
\end{pgfscope}%
\begin{pgfscope}%
\pgfsetbuttcap%
\pgfsetroundjoin%
\definecolor{currentfill}{rgb}{0.000000,0.000000,0.000000}%
\pgfsetfillcolor{currentfill}%
\pgfsetlinewidth{0.803000pt}%
\definecolor{currentstroke}{rgb}{0.000000,0.000000,0.000000}%
\pgfsetstrokecolor{currentstroke}%
\pgfsetdash{}{0pt}%
\pgfsys@defobject{currentmarker}{\pgfqpoint{-0.048611in}{0.000000in}}{\pgfqpoint{-0.000000in}{0.000000in}}{%
\pgfpathmoveto{\pgfqpoint{-0.000000in}{0.000000in}}%
\pgfpathlineto{\pgfqpoint{-0.048611in}{0.000000in}}%
\pgfusepath{stroke,fill}%
}%
\begin{pgfscope}%
\pgfsys@transformshift{0.781402in}{0.858321in}%
\pgfsys@useobject{currentmarker}{}%
\end{pgfscope}%
\end{pgfscope}%
\begin{pgfscope}%
\definecolor{textcolor}{rgb}{0.000000,0.000000,0.000000}%
\pgfsetstrokecolor{textcolor}%
\pgfsetfillcolor{textcolor}%
\pgftext[x=0.545291in, y=0.810096in, left, base]{\color{textcolor}\rmfamily\fontsize{10.000000}{12.000000}\selectfont \(\displaystyle {20}\)}%
\end{pgfscope}%
\begin{pgfscope}%
\pgfsetbuttcap%
\pgfsetroundjoin%
\definecolor{currentfill}{rgb}{0.000000,0.000000,0.000000}%
\pgfsetfillcolor{currentfill}%
\pgfsetlinewidth{0.803000pt}%
\definecolor{currentstroke}{rgb}{0.000000,0.000000,0.000000}%
\pgfsetstrokecolor{currentstroke}%
\pgfsetdash{}{0pt}%
\pgfsys@defobject{currentmarker}{\pgfqpoint{-0.048611in}{0.000000in}}{\pgfqpoint{-0.000000in}{0.000000in}}{%
\pgfpathmoveto{\pgfqpoint{-0.000000in}{0.000000in}}%
\pgfpathlineto{\pgfqpoint{-0.048611in}{0.000000in}}%
\pgfusepath{stroke,fill}%
}%
\begin{pgfscope}%
\pgfsys@transformshift{0.781402in}{1.206792in}%
\pgfsys@useobject{currentmarker}{}%
\end{pgfscope}%
\end{pgfscope}%
\begin{pgfscope}%
\definecolor{textcolor}{rgb}{0.000000,0.000000,0.000000}%
\pgfsetstrokecolor{textcolor}%
\pgfsetfillcolor{textcolor}%
\pgftext[x=0.545291in, y=1.158567in, left, base]{\color{textcolor}\rmfamily\fontsize{10.000000}{12.000000}\selectfont \(\displaystyle {40}\)}%
\end{pgfscope}%
\begin{pgfscope}%
\pgfsetbuttcap%
\pgfsetroundjoin%
\definecolor{currentfill}{rgb}{0.000000,0.000000,0.000000}%
\pgfsetfillcolor{currentfill}%
\pgfsetlinewidth{0.803000pt}%
\definecolor{currentstroke}{rgb}{0.000000,0.000000,0.000000}%
\pgfsetstrokecolor{currentstroke}%
\pgfsetdash{}{0pt}%
\pgfsys@defobject{currentmarker}{\pgfqpoint{-0.048611in}{0.000000in}}{\pgfqpoint{-0.000000in}{0.000000in}}{%
\pgfpathmoveto{\pgfqpoint{-0.000000in}{0.000000in}}%
\pgfpathlineto{\pgfqpoint{-0.048611in}{0.000000in}}%
\pgfusepath{stroke,fill}%
}%
\begin{pgfscope}%
\pgfsys@transformshift{0.781402in}{1.555263in}%
\pgfsys@useobject{currentmarker}{}%
\end{pgfscope}%
\end{pgfscope}%
\begin{pgfscope}%
\definecolor{textcolor}{rgb}{0.000000,0.000000,0.000000}%
\pgfsetstrokecolor{textcolor}%
\pgfsetfillcolor{textcolor}%
\pgftext[x=0.545291in, y=1.507038in, left, base]{\color{textcolor}\rmfamily\fontsize{10.000000}{12.000000}\selectfont \(\displaystyle {60}\)}%
\end{pgfscope}%
\begin{pgfscope}%
\pgfsetbuttcap%
\pgfsetroundjoin%
\definecolor{currentfill}{rgb}{0.000000,0.000000,0.000000}%
\pgfsetfillcolor{currentfill}%
\pgfsetlinewidth{0.803000pt}%
\definecolor{currentstroke}{rgb}{0.000000,0.000000,0.000000}%
\pgfsetstrokecolor{currentstroke}%
\pgfsetdash{}{0pt}%
\pgfsys@defobject{currentmarker}{\pgfqpoint{-0.048611in}{0.000000in}}{\pgfqpoint{-0.000000in}{0.000000in}}{%
\pgfpathmoveto{\pgfqpoint{-0.000000in}{0.000000in}}%
\pgfpathlineto{\pgfqpoint{-0.048611in}{0.000000in}}%
\pgfusepath{stroke,fill}%
}%
\begin{pgfscope}%
\pgfsys@transformshift{0.781402in}{1.903734in}%
\pgfsys@useobject{currentmarker}{}%
\end{pgfscope}%
\end{pgfscope}%
\begin{pgfscope}%
\definecolor{textcolor}{rgb}{0.000000,0.000000,0.000000}%
\pgfsetstrokecolor{textcolor}%
\pgfsetfillcolor{textcolor}%
\pgftext[x=0.545291in, y=1.855508in, left, base]{\color{textcolor}\rmfamily\fontsize{10.000000}{12.000000}\selectfont \(\displaystyle {80}\)}%
\end{pgfscope}%
\begin{pgfscope}%
\pgfsetbuttcap%
\pgfsetroundjoin%
\definecolor{currentfill}{rgb}{0.000000,0.000000,0.000000}%
\pgfsetfillcolor{currentfill}%
\pgfsetlinewidth{0.803000pt}%
\definecolor{currentstroke}{rgb}{0.000000,0.000000,0.000000}%
\pgfsetstrokecolor{currentstroke}%
\pgfsetdash{}{0pt}%
\pgfsys@defobject{currentmarker}{\pgfqpoint{-0.048611in}{0.000000in}}{\pgfqpoint{-0.000000in}{0.000000in}}{%
\pgfpathmoveto{\pgfqpoint{-0.000000in}{0.000000in}}%
\pgfpathlineto{\pgfqpoint{-0.048611in}{0.000000in}}%
\pgfusepath{stroke,fill}%
}%
\begin{pgfscope}%
\pgfsys@transformshift{0.781402in}{2.252204in}%
\pgfsys@useobject{currentmarker}{}%
\end{pgfscope}%
\end{pgfscope}%
\begin{pgfscope}%
\definecolor{textcolor}{rgb}{0.000000,0.000000,0.000000}%
\pgfsetstrokecolor{textcolor}%
\pgfsetfillcolor{textcolor}%
\pgftext[x=0.475846in, y=2.203979in, left, base]{\color{textcolor}\rmfamily\fontsize{10.000000}{12.000000}\selectfont \(\displaystyle {100}\)}%
\end{pgfscope}%
\begin{pgfscope}%
\pgfsetbuttcap%
\pgfsetroundjoin%
\definecolor{currentfill}{rgb}{0.000000,0.000000,0.000000}%
\pgfsetfillcolor{currentfill}%
\pgfsetlinewidth{0.803000pt}%
\definecolor{currentstroke}{rgb}{0.000000,0.000000,0.000000}%
\pgfsetstrokecolor{currentstroke}%
\pgfsetdash{}{0pt}%
\pgfsys@defobject{currentmarker}{\pgfqpoint{-0.048611in}{0.000000in}}{\pgfqpoint{-0.000000in}{0.000000in}}{%
\pgfpathmoveto{\pgfqpoint{-0.000000in}{0.000000in}}%
\pgfpathlineto{\pgfqpoint{-0.048611in}{0.000000in}}%
\pgfusepath{stroke,fill}%
}%
\begin{pgfscope}%
\pgfsys@transformshift{0.781402in}{2.600675in}%
\pgfsys@useobject{currentmarker}{}%
\end{pgfscope}%
\end{pgfscope}%
\begin{pgfscope}%
\definecolor{textcolor}{rgb}{0.000000,0.000000,0.000000}%
\pgfsetstrokecolor{textcolor}%
\pgfsetfillcolor{textcolor}%
\pgftext[x=0.475846in, y=2.552450in, left, base]{\color{textcolor}\rmfamily\fontsize{10.000000}{12.000000}\selectfont \(\displaystyle {120}\)}%
\end{pgfscope}%
\begin{pgfscope}%
\pgfsetbuttcap%
\pgfsetroundjoin%
\definecolor{currentfill}{rgb}{0.000000,0.000000,0.000000}%
\pgfsetfillcolor{currentfill}%
\pgfsetlinewidth{0.803000pt}%
\definecolor{currentstroke}{rgb}{0.000000,0.000000,0.000000}%
\pgfsetstrokecolor{currentstroke}%
\pgfsetdash{}{0pt}%
\pgfsys@defobject{currentmarker}{\pgfqpoint{-0.048611in}{0.000000in}}{\pgfqpoint{-0.000000in}{0.000000in}}{%
\pgfpathmoveto{\pgfqpoint{-0.000000in}{0.000000in}}%
\pgfpathlineto{\pgfqpoint{-0.048611in}{0.000000in}}%
\pgfusepath{stroke,fill}%
}%
\begin{pgfscope}%
\pgfsys@transformshift{0.781402in}{2.949146in}%
\pgfsys@useobject{currentmarker}{}%
\end{pgfscope}%
\end{pgfscope}%
\begin{pgfscope}%
\definecolor{textcolor}{rgb}{0.000000,0.000000,0.000000}%
\pgfsetstrokecolor{textcolor}%
\pgfsetfillcolor{textcolor}%
\pgftext[x=0.475846in, y=2.900921in, left, base]{\color{textcolor}\rmfamily\fontsize{10.000000}{12.000000}\selectfont \(\displaystyle {140}\)}%
\end{pgfscope}%
\begin{pgfscope}%
\definecolor{textcolor}{rgb}{0.000000,0.000000,0.000000}%
\pgfsetstrokecolor{textcolor}%
\pgfsetfillcolor{textcolor}%
\pgftext[x=0.420291in,y=1.740574in,,bottom,rotate=90.000000]{\color{textcolor}\rmfamily\fontsize{10.000000}{12.000000}\selectfont Elapsed time (milliseconds)}%
\end{pgfscope}%
\begin{pgfscope}%
\pgfpathrectangle{\pgfqpoint{0.781402in}{0.386794in}}{\pgfqpoint{4.844695in}{2.707560in}}%
\pgfusepath{clip}%
\pgfsetrectcap%
\pgfsetroundjoin%
\pgfsetlinewidth{1.505625pt}%
\definecolor{currentstroke}{rgb}{0.121569,0.466667,0.705882}%
\pgfsetstrokecolor{currentstroke}%
\pgfsetdash{}{0pt}%
\pgfpathmoveto{\pgfqpoint{1.001616in}{0.566694in}}%
\pgfpathlineto{\pgfqpoint{1.422124in}{0.773964in}}%
\pgfpathlineto{\pgfqpoint{1.864764in}{1.022992in}}%
\pgfpathlineto{\pgfqpoint{2.307404in}{1.254398in}}%
\pgfpathlineto{\pgfqpoint{2.750044in}{1.510803in}}%
\pgfpathlineto{\pgfqpoint{3.192684in}{1.724812in}}%
\pgfpathlineto{\pgfqpoint{3.635324in}{1.954802in}}%
\pgfpathlineto{\pgfqpoint{4.077964in}{2.190726in}}%
\pgfpathlineto{\pgfqpoint{4.520604in}{2.416326in}}%
\pgfpathlineto{\pgfqpoint{4.963245in}{2.656944in}}%
\pgfpathlineto{\pgfqpoint{5.405885in}{2.929349in}}%
\pgfusepath{stroke}%
\end{pgfscope}%
\begin{pgfscope}%
\pgfpathrectangle{\pgfqpoint{0.781402in}{0.386794in}}{\pgfqpoint{4.844695in}{2.707560in}}%
\pgfusepath{clip}%
\pgfsetbuttcap%
\pgfsetroundjoin%
\definecolor{currentfill}{rgb}{0.121569,0.466667,0.705882}%
\pgfsetfillcolor{currentfill}%
\pgfsetlinewidth{1.003750pt}%
\definecolor{currentstroke}{rgb}{0.121569,0.466667,0.705882}%
\pgfsetstrokecolor{currentstroke}%
\pgfsetdash{}{0pt}%
\pgfsys@defobject{currentmarker}{\pgfqpoint{-0.041667in}{-0.041667in}}{\pgfqpoint{0.041667in}{0.041667in}}{%
\pgfpathmoveto{\pgfqpoint{0.000000in}{-0.041667in}}%
\pgfpathcurveto{\pgfqpoint{0.011050in}{-0.041667in}}{\pgfqpoint{0.021649in}{-0.037276in}}{\pgfqpoint{0.029463in}{-0.029463in}}%
\pgfpathcurveto{\pgfqpoint{0.037276in}{-0.021649in}}{\pgfqpoint{0.041667in}{-0.011050in}}{\pgfqpoint{0.041667in}{0.000000in}}%
\pgfpathcurveto{\pgfqpoint{0.041667in}{0.011050in}}{\pgfqpoint{0.037276in}{0.021649in}}{\pgfqpoint{0.029463in}{0.029463in}}%
\pgfpathcurveto{\pgfqpoint{0.021649in}{0.037276in}}{\pgfqpoint{0.011050in}{0.041667in}}{\pgfqpoint{0.000000in}{0.041667in}}%
\pgfpathcurveto{\pgfqpoint{-0.011050in}{0.041667in}}{\pgfqpoint{-0.021649in}{0.037276in}}{\pgfqpoint{-0.029463in}{0.029463in}}%
\pgfpathcurveto{\pgfqpoint{-0.037276in}{0.021649in}}{\pgfqpoint{-0.041667in}{0.011050in}}{\pgfqpoint{-0.041667in}{0.000000in}}%
\pgfpathcurveto{\pgfqpoint{-0.041667in}{-0.011050in}}{\pgfqpoint{-0.037276in}{-0.021649in}}{\pgfqpoint{-0.029463in}{-0.029463in}}%
\pgfpathcurveto{\pgfqpoint{-0.021649in}{-0.037276in}}{\pgfqpoint{-0.011050in}{-0.041667in}}{\pgfqpoint{0.000000in}{-0.041667in}}%
\pgfpathclose%
\pgfusepath{stroke,fill}%
}%
\begin{pgfscope}%
\pgfsys@transformshift{1.001616in}{0.566694in}%
\pgfsys@useobject{currentmarker}{}%
\end{pgfscope}%
\begin{pgfscope}%
\pgfsys@transformshift{1.422124in}{0.773964in}%
\pgfsys@useobject{currentmarker}{}%
\end{pgfscope}%
\begin{pgfscope}%
\pgfsys@transformshift{1.864764in}{1.022992in}%
\pgfsys@useobject{currentmarker}{}%
\end{pgfscope}%
\begin{pgfscope}%
\pgfsys@transformshift{2.307404in}{1.254398in}%
\pgfsys@useobject{currentmarker}{}%
\end{pgfscope}%
\begin{pgfscope}%
\pgfsys@transformshift{2.750044in}{1.510803in}%
\pgfsys@useobject{currentmarker}{}%
\end{pgfscope}%
\begin{pgfscope}%
\pgfsys@transformshift{3.192684in}{1.724812in}%
\pgfsys@useobject{currentmarker}{}%
\end{pgfscope}%
\begin{pgfscope}%
\pgfsys@transformshift{3.635324in}{1.954802in}%
\pgfsys@useobject{currentmarker}{}%
\end{pgfscope}%
\begin{pgfscope}%
\pgfsys@transformshift{4.077964in}{2.190726in}%
\pgfsys@useobject{currentmarker}{}%
\end{pgfscope}%
\begin{pgfscope}%
\pgfsys@transformshift{4.520604in}{2.416326in}%
\pgfsys@useobject{currentmarker}{}%
\end{pgfscope}%
\begin{pgfscope}%
\pgfsys@transformshift{4.963245in}{2.656944in}%
\pgfsys@useobject{currentmarker}{}%
\end{pgfscope}%
\begin{pgfscope}%
\pgfsys@transformshift{5.405885in}{2.929349in}%
\pgfsys@useobject{currentmarker}{}%
\end{pgfscope}%
\end{pgfscope}%
\begin{pgfscope}%
\pgfpathrectangle{\pgfqpoint{0.781402in}{0.386794in}}{\pgfqpoint{4.844695in}{2.707560in}}%
\pgfusepath{clip}%
\pgfsetrectcap%
\pgfsetroundjoin%
\pgfsetlinewidth{1.505625pt}%
\definecolor{currentstroke}{rgb}{1.000000,0.498039,0.054902}%
\pgfsetstrokecolor{currentstroke}%
\pgfsetdash{}{0pt}%
\pgfpathmoveto{\pgfqpoint{1.001616in}{0.509869in}}%
\pgfpathlineto{\pgfqpoint{1.422124in}{0.509867in}}%
\pgfpathlineto{\pgfqpoint{1.864764in}{0.509873in}}%
\pgfpathlineto{\pgfqpoint{2.307404in}{0.509871in}}%
\pgfpathlineto{\pgfqpoint{2.750044in}{0.509871in}}%
\pgfpathlineto{\pgfqpoint{3.192684in}{0.509865in}}%
\pgfpathlineto{\pgfqpoint{3.635324in}{0.509866in}}%
\pgfpathlineto{\pgfqpoint{4.077964in}{0.509870in}}%
\pgfpathlineto{\pgfqpoint{4.520604in}{0.509867in}}%
\pgfpathlineto{\pgfqpoint{4.963245in}{0.509867in}}%
\pgfpathlineto{\pgfqpoint{5.405885in}{0.509867in}}%
\pgfusepath{stroke}%
\end{pgfscope}%
\begin{pgfscope}%
\pgfpathrectangle{\pgfqpoint{0.781402in}{0.386794in}}{\pgfqpoint{4.844695in}{2.707560in}}%
\pgfusepath{clip}%
\pgfsetbuttcap%
\pgfsetroundjoin%
\definecolor{currentfill}{rgb}{1.000000,0.498039,0.054902}%
\pgfsetfillcolor{currentfill}%
\pgfsetlinewidth{1.003750pt}%
\definecolor{currentstroke}{rgb}{1.000000,0.498039,0.054902}%
\pgfsetstrokecolor{currentstroke}%
\pgfsetdash{}{0pt}%
\pgfsys@defobject{currentmarker}{\pgfqpoint{-0.041667in}{-0.041667in}}{\pgfqpoint{0.041667in}{0.041667in}}{%
\pgfpathmoveto{\pgfqpoint{0.000000in}{-0.041667in}}%
\pgfpathcurveto{\pgfqpoint{0.011050in}{-0.041667in}}{\pgfqpoint{0.021649in}{-0.037276in}}{\pgfqpoint{0.029463in}{-0.029463in}}%
\pgfpathcurveto{\pgfqpoint{0.037276in}{-0.021649in}}{\pgfqpoint{0.041667in}{-0.011050in}}{\pgfqpoint{0.041667in}{0.000000in}}%
\pgfpathcurveto{\pgfqpoint{0.041667in}{0.011050in}}{\pgfqpoint{0.037276in}{0.021649in}}{\pgfqpoint{0.029463in}{0.029463in}}%
\pgfpathcurveto{\pgfqpoint{0.021649in}{0.037276in}}{\pgfqpoint{0.011050in}{0.041667in}}{\pgfqpoint{0.000000in}{0.041667in}}%
\pgfpathcurveto{\pgfqpoint{-0.011050in}{0.041667in}}{\pgfqpoint{-0.021649in}{0.037276in}}{\pgfqpoint{-0.029463in}{0.029463in}}%
\pgfpathcurveto{\pgfqpoint{-0.037276in}{0.021649in}}{\pgfqpoint{-0.041667in}{0.011050in}}{\pgfqpoint{-0.041667in}{0.000000in}}%
\pgfpathcurveto{\pgfqpoint{-0.041667in}{-0.011050in}}{\pgfqpoint{-0.037276in}{-0.021649in}}{\pgfqpoint{-0.029463in}{-0.029463in}}%
\pgfpathcurveto{\pgfqpoint{-0.021649in}{-0.037276in}}{\pgfqpoint{-0.011050in}{-0.041667in}}{\pgfqpoint{0.000000in}{-0.041667in}}%
\pgfpathclose%
\pgfusepath{stroke,fill}%
}%
\begin{pgfscope}%
\pgfsys@transformshift{1.001616in}{0.509869in}%
\pgfsys@useobject{currentmarker}{}%
\end{pgfscope}%
\begin{pgfscope}%
\pgfsys@transformshift{1.422124in}{0.509867in}%
\pgfsys@useobject{currentmarker}{}%
\end{pgfscope}%
\begin{pgfscope}%
\pgfsys@transformshift{1.864764in}{0.509873in}%
\pgfsys@useobject{currentmarker}{}%
\end{pgfscope}%
\begin{pgfscope}%
\pgfsys@transformshift{2.307404in}{0.509871in}%
\pgfsys@useobject{currentmarker}{}%
\end{pgfscope}%
\begin{pgfscope}%
\pgfsys@transformshift{2.750044in}{0.509871in}%
\pgfsys@useobject{currentmarker}{}%
\end{pgfscope}%
\begin{pgfscope}%
\pgfsys@transformshift{3.192684in}{0.509865in}%
\pgfsys@useobject{currentmarker}{}%
\end{pgfscope}%
\begin{pgfscope}%
\pgfsys@transformshift{3.635324in}{0.509866in}%
\pgfsys@useobject{currentmarker}{}%
\end{pgfscope}%
\begin{pgfscope}%
\pgfsys@transformshift{4.077964in}{0.509870in}%
\pgfsys@useobject{currentmarker}{}%
\end{pgfscope}%
\begin{pgfscope}%
\pgfsys@transformshift{4.520604in}{0.509867in}%
\pgfsys@useobject{currentmarker}{}%
\end{pgfscope}%
\begin{pgfscope}%
\pgfsys@transformshift{4.963245in}{0.509867in}%
\pgfsys@useobject{currentmarker}{}%
\end{pgfscope}%
\begin{pgfscope}%
\pgfsys@transformshift{5.405885in}{0.509867in}%
\pgfsys@useobject{currentmarker}{}%
\end{pgfscope}%
\end{pgfscope}%
\begin{pgfscope}%
\pgfpathrectangle{\pgfqpoint{0.781402in}{0.386794in}}{\pgfqpoint{4.844695in}{2.707560in}}%
\pgfusepath{clip}%
\pgfsetrectcap%
\pgfsetroundjoin%
\pgfsetlinewidth{1.505625pt}%
\definecolor{currentstroke}{rgb}{0.172549,0.627451,0.172549}%
\pgfsetstrokecolor{currentstroke}%
\pgfsetdash{}{0pt}%
\pgfpathmoveto{\pgfqpoint{1.001616in}{0.567753in}}%
\pgfpathlineto{\pgfqpoint{1.422124in}{0.778856in}}%
\pgfpathlineto{\pgfqpoint{1.864764in}{1.032415in}}%
\pgfpathlineto{\pgfqpoint{2.307404in}{1.267426in}}%
\pgfpathlineto{\pgfqpoint{2.750044in}{1.528306in}}%
\pgfpathlineto{\pgfqpoint{3.192684in}{1.745618in}}%
\pgfpathlineto{\pgfqpoint{3.635324in}{1.980118in}}%
\pgfpathlineto{\pgfqpoint{4.077964in}{2.219946in}}%
\pgfpathlineto{\pgfqpoint{4.520604in}{2.449728in}}%
\pgfpathlineto{\pgfqpoint{4.963245in}{2.694591in}}%
\pgfpathlineto{\pgfqpoint{5.405885in}{2.971283in}}%
\pgfusepath{stroke}%
\end{pgfscope}%
\begin{pgfscope}%
\pgfpathrectangle{\pgfqpoint{0.781402in}{0.386794in}}{\pgfqpoint{4.844695in}{2.707560in}}%
\pgfusepath{clip}%
\pgfsetbuttcap%
\pgfsetroundjoin%
\definecolor{currentfill}{rgb}{0.172549,0.627451,0.172549}%
\pgfsetfillcolor{currentfill}%
\pgfsetlinewidth{1.003750pt}%
\definecolor{currentstroke}{rgb}{0.172549,0.627451,0.172549}%
\pgfsetstrokecolor{currentstroke}%
\pgfsetdash{}{0pt}%
\pgfsys@defobject{currentmarker}{\pgfqpoint{-0.041667in}{-0.041667in}}{\pgfqpoint{0.041667in}{0.041667in}}{%
\pgfpathmoveto{\pgfqpoint{0.000000in}{-0.041667in}}%
\pgfpathcurveto{\pgfqpoint{0.011050in}{-0.041667in}}{\pgfqpoint{0.021649in}{-0.037276in}}{\pgfqpoint{0.029463in}{-0.029463in}}%
\pgfpathcurveto{\pgfqpoint{0.037276in}{-0.021649in}}{\pgfqpoint{0.041667in}{-0.011050in}}{\pgfqpoint{0.041667in}{0.000000in}}%
\pgfpathcurveto{\pgfqpoint{0.041667in}{0.011050in}}{\pgfqpoint{0.037276in}{0.021649in}}{\pgfqpoint{0.029463in}{0.029463in}}%
\pgfpathcurveto{\pgfqpoint{0.021649in}{0.037276in}}{\pgfqpoint{0.011050in}{0.041667in}}{\pgfqpoint{0.000000in}{0.041667in}}%
\pgfpathcurveto{\pgfqpoint{-0.011050in}{0.041667in}}{\pgfqpoint{-0.021649in}{0.037276in}}{\pgfqpoint{-0.029463in}{0.029463in}}%
\pgfpathcurveto{\pgfqpoint{-0.037276in}{0.021649in}}{\pgfqpoint{-0.041667in}{0.011050in}}{\pgfqpoint{-0.041667in}{0.000000in}}%
\pgfpathcurveto{\pgfqpoint{-0.041667in}{-0.011050in}}{\pgfqpoint{-0.037276in}{-0.021649in}}{\pgfqpoint{-0.029463in}{-0.029463in}}%
\pgfpathcurveto{\pgfqpoint{-0.021649in}{-0.037276in}}{\pgfqpoint{-0.011050in}{-0.041667in}}{\pgfqpoint{0.000000in}{-0.041667in}}%
\pgfpathclose%
\pgfusepath{stroke,fill}%
}%
\begin{pgfscope}%
\pgfsys@transformshift{1.001616in}{0.567753in}%
\pgfsys@useobject{currentmarker}{}%
\end{pgfscope}%
\begin{pgfscope}%
\pgfsys@transformshift{1.422124in}{0.778856in}%
\pgfsys@useobject{currentmarker}{}%
\end{pgfscope}%
\begin{pgfscope}%
\pgfsys@transformshift{1.864764in}{1.032415in}%
\pgfsys@useobject{currentmarker}{}%
\end{pgfscope}%
\begin{pgfscope}%
\pgfsys@transformshift{2.307404in}{1.267426in}%
\pgfsys@useobject{currentmarker}{}%
\end{pgfscope}%
\begin{pgfscope}%
\pgfsys@transformshift{2.750044in}{1.528306in}%
\pgfsys@useobject{currentmarker}{}%
\end{pgfscope}%
\begin{pgfscope}%
\pgfsys@transformshift{3.192684in}{1.745618in}%
\pgfsys@useobject{currentmarker}{}%
\end{pgfscope}%
\begin{pgfscope}%
\pgfsys@transformshift{3.635324in}{1.980118in}%
\pgfsys@useobject{currentmarker}{}%
\end{pgfscope}%
\begin{pgfscope}%
\pgfsys@transformshift{4.077964in}{2.219946in}%
\pgfsys@useobject{currentmarker}{}%
\end{pgfscope}%
\begin{pgfscope}%
\pgfsys@transformshift{4.520604in}{2.449728in}%
\pgfsys@useobject{currentmarker}{}%
\end{pgfscope}%
\begin{pgfscope}%
\pgfsys@transformshift{4.963245in}{2.694591in}%
\pgfsys@useobject{currentmarker}{}%
\end{pgfscope}%
\begin{pgfscope}%
\pgfsys@transformshift{5.405885in}{2.971283in}%
\pgfsys@useobject{currentmarker}{}%
\end{pgfscope}%
\end{pgfscope}%
\begin{pgfscope}%
\pgfsetrectcap%
\pgfsetmiterjoin%
\pgfsetlinewidth{0.803000pt}%
\definecolor{currentstroke}{rgb}{0.000000,0.000000,0.000000}%
\pgfsetstrokecolor{currentstroke}%
\pgfsetdash{}{0pt}%
\pgfpathmoveto{\pgfqpoint{0.781402in}{0.386794in}}%
\pgfpathlineto{\pgfqpoint{0.781402in}{3.094354in}}%
\pgfusepath{stroke}%
\end{pgfscope}%
\begin{pgfscope}%
\pgfsetrectcap%
\pgfsetmiterjoin%
\pgfsetlinewidth{0.803000pt}%
\definecolor{currentstroke}{rgb}{0.000000,0.000000,0.000000}%
\pgfsetstrokecolor{currentstroke}%
\pgfsetdash{}{0pt}%
\pgfpathmoveto{\pgfqpoint{5.626098in}{0.386794in}}%
\pgfpathlineto{\pgfqpoint{5.626098in}{3.094354in}}%
\pgfusepath{stroke}%
\end{pgfscope}%
\begin{pgfscope}%
\pgfsetrectcap%
\pgfsetmiterjoin%
\pgfsetlinewidth{0.803000pt}%
\definecolor{currentstroke}{rgb}{0.000000,0.000000,0.000000}%
\pgfsetstrokecolor{currentstroke}%
\pgfsetdash{}{0pt}%
\pgfpathmoveto{\pgfqpoint{0.781402in}{0.386794in}}%
\pgfpathlineto{\pgfqpoint{5.626098in}{0.386794in}}%
\pgfusepath{stroke}%
\end{pgfscope}%
\begin{pgfscope}%
\pgfsetrectcap%
\pgfsetmiterjoin%
\pgfsetlinewidth{0.803000pt}%
\definecolor{currentstroke}{rgb}{0.000000,0.000000,0.000000}%
\pgfsetstrokecolor{currentstroke}%
\pgfsetdash{}{0pt}%
\pgfpathmoveto{\pgfqpoint{0.781402in}{3.094354in}}%
\pgfpathlineto{\pgfqpoint{5.626098in}{3.094354in}}%
\pgfusepath{stroke}%
\end{pgfscope}%
\begin{pgfscope}%
\pgfsetbuttcap%
\pgfsetmiterjoin%
\definecolor{currentfill}{rgb}{1.000000,1.000000,1.000000}%
\pgfsetfillcolor{currentfill}%
\pgfsetfillopacity{0.800000}%
\pgfsetlinewidth{1.003750pt}%
\definecolor{currentstroke}{rgb}{0.800000,0.800000,0.800000}%
\pgfsetstrokecolor{currentstroke}%
\pgfsetstrokeopacity{0.800000}%
\pgfsetdash{}{0pt}%
\pgfpathmoveto{\pgfqpoint{0.878625in}{2.402224in}}%
\pgfpathlineto{\pgfqpoint{2.791822in}{2.402224in}}%
\pgfpathquadraticcurveto{\pgfqpoint{2.819600in}{2.402224in}}{\pgfqpoint{2.819600in}{2.430002in}}%
\pgfpathlineto{\pgfqpoint{2.819600in}{2.997132in}}%
\pgfpathquadraticcurveto{\pgfqpoint{2.819600in}{3.024909in}}{\pgfqpoint{2.791822in}{3.024909in}}%
\pgfpathlineto{\pgfqpoint{0.878625in}{3.024909in}}%
\pgfpathquadraticcurveto{\pgfqpoint{0.850847in}{3.024909in}}{\pgfqpoint{0.850847in}{2.997132in}}%
\pgfpathlineto{\pgfqpoint{0.850847in}{2.430002in}}%
\pgfpathquadraticcurveto{\pgfqpoint{0.850847in}{2.402224in}}{\pgfqpoint{0.878625in}{2.402224in}}%
\pgfpathclose%
\pgfusepath{stroke,fill}%
\end{pgfscope}%
\begin{pgfscope}%
\pgfsetrectcap%
\pgfsetroundjoin%
\pgfsetlinewidth{1.505625pt}%
\definecolor{currentstroke}{rgb}{0.121569,0.466667,0.705882}%
\pgfsetstrokecolor{currentstroke}%
\pgfsetdash{}{0pt}%
\pgfpathmoveto{\pgfqpoint{0.906402in}{2.920743in}}%
\pgfpathlineto{\pgfqpoint{1.184180in}{2.920743in}}%
\pgfusepath{stroke}%
\end{pgfscope}%
\begin{pgfscope}%
\pgfsetbuttcap%
\pgfsetroundjoin%
\definecolor{currentfill}{rgb}{0.121569,0.466667,0.705882}%
\pgfsetfillcolor{currentfill}%
\pgfsetlinewidth{1.003750pt}%
\definecolor{currentstroke}{rgb}{0.121569,0.466667,0.705882}%
\pgfsetstrokecolor{currentstroke}%
\pgfsetdash{}{0pt}%
\pgfsys@defobject{currentmarker}{\pgfqpoint{-0.041667in}{-0.041667in}}{\pgfqpoint{0.041667in}{0.041667in}}{%
\pgfpathmoveto{\pgfqpoint{0.000000in}{-0.041667in}}%
\pgfpathcurveto{\pgfqpoint{0.011050in}{-0.041667in}}{\pgfqpoint{0.021649in}{-0.037276in}}{\pgfqpoint{0.029463in}{-0.029463in}}%
\pgfpathcurveto{\pgfqpoint{0.037276in}{-0.021649in}}{\pgfqpoint{0.041667in}{-0.011050in}}{\pgfqpoint{0.041667in}{0.000000in}}%
\pgfpathcurveto{\pgfqpoint{0.041667in}{0.011050in}}{\pgfqpoint{0.037276in}{0.021649in}}{\pgfqpoint{0.029463in}{0.029463in}}%
\pgfpathcurveto{\pgfqpoint{0.021649in}{0.037276in}}{\pgfqpoint{0.011050in}{0.041667in}}{\pgfqpoint{0.000000in}{0.041667in}}%
\pgfpathcurveto{\pgfqpoint{-0.011050in}{0.041667in}}{\pgfqpoint{-0.021649in}{0.037276in}}{\pgfqpoint{-0.029463in}{0.029463in}}%
\pgfpathcurveto{\pgfqpoint{-0.037276in}{0.021649in}}{\pgfqpoint{-0.041667in}{0.011050in}}{\pgfqpoint{-0.041667in}{0.000000in}}%
\pgfpathcurveto{\pgfqpoint{-0.041667in}{-0.011050in}}{\pgfqpoint{-0.037276in}{-0.021649in}}{\pgfqpoint{-0.029463in}{-0.029463in}}%
\pgfpathcurveto{\pgfqpoint{-0.021649in}{-0.037276in}}{\pgfqpoint{-0.011050in}{-0.041667in}}{\pgfqpoint{0.000000in}{-0.041667in}}%
\pgfpathclose%
\pgfusepath{stroke,fill}%
}%
\begin{pgfscope}%
\pgfsys@transformshift{1.045291in}{2.920743in}%
\pgfsys@useobject{currentmarker}{}%
\end{pgfscope}%
\end{pgfscope}%
\begin{pgfscope}%
\definecolor{textcolor}{rgb}{0.000000,0.000000,0.000000}%
\pgfsetstrokecolor{textcolor}%
\pgfsetfillcolor{textcolor}%
\pgftext[x=1.295291in,y=2.872132in,left,base]{\color{textcolor}\rmfamily\fontsize{10.000000}{12.000000}\selectfont Prefill duration}%
\end{pgfscope}%
\begin{pgfscope}%
\pgfsetrectcap%
\pgfsetroundjoin%
\pgfsetlinewidth{1.505625pt}%
\definecolor{currentstroke}{rgb}{1.000000,0.498039,0.054902}%
\pgfsetstrokecolor{currentstroke}%
\pgfsetdash{}{0pt}%
\pgfpathmoveto{\pgfqpoint{0.906402in}{2.727070in}}%
\pgfpathlineto{\pgfqpoint{1.184180in}{2.727070in}}%
\pgfusepath{stroke}%
\end{pgfscope}%
\begin{pgfscope}%
\pgfsetbuttcap%
\pgfsetroundjoin%
\definecolor{currentfill}{rgb}{1.000000,0.498039,0.054902}%
\pgfsetfillcolor{currentfill}%
\pgfsetlinewidth{1.003750pt}%
\definecolor{currentstroke}{rgb}{1.000000,0.498039,0.054902}%
\pgfsetstrokecolor{currentstroke}%
\pgfsetdash{}{0pt}%
\pgfsys@defobject{currentmarker}{\pgfqpoint{-0.041667in}{-0.041667in}}{\pgfqpoint{0.041667in}{0.041667in}}{%
\pgfpathmoveto{\pgfqpoint{0.000000in}{-0.041667in}}%
\pgfpathcurveto{\pgfqpoint{0.011050in}{-0.041667in}}{\pgfqpoint{0.021649in}{-0.037276in}}{\pgfqpoint{0.029463in}{-0.029463in}}%
\pgfpathcurveto{\pgfqpoint{0.037276in}{-0.021649in}}{\pgfqpoint{0.041667in}{-0.011050in}}{\pgfqpoint{0.041667in}{0.000000in}}%
\pgfpathcurveto{\pgfqpoint{0.041667in}{0.011050in}}{\pgfqpoint{0.037276in}{0.021649in}}{\pgfqpoint{0.029463in}{0.029463in}}%
\pgfpathcurveto{\pgfqpoint{0.021649in}{0.037276in}}{\pgfqpoint{0.011050in}{0.041667in}}{\pgfqpoint{0.000000in}{0.041667in}}%
\pgfpathcurveto{\pgfqpoint{-0.011050in}{0.041667in}}{\pgfqpoint{-0.021649in}{0.037276in}}{\pgfqpoint{-0.029463in}{0.029463in}}%
\pgfpathcurveto{\pgfqpoint{-0.037276in}{0.021649in}}{\pgfqpoint{-0.041667in}{0.011050in}}{\pgfqpoint{-0.041667in}{0.000000in}}%
\pgfpathcurveto{\pgfqpoint{-0.041667in}{-0.011050in}}{\pgfqpoint{-0.037276in}{-0.021649in}}{\pgfqpoint{-0.029463in}{-0.029463in}}%
\pgfpathcurveto{\pgfqpoint{-0.021649in}{-0.037276in}}{\pgfqpoint{-0.011050in}{-0.041667in}}{\pgfqpoint{0.000000in}{-0.041667in}}%
\pgfpathclose%
\pgfusepath{stroke,fill}%
}%
\begin{pgfscope}%
\pgfsys@transformshift{1.045291in}{2.727070in}%
\pgfsys@useobject{currentmarker}{}%
\end{pgfscope}%
\end{pgfscope}%
\begin{pgfscope}%
\definecolor{textcolor}{rgb}{0.000000,0.000000,0.000000}%
\pgfsetstrokecolor{textcolor}%
\pgfsetfillcolor{textcolor}%
\pgftext[x=1.295291in,y=2.678459in,left,base]{\color{textcolor}\rmfamily\fontsize{10.000000}{12.000000}\selectfont Duration without owner}%
\end{pgfscope}%
\begin{pgfscope}%
\pgfsetrectcap%
\pgfsetroundjoin%
\pgfsetlinewidth{1.505625pt}%
\definecolor{currentstroke}{rgb}{0.172549,0.627451,0.172549}%
\pgfsetstrokecolor{currentstroke}%
\pgfsetdash{}{0pt}%
\pgfpathmoveto{\pgfqpoint{0.906402in}{2.533397in}}%
\pgfpathlineto{\pgfqpoint{1.184180in}{2.533397in}}%
\pgfusepath{stroke}%
\end{pgfscope}%
\begin{pgfscope}%
\pgfsetbuttcap%
\pgfsetroundjoin%
\definecolor{currentfill}{rgb}{0.172549,0.627451,0.172549}%
\pgfsetfillcolor{currentfill}%
\pgfsetlinewidth{1.003750pt}%
\definecolor{currentstroke}{rgb}{0.172549,0.627451,0.172549}%
\pgfsetstrokecolor{currentstroke}%
\pgfsetdash{}{0pt}%
\pgfsys@defobject{currentmarker}{\pgfqpoint{-0.041667in}{-0.041667in}}{\pgfqpoint{0.041667in}{0.041667in}}{%
\pgfpathmoveto{\pgfqpoint{0.000000in}{-0.041667in}}%
\pgfpathcurveto{\pgfqpoint{0.011050in}{-0.041667in}}{\pgfqpoint{0.021649in}{-0.037276in}}{\pgfqpoint{0.029463in}{-0.029463in}}%
\pgfpathcurveto{\pgfqpoint{0.037276in}{-0.021649in}}{\pgfqpoint{0.041667in}{-0.011050in}}{\pgfqpoint{0.041667in}{0.000000in}}%
\pgfpathcurveto{\pgfqpoint{0.041667in}{0.011050in}}{\pgfqpoint{0.037276in}{0.021649in}}{\pgfqpoint{0.029463in}{0.029463in}}%
\pgfpathcurveto{\pgfqpoint{0.021649in}{0.037276in}}{\pgfqpoint{0.011050in}{0.041667in}}{\pgfqpoint{0.000000in}{0.041667in}}%
\pgfpathcurveto{\pgfqpoint{-0.011050in}{0.041667in}}{\pgfqpoint{-0.021649in}{0.037276in}}{\pgfqpoint{-0.029463in}{0.029463in}}%
\pgfpathcurveto{\pgfqpoint{-0.037276in}{0.021649in}}{\pgfqpoint{-0.041667in}{0.011050in}}{\pgfqpoint{-0.041667in}{0.000000in}}%
\pgfpathcurveto{\pgfqpoint{-0.041667in}{-0.011050in}}{\pgfqpoint{-0.037276in}{-0.021649in}}{\pgfqpoint{-0.029463in}{-0.029463in}}%
\pgfpathcurveto{\pgfqpoint{-0.021649in}{-0.037276in}}{\pgfqpoint{-0.011050in}{-0.041667in}}{\pgfqpoint{0.000000in}{-0.041667in}}%
\pgfpathclose%
\pgfusepath{stroke,fill}%
}%
\begin{pgfscope}%
\pgfsys@transformshift{1.045291in}{2.533397in}%
\pgfsys@useobject{currentmarker}{}%
\end{pgfscope}%
\end{pgfscope}%
\begin{pgfscope}%
\definecolor{textcolor}{rgb}{0.000000,0.000000,0.000000}%
\pgfsetstrokecolor{textcolor}%
\pgfsetfillcolor{textcolor}%
\pgftext[x=1.295291in,y=2.484786in,left,base]{\color{textcolor}\rmfamily\fontsize{10.000000}{12.000000}\selectfont End to end latency}%
\end{pgfscope}%
\end{pgfpicture}%
\makeatother%
\endgroup%

    \end{center}
    \caption{Migration statistics of a clean vector (2MB huge pages)}
    \label{fig:vectorreadonlyhp}
\end{figure}

In our simplest example, we create a \texttt{vector}, initialize it with the
pre-specified size and migrate it to the destination. Approximately $8$ lines
of code are required on each of the source and destination sides to reproduce
this operation, excluding lines that serve the purpose of gathering statistics.
\autoref{fig:vectorreadonly} depicts the results.

Naturally, the prefill phase takes up most of the transfer time, which grows
linearly by increasing the size of the object. The increasing gap between the
prefill duration and the end to end latency can be attributed to syscalls and
invocation of the signal handler, both of which increase linearly with the
number of pages that are transferred before or after the prefill phase. The
time it takes to turn over the ownership is not impacted by the
size of the object and oscillates between tens of microseconds and hundreds
of microseconds. This is expected as this step consists of a single RDMA SEND.

\autoref{fig:vectorreadonly} shows that we are using about a hundredth of our 100 Gbps
network bandwidth and that the prefill phase is highly CPU bound. To use the
network bandwidth more effectively, we can increase the granularity of our
memory allocation unit by using huge pages. \autoref{fig:vectorreadonlyhp} shows
how using huge pages allows us to use around a fifth of our network bandwidth.
Furthermore, a 500-fold decrease in the number of allocation units decreases
the overal CPU cycles we spend for per-page operations in the memory allocator
and the control and data planes.

\subsection{Migrating a vector while dirtying all of its pages}
\label{sec:dirtyvector}
\begin{figure}[tp]
    \begin{center}
        %% Creator: Matplotlib, PGF backend
%%
%% To include the figure in your LaTeX document, write
%%   \input{<filename>.pgf}
%%
%% Make sure the required packages are loaded in your preamble
%%   \usepackage{pgf}
%%
%% and, on pdftex
%%   \usepackage[utf8]{inputenc}\DeclareUnicodeCharacter{2212}{-}
%%
%% or, on luatex and xetex
%%   \usepackage{unicode-math}
%%
%% Figures using additional raster images can only be included by \input if
%% they are in the same directory as the main LaTeX file. For loading figures
%% from other directories you can use the `import` package
%%   \usepackage{import}
%%
%% and then include the figures with
%%   \import{<path to file>}{<filename>.pgf}
%%
%% Matplotlib used the following preamble
%%
\begingroup%
\makeatletter%
\begin{pgfpicture}%
\pgfpathrectangle{\pgfpointorigin}{\pgfqpoint{6.251220in}{3.516311in}}%
\pgfusepath{use as bounding box, clip}%
\begin{pgfscope}%
\pgfsetbuttcap%
\pgfsetmiterjoin%
\definecolor{currentfill}{rgb}{1.000000,1.000000,1.000000}%
\pgfsetfillcolor{currentfill}%
\pgfsetlinewidth{0.000000pt}%
\definecolor{currentstroke}{rgb}{1.000000,1.000000,1.000000}%
\pgfsetstrokecolor{currentstroke}%
\pgfsetdash{}{0pt}%
\pgfpathmoveto{\pgfqpoint{0.000000in}{0.000000in}}%
\pgfpathlineto{\pgfqpoint{6.251220in}{0.000000in}}%
\pgfpathlineto{\pgfqpoint{6.251220in}{3.516311in}}%
\pgfpathlineto{\pgfqpoint{0.000000in}{3.516311in}}%
\pgfpathclose%
\pgfusepath{fill}%
\end{pgfscope}%
\begin{pgfscope}%
\pgfsetbuttcap%
\pgfsetmiterjoin%
\definecolor{currentfill}{rgb}{1.000000,1.000000,1.000000}%
\pgfsetfillcolor{currentfill}%
\pgfsetlinewidth{0.000000pt}%
\definecolor{currentstroke}{rgb}{0.000000,0.000000,0.000000}%
\pgfsetstrokecolor{currentstroke}%
\pgfsetstrokeopacity{0.000000}%
\pgfsetdash{}{0pt}%
\pgfpathmoveto{\pgfqpoint{0.781402in}{0.386794in}}%
\pgfpathlineto{\pgfqpoint{5.626098in}{0.386794in}}%
\pgfpathlineto{\pgfqpoint{5.626098in}{3.094354in}}%
\pgfpathlineto{\pgfqpoint{0.781402in}{3.094354in}}%
\pgfpathclose%
\pgfusepath{fill}%
\end{pgfscope}%
\begin{pgfscope}%
\pgfsetbuttcap%
\pgfsetroundjoin%
\definecolor{currentfill}{rgb}{0.000000,0.000000,0.000000}%
\pgfsetfillcolor{currentfill}%
\pgfsetlinewidth{0.803000pt}%
\definecolor{currentstroke}{rgb}{0.000000,0.000000,0.000000}%
\pgfsetstrokecolor{currentstroke}%
\pgfsetdash{}{0pt}%
\pgfsys@defobject{currentmarker}{\pgfqpoint{0.000000in}{-0.048611in}}{\pgfqpoint{0.000000in}{0.000000in}}{%
\pgfpathmoveto{\pgfqpoint{0.000000in}{0.000000in}}%
\pgfpathlineto{\pgfqpoint{0.000000in}{-0.048611in}}%
\pgfusepath{stroke,fill}%
}%
\begin{pgfscope}%
\pgfsys@transformshift{1.001616in}{0.386794in}%
\pgfsys@useobject{currentmarker}{}%
\end{pgfscope}%
\end{pgfscope}%
\begin{pgfscope}%
\definecolor{textcolor}{rgb}{0.000000,0.000000,0.000000}%
\pgfsetstrokecolor{textcolor}%
\pgfsetfillcolor{textcolor}%
\pgftext[x=1.001616in,y=0.289572in,,top]{\color{textcolor}\rmfamily\fontsize{10.000000}{12.000000}\selectfont \(\displaystyle {1}\)}%
\end{pgfscope}%
\begin{pgfscope}%
\pgfsetbuttcap%
\pgfsetroundjoin%
\definecolor{currentfill}{rgb}{0.000000,0.000000,0.000000}%
\pgfsetfillcolor{currentfill}%
\pgfsetlinewidth{0.803000pt}%
\definecolor{currentstroke}{rgb}{0.000000,0.000000,0.000000}%
\pgfsetstrokecolor{currentstroke}%
\pgfsetdash{}{0pt}%
\pgfsys@defobject{currentmarker}{\pgfqpoint{0.000000in}{-0.048611in}}{\pgfqpoint{0.000000in}{0.000000in}}{%
\pgfpathmoveto{\pgfqpoint{0.000000in}{0.000000in}}%
\pgfpathlineto{\pgfqpoint{0.000000in}{-0.048611in}}%
\pgfusepath{stroke,fill}%
}%
\begin{pgfscope}%
\pgfsys@transformshift{1.882434in}{0.386794in}%
\pgfsys@useobject{currentmarker}{}%
\end{pgfscope}%
\end{pgfscope}%
\begin{pgfscope}%
\definecolor{textcolor}{rgb}{0.000000,0.000000,0.000000}%
\pgfsetstrokecolor{textcolor}%
\pgfsetfillcolor{textcolor}%
\pgftext[x=1.882434in,y=0.289572in,,top]{\color{textcolor}\rmfamily\fontsize{10.000000}{12.000000}\selectfont \(\displaystyle {20000}\)}%
\end{pgfscope}%
\begin{pgfscope}%
\pgfsetbuttcap%
\pgfsetroundjoin%
\definecolor{currentfill}{rgb}{0.000000,0.000000,0.000000}%
\pgfsetfillcolor{currentfill}%
\pgfsetlinewidth{0.803000pt}%
\definecolor{currentstroke}{rgb}{0.000000,0.000000,0.000000}%
\pgfsetstrokecolor{currentstroke}%
\pgfsetdash{}{0pt}%
\pgfsys@defobject{currentmarker}{\pgfqpoint{0.000000in}{-0.048611in}}{\pgfqpoint{0.000000in}{0.000000in}}{%
\pgfpathmoveto{\pgfqpoint{0.000000in}{0.000000in}}%
\pgfpathlineto{\pgfqpoint{0.000000in}{-0.048611in}}%
\pgfusepath{stroke,fill}%
}%
\begin{pgfscope}%
\pgfsys@transformshift{2.763297in}{0.386794in}%
\pgfsys@useobject{currentmarker}{}%
\end{pgfscope}%
\end{pgfscope}%
\begin{pgfscope}%
\definecolor{textcolor}{rgb}{0.000000,0.000000,0.000000}%
\pgfsetstrokecolor{textcolor}%
\pgfsetfillcolor{textcolor}%
\pgftext[x=2.763297in,y=0.289572in,,top]{\color{textcolor}\rmfamily\fontsize{10.000000}{12.000000}\selectfont \(\displaystyle {40000}\)}%
\end{pgfscope}%
\begin{pgfscope}%
\pgfsetbuttcap%
\pgfsetroundjoin%
\definecolor{currentfill}{rgb}{0.000000,0.000000,0.000000}%
\pgfsetfillcolor{currentfill}%
\pgfsetlinewidth{0.803000pt}%
\definecolor{currentstroke}{rgb}{0.000000,0.000000,0.000000}%
\pgfsetstrokecolor{currentstroke}%
\pgfsetdash{}{0pt}%
\pgfsys@defobject{currentmarker}{\pgfqpoint{0.000000in}{-0.048611in}}{\pgfqpoint{0.000000in}{0.000000in}}{%
\pgfpathmoveto{\pgfqpoint{0.000000in}{0.000000in}}%
\pgfpathlineto{\pgfqpoint{0.000000in}{-0.048611in}}%
\pgfusepath{stroke,fill}%
}%
\begin{pgfscope}%
\pgfsys@transformshift{3.644159in}{0.386794in}%
\pgfsys@useobject{currentmarker}{}%
\end{pgfscope}%
\end{pgfscope}%
\begin{pgfscope}%
\definecolor{textcolor}{rgb}{0.000000,0.000000,0.000000}%
\pgfsetstrokecolor{textcolor}%
\pgfsetfillcolor{textcolor}%
\pgftext[x=3.644159in,y=0.289572in,,top]{\color{textcolor}\rmfamily\fontsize{10.000000}{12.000000}\selectfont \(\displaystyle {60000}\)}%
\end{pgfscope}%
\begin{pgfscope}%
\pgfsetbuttcap%
\pgfsetroundjoin%
\definecolor{currentfill}{rgb}{0.000000,0.000000,0.000000}%
\pgfsetfillcolor{currentfill}%
\pgfsetlinewidth{0.803000pt}%
\definecolor{currentstroke}{rgb}{0.000000,0.000000,0.000000}%
\pgfsetstrokecolor{currentstroke}%
\pgfsetdash{}{0pt}%
\pgfsys@defobject{currentmarker}{\pgfqpoint{0.000000in}{-0.048611in}}{\pgfqpoint{0.000000in}{0.000000in}}{%
\pgfpathmoveto{\pgfqpoint{0.000000in}{0.000000in}}%
\pgfpathlineto{\pgfqpoint{0.000000in}{-0.048611in}}%
\pgfusepath{stroke,fill}%
}%
\begin{pgfscope}%
\pgfsys@transformshift{4.525022in}{0.386794in}%
\pgfsys@useobject{currentmarker}{}%
\end{pgfscope}%
\end{pgfscope}%
\begin{pgfscope}%
\definecolor{textcolor}{rgb}{0.000000,0.000000,0.000000}%
\pgfsetstrokecolor{textcolor}%
\pgfsetfillcolor{textcolor}%
\pgftext[x=4.525022in,y=0.289572in,,top]{\color{textcolor}\rmfamily\fontsize{10.000000}{12.000000}\selectfont \(\displaystyle {80000}\)}%
\end{pgfscope}%
\begin{pgfscope}%
\pgfsetbuttcap%
\pgfsetroundjoin%
\definecolor{currentfill}{rgb}{0.000000,0.000000,0.000000}%
\pgfsetfillcolor{currentfill}%
\pgfsetlinewidth{0.803000pt}%
\definecolor{currentstroke}{rgb}{0.000000,0.000000,0.000000}%
\pgfsetstrokecolor{currentstroke}%
\pgfsetdash{}{0pt}%
\pgfsys@defobject{currentmarker}{\pgfqpoint{0.000000in}{-0.048611in}}{\pgfqpoint{0.000000in}{0.000000in}}{%
\pgfpathmoveto{\pgfqpoint{0.000000in}{0.000000in}}%
\pgfpathlineto{\pgfqpoint{0.000000in}{-0.048611in}}%
\pgfusepath{stroke,fill}%
}%
\begin{pgfscope}%
\pgfsys@transformshift{5.405885in}{0.386794in}%
\pgfsys@useobject{currentmarker}{}%
\end{pgfscope}%
\end{pgfscope}%
\begin{pgfscope}%
\definecolor{textcolor}{rgb}{0.000000,0.000000,0.000000}%
\pgfsetstrokecolor{textcolor}%
\pgfsetfillcolor{textcolor}%
\pgftext[x=5.405885in,y=0.289572in,,top]{\color{textcolor}\rmfamily\fontsize{10.000000}{12.000000}\selectfont \(\displaystyle {100000}\)}%
\end{pgfscope}%
\begin{pgfscope}%
\definecolor{textcolor}{rgb}{0.000000,0.000000,0.000000}%
\pgfsetstrokecolor{textcolor}%
\pgfsetfillcolor{textcolor}%
\pgftext[x=3.203750in,y=0.110560in,,top]{\color{textcolor}\rmfamily\fontsize{10.000000}{12.000000}\selectfont Number of 4KB pages}%
\end{pgfscope}%
\begin{pgfscope}%
\pgfsetbuttcap%
\pgfsetroundjoin%
\definecolor{currentfill}{rgb}{0.000000,0.000000,0.000000}%
\pgfsetfillcolor{currentfill}%
\pgfsetlinewidth{0.803000pt}%
\definecolor{currentstroke}{rgb}{0.000000,0.000000,0.000000}%
\pgfsetstrokecolor{currentstroke}%
\pgfsetdash{}{0pt}%
\pgfsys@defobject{currentmarker}{\pgfqpoint{-0.048611in}{0.000000in}}{\pgfqpoint{-0.000000in}{0.000000in}}{%
\pgfpathmoveto{\pgfqpoint{-0.000000in}{0.000000in}}%
\pgfpathlineto{\pgfqpoint{-0.048611in}{0.000000in}}%
\pgfusepath{stroke,fill}%
}%
\begin{pgfscope}%
\pgfsys@transformshift{0.781402in}{0.509865in}%
\pgfsys@useobject{currentmarker}{}%
\end{pgfscope}%
\end{pgfscope}%
\begin{pgfscope}%
\definecolor{textcolor}{rgb}{0.000000,0.000000,0.000000}%
\pgfsetstrokecolor{textcolor}%
\pgfsetfillcolor{textcolor}%
\pgftext[x=0.614736in, y=0.461639in, left, base]{\color{textcolor}\rmfamily\fontsize{10.000000}{12.000000}\selectfont \(\displaystyle {0}\)}%
\end{pgfscope}%
\begin{pgfscope}%
\pgfsetbuttcap%
\pgfsetroundjoin%
\definecolor{currentfill}{rgb}{0.000000,0.000000,0.000000}%
\pgfsetfillcolor{currentfill}%
\pgfsetlinewidth{0.803000pt}%
\definecolor{currentstroke}{rgb}{0.000000,0.000000,0.000000}%
\pgfsetstrokecolor{currentstroke}%
\pgfsetdash{}{0pt}%
\pgfsys@defobject{currentmarker}{\pgfqpoint{-0.048611in}{0.000000in}}{\pgfqpoint{-0.000000in}{0.000000in}}{%
\pgfpathmoveto{\pgfqpoint{-0.000000in}{0.000000in}}%
\pgfpathlineto{\pgfqpoint{-0.048611in}{0.000000in}}%
\pgfusepath{stroke,fill}%
}%
\begin{pgfscope}%
\pgfsys@transformshift{0.781402in}{0.940778in}%
\pgfsys@useobject{currentmarker}{}%
\end{pgfscope}%
\end{pgfscope}%
\begin{pgfscope}%
\definecolor{textcolor}{rgb}{0.000000,0.000000,0.000000}%
\pgfsetstrokecolor{textcolor}%
\pgfsetfillcolor{textcolor}%
\pgftext[x=0.406402in, y=0.892553in, left, base]{\color{textcolor}\rmfamily\fontsize{10.000000}{12.000000}\selectfont \(\displaystyle {1000}\)}%
\end{pgfscope}%
\begin{pgfscope}%
\pgfsetbuttcap%
\pgfsetroundjoin%
\definecolor{currentfill}{rgb}{0.000000,0.000000,0.000000}%
\pgfsetfillcolor{currentfill}%
\pgfsetlinewidth{0.803000pt}%
\definecolor{currentstroke}{rgb}{0.000000,0.000000,0.000000}%
\pgfsetstrokecolor{currentstroke}%
\pgfsetdash{}{0pt}%
\pgfsys@defobject{currentmarker}{\pgfqpoint{-0.048611in}{0.000000in}}{\pgfqpoint{-0.000000in}{0.000000in}}{%
\pgfpathmoveto{\pgfqpoint{-0.000000in}{0.000000in}}%
\pgfpathlineto{\pgfqpoint{-0.048611in}{0.000000in}}%
\pgfusepath{stroke,fill}%
}%
\begin{pgfscope}%
\pgfsys@transformshift{0.781402in}{1.371692in}%
\pgfsys@useobject{currentmarker}{}%
\end{pgfscope}%
\end{pgfscope}%
\begin{pgfscope}%
\definecolor{textcolor}{rgb}{0.000000,0.000000,0.000000}%
\pgfsetstrokecolor{textcolor}%
\pgfsetfillcolor{textcolor}%
\pgftext[x=0.406402in, y=1.323466in, left, base]{\color{textcolor}\rmfamily\fontsize{10.000000}{12.000000}\selectfont \(\displaystyle {2000}\)}%
\end{pgfscope}%
\begin{pgfscope}%
\pgfsetbuttcap%
\pgfsetroundjoin%
\definecolor{currentfill}{rgb}{0.000000,0.000000,0.000000}%
\pgfsetfillcolor{currentfill}%
\pgfsetlinewidth{0.803000pt}%
\definecolor{currentstroke}{rgb}{0.000000,0.000000,0.000000}%
\pgfsetstrokecolor{currentstroke}%
\pgfsetdash{}{0pt}%
\pgfsys@defobject{currentmarker}{\pgfqpoint{-0.048611in}{0.000000in}}{\pgfqpoint{-0.000000in}{0.000000in}}{%
\pgfpathmoveto{\pgfqpoint{-0.000000in}{0.000000in}}%
\pgfpathlineto{\pgfqpoint{-0.048611in}{0.000000in}}%
\pgfusepath{stroke,fill}%
}%
\begin{pgfscope}%
\pgfsys@transformshift{0.781402in}{1.802605in}%
\pgfsys@useobject{currentmarker}{}%
\end{pgfscope}%
\end{pgfscope}%
\begin{pgfscope}%
\definecolor{textcolor}{rgb}{0.000000,0.000000,0.000000}%
\pgfsetstrokecolor{textcolor}%
\pgfsetfillcolor{textcolor}%
\pgftext[x=0.406402in, y=1.754380in, left, base]{\color{textcolor}\rmfamily\fontsize{10.000000}{12.000000}\selectfont \(\displaystyle {3000}\)}%
\end{pgfscope}%
\begin{pgfscope}%
\pgfsetbuttcap%
\pgfsetroundjoin%
\definecolor{currentfill}{rgb}{0.000000,0.000000,0.000000}%
\pgfsetfillcolor{currentfill}%
\pgfsetlinewidth{0.803000pt}%
\definecolor{currentstroke}{rgb}{0.000000,0.000000,0.000000}%
\pgfsetstrokecolor{currentstroke}%
\pgfsetdash{}{0pt}%
\pgfsys@defobject{currentmarker}{\pgfqpoint{-0.048611in}{0.000000in}}{\pgfqpoint{-0.000000in}{0.000000in}}{%
\pgfpathmoveto{\pgfqpoint{-0.000000in}{0.000000in}}%
\pgfpathlineto{\pgfqpoint{-0.048611in}{0.000000in}}%
\pgfusepath{stroke,fill}%
}%
\begin{pgfscope}%
\pgfsys@transformshift{0.781402in}{2.233519in}%
\pgfsys@useobject{currentmarker}{}%
\end{pgfscope}%
\end{pgfscope}%
\begin{pgfscope}%
\definecolor{textcolor}{rgb}{0.000000,0.000000,0.000000}%
\pgfsetstrokecolor{textcolor}%
\pgfsetfillcolor{textcolor}%
\pgftext[x=0.406402in, y=2.185293in, left, base]{\color{textcolor}\rmfamily\fontsize{10.000000}{12.000000}\selectfont \(\displaystyle {4000}\)}%
\end{pgfscope}%
\begin{pgfscope}%
\pgfsetbuttcap%
\pgfsetroundjoin%
\definecolor{currentfill}{rgb}{0.000000,0.000000,0.000000}%
\pgfsetfillcolor{currentfill}%
\pgfsetlinewidth{0.803000pt}%
\definecolor{currentstroke}{rgb}{0.000000,0.000000,0.000000}%
\pgfsetstrokecolor{currentstroke}%
\pgfsetdash{}{0pt}%
\pgfsys@defobject{currentmarker}{\pgfqpoint{-0.048611in}{0.000000in}}{\pgfqpoint{-0.000000in}{0.000000in}}{%
\pgfpathmoveto{\pgfqpoint{-0.000000in}{0.000000in}}%
\pgfpathlineto{\pgfqpoint{-0.048611in}{0.000000in}}%
\pgfusepath{stroke,fill}%
}%
\begin{pgfscope}%
\pgfsys@transformshift{0.781402in}{2.664432in}%
\pgfsys@useobject{currentmarker}{}%
\end{pgfscope}%
\end{pgfscope}%
\begin{pgfscope}%
\definecolor{textcolor}{rgb}{0.000000,0.000000,0.000000}%
\pgfsetstrokecolor{textcolor}%
\pgfsetfillcolor{textcolor}%
\pgftext[x=0.406402in, y=2.616207in, left, base]{\color{textcolor}\rmfamily\fontsize{10.000000}{12.000000}\selectfont \(\displaystyle {5000}\)}%
\end{pgfscope}%
\begin{pgfscope}%
\definecolor{textcolor}{rgb}{0.000000,0.000000,0.000000}%
\pgfsetstrokecolor{textcolor}%
\pgfsetfillcolor{textcolor}%
\pgftext[x=0.350846in,y=1.740574in,,bottom,rotate=90.000000]{\color{textcolor}\rmfamily\fontsize{10.000000}{12.000000}\selectfont Elapsed time (milliseconds)}%
\end{pgfscope}%
\begin{pgfscope}%
\pgfpathrectangle{\pgfqpoint{0.781402in}{0.386794in}}{\pgfqpoint{4.844695in}{2.707560in}}%
\pgfusepath{clip}%
\pgfsetrectcap%
\pgfsetroundjoin%
\pgfsetlinewidth{1.505625pt}%
\definecolor{currentstroke}{rgb}{0.121569,0.466667,0.705882}%
\pgfsetstrokecolor{currentstroke}%
\pgfsetdash{}{0pt}%
\pgfpathmoveto{\pgfqpoint{1.001616in}{0.510989in}}%
\pgfpathlineto{\pgfqpoint{1.442003in}{0.636670in}}%
\pgfpathlineto{\pgfqpoint{1.882434in}{0.769873in}}%
\pgfpathlineto{\pgfqpoint{2.322866in}{0.897130in}}%
\pgfpathlineto{\pgfqpoint{2.763297in}{1.038113in}}%
\pgfpathlineto{\pgfqpoint{3.203728in}{1.174420in}}%
\pgfpathlineto{\pgfqpoint{3.644159in}{1.303314in}}%
\pgfpathlineto{\pgfqpoint{4.084591in}{1.443472in}}%
\pgfpathlineto{\pgfqpoint{4.525022in}{1.583567in}}%
\pgfpathlineto{\pgfqpoint{4.965453in}{1.693462in}}%
\pgfpathlineto{\pgfqpoint{5.405885in}{1.839996in}}%
\pgfusepath{stroke}%
\end{pgfscope}%
\begin{pgfscope}%
\pgfpathrectangle{\pgfqpoint{0.781402in}{0.386794in}}{\pgfqpoint{4.844695in}{2.707560in}}%
\pgfusepath{clip}%
\pgfsetbuttcap%
\pgfsetroundjoin%
\definecolor{currentfill}{rgb}{0.121569,0.466667,0.705882}%
\pgfsetfillcolor{currentfill}%
\pgfsetlinewidth{1.003750pt}%
\definecolor{currentstroke}{rgb}{0.121569,0.466667,0.705882}%
\pgfsetstrokecolor{currentstroke}%
\pgfsetdash{}{0pt}%
\pgfsys@defobject{currentmarker}{\pgfqpoint{-0.041667in}{-0.041667in}}{\pgfqpoint{0.041667in}{0.041667in}}{%
\pgfpathmoveto{\pgfqpoint{0.000000in}{-0.041667in}}%
\pgfpathcurveto{\pgfqpoint{0.011050in}{-0.041667in}}{\pgfqpoint{0.021649in}{-0.037276in}}{\pgfqpoint{0.029463in}{-0.029463in}}%
\pgfpathcurveto{\pgfqpoint{0.037276in}{-0.021649in}}{\pgfqpoint{0.041667in}{-0.011050in}}{\pgfqpoint{0.041667in}{0.000000in}}%
\pgfpathcurveto{\pgfqpoint{0.041667in}{0.011050in}}{\pgfqpoint{0.037276in}{0.021649in}}{\pgfqpoint{0.029463in}{0.029463in}}%
\pgfpathcurveto{\pgfqpoint{0.021649in}{0.037276in}}{\pgfqpoint{0.011050in}{0.041667in}}{\pgfqpoint{0.000000in}{0.041667in}}%
\pgfpathcurveto{\pgfqpoint{-0.011050in}{0.041667in}}{\pgfqpoint{-0.021649in}{0.037276in}}{\pgfqpoint{-0.029463in}{0.029463in}}%
\pgfpathcurveto{\pgfqpoint{-0.037276in}{0.021649in}}{\pgfqpoint{-0.041667in}{0.011050in}}{\pgfqpoint{-0.041667in}{0.000000in}}%
\pgfpathcurveto{\pgfqpoint{-0.041667in}{-0.011050in}}{\pgfqpoint{-0.037276in}{-0.021649in}}{\pgfqpoint{-0.029463in}{-0.029463in}}%
\pgfpathcurveto{\pgfqpoint{-0.021649in}{-0.037276in}}{\pgfqpoint{-0.011050in}{-0.041667in}}{\pgfqpoint{0.000000in}{-0.041667in}}%
\pgfpathclose%
\pgfusepath{stroke,fill}%
}%
\begin{pgfscope}%
\pgfsys@transformshift{1.001616in}{0.510989in}%
\pgfsys@useobject{currentmarker}{}%
\end{pgfscope}%
\begin{pgfscope}%
\pgfsys@transformshift{1.442003in}{0.636670in}%
\pgfsys@useobject{currentmarker}{}%
\end{pgfscope}%
\begin{pgfscope}%
\pgfsys@transformshift{1.882434in}{0.769873in}%
\pgfsys@useobject{currentmarker}{}%
\end{pgfscope}%
\begin{pgfscope}%
\pgfsys@transformshift{2.322866in}{0.897130in}%
\pgfsys@useobject{currentmarker}{}%
\end{pgfscope}%
\begin{pgfscope}%
\pgfsys@transformshift{2.763297in}{1.038113in}%
\pgfsys@useobject{currentmarker}{}%
\end{pgfscope}%
\begin{pgfscope}%
\pgfsys@transformshift{3.203728in}{1.174420in}%
\pgfsys@useobject{currentmarker}{}%
\end{pgfscope}%
\begin{pgfscope}%
\pgfsys@transformshift{3.644159in}{1.303314in}%
\pgfsys@useobject{currentmarker}{}%
\end{pgfscope}%
\begin{pgfscope}%
\pgfsys@transformshift{4.084591in}{1.443472in}%
\pgfsys@useobject{currentmarker}{}%
\end{pgfscope}%
\begin{pgfscope}%
\pgfsys@transformshift{4.525022in}{1.583567in}%
\pgfsys@useobject{currentmarker}{}%
\end{pgfscope}%
\begin{pgfscope}%
\pgfsys@transformshift{4.965453in}{1.693462in}%
\pgfsys@useobject{currentmarker}{}%
\end{pgfscope}%
\begin{pgfscope}%
\pgfsys@transformshift{5.405885in}{1.839996in}%
\pgfsys@useobject{currentmarker}{}%
\end{pgfscope}%
\end{pgfscope}%
\begin{pgfscope}%
\pgfpathrectangle{\pgfqpoint{0.781402in}{0.386794in}}{\pgfqpoint{4.844695in}{2.707560in}}%
\pgfusepath{clip}%
\pgfsetrectcap%
\pgfsetroundjoin%
\pgfsetlinewidth{1.505625pt}%
\definecolor{currentstroke}{rgb}{1.000000,0.498039,0.054902}%
\pgfsetstrokecolor{currentstroke}%
\pgfsetdash{}{0pt}%
\pgfpathmoveto{\pgfqpoint{1.001616in}{0.510146in}}%
\pgfpathlineto{\pgfqpoint{1.442003in}{0.509865in}}%
\pgfpathlineto{\pgfqpoint{1.882434in}{0.509865in}}%
\pgfpathlineto{\pgfqpoint{2.322866in}{0.509865in}}%
\pgfpathlineto{\pgfqpoint{2.763297in}{0.509865in}}%
\pgfpathlineto{\pgfqpoint{3.203728in}{0.509865in}}%
\pgfpathlineto{\pgfqpoint{3.644159in}{0.510113in}}%
\pgfpathlineto{\pgfqpoint{4.084591in}{0.509865in}}%
\pgfpathlineto{\pgfqpoint{4.525022in}{0.509865in}}%
\pgfpathlineto{\pgfqpoint{4.965453in}{0.509865in}}%
\pgfpathlineto{\pgfqpoint{5.405885in}{0.509865in}}%
\pgfusepath{stroke}%
\end{pgfscope}%
\begin{pgfscope}%
\pgfpathrectangle{\pgfqpoint{0.781402in}{0.386794in}}{\pgfqpoint{4.844695in}{2.707560in}}%
\pgfusepath{clip}%
\pgfsetbuttcap%
\pgfsetroundjoin%
\definecolor{currentfill}{rgb}{1.000000,0.498039,0.054902}%
\pgfsetfillcolor{currentfill}%
\pgfsetlinewidth{1.003750pt}%
\definecolor{currentstroke}{rgb}{1.000000,0.498039,0.054902}%
\pgfsetstrokecolor{currentstroke}%
\pgfsetdash{}{0pt}%
\pgfsys@defobject{currentmarker}{\pgfqpoint{-0.041667in}{-0.041667in}}{\pgfqpoint{0.041667in}{0.041667in}}{%
\pgfpathmoveto{\pgfqpoint{0.000000in}{-0.041667in}}%
\pgfpathcurveto{\pgfqpoint{0.011050in}{-0.041667in}}{\pgfqpoint{0.021649in}{-0.037276in}}{\pgfqpoint{0.029463in}{-0.029463in}}%
\pgfpathcurveto{\pgfqpoint{0.037276in}{-0.021649in}}{\pgfqpoint{0.041667in}{-0.011050in}}{\pgfqpoint{0.041667in}{0.000000in}}%
\pgfpathcurveto{\pgfqpoint{0.041667in}{0.011050in}}{\pgfqpoint{0.037276in}{0.021649in}}{\pgfqpoint{0.029463in}{0.029463in}}%
\pgfpathcurveto{\pgfqpoint{0.021649in}{0.037276in}}{\pgfqpoint{0.011050in}{0.041667in}}{\pgfqpoint{0.000000in}{0.041667in}}%
\pgfpathcurveto{\pgfqpoint{-0.011050in}{0.041667in}}{\pgfqpoint{-0.021649in}{0.037276in}}{\pgfqpoint{-0.029463in}{0.029463in}}%
\pgfpathcurveto{\pgfqpoint{-0.037276in}{0.021649in}}{\pgfqpoint{-0.041667in}{0.011050in}}{\pgfqpoint{-0.041667in}{0.000000in}}%
\pgfpathcurveto{\pgfqpoint{-0.041667in}{-0.011050in}}{\pgfqpoint{-0.037276in}{-0.021649in}}{\pgfqpoint{-0.029463in}{-0.029463in}}%
\pgfpathcurveto{\pgfqpoint{-0.021649in}{-0.037276in}}{\pgfqpoint{-0.011050in}{-0.041667in}}{\pgfqpoint{0.000000in}{-0.041667in}}%
\pgfpathclose%
\pgfusepath{stroke,fill}%
}%
\begin{pgfscope}%
\pgfsys@transformshift{1.001616in}{0.510146in}%
\pgfsys@useobject{currentmarker}{}%
\end{pgfscope}%
\begin{pgfscope}%
\pgfsys@transformshift{1.442003in}{0.509865in}%
\pgfsys@useobject{currentmarker}{}%
\end{pgfscope}%
\begin{pgfscope}%
\pgfsys@transformshift{1.882434in}{0.509865in}%
\pgfsys@useobject{currentmarker}{}%
\end{pgfscope}%
\begin{pgfscope}%
\pgfsys@transformshift{2.322866in}{0.509865in}%
\pgfsys@useobject{currentmarker}{}%
\end{pgfscope}%
\begin{pgfscope}%
\pgfsys@transformshift{2.763297in}{0.509865in}%
\pgfsys@useobject{currentmarker}{}%
\end{pgfscope}%
\begin{pgfscope}%
\pgfsys@transformshift{3.203728in}{0.509865in}%
\pgfsys@useobject{currentmarker}{}%
\end{pgfscope}%
\begin{pgfscope}%
\pgfsys@transformshift{3.644159in}{0.510113in}%
\pgfsys@useobject{currentmarker}{}%
\end{pgfscope}%
\begin{pgfscope}%
\pgfsys@transformshift{4.084591in}{0.509865in}%
\pgfsys@useobject{currentmarker}{}%
\end{pgfscope}%
\begin{pgfscope}%
\pgfsys@transformshift{4.525022in}{0.509865in}%
\pgfsys@useobject{currentmarker}{}%
\end{pgfscope}%
\begin{pgfscope}%
\pgfsys@transformshift{4.965453in}{0.509865in}%
\pgfsys@useobject{currentmarker}{}%
\end{pgfscope}%
\begin{pgfscope}%
\pgfsys@transformshift{5.405885in}{0.509865in}%
\pgfsys@useobject{currentmarker}{}%
\end{pgfscope}%
\end{pgfscope}%
\begin{pgfscope}%
\pgfpathrectangle{\pgfqpoint{0.781402in}{0.386794in}}{\pgfqpoint{4.844695in}{2.707560in}}%
\pgfusepath{clip}%
\pgfsetrectcap%
\pgfsetroundjoin%
\pgfsetlinewidth{1.505625pt}%
\definecolor{currentstroke}{rgb}{0.172549,0.627451,0.172549}%
\pgfsetstrokecolor{currentstroke}%
\pgfsetdash{}{0pt}%
\pgfpathmoveto{\pgfqpoint{1.001616in}{0.509883in}}%
\pgfpathlineto{\pgfqpoint{1.442003in}{0.545971in}}%
\pgfpathlineto{\pgfqpoint{1.882434in}{0.591338in}}%
\pgfpathlineto{\pgfqpoint{2.322866in}{0.636282in}}%
\pgfpathlineto{\pgfqpoint{2.763297in}{0.683650in}}%
\pgfpathlineto{\pgfqpoint{3.203728in}{0.728078in}}%
\pgfpathlineto{\pgfqpoint{3.644159in}{0.776211in}}%
\pgfpathlineto{\pgfqpoint{4.084591in}{0.825757in}}%
\pgfpathlineto{\pgfqpoint{4.525022in}{0.881494in}}%
\pgfpathlineto{\pgfqpoint{4.965453in}{0.924303in}}%
\pgfpathlineto{\pgfqpoint{5.405885in}{0.970340in}}%
\pgfusepath{stroke}%
\end{pgfscope}%
\begin{pgfscope}%
\pgfpathrectangle{\pgfqpoint{0.781402in}{0.386794in}}{\pgfqpoint{4.844695in}{2.707560in}}%
\pgfusepath{clip}%
\pgfsetbuttcap%
\pgfsetroundjoin%
\definecolor{currentfill}{rgb}{0.172549,0.627451,0.172549}%
\pgfsetfillcolor{currentfill}%
\pgfsetlinewidth{1.003750pt}%
\definecolor{currentstroke}{rgb}{0.172549,0.627451,0.172549}%
\pgfsetstrokecolor{currentstroke}%
\pgfsetdash{}{0pt}%
\pgfsys@defobject{currentmarker}{\pgfqpoint{-0.041667in}{-0.041667in}}{\pgfqpoint{0.041667in}{0.041667in}}{%
\pgfpathmoveto{\pgfqpoint{0.000000in}{-0.041667in}}%
\pgfpathcurveto{\pgfqpoint{0.011050in}{-0.041667in}}{\pgfqpoint{0.021649in}{-0.037276in}}{\pgfqpoint{0.029463in}{-0.029463in}}%
\pgfpathcurveto{\pgfqpoint{0.037276in}{-0.021649in}}{\pgfqpoint{0.041667in}{-0.011050in}}{\pgfqpoint{0.041667in}{0.000000in}}%
\pgfpathcurveto{\pgfqpoint{0.041667in}{0.011050in}}{\pgfqpoint{0.037276in}{0.021649in}}{\pgfqpoint{0.029463in}{0.029463in}}%
\pgfpathcurveto{\pgfqpoint{0.021649in}{0.037276in}}{\pgfqpoint{0.011050in}{0.041667in}}{\pgfqpoint{0.000000in}{0.041667in}}%
\pgfpathcurveto{\pgfqpoint{-0.011050in}{0.041667in}}{\pgfqpoint{-0.021649in}{0.037276in}}{\pgfqpoint{-0.029463in}{0.029463in}}%
\pgfpathcurveto{\pgfqpoint{-0.037276in}{0.021649in}}{\pgfqpoint{-0.041667in}{0.011050in}}{\pgfqpoint{-0.041667in}{0.000000in}}%
\pgfpathcurveto{\pgfqpoint{-0.041667in}{-0.011050in}}{\pgfqpoint{-0.037276in}{-0.021649in}}{\pgfqpoint{-0.029463in}{-0.029463in}}%
\pgfpathcurveto{\pgfqpoint{-0.021649in}{-0.037276in}}{\pgfqpoint{-0.011050in}{-0.041667in}}{\pgfqpoint{0.000000in}{-0.041667in}}%
\pgfpathclose%
\pgfusepath{stroke,fill}%
}%
\begin{pgfscope}%
\pgfsys@transformshift{1.001616in}{0.509883in}%
\pgfsys@useobject{currentmarker}{}%
\end{pgfscope}%
\begin{pgfscope}%
\pgfsys@transformshift{1.442003in}{0.545971in}%
\pgfsys@useobject{currentmarker}{}%
\end{pgfscope}%
\begin{pgfscope}%
\pgfsys@transformshift{1.882434in}{0.591338in}%
\pgfsys@useobject{currentmarker}{}%
\end{pgfscope}%
\begin{pgfscope}%
\pgfsys@transformshift{2.322866in}{0.636282in}%
\pgfsys@useobject{currentmarker}{}%
\end{pgfscope}%
\begin{pgfscope}%
\pgfsys@transformshift{2.763297in}{0.683650in}%
\pgfsys@useobject{currentmarker}{}%
\end{pgfscope}%
\begin{pgfscope}%
\pgfsys@transformshift{3.203728in}{0.728078in}%
\pgfsys@useobject{currentmarker}{}%
\end{pgfscope}%
\begin{pgfscope}%
\pgfsys@transformshift{3.644159in}{0.776211in}%
\pgfsys@useobject{currentmarker}{}%
\end{pgfscope}%
\begin{pgfscope}%
\pgfsys@transformshift{4.084591in}{0.825757in}%
\pgfsys@useobject{currentmarker}{}%
\end{pgfscope}%
\begin{pgfscope}%
\pgfsys@transformshift{4.525022in}{0.881494in}%
\pgfsys@useobject{currentmarker}{}%
\end{pgfscope}%
\begin{pgfscope}%
\pgfsys@transformshift{4.965453in}{0.924303in}%
\pgfsys@useobject{currentmarker}{}%
\end{pgfscope}%
\begin{pgfscope}%
\pgfsys@transformshift{5.405885in}{0.970340in}%
\pgfsys@useobject{currentmarker}{}%
\end{pgfscope}%
\end{pgfscope}%
\begin{pgfscope}%
\pgfpathrectangle{\pgfqpoint{0.781402in}{0.386794in}}{\pgfqpoint{4.844695in}{2.707560in}}%
\pgfusepath{clip}%
\pgfsetrectcap%
\pgfsetroundjoin%
\pgfsetlinewidth{1.505625pt}%
\definecolor{currentstroke}{rgb}{0.839216,0.152941,0.156863}%
\pgfsetstrokecolor{currentstroke}%
\pgfsetdash{}{0pt}%
\pgfpathmoveto{\pgfqpoint{1.001616in}{0.511362in}}%
\pgfpathlineto{\pgfqpoint{1.442003in}{0.738373in}}%
\pgfpathlineto{\pgfqpoint{1.882434in}{0.982789in}}%
\pgfpathlineto{\pgfqpoint{2.322866in}{1.213948in}}%
\pgfpathlineto{\pgfqpoint{2.763297in}{1.479409in}}%
\pgfpathlineto{\pgfqpoint{3.203728in}{1.722312in}}%
\pgfpathlineto{\pgfqpoint{3.644159in}{1.963728in}}%
\pgfpathlineto{\pgfqpoint{4.084591in}{2.228331in}}%
\pgfpathlineto{\pgfqpoint{4.525022in}{2.489922in}}%
\pgfpathlineto{\pgfqpoint{4.965453in}{2.712077in}}%
\pgfpathlineto{\pgfqpoint{5.405885in}{2.971283in}}%
\pgfusepath{stroke}%
\end{pgfscope}%
\begin{pgfscope}%
\pgfpathrectangle{\pgfqpoint{0.781402in}{0.386794in}}{\pgfqpoint{4.844695in}{2.707560in}}%
\pgfusepath{clip}%
\pgfsetbuttcap%
\pgfsetroundjoin%
\definecolor{currentfill}{rgb}{0.839216,0.152941,0.156863}%
\pgfsetfillcolor{currentfill}%
\pgfsetlinewidth{1.003750pt}%
\definecolor{currentstroke}{rgb}{0.839216,0.152941,0.156863}%
\pgfsetstrokecolor{currentstroke}%
\pgfsetdash{}{0pt}%
\pgfsys@defobject{currentmarker}{\pgfqpoint{-0.041667in}{-0.041667in}}{\pgfqpoint{0.041667in}{0.041667in}}{%
\pgfpathmoveto{\pgfqpoint{0.000000in}{-0.041667in}}%
\pgfpathcurveto{\pgfqpoint{0.011050in}{-0.041667in}}{\pgfqpoint{0.021649in}{-0.037276in}}{\pgfqpoint{0.029463in}{-0.029463in}}%
\pgfpathcurveto{\pgfqpoint{0.037276in}{-0.021649in}}{\pgfqpoint{0.041667in}{-0.011050in}}{\pgfqpoint{0.041667in}{0.000000in}}%
\pgfpathcurveto{\pgfqpoint{0.041667in}{0.011050in}}{\pgfqpoint{0.037276in}{0.021649in}}{\pgfqpoint{0.029463in}{0.029463in}}%
\pgfpathcurveto{\pgfqpoint{0.021649in}{0.037276in}}{\pgfqpoint{0.011050in}{0.041667in}}{\pgfqpoint{0.000000in}{0.041667in}}%
\pgfpathcurveto{\pgfqpoint{-0.011050in}{0.041667in}}{\pgfqpoint{-0.021649in}{0.037276in}}{\pgfqpoint{-0.029463in}{0.029463in}}%
\pgfpathcurveto{\pgfqpoint{-0.037276in}{0.021649in}}{\pgfqpoint{-0.041667in}{0.011050in}}{\pgfqpoint{-0.041667in}{0.000000in}}%
\pgfpathcurveto{\pgfqpoint{-0.041667in}{-0.011050in}}{\pgfqpoint{-0.037276in}{-0.021649in}}{\pgfqpoint{-0.029463in}{-0.029463in}}%
\pgfpathcurveto{\pgfqpoint{-0.021649in}{-0.037276in}}{\pgfqpoint{-0.011050in}{-0.041667in}}{\pgfqpoint{0.000000in}{-0.041667in}}%
\pgfpathclose%
\pgfusepath{stroke,fill}%
}%
\begin{pgfscope}%
\pgfsys@transformshift{1.001616in}{0.511362in}%
\pgfsys@useobject{currentmarker}{}%
\end{pgfscope}%
\begin{pgfscope}%
\pgfsys@transformshift{1.442003in}{0.738373in}%
\pgfsys@useobject{currentmarker}{}%
\end{pgfscope}%
\begin{pgfscope}%
\pgfsys@transformshift{1.882434in}{0.982789in}%
\pgfsys@useobject{currentmarker}{}%
\end{pgfscope}%
\begin{pgfscope}%
\pgfsys@transformshift{2.322866in}{1.213948in}%
\pgfsys@useobject{currentmarker}{}%
\end{pgfscope}%
\begin{pgfscope}%
\pgfsys@transformshift{2.763297in}{1.479409in}%
\pgfsys@useobject{currentmarker}{}%
\end{pgfscope}%
\begin{pgfscope}%
\pgfsys@transformshift{3.203728in}{1.722312in}%
\pgfsys@useobject{currentmarker}{}%
\end{pgfscope}%
\begin{pgfscope}%
\pgfsys@transformshift{3.644159in}{1.963728in}%
\pgfsys@useobject{currentmarker}{}%
\end{pgfscope}%
\begin{pgfscope}%
\pgfsys@transformshift{4.084591in}{2.228331in}%
\pgfsys@useobject{currentmarker}{}%
\end{pgfscope}%
\begin{pgfscope}%
\pgfsys@transformshift{4.525022in}{2.489922in}%
\pgfsys@useobject{currentmarker}{}%
\end{pgfscope}%
\begin{pgfscope}%
\pgfsys@transformshift{4.965453in}{2.712077in}%
\pgfsys@useobject{currentmarker}{}%
\end{pgfscope}%
\begin{pgfscope}%
\pgfsys@transformshift{5.405885in}{2.971283in}%
\pgfsys@useobject{currentmarker}{}%
\end{pgfscope}%
\end{pgfscope}%
\begin{pgfscope}%
\pgfsetrectcap%
\pgfsetmiterjoin%
\pgfsetlinewidth{0.803000pt}%
\definecolor{currentstroke}{rgb}{0.000000,0.000000,0.000000}%
\pgfsetstrokecolor{currentstroke}%
\pgfsetdash{}{0pt}%
\pgfpathmoveto{\pgfqpoint{0.781402in}{0.386794in}}%
\pgfpathlineto{\pgfqpoint{0.781402in}{3.094354in}}%
\pgfusepath{stroke}%
\end{pgfscope}%
\begin{pgfscope}%
\pgfsetrectcap%
\pgfsetmiterjoin%
\pgfsetlinewidth{0.803000pt}%
\definecolor{currentstroke}{rgb}{0.000000,0.000000,0.000000}%
\pgfsetstrokecolor{currentstroke}%
\pgfsetdash{}{0pt}%
\pgfpathmoveto{\pgfqpoint{5.626098in}{0.386794in}}%
\pgfpathlineto{\pgfqpoint{5.626098in}{3.094354in}}%
\pgfusepath{stroke}%
\end{pgfscope}%
\begin{pgfscope}%
\pgfsetrectcap%
\pgfsetmiterjoin%
\pgfsetlinewidth{0.803000pt}%
\definecolor{currentstroke}{rgb}{0.000000,0.000000,0.000000}%
\pgfsetstrokecolor{currentstroke}%
\pgfsetdash{}{0pt}%
\pgfpathmoveto{\pgfqpoint{0.781402in}{0.386794in}}%
\pgfpathlineto{\pgfqpoint{5.626098in}{0.386794in}}%
\pgfusepath{stroke}%
\end{pgfscope}%
\begin{pgfscope}%
\pgfsetrectcap%
\pgfsetmiterjoin%
\pgfsetlinewidth{0.803000pt}%
\definecolor{currentstroke}{rgb}{0.000000,0.000000,0.000000}%
\pgfsetstrokecolor{currentstroke}%
\pgfsetdash{}{0pt}%
\pgfpathmoveto{\pgfqpoint{0.781402in}{3.094354in}}%
\pgfpathlineto{\pgfqpoint{5.626098in}{3.094354in}}%
\pgfusepath{stroke}%
\end{pgfscope}%
\begin{pgfscope}%
\pgfsetbuttcap%
\pgfsetmiterjoin%
\definecolor{currentfill}{rgb}{1.000000,1.000000,1.000000}%
\pgfsetfillcolor{currentfill}%
\pgfsetfillopacity{0.800000}%
\pgfsetlinewidth{1.003750pt}%
\definecolor{currentstroke}{rgb}{0.800000,0.800000,0.800000}%
\pgfsetstrokecolor{currentstroke}%
\pgfsetstrokeopacity{0.800000}%
\pgfsetdash{}{0pt}%
\pgfpathmoveto{\pgfqpoint{0.878625in}{2.208552in}}%
\pgfpathlineto{\pgfqpoint{3.483567in}{2.208552in}}%
\pgfpathquadraticcurveto{\pgfqpoint{3.511345in}{2.208552in}}{\pgfqpoint{3.511345in}{2.236329in}}%
\pgfpathlineto{\pgfqpoint{3.511345in}{2.997132in}}%
\pgfpathquadraticcurveto{\pgfqpoint{3.511345in}{3.024909in}}{\pgfqpoint{3.483567in}{3.024909in}}%
\pgfpathlineto{\pgfqpoint{0.878625in}{3.024909in}}%
\pgfpathquadraticcurveto{\pgfqpoint{0.850847in}{3.024909in}}{\pgfqpoint{0.850847in}{2.997132in}}%
\pgfpathlineto{\pgfqpoint{0.850847in}{2.236329in}}%
\pgfpathquadraticcurveto{\pgfqpoint{0.850847in}{2.208552in}}{\pgfqpoint{0.878625in}{2.208552in}}%
\pgfpathclose%
\pgfusepath{stroke,fill}%
\end{pgfscope}%
\begin{pgfscope}%
\pgfsetrectcap%
\pgfsetroundjoin%
\pgfsetlinewidth{1.505625pt}%
\definecolor{currentstroke}{rgb}{0.121569,0.466667,0.705882}%
\pgfsetstrokecolor{currentstroke}%
\pgfsetdash{}{0pt}%
\pgfpathmoveto{\pgfqpoint{0.906402in}{2.920743in}}%
\pgfpathlineto{\pgfqpoint{1.184180in}{2.920743in}}%
\pgfusepath{stroke}%
\end{pgfscope}%
\begin{pgfscope}%
\pgfsetbuttcap%
\pgfsetroundjoin%
\definecolor{currentfill}{rgb}{0.121569,0.466667,0.705882}%
\pgfsetfillcolor{currentfill}%
\pgfsetlinewidth{1.003750pt}%
\definecolor{currentstroke}{rgb}{0.121569,0.466667,0.705882}%
\pgfsetstrokecolor{currentstroke}%
\pgfsetdash{}{0pt}%
\pgfsys@defobject{currentmarker}{\pgfqpoint{-0.041667in}{-0.041667in}}{\pgfqpoint{0.041667in}{0.041667in}}{%
\pgfpathmoveto{\pgfqpoint{0.000000in}{-0.041667in}}%
\pgfpathcurveto{\pgfqpoint{0.011050in}{-0.041667in}}{\pgfqpoint{0.021649in}{-0.037276in}}{\pgfqpoint{0.029463in}{-0.029463in}}%
\pgfpathcurveto{\pgfqpoint{0.037276in}{-0.021649in}}{\pgfqpoint{0.041667in}{-0.011050in}}{\pgfqpoint{0.041667in}{0.000000in}}%
\pgfpathcurveto{\pgfqpoint{0.041667in}{0.011050in}}{\pgfqpoint{0.037276in}{0.021649in}}{\pgfqpoint{0.029463in}{0.029463in}}%
\pgfpathcurveto{\pgfqpoint{0.021649in}{0.037276in}}{\pgfqpoint{0.011050in}{0.041667in}}{\pgfqpoint{0.000000in}{0.041667in}}%
\pgfpathcurveto{\pgfqpoint{-0.011050in}{0.041667in}}{\pgfqpoint{-0.021649in}{0.037276in}}{\pgfqpoint{-0.029463in}{0.029463in}}%
\pgfpathcurveto{\pgfqpoint{-0.037276in}{0.021649in}}{\pgfqpoint{-0.041667in}{0.011050in}}{\pgfqpoint{-0.041667in}{0.000000in}}%
\pgfpathcurveto{\pgfqpoint{-0.041667in}{-0.011050in}}{\pgfqpoint{-0.037276in}{-0.021649in}}{\pgfqpoint{-0.029463in}{-0.029463in}}%
\pgfpathcurveto{\pgfqpoint{-0.021649in}{-0.037276in}}{\pgfqpoint{-0.011050in}{-0.041667in}}{\pgfqpoint{0.000000in}{-0.041667in}}%
\pgfpathclose%
\pgfusepath{stroke,fill}%
}%
\begin{pgfscope}%
\pgfsys@transformshift{1.045291in}{2.920743in}%
\pgfsys@useobject{currentmarker}{}%
\end{pgfscope}%
\end{pgfscope}%
\begin{pgfscope}%
\definecolor{textcolor}{rgb}{0.000000,0.000000,0.000000}%
\pgfsetstrokecolor{textcolor}%
\pgfsetfillcolor{textcolor}%
\pgftext[x=1.295291in,y=2.872132in,left,base]{\color{textcolor}\rmfamily\fontsize{10.000000}{12.000000}\selectfont Prefill duration}%
\end{pgfscope}%
\begin{pgfscope}%
\pgfsetrectcap%
\pgfsetroundjoin%
\pgfsetlinewidth{1.505625pt}%
\definecolor{currentstroke}{rgb}{1.000000,0.498039,0.054902}%
\pgfsetstrokecolor{currentstroke}%
\pgfsetdash{}{0pt}%
\pgfpathmoveto{\pgfqpoint{0.906402in}{2.727070in}}%
\pgfpathlineto{\pgfqpoint{1.184180in}{2.727070in}}%
\pgfusepath{stroke}%
\end{pgfscope}%
\begin{pgfscope}%
\pgfsetbuttcap%
\pgfsetroundjoin%
\definecolor{currentfill}{rgb}{1.000000,0.498039,0.054902}%
\pgfsetfillcolor{currentfill}%
\pgfsetlinewidth{1.003750pt}%
\definecolor{currentstroke}{rgb}{1.000000,0.498039,0.054902}%
\pgfsetstrokecolor{currentstroke}%
\pgfsetdash{}{0pt}%
\pgfsys@defobject{currentmarker}{\pgfqpoint{-0.041667in}{-0.041667in}}{\pgfqpoint{0.041667in}{0.041667in}}{%
\pgfpathmoveto{\pgfqpoint{0.000000in}{-0.041667in}}%
\pgfpathcurveto{\pgfqpoint{0.011050in}{-0.041667in}}{\pgfqpoint{0.021649in}{-0.037276in}}{\pgfqpoint{0.029463in}{-0.029463in}}%
\pgfpathcurveto{\pgfqpoint{0.037276in}{-0.021649in}}{\pgfqpoint{0.041667in}{-0.011050in}}{\pgfqpoint{0.041667in}{0.000000in}}%
\pgfpathcurveto{\pgfqpoint{0.041667in}{0.011050in}}{\pgfqpoint{0.037276in}{0.021649in}}{\pgfqpoint{0.029463in}{0.029463in}}%
\pgfpathcurveto{\pgfqpoint{0.021649in}{0.037276in}}{\pgfqpoint{0.011050in}{0.041667in}}{\pgfqpoint{0.000000in}{0.041667in}}%
\pgfpathcurveto{\pgfqpoint{-0.011050in}{0.041667in}}{\pgfqpoint{-0.021649in}{0.037276in}}{\pgfqpoint{-0.029463in}{0.029463in}}%
\pgfpathcurveto{\pgfqpoint{-0.037276in}{0.021649in}}{\pgfqpoint{-0.041667in}{0.011050in}}{\pgfqpoint{-0.041667in}{0.000000in}}%
\pgfpathcurveto{\pgfqpoint{-0.041667in}{-0.011050in}}{\pgfqpoint{-0.037276in}{-0.021649in}}{\pgfqpoint{-0.029463in}{-0.029463in}}%
\pgfpathcurveto{\pgfqpoint{-0.021649in}{-0.037276in}}{\pgfqpoint{-0.011050in}{-0.041667in}}{\pgfqpoint{0.000000in}{-0.041667in}}%
\pgfpathclose%
\pgfusepath{stroke,fill}%
}%
\begin{pgfscope}%
\pgfsys@transformshift{1.045291in}{2.727070in}%
\pgfsys@useobject{currentmarker}{}%
\end{pgfscope}%
\end{pgfscope}%
\begin{pgfscope}%
\definecolor{textcolor}{rgb}{0.000000,0.000000,0.000000}%
\pgfsetstrokecolor{textcolor}%
\pgfsetfillcolor{textcolor}%
\pgftext[x=1.295291in,y=2.678459in,left,base]{\color{textcolor}\rmfamily\fontsize{10.000000}{12.000000}\selectfont Duration without owner}%
\end{pgfscope}%
\begin{pgfscope}%
\pgfsetrectcap%
\pgfsetroundjoin%
\pgfsetlinewidth{1.505625pt}%
\definecolor{currentstroke}{rgb}{0.172549,0.627451,0.172549}%
\pgfsetstrokecolor{currentstroke}%
\pgfsetdash{}{0pt}%
\pgfpathmoveto{\pgfqpoint{0.906402in}{2.533397in}}%
\pgfpathlineto{\pgfqpoint{1.184180in}{2.533397in}}%
\pgfusepath{stroke}%
\end{pgfscope}%
\begin{pgfscope}%
\pgfsetbuttcap%
\pgfsetroundjoin%
\definecolor{currentfill}{rgb}{0.172549,0.627451,0.172549}%
\pgfsetfillcolor{currentfill}%
\pgfsetlinewidth{1.003750pt}%
\definecolor{currentstroke}{rgb}{0.172549,0.627451,0.172549}%
\pgfsetstrokecolor{currentstroke}%
\pgfsetdash{}{0pt}%
\pgfsys@defobject{currentmarker}{\pgfqpoint{-0.041667in}{-0.041667in}}{\pgfqpoint{0.041667in}{0.041667in}}{%
\pgfpathmoveto{\pgfqpoint{0.000000in}{-0.041667in}}%
\pgfpathcurveto{\pgfqpoint{0.011050in}{-0.041667in}}{\pgfqpoint{0.021649in}{-0.037276in}}{\pgfqpoint{0.029463in}{-0.029463in}}%
\pgfpathcurveto{\pgfqpoint{0.037276in}{-0.021649in}}{\pgfqpoint{0.041667in}{-0.011050in}}{\pgfqpoint{0.041667in}{0.000000in}}%
\pgfpathcurveto{\pgfqpoint{0.041667in}{0.011050in}}{\pgfqpoint{0.037276in}{0.021649in}}{\pgfqpoint{0.029463in}{0.029463in}}%
\pgfpathcurveto{\pgfqpoint{0.021649in}{0.037276in}}{\pgfqpoint{0.011050in}{0.041667in}}{\pgfqpoint{0.000000in}{0.041667in}}%
\pgfpathcurveto{\pgfqpoint{-0.011050in}{0.041667in}}{\pgfqpoint{-0.021649in}{0.037276in}}{\pgfqpoint{-0.029463in}{0.029463in}}%
\pgfpathcurveto{\pgfqpoint{-0.037276in}{0.021649in}}{\pgfqpoint{-0.041667in}{0.011050in}}{\pgfqpoint{-0.041667in}{0.000000in}}%
\pgfpathcurveto{\pgfqpoint{-0.041667in}{-0.011050in}}{\pgfqpoint{-0.037276in}{-0.021649in}}{\pgfqpoint{-0.029463in}{-0.029463in}}%
\pgfpathcurveto{\pgfqpoint{-0.021649in}{-0.037276in}}{\pgfqpoint{-0.011050in}{-0.041667in}}{\pgfqpoint{0.000000in}{-0.041667in}}%
\pgfpathclose%
\pgfusepath{stroke,fill}%
}%
\begin{pgfscope}%
\pgfsys@transformshift{1.045291in}{2.533397in}%
\pgfsys@useobject{currentmarker}{}%
\end{pgfscope}%
\end{pgfscope}%
\begin{pgfscope}%
\definecolor{textcolor}{rgb}{0.000000,0.000000,0.000000}%
\pgfsetstrokecolor{textcolor}%
\pgfsetfillcolor{textcolor}%
\pgftext[x=1.295291in,y=2.484786in,left,base]{\color{textcolor}\rmfamily\fontsize{10.000000}{12.000000}\selectfont Time spent transferring dirty pages}%
\end{pgfscope}%
\begin{pgfscope}%
\pgfsetrectcap%
\pgfsetroundjoin%
\pgfsetlinewidth{1.505625pt}%
\definecolor{currentstroke}{rgb}{0.839216,0.152941,0.156863}%
\pgfsetstrokecolor{currentstroke}%
\pgfsetdash{}{0pt}%
\pgfpathmoveto{\pgfqpoint{0.906402in}{2.339724in}}%
\pgfpathlineto{\pgfqpoint{1.184180in}{2.339724in}}%
\pgfusepath{stroke}%
\end{pgfscope}%
\begin{pgfscope}%
\pgfsetbuttcap%
\pgfsetroundjoin%
\definecolor{currentfill}{rgb}{0.839216,0.152941,0.156863}%
\pgfsetfillcolor{currentfill}%
\pgfsetlinewidth{1.003750pt}%
\definecolor{currentstroke}{rgb}{0.839216,0.152941,0.156863}%
\pgfsetstrokecolor{currentstroke}%
\pgfsetdash{}{0pt}%
\pgfsys@defobject{currentmarker}{\pgfqpoint{-0.041667in}{-0.041667in}}{\pgfqpoint{0.041667in}{0.041667in}}{%
\pgfpathmoveto{\pgfqpoint{0.000000in}{-0.041667in}}%
\pgfpathcurveto{\pgfqpoint{0.011050in}{-0.041667in}}{\pgfqpoint{0.021649in}{-0.037276in}}{\pgfqpoint{0.029463in}{-0.029463in}}%
\pgfpathcurveto{\pgfqpoint{0.037276in}{-0.021649in}}{\pgfqpoint{0.041667in}{-0.011050in}}{\pgfqpoint{0.041667in}{0.000000in}}%
\pgfpathcurveto{\pgfqpoint{0.041667in}{0.011050in}}{\pgfqpoint{0.037276in}{0.021649in}}{\pgfqpoint{0.029463in}{0.029463in}}%
\pgfpathcurveto{\pgfqpoint{0.021649in}{0.037276in}}{\pgfqpoint{0.011050in}{0.041667in}}{\pgfqpoint{0.000000in}{0.041667in}}%
\pgfpathcurveto{\pgfqpoint{-0.011050in}{0.041667in}}{\pgfqpoint{-0.021649in}{0.037276in}}{\pgfqpoint{-0.029463in}{0.029463in}}%
\pgfpathcurveto{\pgfqpoint{-0.037276in}{0.021649in}}{\pgfqpoint{-0.041667in}{0.011050in}}{\pgfqpoint{-0.041667in}{0.000000in}}%
\pgfpathcurveto{\pgfqpoint{-0.041667in}{-0.011050in}}{\pgfqpoint{-0.037276in}{-0.021649in}}{\pgfqpoint{-0.029463in}{-0.029463in}}%
\pgfpathcurveto{\pgfqpoint{-0.021649in}{-0.037276in}}{\pgfqpoint{-0.011050in}{-0.041667in}}{\pgfqpoint{0.000000in}{-0.041667in}}%
\pgfpathclose%
\pgfusepath{stroke,fill}%
}%
\begin{pgfscope}%
\pgfsys@transformshift{1.045291in}{2.339724in}%
\pgfsys@useobject{currentmarker}{}%
\end{pgfscope}%
\end{pgfscope}%
\begin{pgfscope}%
\definecolor{textcolor}{rgb}{0.000000,0.000000,0.000000}%
\pgfsetstrokecolor{textcolor}%
\pgfsetfillcolor{textcolor}%
\pgftext[x=1.295291in,y=2.291113in,left,base]{\color{textcolor}\rmfamily\fontsize{10.000000}{12.000000}\selectfont End to end latency}%
\end{pgfscope}%
\end{pgfpicture}%
\makeatother%
\endgroup%

    \end{center}
    \caption{Migration statistics of a vector with all pages dirty (4KB pages)}
    \label{fig:vectorwriteall}
\end{figure}

\begin{figure}[tp]
    \begin{center}
        %% Creator: Matplotlib, PGF backend
%%
%% To include the figure in your LaTeX document, write
%%   \input{<filename>.pgf}
%%
%% Make sure the required packages are loaded in your preamble
%%   \usepackage{pgf}
%%
%% and, on pdftex
%%   \usepackage[utf8]{inputenc}\DeclareUnicodeCharacter{2212}{-}
%%
%% or, on luatex and xetex
%%   \usepackage{unicode-math}
%%
%% Figures using additional raster images can only be included by \input if
%% they are in the same directory as the main LaTeX file. For loading figures
%% from other directories you can use the `import` package
%%   \usepackage{import}
%%
%% and then include the figures with
%%   \import{<path to file>}{<filename>.pgf}
%%
%% Matplotlib used the following preamble
%%
\begingroup%
\makeatletter%
\begin{pgfpicture}%
\pgfpathrectangle{\pgfpointorigin}{\pgfqpoint{6.251220in}{3.516311in}}%
\pgfusepath{use as bounding box, clip}%
\begin{pgfscope}%
\pgfsetbuttcap%
\pgfsetmiterjoin%
\definecolor{currentfill}{rgb}{1.000000,1.000000,1.000000}%
\pgfsetfillcolor{currentfill}%
\pgfsetlinewidth{0.000000pt}%
\definecolor{currentstroke}{rgb}{1.000000,1.000000,1.000000}%
\pgfsetstrokecolor{currentstroke}%
\pgfsetdash{}{0pt}%
\pgfpathmoveto{\pgfqpoint{0.000000in}{0.000000in}}%
\pgfpathlineto{\pgfqpoint{6.251220in}{0.000000in}}%
\pgfpathlineto{\pgfqpoint{6.251220in}{3.516311in}}%
\pgfpathlineto{\pgfqpoint{0.000000in}{3.516311in}}%
\pgfpathclose%
\pgfusepath{fill}%
\end{pgfscope}%
\begin{pgfscope}%
\pgfsetbuttcap%
\pgfsetmiterjoin%
\definecolor{currentfill}{rgb}{1.000000,1.000000,1.000000}%
\pgfsetfillcolor{currentfill}%
\pgfsetlinewidth{0.000000pt}%
\definecolor{currentstroke}{rgb}{0.000000,0.000000,0.000000}%
\pgfsetstrokecolor{currentstroke}%
\pgfsetstrokeopacity{0.000000}%
\pgfsetdash{}{0pt}%
\pgfpathmoveto{\pgfqpoint{0.781402in}{0.386794in}}%
\pgfpathlineto{\pgfqpoint{5.626098in}{0.386794in}}%
\pgfpathlineto{\pgfqpoint{5.626098in}{3.094354in}}%
\pgfpathlineto{\pgfqpoint{0.781402in}{3.094354in}}%
\pgfpathclose%
\pgfusepath{fill}%
\end{pgfscope}%
\begin{pgfscope}%
\pgfsetbuttcap%
\pgfsetroundjoin%
\definecolor{currentfill}{rgb}{0.000000,0.000000,0.000000}%
\pgfsetfillcolor{currentfill}%
\pgfsetlinewidth{0.803000pt}%
\definecolor{currentstroke}{rgb}{0.000000,0.000000,0.000000}%
\pgfsetstrokecolor{currentstroke}%
\pgfsetdash{}{0pt}%
\pgfsys@defobject{currentmarker}{\pgfqpoint{0.000000in}{-0.048611in}}{\pgfqpoint{0.000000in}{0.000000in}}{%
\pgfpathmoveto{\pgfqpoint{0.000000in}{0.000000in}}%
\pgfpathlineto{\pgfqpoint{0.000000in}{-0.048611in}}%
\pgfusepath{stroke,fill}%
}%
\begin{pgfscope}%
\pgfsys@transformshift{1.001616in}{0.386794in}%
\pgfsys@useobject{currentmarker}{}%
\end{pgfscope}%
\end{pgfscope}%
\begin{pgfscope}%
\definecolor{textcolor}{rgb}{0.000000,0.000000,0.000000}%
\pgfsetstrokecolor{textcolor}%
\pgfsetfillcolor{textcolor}%
\pgftext[x=1.001616in,y=0.289572in,,top]{\color{textcolor}\rmfamily\fontsize{10.000000}{12.000000}\selectfont \(\displaystyle {1}\)}%
\end{pgfscope}%
\begin{pgfscope}%
\pgfsetbuttcap%
\pgfsetroundjoin%
\definecolor{currentfill}{rgb}{0.000000,0.000000,0.000000}%
\pgfsetfillcolor{currentfill}%
\pgfsetlinewidth{0.803000pt}%
\definecolor{currentstroke}{rgb}{0.000000,0.000000,0.000000}%
\pgfsetstrokecolor{currentstroke}%
\pgfsetdash{}{0pt}%
\pgfsys@defobject{currentmarker}{\pgfqpoint{0.000000in}{-0.048611in}}{\pgfqpoint{0.000000in}{0.000000in}}{%
\pgfpathmoveto{\pgfqpoint{0.000000in}{0.000000in}}%
\pgfpathlineto{\pgfqpoint{0.000000in}{-0.048611in}}%
\pgfusepath{stroke,fill}%
}%
\begin{pgfscope}%
\pgfsys@transformshift{1.532784in}{0.386794in}%
\pgfsys@useobject{currentmarker}{}%
\end{pgfscope}%
\end{pgfscope}%
\begin{pgfscope}%
\definecolor{textcolor}{rgb}{0.000000,0.000000,0.000000}%
\pgfsetstrokecolor{textcolor}%
\pgfsetfillcolor{textcolor}%
\pgftext[x=1.532784in,y=0.289572in,,top]{\color{textcolor}\rmfamily\fontsize{10.000000}{12.000000}\selectfont \(\displaystyle {25}\)}%
\end{pgfscope}%
\begin{pgfscope}%
\pgfsetbuttcap%
\pgfsetroundjoin%
\definecolor{currentfill}{rgb}{0.000000,0.000000,0.000000}%
\pgfsetfillcolor{currentfill}%
\pgfsetlinewidth{0.803000pt}%
\definecolor{currentstroke}{rgb}{0.000000,0.000000,0.000000}%
\pgfsetstrokecolor{currentstroke}%
\pgfsetdash{}{0pt}%
\pgfsys@defobject{currentmarker}{\pgfqpoint{0.000000in}{-0.048611in}}{\pgfqpoint{0.000000in}{0.000000in}}{%
\pgfpathmoveto{\pgfqpoint{0.000000in}{0.000000in}}%
\pgfpathlineto{\pgfqpoint{0.000000in}{-0.048611in}}%
\pgfusepath{stroke,fill}%
}%
\begin{pgfscope}%
\pgfsys@transformshift{2.086084in}{0.386794in}%
\pgfsys@useobject{currentmarker}{}%
\end{pgfscope}%
\end{pgfscope}%
\begin{pgfscope}%
\definecolor{textcolor}{rgb}{0.000000,0.000000,0.000000}%
\pgfsetstrokecolor{textcolor}%
\pgfsetfillcolor{textcolor}%
\pgftext[x=2.086084in,y=0.289572in,,top]{\color{textcolor}\rmfamily\fontsize{10.000000}{12.000000}\selectfont \(\displaystyle {50}\)}%
\end{pgfscope}%
\begin{pgfscope}%
\pgfsetbuttcap%
\pgfsetroundjoin%
\definecolor{currentfill}{rgb}{0.000000,0.000000,0.000000}%
\pgfsetfillcolor{currentfill}%
\pgfsetlinewidth{0.803000pt}%
\definecolor{currentstroke}{rgb}{0.000000,0.000000,0.000000}%
\pgfsetstrokecolor{currentstroke}%
\pgfsetdash{}{0pt}%
\pgfsys@defobject{currentmarker}{\pgfqpoint{0.000000in}{-0.048611in}}{\pgfqpoint{0.000000in}{0.000000in}}{%
\pgfpathmoveto{\pgfqpoint{0.000000in}{0.000000in}}%
\pgfpathlineto{\pgfqpoint{0.000000in}{-0.048611in}}%
\pgfusepath{stroke,fill}%
}%
\begin{pgfscope}%
\pgfsys@transformshift{2.639384in}{0.386794in}%
\pgfsys@useobject{currentmarker}{}%
\end{pgfscope}%
\end{pgfscope}%
\begin{pgfscope}%
\definecolor{textcolor}{rgb}{0.000000,0.000000,0.000000}%
\pgfsetstrokecolor{textcolor}%
\pgfsetfillcolor{textcolor}%
\pgftext[x=2.639384in,y=0.289572in,,top]{\color{textcolor}\rmfamily\fontsize{10.000000}{12.000000}\selectfont \(\displaystyle {75}\)}%
\end{pgfscope}%
\begin{pgfscope}%
\pgfsetbuttcap%
\pgfsetroundjoin%
\definecolor{currentfill}{rgb}{0.000000,0.000000,0.000000}%
\pgfsetfillcolor{currentfill}%
\pgfsetlinewidth{0.803000pt}%
\definecolor{currentstroke}{rgb}{0.000000,0.000000,0.000000}%
\pgfsetstrokecolor{currentstroke}%
\pgfsetdash{}{0pt}%
\pgfsys@defobject{currentmarker}{\pgfqpoint{0.000000in}{-0.048611in}}{\pgfqpoint{0.000000in}{0.000000in}}{%
\pgfpathmoveto{\pgfqpoint{0.000000in}{0.000000in}}%
\pgfpathlineto{\pgfqpoint{0.000000in}{-0.048611in}}%
\pgfusepath{stroke,fill}%
}%
\begin{pgfscope}%
\pgfsys@transformshift{3.192684in}{0.386794in}%
\pgfsys@useobject{currentmarker}{}%
\end{pgfscope}%
\end{pgfscope}%
\begin{pgfscope}%
\definecolor{textcolor}{rgb}{0.000000,0.000000,0.000000}%
\pgfsetstrokecolor{textcolor}%
\pgfsetfillcolor{textcolor}%
\pgftext[x=3.192684in,y=0.289572in,,top]{\color{textcolor}\rmfamily\fontsize{10.000000}{12.000000}\selectfont \(\displaystyle {100}\)}%
\end{pgfscope}%
\begin{pgfscope}%
\pgfsetbuttcap%
\pgfsetroundjoin%
\definecolor{currentfill}{rgb}{0.000000,0.000000,0.000000}%
\pgfsetfillcolor{currentfill}%
\pgfsetlinewidth{0.803000pt}%
\definecolor{currentstroke}{rgb}{0.000000,0.000000,0.000000}%
\pgfsetstrokecolor{currentstroke}%
\pgfsetdash{}{0pt}%
\pgfsys@defobject{currentmarker}{\pgfqpoint{0.000000in}{-0.048611in}}{\pgfqpoint{0.000000in}{0.000000in}}{%
\pgfpathmoveto{\pgfqpoint{0.000000in}{0.000000in}}%
\pgfpathlineto{\pgfqpoint{0.000000in}{-0.048611in}}%
\pgfusepath{stroke,fill}%
}%
\begin{pgfscope}%
\pgfsys@transformshift{3.745984in}{0.386794in}%
\pgfsys@useobject{currentmarker}{}%
\end{pgfscope}%
\end{pgfscope}%
\begin{pgfscope}%
\definecolor{textcolor}{rgb}{0.000000,0.000000,0.000000}%
\pgfsetstrokecolor{textcolor}%
\pgfsetfillcolor{textcolor}%
\pgftext[x=3.745984in,y=0.289572in,,top]{\color{textcolor}\rmfamily\fontsize{10.000000}{12.000000}\selectfont \(\displaystyle {125}\)}%
\end{pgfscope}%
\begin{pgfscope}%
\pgfsetbuttcap%
\pgfsetroundjoin%
\definecolor{currentfill}{rgb}{0.000000,0.000000,0.000000}%
\pgfsetfillcolor{currentfill}%
\pgfsetlinewidth{0.803000pt}%
\definecolor{currentstroke}{rgb}{0.000000,0.000000,0.000000}%
\pgfsetstrokecolor{currentstroke}%
\pgfsetdash{}{0pt}%
\pgfsys@defobject{currentmarker}{\pgfqpoint{0.000000in}{-0.048611in}}{\pgfqpoint{0.000000in}{0.000000in}}{%
\pgfpathmoveto{\pgfqpoint{0.000000in}{0.000000in}}%
\pgfpathlineto{\pgfqpoint{0.000000in}{-0.048611in}}%
\pgfusepath{stroke,fill}%
}%
\begin{pgfscope}%
\pgfsys@transformshift{4.299284in}{0.386794in}%
\pgfsys@useobject{currentmarker}{}%
\end{pgfscope}%
\end{pgfscope}%
\begin{pgfscope}%
\definecolor{textcolor}{rgb}{0.000000,0.000000,0.000000}%
\pgfsetstrokecolor{textcolor}%
\pgfsetfillcolor{textcolor}%
\pgftext[x=4.299284in,y=0.289572in,,top]{\color{textcolor}\rmfamily\fontsize{10.000000}{12.000000}\selectfont \(\displaystyle {150}\)}%
\end{pgfscope}%
\begin{pgfscope}%
\pgfsetbuttcap%
\pgfsetroundjoin%
\definecolor{currentfill}{rgb}{0.000000,0.000000,0.000000}%
\pgfsetfillcolor{currentfill}%
\pgfsetlinewidth{0.803000pt}%
\definecolor{currentstroke}{rgb}{0.000000,0.000000,0.000000}%
\pgfsetstrokecolor{currentstroke}%
\pgfsetdash{}{0pt}%
\pgfsys@defobject{currentmarker}{\pgfqpoint{0.000000in}{-0.048611in}}{\pgfqpoint{0.000000in}{0.000000in}}{%
\pgfpathmoveto{\pgfqpoint{0.000000in}{0.000000in}}%
\pgfpathlineto{\pgfqpoint{0.000000in}{-0.048611in}}%
\pgfusepath{stroke,fill}%
}%
\begin{pgfscope}%
\pgfsys@transformshift{4.852584in}{0.386794in}%
\pgfsys@useobject{currentmarker}{}%
\end{pgfscope}%
\end{pgfscope}%
\begin{pgfscope}%
\definecolor{textcolor}{rgb}{0.000000,0.000000,0.000000}%
\pgfsetstrokecolor{textcolor}%
\pgfsetfillcolor{textcolor}%
\pgftext[x=4.852584in,y=0.289572in,,top]{\color{textcolor}\rmfamily\fontsize{10.000000}{12.000000}\selectfont \(\displaystyle {175}\)}%
\end{pgfscope}%
\begin{pgfscope}%
\pgfsetbuttcap%
\pgfsetroundjoin%
\definecolor{currentfill}{rgb}{0.000000,0.000000,0.000000}%
\pgfsetfillcolor{currentfill}%
\pgfsetlinewidth{0.803000pt}%
\definecolor{currentstroke}{rgb}{0.000000,0.000000,0.000000}%
\pgfsetstrokecolor{currentstroke}%
\pgfsetdash{}{0pt}%
\pgfsys@defobject{currentmarker}{\pgfqpoint{0.000000in}{-0.048611in}}{\pgfqpoint{0.000000in}{0.000000in}}{%
\pgfpathmoveto{\pgfqpoint{0.000000in}{0.000000in}}%
\pgfpathlineto{\pgfqpoint{0.000000in}{-0.048611in}}%
\pgfusepath{stroke,fill}%
}%
\begin{pgfscope}%
\pgfsys@transformshift{5.405885in}{0.386794in}%
\pgfsys@useobject{currentmarker}{}%
\end{pgfscope}%
\end{pgfscope}%
\begin{pgfscope}%
\definecolor{textcolor}{rgb}{0.000000,0.000000,0.000000}%
\pgfsetstrokecolor{textcolor}%
\pgfsetfillcolor{textcolor}%
\pgftext[x=5.405885in,y=0.289572in,,top]{\color{textcolor}\rmfamily\fontsize{10.000000}{12.000000}\selectfont \(\displaystyle {200}\)}%
\end{pgfscope}%
\begin{pgfscope}%
\definecolor{textcolor}{rgb}{0.000000,0.000000,0.000000}%
\pgfsetstrokecolor{textcolor}%
\pgfsetfillcolor{textcolor}%
\pgftext[x=3.203750in,y=0.110560in,,top]{\color{textcolor}\rmfamily\fontsize{10.000000}{12.000000}\selectfont Number of 2MB pages}%
\end{pgfscope}%
\begin{pgfscope}%
\pgfsetbuttcap%
\pgfsetroundjoin%
\definecolor{currentfill}{rgb}{0.000000,0.000000,0.000000}%
\pgfsetfillcolor{currentfill}%
\pgfsetlinewidth{0.803000pt}%
\definecolor{currentstroke}{rgb}{0.000000,0.000000,0.000000}%
\pgfsetstrokecolor{currentstroke}%
\pgfsetdash{}{0pt}%
\pgfsys@defobject{currentmarker}{\pgfqpoint{-0.048611in}{0.000000in}}{\pgfqpoint{-0.000000in}{0.000000in}}{%
\pgfpathmoveto{\pgfqpoint{-0.000000in}{0.000000in}}%
\pgfpathlineto{\pgfqpoint{-0.048611in}{0.000000in}}%
\pgfusepath{stroke,fill}%
}%
\begin{pgfscope}%
\pgfsys@transformshift{0.781402in}{0.509856in}%
\pgfsys@useobject{currentmarker}{}%
\end{pgfscope}%
\end{pgfscope}%
\begin{pgfscope}%
\definecolor{textcolor}{rgb}{0.000000,0.000000,0.000000}%
\pgfsetstrokecolor{textcolor}%
\pgfsetfillcolor{textcolor}%
\pgftext[x=0.614736in, y=0.461630in, left, base]{\color{textcolor}\rmfamily\fontsize{10.000000}{12.000000}\selectfont \(\displaystyle {0}\)}%
\end{pgfscope}%
\begin{pgfscope}%
\pgfsetbuttcap%
\pgfsetroundjoin%
\definecolor{currentfill}{rgb}{0.000000,0.000000,0.000000}%
\pgfsetfillcolor{currentfill}%
\pgfsetlinewidth{0.803000pt}%
\definecolor{currentstroke}{rgb}{0.000000,0.000000,0.000000}%
\pgfsetstrokecolor{currentstroke}%
\pgfsetdash{}{0pt}%
\pgfsys@defobject{currentmarker}{\pgfqpoint{-0.048611in}{0.000000in}}{\pgfqpoint{-0.000000in}{0.000000in}}{%
\pgfpathmoveto{\pgfqpoint{-0.000000in}{0.000000in}}%
\pgfpathlineto{\pgfqpoint{-0.048611in}{0.000000in}}%
\pgfusepath{stroke,fill}%
}%
\begin{pgfscope}%
\pgfsys@transformshift{0.781402in}{0.990036in}%
\pgfsys@useobject{currentmarker}{}%
\end{pgfscope}%
\end{pgfscope}%
\begin{pgfscope}%
\definecolor{textcolor}{rgb}{0.000000,0.000000,0.000000}%
\pgfsetstrokecolor{textcolor}%
\pgfsetfillcolor{textcolor}%
\pgftext[x=0.545291in, y=0.941811in, left, base]{\color{textcolor}\rmfamily\fontsize{10.000000}{12.000000}\selectfont \(\displaystyle {50}\)}%
\end{pgfscope}%
\begin{pgfscope}%
\pgfsetbuttcap%
\pgfsetroundjoin%
\definecolor{currentfill}{rgb}{0.000000,0.000000,0.000000}%
\pgfsetfillcolor{currentfill}%
\pgfsetlinewidth{0.803000pt}%
\definecolor{currentstroke}{rgb}{0.000000,0.000000,0.000000}%
\pgfsetstrokecolor{currentstroke}%
\pgfsetdash{}{0pt}%
\pgfsys@defobject{currentmarker}{\pgfqpoint{-0.048611in}{0.000000in}}{\pgfqpoint{-0.000000in}{0.000000in}}{%
\pgfpathmoveto{\pgfqpoint{-0.000000in}{0.000000in}}%
\pgfpathlineto{\pgfqpoint{-0.048611in}{0.000000in}}%
\pgfusepath{stroke,fill}%
}%
\begin{pgfscope}%
\pgfsys@transformshift{0.781402in}{1.470216in}%
\pgfsys@useobject{currentmarker}{}%
\end{pgfscope}%
\end{pgfscope}%
\begin{pgfscope}%
\definecolor{textcolor}{rgb}{0.000000,0.000000,0.000000}%
\pgfsetstrokecolor{textcolor}%
\pgfsetfillcolor{textcolor}%
\pgftext[x=0.475846in, y=1.421991in, left, base]{\color{textcolor}\rmfamily\fontsize{10.000000}{12.000000}\selectfont \(\displaystyle {100}\)}%
\end{pgfscope}%
\begin{pgfscope}%
\pgfsetbuttcap%
\pgfsetroundjoin%
\definecolor{currentfill}{rgb}{0.000000,0.000000,0.000000}%
\pgfsetfillcolor{currentfill}%
\pgfsetlinewidth{0.803000pt}%
\definecolor{currentstroke}{rgb}{0.000000,0.000000,0.000000}%
\pgfsetstrokecolor{currentstroke}%
\pgfsetdash{}{0pt}%
\pgfsys@defobject{currentmarker}{\pgfqpoint{-0.048611in}{0.000000in}}{\pgfqpoint{-0.000000in}{0.000000in}}{%
\pgfpathmoveto{\pgfqpoint{-0.000000in}{0.000000in}}%
\pgfpathlineto{\pgfqpoint{-0.048611in}{0.000000in}}%
\pgfusepath{stroke,fill}%
}%
\begin{pgfscope}%
\pgfsys@transformshift{0.781402in}{1.950397in}%
\pgfsys@useobject{currentmarker}{}%
\end{pgfscope}%
\end{pgfscope}%
\begin{pgfscope}%
\definecolor{textcolor}{rgb}{0.000000,0.000000,0.000000}%
\pgfsetstrokecolor{textcolor}%
\pgfsetfillcolor{textcolor}%
\pgftext[x=0.475846in, y=1.902171in, left, base]{\color{textcolor}\rmfamily\fontsize{10.000000}{12.000000}\selectfont \(\displaystyle {150}\)}%
\end{pgfscope}%
\begin{pgfscope}%
\pgfsetbuttcap%
\pgfsetroundjoin%
\definecolor{currentfill}{rgb}{0.000000,0.000000,0.000000}%
\pgfsetfillcolor{currentfill}%
\pgfsetlinewidth{0.803000pt}%
\definecolor{currentstroke}{rgb}{0.000000,0.000000,0.000000}%
\pgfsetstrokecolor{currentstroke}%
\pgfsetdash{}{0pt}%
\pgfsys@defobject{currentmarker}{\pgfqpoint{-0.048611in}{0.000000in}}{\pgfqpoint{-0.000000in}{0.000000in}}{%
\pgfpathmoveto{\pgfqpoint{-0.000000in}{0.000000in}}%
\pgfpathlineto{\pgfqpoint{-0.048611in}{0.000000in}}%
\pgfusepath{stroke,fill}%
}%
\begin{pgfscope}%
\pgfsys@transformshift{0.781402in}{2.430577in}%
\pgfsys@useobject{currentmarker}{}%
\end{pgfscope}%
\end{pgfscope}%
\begin{pgfscope}%
\definecolor{textcolor}{rgb}{0.000000,0.000000,0.000000}%
\pgfsetstrokecolor{textcolor}%
\pgfsetfillcolor{textcolor}%
\pgftext[x=0.475846in, y=2.382352in, left, base]{\color{textcolor}\rmfamily\fontsize{10.000000}{12.000000}\selectfont \(\displaystyle {200}\)}%
\end{pgfscope}%
\begin{pgfscope}%
\pgfsetbuttcap%
\pgfsetroundjoin%
\definecolor{currentfill}{rgb}{0.000000,0.000000,0.000000}%
\pgfsetfillcolor{currentfill}%
\pgfsetlinewidth{0.803000pt}%
\definecolor{currentstroke}{rgb}{0.000000,0.000000,0.000000}%
\pgfsetstrokecolor{currentstroke}%
\pgfsetdash{}{0pt}%
\pgfsys@defobject{currentmarker}{\pgfqpoint{-0.048611in}{0.000000in}}{\pgfqpoint{-0.000000in}{0.000000in}}{%
\pgfpathmoveto{\pgfqpoint{-0.000000in}{0.000000in}}%
\pgfpathlineto{\pgfqpoint{-0.048611in}{0.000000in}}%
\pgfusepath{stroke,fill}%
}%
\begin{pgfscope}%
\pgfsys@transformshift{0.781402in}{2.910757in}%
\pgfsys@useobject{currentmarker}{}%
\end{pgfscope}%
\end{pgfscope}%
\begin{pgfscope}%
\definecolor{textcolor}{rgb}{0.000000,0.000000,0.000000}%
\pgfsetstrokecolor{textcolor}%
\pgfsetfillcolor{textcolor}%
\pgftext[x=0.475846in, y=2.862532in, left, base]{\color{textcolor}\rmfamily\fontsize{10.000000}{12.000000}\selectfont \(\displaystyle {250}\)}%
\end{pgfscope}%
\begin{pgfscope}%
\definecolor{textcolor}{rgb}{0.000000,0.000000,0.000000}%
\pgfsetstrokecolor{textcolor}%
\pgfsetfillcolor{textcolor}%
\pgftext[x=0.420291in,y=1.740574in,,bottom,rotate=90.000000]{\color{textcolor}\rmfamily\fontsize{10.000000}{12.000000}\selectfont Elapsed time (milliseconds)}%
\end{pgfscope}%
\begin{pgfscope}%
\pgfpathrectangle{\pgfqpoint{0.781402in}{0.386794in}}{\pgfqpoint{4.844695in}{2.707560in}}%
\pgfusepath{clip}%
\pgfsetrectcap%
\pgfsetroundjoin%
\pgfsetlinewidth{1.505625pt}%
\definecolor{currentstroke}{rgb}{0.121569,0.466667,0.705882}%
\pgfsetstrokecolor{currentstroke}%
\pgfsetdash{}{0pt}%
\pgfpathmoveto{\pgfqpoint{1.001616in}{0.538691in}}%
\pgfpathlineto{\pgfqpoint{1.422124in}{0.653764in}}%
\pgfpathlineto{\pgfqpoint{1.864764in}{0.790813in}}%
\pgfpathlineto{\pgfqpoint{2.307404in}{0.925940in}}%
\pgfpathlineto{\pgfqpoint{2.750044in}{1.049097in}}%
\pgfpathlineto{\pgfqpoint{3.192684in}{1.176169in}}%
\pgfpathlineto{\pgfqpoint{3.635324in}{1.307294in}}%
\pgfpathlineto{\pgfqpoint{4.077964in}{1.434451in}}%
\pgfpathlineto{\pgfqpoint{4.520604in}{1.562454in}}%
\pgfpathlineto{\pgfqpoint{4.963245in}{1.683361in}}%
\pgfpathlineto{\pgfqpoint{5.405885in}{1.815963in}}%
\pgfusepath{stroke}%
\end{pgfscope}%
\begin{pgfscope}%
\pgfpathrectangle{\pgfqpoint{0.781402in}{0.386794in}}{\pgfqpoint{4.844695in}{2.707560in}}%
\pgfusepath{clip}%
\pgfsetbuttcap%
\pgfsetroundjoin%
\definecolor{currentfill}{rgb}{0.121569,0.466667,0.705882}%
\pgfsetfillcolor{currentfill}%
\pgfsetlinewidth{1.003750pt}%
\definecolor{currentstroke}{rgb}{0.121569,0.466667,0.705882}%
\pgfsetstrokecolor{currentstroke}%
\pgfsetdash{}{0pt}%
\pgfsys@defobject{currentmarker}{\pgfqpoint{-0.041667in}{-0.041667in}}{\pgfqpoint{0.041667in}{0.041667in}}{%
\pgfpathmoveto{\pgfqpoint{0.000000in}{-0.041667in}}%
\pgfpathcurveto{\pgfqpoint{0.011050in}{-0.041667in}}{\pgfqpoint{0.021649in}{-0.037276in}}{\pgfqpoint{0.029463in}{-0.029463in}}%
\pgfpathcurveto{\pgfqpoint{0.037276in}{-0.021649in}}{\pgfqpoint{0.041667in}{-0.011050in}}{\pgfqpoint{0.041667in}{0.000000in}}%
\pgfpathcurveto{\pgfqpoint{0.041667in}{0.011050in}}{\pgfqpoint{0.037276in}{0.021649in}}{\pgfqpoint{0.029463in}{0.029463in}}%
\pgfpathcurveto{\pgfqpoint{0.021649in}{0.037276in}}{\pgfqpoint{0.011050in}{0.041667in}}{\pgfqpoint{0.000000in}{0.041667in}}%
\pgfpathcurveto{\pgfqpoint{-0.011050in}{0.041667in}}{\pgfqpoint{-0.021649in}{0.037276in}}{\pgfqpoint{-0.029463in}{0.029463in}}%
\pgfpathcurveto{\pgfqpoint{-0.037276in}{0.021649in}}{\pgfqpoint{-0.041667in}{0.011050in}}{\pgfqpoint{-0.041667in}{0.000000in}}%
\pgfpathcurveto{\pgfqpoint{-0.041667in}{-0.011050in}}{\pgfqpoint{-0.037276in}{-0.021649in}}{\pgfqpoint{-0.029463in}{-0.029463in}}%
\pgfpathcurveto{\pgfqpoint{-0.021649in}{-0.037276in}}{\pgfqpoint{-0.011050in}{-0.041667in}}{\pgfqpoint{0.000000in}{-0.041667in}}%
\pgfpathclose%
\pgfusepath{stroke,fill}%
}%
\begin{pgfscope}%
\pgfsys@transformshift{1.001616in}{0.538691in}%
\pgfsys@useobject{currentmarker}{}%
\end{pgfscope}%
\begin{pgfscope}%
\pgfsys@transformshift{1.422124in}{0.653764in}%
\pgfsys@useobject{currentmarker}{}%
\end{pgfscope}%
\begin{pgfscope}%
\pgfsys@transformshift{1.864764in}{0.790813in}%
\pgfsys@useobject{currentmarker}{}%
\end{pgfscope}%
\begin{pgfscope}%
\pgfsys@transformshift{2.307404in}{0.925940in}%
\pgfsys@useobject{currentmarker}{}%
\end{pgfscope}%
\begin{pgfscope}%
\pgfsys@transformshift{2.750044in}{1.049097in}%
\pgfsys@useobject{currentmarker}{}%
\end{pgfscope}%
\begin{pgfscope}%
\pgfsys@transformshift{3.192684in}{1.176169in}%
\pgfsys@useobject{currentmarker}{}%
\end{pgfscope}%
\begin{pgfscope}%
\pgfsys@transformshift{3.635324in}{1.307294in}%
\pgfsys@useobject{currentmarker}{}%
\end{pgfscope}%
\begin{pgfscope}%
\pgfsys@transformshift{4.077964in}{1.434451in}%
\pgfsys@useobject{currentmarker}{}%
\end{pgfscope}%
\begin{pgfscope}%
\pgfsys@transformshift{4.520604in}{1.562454in}%
\pgfsys@useobject{currentmarker}{}%
\end{pgfscope}%
\begin{pgfscope}%
\pgfsys@transformshift{4.963245in}{1.683361in}%
\pgfsys@useobject{currentmarker}{}%
\end{pgfscope}%
\begin{pgfscope}%
\pgfsys@transformshift{5.405885in}{1.815963in}%
\pgfsys@useobject{currentmarker}{}%
\end{pgfscope}%
\end{pgfscope}%
\begin{pgfscope}%
\pgfpathrectangle{\pgfqpoint{0.781402in}{0.386794in}}{\pgfqpoint{4.844695in}{2.707560in}}%
\pgfusepath{clip}%
\pgfsetrectcap%
\pgfsetroundjoin%
\pgfsetlinewidth{1.505625pt}%
\definecolor{currentstroke}{rgb}{1.000000,0.498039,0.054902}%
\pgfsetstrokecolor{currentstroke}%
\pgfsetdash{}{0pt}%
\pgfpathmoveto{\pgfqpoint{1.001616in}{0.509870in}}%
\pgfpathlineto{\pgfqpoint{1.422124in}{0.509866in}}%
\pgfpathlineto{\pgfqpoint{1.864764in}{0.509868in}}%
\pgfpathlineto{\pgfqpoint{2.307404in}{0.509866in}}%
\pgfpathlineto{\pgfqpoint{2.750044in}{0.509865in}}%
\pgfpathlineto{\pgfqpoint{3.192684in}{0.509870in}}%
\pgfpathlineto{\pgfqpoint{3.635324in}{0.509867in}}%
\pgfpathlineto{\pgfqpoint{4.077964in}{0.509868in}}%
\pgfpathlineto{\pgfqpoint{4.520604in}{0.509872in}}%
\pgfpathlineto{\pgfqpoint{4.963245in}{0.509869in}}%
\pgfpathlineto{\pgfqpoint{5.405885in}{0.509866in}}%
\pgfusepath{stroke}%
\end{pgfscope}%
\begin{pgfscope}%
\pgfpathrectangle{\pgfqpoint{0.781402in}{0.386794in}}{\pgfqpoint{4.844695in}{2.707560in}}%
\pgfusepath{clip}%
\pgfsetbuttcap%
\pgfsetroundjoin%
\definecolor{currentfill}{rgb}{1.000000,0.498039,0.054902}%
\pgfsetfillcolor{currentfill}%
\pgfsetlinewidth{1.003750pt}%
\definecolor{currentstroke}{rgb}{1.000000,0.498039,0.054902}%
\pgfsetstrokecolor{currentstroke}%
\pgfsetdash{}{0pt}%
\pgfsys@defobject{currentmarker}{\pgfqpoint{-0.041667in}{-0.041667in}}{\pgfqpoint{0.041667in}{0.041667in}}{%
\pgfpathmoveto{\pgfqpoint{0.000000in}{-0.041667in}}%
\pgfpathcurveto{\pgfqpoint{0.011050in}{-0.041667in}}{\pgfqpoint{0.021649in}{-0.037276in}}{\pgfqpoint{0.029463in}{-0.029463in}}%
\pgfpathcurveto{\pgfqpoint{0.037276in}{-0.021649in}}{\pgfqpoint{0.041667in}{-0.011050in}}{\pgfqpoint{0.041667in}{0.000000in}}%
\pgfpathcurveto{\pgfqpoint{0.041667in}{0.011050in}}{\pgfqpoint{0.037276in}{0.021649in}}{\pgfqpoint{0.029463in}{0.029463in}}%
\pgfpathcurveto{\pgfqpoint{0.021649in}{0.037276in}}{\pgfqpoint{0.011050in}{0.041667in}}{\pgfqpoint{0.000000in}{0.041667in}}%
\pgfpathcurveto{\pgfqpoint{-0.011050in}{0.041667in}}{\pgfqpoint{-0.021649in}{0.037276in}}{\pgfqpoint{-0.029463in}{0.029463in}}%
\pgfpathcurveto{\pgfqpoint{-0.037276in}{0.021649in}}{\pgfqpoint{-0.041667in}{0.011050in}}{\pgfqpoint{-0.041667in}{0.000000in}}%
\pgfpathcurveto{\pgfqpoint{-0.041667in}{-0.011050in}}{\pgfqpoint{-0.037276in}{-0.021649in}}{\pgfqpoint{-0.029463in}{-0.029463in}}%
\pgfpathcurveto{\pgfqpoint{-0.021649in}{-0.037276in}}{\pgfqpoint{-0.011050in}{-0.041667in}}{\pgfqpoint{0.000000in}{-0.041667in}}%
\pgfpathclose%
\pgfusepath{stroke,fill}%
}%
\begin{pgfscope}%
\pgfsys@transformshift{1.001616in}{0.509870in}%
\pgfsys@useobject{currentmarker}{}%
\end{pgfscope}%
\begin{pgfscope}%
\pgfsys@transformshift{1.422124in}{0.509866in}%
\pgfsys@useobject{currentmarker}{}%
\end{pgfscope}%
\begin{pgfscope}%
\pgfsys@transformshift{1.864764in}{0.509868in}%
\pgfsys@useobject{currentmarker}{}%
\end{pgfscope}%
\begin{pgfscope}%
\pgfsys@transformshift{2.307404in}{0.509866in}%
\pgfsys@useobject{currentmarker}{}%
\end{pgfscope}%
\begin{pgfscope}%
\pgfsys@transformshift{2.750044in}{0.509865in}%
\pgfsys@useobject{currentmarker}{}%
\end{pgfscope}%
\begin{pgfscope}%
\pgfsys@transformshift{3.192684in}{0.509870in}%
\pgfsys@useobject{currentmarker}{}%
\end{pgfscope}%
\begin{pgfscope}%
\pgfsys@transformshift{3.635324in}{0.509867in}%
\pgfsys@useobject{currentmarker}{}%
\end{pgfscope}%
\begin{pgfscope}%
\pgfsys@transformshift{4.077964in}{0.509868in}%
\pgfsys@useobject{currentmarker}{}%
\end{pgfscope}%
\begin{pgfscope}%
\pgfsys@transformshift{4.520604in}{0.509872in}%
\pgfsys@useobject{currentmarker}{}%
\end{pgfscope}%
\begin{pgfscope}%
\pgfsys@transformshift{4.963245in}{0.509869in}%
\pgfsys@useobject{currentmarker}{}%
\end{pgfscope}%
\begin{pgfscope}%
\pgfsys@transformshift{5.405885in}{0.509866in}%
\pgfsys@useobject{currentmarker}{}%
\end{pgfscope}%
\end{pgfscope}%
\begin{pgfscope}%
\pgfpathrectangle{\pgfqpoint{0.781402in}{0.386794in}}{\pgfqpoint{4.844695in}{2.707560in}}%
\pgfusepath{clip}%
\pgfsetrectcap%
\pgfsetroundjoin%
\pgfsetlinewidth{1.505625pt}%
\definecolor{currentstroke}{rgb}{0.172549,0.627451,0.172549}%
\pgfsetstrokecolor{currentstroke}%
\pgfsetdash{}{0pt}%
\pgfpathmoveto{\pgfqpoint{1.001616in}{0.511996in}}%
\pgfpathlineto{\pgfqpoint{1.422124in}{0.547502in}}%
\pgfpathlineto{\pgfqpoint{1.864764in}{0.583834in}}%
\pgfpathlineto{\pgfqpoint{2.307404in}{0.619413in}}%
\pgfpathlineto{\pgfqpoint{2.750044in}{0.660666in}}%
\pgfpathlineto{\pgfqpoint{3.192684in}{0.697100in}}%
\pgfpathlineto{\pgfqpoint{3.635324in}{0.740352in}}%
\pgfpathlineto{\pgfqpoint{4.077964in}{0.785074in}}%
\pgfpathlineto{\pgfqpoint{4.520604in}{0.824987in}}%
\pgfpathlineto{\pgfqpoint{4.963245in}{0.847650in}}%
\pgfpathlineto{\pgfqpoint{5.405885in}{0.901817in}}%
\pgfusepath{stroke}%
\end{pgfscope}%
\begin{pgfscope}%
\pgfpathrectangle{\pgfqpoint{0.781402in}{0.386794in}}{\pgfqpoint{4.844695in}{2.707560in}}%
\pgfusepath{clip}%
\pgfsetbuttcap%
\pgfsetroundjoin%
\definecolor{currentfill}{rgb}{0.172549,0.627451,0.172549}%
\pgfsetfillcolor{currentfill}%
\pgfsetlinewidth{1.003750pt}%
\definecolor{currentstroke}{rgb}{0.172549,0.627451,0.172549}%
\pgfsetstrokecolor{currentstroke}%
\pgfsetdash{}{0pt}%
\pgfsys@defobject{currentmarker}{\pgfqpoint{-0.041667in}{-0.041667in}}{\pgfqpoint{0.041667in}{0.041667in}}{%
\pgfpathmoveto{\pgfqpoint{0.000000in}{-0.041667in}}%
\pgfpathcurveto{\pgfqpoint{0.011050in}{-0.041667in}}{\pgfqpoint{0.021649in}{-0.037276in}}{\pgfqpoint{0.029463in}{-0.029463in}}%
\pgfpathcurveto{\pgfqpoint{0.037276in}{-0.021649in}}{\pgfqpoint{0.041667in}{-0.011050in}}{\pgfqpoint{0.041667in}{0.000000in}}%
\pgfpathcurveto{\pgfqpoint{0.041667in}{0.011050in}}{\pgfqpoint{0.037276in}{0.021649in}}{\pgfqpoint{0.029463in}{0.029463in}}%
\pgfpathcurveto{\pgfqpoint{0.021649in}{0.037276in}}{\pgfqpoint{0.011050in}{0.041667in}}{\pgfqpoint{0.000000in}{0.041667in}}%
\pgfpathcurveto{\pgfqpoint{-0.011050in}{0.041667in}}{\pgfqpoint{-0.021649in}{0.037276in}}{\pgfqpoint{-0.029463in}{0.029463in}}%
\pgfpathcurveto{\pgfqpoint{-0.037276in}{0.021649in}}{\pgfqpoint{-0.041667in}{0.011050in}}{\pgfqpoint{-0.041667in}{0.000000in}}%
\pgfpathcurveto{\pgfqpoint{-0.041667in}{-0.011050in}}{\pgfqpoint{-0.037276in}{-0.021649in}}{\pgfqpoint{-0.029463in}{-0.029463in}}%
\pgfpathcurveto{\pgfqpoint{-0.021649in}{-0.037276in}}{\pgfqpoint{-0.011050in}{-0.041667in}}{\pgfqpoint{0.000000in}{-0.041667in}}%
\pgfpathclose%
\pgfusepath{stroke,fill}%
}%
\begin{pgfscope}%
\pgfsys@transformshift{1.001616in}{0.511996in}%
\pgfsys@useobject{currentmarker}{}%
\end{pgfscope}%
\begin{pgfscope}%
\pgfsys@transformshift{1.422124in}{0.547502in}%
\pgfsys@useobject{currentmarker}{}%
\end{pgfscope}%
\begin{pgfscope}%
\pgfsys@transformshift{1.864764in}{0.583834in}%
\pgfsys@useobject{currentmarker}{}%
\end{pgfscope}%
\begin{pgfscope}%
\pgfsys@transformshift{2.307404in}{0.619413in}%
\pgfsys@useobject{currentmarker}{}%
\end{pgfscope}%
\begin{pgfscope}%
\pgfsys@transformshift{2.750044in}{0.660666in}%
\pgfsys@useobject{currentmarker}{}%
\end{pgfscope}%
\begin{pgfscope}%
\pgfsys@transformshift{3.192684in}{0.697100in}%
\pgfsys@useobject{currentmarker}{}%
\end{pgfscope}%
\begin{pgfscope}%
\pgfsys@transformshift{3.635324in}{0.740352in}%
\pgfsys@useobject{currentmarker}{}%
\end{pgfscope}%
\begin{pgfscope}%
\pgfsys@transformshift{4.077964in}{0.785074in}%
\pgfsys@useobject{currentmarker}{}%
\end{pgfscope}%
\begin{pgfscope}%
\pgfsys@transformshift{4.520604in}{0.824987in}%
\pgfsys@useobject{currentmarker}{}%
\end{pgfscope}%
\begin{pgfscope}%
\pgfsys@transformshift{4.963245in}{0.847650in}%
\pgfsys@useobject{currentmarker}{}%
\end{pgfscope}%
\begin{pgfscope}%
\pgfsys@transformshift{5.405885in}{0.901817in}%
\pgfsys@useobject{currentmarker}{}%
\end{pgfscope}%
\end{pgfscope}%
\begin{pgfscope}%
\pgfpathrectangle{\pgfqpoint{0.781402in}{0.386794in}}{\pgfqpoint{4.844695in}{2.707560in}}%
\pgfusepath{clip}%
\pgfsetrectcap%
\pgfsetroundjoin%
\pgfsetlinewidth{1.505625pt}%
\definecolor{currentstroke}{rgb}{0.839216,0.152941,0.156863}%
\pgfsetstrokecolor{currentstroke}%
\pgfsetdash{}{0pt}%
\pgfpathmoveto{\pgfqpoint{1.001616in}{0.543101in}}%
\pgfpathlineto{\pgfqpoint{1.422124in}{0.788850in}}%
\pgfpathlineto{\pgfqpoint{1.864764in}{1.016269in}}%
\pgfpathlineto{\pgfqpoint{2.307404in}{1.263417in}}%
\pgfpathlineto{\pgfqpoint{2.750044in}{1.523859in}}%
\pgfpathlineto{\pgfqpoint{3.192684in}{1.763462in}}%
\pgfpathlineto{\pgfqpoint{3.635324in}{2.024801in}}%
\pgfpathlineto{\pgfqpoint{4.077964in}{2.265805in}}%
\pgfpathlineto{\pgfqpoint{4.520604in}{2.487341in}}%
\pgfpathlineto{\pgfqpoint{4.963245in}{2.726813in}}%
\pgfpathlineto{\pgfqpoint{5.405885in}{2.971283in}}%
\pgfusepath{stroke}%
\end{pgfscope}%
\begin{pgfscope}%
\pgfpathrectangle{\pgfqpoint{0.781402in}{0.386794in}}{\pgfqpoint{4.844695in}{2.707560in}}%
\pgfusepath{clip}%
\pgfsetbuttcap%
\pgfsetroundjoin%
\definecolor{currentfill}{rgb}{0.839216,0.152941,0.156863}%
\pgfsetfillcolor{currentfill}%
\pgfsetlinewidth{1.003750pt}%
\definecolor{currentstroke}{rgb}{0.839216,0.152941,0.156863}%
\pgfsetstrokecolor{currentstroke}%
\pgfsetdash{}{0pt}%
\pgfsys@defobject{currentmarker}{\pgfqpoint{-0.041667in}{-0.041667in}}{\pgfqpoint{0.041667in}{0.041667in}}{%
\pgfpathmoveto{\pgfqpoint{0.000000in}{-0.041667in}}%
\pgfpathcurveto{\pgfqpoint{0.011050in}{-0.041667in}}{\pgfqpoint{0.021649in}{-0.037276in}}{\pgfqpoint{0.029463in}{-0.029463in}}%
\pgfpathcurveto{\pgfqpoint{0.037276in}{-0.021649in}}{\pgfqpoint{0.041667in}{-0.011050in}}{\pgfqpoint{0.041667in}{0.000000in}}%
\pgfpathcurveto{\pgfqpoint{0.041667in}{0.011050in}}{\pgfqpoint{0.037276in}{0.021649in}}{\pgfqpoint{0.029463in}{0.029463in}}%
\pgfpathcurveto{\pgfqpoint{0.021649in}{0.037276in}}{\pgfqpoint{0.011050in}{0.041667in}}{\pgfqpoint{0.000000in}{0.041667in}}%
\pgfpathcurveto{\pgfqpoint{-0.011050in}{0.041667in}}{\pgfqpoint{-0.021649in}{0.037276in}}{\pgfqpoint{-0.029463in}{0.029463in}}%
\pgfpathcurveto{\pgfqpoint{-0.037276in}{0.021649in}}{\pgfqpoint{-0.041667in}{0.011050in}}{\pgfqpoint{-0.041667in}{0.000000in}}%
\pgfpathcurveto{\pgfqpoint{-0.041667in}{-0.011050in}}{\pgfqpoint{-0.037276in}{-0.021649in}}{\pgfqpoint{-0.029463in}{-0.029463in}}%
\pgfpathcurveto{\pgfqpoint{-0.021649in}{-0.037276in}}{\pgfqpoint{-0.011050in}{-0.041667in}}{\pgfqpoint{0.000000in}{-0.041667in}}%
\pgfpathclose%
\pgfusepath{stroke,fill}%
}%
\begin{pgfscope}%
\pgfsys@transformshift{1.001616in}{0.543101in}%
\pgfsys@useobject{currentmarker}{}%
\end{pgfscope}%
\begin{pgfscope}%
\pgfsys@transformshift{1.422124in}{0.788850in}%
\pgfsys@useobject{currentmarker}{}%
\end{pgfscope}%
\begin{pgfscope}%
\pgfsys@transformshift{1.864764in}{1.016269in}%
\pgfsys@useobject{currentmarker}{}%
\end{pgfscope}%
\begin{pgfscope}%
\pgfsys@transformshift{2.307404in}{1.263417in}%
\pgfsys@useobject{currentmarker}{}%
\end{pgfscope}%
\begin{pgfscope}%
\pgfsys@transformshift{2.750044in}{1.523859in}%
\pgfsys@useobject{currentmarker}{}%
\end{pgfscope}%
\begin{pgfscope}%
\pgfsys@transformshift{3.192684in}{1.763462in}%
\pgfsys@useobject{currentmarker}{}%
\end{pgfscope}%
\begin{pgfscope}%
\pgfsys@transformshift{3.635324in}{2.024801in}%
\pgfsys@useobject{currentmarker}{}%
\end{pgfscope}%
\begin{pgfscope}%
\pgfsys@transformshift{4.077964in}{2.265805in}%
\pgfsys@useobject{currentmarker}{}%
\end{pgfscope}%
\begin{pgfscope}%
\pgfsys@transformshift{4.520604in}{2.487341in}%
\pgfsys@useobject{currentmarker}{}%
\end{pgfscope}%
\begin{pgfscope}%
\pgfsys@transformshift{4.963245in}{2.726813in}%
\pgfsys@useobject{currentmarker}{}%
\end{pgfscope}%
\begin{pgfscope}%
\pgfsys@transformshift{5.405885in}{2.971283in}%
\pgfsys@useobject{currentmarker}{}%
\end{pgfscope}%
\end{pgfscope}%
\begin{pgfscope}%
\pgfsetrectcap%
\pgfsetmiterjoin%
\pgfsetlinewidth{0.803000pt}%
\definecolor{currentstroke}{rgb}{0.000000,0.000000,0.000000}%
\pgfsetstrokecolor{currentstroke}%
\pgfsetdash{}{0pt}%
\pgfpathmoveto{\pgfqpoint{0.781402in}{0.386794in}}%
\pgfpathlineto{\pgfqpoint{0.781402in}{3.094354in}}%
\pgfusepath{stroke}%
\end{pgfscope}%
\begin{pgfscope}%
\pgfsetrectcap%
\pgfsetmiterjoin%
\pgfsetlinewidth{0.803000pt}%
\definecolor{currentstroke}{rgb}{0.000000,0.000000,0.000000}%
\pgfsetstrokecolor{currentstroke}%
\pgfsetdash{}{0pt}%
\pgfpathmoveto{\pgfqpoint{5.626098in}{0.386794in}}%
\pgfpathlineto{\pgfqpoint{5.626098in}{3.094354in}}%
\pgfusepath{stroke}%
\end{pgfscope}%
\begin{pgfscope}%
\pgfsetrectcap%
\pgfsetmiterjoin%
\pgfsetlinewidth{0.803000pt}%
\definecolor{currentstroke}{rgb}{0.000000,0.000000,0.000000}%
\pgfsetstrokecolor{currentstroke}%
\pgfsetdash{}{0pt}%
\pgfpathmoveto{\pgfqpoint{0.781402in}{0.386794in}}%
\pgfpathlineto{\pgfqpoint{5.626098in}{0.386794in}}%
\pgfusepath{stroke}%
\end{pgfscope}%
\begin{pgfscope}%
\pgfsetrectcap%
\pgfsetmiterjoin%
\pgfsetlinewidth{0.803000pt}%
\definecolor{currentstroke}{rgb}{0.000000,0.000000,0.000000}%
\pgfsetstrokecolor{currentstroke}%
\pgfsetdash{}{0pt}%
\pgfpathmoveto{\pgfqpoint{0.781402in}{3.094354in}}%
\pgfpathlineto{\pgfqpoint{5.626098in}{3.094354in}}%
\pgfusepath{stroke}%
\end{pgfscope}%
\begin{pgfscope}%
\pgfsetbuttcap%
\pgfsetmiterjoin%
\definecolor{currentfill}{rgb}{1.000000,1.000000,1.000000}%
\pgfsetfillcolor{currentfill}%
\pgfsetfillopacity{0.800000}%
\pgfsetlinewidth{1.003750pt}%
\definecolor{currentstroke}{rgb}{0.800000,0.800000,0.800000}%
\pgfsetstrokecolor{currentstroke}%
\pgfsetstrokeopacity{0.800000}%
\pgfsetdash{}{0pt}%
\pgfpathmoveto{\pgfqpoint{0.878625in}{2.208552in}}%
\pgfpathlineto{\pgfqpoint{3.483567in}{2.208552in}}%
\pgfpathquadraticcurveto{\pgfqpoint{3.511345in}{2.208552in}}{\pgfqpoint{3.511345in}{2.236329in}}%
\pgfpathlineto{\pgfqpoint{3.511345in}{2.997132in}}%
\pgfpathquadraticcurveto{\pgfqpoint{3.511345in}{3.024909in}}{\pgfqpoint{3.483567in}{3.024909in}}%
\pgfpathlineto{\pgfqpoint{0.878625in}{3.024909in}}%
\pgfpathquadraticcurveto{\pgfqpoint{0.850847in}{3.024909in}}{\pgfqpoint{0.850847in}{2.997132in}}%
\pgfpathlineto{\pgfqpoint{0.850847in}{2.236329in}}%
\pgfpathquadraticcurveto{\pgfqpoint{0.850847in}{2.208552in}}{\pgfqpoint{0.878625in}{2.208552in}}%
\pgfpathclose%
\pgfusepath{stroke,fill}%
\end{pgfscope}%
\begin{pgfscope}%
\pgfsetrectcap%
\pgfsetroundjoin%
\pgfsetlinewidth{1.505625pt}%
\definecolor{currentstroke}{rgb}{0.121569,0.466667,0.705882}%
\pgfsetstrokecolor{currentstroke}%
\pgfsetdash{}{0pt}%
\pgfpathmoveto{\pgfqpoint{0.906402in}{2.920743in}}%
\pgfpathlineto{\pgfqpoint{1.184180in}{2.920743in}}%
\pgfusepath{stroke}%
\end{pgfscope}%
\begin{pgfscope}%
\pgfsetbuttcap%
\pgfsetroundjoin%
\definecolor{currentfill}{rgb}{0.121569,0.466667,0.705882}%
\pgfsetfillcolor{currentfill}%
\pgfsetlinewidth{1.003750pt}%
\definecolor{currentstroke}{rgb}{0.121569,0.466667,0.705882}%
\pgfsetstrokecolor{currentstroke}%
\pgfsetdash{}{0pt}%
\pgfsys@defobject{currentmarker}{\pgfqpoint{-0.041667in}{-0.041667in}}{\pgfqpoint{0.041667in}{0.041667in}}{%
\pgfpathmoveto{\pgfqpoint{0.000000in}{-0.041667in}}%
\pgfpathcurveto{\pgfqpoint{0.011050in}{-0.041667in}}{\pgfqpoint{0.021649in}{-0.037276in}}{\pgfqpoint{0.029463in}{-0.029463in}}%
\pgfpathcurveto{\pgfqpoint{0.037276in}{-0.021649in}}{\pgfqpoint{0.041667in}{-0.011050in}}{\pgfqpoint{0.041667in}{0.000000in}}%
\pgfpathcurveto{\pgfqpoint{0.041667in}{0.011050in}}{\pgfqpoint{0.037276in}{0.021649in}}{\pgfqpoint{0.029463in}{0.029463in}}%
\pgfpathcurveto{\pgfqpoint{0.021649in}{0.037276in}}{\pgfqpoint{0.011050in}{0.041667in}}{\pgfqpoint{0.000000in}{0.041667in}}%
\pgfpathcurveto{\pgfqpoint{-0.011050in}{0.041667in}}{\pgfqpoint{-0.021649in}{0.037276in}}{\pgfqpoint{-0.029463in}{0.029463in}}%
\pgfpathcurveto{\pgfqpoint{-0.037276in}{0.021649in}}{\pgfqpoint{-0.041667in}{0.011050in}}{\pgfqpoint{-0.041667in}{0.000000in}}%
\pgfpathcurveto{\pgfqpoint{-0.041667in}{-0.011050in}}{\pgfqpoint{-0.037276in}{-0.021649in}}{\pgfqpoint{-0.029463in}{-0.029463in}}%
\pgfpathcurveto{\pgfqpoint{-0.021649in}{-0.037276in}}{\pgfqpoint{-0.011050in}{-0.041667in}}{\pgfqpoint{0.000000in}{-0.041667in}}%
\pgfpathclose%
\pgfusepath{stroke,fill}%
}%
\begin{pgfscope}%
\pgfsys@transformshift{1.045291in}{2.920743in}%
\pgfsys@useobject{currentmarker}{}%
\end{pgfscope}%
\end{pgfscope}%
\begin{pgfscope}%
\definecolor{textcolor}{rgb}{0.000000,0.000000,0.000000}%
\pgfsetstrokecolor{textcolor}%
\pgfsetfillcolor{textcolor}%
\pgftext[x=1.295291in,y=2.872132in,left,base]{\color{textcolor}\rmfamily\fontsize{10.000000}{12.000000}\selectfont Prefill duration}%
\end{pgfscope}%
\begin{pgfscope}%
\pgfsetrectcap%
\pgfsetroundjoin%
\pgfsetlinewidth{1.505625pt}%
\definecolor{currentstroke}{rgb}{1.000000,0.498039,0.054902}%
\pgfsetstrokecolor{currentstroke}%
\pgfsetdash{}{0pt}%
\pgfpathmoveto{\pgfqpoint{0.906402in}{2.727070in}}%
\pgfpathlineto{\pgfqpoint{1.184180in}{2.727070in}}%
\pgfusepath{stroke}%
\end{pgfscope}%
\begin{pgfscope}%
\pgfsetbuttcap%
\pgfsetroundjoin%
\definecolor{currentfill}{rgb}{1.000000,0.498039,0.054902}%
\pgfsetfillcolor{currentfill}%
\pgfsetlinewidth{1.003750pt}%
\definecolor{currentstroke}{rgb}{1.000000,0.498039,0.054902}%
\pgfsetstrokecolor{currentstroke}%
\pgfsetdash{}{0pt}%
\pgfsys@defobject{currentmarker}{\pgfqpoint{-0.041667in}{-0.041667in}}{\pgfqpoint{0.041667in}{0.041667in}}{%
\pgfpathmoveto{\pgfqpoint{0.000000in}{-0.041667in}}%
\pgfpathcurveto{\pgfqpoint{0.011050in}{-0.041667in}}{\pgfqpoint{0.021649in}{-0.037276in}}{\pgfqpoint{0.029463in}{-0.029463in}}%
\pgfpathcurveto{\pgfqpoint{0.037276in}{-0.021649in}}{\pgfqpoint{0.041667in}{-0.011050in}}{\pgfqpoint{0.041667in}{0.000000in}}%
\pgfpathcurveto{\pgfqpoint{0.041667in}{0.011050in}}{\pgfqpoint{0.037276in}{0.021649in}}{\pgfqpoint{0.029463in}{0.029463in}}%
\pgfpathcurveto{\pgfqpoint{0.021649in}{0.037276in}}{\pgfqpoint{0.011050in}{0.041667in}}{\pgfqpoint{0.000000in}{0.041667in}}%
\pgfpathcurveto{\pgfqpoint{-0.011050in}{0.041667in}}{\pgfqpoint{-0.021649in}{0.037276in}}{\pgfqpoint{-0.029463in}{0.029463in}}%
\pgfpathcurveto{\pgfqpoint{-0.037276in}{0.021649in}}{\pgfqpoint{-0.041667in}{0.011050in}}{\pgfqpoint{-0.041667in}{0.000000in}}%
\pgfpathcurveto{\pgfqpoint{-0.041667in}{-0.011050in}}{\pgfqpoint{-0.037276in}{-0.021649in}}{\pgfqpoint{-0.029463in}{-0.029463in}}%
\pgfpathcurveto{\pgfqpoint{-0.021649in}{-0.037276in}}{\pgfqpoint{-0.011050in}{-0.041667in}}{\pgfqpoint{0.000000in}{-0.041667in}}%
\pgfpathclose%
\pgfusepath{stroke,fill}%
}%
\begin{pgfscope}%
\pgfsys@transformshift{1.045291in}{2.727070in}%
\pgfsys@useobject{currentmarker}{}%
\end{pgfscope}%
\end{pgfscope}%
\begin{pgfscope}%
\definecolor{textcolor}{rgb}{0.000000,0.000000,0.000000}%
\pgfsetstrokecolor{textcolor}%
\pgfsetfillcolor{textcolor}%
\pgftext[x=1.295291in,y=2.678459in,left,base]{\color{textcolor}\rmfamily\fontsize{10.000000}{12.000000}\selectfont Duration without owner}%
\end{pgfscope}%
\begin{pgfscope}%
\pgfsetrectcap%
\pgfsetroundjoin%
\pgfsetlinewidth{1.505625pt}%
\definecolor{currentstroke}{rgb}{0.172549,0.627451,0.172549}%
\pgfsetstrokecolor{currentstroke}%
\pgfsetdash{}{0pt}%
\pgfpathmoveto{\pgfqpoint{0.906402in}{2.533397in}}%
\pgfpathlineto{\pgfqpoint{1.184180in}{2.533397in}}%
\pgfusepath{stroke}%
\end{pgfscope}%
\begin{pgfscope}%
\pgfsetbuttcap%
\pgfsetroundjoin%
\definecolor{currentfill}{rgb}{0.172549,0.627451,0.172549}%
\pgfsetfillcolor{currentfill}%
\pgfsetlinewidth{1.003750pt}%
\definecolor{currentstroke}{rgb}{0.172549,0.627451,0.172549}%
\pgfsetstrokecolor{currentstroke}%
\pgfsetdash{}{0pt}%
\pgfsys@defobject{currentmarker}{\pgfqpoint{-0.041667in}{-0.041667in}}{\pgfqpoint{0.041667in}{0.041667in}}{%
\pgfpathmoveto{\pgfqpoint{0.000000in}{-0.041667in}}%
\pgfpathcurveto{\pgfqpoint{0.011050in}{-0.041667in}}{\pgfqpoint{0.021649in}{-0.037276in}}{\pgfqpoint{0.029463in}{-0.029463in}}%
\pgfpathcurveto{\pgfqpoint{0.037276in}{-0.021649in}}{\pgfqpoint{0.041667in}{-0.011050in}}{\pgfqpoint{0.041667in}{0.000000in}}%
\pgfpathcurveto{\pgfqpoint{0.041667in}{0.011050in}}{\pgfqpoint{0.037276in}{0.021649in}}{\pgfqpoint{0.029463in}{0.029463in}}%
\pgfpathcurveto{\pgfqpoint{0.021649in}{0.037276in}}{\pgfqpoint{0.011050in}{0.041667in}}{\pgfqpoint{0.000000in}{0.041667in}}%
\pgfpathcurveto{\pgfqpoint{-0.011050in}{0.041667in}}{\pgfqpoint{-0.021649in}{0.037276in}}{\pgfqpoint{-0.029463in}{0.029463in}}%
\pgfpathcurveto{\pgfqpoint{-0.037276in}{0.021649in}}{\pgfqpoint{-0.041667in}{0.011050in}}{\pgfqpoint{-0.041667in}{0.000000in}}%
\pgfpathcurveto{\pgfqpoint{-0.041667in}{-0.011050in}}{\pgfqpoint{-0.037276in}{-0.021649in}}{\pgfqpoint{-0.029463in}{-0.029463in}}%
\pgfpathcurveto{\pgfqpoint{-0.021649in}{-0.037276in}}{\pgfqpoint{-0.011050in}{-0.041667in}}{\pgfqpoint{0.000000in}{-0.041667in}}%
\pgfpathclose%
\pgfusepath{stroke,fill}%
}%
\begin{pgfscope}%
\pgfsys@transformshift{1.045291in}{2.533397in}%
\pgfsys@useobject{currentmarker}{}%
\end{pgfscope}%
\end{pgfscope}%
\begin{pgfscope}%
\definecolor{textcolor}{rgb}{0.000000,0.000000,0.000000}%
\pgfsetstrokecolor{textcolor}%
\pgfsetfillcolor{textcolor}%
\pgftext[x=1.295291in,y=2.484786in,left,base]{\color{textcolor}\rmfamily\fontsize{10.000000}{12.000000}\selectfont Time spent transferring dirty pages}%
\end{pgfscope}%
\begin{pgfscope}%
\pgfsetrectcap%
\pgfsetroundjoin%
\pgfsetlinewidth{1.505625pt}%
\definecolor{currentstroke}{rgb}{0.839216,0.152941,0.156863}%
\pgfsetstrokecolor{currentstroke}%
\pgfsetdash{}{0pt}%
\pgfpathmoveto{\pgfqpoint{0.906402in}{2.339724in}}%
\pgfpathlineto{\pgfqpoint{1.184180in}{2.339724in}}%
\pgfusepath{stroke}%
\end{pgfscope}%
\begin{pgfscope}%
\pgfsetbuttcap%
\pgfsetroundjoin%
\definecolor{currentfill}{rgb}{0.839216,0.152941,0.156863}%
\pgfsetfillcolor{currentfill}%
\pgfsetlinewidth{1.003750pt}%
\definecolor{currentstroke}{rgb}{0.839216,0.152941,0.156863}%
\pgfsetstrokecolor{currentstroke}%
\pgfsetdash{}{0pt}%
\pgfsys@defobject{currentmarker}{\pgfqpoint{-0.041667in}{-0.041667in}}{\pgfqpoint{0.041667in}{0.041667in}}{%
\pgfpathmoveto{\pgfqpoint{0.000000in}{-0.041667in}}%
\pgfpathcurveto{\pgfqpoint{0.011050in}{-0.041667in}}{\pgfqpoint{0.021649in}{-0.037276in}}{\pgfqpoint{0.029463in}{-0.029463in}}%
\pgfpathcurveto{\pgfqpoint{0.037276in}{-0.021649in}}{\pgfqpoint{0.041667in}{-0.011050in}}{\pgfqpoint{0.041667in}{0.000000in}}%
\pgfpathcurveto{\pgfqpoint{0.041667in}{0.011050in}}{\pgfqpoint{0.037276in}{0.021649in}}{\pgfqpoint{0.029463in}{0.029463in}}%
\pgfpathcurveto{\pgfqpoint{0.021649in}{0.037276in}}{\pgfqpoint{0.011050in}{0.041667in}}{\pgfqpoint{0.000000in}{0.041667in}}%
\pgfpathcurveto{\pgfqpoint{-0.011050in}{0.041667in}}{\pgfqpoint{-0.021649in}{0.037276in}}{\pgfqpoint{-0.029463in}{0.029463in}}%
\pgfpathcurveto{\pgfqpoint{-0.037276in}{0.021649in}}{\pgfqpoint{-0.041667in}{0.011050in}}{\pgfqpoint{-0.041667in}{0.000000in}}%
\pgfpathcurveto{\pgfqpoint{-0.041667in}{-0.011050in}}{\pgfqpoint{-0.037276in}{-0.021649in}}{\pgfqpoint{-0.029463in}{-0.029463in}}%
\pgfpathcurveto{\pgfqpoint{-0.021649in}{-0.037276in}}{\pgfqpoint{-0.011050in}{-0.041667in}}{\pgfqpoint{0.000000in}{-0.041667in}}%
\pgfpathclose%
\pgfusepath{stroke,fill}%
}%
\begin{pgfscope}%
\pgfsys@transformshift{1.045291in}{2.339724in}%
\pgfsys@useobject{currentmarker}{}%
\end{pgfscope}%
\end{pgfscope}%
\begin{pgfscope}%
\definecolor{textcolor}{rgb}{0.000000,0.000000,0.000000}%
\pgfsetstrokecolor{textcolor}%
\pgfsetfillcolor{textcolor}%
\pgftext[x=1.295291in,y=2.291113in,left,base]{\color{textcolor}\rmfamily\fontsize{10.000000}{12.000000}\selectfont End to end latency}%
\end{pgfscope}%
\end{pgfpicture}%
\makeatother%
\endgroup%

    \end{center}
    \caption{Migration statistics of a vector with all pages dirty (2MB huge pages)}
    \label{fig:vectorwriteallhp}
\end{figure}

In this micro-benchmark, we create a vector, dirty all of its pages after the
prefill phase has finished, and then finalize the transfer. Compared to the
clean scenario in \autoref{sec:cleanvec}, we need to spend extra time to
retransfer the dirty pages.

\autoref{fig:vectorwriteall} shows the result. As we would expect, the time it takes to turn over
the object ownership remains unchanged, however based on the usage of the
application, there will be a period during which the object is read-only on the
sender's side and the writes will be delayed on the receiver's side. We
explore the implications of this in \autoref{sec:evalmigfriendly} and
\autoref{sec:evalgenericobj}.

Although the same set of pages are sent/received during the prefill phase and
transferring of the dirty pages, the former takes considerably longer. This
happens because we do not need to pin the object memory to physical memory all
over again on each side.

Similar to the case with a read-only vector, we repeat the experiment with huge
pages. \autoref{fig:vectorwriteallhp} shows the result. The result resembles
those in \autoref{sec:cleanvec}, except here the elapsed time excluding the
prefill duration and the dirty page transfer duration is still significant. This
is contributed by the source machine having to execute the signal
handler while looping over the pages and dirtying them, after the
prefill phase and before the final transfer phase.


\section{Case study: bloom filter}
\label{sec:evalmigfriendly}

\begin{figure}[tp]
    \begin{center}
        %% Creator: Matplotlib, PGF backend
%%
%% To include the figure in your LaTeX document, write
%%   \input{<filename>.pgf}
%%
%% Make sure the required packages are loaded in your preamble
%%   \usepackage{pgf}
%%
%% and, on pdftex
%%   \usepackage[utf8]{inputenc}\DeclareUnicodeCharacter{2212}{-}
%%
%% or, on luatex and xetex
%%   \usepackage{unicode-math}
%%
%% Figures using additional raster images can only be included by \input if
%% they are in the same directory as the main LaTeX file. For loading figures
%% from other directories you can use the `import` package
%%   \usepackage{import}
%%
%% and then include the figures with
%%   \import{<path to file>}{<filename>.pgf}
%%
%% Matplotlib used the following preamble
%%
\begingroup%
\makeatletter%
\begin{pgfpicture}%
\pgfpathrectangle{\pgfpointorigin}{\pgfqpoint{6.251220in}{7.032623in}}%
\pgfusepath{use as bounding box, clip}%
\begin{pgfscope}%
\pgfsetbuttcap%
\pgfsetmiterjoin%
\definecolor{currentfill}{rgb}{1.000000,1.000000,1.000000}%
\pgfsetfillcolor{currentfill}%
\pgfsetlinewidth{0.000000pt}%
\definecolor{currentstroke}{rgb}{1.000000,1.000000,1.000000}%
\pgfsetstrokecolor{currentstroke}%
\pgfsetdash{}{0pt}%
\pgfpathmoveto{\pgfqpoint{0.000000in}{0.000000in}}%
\pgfpathlineto{\pgfqpoint{6.251220in}{0.000000in}}%
\pgfpathlineto{\pgfqpoint{6.251220in}{7.032623in}}%
\pgfpathlineto{\pgfqpoint{0.000000in}{7.032623in}}%
\pgfpathclose%
\pgfusepath{fill}%
\end{pgfscope}%
\begin{pgfscope}%
\pgfsetbuttcap%
\pgfsetmiterjoin%
\definecolor{currentfill}{rgb}{1.000000,1.000000,1.000000}%
\pgfsetfillcolor{currentfill}%
\pgfsetlinewidth{0.000000pt}%
\definecolor{currentstroke}{rgb}{0.000000,0.000000,0.000000}%
\pgfsetstrokecolor{currentstroke}%
\pgfsetstrokeopacity{0.000000}%
\pgfsetdash{}{0pt}%
\pgfpathmoveto{\pgfqpoint{0.781402in}{0.773588in}}%
\pgfpathlineto{\pgfqpoint{5.626098in}{0.773588in}}%
\pgfpathlineto{\pgfqpoint{5.626098in}{6.188708in}}%
\pgfpathlineto{\pgfqpoint{0.781402in}{6.188708in}}%
\pgfpathclose%
\pgfusepath{fill}%
\end{pgfscope}%
\begin{pgfscope}%
\pgfpathrectangle{\pgfqpoint{0.781402in}{0.773588in}}{\pgfqpoint{4.844695in}{5.415119in}}%
\pgfusepath{clip}%
\pgfsetbuttcap%
\pgfsetroundjoin%
\definecolor{currentfill}{rgb}{0.121569,0.466667,0.705882}%
\pgfsetfillcolor{currentfill}%
\pgfsetlinewidth{0.000000pt}%
\definecolor{currentstroke}{rgb}{0.000000,0.000000,0.000000}%
\pgfsetstrokecolor{currentstroke}%
\pgfsetdash{}{0pt}%
\pgfpathmoveto{\pgfqpoint{1.001616in}{0.773588in}}%
\pgfpathlineto{\pgfqpoint{1.001616in}{0.773588in}}%
\pgfpathlineto{\pgfqpoint{1.042993in}{0.773588in}}%
\pgfpathlineto{\pgfqpoint{1.084375in}{0.773588in}}%
\pgfpathlineto{\pgfqpoint{1.125752in}{0.773588in}}%
\pgfpathlineto{\pgfqpoint{1.167091in}{0.773588in}}%
\pgfpathlineto{\pgfqpoint{1.208458in}{0.773588in}}%
\pgfpathlineto{\pgfqpoint{1.249856in}{0.773588in}}%
\pgfpathlineto{\pgfqpoint{1.291280in}{0.773588in}}%
\pgfpathlineto{\pgfqpoint{1.332721in}{0.773588in}}%
\pgfpathlineto{\pgfqpoint{1.374186in}{0.773588in}}%
\pgfpathlineto{\pgfqpoint{1.415629in}{0.773588in}}%
\pgfpathlineto{\pgfqpoint{1.457070in}{0.773588in}}%
\pgfpathlineto{\pgfqpoint{1.498540in}{0.773588in}}%
\pgfpathlineto{\pgfqpoint{1.539972in}{0.773588in}}%
\pgfpathlineto{\pgfqpoint{1.581396in}{0.773588in}}%
\pgfpathlineto{\pgfqpoint{1.622829in}{0.773588in}}%
\pgfpathlineto{\pgfqpoint{1.664297in}{0.773588in}}%
\pgfpathlineto{\pgfqpoint{1.705741in}{0.773588in}}%
\pgfpathlineto{\pgfqpoint{1.747243in}{0.773588in}}%
\pgfpathlineto{\pgfqpoint{1.795378in}{0.773588in}}%
\pgfpathlineto{\pgfqpoint{1.849942in}{0.773588in}}%
\pgfpathlineto{\pgfqpoint{1.905453in}{0.773588in}}%
\pgfpathlineto{\pgfqpoint{1.963640in}{0.773588in}}%
\pgfpathlineto{\pgfqpoint{2.021509in}{0.773588in}}%
\pgfpathlineto{\pgfqpoint{2.078568in}{0.773588in}}%
\pgfpathlineto{\pgfqpoint{2.134427in}{0.773588in}}%
\pgfpathlineto{\pgfqpoint{2.188824in}{0.773588in}}%
\pgfpathlineto{\pgfqpoint{2.244502in}{0.773588in}}%
\pgfpathlineto{\pgfqpoint{2.300442in}{0.773588in}}%
\pgfpathlineto{\pgfqpoint{2.355457in}{0.773588in}}%
\pgfpathlineto{\pgfqpoint{2.411446in}{0.773588in}}%
\pgfpathlineto{\pgfqpoint{2.469564in}{0.773588in}}%
\pgfpathlineto{\pgfqpoint{2.529907in}{0.773588in}}%
\pgfpathlineto{\pgfqpoint{2.589001in}{0.773588in}}%
\pgfpathlineto{\pgfqpoint{2.646993in}{0.773588in}}%
\pgfpathlineto{\pgfqpoint{2.704011in}{0.773588in}}%
\pgfpathlineto{\pgfqpoint{2.760229in}{0.773588in}}%
\pgfpathlineto{\pgfqpoint{2.815997in}{0.773588in}}%
\pgfpathlineto{\pgfqpoint{2.869949in}{0.773588in}}%
\pgfpathlineto{\pgfqpoint{2.921979in}{0.773588in}}%
\pgfpathlineto{\pgfqpoint{2.972423in}{0.773588in}}%
\pgfpathlineto{\pgfqpoint{3.021960in}{0.773588in}}%
\pgfpathlineto{\pgfqpoint{3.070408in}{0.773588in}}%
\pgfpathlineto{\pgfqpoint{3.117833in}{0.773588in}}%
\pgfpathlineto{\pgfqpoint{3.164694in}{0.773588in}}%
\pgfpathlineto{\pgfqpoint{3.210786in}{0.773588in}}%
\pgfpathlineto{\pgfqpoint{3.255737in}{0.773588in}}%
\pgfpathlineto{\pgfqpoint{3.297077in}{0.773588in}}%
\pgfpathlineto{\pgfqpoint{3.337163in}{0.773588in}}%
\pgfpathlineto{\pgfqpoint{3.376505in}{0.773588in}}%
\pgfpathlineto{\pgfqpoint{3.415888in}{0.773588in}}%
\pgfpathlineto{\pgfqpoint{3.455225in}{0.773588in}}%
\pgfpathlineto{\pgfqpoint{3.494517in}{0.773588in}}%
\pgfpathlineto{\pgfqpoint{3.533833in}{0.773588in}}%
\pgfpathlineto{\pgfqpoint{3.573169in}{0.773588in}}%
\pgfpathlineto{\pgfqpoint{3.612456in}{0.773588in}}%
\pgfpathlineto{\pgfqpoint{3.651783in}{0.773588in}}%
\pgfpathlineto{\pgfqpoint{3.691109in}{0.773588in}}%
\pgfpathlineto{\pgfqpoint{3.730402in}{0.773588in}}%
\pgfpathlineto{\pgfqpoint{3.769397in}{0.773588in}}%
\pgfpathlineto{\pgfqpoint{3.806671in}{0.773588in}}%
\pgfpathlineto{\pgfqpoint{3.843637in}{0.773588in}}%
\pgfpathlineto{\pgfqpoint{3.880608in}{0.773588in}}%
\pgfpathlineto{\pgfqpoint{3.917591in}{0.773588in}}%
\pgfpathlineto{\pgfqpoint{3.954566in}{0.773588in}}%
\pgfpathlineto{\pgfqpoint{3.991588in}{0.773588in}}%
\pgfpathlineto{\pgfqpoint{4.028745in}{0.773588in}}%
\pgfpathlineto{\pgfqpoint{4.065634in}{0.773588in}}%
\pgfpathlineto{\pgfqpoint{4.101899in}{0.773588in}}%
\pgfpathlineto{\pgfqpoint{4.138055in}{0.773588in}}%
\pgfpathlineto{\pgfqpoint{4.173852in}{0.773588in}}%
\pgfpathlineto{\pgfqpoint{4.209218in}{0.773588in}}%
\pgfpathlineto{\pgfqpoint{4.244034in}{0.773588in}}%
\pgfpathlineto{\pgfqpoint{4.278250in}{0.773588in}}%
\pgfpathlineto{\pgfqpoint{4.311749in}{0.773588in}}%
\pgfpathlineto{\pgfqpoint{4.344452in}{0.773588in}}%
\pgfpathlineto{\pgfqpoint{4.376239in}{0.773588in}}%
\pgfpathlineto{\pgfqpoint{4.407030in}{0.773588in}}%
\pgfpathlineto{\pgfqpoint{4.436765in}{0.773588in}}%
\pgfpathlineto{\pgfqpoint{4.465281in}{0.773588in}}%
\pgfpathlineto{\pgfqpoint{4.492692in}{0.773588in}}%
\pgfpathlineto{\pgfqpoint{4.518427in}{0.773588in}}%
\pgfpathlineto{\pgfqpoint{4.543529in}{0.773588in}}%
\pgfpathlineto{\pgfqpoint{4.568473in}{0.773588in}}%
\pgfpathlineto{\pgfqpoint{4.593377in}{0.773588in}}%
\pgfpathlineto{\pgfqpoint{4.618263in}{0.773588in}}%
\pgfpathlineto{\pgfqpoint{4.643083in}{0.773588in}}%
\pgfpathlineto{\pgfqpoint{4.667884in}{0.773588in}}%
\pgfpathlineto{\pgfqpoint{4.692637in}{0.773588in}}%
\pgfpathlineto{\pgfqpoint{4.717447in}{0.773588in}}%
\pgfpathlineto{\pgfqpoint{4.742205in}{0.773588in}}%
\pgfpathlineto{\pgfqpoint{4.766979in}{0.773588in}}%
\pgfpathlineto{\pgfqpoint{4.791764in}{0.773588in}}%
\pgfpathlineto{\pgfqpoint{4.816503in}{0.773588in}}%
\pgfpathlineto{\pgfqpoint{4.846540in}{0.773588in}}%
\pgfpathlineto{\pgfqpoint{4.883926in}{0.773588in}}%
\pgfpathlineto{\pgfqpoint{4.921196in}{0.773588in}}%
\pgfpathlineto{\pgfqpoint{4.958462in}{0.773588in}}%
\pgfpathlineto{\pgfqpoint{4.995714in}{0.773588in}}%
\pgfpathlineto{\pgfqpoint{5.033005in}{0.773588in}}%
\pgfpathlineto{\pgfqpoint{5.070304in}{0.773588in}}%
\pgfpathlineto{\pgfqpoint{5.107610in}{0.773588in}}%
\pgfpathlineto{\pgfqpoint{5.144914in}{0.773588in}}%
\pgfpathlineto{\pgfqpoint{5.182186in}{0.773588in}}%
\pgfpathlineto{\pgfqpoint{5.219442in}{0.773588in}}%
\pgfpathlineto{\pgfqpoint{5.256693in}{0.773588in}}%
\pgfpathlineto{\pgfqpoint{5.293977in}{0.773588in}}%
\pgfpathlineto{\pgfqpoint{5.331273in}{0.773588in}}%
\pgfpathlineto{\pgfqpoint{5.368577in}{0.773588in}}%
\pgfpathlineto{\pgfqpoint{5.405885in}{0.773588in}}%
\pgfpathlineto{\pgfqpoint{5.405885in}{2.097942in}}%
\pgfpathlineto{\pgfqpoint{5.405885in}{2.097942in}}%
\pgfpathlineto{\pgfqpoint{5.368577in}{2.097942in}}%
\pgfpathlineto{\pgfqpoint{5.331273in}{2.097942in}}%
\pgfpathlineto{\pgfqpoint{5.293977in}{2.097942in}}%
\pgfpathlineto{\pgfqpoint{5.256693in}{2.097942in}}%
\pgfpathlineto{\pgfqpoint{5.219442in}{2.097942in}}%
\pgfpathlineto{\pgfqpoint{5.182186in}{2.097942in}}%
\pgfpathlineto{\pgfqpoint{5.144914in}{2.097942in}}%
\pgfpathlineto{\pgfqpoint{5.107610in}{2.097942in}}%
\pgfpathlineto{\pgfqpoint{5.070304in}{2.097942in}}%
\pgfpathlineto{\pgfqpoint{5.033005in}{2.097942in}}%
\pgfpathlineto{\pgfqpoint{4.995714in}{2.097942in}}%
\pgfpathlineto{\pgfqpoint{4.958462in}{2.097942in}}%
\pgfpathlineto{\pgfqpoint{4.921196in}{2.097942in}}%
\pgfpathlineto{\pgfqpoint{4.883926in}{2.097942in}}%
\pgfpathlineto{\pgfqpoint{4.846540in}{2.097641in}}%
\pgfpathlineto{\pgfqpoint{4.816503in}{2.096840in}}%
\pgfpathlineto{\pgfqpoint{4.791764in}{2.095655in}}%
\pgfpathlineto{\pgfqpoint{4.766979in}{2.094194in}}%
\pgfpathlineto{\pgfqpoint{4.742205in}{2.095755in}}%
\pgfpathlineto{\pgfqpoint{4.717447in}{2.097702in}}%
\pgfpathlineto{\pgfqpoint{4.692637in}{2.092538in}}%
\pgfpathlineto{\pgfqpoint{4.667884in}{2.093673in}}%
\pgfpathlineto{\pgfqpoint{4.643083in}{2.094865in}}%
\pgfpathlineto{\pgfqpoint{4.618263in}{2.096784in}}%
\pgfpathlineto{\pgfqpoint{4.593377in}{2.097212in}}%
\pgfpathlineto{\pgfqpoint{4.568473in}{2.095802in}}%
\pgfpathlineto{\pgfqpoint{4.543529in}{2.096725in}}%
\pgfpathlineto{\pgfqpoint{4.518427in}{2.079197in}}%
\pgfpathlineto{\pgfqpoint{4.492692in}{1.982241in}}%
\pgfpathlineto{\pgfqpoint{4.465281in}{1.939656in}}%
\pgfpathlineto{\pgfqpoint{4.436765in}{1.835427in}}%
\pgfpathlineto{\pgfqpoint{4.407030in}{1.664611in}}%
\pgfpathlineto{\pgfqpoint{4.376239in}{1.526622in}}%
\pgfpathlineto{\pgfqpoint{4.344452in}{1.375846in}}%
\pgfpathlineto{\pgfqpoint{4.311749in}{1.253505in}}%
\pgfpathlineto{\pgfqpoint{4.278250in}{1.148913in}}%
\pgfpathlineto{\pgfqpoint{4.244034in}{1.061085in}}%
\pgfpathlineto{\pgfqpoint{4.209218in}{0.963881in}}%
\pgfpathlineto{\pgfqpoint{4.173852in}{0.905273in}}%
\pgfpathlineto{\pgfqpoint{4.138055in}{0.865780in}}%
\pgfpathlineto{\pgfqpoint{4.101899in}{0.818043in}}%
\pgfpathlineto{\pgfqpoint{4.065634in}{0.773588in}}%
\pgfpathlineto{\pgfqpoint{4.028745in}{0.773588in}}%
\pgfpathlineto{\pgfqpoint{3.991588in}{0.773588in}}%
\pgfpathlineto{\pgfqpoint{3.954566in}{0.773588in}}%
\pgfpathlineto{\pgfqpoint{3.917591in}{0.773588in}}%
\pgfpathlineto{\pgfqpoint{3.880608in}{0.773588in}}%
\pgfpathlineto{\pgfqpoint{3.843637in}{0.773588in}}%
\pgfpathlineto{\pgfqpoint{3.806671in}{0.773588in}}%
\pgfpathlineto{\pgfqpoint{3.769397in}{0.773588in}}%
\pgfpathlineto{\pgfqpoint{3.730402in}{0.773588in}}%
\pgfpathlineto{\pgfqpoint{3.691109in}{0.773588in}}%
\pgfpathlineto{\pgfqpoint{3.651783in}{0.773588in}}%
\pgfpathlineto{\pgfqpoint{3.612456in}{0.773588in}}%
\pgfpathlineto{\pgfqpoint{3.573169in}{0.773588in}}%
\pgfpathlineto{\pgfqpoint{3.533833in}{0.773588in}}%
\pgfpathlineto{\pgfqpoint{3.494517in}{0.773588in}}%
\pgfpathlineto{\pgfqpoint{3.455225in}{0.773588in}}%
\pgfpathlineto{\pgfqpoint{3.415888in}{0.773588in}}%
\pgfpathlineto{\pgfqpoint{3.376505in}{0.773588in}}%
\pgfpathlineto{\pgfqpoint{3.337163in}{0.773588in}}%
\pgfpathlineto{\pgfqpoint{3.297077in}{0.773588in}}%
\pgfpathlineto{\pgfqpoint{3.255737in}{0.773588in}}%
\pgfpathlineto{\pgfqpoint{3.210786in}{0.773588in}}%
\pgfpathlineto{\pgfqpoint{3.164694in}{0.773588in}}%
\pgfpathlineto{\pgfqpoint{3.117833in}{0.773588in}}%
\pgfpathlineto{\pgfqpoint{3.070408in}{0.773588in}}%
\pgfpathlineto{\pgfqpoint{3.021960in}{0.773588in}}%
\pgfpathlineto{\pgfqpoint{2.972423in}{0.773588in}}%
\pgfpathlineto{\pgfqpoint{2.921979in}{0.773588in}}%
\pgfpathlineto{\pgfqpoint{2.869949in}{0.773588in}}%
\pgfpathlineto{\pgfqpoint{2.815997in}{0.773588in}}%
\pgfpathlineto{\pgfqpoint{2.760229in}{0.773588in}}%
\pgfpathlineto{\pgfqpoint{2.704011in}{0.773588in}}%
\pgfpathlineto{\pgfqpoint{2.646993in}{0.773588in}}%
\pgfpathlineto{\pgfqpoint{2.589001in}{0.773588in}}%
\pgfpathlineto{\pgfqpoint{2.529907in}{0.773588in}}%
\pgfpathlineto{\pgfqpoint{2.469564in}{0.773588in}}%
\pgfpathlineto{\pgfqpoint{2.411446in}{0.773588in}}%
\pgfpathlineto{\pgfqpoint{2.355457in}{0.773588in}}%
\pgfpathlineto{\pgfqpoint{2.300442in}{0.773588in}}%
\pgfpathlineto{\pgfqpoint{2.244502in}{0.773588in}}%
\pgfpathlineto{\pgfqpoint{2.188824in}{0.773588in}}%
\pgfpathlineto{\pgfqpoint{2.134427in}{0.773588in}}%
\pgfpathlineto{\pgfqpoint{2.078568in}{0.773588in}}%
\pgfpathlineto{\pgfqpoint{2.021509in}{0.773588in}}%
\pgfpathlineto{\pgfqpoint{1.963640in}{0.773588in}}%
\pgfpathlineto{\pgfqpoint{1.905453in}{0.773588in}}%
\pgfpathlineto{\pgfqpoint{1.849942in}{0.773588in}}%
\pgfpathlineto{\pgfqpoint{1.795378in}{0.773588in}}%
\pgfpathlineto{\pgfqpoint{1.747243in}{0.773588in}}%
\pgfpathlineto{\pgfqpoint{1.705741in}{0.773588in}}%
\pgfpathlineto{\pgfqpoint{1.664297in}{0.773588in}}%
\pgfpathlineto{\pgfqpoint{1.622829in}{0.773588in}}%
\pgfpathlineto{\pgfqpoint{1.581396in}{0.773588in}}%
\pgfpathlineto{\pgfqpoint{1.539972in}{0.773588in}}%
\pgfpathlineto{\pgfqpoint{1.498540in}{0.773588in}}%
\pgfpathlineto{\pgfqpoint{1.457070in}{0.773588in}}%
\pgfpathlineto{\pgfqpoint{1.415629in}{0.773588in}}%
\pgfpathlineto{\pgfqpoint{1.374186in}{0.773588in}}%
\pgfpathlineto{\pgfqpoint{1.332721in}{0.773588in}}%
\pgfpathlineto{\pgfqpoint{1.291280in}{0.773588in}}%
\pgfpathlineto{\pgfqpoint{1.249856in}{0.773588in}}%
\pgfpathlineto{\pgfqpoint{1.208458in}{0.773588in}}%
\pgfpathlineto{\pgfqpoint{1.167091in}{0.773588in}}%
\pgfpathlineto{\pgfqpoint{1.125752in}{0.773588in}}%
\pgfpathlineto{\pgfqpoint{1.084375in}{0.773588in}}%
\pgfpathlineto{\pgfqpoint{1.042993in}{0.773588in}}%
\pgfpathlineto{\pgfqpoint{1.001616in}{0.773588in}}%
\pgfpathclose%
\pgfusepath{fill}%
\end{pgfscope}%
\begin{pgfscope}%
\pgfpathrectangle{\pgfqpoint{0.781402in}{0.773588in}}{\pgfqpoint{4.844695in}{5.415119in}}%
\pgfusepath{clip}%
\pgfsetbuttcap%
\pgfsetroundjoin%
\definecolor{currentfill}{rgb}{1.000000,0.498039,0.054902}%
\pgfsetfillcolor{currentfill}%
\pgfsetlinewidth{0.000000pt}%
\definecolor{currentstroke}{rgb}{0.000000,0.000000,0.000000}%
\pgfsetstrokecolor{currentstroke}%
\pgfsetdash{}{0pt}%
\pgfpathmoveto{\pgfqpoint{1.001616in}{0.773588in}}%
\pgfpathlineto{\pgfqpoint{1.001616in}{0.773588in}}%
\pgfpathlineto{\pgfqpoint{1.042993in}{0.773588in}}%
\pgfpathlineto{\pgfqpoint{1.084375in}{0.773588in}}%
\pgfpathlineto{\pgfqpoint{1.125752in}{0.773588in}}%
\pgfpathlineto{\pgfqpoint{1.167091in}{0.773588in}}%
\pgfpathlineto{\pgfqpoint{1.208458in}{0.773588in}}%
\pgfpathlineto{\pgfqpoint{1.249856in}{0.773588in}}%
\pgfpathlineto{\pgfqpoint{1.291280in}{0.773588in}}%
\pgfpathlineto{\pgfqpoint{1.332721in}{0.773588in}}%
\pgfpathlineto{\pgfqpoint{1.374186in}{0.773588in}}%
\pgfpathlineto{\pgfqpoint{1.415629in}{0.773588in}}%
\pgfpathlineto{\pgfqpoint{1.457070in}{0.773588in}}%
\pgfpathlineto{\pgfqpoint{1.498540in}{0.773588in}}%
\pgfpathlineto{\pgfqpoint{1.539972in}{0.773588in}}%
\pgfpathlineto{\pgfqpoint{1.581396in}{0.773588in}}%
\pgfpathlineto{\pgfqpoint{1.622829in}{0.773588in}}%
\pgfpathlineto{\pgfqpoint{1.664297in}{0.773588in}}%
\pgfpathlineto{\pgfqpoint{1.705741in}{0.773588in}}%
\pgfpathlineto{\pgfqpoint{1.747243in}{0.773588in}}%
\pgfpathlineto{\pgfqpoint{1.795378in}{0.773588in}}%
\pgfpathlineto{\pgfqpoint{1.849942in}{0.773588in}}%
\pgfpathlineto{\pgfqpoint{1.905453in}{0.773588in}}%
\pgfpathlineto{\pgfqpoint{1.963640in}{0.773588in}}%
\pgfpathlineto{\pgfqpoint{2.021509in}{0.773588in}}%
\pgfpathlineto{\pgfqpoint{2.078568in}{0.773588in}}%
\pgfpathlineto{\pgfqpoint{2.134427in}{0.773588in}}%
\pgfpathlineto{\pgfqpoint{2.188824in}{0.773588in}}%
\pgfpathlineto{\pgfqpoint{2.244502in}{0.773588in}}%
\pgfpathlineto{\pgfqpoint{2.300442in}{0.773588in}}%
\pgfpathlineto{\pgfqpoint{2.355457in}{0.773588in}}%
\pgfpathlineto{\pgfqpoint{2.411446in}{0.773588in}}%
\pgfpathlineto{\pgfqpoint{2.469564in}{0.773588in}}%
\pgfpathlineto{\pgfqpoint{2.529907in}{0.773588in}}%
\pgfpathlineto{\pgfqpoint{2.589001in}{0.773588in}}%
\pgfpathlineto{\pgfqpoint{2.646993in}{0.773588in}}%
\pgfpathlineto{\pgfqpoint{2.704011in}{0.773588in}}%
\pgfpathlineto{\pgfqpoint{2.760229in}{0.773588in}}%
\pgfpathlineto{\pgfqpoint{2.815997in}{0.773588in}}%
\pgfpathlineto{\pgfqpoint{2.869949in}{0.773588in}}%
\pgfpathlineto{\pgfqpoint{2.921979in}{0.773588in}}%
\pgfpathlineto{\pgfqpoint{2.972423in}{0.773588in}}%
\pgfpathlineto{\pgfqpoint{3.021960in}{0.773588in}}%
\pgfpathlineto{\pgfqpoint{3.070408in}{0.773588in}}%
\pgfpathlineto{\pgfqpoint{3.117833in}{0.773588in}}%
\pgfpathlineto{\pgfqpoint{3.164694in}{0.773588in}}%
\pgfpathlineto{\pgfqpoint{3.210786in}{0.773588in}}%
\pgfpathlineto{\pgfqpoint{3.255737in}{0.773588in}}%
\pgfpathlineto{\pgfqpoint{3.297077in}{0.773588in}}%
\pgfpathlineto{\pgfqpoint{3.337163in}{0.773588in}}%
\pgfpathlineto{\pgfqpoint{3.376505in}{0.773588in}}%
\pgfpathlineto{\pgfqpoint{3.415888in}{0.773588in}}%
\pgfpathlineto{\pgfqpoint{3.455225in}{0.773588in}}%
\pgfpathlineto{\pgfqpoint{3.494517in}{0.773588in}}%
\pgfpathlineto{\pgfqpoint{3.533833in}{0.773588in}}%
\pgfpathlineto{\pgfqpoint{3.573169in}{0.773588in}}%
\pgfpathlineto{\pgfqpoint{3.612456in}{0.773588in}}%
\pgfpathlineto{\pgfqpoint{3.651783in}{0.773588in}}%
\pgfpathlineto{\pgfqpoint{3.691109in}{0.773588in}}%
\pgfpathlineto{\pgfqpoint{3.730402in}{0.773588in}}%
\pgfpathlineto{\pgfqpoint{3.769397in}{0.773588in}}%
\pgfpathlineto{\pgfqpoint{3.806671in}{0.773588in}}%
\pgfpathlineto{\pgfqpoint{3.843637in}{0.773588in}}%
\pgfpathlineto{\pgfqpoint{3.880608in}{0.773588in}}%
\pgfpathlineto{\pgfqpoint{3.917591in}{0.773588in}}%
\pgfpathlineto{\pgfqpoint{3.954566in}{0.773588in}}%
\pgfpathlineto{\pgfqpoint{3.991588in}{0.773588in}}%
\pgfpathlineto{\pgfqpoint{4.028745in}{0.773588in}}%
\pgfpathlineto{\pgfqpoint{4.065634in}{0.773588in}}%
\pgfpathlineto{\pgfqpoint{4.101899in}{0.818043in}}%
\pgfpathlineto{\pgfqpoint{4.138055in}{0.865780in}}%
\pgfpathlineto{\pgfqpoint{4.173852in}{0.905273in}}%
\pgfpathlineto{\pgfqpoint{4.209218in}{0.963881in}}%
\pgfpathlineto{\pgfqpoint{4.244034in}{1.061085in}}%
\pgfpathlineto{\pgfqpoint{4.278250in}{1.148913in}}%
\pgfpathlineto{\pgfqpoint{4.311749in}{1.253505in}}%
\pgfpathlineto{\pgfqpoint{4.344452in}{1.375846in}}%
\pgfpathlineto{\pgfqpoint{4.376239in}{1.526622in}}%
\pgfpathlineto{\pgfqpoint{4.407030in}{1.664611in}}%
\pgfpathlineto{\pgfqpoint{4.436765in}{1.835427in}}%
\pgfpathlineto{\pgfqpoint{4.465281in}{1.939656in}}%
\pgfpathlineto{\pgfqpoint{4.492692in}{1.982241in}}%
\pgfpathlineto{\pgfqpoint{4.518427in}{2.079197in}}%
\pgfpathlineto{\pgfqpoint{4.543529in}{2.096725in}}%
\pgfpathlineto{\pgfqpoint{4.568473in}{2.095802in}}%
\pgfpathlineto{\pgfqpoint{4.593377in}{2.097212in}}%
\pgfpathlineto{\pgfqpoint{4.618263in}{2.096784in}}%
\pgfpathlineto{\pgfqpoint{4.643083in}{2.094865in}}%
\pgfpathlineto{\pgfqpoint{4.667884in}{2.093673in}}%
\pgfpathlineto{\pgfqpoint{4.692637in}{2.092538in}}%
\pgfpathlineto{\pgfqpoint{4.717447in}{2.097702in}}%
\pgfpathlineto{\pgfqpoint{4.742205in}{2.095755in}}%
\pgfpathlineto{\pgfqpoint{4.766979in}{2.094194in}}%
\pgfpathlineto{\pgfqpoint{4.791764in}{2.095655in}}%
\pgfpathlineto{\pgfqpoint{4.816503in}{2.096840in}}%
\pgfpathlineto{\pgfqpoint{4.846540in}{2.097641in}}%
\pgfpathlineto{\pgfqpoint{4.883926in}{2.097942in}}%
\pgfpathlineto{\pgfqpoint{4.921196in}{2.097942in}}%
\pgfpathlineto{\pgfqpoint{4.958462in}{2.097942in}}%
\pgfpathlineto{\pgfqpoint{4.995714in}{2.097942in}}%
\pgfpathlineto{\pgfqpoint{5.033005in}{2.097942in}}%
\pgfpathlineto{\pgfqpoint{5.070304in}{2.097942in}}%
\pgfpathlineto{\pgfqpoint{5.107610in}{2.097942in}}%
\pgfpathlineto{\pgfqpoint{5.144914in}{2.097942in}}%
\pgfpathlineto{\pgfqpoint{5.182186in}{2.097942in}}%
\pgfpathlineto{\pgfqpoint{5.219442in}{2.097942in}}%
\pgfpathlineto{\pgfqpoint{5.256693in}{2.097942in}}%
\pgfpathlineto{\pgfqpoint{5.293977in}{2.097942in}}%
\pgfpathlineto{\pgfqpoint{5.331273in}{2.097942in}}%
\pgfpathlineto{\pgfqpoint{5.368577in}{2.097942in}}%
\pgfpathlineto{\pgfqpoint{5.405885in}{2.097942in}}%
\pgfpathlineto{\pgfqpoint{5.405885in}{3.363104in}}%
\pgfpathlineto{\pgfqpoint{5.405885in}{3.363104in}}%
\pgfpathlineto{\pgfqpoint{5.368577in}{3.363104in}}%
\pgfpathlineto{\pgfqpoint{5.331273in}{3.363104in}}%
\pgfpathlineto{\pgfqpoint{5.293977in}{3.363104in}}%
\pgfpathlineto{\pgfqpoint{5.256693in}{3.363104in}}%
\pgfpathlineto{\pgfqpoint{5.219442in}{3.363104in}}%
\pgfpathlineto{\pgfqpoint{5.182186in}{3.363104in}}%
\pgfpathlineto{\pgfqpoint{5.144914in}{3.363104in}}%
\pgfpathlineto{\pgfqpoint{5.107610in}{3.363104in}}%
\pgfpathlineto{\pgfqpoint{5.070304in}{3.363104in}}%
\pgfpathlineto{\pgfqpoint{5.033005in}{3.363104in}}%
\pgfpathlineto{\pgfqpoint{4.995714in}{3.363104in}}%
\pgfpathlineto{\pgfqpoint{4.958462in}{3.363104in}}%
\pgfpathlineto{\pgfqpoint{4.921196in}{3.363104in}}%
\pgfpathlineto{\pgfqpoint{4.883926in}{3.363104in}}%
\pgfpathlineto{\pgfqpoint{4.846540in}{3.362461in}}%
\pgfpathlineto{\pgfqpoint{4.816503in}{3.359250in}}%
\pgfpathlineto{\pgfqpoint{4.791764in}{3.355505in}}%
\pgfpathlineto{\pgfqpoint{4.766979in}{3.352900in}}%
\pgfpathlineto{\pgfqpoint{4.742205in}{3.354473in}}%
\pgfpathlineto{\pgfqpoint{4.717447in}{3.354720in}}%
\pgfpathlineto{\pgfqpoint{4.692637in}{3.346168in}}%
\pgfpathlineto{\pgfqpoint{4.667884in}{3.343932in}}%
\pgfpathlineto{\pgfqpoint{4.643083in}{3.337346in}}%
\pgfpathlineto{\pgfqpoint{4.618263in}{3.333835in}}%
\pgfpathlineto{\pgfqpoint{4.593377in}{3.325424in}}%
\pgfpathlineto{\pgfqpoint{4.568473in}{3.307655in}}%
\pgfpathlineto{\pgfqpoint{4.543529in}{3.285500in}}%
\pgfpathlineto{\pgfqpoint{4.518427in}{3.212436in}}%
\pgfpathlineto{\pgfqpoint{4.492692in}{2.759135in}}%
\pgfpathlineto{\pgfqpoint{4.465281in}{2.409996in}}%
\pgfpathlineto{\pgfqpoint{4.436765in}{2.147956in}}%
\pgfpathlineto{\pgfqpoint{4.407030in}{1.886287in}}%
\pgfpathlineto{\pgfqpoint{4.376239in}{1.696774in}}%
\pgfpathlineto{\pgfqpoint{4.344452in}{1.505100in}}%
\pgfpathlineto{\pgfqpoint{4.311749in}{1.354290in}}%
\pgfpathlineto{\pgfqpoint{4.278250in}{1.229140in}}%
\pgfpathlineto{\pgfqpoint{4.244034in}{1.123699in}}%
\pgfpathlineto{\pgfqpoint{4.209218in}{1.012308in}}%
\pgfpathlineto{\pgfqpoint{4.173852in}{0.943149in}}%
\pgfpathlineto{\pgfqpoint{4.138055in}{0.879063in}}%
\pgfpathlineto{\pgfqpoint{4.101899in}{0.818043in}}%
\pgfpathlineto{\pgfqpoint{4.065634in}{0.773588in}}%
\pgfpathlineto{\pgfqpoint{4.028745in}{0.773588in}}%
\pgfpathlineto{\pgfqpoint{3.991588in}{0.773588in}}%
\pgfpathlineto{\pgfqpoint{3.954566in}{0.773588in}}%
\pgfpathlineto{\pgfqpoint{3.917591in}{0.773588in}}%
\pgfpathlineto{\pgfqpoint{3.880608in}{0.773588in}}%
\pgfpathlineto{\pgfqpoint{3.843637in}{0.773588in}}%
\pgfpathlineto{\pgfqpoint{3.806671in}{0.773588in}}%
\pgfpathlineto{\pgfqpoint{3.769397in}{0.773588in}}%
\pgfpathlineto{\pgfqpoint{3.730402in}{0.773588in}}%
\pgfpathlineto{\pgfqpoint{3.691109in}{0.773588in}}%
\pgfpathlineto{\pgfqpoint{3.651783in}{0.773588in}}%
\pgfpathlineto{\pgfqpoint{3.612456in}{0.773588in}}%
\pgfpathlineto{\pgfqpoint{3.573169in}{0.773588in}}%
\pgfpathlineto{\pgfqpoint{3.533833in}{0.773588in}}%
\pgfpathlineto{\pgfqpoint{3.494517in}{0.773588in}}%
\pgfpathlineto{\pgfqpoint{3.455225in}{0.773588in}}%
\pgfpathlineto{\pgfqpoint{3.415888in}{0.773588in}}%
\pgfpathlineto{\pgfqpoint{3.376505in}{0.773588in}}%
\pgfpathlineto{\pgfqpoint{3.337163in}{0.773588in}}%
\pgfpathlineto{\pgfqpoint{3.297077in}{0.773588in}}%
\pgfpathlineto{\pgfqpoint{3.255737in}{0.773588in}}%
\pgfpathlineto{\pgfqpoint{3.210786in}{0.773588in}}%
\pgfpathlineto{\pgfqpoint{3.164694in}{0.773588in}}%
\pgfpathlineto{\pgfqpoint{3.117833in}{0.773588in}}%
\pgfpathlineto{\pgfqpoint{3.070408in}{0.773588in}}%
\pgfpathlineto{\pgfqpoint{3.021960in}{0.773588in}}%
\pgfpathlineto{\pgfqpoint{2.972423in}{0.773588in}}%
\pgfpathlineto{\pgfqpoint{2.921979in}{0.773588in}}%
\pgfpathlineto{\pgfqpoint{2.869949in}{0.773588in}}%
\pgfpathlineto{\pgfqpoint{2.815997in}{0.773588in}}%
\pgfpathlineto{\pgfqpoint{2.760229in}{0.773588in}}%
\pgfpathlineto{\pgfqpoint{2.704011in}{0.773588in}}%
\pgfpathlineto{\pgfqpoint{2.646993in}{0.773588in}}%
\pgfpathlineto{\pgfqpoint{2.589001in}{0.773588in}}%
\pgfpathlineto{\pgfqpoint{2.529907in}{0.773588in}}%
\pgfpathlineto{\pgfqpoint{2.469564in}{0.773588in}}%
\pgfpathlineto{\pgfqpoint{2.411446in}{0.773588in}}%
\pgfpathlineto{\pgfqpoint{2.355457in}{0.773588in}}%
\pgfpathlineto{\pgfqpoint{2.300442in}{0.773588in}}%
\pgfpathlineto{\pgfqpoint{2.244502in}{0.773588in}}%
\pgfpathlineto{\pgfqpoint{2.188824in}{0.773588in}}%
\pgfpathlineto{\pgfqpoint{2.134427in}{0.773588in}}%
\pgfpathlineto{\pgfqpoint{2.078568in}{0.773588in}}%
\pgfpathlineto{\pgfqpoint{2.021509in}{0.773588in}}%
\pgfpathlineto{\pgfqpoint{1.963640in}{0.773588in}}%
\pgfpathlineto{\pgfqpoint{1.905453in}{0.773588in}}%
\pgfpathlineto{\pgfqpoint{1.849942in}{0.773588in}}%
\pgfpathlineto{\pgfqpoint{1.795378in}{0.773588in}}%
\pgfpathlineto{\pgfqpoint{1.747243in}{0.773588in}}%
\pgfpathlineto{\pgfqpoint{1.705741in}{0.773588in}}%
\pgfpathlineto{\pgfqpoint{1.664297in}{0.773588in}}%
\pgfpathlineto{\pgfqpoint{1.622829in}{0.773588in}}%
\pgfpathlineto{\pgfqpoint{1.581396in}{0.773588in}}%
\pgfpathlineto{\pgfqpoint{1.539972in}{0.773588in}}%
\pgfpathlineto{\pgfqpoint{1.498540in}{0.773588in}}%
\pgfpathlineto{\pgfqpoint{1.457070in}{0.773588in}}%
\pgfpathlineto{\pgfqpoint{1.415629in}{0.773588in}}%
\pgfpathlineto{\pgfqpoint{1.374186in}{0.773588in}}%
\pgfpathlineto{\pgfqpoint{1.332721in}{0.773588in}}%
\pgfpathlineto{\pgfqpoint{1.291280in}{0.773588in}}%
\pgfpathlineto{\pgfqpoint{1.249856in}{0.773588in}}%
\pgfpathlineto{\pgfqpoint{1.208458in}{0.773588in}}%
\pgfpathlineto{\pgfqpoint{1.167091in}{0.773588in}}%
\pgfpathlineto{\pgfqpoint{1.125752in}{0.773588in}}%
\pgfpathlineto{\pgfqpoint{1.084375in}{0.773588in}}%
\pgfpathlineto{\pgfqpoint{1.042993in}{0.773588in}}%
\pgfpathlineto{\pgfqpoint{1.001616in}{0.773588in}}%
\pgfpathclose%
\pgfusepath{fill}%
\end{pgfscope}%
\begin{pgfscope}%
\pgfpathrectangle{\pgfqpoint{0.781402in}{0.773588in}}{\pgfqpoint{4.844695in}{5.415119in}}%
\pgfusepath{clip}%
\pgfsetbuttcap%
\pgfsetroundjoin%
\definecolor{currentfill}{rgb}{0.172549,0.627451,0.172549}%
\pgfsetfillcolor{currentfill}%
\pgfsetlinewidth{0.000000pt}%
\definecolor{currentstroke}{rgb}{0.000000,0.000000,0.000000}%
\pgfsetstrokecolor{currentstroke}%
\pgfsetdash{}{0pt}%
\pgfpathmoveto{\pgfqpoint{1.001616in}{1.341197in}}%
\pgfpathlineto{\pgfqpoint{1.001616in}{0.773588in}}%
\pgfpathlineto{\pgfqpoint{1.042993in}{0.773588in}}%
\pgfpathlineto{\pgfqpoint{1.084375in}{0.773588in}}%
\pgfpathlineto{\pgfqpoint{1.125752in}{0.773588in}}%
\pgfpathlineto{\pgfqpoint{1.167091in}{0.773588in}}%
\pgfpathlineto{\pgfqpoint{1.208458in}{0.773588in}}%
\pgfpathlineto{\pgfqpoint{1.249856in}{0.773588in}}%
\pgfpathlineto{\pgfqpoint{1.291280in}{0.773588in}}%
\pgfpathlineto{\pgfqpoint{1.332721in}{0.773588in}}%
\pgfpathlineto{\pgfqpoint{1.374186in}{0.773588in}}%
\pgfpathlineto{\pgfqpoint{1.415629in}{0.773588in}}%
\pgfpathlineto{\pgfqpoint{1.457070in}{0.773588in}}%
\pgfpathlineto{\pgfqpoint{1.498540in}{0.773588in}}%
\pgfpathlineto{\pgfqpoint{1.539972in}{0.773588in}}%
\pgfpathlineto{\pgfqpoint{1.581396in}{0.773588in}}%
\pgfpathlineto{\pgfqpoint{1.622829in}{0.773588in}}%
\pgfpathlineto{\pgfqpoint{1.664297in}{0.773588in}}%
\pgfpathlineto{\pgfqpoint{1.705741in}{0.773588in}}%
\pgfpathlineto{\pgfqpoint{1.747243in}{0.773588in}}%
\pgfpathlineto{\pgfqpoint{1.795378in}{0.773588in}}%
\pgfpathlineto{\pgfqpoint{1.849942in}{0.773588in}}%
\pgfpathlineto{\pgfqpoint{1.905453in}{0.773588in}}%
\pgfpathlineto{\pgfqpoint{1.963640in}{0.773588in}}%
\pgfpathlineto{\pgfqpoint{2.021509in}{0.773588in}}%
\pgfpathlineto{\pgfqpoint{2.078568in}{0.773588in}}%
\pgfpathlineto{\pgfqpoint{2.134427in}{0.773588in}}%
\pgfpathlineto{\pgfqpoint{2.188824in}{0.773588in}}%
\pgfpathlineto{\pgfqpoint{2.244502in}{0.773588in}}%
\pgfpathlineto{\pgfqpoint{2.300442in}{0.773588in}}%
\pgfpathlineto{\pgfqpoint{2.355457in}{0.773588in}}%
\pgfpathlineto{\pgfqpoint{2.411446in}{0.773588in}}%
\pgfpathlineto{\pgfqpoint{2.469564in}{0.773588in}}%
\pgfpathlineto{\pgfqpoint{2.529907in}{0.773588in}}%
\pgfpathlineto{\pgfqpoint{2.589001in}{0.773588in}}%
\pgfpathlineto{\pgfqpoint{2.646993in}{0.773588in}}%
\pgfpathlineto{\pgfqpoint{2.704011in}{0.773588in}}%
\pgfpathlineto{\pgfqpoint{2.760229in}{0.773588in}}%
\pgfpathlineto{\pgfqpoint{2.815997in}{0.773588in}}%
\pgfpathlineto{\pgfqpoint{2.869949in}{0.773588in}}%
\pgfpathlineto{\pgfqpoint{2.921979in}{0.773588in}}%
\pgfpathlineto{\pgfqpoint{2.972423in}{0.773588in}}%
\pgfpathlineto{\pgfqpoint{3.021960in}{0.773588in}}%
\pgfpathlineto{\pgfqpoint{3.070408in}{0.773588in}}%
\pgfpathlineto{\pgfqpoint{3.117833in}{0.773588in}}%
\pgfpathlineto{\pgfqpoint{3.164694in}{0.773588in}}%
\pgfpathlineto{\pgfqpoint{3.210786in}{0.773588in}}%
\pgfpathlineto{\pgfqpoint{3.255737in}{0.773588in}}%
\pgfpathlineto{\pgfqpoint{3.297077in}{0.773588in}}%
\pgfpathlineto{\pgfqpoint{3.337163in}{0.773588in}}%
\pgfpathlineto{\pgfqpoint{3.376505in}{0.773588in}}%
\pgfpathlineto{\pgfqpoint{3.415888in}{0.773588in}}%
\pgfpathlineto{\pgfqpoint{3.455225in}{0.773588in}}%
\pgfpathlineto{\pgfqpoint{3.494517in}{0.773588in}}%
\pgfpathlineto{\pgfqpoint{3.533833in}{0.773588in}}%
\pgfpathlineto{\pgfqpoint{3.573169in}{0.773588in}}%
\pgfpathlineto{\pgfqpoint{3.612456in}{0.773588in}}%
\pgfpathlineto{\pgfqpoint{3.651783in}{0.773588in}}%
\pgfpathlineto{\pgfqpoint{3.691109in}{0.773588in}}%
\pgfpathlineto{\pgfqpoint{3.730402in}{0.773588in}}%
\pgfpathlineto{\pgfqpoint{3.769397in}{0.773588in}}%
\pgfpathlineto{\pgfqpoint{3.806671in}{0.773588in}}%
\pgfpathlineto{\pgfqpoint{3.843637in}{0.773588in}}%
\pgfpathlineto{\pgfqpoint{3.880608in}{0.773588in}}%
\pgfpathlineto{\pgfqpoint{3.917591in}{0.773588in}}%
\pgfpathlineto{\pgfqpoint{3.954566in}{0.773588in}}%
\pgfpathlineto{\pgfqpoint{3.991588in}{0.773588in}}%
\pgfpathlineto{\pgfqpoint{4.028745in}{0.773588in}}%
\pgfpathlineto{\pgfqpoint{4.065634in}{0.773588in}}%
\pgfpathlineto{\pgfqpoint{4.101899in}{0.818043in}}%
\pgfpathlineto{\pgfqpoint{4.138055in}{0.879063in}}%
\pgfpathlineto{\pgfqpoint{4.173852in}{0.943149in}}%
\pgfpathlineto{\pgfqpoint{4.209218in}{1.012308in}}%
\pgfpathlineto{\pgfqpoint{4.244034in}{1.123699in}}%
\pgfpathlineto{\pgfqpoint{4.278250in}{1.229140in}}%
\pgfpathlineto{\pgfqpoint{4.311749in}{1.354290in}}%
\pgfpathlineto{\pgfqpoint{4.344452in}{1.505100in}}%
\pgfpathlineto{\pgfqpoint{4.376239in}{1.696774in}}%
\pgfpathlineto{\pgfqpoint{4.407030in}{1.886287in}}%
\pgfpathlineto{\pgfqpoint{4.436765in}{2.147956in}}%
\pgfpathlineto{\pgfqpoint{4.465281in}{2.409996in}}%
\pgfpathlineto{\pgfqpoint{4.492692in}{2.759135in}}%
\pgfpathlineto{\pgfqpoint{4.518427in}{3.212436in}}%
\pgfpathlineto{\pgfqpoint{4.543529in}{3.285500in}}%
\pgfpathlineto{\pgfqpoint{4.568473in}{3.307655in}}%
\pgfpathlineto{\pgfqpoint{4.593377in}{3.325424in}}%
\pgfpathlineto{\pgfqpoint{4.618263in}{3.333835in}}%
\pgfpathlineto{\pgfqpoint{4.643083in}{3.337346in}}%
\pgfpathlineto{\pgfqpoint{4.667884in}{3.343932in}}%
\pgfpathlineto{\pgfqpoint{4.692637in}{3.346168in}}%
\pgfpathlineto{\pgfqpoint{4.717447in}{3.354720in}}%
\pgfpathlineto{\pgfqpoint{4.742205in}{3.354473in}}%
\pgfpathlineto{\pgfqpoint{4.766979in}{3.352900in}}%
\pgfpathlineto{\pgfqpoint{4.791764in}{3.355505in}}%
\pgfpathlineto{\pgfqpoint{4.816503in}{3.359250in}}%
\pgfpathlineto{\pgfqpoint{4.846540in}{3.362461in}}%
\pgfpathlineto{\pgfqpoint{4.883926in}{3.363104in}}%
\pgfpathlineto{\pgfqpoint{4.921196in}{3.363104in}}%
\pgfpathlineto{\pgfqpoint{4.958462in}{3.363104in}}%
\pgfpathlineto{\pgfqpoint{4.995714in}{3.363104in}}%
\pgfpathlineto{\pgfqpoint{5.033005in}{3.363104in}}%
\pgfpathlineto{\pgfqpoint{5.070304in}{3.363104in}}%
\pgfpathlineto{\pgfqpoint{5.107610in}{3.363104in}}%
\pgfpathlineto{\pgfqpoint{5.144914in}{3.363104in}}%
\pgfpathlineto{\pgfqpoint{5.182186in}{3.363104in}}%
\pgfpathlineto{\pgfqpoint{5.219442in}{3.363104in}}%
\pgfpathlineto{\pgfqpoint{5.256693in}{3.363104in}}%
\pgfpathlineto{\pgfqpoint{5.293977in}{3.363104in}}%
\pgfpathlineto{\pgfqpoint{5.331273in}{3.363104in}}%
\pgfpathlineto{\pgfqpoint{5.368577in}{3.363104in}}%
\pgfpathlineto{\pgfqpoint{5.405885in}{3.363104in}}%
\pgfpathlineto{\pgfqpoint{5.405885in}{4.664590in}}%
\pgfpathlineto{\pgfqpoint{5.405885in}{4.664590in}}%
\pgfpathlineto{\pgfqpoint{5.368577in}{4.667255in}}%
\pgfpathlineto{\pgfqpoint{5.331273in}{4.671324in}}%
\pgfpathlineto{\pgfqpoint{5.293977in}{4.671707in}}%
\pgfpathlineto{\pgfqpoint{5.256693in}{4.667846in}}%
\pgfpathlineto{\pgfqpoint{5.219442in}{4.669695in}}%
\pgfpathlineto{\pgfqpoint{5.182186in}{4.670181in}}%
\pgfpathlineto{\pgfqpoint{5.144914in}{4.669325in}}%
\pgfpathlineto{\pgfqpoint{5.107610in}{4.669076in}}%
\pgfpathlineto{\pgfqpoint{5.070304in}{4.669476in}}%
\pgfpathlineto{\pgfqpoint{5.033005in}{4.670555in}}%
\pgfpathlineto{\pgfqpoint{4.995714in}{4.670339in}}%
\pgfpathlineto{\pgfqpoint{4.958462in}{4.670386in}}%
\pgfpathlineto{\pgfqpoint{4.921196in}{4.668093in}}%
\pgfpathlineto{\pgfqpoint{4.883926in}{4.668401in}}%
\pgfpathlineto{\pgfqpoint{4.846540in}{4.661995in}}%
\pgfpathlineto{\pgfqpoint{4.816503in}{4.673521in}}%
\pgfpathlineto{\pgfqpoint{4.791764in}{4.660915in}}%
\pgfpathlineto{\pgfqpoint{4.766979in}{4.665676in}}%
\pgfpathlineto{\pgfqpoint{4.742205in}{4.656381in}}%
\pgfpathlineto{\pgfqpoint{4.717447in}{4.664977in}}%
\pgfpathlineto{\pgfqpoint{4.692637in}{4.655583in}}%
\pgfpathlineto{\pgfqpoint{4.667884in}{4.653380in}}%
\pgfpathlineto{\pgfqpoint{4.643083in}{4.649944in}}%
\pgfpathlineto{\pgfqpoint{4.618263in}{4.636646in}}%
\pgfpathlineto{\pgfqpoint{4.593377in}{4.637062in}}%
\pgfpathlineto{\pgfqpoint{4.568473in}{4.609645in}}%
\pgfpathlineto{\pgfqpoint{4.543529in}{4.595634in}}%
\pgfpathlineto{\pgfqpoint{4.518427in}{4.520896in}}%
\pgfpathlineto{\pgfqpoint{4.492692in}{4.073421in}}%
\pgfpathlineto{\pgfqpoint{4.465281in}{3.731008in}}%
\pgfpathlineto{\pgfqpoint{4.436765in}{3.461934in}}%
\pgfpathlineto{\pgfqpoint{4.407030in}{3.201960in}}%
\pgfpathlineto{\pgfqpoint{4.376239in}{3.014489in}}%
\pgfpathlineto{\pgfqpoint{4.344452in}{2.818283in}}%
\pgfpathlineto{\pgfqpoint{4.311749in}{2.668023in}}%
\pgfpathlineto{\pgfqpoint{4.278250in}{2.544748in}}%
\pgfpathlineto{\pgfqpoint{4.244034in}{2.442066in}}%
\pgfpathlineto{\pgfqpoint{4.209218in}{2.328778in}}%
\pgfpathlineto{\pgfqpoint{4.173852in}{2.257032in}}%
\pgfpathlineto{\pgfqpoint{4.138055in}{2.191613in}}%
\pgfpathlineto{\pgfqpoint{4.101899in}{2.133556in}}%
\pgfpathlineto{\pgfqpoint{4.065634in}{2.091875in}}%
\pgfpathlineto{\pgfqpoint{4.028745in}{2.076258in}}%
\pgfpathlineto{\pgfqpoint{3.991588in}{2.089416in}}%
\pgfpathlineto{\pgfqpoint{3.954566in}{2.090672in}}%
\pgfpathlineto{\pgfqpoint{3.917591in}{2.091743in}}%
\pgfpathlineto{\pgfqpoint{3.880608in}{2.089291in}}%
\pgfpathlineto{\pgfqpoint{3.843637in}{2.091515in}}%
\pgfpathlineto{\pgfqpoint{3.806671in}{2.093222in}}%
\pgfpathlineto{\pgfqpoint{3.769397in}{1.762417in}}%
\pgfpathlineto{\pgfqpoint{3.730402in}{1.362032in}}%
\pgfpathlineto{\pgfqpoint{3.691109in}{1.361860in}}%
\pgfpathlineto{\pgfqpoint{3.651783in}{1.360511in}}%
\pgfpathlineto{\pgfqpoint{3.612456in}{1.360497in}}%
\pgfpathlineto{\pgfqpoint{3.573169in}{1.360712in}}%
\pgfpathlineto{\pgfqpoint{3.533833in}{1.359435in}}%
\pgfpathlineto{\pgfqpoint{3.494517in}{1.360712in}}%
\pgfpathlineto{\pgfqpoint{3.455225in}{1.359787in}}%
\pgfpathlineto{\pgfqpoint{3.415888in}{1.362416in}}%
\pgfpathlineto{\pgfqpoint{3.376505in}{1.356972in}}%
\pgfpathlineto{\pgfqpoint{3.337163in}{1.362396in}}%
\pgfpathlineto{\pgfqpoint{3.297077in}{1.339133in}}%
\pgfpathlineto{\pgfqpoint{3.255737in}{1.343240in}}%
\pgfpathlineto{\pgfqpoint{3.210786in}{1.333897in}}%
\pgfpathlineto{\pgfqpoint{3.164694in}{1.330200in}}%
\pgfpathlineto{\pgfqpoint{3.117833in}{1.328486in}}%
\pgfpathlineto{\pgfqpoint{3.070408in}{1.329105in}}%
\pgfpathlineto{\pgfqpoint{3.021960in}{1.329435in}}%
\pgfpathlineto{\pgfqpoint{2.972423in}{1.324644in}}%
\pgfpathlineto{\pgfqpoint{2.921979in}{1.325477in}}%
\pgfpathlineto{\pgfqpoint{2.869949in}{1.322957in}}%
\pgfpathlineto{\pgfqpoint{2.815997in}{1.314994in}}%
\pgfpathlineto{\pgfqpoint{2.760229in}{1.315429in}}%
\pgfpathlineto{\pgfqpoint{2.704011in}{1.318014in}}%
\pgfpathlineto{\pgfqpoint{2.646993in}{1.317405in}}%
\pgfpathlineto{\pgfqpoint{2.589001in}{1.315048in}}%
\pgfpathlineto{\pgfqpoint{2.529907in}{1.314789in}}%
\pgfpathlineto{\pgfqpoint{2.469564in}{1.313535in}}%
\pgfpathlineto{\pgfqpoint{2.411446in}{1.312654in}}%
\pgfpathlineto{\pgfqpoint{2.355457in}{1.314666in}}%
\pgfpathlineto{\pgfqpoint{2.300442in}{1.315084in}}%
\pgfpathlineto{\pgfqpoint{2.244502in}{1.315276in}}%
\pgfpathlineto{\pgfqpoint{2.188824in}{1.314691in}}%
\pgfpathlineto{\pgfqpoint{2.134427in}{1.316378in}}%
\pgfpathlineto{\pgfqpoint{2.078568in}{1.312392in}}%
\pgfpathlineto{\pgfqpoint{2.021509in}{1.313651in}}%
\pgfpathlineto{\pgfqpoint{1.963640in}{1.309627in}}%
\pgfpathlineto{\pgfqpoint{1.905453in}{1.314036in}}%
\pgfpathlineto{\pgfqpoint{1.849942in}{1.317586in}}%
\pgfpathlineto{\pgfqpoint{1.795378in}{1.311311in}}%
\pgfpathlineto{\pgfqpoint{1.747243in}{1.363597in}}%
\pgfpathlineto{\pgfqpoint{1.705741in}{1.363465in}}%
\pgfpathlineto{\pgfqpoint{1.664297in}{1.363974in}}%
\pgfpathlineto{\pgfqpoint{1.622829in}{1.364053in}}%
\pgfpathlineto{\pgfqpoint{1.581396in}{1.362069in}}%
\pgfpathlineto{\pgfqpoint{1.539972in}{1.364675in}}%
\pgfpathlineto{\pgfqpoint{1.498540in}{1.364751in}}%
\pgfpathlineto{\pgfqpoint{1.457070in}{1.362567in}}%
\pgfpathlineto{\pgfqpoint{1.415629in}{1.361473in}}%
\pgfpathlineto{\pgfqpoint{1.374186in}{1.360805in}}%
\pgfpathlineto{\pgfqpoint{1.332721in}{1.362715in}}%
\pgfpathlineto{\pgfqpoint{1.291280in}{1.364344in}}%
\pgfpathlineto{\pgfqpoint{1.249856in}{1.360154in}}%
\pgfpathlineto{\pgfqpoint{1.208458in}{1.362371in}}%
\pgfpathlineto{\pgfqpoint{1.167091in}{1.360322in}}%
\pgfpathlineto{\pgfqpoint{1.125752in}{1.364467in}}%
\pgfpathlineto{\pgfqpoint{1.084375in}{1.364769in}}%
\pgfpathlineto{\pgfqpoint{1.042993in}{1.366427in}}%
\pgfpathlineto{\pgfqpoint{1.001616in}{1.341197in}}%
\pgfpathclose%
\pgfusepath{fill}%
\end{pgfscope}%
\begin{pgfscope}%
\pgfpathrectangle{\pgfqpoint{0.781402in}{0.773588in}}{\pgfqpoint{4.844695in}{5.415119in}}%
\pgfusepath{clip}%
\pgfsetbuttcap%
\pgfsetroundjoin%
\definecolor{currentfill}{rgb}{0.839216,0.152941,0.156863}%
\pgfsetfillcolor{currentfill}%
\pgfsetlinewidth{0.000000pt}%
\definecolor{currentstroke}{rgb}{0.000000,0.000000,0.000000}%
\pgfsetstrokecolor{currentstroke}%
\pgfsetdash{}{0pt}%
\pgfpathmoveto{\pgfqpoint{1.001616in}{1.866345in}}%
\pgfpathlineto{\pgfqpoint{1.001616in}{1.341197in}}%
\pgfpathlineto{\pgfqpoint{1.042993in}{1.366427in}}%
\pgfpathlineto{\pgfqpoint{1.084375in}{1.364769in}}%
\pgfpathlineto{\pgfqpoint{1.125752in}{1.364467in}}%
\pgfpathlineto{\pgfqpoint{1.167091in}{1.360322in}}%
\pgfpathlineto{\pgfqpoint{1.208458in}{1.362371in}}%
\pgfpathlineto{\pgfqpoint{1.249856in}{1.360154in}}%
\pgfpathlineto{\pgfqpoint{1.291280in}{1.364344in}}%
\pgfpathlineto{\pgfqpoint{1.332721in}{1.362715in}}%
\pgfpathlineto{\pgfqpoint{1.374186in}{1.360805in}}%
\pgfpathlineto{\pgfqpoint{1.415629in}{1.361473in}}%
\pgfpathlineto{\pgfqpoint{1.457070in}{1.362567in}}%
\pgfpathlineto{\pgfqpoint{1.498540in}{1.364751in}}%
\pgfpathlineto{\pgfqpoint{1.539972in}{1.364675in}}%
\pgfpathlineto{\pgfqpoint{1.581396in}{1.362069in}}%
\pgfpathlineto{\pgfqpoint{1.622829in}{1.364053in}}%
\pgfpathlineto{\pgfqpoint{1.664297in}{1.363974in}}%
\pgfpathlineto{\pgfqpoint{1.705741in}{1.363465in}}%
\pgfpathlineto{\pgfqpoint{1.747243in}{1.363597in}}%
\pgfpathlineto{\pgfqpoint{1.795378in}{1.311311in}}%
\pgfpathlineto{\pgfqpoint{1.849942in}{1.317586in}}%
\pgfpathlineto{\pgfqpoint{1.905453in}{1.314036in}}%
\pgfpathlineto{\pgfqpoint{1.963640in}{1.309627in}}%
\pgfpathlineto{\pgfqpoint{2.021509in}{1.313651in}}%
\pgfpathlineto{\pgfqpoint{2.078568in}{1.312392in}}%
\pgfpathlineto{\pgfqpoint{2.134427in}{1.316378in}}%
\pgfpathlineto{\pgfqpoint{2.188824in}{1.314691in}}%
\pgfpathlineto{\pgfqpoint{2.244502in}{1.315276in}}%
\pgfpathlineto{\pgfqpoint{2.300442in}{1.315084in}}%
\pgfpathlineto{\pgfqpoint{2.355457in}{1.314666in}}%
\pgfpathlineto{\pgfqpoint{2.411446in}{1.312654in}}%
\pgfpathlineto{\pgfqpoint{2.469564in}{1.313535in}}%
\pgfpathlineto{\pgfqpoint{2.529907in}{1.314789in}}%
\pgfpathlineto{\pgfqpoint{2.589001in}{1.315048in}}%
\pgfpathlineto{\pgfqpoint{2.646993in}{1.317405in}}%
\pgfpathlineto{\pgfqpoint{2.704011in}{1.318014in}}%
\pgfpathlineto{\pgfqpoint{2.760229in}{1.315429in}}%
\pgfpathlineto{\pgfqpoint{2.815997in}{1.314994in}}%
\pgfpathlineto{\pgfqpoint{2.869949in}{1.322957in}}%
\pgfpathlineto{\pgfqpoint{2.921979in}{1.325477in}}%
\pgfpathlineto{\pgfqpoint{2.972423in}{1.324644in}}%
\pgfpathlineto{\pgfqpoint{3.021960in}{1.329435in}}%
\pgfpathlineto{\pgfqpoint{3.070408in}{1.329105in}}%
\pgfpathlineto{\pgfqpoint{3.117833in}{1.328486in}}%
\pgfpathlineto{\pgfqpoint{3.164694in}{1.330200in}}%
\pgfpathlineto{\pgfqpoint{3.210786in}{1.333897in}}%
\pgfpathlineto{\pgfqpoint{3.255737in}{1.343240in}}%
\pgfpathlineto{\pgfqpoint{3.297077in}{1.339133in}}%
\pgfpathlineto{\pgfqpoint{3.337163in}{1.362396in}}%
\pgfpathlineto{\pgfqpoint{3.376505in}{1.356972in}}%
\pgfpathlineto{\pgfqpoint{3.415888in}{1.362416in}}%
\pgfpathlineto{\pgfqpoint{3.455225in}{1.359787in}}%
\pgfpathlineto{\pgfqpoint{3.494517in}{1.360712in}}%
\pgfpathlineto{\pgfqpoint{3.533833in}{1.359435in}}%
\pgfpathlineto{\pgfqpoint{3.573169in}{1.360712in}}%
\pgfpathlineto{\pgfqpoint{3.612456in}{1.360497in}}%
\pgfpathlineto{\pgfqpoint{3.651783in}{1.360511in}}%
\pgfpathlineto{\pgfqpoint{3.691109in}{1.361860in}}%
\pgfpathlineto{\pgfqpoint{3.730402in}{1.362032in}}%
\pgfpathlineto{\pgfqpoint{3.769397in}{1.762417in}}%
\pgfpathlineto{\pgfqpoint{3.806671in}{2.093222in}}%
\pgfpathlineto{\pgfqpoint{3.843637in}{2.091515in}}%
\pgfpathlineto{\pgfqpoint{3.880608in}{2.089291in}}%
\pgfpathlineto{\pgfqpoint{3.917591in}{2.091743in}}%
\pgfpathlineto{\pgfqpoint{3.954566in}{2.090672in}}%
\pgfpathlineto{\pgfqpoint{3.991588in}{2.089416in}}%
\pgfpathlineto{\pgfqpoint{4.028745in}{2.076258in}}%
\pgfpathlineto{\pgfqpoint{4.065634in}{2.091875in}}%
\pgfpathlineto{\pgfqpoint{4.101899in}{2.133556in}}%
\pgfpathlineto{\pgfqpoint{4.138055in}{2.191613in}}%
\pgfpathlineto{\pgfqpoint{4.173852in}{2.257032in}}%
\pgfpathlineto{\pgfqpoint{4.209218in}{2.328778in}}%
\pgfpathlineto{\pgfqpoint{4.244034in}{2.442066in}}%
\pgfpathlineto{\pgfqpoint{4.278250in}{2.544748in}}%
\pgfpathlineto{\pgfqpoint{4.311749in}{2.668023in}}%
\pgfpathlineto{\pgfqpoint{4.344452in}{2.818283in}}%
\pgfpathlineto{\pgfqpoint{4.376239in}{3.014489in}}%
\pgfpathlineto{\pgfqpoint{4.407030in}{3.201960in}}%
\pgfpathlineto{\pgfqpoint{4.436765in}{3.461934in}}%
\pgfpathlineto{\pgfqpoint{4.465281in}{3.731008in}}%
\pgfpathlineto{\pgfqpoint{4.492692in}{4.073421in}}%
\pgfpathlineto{\pgfqpoint{4.518427in}{4.520896in}}%
\pgfpathlineto{\pgfqpoint{4.543529in}{4.595634in}}%
\pgfpathlineto{\pgfqpoint{4.568473in}{4.609645in}}%
\pgfpathlineto{\pgfqpoint{4.593377in}{4.637062in}}%
\pgfpathlineto{\pgfqpoint{4.618263in}{4.636646in}}%
\pgfpathlineto{\pgfqpoint{4.643083in}{4.649944in}}%
\pgfpathlineto{\pgfqpoint{4.667884in}{4.653380in}}%
\pgfpathlineto{\pgfqpoint{4.692637in}{4.655583in}}%
\pgfpathlineto{\pgfqpoint{4.717447in}{4.664977in}}%
\pgfpathlineto{\pgfqpoint{4.742205in}{4.656381in}}%
\pgfpathlineto{\pgfqpoint{4.766979in}{4.665676in}}%
\pgfpathlineto{\pgfqpoint{4.791764in}{4.660915in}}%
\pgfpathlineto{\pgfqpoint{4.816503in}{4.673521in}}%
\pgfpathlineto{\pgfqpoint{4.846540in}{4.661995in}}%
\pgfpathlineto{\pgfqpoint{4.883926in}{4.668401in}}%
\pgfpathlineto{\pgfqpoint{4.921196in}{4.668093in}}%
\pgfpathlineto{\pgfqpoint{4.958462in}{4.670386in}}%
\pgfpathlineto{\pgfqpoint{4.995714in}{4.670339in}}%
\pgfpathlineto{\pgfqpoint{5.033005in}{4.670555in}}%
\pgfpathlineto{\pgfqpoint{5.070304in}{4.669476in}}%
\pgfpathlineto{\pgfqpoint{5.107610in}{4.669076in}}%
\pgfpathlineto{\pgfqpoint{5.144914in}{4.669325in}}%
\pgfpathlineto{\pgfqpoint{5.182186in}{4.670181in}}%
\pgfpathlineto{\pgfqpoint{5.219442in}{4.669695in}}%
\pgfpathlineto{\pgfqpoint{5.256693in}{4.667846in}}%
\pgfpathlineto{\pgfqpoint{5.293977in}{4.671707in}}%
\pgfpathlineto{\pgfqpoint{5.331273in}{4.671324in}}%
\pgfpathlineto{\pgfqpoint{5.368577in}{4.667255in}}%
\pgfpathlineto{\pgfqpoint{5.405885in}{4.664590in}}%
\pgfpathlineto{\pgfqpoint{5.405885in}{5.907459in}}%
\pgfpathlineto{\pgfqpoint{5.405885in}{5.907459in}}%
\pgfpathlineto{\pgfqpoint{5.368577in}{5.919721in}}%
\pgfpathlineto{\pgfqpoint{5.331273in}{5.925136in}}%
\pgfpathlineto{\pgfqpoint{5.293977in}{5.924980in}}%
\pgfpathlineto{\pgfqpoint{5.256693in}{5.924291in}}%
\pgfpathlineto{\pgfqpoint{5.219442in}{5.923400in}}%
\pgfpathlineto{\pgfqpoint{5.182186in}{5.923092in}}%
\pgfpathlineto{\pgfqpoint{5.144914in}{5.918788in}}%
\pgfpathlineto{\pgfqpoint{5.107610in}{5.921473in}}%
\pgfpathlineto{\pgfqpoint{5.070304in}{5.923674in}}%
\pgfpathlineto{\pgfqpoint{5.033005in}{5.924518in}}%
\pgfpathlineto{\pgfqpoint{4.995714in}{5.925125in}}%
\pgfpathlineto{\pgfqpoint{4.958462in}{5.923366in}}%
\pgfpathlineto{\pgfqpoint{4.921196in}{5.920973in}}%
\pgfpathlineto{\pgfqpoint{4.883926in}{5.921248in}}%
\pgfpathlineto{\pgfqpoint{4.846540in}{5.906757in}}%
\pgfpathlineto{\pgfqpoint{4.816503in}{5.930845in}}%
\pgfpathlineto{\pgfqpoint{4.791764in}{5.909307in}}%
\pgfpathlineto{\pgfqpoint{4.766979in}{5.925260in}}%
\pgfpathlineto{\pgfqpoint{4.742205in}{5.908598in}}%
\pgfpathlineto{\pgfqpoint{4.717447in}{5.920974in}}%
\pgfpathlineto{\pgfqpoint{4.692637in}{5.908731in}}%
\pgfpathlineto{\pgfqpoint{4.667884in}{5.906386in}}%
\pgfpathlineto{\pgfqpoint{4.643083in}{5.908976in}}%
\pgfpathlineto{\pgfqpoint{4.618263in}{5.885322in}}%
\pgfpathlineto{\pgfqpoint{4.593377in}{5.892171in}}%
\pgfpathlineto{\pgfqpoint{4.568473in}{5.862978in}}%
\pgfpathlineto{\pgfqpoint{4.543529in}{5.851222in}}%
\pgfpathlineto{\pgfqpoint{4.518427in}{5.771935in}}%
\pgfpathlineto{\pgfqpoint{4.492692in}{5.331064in}}%
\pgfpathlineto{\pgfqpoint{4.465281in}{4.995191in}}%
\pgfpathlineto{\pgfqpoint{4.436765in}{4.721643in}}%
\pgfpathlineto{\pgfqpoint{4.407030in}{4.462792in}}%
\pgfpathlineto{\pgfqpoint{4.376239in}{4.281051in}}%
\pgfpathlineto{\pgfqpoint{4.344452in}{4.080060in}}%
\pgfpathlineto{\pgfqpoint{4.311749in}{3.929436in}}%
\pgfpathlineto{\pgfqpoint{4.278250in}{3.807579in}}%
\pgfpathlineto{\pgfqpoint{4.244034in}{3.705878in}}%
\pgfpathlineto{\pgfqpoint{4.209218in}{3.595131in}}%
\pgfpathlineto{\pgfqpoint{4.173852in}{3.521340in}}%
\pgfpathlineto{\pgfqpoint{4.138055in}{3.453135in}}%
\pgfpathlineto{\pgfqpoint{4.101899in}{3.395896in}}%
\pgfpathlineto{\pgfqpoint{4.065634in}{3.356544in}}%
\pgfpathlineto{\pgfqpoint{4.028745in}{3.337413in}}%
\pgfpathlineto{\pgfqpoint{3.991588in}{3.348610in}}%
\pgfpathlineto{\pgfqpoint{3.954566in}{3.352087in}}%
\pgfpathlineto{\pgfqpoint{3.917591in}{3.352081in}}%
\pgfpathlineto{\pgfqpoint{3.880608in}{3.352768in}}%
\pgfpathlineto{\pgfqpoint{3.843637in}{3.356373in}}%
\pgfpathlineto{\pgfqpoint{3.806671in}{3.359561in}}%
\pgfpathlineto{\pgfqpoint{3.769397in}{3.016708in}}%
\pgfpathlineto{\pgfqpoint{3.730402in}{2.616404in}}%
\pgfpathlineto{\pgfqpoint{3.691109in}{2.615119in}}%
\pgfpathlineto{\pgfqpoint{3.651783in}{2.612524in}}%
\pgfpathlineto{\pgfqpoint{3.612456in}{2.614373in}}%
\pgfpathlineto{\pgfqpoint{3.573169in}{2.614163in}}%
\pgfpathlineto{\pgfqpoint{3.533833in}{2.611885in}}%
\pgfpathlineto{\pgfqpoint{3.494517in}{2.616014in}}%
\pgfpathlineto{\pgfqpoint{3.455225in}{2.613496in}}%
\pgfpathlineto{\pgfqpoint{3.415888in}{2.615197in}}%
\pgfpathlineto{\pgfqpoint{3.376505in}{2.610689in}}%
\pgfpathlineto{\pgfqpoint{3.337163in}{2.615990in}}%
\pgfpathlineto{\pgfqpoint{3.297077in}{2.533652in}}%
\pgfpathlineto{\pgfqpoint{3.255737in}{2.179896in}}%
\pgfpathlineto{\pgfqpoint{3.210786in}{1.815662in}}%
\pgfpathlineto{\pgfqpoint{3.164694in}{1.802395in}}%
\pgfpathlineto{\pgfqpoint{3.117833in}{1.784407in}}%
\pgfpathlineto{\pgfqpoint{3.070408in}{1.768742in}}%
\pgfpathlineto{\pgfqpoint{3.021960in}{1.749076in}}%
\pgfpathlineto{\pgfqpoint{2.972423in}{1.725347in}}%
\pgfpathlineto{\pgfqpoint{2.921979in}{1.705542in}}%
\pgfpathlineto{\pgfqpoint{2.869949in}{1.674784in}}%
\pgfpathlineto{\pgfqpoint{2.815997in}{1.641840in}}%
\pgfpathlineto{\pgfqpoint{2.760229in}{1.618898in}}%
\pgfpathlineto{\pgfqpoint{2.704011in}{1.622280in}}%
\pgfpathlineto{\pgfqpoint{2.646993in}{1.600676in}}%
\pgfpathlineto{\pgfqpoint{2.589001in}{1.588521in}}%
\pgfpathlineto{\pgfqpoint{2.529907in}{1.567887in}}%
\pgfpathlineto{\pgfqpoint{2.469564in}{1.574774in}}%
\pgfpathlineto{\pgfqpoint{2.411446in}{1.613791in}}%
\pgfpathlineto{\pgfqpoint{2.355457in}{1.641934in}}%
\pgfpathlineto{\pgfqpoint{2.300442in}{1.633385in}}%
\pgfpathlineto{\pgfqpoint{2.244502in}{1.624399in}}%
\pgfpathlineto{\pgfqpoint{2.188824in}{1.638642in}}%
\pgfpathlineto{\pgfqpoint{2.134427in}{1.651335in}}%
\pgfpathlineto{\pgfqpoint{2.078568in}{1.603821in}}%
\pgfpathlineto{\pgfqpoint{2.021509in}{1.616476in}}%
\pgfpathlineto{\pgfqpoint{1.963640in}{1.585604in}}%
\pgfpathlineto{\pgfqpoint{1.905453in}{1.598895in}}%
\pgfpathlineto{\pgfqpoint{1.849942in}{1.687110in}}%
\pgfpathlineto{\pgfqpoint{1.795378in}{1.640816in}}%
\pgfpathlineto{\pgfqpoint{1.747243in}{1.926389in}}%
\pgfpathlineto{\pgfqpoint{1.705741in}{1.927812in}}%
\pgfpathlineto{\pgfqpoint{1.664297in}{1.925967in}}%
\pgfpathlineto{\pgfqpoint{1.622829in}{1.926168in}}%
\pgfpathlineto{\pgfqpoint{1.581396in}{1.922523in}}%
\pgfpathlineto{\pgfqpoint{1.539972in}{1.926730in}}%
\pgfpathlineto{\pgfqpoint{1.498540in}{1.927486in}}%
\pgfpathlineto{\pgfqpoint{1.457070in}{1.925064in}}%
\pgfpathlineto{\pgfqpoint{1.415629in}{1.924122in}}%
\pgfpathlineto{\pgfqpoint{1.374186in}{1.922103in}}%
\pgfpathlineto{\pgfqpoint{1.332721in}{1.925826in}}%
\pgfpathlineto{\pgfqpoint{1.291280in}{1.926536in}}%
\pgfpathlineto{\pgfqpoint{1.249856in}{1.923914in}}%
\pgfpathlineto{\pgfqpoint{1.208458in}{1.925928in}}%
\pgfpathlineto{\pgfqpoint{1.167091in}{1.924419in}}%
\pgfpathlineto{\pgfqpoint{1.125752in}{1.927518in}}%
\pgfpathlineto{\pgfqpoint{1.084375in}{1.924975in}}%
\pgfpathlineto{\pgfqpoint{1.042993in}{1.930978in}}%
\pgfpathlineto{\pgfqpoint{1.001616in}{1.866345in}}%
\pgfpathclose%
\pgfusepath{fill}%
\end{pgfscope}%
\begin{pgfscope}%
\pgfpathrectangle{\pgfqpoint{0.781402in}{0.773588in}}{\pgfqpoint{4.844695in}{5.415119in}}%
\pgfusepath{clip}%
\pgfsetbuttcap%
\pgfsetroundjoin%
\definecolor{currentfill}{rgb}{0.580392,0.403922,0.741176}%
\pgfsetfillcolor{currentfill}%
\pgfsetlinewidth{0.000000pt}%
\definecolor{currentstroke}{rgb}{0.000000,0.000000,0.000000}%
\pgfsetstrokecolor{currentstroke}%
\pgfsetdash{}{0pt}%
\pgfpathmoveto{\pgfqpoint{1.001616in}{2.423881in}}%
\pgfpathlineto{\pgfqpoint{1.001616in}{1.866345in}}%
\pgfpathlineto{\pgfqpoint{1.042993in}{1.930978in}}%
\pgfpathlineto{\pgfqpoint{1.084375in}{1.924975in}}%
\pgfpathlineto{\pgfqpoint{1.125752in}{1.927518in}}%
\pgfpathlineto{\pgfqpoint{1.167091in}{1.924419in}}%
\pgfpathlineto{\pgfqpoint{1.208458in}{1.925928in}}%
\pgfpathlineto{\pgfqpoint{1.249856in}{1.923914in}}%
\pgfpathlineto{\pgfqpoint{1.291280in}{1.926536in}}%
\pgfpathlineto{\pgfqpoint{1.332721in}{1.925826in}}%
\pgfpathlineto{\pgfqpoint{1.374186in}{1.922103in}}%
\pgfpathlineto{\pgfqpoint{1.415629in}{1.924122in}}%
\pgfpathlineto{\pgfqpoint{1.457070in}{1.925064in}}%
\pgfpathlineto{\pgfqpoint{1.498540in}{1.927486in}}%
\pgfpathlineto{\pgfqpoint{1.539972in}{1.926730in}}%
\pgfpathlineto{\pgfqpoint{1.581396in}{1.922523in}}%
\pgfpathlineto{\pgfqpoint{1.622829in}{1.926168in}}%
\pgfpathlineto{\pgfqpoint{1.664297in}{1.925967in}}%
\pgfpathlineto{\pgfqpoint{1.705741in}{1.927812in}}%
\pgfpathlineto{\pgfqpoint{1.747243in}{1.926389in}}%
\pgfpathlineto{\pgfqpoint{1.795378in}{1.640816in}}%
\pgfpathlineto{\pgfqpoint{1.849942in}{1.687110in}}%
\pgfpathlineto{\pgfqpoint{1.905453in}{1.598895in}}%
\pgfpathlineto{\pgfqpoint{1.963640in}{1.585604in}}%
\pgfpathlineto{\pgfqpoint{2.021509in}{1.616476in}}%
\pgfpathlineto{\pgfqpoint{2.078568in}{1.603821in}}%
\pgfpathlineto{\pgfqpoint{2.134427in}{1.651335in}}%
\pgfpathlineto{\pgfqpoint{2.188824in}{1.638642in}}%
\pgfpathlineto{\pgfqpoint{2.244502in}{1.624399in}}%
\pgfpathlineto{\pgfqpoint{2.300442in}{1.633385in}}%
\pgfpathlineto{\pgfqpoint{2.355457in}{1.641934in}}%
\pgfpathlineto{\pgfqpoint{2.411446in}{1.613791in}}%
\pgfpathlineto{\pgfqpoint{2.469564in}{1.574774in}}%
\pgfpathlineto{\pgfqpoint{2.529907in}{1.567887in}}%
\pgfpathlineto{\pgfqpoint{2.589001in}{1.588521in}}%
\pgfpathlineto{\pgfqpoint{2.646993in}{1.600676in}}%
\pgfpathlineto{\pgfqpoint{2.704011in}{1.622280in}}%
\pgfpathlineto{\pgfqpoint{2.760229in}{1.618898in}}%
\pgfpathlineto{\pgfqpoint{2.815997in}{1.641840in}}%
\pgfpathlineto{\pgfqpoint{2.869949in}{1.674784in}}%
\pgfpathlineto{\pgfqpoint{2.921979in}{1.705542in}}%
\pgfpathlineto{\pgfqpoint{2.972423in}{1.725347in}}%
\pgfpathlineto{\pgfqpoint{3.021960in}{1.749076in}}%
\pgfpathlineto{\pgfqpoint{3.070408in}{1.768742in}}%
\pgfpathlineto{\pgfqpoint{3.117833in}{1.784407in}}%
\pgfpathlineto{\pgfqpoint{3.164694in}{1.802395in}}%
\pgfpathlineto{\pgfqpoint{3.210786in}{1.815662in}}%
\pgfpathlineto{\pgfqpoint{3.255737in}{2.179896in}}%
\pgfpathlineto{\pgfqpoint{3.297077in}{2.533652in}}%
\pgfpathlineto{\pgfqpoint{3.337163in}{2.615990in}}%
\pgfpathlineto{\pgfqpoint{3.376505in}{2.610689in}}%
\pgfpathlineto{\pgfqpoint{3.415888in}{2.615197in}}%
\pgfpathlineto{\pgfqpoint{3.455225in}{2.613496in}}%
\pgfpathlineto{\pgfqpoint{3.494517in}{2.616014in}}%
\pgfpathlineto{\pgfqpoint{3.533833in}{2.611885in}}%
\pgfpathlineto{\pgfqpoint{3.573169in}{2.614163in}}%
\pgfpathlineto{\pgfqpoint{3.612456in}{2.614373in}}%
\pgfpathlineto{\pgfqpoint{3.651783in}{2.612524in}}%
\pgfpathlineto{\pgfqpoint{3.691109in}{2.615119in}}%
\pgfpathlineto{\pgfqpoint{3.730402in}{2.616404in}}%
\pgfpathlineto{\pgfqpoint{3.769397in}{3.016708in}}%
\pgfpathlineto{\pgfqpoint{3.806671in}{3.359561in}}%
\pgfpathlineto{\pgfqpoint{3.843637in}{3.356373in}}%
\pgfpathlineto{\pgfqpoint{3.880608in}{3.352768in}}%
\pgfpathlineto{\pgfqpoint{3.917591in}{3.352081in}}%
\pgfpathlineto{\pgfqpoint{3.954566in}{3.352087in}}%
\pgfpathlineto{\pgfqpoint{3.991588in}{3.348610in}}%
\pgfpathlineto{\pgfqpoint{4.028745in}{3.337413in}}%
\pgfpathlineto{\pgfqpoint{4.065634in}{3.356544in}}%
\pgfpathlineto{\pgfqpoint{4.101899in}{3.395896in}}%
\pgfpathlineto{\pgfqpoint{4.138055in}{3.453135in}}%
\pgfpathlineto{\pgfqpoint{4.173852in}{3.521340in}}%
\pgfpathlineto{\pgfqpoint{4.209218in}{3.595131in}}%
\pgfpathlineto{\pgfqpoint{4.244034in}{3.705878in}}%
\pgfpathlineto{\pgfqpoint{4.278250in}{3.807579in}}%
\pgfpathlineto{\pgfqpoint{4.311749in}{3.929436in}}%
\pgfpathlineto{\pgfqpoint{4.344452in}{4.080060in}}%
\pgfpathlineto{\pgfqpoint{4.376239in}{4.281051in}}%
\pgfpathlineto{\pgfqpoint{4.407030in}{4.462792in}}%
\pgfpathlineto{\pgfqpoint{4.436765in}{4.721643in}}%
\pgfpathlineto{\pgfqpoint{4.465281in}{4.995191in}}%
\pgfpathlineto{\pgfqpoint{4.492692in}{5.331064in}}%
\pgfpathlineto{\pgfqpoint{4.518427in}{5.771935in}}%
\pgfpathlineto{\pgfqpoint{4.543529in}{5.851222in}}%
\pgfpathlineto{\pgfqpoint{4.568473in}{5.862978in}}%
\pgfpathlineto{\pgfqpoint{4.593377in}{5.892171in}}%
\pgfpathlineto{\pgfqpoint{4.618263in}{5.885322in}}%
\pgfpathlineto{\pgfqpoint{4.643083in}{5.908976in}}%
\pgfpathlineto{\pgfqpoint{4.667884in}{5.906386in}}%
\pgfpathlineto{\pgfqpoint{4.692637in}{5.908731in}}%
\pgfpathlineto{\pgfqpoint{4.717447in}{5.920974in}}%
\pgfpathlineto{\pgfqpoint{4.742205in}{5.908598in}}%
\pgfpathlineto{\pgfqpoint{4.766979in}{5.925260in}}%
\pgfpathlineto{\pgfqpoint{4.791764in}{5.909307in}}%
\pgfpathlineto{\pgfqpoint{4.816503in}{5.930845in}}%
\pgfpathlineto{\pgfqpoint{4.846540in}{5.906757in}}%
\pgfpathlineto{\pgfqpoint{4.883926in}{5.921248in}}%
\pgfpathlineto{\pgfqpoint{4.921196in}{5.920973in}}%
\pgfpathlineto{\pgfqpoint{4.958462in}{5.923366in}}%
\pgfpathlineto{\pgfqpoint{4.995714in}{5.925125in}}%
\pgfpathlineto{\pgfqpoint{5.033005in}{5.924518in}}%
\pgfpathlineto{\pgfqpoint{5.070304in}{5.923674in}}%
\pgfpathlineto{\pgfqpoint{5.107610in}{5.921473in}}%
\pgfpathlineto{\pgfqpoint{5.144914in}{5.918788in}}%
\pgfpathlineto{\pgfqpoint{5.182186in}{5.923092in}}%
\pgfpathlineto{\pgfqpoint{5.219442in}{5.923400in}}%
\pgfpathlineto{\pgfqpoint{5.256693in}{5.924291in}}%
\pgfpathlineto{\pgfqpoint{5.293977in}{5.924980in}}%
\pgfpathlineto{\pgfqpoint{5.331273in}{5.925136in}}%
\pgfpathlineto{\pgfqpoint{5.368577in}{5.919721in}}%
\pgfpathlineto{\pgfqpoint{5.405885in}{5.907459in}}%
\pgfpathlineto{\pgfqpoint{5.405885in}{5.907459in}}%
\pgfpathlineto{\pgfqpoint{5.405885in}{5.907459in}}%
\pgfpathlineto{\pgfqpoint{5.368577in}{5.919721in}}%
\pgfpathlineto{\pgfqpoint{5.331273in}{5.925136in}}%
\pgfpathlineto{\pgfqpoint{5.293977in}{5.924980in}}%
\pgfpathlineto{\pgfqpoint{5.256693in}{5.924291in}}%
\pgfpathlineto{\pgfqpoint{5.219442in}{5.923400in}}%
\pgfpathlineto{\pgfqpoint{5.182186in}{5.923092in}}%
\pgfpathlineto{\pgfqpoint{5.144914in}{5.918788in}}%
\pgfpathlineto{\pgfqpoint{5.107610in}{5.921473in}}%
\pgfpathlineto{\pgfqpoint{5.070304in}{5.923674in}}%
\pgfpathlineto{\pgfqpoint{5.033005in}{5.924518in}}%
\pgfpathlineto{\pgfqpoint{4.995714in}{5.925125in}}%
\pgfpathlineto{\pgfqpoint{4.958462in}{5.923366in}}%
\pgfpathlineto{\pgfqpoint{4.921196in}{5.920973in}}%
\pgfpathlineto{\pgfqpoint{4.883926in}{5.921248in}}%
\pgfpathlineto{\pgfqpoint{4.846540in}{5.906757in}}%
\pgfpathlineto{\pgfqpoint{4.816503in}{5.930845in}}%
\pgfpathlineto{\pgfqpoint{4.791764in}{5.909307in}}%
\pgfpathlineto{\pgfqpoint{4.766979in}{5.925260in}}%
\pgfpathlineto{\pgfqpoint{4.742205in}{5.908598in}}%
\pgfpathlineto{\pgfqpoint{4.717447in}{5.920974in}}%
\pgfpathlineto{\pgfqpoint{4.692637in}{5.908731in}}%
\pgfpathlineto{\pgfqpoint{4.667884in}{5.906386in}}%
\pgfpathlineto{\pgfqpoint{4.643083in}{5.908976in}}%
\pgfpathlineto{\pgfqpoint{4.618263in}{5.885322in}}%
\pgfpathlineto{\pgfqpoint{4.593377in}{5.892171in}}%
\pgfpathlineto{\pgfqpoint{4.568473in}{5.862978in}}%
\pgfpathlineto{\pgfqpoint{4.543529in}{5.851222in}}%
\pgfpathlineto{\pgfqpoint{4.518427in}{5.771935in}}%
\pgfpathlineto{\pgfqpoint{4.492692in}{5.331064in}}%
\pgfpathlineto{\pgfqpoint{4.465281in}{4.995191in}}%
\pgfpathlineto{\pgfqpoint{4.436765in}{4.721643in}}%
\pgfpathlineto{\pgfqpoint{4.407030in}{4.462792in}}%
\pgfpathlineto{\pgfqpoint{4.376239in}{4.281051in}}%
\pgfpathlineto{\pgfqpoint{4.344452in}{4.080060in}}%
\pgfpathlineto{\pgfqpoint{4.311749in}{3.929436in}}%
\pgfpathlineto{\pgfqpoint{4.278250in}{3.807579in}}%
\pgfpathlineto{\pgfqpoint{4.244034in}{3.705878in}}%
\pgfpathlineto{\pgfqpoint{4.209218in}{3.595131in}}%
\pgfpathlineto{\pgfqpoint{4.173852in}{3.521340in}}%
\pgfpathlineto{\pgfqpoint{4.138055in}{3.453135in}}%
\pgfpathlineto{\pgfqpoint{4.101899in}{3.395896in}}%
\pgfpathlineto{\pgfqpoint{4.065634in}{3.356544in}}%
\pgfpathlineto{\pgfqpoint{4.028745in}{3.337413in}}%
\pgfpathlineto{\pgfqpoint{3.991588in}{3.348610in}}%
\pgfpathlineto{\pgfqpoint{3.954566in}{3.352087in}}%
\pgfpathlineto{\pgfqpoint{3.917591in}{3.352081in}}%
\pgfpathlineto{\pgfqpoint{3.880608in}{3.352768in}}%
\pgfpathlineto{\pgfqpoint{3.843637in}{3.356373in}}%
\pgfpathlineto{\pgfqpoint{3.806671in}{3.359561in}}%
\pgfpathlineto{\pgfqpoint{3.769397in}{3.279789in}}%
\pgfpathlineto{\pgfqpoint{3.730402in}{3.201512in}}%
\pgfpathlineto{\pgfqpoint{3.691109in}{3.200491in}}%
\pgfpathlineto{\pgfqpoint{3.651783in}{3.198256in}}%
\pgfpathlineto{\pgfqpoint{3.612456in}{3.201385in}}%
\pgfpathlineto{\pgfqpoint{3.573169in}{3.201452in}}%
\pgfpathlineto{\pgfqpoint{3.533833in}{3.198066in}}%
\pgfpathlineto{\pgfqpoint{3.494517in}{3.204856in}}%
\pgfpathlineto{\pgfqpoint{3.455225in}{3.202255in}}%
\pgfpathlineto{\pgfqpoint{3.415888in}{3.202131in}}%
\pgfpathlineto{\pgfqpoint{3.376505in}{3.194553in}}%
\pgfpathlineto{\pgfqpoint{3.337163in}{3.200877in}}%
\pgfpathlineto{\pgfqpoint{3.297077in}{3.099470in}}%
\pgfpathlineto{\pgfqpoint{3.255737in}{2.746544in}}%
\pgfpathlineto{\pgfqpoint{3.210786in}{2.373235in}}%
\pgfpathlineto{\pgfqpoint{3.164694in}{2.357668in}}%
\pgfpathlineto{\pgfqpoint{3.117833in}{2.339936in}}%
\pgfpathlineto{\pgfqpoint{3.070408in}{2.323635in}}%
\pgfpathlineto{\pgfqpoint{3.021960in}{2.301234in}}%
\pgfpathlineto{\pgfqpoint{2.972423in}{2.276296in}}%
\pgfpathlineto{\pgfqpoint{2.921979in}{2.259215in}}%
\pgfpathlineto{\pgfqpoint{2.869949in}{2.223093in}}%
\pgfpathlineto{\pgfqpoint{2.815997in}{2.184370in}}%
\pgfpathlineto{\pgfqpoint{2.760229in}{2.162658in}}%
\pgfpathlineto{\pgfqpoint{2.704011in}{2.165773in}}%
\pgfpathlineto{\pgfqpoint{2.646993in}{2.144592in}}%
\pgfpathlineto{\pgfqpoint{2.589001in}{2.132370in}}%
\pgfpathlineto{\pgfqpoint{2.529907in}{2.108492in}}%
\pgfpathlineto{\pgfqpoint{2.469564in}{2.116072in}}%
\pgfpathlineto{\pgfqpoint{2.411446in}{2.153662in}}%
\pgfpathlineto{\pgfqpoint{2.355457in}{2.185053in}}%
\pgfpathlineto{\pgfqpoint{2.300442in}{2.176320in}}%
\pgfpathlineto{\pgfqpoint{2.244502in}{2.166321in}}%
\pgfpathlineto{\pgfqpoint{2.188824in}{2.182296in}}%
\pgfpathlineto{\pgfqpoint{2.134427in}{2.196483in}}%
\pgfpathlineto{\pgfqpoint{2.078568in}{2.140756in}}%
\pgfpathlineto{\pgfqpoint{2.021509in}{2.156892in}}%
\pgfpathlineto{\pgfqpoint{1.963640in}{2.122721in}}%
\pgfpathlineto{\pgfqpoint{1.905453in}{2.136216in}}%
\pgfpathlineto{\pgfqpoint{1.849942in}{2.228684in}}%
\pgfpathlineto{\pgfqpoint{1.795378in}{2.179073in}}%
\pgfpathlineto{\pgfqpoint{1.747243in}{2.514716in}}%
\pgfpathlineto{\pgfqpoint{1.705741in}{2.516533in}}%
\pgfpathlineto{\pgfqpoint{1.664297in}{2.515981in}}%
\pgfpathlineto{\pgfqpoint{1.622829in}{2.516396in}}%
\pgfpathlineto{\pgfqpoint{1.581396in}{2.512910in}}%
\pgfpathlineto{\pgfqpoint{1.539972in}{2.514893in}}%
\pgfpathlineto{\pgfqpoint{1.498540in}{2.515524in}}%
\pgfpathlineto{\pgfqpoint{1.457070in}{2.515639in}}%
\pgfpathlineto{\pgfqpoint{1.415629in}{2.515241in}}%
\pgfpathlineto{\pgfqpoint{1.374186in}{2.513631in}}%
\pgfpathlineto{\pgfqpoint{1.332721in}{2.514045in}}%
\pgfpathlineto{\pgfqpoint{1.291280in}{2.515772in}}%
\pgfpathlineto{\pgfqpoint{1.249856in}{2.515562in}}%
\pgfpathlineto{\pgfqpoint{1.208458in}{2.517861in}}%
\pgfpathlineto{\pgfqpoint{1.167091in}{2.518390in}}%
\pgfpathlineto{\pgfqpoint{1.125752in}{2.518898in}}%
\pgfpathlineto{\pgfqpoint{1.084375in}{2.514891in}}%
\pgfpathlineto{\pgfqpoint{1.042993in}{2.519575in}}%
\pgfpathlineto{\pgfqpoint{1.001616in}{2.423881in}}%
\pgfpathclose%
\pgfusepath{fill}%
\end{pgfscope}%
\begin{pgfscope}%
\pgfpathrectangle{\pgfqpoint{0.781402in}{0.773588in}}{\pgfqpoint{4.844695in}{5.415119in}}%
\pgfusepath{clip}%
\pgfsetbuttcap%
\pgfsetroundjoin%
\definecolor{currentfill}{rgb}{0.549020,0.337255,0.294118}%
\pgfsetfillcolor{currentfill}%
\pgfsetlinewidth{0.000000pt}%
\definecolor{currentstroke}{rgb}{0.000000,0.000000,0.000000}%
\pgfsetstrokecolor{currentstroke}%
\pgfsetdash{}{0pt}%
\pgfpathmoveto{\pgfqpoint{1.001616in}{2.946160in}}%
\pgfpathlineto{\pgfqpoint{1.001616in}{2.423881in}}%
\pgfpathlineto{\pgfqpoint{1.042993in}{2.519575in}}%
\pgfpathlineto{\pgfqpoint{1.084375in}{2.514891in}}%
\pgfpathlineto{\pgfqpoint{1.125752in}{2.518898in}}%
\pgfpathlineto{\pgfqpoint{1.167091in}{2.518390in}}%
\pgfpathlineto{\pgfqpoint{1.208458in}{2.517861in}}%
\pgfpathlineto{\pgfqpoint{1.249856in}{2.515562in}}%
\pgfpathlineto{\pgfqpoint{1.291280in}{2.515772in}}%
\pgfpathlineto{\pgfqpoint{1.332721in}{2.514045in}}%
\pgfpathlineto{\pgfqpoint{1.374186in}{2.513631in}}%
\pgfpathlineto{\pgfqpoint{1.415629in}{2.515241in}}%
\pgfpathlineto{\pgfqpoint{1.457070in}{2.515639in}}%
\pgfpathlineto{\pgfqpoint{1.498540in}{2.515524in}}%
\pgfpathlineto{\pgfqpoint{1.539972in}{2.514893in}}%
\pgfpathlineto{\pgfqpoint{1.581396in}{2.512910in}}%
\pgfpathlineto{\pgfqpoint{1.622829in}{2.516396in}}%
\pgfpathlineto{\pgfqpoint{1.664297in}{2.515981in}}%
\pgfpathlineto{\pgfqpoint{1.705741in}{2.516533in}}%
\pgfpathlineto{\pgfqpoint{1.747243in}{2.514716in}}%
\pgfpathlineto{\pgfqpoint{1.795378in}{2.179073in}}%
\pgfpathlineto{\pgfqpoint{1.849942in}{2.228684in}}%
\pgfpathlineto{\pgfqpoint{1.905453in}{2.136216in}}%
\pgfpathlineto{\pgfqpoint{1.963640in}{2.122721in}}%
\pgfpathlineto{\pgfqpoint{2.021509in}{2.156892in}}%
\pgfpathlineto{\pgfqpoint{2.078568in}{2.140756in}}%
\pgfpathlineto{\pgfqpoint{2.134427in}{2.196483in}}%
\pgfpathlineto{\pgfqpoint{2.188824in}{2.182296in}}%
\pgfpathlineto{\pgfqpoint{2.244502in}{2.166321in}}%
\pgfpathlineto{\pgfqpoint{2.300442in}{2.176320in}}%
\pgfpathlineto{\pgfqpoint{2.355457in}{2.185053in}}%
\pgfpathlineto{\pgfqpoint{2.411446in}{2.153662in}}%
\pgfpathlineto{\pgfqpoint{2.469564in}{2.116072in}}%
\pgfpathlineto{\pgfqpoint{2.529907in}{2.108492in}}%
\pgfpathlineto{\pgfqpoint{2.589001in}{2.132370in}}%
\pgfpathlineto{\pgfqpoint{2.646993in}{2.144592in}}%
\pgfpathlineto{\pgfqpoint{2.704011in}{2.165773in}}%
\pgfpathlineto{\pgfqpoint{2.760229in}{2.162658in}}%
\pgfpathlineto{\pgfqpoint{2.815997in}{2.184370in}}%
\pgfpathlineto{\pgfqpoint{2.869949in}{2.223093in}}%
\pgfpathlineto{\pgfqpoint{2.921979in}{2.259215in}}%
\pgfpathlineto{\pgfqpoint{2.972423in}{2.276296in}}%
\pgfpathlineto{\pgfqpoint{3.021960in}{2.301234in}}%
\pgfpathlineto{\pgfqpoint{3.070408in}{2.323635in}}%
\pgfpathlineto{\pgfqpoint{3.117833in}{2.339936in}}%
\pgfpathlineto{\pgfqpoint{3.164694in}{2.357668in}}%
\pgfpathlineto{\pgfqpoint{3.210786in}{2.373235in}}%
\pgfpathlineto{\pgfqpoint{3.255737in}{2.746544in}}%
\pgfpathlineto{\pgfqpoint{3.297077in}{3.099470in}}%
\pgfpathlineto{\pgfqpoint{3.337163in}{3.200877in}}%
\pgfpathlineto{\pgfqpoint{3.376505in}{3.194553in}}%
\pgfpathlineto{\pgfqpoint{3.415888in}{3.202131in}}%
\pgfpathlineto{\pgfqpoint{3.455225in}{3.202255in}}%
\pgfpathlineto{\pgfqpoint{3.494517in}{3.204856in}}%
\pgfpathlineto{\pgfqpoint{3.533833in}{3.198066in}}%
\pgfpathlineto{\pgfqpoint{3.573169in}{3.201452in}}%
\pgfpathlineto{\pgfqpoint{3.612456in}{3.201385in}}%
\pgfpathlineto{\pgfqpoint{3.651783in}{3.198256in}}%
\pgfpathlineto{\pgfqpoint{3.691109in}{3.200491in}}%
\pgfpathlineto{\pgfqpoint{3.730402in}{3.201512in}}%
\pgfpathlineto{\pgfqpoint{3.769397in}{3.279789in}}%
\pgfpathlineto{\pgfqpoint{3.806671in}{3.359561in}}%
\pgfpathlineto{\pgfqpoint{3.843637in}{3.356373in}}%
\pgfpathlineto{\pgfqpoint{3.880608in}{3.352768in}}%
\pgfpathlineto{\pgfqpoint{3.917591in}{3.352081in}}%
\pgfpathlineto{\pgfqpoint{3.954566in}{3.352087in}}%
\pgfpathlineto{\pgfqpoint{3.991588in}{3.348610in}}%
\pgfpathlineto{\pgfqpoint{4.028745in}{3.337413in}}%
\pgfpathlineto{\pgfqpoint{4.065634in}{3.356544in}}%
\pgfpathlineto{\pgfqpoint{4.101899in}{3.395896in}}%
\pgfpathlineto{\pgfqpoint{4.138055in}{3.453135in}}%
\pgfpathlineto{\pgfqpoint{4.173852in}{3.521340in}}%
\pgfpathlineto{\pgfqpoint{4.209218in}{3.595131in}}%
\pgfpathlineto{\pgfqpoint{4.244034in}{3.705878in}}%
\pgfpathlineto{\pgfqpoint{4.278250in}{3.807579in}}%
\pgfpathlineto{\pgfqpoint{4.311749in}{3.929436in}}%
\pgfpathlineto{\pgfqpoint{4.344452in}{4.080060in}}%
\pgfpathlineto{\pgfqpoint{4.376239in}{4.281051in}}%
\pgfpathlineto{\pgfqpoint{4.407030in}{4.462792in}}%
\pgfpathlineto{\pgfqpoint{4.436765in}{4.721643in}}%
\pgfpathlineto{\pgfqpoint{4.465281in}{4.995191in}}%
\pgfpathlineto{\pgfqpoint{4.492692in}{5.331064in}}%
\pgfpathlineto{\pgfqpoint{4.518427in}{5.771935in}}%
\pgfpathlineto{\pgfqpoint{4.543529in}{5.851222in}}%
\pgfpathlineto{\pgfqpoint{4.568473in}{5.862978in}}%
\pgfpathlineto{\pgfqpoint{4.593377in}{5.892171in}}%
\pgfpathlineto{\pgfqpoint{4.618263in}{5.885322in}}%
\pgfpathlineto{\pgfqpoint{4.643083in}{5.908976in}}%
\pgfpathlineto{\pgfqpoint{4.667884in}{5.906386in}}%
\pgfpathlineto{\pgfqpoint{4.692637in}{5.908731in}}%
\pgfpathlineto{\pgfqpoint{4.717447in}{5.920974in}}%
\pgfpathlineto{\pgfqpoint{4.742205in}{5.908598in}}%
\pgfpathlineto{\pgfqpoint{4.766979in}{5.925260in}}%
\pgfpathlineto{\pgfqpoint{4.791764in}{5.909307in}}%
\pgfpathlineto{\pgfqpoint{4.816503in}{5.930845in}}%
\pgfpathlineto{\pgfqpoint{4.846540in}{5.906757in}}%
\pgfpathlineto{\pgfqpoint{4.883926in}{5.921248in}}%
\pgfpathlineto{\pgfqpoint{4.921196in}{5.920973in}}%
\pgfpathlineto{\pgfqpoint{4.958462in}{5.923366in}}%
\pgfpathlineto{\pgfqpoint{4.995714in}{5.925125in}}%
\pgfpathlineto{\pgfqpoint{5.033005in}{5.924518in}}%
\pgfpathlineto{\pgfqpoint{5.070304in}{5.923674in}}%
\pgfpathlineto{\pgfqpoint{5.107610in}{5.921473in}}%
\pgfpathlineto{\pgfqpoint{5.144914in}{5.918788in}}%
\pgfpathlineto{\pgfqpoint{5.182186in}{5.923092in}}%
\pgfpathlineto{\pgfqpoint{5.219442in}{5.923400in}}%
\pgfpathlineto{\pgfqpoint{5.256693in}{5.924291in}}%
\pgfpathlineto{\pgfqpoint{5.293977in}{5.924980in}}%
\pgfpathlineto{\pgfqpoint{5.331273in}{5.925136in}}%
\pgfpathlineto{\pgfqpoint{5.368577in}{5.919721in}}%
\pgfpathlineto{\pgfqpoint{5.405885in}{5.907459in}}%
\pgfpathlineto{\pgfqpoint{5.405885in}{5.907459in}}%
\pgfpathlineto{\pgfqpoint{5.405885in}{5.907459in}}%
\pgfpathlineto{\pgfqpoint{5.368577in}{5.919721in}}%
\pgfpathlineto{\pgfqpoint{5.331273in}{5.925136in}}%
\pgfpathlineto{\pgfqpoint{5.293977in}{5.924980in}}%
\pgfpathlineto{\pgfqpoint{5.256693in}{5.924291in}}%
\pgfpathlineto{\pgfqpoint{5.219442in}{5.923400in}}%
\pgfpathlineto{\pgfqpoint{5.182186in}{5.923092in}}%
\pgfpathlineto{\pgfqpoint{5.144914in}{5.918788in}}%
\pgfpathlineto{\pgfqpoint{5.107610in}{5.921473in}}%
\pgfpathlineto{\pgfqpoint{5.070304in}{5.923674in}}%
\pgfpathlineto{\pgfqpoint{5.033005in}{5.924518in}}%
\pgfpathlineto{\pgfqpoint{4.995714in}{5.925125in}}%
\pgfpathlineto{\pgfqpoint{4.958462in}{5.923366in}}%
\pgfpathlineto{\pgfqpoint{4.921196in}{5.920973in}}%
\pgfpathlineto{\pgfqpoint{4.883926in}{5.921248in}}%
\pgfpathlineto{\pgfqpoint{4.846540in}{5.906757in}}%
\pgfpathlineto{\pgfqpoint{4.816503in}{5.930845in}}%
\pgfpathlineto{\pgfqpoint{4.791764in}{5.909307in}}%
\pgfpathlineto{\pgfqpoint{4.766979in}{5.925260in}}%
\pgfpathlineto{\pgfqpoint{4.742205in}{5.908598in}}%
\pgfpathlineto{\pgfqpoint{4.717447in}{5.920974in}}%
\pgfpathlineto{\pgfqpoint{4.692637in}{5.908731in}}%
\pgfpathlineto{\pgfqpoint{4.667884in}{5.906386in}}%
\pgfpathlineto{\pgfqpoint{4.643083in}{5.908976in}}%
\pgfpathlineto{\pgfqpoint{4.618263in}{5.885322in}}%
\pgfpathlineto{\pgfqpoint{4.593377in}{5.892171in}}%
\pgfpathlineto{\pgfqpoint{4.568473in}{5.862978in}}%
\pgfpathlineto{\pgfqpoint{4.543529in}{5.851222in}}%
\pgfpathlineto{\pgfqpoint{4.518427in}{5.771935in}}%
\pgfpathlineto{\pgfqpoint{4.492692in}{5.331064in}}%
\pgfpathlineto{\pgfqpoint{4.465281in}{4.995191in}}%
\pgfpathlineto{\pgfqpoint{4.436765in}{4.721643in}}%
\pgfpathlineto{\pgfqpoint{4.407030in}{4.462792in}}%
\pgfpathlineto{\pgfqpoint{4.376239in}{4.281051in}}%
\pgfpathlineto{\pgfqpoint{4.344452in}{4.080060in}}%
\pgfpathlineto{\pgfqpoint{4.311749in}{3.929436in}}%
\pgfpathlineto{\pgfqpoint{4.278250in}{3.807579in}}%
\pgfpathlineto{\pgfqpoint{4.244034in}{3.705878in}}%
\pgfpathlineto{\pgfqpoint{4.209218in}{3.595131in}}%
\pgfpathlineto{\pgfqpoint{4.173852in}{3.521340in}}%
\pgfpathlineto{\pgfqpoint{4.138055in}{3.453135in}}%
\pgfpathlineto{\pgfqpoint{4.101899in}{3.395896in}}%
\pgfpathlineto{\pgfqpoint{4.065634in}{3.356544in}}%
\pgfpathlineto{\pgfqpoint{4.028745in}{3.337413in}}%
\pgfpathlineto{\pgfqpoint{3.991588in}{3.348610in}}%
\pgfpathlineto{\pgfqpoint{3.954566in}{3.352087in}}%
\pgfpathlineto{\pgfqpoint{3.917591in}{3.352081in}}%
\pgfpathlineto{\pgfqpoint{3.880608in}{3.352768in}}%
\pgfpathlineto{\pgfqpoint{3.843637in}{3.356373in}}%
\pgfpathlineto{\pgfqpoint{3.806671in}{3.359561in}}%
\pgfpathlineto{\pgfqpoint{3.769397in}{3.279789in}}%
\pgfpathlineto{\pgfqpoint{3.730402in}{3.201512in}}%
\pgfpathlineto{\pgfqpoint{3.691109in}{3.200491in}}%
\pgfpathlineto{\pgfqpoint{3.651783in}{3.198256in}}%
\pgfpathlineto{\pgfqpoint{3.612456in}{3.201385in}}%
\pgfpathlineto{\pgfqpoint{3.573169in}{3.201452in}}%
\pgfpathlineto{\pgfqpoint{3.533833in}{3.198066in}}%
\pgfpathlineto{\pgfqpoint{3.494517in}{3.204856in}}%
\pgfpathlineto{\pgfqpoint{3.455225in}{3.202255in}}%
\pgfpathlineto{\pgfqpoint{3.415888in}{3.202131in}}%
\pgfpathlineto{\pgfqpoint{3.376505in}{3.194553in}}%
\pgfpathlineto{\pgfqpoint{3.337163in}{3.200877in}}%
\pgfpathlineto{\pgfqpoint{3.297077in}{3.099470in}}%
\pgfpathlineto{\pgfqpoint{3.255737in}{3.006630in}}%
\pgfpathlineto{\pgfqpoint{3.210786in}{2.855593in}}%
\pgfpathlineto{\pgfqpoint{3.164694in}{2.827309in}}%
\pgfpathlineto{\pgfqpoint{3.117833in}{2.797983in}}%
\pgfpathlineto{\pgfqpoint{3.070408in}{2.763368in}}%
\pgfpathlineto{\pgfqpoint{3.021960in}{2.721433in}}%
\pgfpathlineto{\pgfqpoint{2.972423in}{2.679835in}}%
\pgfpathlineto{\pgfqpoint{2.921979in}{2.639209in}}%
\pgfpathlineto{\pgfqpoint{2.869949in}{2.574241in}}%
\pgfpathlineto{\pgfqpoint{2.815997in}{2.508103in}}%
\pgfpathlineto{\pgfqpoint{2.760229in}{2.466942in}}%
\pgfpathlineto{\pgfqpoint{2.704011in}{2.467193in}}%
\pgfpathlineto{\pgfqpoint{2.646993in}{2.430474in}}%
\pgfpathlineto{\pgfqpoint{2.589001in}{2.406564in}}%
\pgfpathlineto{\pgfqpoint{2.529907in}{2.359749in}}%
\pgfpathlineto{\pgfqpoint{2.469564in}{2.379650in}}%
\pgfpathlineto{\pgfqpoint{2.411446in}{2.452689in}}%
\pgfpathlineto{\pgfqpoint{2.355457in}{2.510297in}}%
\pgfpathlineto{\pgfqpoint{2.300442in}{2.494798in}}%
\pgfpathlineto{\pgfqpoint{2.244502in}{2.474823in}}%
\pgfpathlineto{\pgfqpoint{2.188824in}{2.508172in}}%
\pgfpathlineto{\pgfqpoint{2.134427in}{2.531703in}}%
\pgfpathlineto{\pgfqpoint{2.078568in}{2.431374in}}%
\pgfpathlineto{\pgfqpoint{2.021509in}{2.463064in}}%
\pgfpathlineto{\pgfqpoint{1.963640in}{2.398288in}}%
\pgfpathlineto{\pgfqpoint{1.905453in}{2.420577in}}%
\pgfpathlineto{\pgfqpoint{1.849942in}{2.593795in}}%
\pgfpathlineto{\pgfqpoint{1.795378in}{2.508748in}}%
\pgfpathlineto{\pgfqpoint{1.747243in}{3.074592in}}%
\pgfpathlineto{\pgfqpoint{1.705741in}{3.074584in}}%
\pgfpathlineto{\pgfqpoint{1.664297in}{3.076473in}}%
\pgfpathlineto{\pgfqpoint{1.622829in}{3.076781in}}%
\pgfpathlineto{\pgfqpoint{1.581396in}{3.077315in}}%
\pgfpathlineto{\pgfqpoint{1.539972in}{3.077723in}}%
\pgfpathlineto{\pgfqpoint{1.498540in}{3.076889in}}%
\pgfpathlineto{\pgfqpoint{1.457070in}{3.074954in}}%
\pgfpathlineto{\pgfqpoint{1.415629in}{3.077147in}}%
\pgfpathlineto{\pgfqpoint{1.374186in}{3.074872in}}%
\pgfpathlineto{\pgfqpoint{1.332721in}{3.075079in}}%
\pgfpathlineto{\pgfqpoint{1.291280in}{3.076669in}}%
\pgfpathlineto{\pgfqpoint{1.249856in}{3.077450in}}%
\pgfpathlineto{\pgfqpoint{1.208458in}{3.080244in}}%
\pgfpathlineto{\pgfqpoint{1.167091in}{3.080759in}}%
\pgfpathlineto{\pgfqpoint{1.125752in}{3.082011in}}%
\pgfpathlineto{\pgfqpoint{1.084375in}{3.077983in}}%
\pgfpathlineto{\pgfqpoint{1.042993in}{3.080930in}}%
\pgfpathlineto{\pgfqpoint{1.001616in}{2.946160in}}%
\pgfpathclose%
\pgfusepath{fill}%
\end{pgfscope}%
\begin{pgfscope}%
\pgfsetbuttcap%
\pgfsetroundjoin%
\definecolor{currentfill}{rgb}{0.000000,0.000000,0.000000}%
\pgfsetfillcolor{currentfill}%
\pgfsetlinewidth{0.803000pt}%
\definecolor{currentstroke}{rgb}{0.000000,0.000000,0.000000}%
\pgfsetstrokecolor{currentstroke}%
\pgfsetdash{}{0pt}%
\pgfsys@defobject{currentmarker}{\pgfqpoint{0.000000in}{-0.048611in}}{\pgfqpoint{0.000000in}{0.000000in}}{%
\pgfpathmoveto{\pgfqpoint{0.000000in}{0.000000in}}%
\pgfpathlineto{\pgfqpoint{0.000000in}{-0.048611in}}%
\pgfusepath{stroke,fill}%
}%
\begin{pgfscope}%
\pgfsys@transformshift{0.981539in}{0.773588in}%
\pgfsys@useobject{currentmarker}{}%
\end{pgfscope}%
\end{pgfscope}%
\begin{pgfscope}%
\definecolor{textcolor}{rgb}{0.000000,0.000000,0.000000}%
\pgfsetstrokecolor{textcolor}%
\pgfsetfillcolor{textcolor}%
\pgftext[x=0.981539in,y=0.676366in,,top]{\color{textcolor}\rmfamily\fontsize{10.000000}{12.000000}\selectfont \(\displaystyle {0}\)}%
\end{pgfscope}%
\begin{pgfscope}%
\pgfsetbuttcap%
\pgfsetroundjoin%
\definecolor{currentfill}{rgb}{0.000000,0.000000,0.000000}%
\pgfsetfillcolor{currentfill}%
\pgfsetlinewidth{0.803000pt}%
\definecolor{currentstroke}{rgb}{0.000000,0.000000,0.000000}%
\pgfsetstrokecolor{currentstroke}%
\pgfsetdash{}{0pt}%
\pgfsys@defobject{currentmarker}{\pgfqpoint{0.000000in}{-0.048611in}}{\pgfqpoint{0.000000in}{0.000000in}}{%
\pgfpathmoveto{\pgfqpoint{0.000000in}{0.000000in}}%
\pgfpathlineto{\pgfqpoint{0.000000in}{-0.048611in}}%
\pgfusepath{stroke,fill}%
}%
\begin{pgfscope}%
\pgfsys@transformshift{1.638927in}{0.773588in}%
\pgfsys@useobject{currentmarker}{}%
\end{pgfscope}%
\end{pgfscope}%
\begin{pgfscope}%
\definecolor{textcolor}{rgb}{0.000000,0.000000,0.000000}%
\pgfsetstrokecolor{textcolor}%
\pgfsetfillcolor{textcolor}%
\pgftext[x=1.638927in,y=0.676366in,,top]{\color{textcolor}\rmfamily\fontsize{10.000000}{12.000000}\selectfont \(\displaystyle {2000}\)}%
\end{pgfscope}%
\begin{pgfscope}%
\pgfsetbuttcap%
\pgfsetroundjoin%
\definecolor{currentfill}{rgb}{0.000000,0.000000,0.000000}%
\pgfsetfillcolor{currentfill}%
\pgfsetlinewidth{0.803000pt}%
\definecolor{currentstroke}{rgb}{0.000000,0.000000,0.000000}%
\pgfsetstrokecolor{currentstroke}%
\pgfsetdash{}{0pt}%
\pgfsys@defobject{currentmarker}{\pgfqpoint{0.000000in}{-0.048611in}}{\pgfqpoint{0.000000in}{0.000000in}}{%
\pgfpathmoveto{\pgfqpoint{0.000000in}{0.000000in}}%
\pgfpathlineto{\pgfqpoint{0.000000in}{-0.048611in}}%
\pgfusepath{stroke,fill}%
}%
\begin{pgfscope}%
\pgfsys@transformshift{2.296314in}{0.773588in}%
\pgfsys@useobject{currentmarker}{}%
\end{pgfscope}%
\end{pgfscope}%
\begin{pgfscope}%
\definecolor{textcolor}{rgb}{0.000000,0.000000,0.000000}%
\pgfsetstrokecolor{textcolor}%
\pgfsetfillcolor{textcolor}%
\pgftext[x=2.296314in,y=0.676366in,,top]{\color{textcolor}\rmfamily\fontsize{10.000000}{12.000000}\selectfont \(\displaystyle {4000}\)}%
\end{pgfscope}%
\begin{pgfscope}%
\pgfsetbuttcap%
\pgfsetroundjoin%
\definecolor{currentfill}{rgb}{0.000000,0.000000,0.000000}%
\pgfsetfillcolor{currentfill}%
\pgfsetlinewidth{0.803000pt}%
\definecolor{currentstroke}{rgb}{0.000000,0.000000,0.000000}%
\pgfsetstrokecolor{currentstroke}%
\pgfsetdash{}{0pt}%
\pgfsys@defobject{currentmarker}{\pgfqpoint{0.000000in}{-0.048611in}}{\pgfqpoint{0.000000in}{0.000000in}}{%
\pgfpathmoveto{\pgfqpoint{0.000000in}{0.000000in}}%
\pgfpathlineto{\pgfqpoint{0.000000in}{-0.048611in}}%
\pgfusepath{stroke,fill}%
}%
\begin{pgfscope}%
\pgfsys@transformshift{2.953702in}{0.773588in}%
\pgfsys@useobject{currentmarker}{}%
\end{pgfscope}%
\end{pgfscope}%
\begin{pgfscope}%
\definecolor{textcolor}{rgb}{0.000000,0.000000,0.000000}%
\pgfsetstrokecolor{textcolor}%
\pgfsetfillcolor{textcolor}%
\pgftext[x=2.953702in,y=0.676366in,,top]{\color{textcolor}\rmfamily\fontsize{10.000000}{12.000000}\selectfont \(\displaystyle {6000}\)}%
\end{pgfscope}%
\begin{pgfscope}%
\pgfsetbuttcap%
\pgfsetroundjoin%
\definecolor{currentfill}{rgb}{0.000000,0.000000,0.000000}%
\pgfsetfillcolor{currentfill}%
\pgfsetlinewidth{0.803000pt}%
\definecolor{currentstroke}{rgb}{0.000000,0.000000,0.000000}%
\pgfsetstrokecolor{currentstroke}%
\pgfsetdash{}{0pt}%
\pgfsys@defobject{currentmarker}{\pgfqpoint{0.000000in}{-0.048611in}}{\pgfqpoint{0.000000in}{0.000000in}}{%
\pgfpathmoveto{\pgfqpoint{0.000000in}{0.000000in}}%
\pgfpathlineto{\pgfqpoint{0.000000in}{-0.048611in}}%
\pgfusepath{stroke,fill}%
}%
\begin{pgfscope}%
\pgfsys@transformshift{3.611089in}{0.773588in}%
\pgfsys@useobject{currentmarker}{}%
\end{pgfscope}%
\end{pgfscope}%
\begin{pgfscope}%
\definecolor{textcolor}{rgb}{0.000000,0.000000,0.000000}%
\pgfsetstrokecolor{textcolor}%
\pgfsetfillcolor{textcolor}%
\pgftext[x=3.611089in,y=0.676366in,,top]{\color{textcolor}\rmfamily\fontsize{10.000000}{12.000000}\selectfont \(\displaystyle {8000}\)}%
\end{pgfscope}%
\begin{pgfscope}%
\pgfsetbuttcap%
\pgfsetroundjoin%
\definecolor{currentfill}{rgb}{0.000000,0.000000,0.000000}%
\pgfsetfillcolor{currentfill}%
\pgfsetlinewidth{0.803000pt}%
\definecolor{currentstroke}{rgb}{0.000000,0.000000,0.000000}%
\pgfsetstrokecolor{currentstroke}%
\pgfsetdash{}{0pt}%
\pgfsys@defobject{currentmarker}{\pgfqpoint{0.000000in}{-0.048611in}}{\pgfqpoint{0.000000in}{0.000000in}}{%
\pgfpathmoveto{\pgfqpoint{0.000000in}{0.000000in}}%
\pgfpathlineto{\pgfqpoint{0.000000in}{-0.048611in}}%
\pgfusepath{stroke,fill}%
}%
\begin{pgfscope}%
\pgfsys@transformshift{4.268476in}{0.773588in}%
\pgfsys@useobject{currentmarker}{}%
\end{pgfscope}%
\end{pgfscope}%
\begin{pgfscope}%
\definecolor{textcolor}{rgb}{0.000000,0.000000,0.000000}%
\pgfsetstrokecolor{textcolor}%
\pgfsetfillcolor{textcolor}%
\pgftext[x=4.268476in,y=0.676366in,,top]{\color{textcolor}\rmfamily\fontsize{10.000000}{12.000000}\selectfont \(\displaystyle {10000}\)}%
\end{pgfscope}%
\begin{pgfscope}%
\pgfsetbuttcap%
\pgfsetroundjoin%
\definecolor{currentfill}{rgb}{0.000000,0.000000,0.000000}%
\pgfsetfillcolor{currentfill}%
\pgfsetlinewidth{0.803000pt}%
\definecolor{currentstroke}{rgb}{0.000000,0.000000,0.000000}%
\pgfsetstrokecolor{currentstroke}%
\pgfsetdash{}{0pt}%
\pgfsys@defobject{currentmarker}{\pgfqpoint{0.000000in}{-0.048611in}}{\pgfqpoint{0.000000in}{0.000000in}}{%
\pgfpathmoveto{\pgfqpoint{0.000000in}{0.000000in}}%
\pgfpathlineto{\pgfqpoint{0.000000in}{-0.048611in}}%
\pgfusepath{stroke,fill}%
}%
\begin{pgfscope}%
\pgfsys@transformshift{4.925864in}{0.773588in}%
\pgfsys@useobject{currentmarker}{}%
\end{pgfscope}%
\end{pgfscope}%
\begin{pgfscope}%
\definecolor{textcolor}{rgb}{0.000000,0.000000,0.000000}%
\pgfsetstrokecolor{textcolor}%
\pgfsetfillcolor{textcolor}%
\pgftext[x=4.925864in,y=0.676366in,,top]{\color{textcolor}\rmfamily\fontsize{10.000000}{12.000000}\selectfont \(\displaystyle {12000}\)}%
\end{pgfscope}%
\begin{pgfscope}%
\pgfsetbuttcap%
\pgfsetroundjoin%
\definecolor{currentfill}{rgb}{0.000000,0.000000,0.000000}%
\pgfsetfillcolor{currentfill}%
\pgfsetlinewidth{0.803000pt}%
\definecolor{currentstroke}{rgb}{0.000000,0.000000,0.000000}%
\pgfsetstrokecolor{currentstroke}%
\pgfsetdash{}{0pt}%
\pgfsys@defobject{currentmarker}{\pgfqpoint{0.000000in}{-0.048611in}}{\pgfqpoint{0.000000in}{0.000000in}}{%
\pgfpathmoveto{\pgfqpoint{0.000000in}{0.000000in}}%
\pgfpathlineto{\pgfqpoint{0.000000in}{-0.048611in}}%
\pgfusepath{stroke,fill}%
}%
\begin{pgfscope}%
\pgfsys@transformshift{5.583251in}{0.773588in}%
\pgfsys@useobject{currentmarker}{}%
\end{pgfscope}%
\end{pgfscope}%
\begin{pgfscope}%
\definecolor{textcolor}{rgb}{0.000000,0.000000,0.000000}%
\pgfsetstrokecolor{textcolor}%
\pgfsetfillcolor{textcolor}%
\pgftext[x=5.583251in,y=0.676366in,,top]{\color{textcolor}\rmfamily\fontsize{10.000000}{12.000000}\selectfont \(\displaystyle {14000}\)}%
\end{pgfscope}%
\begin{pgfscope}%
\definecolor{textcolor}{rgb}{0.000000,0.000000,0.000000}%
\pgfsetstrokecolor{textcolor}%
\pgfsetfillcolor{textcolor}%
\pgftext[x=3.203750in,y=0.497354in,,top]{\color{textcolor}\rmfamily\fontsize{10.000000}{12.000000}\selectfont Time (milliseconds)}%
\end{pgfscope}%
\begin{pgfscope}%
\pgfsetbuttcap%
\pgfsetroundjoin%
\definecolor{currentfill}{rgb}{0.000000,0.000000,0.000000}%
\pgfsetfillcolor{currentfill}%
\pgfsetlinewidth{0.803000pt}%
\definecolor{currentstroke}{rgb}{0.000000,0.000000,0.000000}%
\pgfsetstrokecolor{currentstroke}%
\pgfsetdash{}{0pt}%
\pgfsys@defobject{currentmarker}{\pgfqpoint{-0.048611in}{0.000000in}}{\pgfqpoint{-0.000000in}{0.000000in}}{%
\pgfpathmoveto{\pgfqpoint{-0.000000in}{0.000000in}}%
\pgfpathlineto{\pgfqpoint{-0.048611in}{0.000000in}}%
\pgfusepath{stroke,fill}%
}%
\begin{pgfscope}%
\pgfsys@transformshift{0.781402in}{0.773588in}%
\pgfsys@useobject{currentmarker}{}%
\end{pgfscope}%
\end{pgfscope}%
\begin{pgfscope}%
\definecolor{textcolor}{rgb}{0.000000,0.000000,0.000000}%
\pgfsetstrokecolor{textcolor}%
\pgfsetfillcolor{textcolor}%
\pgftext[x=0.614736in, y=0.725363in, left, base]{\color{textcolor}\rmfamily\fontsize{10.000000}{12.000000}\selectfont \(\displaystyle {0}\)}%
\end{pgfscope}%
\begin{pgfscope}%
\pgfsetbuttcap%
\pgfsetroundjoin%
\definecolor{currentfill}{rgb}{0.000000,0.000000,0.000000}%
\pgfsetfillcolor{currentfill}%
\pgfsetlinewidth{0.803000pt}%
\definecolor{currentstroke}{rgb}{0.000000,0.000000,0.000000}%
\pgfsetstrokecolor{currentstroke}%
\pgfsetdash{}{0pt}%
\pgfsys@defobject{currentmarker}{\pgfqpoint{-0.048611in}{0.000000in}}{\pgfqpoint{-0.000000in}{0.000000in}}{%
\pgfpathmoveto{\pgfqpoint{-0.000000in}{0.000000in}}%
\pgfpathlineto{\pgfqpoint{-0.048611in}{0.000000in}}%
\pgfusepath{stroke,fill}%
}%
\begin{pgfscope}%
\pgfsys@transformshift{0.781402in}{1.934933in}%
\pgfsys@useobject{currentmarker}{}%
\end{pgfscope}%
\end{pgfscope}%
\begin{pgfscope}%
\definecolor{textcolor}{rgb}{0.000000,0.000000,0.000000}%
\pgfsetstrokecolor{textcolor}%
\pgfsetfillcolor{textcolor}%
\pgftext[x=0.614736in, y=1.886708in, left, base]{\color{textcolor}\rmfamily\fontsize{10.000000}{12.000000}\selectfont \(\displaystyle {2}\)}%
\end{pgfscope}%
\begin{pgfscope}%
\pgfsetbuttcap%
\pgfsetroundjoin%
\definecolor{currentfill}{rgb}{0.000000,0.000000,0.000000}%
\pgfsetfillcolor{currentfill}%
\pgfsetlinewidth{0.803000pt}%
\definecolor{currentstroke}{rgb}{0.000000,0.000000,0.000000}%
\pgfsetstrokecolor{currentstroke}%
\pgfsetdash{}{0pt}%
\pgfsys@defobject{currentmarker}{\pgfqpoint{-0.048611in}{0.000000in}}{\pgfqpoint{-0.000000in}{0.000000in}}{%
\pgfpathmoveto{\pgfqpoint{-0.000000in}{0.000000in}}%
\pgfpathlineto{\pgfqpoint{-0.048611in}{0.000000in}}%
\pgfusepath{stroke,fill}%
}%
\begin{pgfscope}%
\pgfsys@transformshift{0.781402in}{3.096277in}%
\pgfsys@useobject{currentmarker}{}%
\end{pgfscope}%
\end{pgfscope}%
\begin{pgfscope}%
\definecolor{textcolor}{rgb}{0.000000,0.000000,0.000000}%
\pgfsetstrokecolor{textcolor}%
\pgfsetfillcolor{textcolor}%
\pgftext[x=0.614736in, y=3.048052in, left, base]{\color{textcolor}\rmfamily\fontsize{10.000000}{12.000000}\selectfont \(\displaystyle {4}\)}%
\end{pgfscope}%
\begin{pgfscope}%
\pgfsetbuttcap%
\pgfsetroundjoin%
\definecolor{currentfill}{rgb}{0.000000,0.000000,0.000000}%
\pgfsetfillcolor{currentfill}%
\pgfsetlinewidth{0.803000pt}%
\definecolor{currentstroke}{rgb}{0.000000,0.000000,0.000000}%
\pgfsetstrokecolor{currentstroke}%
\pgfsetdash{}{0pt}%
\pgfsys@defobject{currentmarker}{\pgfqpoint{-0.048611in}{0.000000in}}{\pgfqpoint{-0.000000in}{0.000000in}}{%
\pgfpathmoveto{\pgfqpoint{-0.000000in}{0.000000in}}%
\pgfpathlineto{\pgfqpoint{-0.048611in}{0.000000in}}%
\pgfusepath{stroke,fill}%
}%
\begin{pgfscope}%
\pgfsys@transformshift{0.781402in}{4.257622in}%
\pgfsys@useobject{currentmarker}{}%
\end{pgfscope}%
\end{pgfscope}%
\begin{pgfscope}%
\definecolor{textcolor}{rgb}{0.000000,0.000000,0.000000}%
\pgfsetstrokecolor{textcolor}%
\pgfsetfillcolor{textcolor}%
\pgftext[x=0.614736in, y=4.209396in, left, base]{\color{textcolor}\rmfamily\fontsize{10.000000}{12.000000}\selectfont \(\displaystyle {6}\)}%
\end{pgfscope}%
\begin{pgfscope}%
\pgfsetbuttcap%
\pgfsetroundjoin%
\definecolor{currentfill}{rgb}{0.000000,0.000000,0.000000}%
\pgfsetfillcolor{currentfill}%
\pgfsetlinewidth{0.803000pt}%
\definecolor{currentstroke}{rgb}{0.000000,0.000000,0.000000}%
\pgfsetstrokecolor{currentstroke}%
\pgfsetdash{}{0pt}%
\pgfsys@defobject{currentmarker}{\pgfqpoint{-0.048611in}{0.000000in}}{\pgfqpoint{-0.000000in}{0.000000in}}{%
\pgfpathmoveto{\pgfqpoint{-0.000000in}{0.000000in}}%
\pgfpathlineto{\pgfqpoint{-0.048611in}{0.000000in}}%
\pgfusepath{stroke,fill}%
}%
\begin{pgfscope}%
\pgfsys@transformshift{0.781402in}{5.418966in}%
\pgfsys@useobject{currentmarker}{}%
\end{pgfscope}%
\end{pgfscope}%
\begin{pgfscope}%
\definecolor{textcolor}{rgb}{0.000000,0.000000,0.000000}%
\pgfsetstrokecolor{textcolor}%
\pgfsetfillcolor{textcolor}%
\pgftext[x=0.614736in, y=5.370741in, left, base]{\color{textcolor}\rmfamily\fontsize{10.000000}{12.000000}\selectfont \(\displaystyle {8}\)}%
\end{pgfscope}%
\begin{pgfscope}%
\definecolor{textcolor}{rgb}{0.000000,0.000000,0.000000}%
\pgfsetstrokecolor{textcolor}%
\pgfsetfillcolor{textcolor}%
\pgftext[x=0.559180in,y=3.481148in,,bottom,rotate=90.000000]{\color{textcolor}\rmfamily\fontsize{10.000000}{12.000000}\selectfont Throughput (million operations/second)}%
\end{pgfscope}%
\begin{pgfscope}%
\pgfpathrectangle{\pgfqpoint{0.781402in}{0.773588in}}{\pgfqpoint{4.844695in}{5.415119in}}%
\pgfusepath{clip}%
\pgfsetrectcap%
\pgfsetroundjoin%
\pgfsetlinewidth{1.505625pt}%
\definecolor{currentstroke}{rgb}{0.000000,0.000000,1.000000}%
\pgfsetstrokecolor{currentstroke}%
\pgfsetdash{}{0pt}%
\pgfpathmoveto{\pgfqpoint{0.983123in}{0.773588in}}%
\pgfpathlineto{\pgfqpoint{0.983123in}{6.188708in}}%
\pgfusepath{stroke}%
\end{pgfscope}%
\begin{pgfscope}%
\pgfpathrectangle{\pgfqpoint{0.781402in}{0.773588in}}{\pgfqpoint{4.844695in}{5.415119in}}%
\pgfusepath{clip}%
\pgfsetrectcap%
\pgfsetroundjoin%
\pgfsetlinewidth{1.505625pt}%
\definecolor{currentstroke}{rgb}{0.750000,0.750000,0.000000}%
\pgfsetstrokecolor{currentstroke}%
\pgfsetdash{}{0pt}%
\pgfpathmoveto{\pgfqpoint{1.766893in}{0.773588in}}%
\pgfpathlineto{\pgfqpoint{1.766893in}{6.188708in}}%
\pgfusepath{stroke}%
\end{pgfscope}%
\begin{pgfscope}%
\pgfpathrectangle{\pgfqpoint{0.781402in}{0.773588in}}{\pgfqpoint{4.844695in}{5.415119in}}%
\pgfusepath{clip}%
\pgfsetrectcap%
\pgfsetroundjoin%
\pgfsetlinewidth{1.505625pt}%
\definecolor{currentstroke}{rgb}{0.750000,0.000000,0.750000}%
\pgfsetstrokecolor{currentstroke}%
\pgfsetdash{}{0pt}%
\pgfpathmoveto{\pgfqpoint{3.253738in}{0.773588in}}%
\pgfpathlineto{\pgfqpoint{3.253738in}{6.188708in}}%
\pgfusepath{stroke}%
\end{pgfscope}%
\begin{pgfscope}%
\pgfpathrectangle{\pgfqpoint{0.781402in}{0.773588in}}{\pgfqpoint{4.844695in}{5.415119in}}%
\pgfusepath{clip}%
\pgfsetrectcap%
\pgfsetroundjoin%
\pgfsetlinewidth{1.505625pt}%
\definecolor{currentstroke}{rgb}{1.000000,0.000000,0.000000}%
\pgfsetstrokecolor{currentstroke}%
\pgfsetdash{}{0pt}%
\pgfpathmoveto{\pgfqpoint{3.766609in}{0.773588in}}%
\pgfpathlineto{\pgfqpoint{3.766609in}{6.188708in}}%
\pgfusepath{stroke}%
\end{pgfscope}%
\begin{pgfscope}%
\pgfpathrectangle{\pgfqpoint{0.781402in}{0.773588in}}{\pgfqpoint{4.844695in}{5.415119in}}%
\pgfusepath{clip}%
\pgfsetrectcap%
\pgfsetroundjoin%
\pgfsetlinewidth{1.505625pt}%
\definecolor{currentstroke}{rgb}{0.000000,0.500000,0.000000}%
\pgfsetstrokecolor{currentstroke}%
\pgfsetdash{}{0pt}%
\pgfpathmoveto{\pgfqpoint{4.001288in}{0.773588in}}%
\pgfpathlineto{\pgfqpoint{4.001288in}{6.188708in}}%
\pgfusepath{stroke}%
\end{pgfscope}%
\begin{pgfscope}%
\pgfpathrectangle{\pgfqpoint{0.781402in}{0.773588in}}{\pgfqpoint{4.844695in}{5.415119in}}%
\pgfusepath{clip}%
\pgfsetrectcap%
\pgfsetroundjoin%
\pgfsetlinewidth{1.505625pt}%
\definecolor{currentstroke}{rgb}{0.000000,0.750000,0.750000}%
\pgfsetstrokecolor{currentstroke}%
\pgfsetdash{}{0pt}%
\pgfpathmoveto{\pgfqpoint{4.500409in}{0.773588in}}%
\pgfpathlineto{\pgfqpoint{4.500409in}{6.188708in}}%
\pgfusepath{stroke}%
\end{pgfscope}%
\begin{pgfscope}%
\pgfpathrectangle{\pgfqpoint{0.781402in}{0.773588in}}{\pgfqpoint{4.844695in}{5.415119in}}%
\pgfusepath{clip}%
\pgfsetrectcap%
\pgfsetroundjoin%
\pgfsetlinewidth{1.505625pt}%
\definecolor{currentstroke}{rgb}{0.750000,0.000000,0.750000}%
\pgfsetstrokecolor{currentstroke}%
\pgfsetdash{}{0pt}%
\pgfpathmoveto{\pgfqpoint{4.500413in}{0.773588in}}%
\pgfpathlineto{\pgfqpoint{4.500413in}{6.188708in}}%
\pgfusepath{stroke}%
\end{pgfscope}%
\begin{pgfscope}%
\pgfsetrectcap%
\pgfsetmiterjoin%
\pgfsetlinewidth{0.803000pt}%
\definecolor{currentstroke}{rgb}{0.000000,0.000000,0.000000}%
\pgfsetstrokecolor{currentstroke}%
\pgfsetdash{}{0pt}%
\pgfpathmoveto{\pgfqpoint{0.781402in}{0.773588in}}%
\pgfpathlineto{\pgfqpoint{0.781402in}{6.188708in}}%
\pgfusepath{stroke}%
\end{pgfscope}%
\begin{pgfscope}%
\pgfsetrectcap%
\pgfsetmiterjoin%
\pgfsetlinewidth{0.803000pt}%
\definecolor{currentstroke}{rgb}{0.000000,0.000000,0.000000}%
\pgfsetstrokecolor{currentstroke}%
\pgfsetdash{}{0pt}%
\pgfpathmoveto{\pgfqpoint{5.626098in}{0.773588in}}%
\pgfpathlineto{\pgfqpoint{5.626098in}{6.188708in}}%
\pgfusepath{stroke}%
\end{pgfscope}%
\begin{pgfscope}%
\pgfsetrectcap%
\pgfsetmiterjoin%
\pgfsetlinewidth{0.803000pt}%
\definecolor{currentstroke}{rgb}{0.000000,0.000000,0.000000}%
\pgfsetstrokecolor{currentstroke}%
\pgfsetdash{}{0pt}%
\pgfpathmoveto{\pgfqpoint{0.781402in}{0.773588in}}%
\pgfpathlineto{\pgfqpoint{5.626098in}{0.773588in}}%
\pgfusepath{stroke}%
\end{pgfscope}%
\begin{pgfscope}%
\pgfsetrectcap%
\pgfsetmiterjoin%
\pgfsetlinewidth{0.803000pt}%
\definecolor{currentstroke}{rgb}{0.000000,0.000000,0.000000}%
\pgfsetstrokecolor{currentstroke}%
\pgfsetdash{}{0pt}%
\pgfpathmoveto{\pgfqpoint{0.781402in}{6.188708in}}%
\pgfpathlineto{\pgfqpoint{5.626098in}{6.188708in}}%
\pgfusepath{stroke}%
\end{pgfscope}%
\begin{pgfscope}%
\pgfsetbuttcap%
\pgfsetmiterjoin%
\definecolor{currentfill}{rgb}{1.000000,1.000000,1.000000}%
\pgfsetfillcolor{currentfill}%
\pgfsetfillopacity{0.800000}%
\pgfsetlinewidth{1.003750pt}%
\definecolor{currentstroke}{rgb}{0.800000,0.800000,0.800000}%
\pgfsetstrokecolor{currentstroke}%
\pgfsetstrokeopacity{0.800000}%
\pgfsetdash{}{0pt}%
\pgfpathmoveto{\pgfqpoint{0.878625in}{3.559851in}}%
\pgfpathlineto{\pgfqpoint{3.799925in}{3.559851in}}%
\pgfpathquadraticcurveto{\pgfqpoint{3.827703in}{3.559851in}}{\pgfqpoint{3.827703in}{3.587628in}}%
\pgfpathlineto{\pgfqpoint{3.827703in}{6.091486in}}%
\pgfpathquadraticcurveto{\pgfqpoint{3.827703in}{6.119263in}}{\pgfqpoint{3.799925in}{6.119263in}}%
\pgfpathlineto{\pgfqpoint{0.878625in}{6.119263in}}%
\pgfpathquadraticcurveto{\pgfqpoint{0.850847in}{6.119263in}}{\pgfqpoint{0.850847in}{6.091486in}}%
\pgfpathlineto{\pgfqpoint{0.850847in}{3.587628in}}%
\pgfpathquadraticcurveto{\pgfqpoint{0.850847in}{3.559851in}}{\pgfqpoint{0.878625in}{3.559851in}}%
\pgfpathclose%
\pgfusepath{stroke,fill}%
\end{pgfscope}%
\begin{pgfscope}%
\pgfsetrectcap%
\pgfsetroundjoin%
\pgfsetlinewidth{1.505625pt}%
\definecolor{currentstroke}{rgb}{0.000000,0.000000,1.000000}%
\pgfsetstrokecolor{currentstroke}%
\pgfsetdash{}{0pt}%
\pgfpathmoveto{\pgfqpoint{0.906402in}{6.015097in}}%
\pgfpathlineto{\pgfqpoint{1.184180in}{6.015097in}}%
\pgfusepath{stroke}%
\end{pgfscope}%
\begin{pgfscope}%
\definecolor{textcolor}{rgb}{0.000000,0.000000,0.000000}%
\pgfsetstrokecolor{textcolor}%
\pgfsetfillcolor{textcolor}%
\pgftext[x=1.295291in,y=5.966486in,left,base]{\color{textcolor}\rmfamily\fontsize{10.000000}{12.000000}\selectfont Started migration}%
\end{pgfscope}%
\begin{pgfscope}%
\pgfsetrectcap%
\pgfsetroundjoin%
\pgfsetlinewidth{1.505625pt}%
\definecolor{currentstroke}{rgb}{0.750000,0.750000,0.000000}%
\pgfsetstrokecolor{currentstroke}%
\pgfsetdash{}{0pt}%
\pgfpathmoveto{\pgfqpoint{0.906402in}{5.821424in}}%
\pgfpathlineto{\pgfqpoint{1.184180in}{5.821424in}}%
\pgfusepath{stroke}%
\end{pgfscope}%
\begin{pgfscope}%
\definecolor{textcolor}{rgb}{0.000000,0.000000,0.000000}%
\pgfsetstrokecolor{textcolor}%
\pgfsetfillcolor{textcolor}%
\pgftext[x=1.295291in,y=5.772813in,left,base]{\color{textcolor}\rmfamily\fontsize{10.000000}{12.000000}\selectfont Started prefill writes}%
\end{pgfscope}%
\begin{pgfscope}%
\pgfsetrectcap%
\pgfsetroundjoin%
\pgfsetlinewidth{1.505625pt}%
\definecolor{currentstroke}{rgb}{0.750000,0.000000,0.750000}%
\pgfsetstrokecolor{currentstroke}%
\pgfsetdash{}{0pt}%
\pgfpathmoveto{\pgfqpoint{0.906402in}{5.627751in}}%
\pgfpathlineto{\pgfqpoint{1.184180in}{5.627751in}}%
\pgfusepath{stroke}%
\end{pgfscope}%
\begin{pgfscope}%
\definecolor{textcolor}{rgb}{0.000000,0.000000,0.000000}%
\pgfsetstrokecolor{textcolor}%
\pgfsetfillcolor{textcolor}%
\pgftext[x=1.295291in,y=5.579140in,left,base]{\color{textcolor}\rmfamily\fontsize{10.000000}{12.000000}\selectfont Finished prefill writes}%
\end{pgfscope}%
\begin{pgfscope}%
\pgfsetrectcap%
\pgfsetroundjoin%
\pgfsetlinewidth{1.505625pt}%
\definecolor{currentstroke}{rgb}{1.000000,0.000000,0.000000}%
\pgfsetstrokecolor{currentstroke}%
\pgfsetdash{}{0pt}%
\pgfpathmoveto{\pgfqpoint{0.906402in}{5.434078in}}%
\pgfpathlineto{\pgfqpoint{1.184180in}{5.434078in}}%
\pgfusepath{stroke}%
\end{pgfscope}%
\begin{pgfscope}%
\definecolor{textcolor}{rgb}{0.000000,0.000000,0.000000}%
\pgfsetstrokecolor{textcolor}%
\pgfsetfillcolor{textcolor}%
\pgftext[x=1.295291in,y=5.385467in,left,base]{\color{textcolor}\rmfamily\fontsize{10.000000}{12.000000}\selectfont Transferred ownership to the destination}%
\end{pgfscope}%
\begin{pgfscope}%
\pgfsetrectcap%
\pgfsetroundjoin%
\pgfsetlinewidth{1.505625pt}%
\definecolor{currentstroke}{rgb}{0.000000,0.500000,0.000000}%
\pgfsetstrokecolor{currentstroke}%
\pgfsetdash{}{0pt}%
\pgfpathmoveto{\pgfqpoint{0.906402in}{5.240406in}}%
\pgfpathlineto{\pgfqpoint{1.184180in}{5.240406in}}%
\pgfusepath{stroke}%
\end{pgfscope}%
\begin{pgfscope}%
\definecolor{textcolor}{rgb}{0.000000,0.000000,0.000000}%
\pgfsetstrokecolor{textcolor}%
\pgfsetfillcolor{textcolor}%
\pgftext[x=1.295291in,y=5.191794in,left,base]{\color{textcolor}\rmfamily\fontsize{10.000000}{12.000000}\selectfont Started reading dirty pages}%
\end{pgfscope}%
\begin{pgfscope}%
\pgfsetrectcap%
\pgfsetroundjoin%
\pgfsetlinewidth{1.505625pt}%
\definecolor{currentstroke}{rgb}{0.000000,0.750000,0.750000}%
\pgfsetstrokecolor{currentstroke}%
\pgfsetdash{}{0pt}%
\pgfpathmoveto{\pgfqpoint{0.906402in}{5.046733in}}%
\pgfpathlineto{\pgfqpoint{1.184180in}{5.046733in}}%
\pgfusepath{stroke}%
\end{pgfscope}%
\begin{pgfscope}%
\definecolor{textcolor}{rgb}{0.000000,0.000000,0.000000}%
\pgfsetstrokecolor{textcolor}%
\pgfsetfillcolor{textcolor}%
\pgftext[x=1.295291in,y=4.998122in,left,base]{\color{textcolor}\rmfamily\fontsize{10.000000}{12.000000}\selectfont Finished reading dirty pages}%
\end{pgfscope}%
\begin{pgfscope}%
\pgfsetrectcap%
\pgfsetroundjoin%
\pgfsetlinewidth{1.505625pt}%
\definecolor{currentstroke}{rgb}{0.750000,0.000000,0.750000}%
\pgfsetstrokecolor{currentstroke}%
\pgfsetdash{}{0pt}%
\pgfpathmoveto{\pgfqpoint{0.906402in}{4.853060in}}%
\pgfpathlineto{\pgfqpoint{1.184180in}{4.853060in}}%
\pgfusepath{stroke}%
\end{pgfscope}%
\begin{pgfscope}%
\definecolor{textcolor}{rgb}{0.000000,0.000000,0.000000}%
\pgfsetstrokecolor{textcolor}%
\pgfsetfillcolor{textcolor}%
\pgftext[x=1.295291in,y=4.804449in,left,base]{\color{textcolor}\rmfamily\fontsize{10.000000}{12.000000}\selectfont Finished migration}%
\end{pgfscope}%
\begin{pgfscope}%
\pgfsetbuttcap%
\pgfsetmiterjoin%
\definecolor{currentfill}{rgb}{0.121569,0.466667,0.705882}%
\pgfsetfillcolor{currentfill}%
\pgfsetlinewidth{0.000000pt}%
\definecolor{currentstroke}{rgb}{0.000000,0.000000,0.000000}%
\pgfsetstrokecolor{currentstroke}%
\pgfsetstrokeopacity{0.000000}%
\pgfsetdash{}{0pt}%
\pgfpathmoveto{\pgfqpoint{0.906402in}{4.610776in}}%
\pgfpathlineto{\pgfqpoint{1.184180in}{4.610776in}}%
\pgfpathlineto{\pgfqpoint{1.184180in}{4.707998in}}%
\pgfpathlineto{\pgfqpoint{0.906402in}{4.707998in}}%
\pgfpathclose%
\pgfusepath{fill}%
\end{pgfscope}%
\begin{pgfscope}%
\definecolor{textcolor}{rgb}{0.000000,0.000000,0.000000}%
\pgfsetstrokecolor{textcolor}%
\pgfsetfillcolor{textcolor}%
\pgftext[x=1.295291in,y=4.610776in,left,base]{\color{textcolor}\rmfamily\fontsize{10.000000}{12.000000}\selectfont BF1 read at destination}%
\end{pgfscope}%
\begin{pgfscope}%
\pgfsetbuttcap%
\pgfsetmiterjoin%
\definecolor{currentfill}{rgb}{1.000000,0.498039,0.054902}%
\pgfsetfillcolor{currentfill}%
\pgfsetlinewidth{0.000000pt}%
\definecolor{currentstroke}{rgb}{0.000000,0.000000,0.000000}%
\pgfsetstrokecolor{currentstroke}%
\pgfsetstrokeopacity{0.000000}%
\pgfsetdash{}{0pt}%
\pgfpathmoveto{\pgfqpoint{0.906402in}{4.417103in}}%
\pgfpathlineto{\pgfqpoint{1.184180in}{4.417103in}}%
\pgfpathlineto{\pgfqpoint{1.184180in}{4.514326in}}%
\pgfpathlineto{\pgfqpoint{0.906402in}{4.514326in}}%
\pgfpathclose%
\pgfusepath{fill}%
\end{pgfscope}%
\begin{pgfscope}%
\definecolor{textcolor}{rgb}{0.000000,0.000000,0.000000}%
\pgfsetstrokecolor{textcolor}%
\pgfsetfillcolor{textcolor}%
\pgftext[x=1.295291in,y=4.417103in,left,base]{\color{textcolor}\rmfamily\fontsize{10.000000}{12.000000}\selectfont BF1 write at destination}%
\end{pgfscope}%
\begin{pgfscope}%
\pgfsetbuttcap%
\pgfsetmiterjoin%
\definecolor{currentfill}{rgb}{0.172549,0.627451,0.172549}%
\pgfsetfillcolor{currentfill}%
\pgfsetlinewidth{0.000000pt}%
\definecolor{currentstroke}{rgb}{0.000000,0.000000,0.000000}%
\pgfsetstrokecolor{currentstroke}%
\pgfsetstrokeopacity{0.000000}%
\pgfsetdash{}{0pt}%
\pgfpathmoveto{\pgfqpoint{0.906402in}{4.223431in}}%
\pgfpathlineto{\pgfqpoint{1.184180in}{4.223431in}}%
\pgfpathlineto{\pgfqpoint{1.184180in}{4.320653in}}%
\pgfpathlineto{\pgfqpoint{0.906402in}{4.320653in}}%
\pgfpathclose%
\pgfusepath{fill}%
\end{pgfscope}%
\begin{pgfscope}%
\definecolor{textcolor}{rgb}{0.000000,0.000000,0.000000}%
\pgfsetstrokecolor{textcolor}%
\pgfsetfillcolor{textcolor}%
\pgftext[x=1.295291in,y=4.223431in,left,base]{\color{textcolor}\rmfamily\fontsize{10.000000}{12.000000}\selectfont BF2 read at source}%
\end{pgfscope}%
\begin{pgfscope}%
\pgfsetbuttcap%
\pgfsetmiterjoin%
\definecolor{currentfill}{rgb}{0.839216,0.152941,0.156863}%
\pgfsetfillcolor{currentfill}%
\pgfsetlinewidth{0.000000pt}%
\definecolor{currentstroke}{rgb}{0.000000,0.000000,0.000000}%
\pgfsetstrokecolor{currentstroke}%
\pgfsetstrokeopacity{0.000000}%
\pgfsetdash{}{0pt}%
\pgfpathmoveto{\pgfqpoint{0.906402in}{4.029758in}}%
\pgfpathlineto{\pgfqpoint{1.184180in}{4.029758in}}%
\pgfpathlineto{\pgfqpoint{1.184180in}{4.126980in}}%
\pgfpathlineto{\pgfqpoint{0.906402in}{4.126980in}}%
\pgfpathclose%
\pgfusepath{fill}%
\end{pgfscope}%
\begin{pgfscope}%
\definecolor{textcolor}{rgb}{0.000000,0.000000,0.000000}%
\pgfsetstrokecolor{textcolor}%
\pgfsetfillcolor{textcolor}%
\pgftext[x=1.295291in,y=4.029758in,left,base]{\color{textcolor}\rmfamily\fontsize{10.000000}{12.000000}\selectfont BF2 write at source}%
\end{pgfscope}%
\begin{pgfscope}%
\pgfsetbuttcap%
\pgfsetmiterjoin%
\definecolor{currentfill}{rgb}{0.580392,0.403922,0.741176}%
\pgfsetfillcolor{currentfill}%
\pgfsetlinewidth{0.000000pt}%
\definecolor{currentstroke}{rgb}{0.000000,0.000000,0.000000}%
\pgfsetstrokecolor{currentstroke}%
\pgfsetstrokeopacity{0.000000}%
\pgfsetdash{}{0pt}%
\pgfpathmoveto{\pgfqpoint{0.906402in}{3.836085in}}%
\pgfpathlineto{\pgfqpoint{1.184180in}{3.836085in}}%
\pgfpathlineto{\pgfqpoint{1.184180in}{3.933307in}}%
\pgfpathlineto{\pgfqpoint{0.906402in}{3.933307in}}%
\pgfpathclose%
\pgfusepath{fill}%
\end{pgfscope}%
\begin{pgfscope}%
\definecolor{textcolor}{rgb}{0.000000,0.000000,0.000000}%
\pgfsetstrokecolor{textcolor}%
\pgfsetfillcolor{textcolor}%
\pgftext[x=1.295291in,y=3.836085in,left,base]{\color{textcolor}\rmfamily\fontsize{10.000000}{12.000000}\selectfont BF1 read at source}%
\end{pgfscope}%
\begin{pgfscope}%
\pgfsetbuttcap%
\pgfsetmiterjoin%
\definecolor{currentfill}{rgb}{0.549020,0.337255,0.294118}%
\pgfsetfillcolor{currentfill}%
\pgfsetlinewidth{0.000000pt}%
\definecolor{currentstroke}{rgb}{0.000000,0.000000,0.000000}%
\pgfsetstrokecolor{currentstroke}%
\pgfsetstrokeopacity{0.000000}%
\pgfsetdash{}{0pt}%
\pgfpathmoveto{\pgfqpoint{0.906402in}{3.642412in}}%
\pgfpathlineto{\pgfqpoint{1.184180in}{3.642412in}}%
\pgfpathlineto{\pgfqpoint{1.184180in}{3.739634in}}%
\pgfpathlineto{\pgfqpoint{0.906402in}{3.739634in}}%
\pgfpathclose%
\pgfusepath{fill}%
\end{pgfscope}%
\begin{pgfscope}%
\definecolor{textcolor}{rgb}{0.000000,0.000000,0.000000}%
\pgfsetstrokecolor{textcolor}%
\pgfsetfillcolor{textcolor}%
\pgftext[x=1.295291in,y=3.642412in,left,base]{\color{textcolor}\rmfamily\fontsize{10.000000}{12.000000}\selectfont BF1 write at source}%
\end{pgfscope}%
\end{pgfpicture}%
\makeatother%
\endgroup%

    \end{center}
    \caption{Migration timeline of a bloom filter (4KB pages)}
    \label{fig:bloomfilter}
\end{figure}


\begin{figure}[tp]
    \begin{center}
        %% Creator: Matplotlib, PGF backend
%%
%% To include the figure in your LaTeX document, write
%%   \input{<filename>.pgf}
%%
%% Make sure the required packages are loaded in your preamble
%%   \usepackage{pgf}
%%
%% and, on pdftex
%%   \usepackage[utf8]{inputenc}\DeclareUnicodeCharacter{2212}{-}
%%
%% or, on luatex and xetex
%%   \usepackage{unicode-math}
%%
%% Figures using additional raster images can only be included by \input if
%% they are in the same directory as the main LaTeX file. For loading figures
%% from other directories you can use the `import` package
%%   \usepackage{import}
%%
%% and then include the figures with
%%   \import{<path to file>}{<filename>.pgf}
%%
%% Matplotlib used the following preamble
%%
\begingroup%
\makeatletter%
\begin{pgfpicture}%
\pgfpathrectangle{\pgfpointorigin}{\pgfqpoint{6.251220in}{7.032623in}}%
\pgfusepath{use as bounding box, clip}%
\begin{pgfscope}%
\pgfsetbuttcap%
\pgfsetmiterjoin%
\definecolor{currentfill}{rgb}{1.000000,1.000000,1.000000}%
\pgfsetfillcolor{currentfill}%
\pgfsetlinewidth{0.000000pt}%
\definecolor{currentstroke}{rgb}{1.000000,1.000000,1.000000}%
\pgfsetstrokecolor{currentstroke}%
\pgfsetdash{}{0pt}%
\pgfpathmoveto{\pgfqpoint{0.000000in}{0.000000in}}%
\pgfpathlineto{\pgfqpoint{6.251220in}{0.000000in}}%
\pgfpathlineto{\pgfqpoint{6.251220in}{7.032623in}}%
\pgfpathlineto{\pgfqpoint{0.000000in}{7.032623in}}%
\pgfpathclose%
\pgfusepath{fill}%
\end{pgfscope}%
\begin{pgfscope}%
\pgfsetbuttcap%
\pgfsetmiterjoin%
\definecolor{currentfill}{rgb}{1.000000,1.000000,1.000000}%
\pgfsetfillcolor{currentfill}%
\pgfsetlinewidth{0.000000pt}%
\definecolor{currentstroke}{rgb}{0.000000,0.000000,0.000000}%
\pgfsetstrokecolor{currentstroke}%
\pgfsetstrokeopacity{0.000000}%
\pgfsetdash{}{0pt}%
\pgfpathmoveto{\pgfqpoint{0.781402in}{0.773588in}}%
\pgfpathlineto{\pgfqpoint{5.626098in}{0.773588in}}%
\pgfpathlineto{\pgfqpoint{5.626098in}{6.188708in}}%
\pgfpathlineto{\pgfqpoint{0.781402in}{6.188708in}}%
\pgfpathclose%
\pgfusepath{fill}%
\end{pgfscope}%
\begin{pgfscope}%
\pgfpathrectangle{\pgfqpoint{0.781402in}{0.773588in}}{\pgfqpoint{4.844695in}{5.415119in}}%
\pgfusepath{clip}%
\pgfsetbuttcap%
\pgfsetroundjoin%
\definecolor{currentfill}{rgb}{0.121569,0.466667,0.705882}%
\pgfsetfillcolor{currentfill}%
\pgfsetlinewidth{0.000000pt}%
\definecolor{currentstroke}{rgb}{0.000000,0.000000,0.000000}%
\pgfsetstrokecolor{currentstroke}%
\pgfsetdash{}{0pt}%
\pgfpathmoveto{\pgfqpoint{1.001616in}{0.773588in}}%
\pgfpathlineto{\pgfqpoint{1.001616in}{0.773588in}}%
\pgfpathlineto{\pgfqpoint{1.015150in}{0.773588in}}%
\pgfpathlineto{\pgfqpoint{1.028303in}{0.773588in}}%
\pgfpathlineto{\pgfqpoint{1.041322in}{0.773588in}}%
\pgfpathlineto{\pgfqpoint{1.054264in}{0.773588in}}%
\pgfpathlineto{\pgfqpoint{1.067143in}{0.773588in}}%
\pgfpathlineto{\pgfqpoint{1.079972in}{0.773588in}}%
\pgfpathlineto{\pgfqpoint{1.092853in}{0.773588in}}%
\pgfpathlineto{\pgfqpoint{1.105655in}{0.773588in}}%
\pgfpathlineto{\pgfqpoint{1.118303in}{0.773588in}}%
\pgfpathlineto{\pgfqpoint{1.131014in}{0.773588in}}%
\pgfpathlineto{\pgfqpoint{1.143621in}{0.773588in}}%
\pgfpathlineto{\pgfqpoint{1.156226in}{0.773588in}}%
\pgfpathlineto{\pgfqpoint{1.168924in}{0.773588in}}%
\pgfpathlineto{\pgfqpoint{1.181503in}{0.773588in}}%
\pgfpathlineto{\pgfqpoint{1.194024in}{0.773588in}}%
\pgfpathlineto{\pgfqpoint{1.206622in}{0.773588in}}%
\pgfpathlineto{\pgfqpoint{1.219183in}{0.773588in}}%
\pgfpathlineto{\pgfqpoint{1.231699in}{0.773588in}}%
\pgfpathlineto{\pgfqpoint{1.244104in}{0.773588in}}%
\pgfpathlineto{\pgfqpoint{1.256724in}{0.773588in}}%
\pgfpathlineto{\pgfqpoint{1.269255in}{0.773588in}}%
\pgfpathlineto{\pgfqpoint{1.281753in}{0.773588in}}%
\pgfpathlineto{\pgfqpoint{1.294165in}{0.773588in}}%
\pgfpathlineto{\pgfqpoint{1.306603in}{0.773588in}}%
\pgfpathlineto{\pgfqpoint{1.319073in}{0.773588in}}%
\pgfpathlineto{\pgfqpoint{1.331552in}{0.773588in}}%
\pgfpathlineto{\pgfqpoint{1.343969in}{0.773588in}}%
\pgfpathlineto{\pgfqpoint{1.356458in}{0.773588in}}%
\pgfpathlineto{\pgfqpoint{1.368992in}{0.773588in}}%
\pgfpathlineto{\pgfqpoint{1.381448in}{0.773588in}}%
\pgfpathlineto{\pgfqpoint{1.393906in}{0.773588in}}%
\pgfpathlineto{\pgfqpoint{1.406401in}{0.773588in}}%
\pgfpathlineto{\pgfqpoint{1.418893in}{0.773588in}}%
\pgfpathlineto{\pgfqpoint{1.431382in}{0.773588in}}%
\pgfpathlineto{\pgfqpoint{1.443756in}{0.773588in}}%
\pgfpathlineto{\pgfqpoint{1.456188in}{0.773588in}}%
\pgfpathlineto{\pgfqpoint{1.468660in}{0.773588in}}%
\pgfpathlineto{\pgfqpoint{1.481163in}{0.773588in}}%
\pgfpathlineto{\pgfqpoint{1.493578in}{0.773588in}}%
\pgfpathlineto{\pgfqpoint{1.506021in}{0.773588in}}%
\pgfpathlineto{\pgfqpoint{1.519060in}{0.773588in}}%
\pgfpathlineto{\pgfqpoint{1.531474in}{0.773588in}}%
\pgfpathlineto{\pgfqpoint{1.543839in}{0.773588in}}%
\pgfpathlineto{\pgfqpoint{1.556290in}{0.773588in}}%
\pgfpathlineto{\pgfqpoint{1.568672in}{0.773588in}}%
\pgfpathlineto{\pgfqpoint{1.581070in}{0.773588in}}%
\pgfpathlineto{\pgfqpoint{1.593430in}{0.773588in}}%
\pgfpathlineto{\pgfqpoint{1.605869in}{0.773588in}}%
\pgfpathlineto{\pgfqpoint{1.618336in}{0.773588in}}%
\pgfpathlineto{\pgfqpoint{1.630788in}{0.773588in}}%
\pgfpathlineto{\pgfqpoint{1.643249in}{0.773588in}}%
\pgfpathlineto{\pgfqpoint{1.655673in}{0.773588in}}%
\pgfpathlineto{\pgfqpoint{1.668187in}{0.773588in}}%
\pgfpathlineto{\pgfqpoint{1.680578in}{0.773588in}}%
\pgfpathlineto{\pgfqpoint{1.693005in}{0.773588in}}%
\pgfpathlineto{\pgfqpoint{1.705498in}{0.773588in}}%
\pgfpathlineto{\pgfqpoint{1.717832in}{0.773588in}}%
\pgfpathlineto{\pgfqpoint{1.730200in}{0.773588in}}%
\pgfpathlineto{\pgfqpoint{1.742593in}{0.773588in}}%
\pgfpathlineto{\pgfqpoint{1.755070in}{0.773588in}}%
\pgfpathlineto{\pgfqpoint{1.767480in}{0.773588in}}%
\pgfpathlineto{\pgfqpoint{1.779913in}{0.773588in}}%
\pgfpathlineto{\pgfqpoint{1.792355in}{0.773588in}}%
\pgfpathlineto{\pgfqpoint{1.804716in}{0.773588in}}%
\pgfpathlineto{\pgfqpoint{1.817148in}{0.773588in}}%
\pgfpathlineto{\pgfqpoint{1.829628in}{0.773588in}}%
\pgfpathlineto{\pgfqpoint{1.842065in}{0.773588in}}%
\pgfpathlineto{\pgfqpoint{1.854433in}{0.773588in}}%
\pgfpathlineto{\pgfqpoint{1.866747in}{0.773588in}}%
\pgfpathlineto{\pgfqpoint{1.879180in}{0.773588in}}%
\pgfpathlineto{\pgfqpoint{1.891620in}{0.773588in}}%
\pgfpathlineto{\pgfqpoint{1.904130in}{0.773588in}}%
\pgfpathlineto{\pgfqpoint{1.916611in}{0.773588in}}%
\pgfpathlineto{\pgfqpoint{1.929118in}{0.773588in}}%
\pgfpathlineto{\pgfqpoint{1.941620in}{0.773588in}}%
\pgfpathlineto{\pgfqpoint{1.954108in}{0.773588in}}%
\pgfpathlineto{\pgfqpoint{1.966664in}{0.773588in}}%
\pgfpathlineto{\pgfqpoint{1.979161in}{0.773588in}}%
\pgfpathlineto{\pgfqpoint{1.991646in}{0.773588in}}%
\pgfpathlineto{\pgfqpoint{2.004212in}{0.773588in}}%
\pgfpathlineto{\pgfqpoint{2.016889in}{0.773588in}}%
\pgfpathlineto{\pgfqpoint{2.030840in}{0.773588in}}%
\pgfpathlineto{\pgfqpoint{2.043350in}{0.773588in}}%
\pgfpathlineto{\pgfqpoint{2.055871in}{0.773588in}}%
\pgfpathlineto{\pgfqpoint{2.068360in}{0.773588in}}%
\pgfpathlineto{\pgfqpoint{2.080914in}{0.773588in}}%
\pgfpathlineto{\pgfqpoint{2.093434in}{0.773588in}}%
\pgfpathlineto{\pgfqpoint{2.106029in}{0.773588in}}%
\pgfpathlineto{\pgfqpoint{2.118500in}{0.773588in}}%
\pgfpathlineto{\pgfqpoint{2.130987in}{0.773588in}}%
\pgfpathlineto{\pgfqpoint{2.143512in}{0.773588in}}%
\pgfpathlineto{\pgfqpoint{2.156064in}{0.773588in}}%
\pgfpathlineto{\pgfqpoint{2.168659in}{0.773588in}}%
\pgfpathlineto{\pgfqpoint{2.181172in}{0.773588in}}%
\pgfpathlineto{\pgfqpoint{2.193673in}{0.773588in}}%
\pgfpathlineto{\pgfqpoint{2.206174in}{0.773588in}}%
\pgfpathlineto{\pgfqpoint{2.218726in}{0.773588in}}%
\pgfpathlineto{\pgfqpoint{2.231301in}{0.773588in}}%
\pgfpathlineto{\pgfqpoint{2.243791in}{0.773588in}}%
\pgfpathlineto{\pgfqpoint{2.256287in}{0.773588in}}%
\pgfpathlineto{\pgfqpoint{2.268761in}{0.773588in}}%
\pgfpathlineto{\pgfqpoint{2.281255in}{0.773588in}}%
\pgfpathlineto{\pgfqpoint{2.293873in}{0.773588in}}%
\pgfpathlineto{\pgfqpoint{2.306399in}{0.773588in}}%
\pgfpathlineto{\pgfqpoint{2.318891in}{0.773588in}}%
\pgfpathlineto{\pgfqpoint{2.331395in}{0.773588in}}%
\pgfpathlineto{\pgfqpoint{2.343909in}{0.773588in}}%
\pgfpathlineto{\pgfqpoint{2.356417in}{0.773588in}}%
\pgfpathlineto{\pgfqpoint{2.368880in}{0.773588in}}%
\pgfpathlineto{\pgfqpoint{2.381372in}{0.773588in}}%
\pgfpathlineto{\pgfqpoint{2.393733in}{0.773588in}}%
\pgfpathlineto{\pgfqpoint{2.406207in}{0.773588in}}%
\pgfpathlineto{\pgfqpoint{2.418633in}{0.773588in}}%
\pgfpathlineto{\pgfqpoint{2.433282in}{0.773588in}}%
\pgfpathlineto{\pgfqpoint{2.457197in}{0.773588in}}%
\pgfpathlineto{\pgfqpoint{2.477723in}{0.773588in}}%
\pgfpathlineto{\pgfqpoint{2.494737in}{0.773588in}}%
\pgfpathlineto{\pgfqpoint{2.515773in}{0.773588in}}%
\pgfpathlineto{\pgfqpoint{2.538352in}{0.773588in}}%
\pgfpathlineto{\pgfqpoint{2.558886in}{0.773588in}}%
\pgfpathlineto{\pgfqpoint{2.578521in}{0.773588in}}%
\pgfpathlineto{\pgfqpoint{2.598064in}{0.773588in}}%
\pgfpathlineto{\pgfqpoint{2.616149in}{0.773588in}}%
\pgfpathlineto{\pgfqpoint{2.636069in}{0.773588in}}%
\pgfpathlineto{\pgfqpoint{2.657274in}{0.773588in}}%
\pgfpathlineto{\pgfqpoint{2.676022in}{0.773588in}}%
\pgfpathlineto{\pgfqpoint{2.693540in}{0.773588in}}%
\pgfpathlineto{\pgfqpoint{2.710992in}{0.773588in}}%
\pgfpathlineto{\pgfqpoint{2.727722in}{0.773588in}}%
\pgfpathlineto{\pgfqpoint{2.744840in}{0.773588in}}%
\pgfpathlineto{\pgfqpoint{2.763118in}{0.773588in}}%
\pgfpathlineto{\pgfqpoint{2.780862in}{0.773588in}}%
\pgfpathlineto{\pgfqpoint{2.798917in}{0.773588in}}%
\pgfpathlineto{\pgfqpoint{2.815750in}{0.773588in}}%
\pgfpathlineto{\pgfqpoint{2.831972in}{0.773588in}}%
\pgfpathlineto{\pgfqpoint{2.848081in}{0.773588in}}%
\pgfpathlineto{\pgfqpoint{2.864455in}{0.773588in}}%
\pgfpathlineto{\pgfqpoint{2.880921in}{0.773588in}}%
\pgfpathlineto{\pgfqpoint{2.897563in}{0.773588in}}%
\pgfpathlineto{\pgfqpoint{2.914048in}{0.773588in}}%
\pgfpathlineto{\pgfqpoint{2.930775in}{0.773588in}}%
\pgfpathlineto{\pgfqpoint{2.949130in}{0.773588in}}%
\pgfpathlineto{\pgfqpoint{2.968433in}{0.773588in}}%
\pgfpathlineto{\pgfqpoint{2.985853in}{0.773588in}}%
\pgfpathlineto{\pgfqpoint{3.001921in}{0.773588in}}%
\pgfpathlineto{\pgfqpoint{3.018968in}{0.773588in}}%
\pgfpathlineto{\pgfqpoint{3.036276in}{0.773588in}}%
\pgfpathlineto{\pgfqpoint{3.052708in}{0.773588in}}%
\pgfpathlineto{\pgfqpoint{3.069003in}{0.773588in}}%
\pgfpathlineto{\pgfqpoint{3.085686in}{0.773588in}}%
\pgfpathlineto{\pgfqpoint{3.102316in}{0.773588in}}%
\pgfpathlineto{\pgfqpoint{3.118692in}{0.773588in}}%
\pgfpathlineto{\pgfqpoint{3.135319in}{0.773588in}}%
\pgfpathlineto{\pgfqpoint{3.152584in}{0.773588in}}%
\pgfpathlineto{\pgfqpoint{3.169651in}{0.773588in}}%
\pgfpathlineto{\pgfqpoint{3.185752in}{0.773588in}}%
\pgfpathlineto{\pgfqpoint{3.201602in}{0.773588in}}%
\pgfpathlineto{\pgfqpoint{3.217213in}{0.773588in}}%
\pgfpathlineto{\pgfqpoint{3.232763in}{0.773588in}}%
\pgfpathlineto{\pgfqpoint{3.248207in}{0.773588in}}%
\pgfpathlineto{\pgfqpoint{3.263722in}{0.773588in}}%
\pgfpathlineto{\pgfqpoint{3.279179in}{0.773588in}}%
\pgfpathlineto{\pgfqpoint{3.295281in}{0.773588in}}%
\pgfpathlineto{\pgfqpoint{3.313329in}{0.773588in}}%
\pgfpathlineto{\pgfqpoint{3.328928in}{0.773588in}}%
\pgfpathlineto{\pgfqpoint{3.344594in}{0.773588in}}%
\pgfpathlineto{\pgfqpoint{3.359992in}{0.773588in}}%
\pgfpathlineto{\pgfqpoint{3.375355in}{0.773588in}}%
\pgfpathlineto{\pgfqpoint{3.390722in}{0.773588in}}%
\pgfpathlineto{\pgfqpoint{3.405908in}{0.773588in}}%
\pgfpathlineto{\pgfqpoint{3.421182in}{0.773588in}}%
\pgfpathlineto{\pgfqpoint{3.436431in}{0.773588in}}%
\pgfpathlineto{\pgfqpoint{3.451777in}{0.773588in}}%
\pgfpathlineto{\pgfqpoint{3.467311in}{0.773588in}}%
\pgfpathlineto{\pgfqpoint{3.483019in}{0.773588in}}%
\pgfpathlineto{\pgfqpoint{3.498190in}{0.773588in}}%
\pgfpathlineto{\pgfqpoint{3.513323in}{0.773588in}}%
\pgfpathlineto{\pgfqpoint{3.528239in}{0.773588in}}%
\pgfpathlineto{\pgfqpoint{3.542041in}{0.773588in}}%
\pgfpathlineto{\pgfqpoint{3.554208in}{0.773588in}}%
\pgfpathlineto{\pgfqpoint{3.566335in}{0.773588in}}%
\pgfpathlineto{\pgfqpoint{3.578260in}{0.773588in}}%
\pgfpathlineto{\pgfqpoint{3.590359in}{0.773588in}}%
\pgfpathlineto{\pgfqpoint{3.602917in}{0.773588in}}%
\pgfpathlineto{\pgfqpoint{3.615064in}{0.773588in}}%
\pgfpathlineto{\pgfqpoint{3.627108in}{0.773588in}}%
\pgfpathlineto{\pgfqpoint{3.639052in}{0.773588in}}%
\pgfpathlineto{\pgfqpoint{3.651005in}{0.773588in}}%
\pgfpathlineto{\pgfqpoint{3.663361in}{0.773588in}}%
\pgfpathlineto{\pgfqpoint{3.675639in}{0.773588in}}%
\pgfpathlineto{\pgfqpoint{3.687677in}{0.773588in}}%
\pgfpathlineto{\pgfqpoint{3.699766in}{0.773588in}}%
\pgfpathlineto{\pgfqpoint{3.711790in}{0.773588in}}%
\pgfpathlineto{\pgfqpoint{3.723925in}{0.773588in}}%
\pgfpathlineto{\pgfqpoint{3.736218in}{0.773588in}}%
\pgfpathlineto{\pgfqpoint{3.748246in}{0.773588in}}%
\pgfpathlineto{\pgfqpoint{3.760252in}{0.773588in}}%
\pgfpathlineto{\pgfqpoint{3.772194in}{0.773588in}}%
\pgfpathlineto{\pgfqpoint{3.784213in}{0.773588in}}%
\pgfpathlineto{\pgfqpoint{3.796179in}{0.773588in}}%
\pgfpathlineto{\pgfqpoint{3.807700in}{0.773588in}}%
\pgfpathlineto{\pgfqpoint{3.819207in}{0.773588in}}%
\pgfpathlineto{\pgfqpoint{3.830748in}{0.773588in}}%
\pgfpathlineto{\pgfqpoint{3.842268in}{0.773588in}}%
\pgfpathlineto{\pgfqpoint{3.853773in}{0.773588in}}%
\pgfpathlineto{\pgfqpoint{3.865394in}{0.773588in}}%
\pgfpathlineto{\pgfqpoint{3.876999in}{0.773588in}}%
\pgfpathlineto{\pgfqpoint{3.888612in}{0.773588in}}%
\pgfpathlineto{\pgfqpoint{3.900120in}{0.773588in}}%
\pgfpathlineto{\pgfqpoint{3.911548in}{0.773588in}}%
\pgfpathlineto{\pgfqpoint{3.923007in}{0.773588in}}%
\pgfpathlineto{\pgfqpoint{3.934526in}{0.773588in}}%
\pgfpathlineto{\pgfqpoint{3.945995in}{0.773588in}}%
\pgfpathlineto{\pgfqpoint{3.957345in}{0.773588in}}%
\pgfpathlineto{\pgfqpoint{3.968800in}{0.773588in}}%
\pgfpathlineto{\pgfqpoint{3.980165in}{0.773588in}}%
\pgfpathlineto{\pgfqpoint{3.991607in}{0.773588in}}%
\pgfpathlineto{\pgfqpoint{4.003035in}{0.773588in}}%
\pgfpathlineto{\pgfqpoint{4.014312in}{0.773588in}}%
\pgfpathlineto{\pgfqpoint{4.025578in}{0.773588in}}%
\pgfpathlineto{\pgfqpoint{4.036918in}{0.773588in}}%
\pgfpathlineto{\pgfqpoint{4.048157in}{0.773588in}}%
\pgfpathlineto{\pgfqpoint{4.059512in}{0.773588in}}%
\pgfpathlineto{\pgfqpoint{4.070994in}{0.773588in}}%
\pgfpathlineto{\pgfqpoint{4.082264in}{0.773588in}}%
\pgfpathlineto{\pgfqpoint{4.093537in}{0.773588in}}%
\pgfpathlineto{\pgfqpoint{4.104772in}{0.773588in}}%
\pgfpathlineto{\pgfqpoint{4.116014in}{0.773588in}}%
\pgfpathlineto{\pgfqpoint{4.127322in}{0.773588in}}%
\pgfpathlineto{\pgfqpoint{4.138598in}{0.773588in}}%
\pgfpathlineto{\pgfqpoint{4.149813in}{0.773588in}}%
\pgfpathlineto{\pgfqpoint{4.160983in}{0.773588in}}%
\pgfpathlineto{\pgfqpoint{4.172101in}{0.773588in}}%
\pgfpathlineto{\pgfqpoint{4.183285in}{0.773588in}}%
\pgfpathlineto{\pgfqpoint{4.194400in}{0.773588in}}%
\pgfpathlineto{\pgfqpoint{4.205594in}{0.773588in}}%
\pgfpathlineto{\pgfqpoint{4.216619in}{0.773588in}}%
\pgfpathlineto{\pgfqpoint{4.227703in}{0.773588in}}%
\pgfpathlineto{\pgfqpoint{4.238844in}{0.773588in}}%
\pgfpathlineto{\pgfqpoint{4.249866in}{0.773588in}}%
\pgfpathlineto{\pgfqpoint{4.260400in}{0.773588in}}%
\pgfpathlineto{\pgfqpoint{4.270972in}{0.773588in}}%
\pgfpathlineto{\pgfqpoint{4.280743in}{0.773588in}}%
\pgfpathlineto{\pgfqpoint{4.290581in}{0.773588in}}%
\pgfpathlineto{\pgfqpoint{4.300486in}{0.773588in}}%
\pgfpathlineto{\pgfqpoint{4.310022in}{0.773588in}}%
\pgfpathlineto{\pgfqpoint{4.319388in}{0.773588in}}%
\pgfpathlineto{\pgfqpoint{4.328717in}{0.773588in}}%
\pgfpathlineto{\pgfqpoint{4.337933in}{0.773588in}}%
\pgfpathlineto{\pgfqpoint{4.346774in}{0.773588in}}%
\pgfpathlineto{\pgfqpoint{4.355507in}{0.773588in}}%
\pgfpathlineto{\pgfqpoint{4.364219in}{0.773588in}}%
\pgfpathlineto{\pgfqpoint{4.373001in}{0.773588in}}%
\pgfpathlineto{\pgfqpoint{4.381765in}{0.773588in}}%
\pgfpathlineto{\pgfqpoint{4.390451in}{0.773588in}}%
\pgfpathlineto{\pgfqpoint{4.399156in}{0.773588in}}%
\pgfpathlineto{\pgfqpoint{4.407811in}{0.773588in}}%
\pgfpathlineto{\pgfqpoint{4.416522in}{0.773588in}}%
\pgfpathlineto{\pgfqpoint{4.425248in}{0.773588in}}%
\pgfpathlineto{\pgfqpoint{4.433894in}{0.773588in}}%
\pgfpathlineto{\pgfqpoint{4.442591in}{0.773588in}}%
\pgfpathlineto{\pgfqpoint{4.451305in}{0.773588in}}%
\pgfpathlineto{\pgfqpoint{4.460011in}{0.773588in}}%
\pgfpathlineto{\pgfqpoint{4.468771in}{0.773588in}}%
\pgfpathlineto{\pgfqpoint{4.477466in}{0.773588in}}%
\pgfpathlineto{\pgfqpoint{4.486071in}{0.773588in}}%
\pgfpathlineto{\pgfqpoint{4.494752in}{0.773588in}}%
\pgfpathlineto{\pgfqpoint{4.503421in}{0.773588in}}%
\pgfpathlineto{\pgfqpoint{4.512084in}{0.773588in}}%
\pgfpathlineto{\pgfqpoint{4.520678in}{0.773588in}}%
\pgfpathlineto{\pgfqpoint{4.529276in}{0.773588in}}%
\pgfpathlineto{\pgfqpoint{4.537876in}{0.773588in}}%
\pgfpathlineto{\pgfqpoint{4.546439in}{0.773588in}}%
\pgfpathlineto{\pgfqpoint{4.554980in}{0.773588in}}%
\pgfpathlineto{\pgfqpoint{4.563550in}{0.773588in}}%
\pgfpathlineto{\pgfqpoint{4.572097in}{0.773588in}}%
\pgfpathlineto{\pgfqpoint{4.580727in}{0.773588in}}%
\pgfpathlineto{\pgfqpoint{4.589255in}{0.773588in}}%
\pgfpathlineto{\pgfqpoint{4.597719in}{0.773588in}}%
\pgfpathlineto{\pgfqpoint{4.606211in}{0.773588in}}%
\pgfpathlineto{\pgfqpoint{4.614783in}{0.773588in}}%
\pgfpathlineto{\pgfqpoint{4.623282in}{0.773588in}}%
\pgfpathlineto{\pgfqpoint{4.631733in}{0.773588in}}%
\pgfpathlineto{\pgfqpoint{4.640221in}{0.773588in}}%
\pgfpathlineto{\pgfqpoint{4.648847in}{0.773588in}}%
\pgfpathlineto{\pgfqpoint{4.657349in}{0.773588in}}%
\pgfpathlineto{\pgfqpoint{4.665825in}{0.773588in}}%
\pgfpathlineto{\pgfqpoint{4.674255in}{0.773588in}}%
\pgfpathlineto{\pgfqpoint{4.682643in}{0.773588in}}%
\pgfpathlineto{\pgfqpoint{4.691071in}{0.773588in}}%
\pgfpathlineto{\pgfqpoint{4.699465in}{0.773588in}}%
\pgfpathlineto{\pgfqpoint{4.707865in}{0.773588in}}%
\pgfpathlineto{\pgfqpoint{4.716247in}{0.773588in}}%
\pgfpathlineto{\pgfqpoint{4.724666in}{0.773588in}}%
\pgfpathlineto{\pgfqpoint{4.733049in}{0.773588in}}%
\pgfpathlineto{\pgfqpoint{4.741378in}{0.773588in}}%
\pgfpathlineto{\pgfqpoint{4.749750in}{0.773588in}}%
\pgfpathlineto{\pgfqpoint{4.758119in}{0.773588in}}%
\pgfpathlineto{\pgfqpoint{4.766528in}{0.773588in}}%
\pgfpathlineto{\pgfqpoint{4.774881in}{0.773588in}}%
\pgfpathlineto{\pgfqpoint{4.783272in}{0.773588in}}%
\pgfpathlineto{\pgfqpoint{4.791573in}{0.773588in}}%
\pgfpathlineto{\pgfqpoint{4.799850in}{0.773588in}}%
\pgfpathlineto{\pgfqpoint{4.808134in}{0.773588in}}%
\pgfpathlineto{\pgfqpoint{4.816444in}{0.773588in}}%
\pgfpathlineto{\pgfqpoint{4.824772in}{0.773588in}}%
\pgfpathlineto{\pgfqpoint{4.833061in}{0.773588in}}%
\pgfpathlineto{\pgfqpoint{4.841325in}{0.773588in}}%
\pgfpathlineto{\pgfqpoint{4.849575in}{0.773588in}}%
\pgfpathlineto{\pgfqpoint{4.857830in}{0.773588in}}%
\pgfpathlineto{\pgfqpoint{4.866069in}{0.773588in}}%
\pgfpathlineto{\pgfqpoint{4.874381in}{0.773588in}}%
\pgfpathlineto{\pgfqpoint{4.882658in}{0.773588in}}%
\pgfpathlineto{\pgfqpoint{4.890980in}{0.773588in}}%
\pgfpathlineto{\pgfqpoint{4.899205in}{0.773588in}}%
\pgfpathlineto{\pgfqpoint{4.907476in}{0.773588in}}%
\pgfpathlineto{\pgfqpoint{4.915736in}{0.773588in}}%
\pgfpathlineto{\pgfqpoint{4.924041in}{0.773588in}}%
\pgfpathlineto{\pgfqpoint{4.932347in}{0.773588in}}%
\pgfpathlineto{\pgfqpoint{4.940566in}{0.773588in}}%
\pgfpathlineto{\pgfqpoint{4.948774in}{0.773588in}}%
\pgfpathlineto{\pgfqpoint{4.956992in}{0.773588in}}%
\pgfpathlineto{\pgfqpoint{4.965138in}{0.773588in}}%
\pgfpathlineto{\pgfqpoint{4.973428in}{0.773588in}}%
\pgfpathlineto{\pgfqpoint{4.981642in}{0.773588in}}%
\pgfpathlineto{\pgfqpoint{4.989794in}{0.773588in}}%
\pgfpathlineto{\pgfqpoint{4.997948in}{0.773588in}}%
\pgfpathlineto{\pgfqpoint{5.006134in}{0.773588in}}%
\pgfpathlineto{\pgfqpoint{5.014317in}{0.773588in}}%
\pgfpathlineto{\pgfqpoint{5.022451in}{0.773588in}}%
\pgfpathlineto{\pgfqpoint{5.030602in}{0.773588in}}%
\pgfpathlineto{\pgfqpoint{5.038759in}{0.773588in}}%
\pgfpathlineto{\pgfqpoint{5.046876in}{0.773588in}}%
\pgfpathlineto{\pgfqpoint{5.055031in}{0.773588in}}%
\pgfpathlineto{\pgfqpoint{5.063216in}{0.773588in}}%
\pgfpathlineto{\pgfqpoint{5.071391in}{0.773588in}}%
\pgfpathlineto{\pgfqpoint{5.079506in}{0.773588in}}%
\pgfpathlineto{\pgfqpoint{5.087675in}{0.773588in}}%
\pgfpathlineto{\pgfqpoint{5.095819in}{0.773588in}}%
\pgfpathlineto{\pgfqpoint{5.103859in}{0.773588in}}%
\pgfpathlineto{\pgfqpoint{5.111982in}{0.773588in}}%
\pgfpathlineto{\pgfqpoint{5.120128in}{0.773588in}}%
\pgfpathlineto{\pgfqpoint{5.128231in}{0.773588in}}%
\pgfpathlineto{\pgfqpoint{5.136344in}{0.773588in}}%
\pgfpathlineto{\pgfqpoint{5.144370in}{0.773588in}}%
\pgfpathlineto{\pgfqpoint{5.152426in}{0.773588in}}%
\pgfpathlineto{\pgfqpoint{5.160447in}{0.773588in}}%
\pgfpathlineto{\pgfqpoint{5.168547in}{0.773588in}}%
\pgfpathlineto{\pgfqpoint{5.176647in}{0.773588in}}%
\pgfpathlineto{\pgfqpoint{5.184747in}{0.773588in}}%
\pgfpathlineto{\pgfqpoint{5.192761in}{0.773588in}}%
\pgfpathlineto{\pgfqpoint{5.200810in}{0.773588in}}%
\pgfpathlineto{\pgfqpoint{5.208810in}{0.773588in}}%
\pgfpathlineto{\pgfqpoint{5.216856in}{0.773588in}}%
\pgfpathlineto{\pgfqpoint{5.224894in}{0.773588in}}%
\pgfpathlineto{\pgfqpoint{5.233040in}{0.773588in}}%
\pgfpathlineto{\pgfqpoint{5.245361in}{0.773588in}}%
\pgfpathlineto{\pgfqpoint{5.253575in}{0.773588in}}%
\pgfpathlineto{\pgfqpoint{5.261632in}{0.773588in}}%
\pgfpathlineto{\pgfqpoint{5.269666in}{0.773588in}}%
\pgfpathlineto{\pgfqpoint{5.277683in}{0.773588in}}%
\pgfpathlineto{\pgfqpoint{5.285748in}{0.773588in}}%
\pgfpathlineto{\pgfqpoint{5.293795in}{0.773588in}}%
\pgfpathlineto{\pgfqpoint{5.301824in}{0.773588in}}%
\pgfpathlineto{\pgfqpoint{5.309888in}{0.773588in}}%
\pgfpathlineto{\pgfqpoint{5.317933in}{0.773588in}}%
\pgfpathlineto{\pgfqpoint{5.327487in}{0.773588in}}%
\pgfpathlineto{\pgfqpoint{5.338668in}{0.773588in}}%
\pgfpathlineto{\pgfqpoint{5.349820in}{0.773588in}}%
\pgfpathlineto{\pgfqpoint{5.360970in}{0.773588in}}%
\pgfpathlineto{\pgfqpoint{5.372247in}{0.773588in}}%
\pgfpathlineto{\pgfqpoint{5.383480in}{0.773588in}}%
\pgfpathlineto{\pgfqpoint{5.394653in}{0.773588in}}%
\pgfpathlineto{\pgfqpoint{5.405885in}{0.773588in}}%
\pgfpathlineto{\pgfqpoint{5.405885in}{2.078279in}}%
\pgfpathlineto{\pgfqpoint{5.405885in}{2.078279in}}%
\pgfpathlineto{\pgfqpoint{5.394653in}{2.078279in}}%
\pgfpathlineto{\pgfqpoint{5.383480in}{2.078279in}}%
\pgfpathlineto{\pgfqpoint{5.372247in}{2.078279in}}%
\pgfpathlineto{\pgfqpoint{5.360970in}{2.078279in}}%
\pgfpathlineto{\pgfqpoint{5.349820in}{2.078279in}}%
\pgfpathlineto{\pgfqpoint{5.338668in}{2.078279in}}%
\pgfpathlineto{\pgfqpoint{5.327487in}{2.078279in}}%
\pgfpathlineto{\pgfqpoint{5.317933in}{2.013298in}}%
\pgfpathlineto{\pgfqpoint{5.309888in}{2.004593in}}%
\pgfpathlineto{\pgfqpoint{5.301824in}{2.021439in}}%
\pgfpathlineto{\pgfqpoint{5.293795in}{2.002070in}}%
\pgfpathlineto{\pgfqpoint{5.285748in}{2.011043in}}%
\pgfpathlineto{\pgfqpoint{5.277683in}{2.012220in}}%
\pgfpathlineto{\pgfqpoint{5.269666in}{1.992287in}}%
\pgfpathlineto{\pgfqpoint{5.261632in}{2.004198in}}%
\pgfpathlineto{\pgfqpoint{5.253575in}{1.997690in}}%
\pgfpathlineto{\pgfqpoint{5.245361in}{2.019855in}}%
\pgfpathlineto{\pgfqpoint{5.233040in}{2.017139in}}%
\pgfpathlineto{\pgfqpoint{5.224894in}{2.006818in}}%
\pgfpathlineto{\pgfqpoint{5.216856in}{2.017889in}}%
\pgfpathlineto{\pgfqpoint{5.208810in}{2.025945in}}%
\pgfpathlineto{\pgfqpoint{5.200810in}{2.003568in}}%
\pgfpathlineto{\pgfqpoint{5.192761in}{2.016126in}}%
\pgfpathlineto{\pgfqpoint{5.184747in}{2.024057in}}%
\pgfpathlineto{\pgfqpoint{5.176647in}{1.992237in}}%
\pgfpathlineto{\pgfqpoint{5.168547in}{2.008400in}}%
\pgfpathlineto{\pgfqpoint{5.160447in}{2.011538in}}%
\pgfpathlineto{\pgfqpoint{5.152426in}{1.988997in}}%
\pgfpathlineto{\pgfqpoint{5.144370in}{1.987290in}}%
\pgfpathlineto{\pgfqpoint{5.136344in}{2.024205in}}%
\pgfpathlineto{\pgfqpoint{5.128231in}{2.001004in}}%
\pgfpathlineto{\pgfqpoint{5.120128in}{2.016296in}}%
\pgfpathlineto{\pgfqpoint{5.111982in}{2.006399in}}%
\pgfpathlineto{\pgfqpoint{5.103859in}{2.013410in}}%
\pgfpathlineto{\pgfqpoint{5.095819in}{2.018341in}}%
\pgfpathlineto{\pgfqpoint{5.087675in}{2.014413in}}%
\pgfpathlineto{\pgfqpoint{5.079506in}{1.994744in}}%
\pgfpathlineto{\pgfqpoint{5.071391in}{1.995421in}}%
\pgfpathlineto{\pgfqpoint{5.063216in}{2.010850in}}%
\pgfpathlineto{\pgfqpoint{5.055031in}{2.019743in}}%
\pgfpathlineto{\pgfqpoint{5.046876in}{1.994952in}}%
\pgfpathlineto{\pgfqpoint{5.038759in}{2.006790in}}%
\pgfpathlineto{\pgfqpoint{5.030602in}{2.014069in}}%
\pgfpathlineto{\pgfqpoint{5.022451in}{2.017700in}}%
\pgfpathlineto{\pgfqpoint{5.014317in}{1.997604in}}%
\pgfpathlineto{\pgfqpoint{5.006134in}{2.000368in}}%
\pgfpathlineto{\pgfqpoint{4.997948in}{2.023285in}}%
\pgfpathlineto{\pgfqpoint{4.989794in}{2.006792in}}%
\pgfpathlineto{\pgfqpoint{4.981642in}{1.996711in}}%
\pgfpathlineto{\pgfqpoint{4.973428in}{2.002030in}}%
\pgfpathlineto{\pgfqpoint{4.965138in}{1.984777in}}%
\pgfpathlineto{\pgfqpoint{4.956992in}{2.006114in}}%
\pgfpathlineto{\pgfqpoint{4.948774in}{2.012384in}}%
\pgfpathlineto{\pgfqpoint{4.940566in}{2.001328in}}%
\pgfpathlineto{\pgfqpoint{4.932347in}{2.019887in}}%
\pgfpathlineto{\pgfqpoint{4.924041in}{2.002932in}}%
\pgfpathlineto{\pgfqpoint{4.915736in}{2.024372in}}%
\pgfpathlineto{\pgfqpoint{4.907476in}{2.008683in}}%
\pgfpathlineto{\pgfqpoint{4.899205in}{2.033603in}}%
\pgfpathlineto{\pgfqpoint{4.890980in}{2.022830in}}%
\pgfpathlineto{\pgfqpoint{4.882658in}{2.008392in}}%
\pgfpathlineto{\pgfqpoint{4.874381in}{2.005115in}}%
\pgfpathlineto{\pgfqpoint{4.866069in}{2.020263in}}%
\pgfpathlineto{\pgfqpoint{4.857830in}{2.029836in}}%
\pgfpathlineto{\pgfqpoint{4.849575in}{2.001721in}}%
\pgfpathlineto{\pgfqpoint{4.841325in}{2.028311in}}%
\pgfpathlineto{\pgfqpoint{4.833061in}{2.010650in}}%
\pgfpathlineto{\pgfqpoint{4.824772in}{2.010632in}}%
\pgfpathlineto{\pgfqpoint{4.816444in}{2.007052in}}%
\pgfpathlineto{\pgfqpoint{4.808134in}{2.012887in}}%
\pgfpathlineto{\pgfqpoint{4.799850in}{2.012329in}}%
\pgfpathlineto{\pgfqpoint{4.791573in}{2.008402in}}%
\pgfpathlineto{\pgfqpoint{4.783272in}{2.005253in}}%
\pgfpathlineto{\pgfqpoint{4.774881in}{2.016107in}}%
\pgfpathlineto{\pgfqpoint{4.766528in}{2.008808in}}%
\pgfpathlineto{\pgfqpoint{4.758119in}{2.005010in}}%
\pgfpathlineto{\pgfqpoint{4.749750in}{2.007352in}}%
\pgfpathlineto{\pgfqpoint{4.741378in}{1.994153in}}%
\pgfpathlineto{\pgfqpoint{4.733049in}{2.015841in}}%
\pgfpathlineto{\pgfqpoint{4.724666in}{1.995905in}}%
\pgfpathlineto{\pgfqpoint{4.716247in}{2.016652in}}%
\pgfpathlineto{\pgfqpoint{4.707865in}{2.005723in}}%
\pgfpathlineto{\pgfqpoint{4.699465in}{2.021339in}}%
\pgfpathlineto{\pgfqpoint{4.691071in}{2.016009in}}%
\pgfpathlineto{\pgfqpoint{4.682643in}{2.003883in}}%
\pgfpathlineto{\pgfqpoint{4.674255in}{1.990279in}}%
\pgfpathlineto{\pgfqpoint{4.665825in}{1.968337in}}%
\pgfpathlineto{\pgfqpoint{4.657349in}{2.000882in}}%
\pgfpathlineto{\pgfqpoint{4.648847in}{1.949582in}}%
\pgfpathlineto{\pgfqpoint{4.640221in}{1.967260in}}%
\pgfpathlineto{\pgfqpoint{4.631733in}{1.996046in}}%
\pgfpathlineto{\pgfqpoint{4.623282in}{2.003276in}}%
\pgfpathlineto{\pgfqpoint{4.614783in}{1.991947in}}%
\pgfpathlineto{\pgfqpoint{4.606211in}{1.975203in}}%
\pgfpathlineto{\pgfqpoint{4.597719in}{2.003030in}}%
\pgfpathlineto{\pgfqpoint{4.589255in}{1.996773in}}%
\pgfpathlineto{\pgfqpoint{4.580727in}{1.987710in}}%
\pgfpathlineto{\pgfqpoint{4.572097in}{1.977375in}}%
\pgfpathlineto{\pgfqpoint{4.563550in}{1.976320in}}%
\pgfpathlineto{\pgfqpoint{4.554980in}{1.981305in}}%
\pgfpathlineto{\pgfqpoint{4.546439in}{1.990194in}}%
\pgfpathlineto{\pgfqpoint{4.537876in}{1.991824in}}%
\pgfpathlineto{\pgfqpoint{4.529276in}{1.962817in}}%
\pgfpathlineto{\pgfqpoint{4.520678in}{1.983090in}}%
\pgfpathlineto{\pgfqpoint{4.512084in}{1.961443in}}%
\pgfpathlineto{\pgfqpoint{4.503421in}{1.972271in}}%
\pgfpathlineto{\pgfqpoint{4.494752in}{1.961054in}}%
\pgfpathlineto{\pgfqpoint{4.486071in}{1.980098in}}%
\pgfpathlineto{\pgfqpoint{4.477466in}{1.976629in}}%
\pgfpathlineto{\pgfqpoint{4.468771in}{1.959528in}}%
\pgfpathlineto{\pgfqpoint{4.460011in}{1.975687in}}%
\pgfpathlineto{\pgfqpoint{4.451305in}{1.951847in}}%
\pgfpathlineto{\pgfqpoint{4.442591in}{1.963719in}}%
\pgfpathlineto{\pgfqpoint{4.433894in}{1.966871in}}%
\pgfpathlineto{\pgfqpoint{4.425248in}{1.939798in}}%
\pgfpathlineto{\pgfqpoint{4.416522in}{1.958386in}}%
\pgfpathlineto{\pgfqpoint{4.407811in}{1.957148in}}%
\pgfpathlineto{\pgfqpoint{4.399156in}{1.959803in}}%
\pgfpathlineto{\pgfqpoint{4.390451in}{1.949235in}}%
\pgfpathlineto{\pgfqpoint{4.381765in}{1.976384in}}%
\pgfpathlineto{\pgfqpoint{4.373001in}{1.913286in}}%
\pgfpathlineto{\pgfqpoint{4.364219in}{1.927117in}}%
\pgfpathlineto{\pgfqpoint{4.355507in}{1.934075in}}%
\pgfpathlineto{\pgfqpoint{4.346774in}{1.927275in}}%
\pgfpathlineto{\pgfqpoint{4.337933in}{1.902762in}}%
\pgfpathlineto{\pgfqpoint{4.328717in}{1.840369in}}%
\pgfpathlineto{\pgfqpoint{4.319388in}{1.885959in}}%
\pgfpathlineto{\pgfqpoint{4.310022in}{1.842558in}}%
\pgfpathlineto{\pgfqpoint{4.300486in}{1.905912in}}%
\pgfpathlineto{\pgfqpoint{4.290581in}{1.840707in}}%
\pgfpathlineto{\pgfqpoint{4.280743in}{1.858238in}}%
\pgfpathlineto{\pgfqpoint{4.270972in}{1.366989in}}%
\pgfpathlineto{\pgfqpoint{4.260400in}{1.712068in}}%
\pgfpathlineto{\pgfqpoint{4.249866in}{0.923295in}}%
\pgfpathlineto{\pgfqpoint{4.238844in}{0.874109in}}%
\pgfpathlineto{\pgfqpoint{4.227703in}{0.908748in}}%
\pgfpathlineto{\pgfqpoint{4.216619in}{0.834565in}}%
\pgfpathlineto{\pgfqpoint{4.205594in}{0.820220in}}%
\pgfpathlineto{\pgfqpoint{4.194400in}{0.827586in}}%
\pgfpathlineto{\pgfqpoint{4.183285in}{0.858315in}}%
\pgfpathlineto{\pgfqpoint{4.172101in}{0.902780in}}%
\pgfpathlineto{\pgfqpoint{4.160983in}{0.805638in}}%
\pgfpathlineto{\pgfqpoint{4.149813in}{0.797653in}}%
\pgfpathlineto{\pgfqpoint{4.138598in}{0.797025in}}%
\pgfpathlineto{\pgfqpoint{4.127322in}{0.795009in}}%
\pgfpathlineto{\pgfqpoint{4.116014in}{0.793664in}}%
\pgfpathlineto{\pgfqpoint{4.104772in}{0.795244in}}%
\pgfpathlineto{\pgfqpoint{4.093537in}{0.797517in}}%
\pgfpathlineto{\pgfqpoint{4.082264in}{0.797517in}}%
\pgfpathlineto{\pgfqpoint{4.070994in}{0.793986in}}%
\pgfpathlineto{\pgfqpoint{4.059512in}{0.793411in}}%
\pgfpathlineto{\pgfqpoint{4.048157in}{0.784374in}}%
\pgfpathlineto{\pgfqpoint{4.036918in}{0.781674in}}%
\pgfpathlineto{\pgfqpoint{4.025578in}{0.780138in}}%
\pgfpathlineto{\pgfqpoint{4.014312in}{0.773588in}}%
\pgfpathlineto{\pgfqpoint{4.003035in}{0.773588in}}%
\pgfpathlineto{\pgfqpoint{3.991607in}{0.773588in}}%
\pgfpathlineto{\pgfqpoint{3.980165in}{0.773588in}}%
\pgfpathlineto{\pgfqpoint{3.968800in}{0.773588in}}%
\pgfpathlineto{\pgfqpoint{3.957345in}{0.773588in}}%
\pgfpathlineto{\pgfqpoint{3.945995in}{0.773588in}}%
\pgfpathlineto{\pgfqpoint{3.934526in}{0.773588in}}%
\pgfpathlineto{\pgfqpoint{3.923007in}{0.773588in}}%
\pgfpathlineto{\pgfqpoint{3.911548in}{0.773588in}}%
\pgfpathlineto{\pgfqpoint{3.900120in}{0.773588in}}%
\pgfpathlineto{\pgfqpoint{3.888612in}{0.773588in}}%
\pgfpathlineto{\pgfqpoint{3.876999in}{0.773588in}}%
\pgfpathlineto{\pgfqpoint{3.865394in}{0.773588in}}%
\pgfpathlineto{\pgfqpoint{3.853773in}{0.773588in}}%
\pgfpathlineto{\pgfqpoint{3.842268in}{0.773588in}}%
\pgfpathlineto{\pgfqpoint{3.830748in}{0.773588in}}%
\pgfpathlineto{\pgfqpoint{3.819207in}{0.773588in}}%
\pgfpathlineto{\pgfqpoint{3.807700in}{0.773588in}}%
\pgfpathlineto{\pgfqpoint{3.796179in}{0.773588in}}%
\pgfpathlineto{\pgfqpoint{3.784213in}{0.773588in}}%
\pgfpathlineto{\pgfqpoint{3.772194in}{0.773588in}}%
\pgfpathlineto{\pgfqpoint{3.760252in}{0.773588in}}%
\pgfpathlineto{\pgfqpoint{3.748246in}{0.773588in}}%
\pgfpathlineto{\pgfqpoint{3.736218in}{0.773588in}}%
\pgfpathlineto{\pgfqpoint{3.723925in}{0.773588in}}%
\pgfpathlineto{\pgfqpoint{3.711790in}{0.773588in}}%
\pgfpathlineto{\pgfqpoint{3.699766in}{0.773588in}}%
\pgfpathlineto{\pgfqpoint{3.687677in}{0.773588in}}%
\pgfpathlineto{\pgfqpoint{3.675639in}{0.773588in}}%
\pgfpathlineto{\pgfqpoint{3.663361in}{0.773588in}}%
\pgfpathlineto{\pgfqpoint{3.651005in}{0.773588in}}%
\pgfpathlineto{\pgfqpoint{3.639052in}{0.773588in}}%
\pgfpathlineto{\pgfqpoint{3.627108in}{0.773588in}}%
\pgfpathlineto{\pgfqpoint{3.615064in}{0.773588in}}%
\pgfpathlineto{\pgfqpoint{3.602917in}{0.773588in}}%
\pgfpathlineto{\pgfqpoint{3.590359in}{0.773588in}}%
\pgfpathlineto{\pgfqpoint{3.578260in}{0.773588in}}%
\pgfpathlineto{\pgfqpoint{3.566335in}{0.773588in}}%
\pgfpathlineto{\pgfqpoint{3.554208in}{0.773588in}}%
\pgfpathlineto{\pgfqpoint{3.542041in}{0.773588in}}%
\pgfpathlineto{\pgfqpoint{3.528239in}{0.773588in}}%
\pgfpathlineto{\pgfqpoint{3.513323in}{0.773588in}}%
\pgfpathlineto{\pgfqpoint{3.498190in}{0.773588in}}%
\pgfpathlineto{\pgfqpoint{3.483019in}{0.773588in}}%
\pgfpathlineto{\pgfqpoint{3.467311in}{0.773588in}}%
\pgfpathlineto{\pgfqpoint{3.451777in}{0.773588in}}%
\pgfpathlineto{\pgfqpoint{3.436431in}{0.773588in}}%
\pgfpathlineto{\pgfqpoint{3.421182in}{0.773588in}}%
\pgfpathlineto{\pgfqpoint{3.405908in}{0.773588in}}%
\pgfpathlineto{\pgfqpoint{3.390722in}{0.773588in}}%
\pgfpathlineto{\pgfqpoint{3.375355in}{0.773588in}}%
\pgfpathlineto{\pgfqpoint{3.359992in}{0.773588in}}%
\pgfpathlineto{\pgfqpoint{3.344594in}{0.773588in}}%
\pgfpathlineto{\pgfqpoint{3.328928in}{0.773588in}}%
\pgfpathlineto{\pgfqpoint{3.313329in}{0.773588in}}%
\pgfpathlineto{\pgfqpoint{3.295281in}{0.773588in}}%
\pgfpathlineto{\pgfqpoint{3.279179in}{0.773588in}}%
\pgfpathlineto{\pgfqpoint{3.263722in}{0.773588in}}%
\pgfpathlineto{\pgfqpoint{3.248207in}{0.773588in}}%
\pgfpathlineto{\pgfqpoint{3.232763in}{0.773588in}}%
\pgfpathlineto{\pgfqpoint{3.217213in}{0.773588in}}%
\pgfpathlineto{\pgfqpoint{3.201602in}{0.773588in}}%
\pgfpathlineto{\pgfqpoint{3.185752in}{0.773588in}}%
\pgfpathlineto{\pgfqpoint{3.169651in}{0.773588in}}%
\pgfpathlineto{\pgfqpoint{3.152584in}{0.773588in}}%
\pgfpathlineto{\pgfqpoint{3.135319in}{0.773588in}}%
\pgfpathlineto{\pgfqpoint{3.118692in}{0.773588in}}%
\pgfpathlineto{\pgfqpoint{3.102316in}{0.773588in}}%
\pgfpathlineto{\pgfqpoint{3.085686in}{0.773588in}}%
\pgfpathlineto{\pgfqpoint{3.069003in}{0.773588in}}%
\pgfpathlineto{\pgfqpoint{3.052708in}{0.773588in}}%
\pgfpathlineto{\pgfqpoint{3.036276in}{0.773588in}}%
\pgfpathlineto{\pgfqpoint{3.018968in}{0.773588in}}%
\pgfpathlineto{\pgfqpoint{3.001921in}{0.773588in}}%
\pgfpathlineto{\pgfqpoint{2.985853in}{0.773588in}}%
\pgfpathlineto{\pgfqpoint{2.968433in}{0.773588in}}%
\pgfpathlineto{\pgfqpoint{2.949130in}{0.773588in}}%
\pgfpathlineto{\pgfqpoint{2.930775in}{0.773588in}}%
\pgfpathlineto{\pgfqpoint{2.914048in}{0.773588in}}%
\pgfpathlineto{\pgfqpoint{2.897563in}{0.773588in}}%
\pgfpathlineto{\pgfqpoint{2.880921in}{0.773588in}}%
\pgfpathlineto{\pgfqpoint{2.864455in}{0.773588in}}%
\pgfpathlineto{\pgfqpoint{2.848081in}{0.773588in}}%
\pgfpathlineto{\pgfqpoint{2.831972in}{0.773588in}}%
\pgfpathlineto{\pgfqpoint{2.815750in}{0.773588in}}%
\pgfpathlineto{\pgfqpoint{2.798917in}{0.773588in}}%
\pgfpathlineto{\pgfqpoint{2.780862in}{0.773588in}}%
\pgfpathlineto{\pgfqpoint{2.763118in}{0.773588in}}%
\pgfpathlineto{\pgfqpoint{2.744840in}{0.773588in}}%
\pgfpathlineto{\pgfqpoint{2.727722in}{0.773588in}}%
\pgfpathlineto{\pgfqpoint{2.710992in}{0.773588in}}%
\pgfpathlineto{\pgfqpoint{2.693540in}{0.773588in}}%
\pgfpathlineto{\pgfqpoint{2.676022in}{0.773588in}}%
\pgfpathlineto{\pgfqpoint{2.657274in}{0.773588in}}%
\pgfpathlineto{\pgfqpoint{2.636069in}{0.773588in}}%
\pgfpathlineto{\pgfqpoint{2.616149in}{0.773588in}}%
\pgfpathlineto{\pgfqpoint{2.598064in}{0.773588in}}%
\pgfpathlineto{\pgfqpoint{2.578521in}{0.773588in}}%
\pgfpathlineto{\pgfqpoint{2.558886in}{0.773588in}}%
\pgfpathlineto{\pgfqpoint{2.538352in}{0.773588in}}%
\pgfpathlineto{\pgfqpoint{2.515773in}{0.773588in}}%
\pgfpathlineto{\pgfqpoint{2.494737in}{0.773588in}}%
\pgfpathlineto{\pgfqpoint{2.477723in}{0.773588in}}%
\pgfpathlineto{\pgfqpoint{2.457197in}{0.773588in}}%
\pgfpathlineto{\pgfqpoint{2.433282in}{0.773588in}}%
\pgfpathlineto{\pgfqpoint{2.418633in}{0.773588in}}%
\pgfpathlineto{\pgfqpoint{2.406207in}{0.773588in}}%
\pgfpathlineto{\pgfqpoint{2.393733in}{0.773588in}}%
\pgfpathlineto{\pgfqpoint{2.381372in}{0.773588in}}%
\pgfpathlineto{\pgfqpoint{2.368880in}{0.773588in}}%
\pgfpathlineto{\pgfqpoint{2.356417in}{0.773588in}}%
\pgfpathlineto{\pgfqpoint{2.343909in}{0.773588in}}%
\pgfpathlineto{\pgfqpoint{2.331395in}{0.773588in}}%
\pgfpathlineto{\pgfqpoint{2.318891in}{0.773588in}}%
\pgfpathlineto{\pgfqpoint{2.306399in}{0.773588in}}%
\pgfpathlineto{\pgfqpoint{2.293873in}{0.773588in}}%
\pgfpathlineto{\pgfqpoint{2.281255in}{0.773588in}}%
\pgfpathlineto{\pgfqpoint{2.268761in}{0.773588in}}%
\pgfpathlineto{\pgfqpoint{2.256287in}{0.773588in}}%
\pgfpathlineto{\pgfqpoint{2.243791in}{0.773588in}}%
\pgfpathlineto{\pgfqpoint{2.231301in}{0.773588in}}%
\pgfpathlineto{\pgfqpoint{2.218726in}{0.773588in}}%
\pgfpathlineto{\pgfqpoint{2.206174in}{0.773588in}}%
\pgfpathlineto{\pgfqpoint{2.193673in}{0.773588in}}%
\pgfpathlineto{\pgfqpoint{2.181172in}{0.773588in}}%
\pgfpathlineto{\pgfqpoint{2.168659in}{0.773588in}}%
\pgfpathlineto{\pgfqpoint{2.156064in}{0.773588in}}%
\pgfpathlineto{\pgfqpoint{2.143512in}{0.773588in}}%
\pgfpathlineto{\pgfqpoint{2.130987in}{0.773588in}}%
\pgfpathlineto{\pgfqpoint{2.118500in}{0.773588in}}%
\pgfpathlineto{\pgfqpoint{2.106029in}{0.773588in}}%
\pgfpathlineto{\pgfqpoint{2.093434in}{0.773588in}}%
\pgfpathlineto{\pgfqpoint{2.080914in}{0.773588in}}%
\pgfpathlineto{\pgfqpoint{2.068360in}{0.773588in}}%
\pgfpathlineto{\pgfqpoint{2.055871in}{0.773588in}}%
\pgfpathlineto{\pgfqpoint{2.043350in}{0.773588in}}%
\pgfpathlineto{\pgfqpoint{2.030840in}{0.773588in}}%
\pgfpathlineto{\pgfqpoint{2.016889in}{0.773588in}}%
\pgfpathlineto{\pgfqpoint{2.004212in}{0.773588in}}%
\pgfpathlineto{\pgfqpoint{1.991646in}{0.773588in}}%
\pgfpathlineto{\pgfqpoint{1.979161in}{0.773588in}}%
\pgfpathlineto{\pgfqpoint{1.966664in}{0.773588in}}%
\pgfpathlineto{\pgfqpoint{1.954108in}{0.773588in}}%
\pgfpathlineto{\pgfqpoint{1.941620in}{0.773588in}}%
\pgfpathlineto{\pgfqpoint{1.929118in}{0.773588in}}%
\pgfpathlineto{\pgfqpoint{1.916611in}{0.773588in}}%
\pgfpathlineto{\pgfqpoint{1.904130in}{0.773588in}}%
\pgfpathlineto{\pgfqpoint{1.891620in}{0.773588in}}%
\pgfpathlineto{\pgfqpoint{1.879180in}{0.773588in}}%
\pgfpathlineto{\pgfqpoint{1.866747in}{0.773588in}}%
\pgfpathlineto{\pgfqpoint{1.854433in}{0.773588in}}%
\pgfpathlineto{\pgfqpoint{1.842065in}{0.773588in}}%
\pgfpathlineto{\pgfqpoint{1.829628in}{0.773588in}}%
\pgfpathlineto{\pgfqpoint{1.817148in}{0.773588in}}%
\pgfpathlineto{\pgfqpoint{1.804716in}{0.773588in}}%
\pgfpathlineto{\pgfqpoint{1.792355in}{0.773588in}}%
\pgfpathlineto{\pgfqpoint{1.779913in}{0.773588in}}%
\pgfpathlineto{\pgfqpoint{1.767480in}{0.773588in}}%
\pgfpathlineto{\pgfqpoint{1.755070in}{0.773588in}}%
\pgfpathlineto{\pgfqpoint{1.742593in}{0.773588in}}%
\pgfpathlineto{\pgfqpoint{1.730200in}{0.773588in}}%
\pgfpathlineto{\pgfqpoint{1.717832in}{0.773588in}}%
\pgfpathlineto{\pgfqpoint{1.705498in}{0.773588in}}%
\pgfpathlineto{\pgfqpoint{1.693005in}{0.773588in}}%
\pgfpathlineto{\pgfqpoint{1.680578in}{0.773588in}}%
\pgfpathlineto{\pgfqpoint{1.668187in}{0.773588in}}%
\pgfpathlineto{\pgfqpoint{1.655673in}{0.773588in}}%
\pgfpathlineto{\pgfqpoint{1.643249in}{0.773588in}}%
\pgfpathlineto{\pgfqpoint{1.630788in}{0.773588in}}%
\pgfpathlineto{\pgfqpoint{1.618336in}{0.773588in}}%
\pgfpathlineto{\pgfqpoint{1.605869in}{0.773588in}}%
\pgfpathlineto{\pgfqpoint{1.593430in}{0.773588in}}%
\pgfpathlineto{\pgfqpoint{1.581070in}{0.773588in}}%
\pgfpathlineto{\pgfqpoint{1.568672in}{0.773588in}}%
\pgfpathlineto{\pgfqpoint{1.556290in}{0.773588in}}%
\pgfpathlineto{\pgfqpoint{1.543839in}{0.773588in}}%
\pgfpathlineto{\pgfqpoint{1.531474in}{0.773588in}}%
\pgfpathlineto{\pgfqpoint{1.519060in}{0.773588in}}%
\pgfpathlineto{\pgfqpoint{1.506021in}{0.773588in}}%
\pgfpathlineto{\pgfqpoint{1.493578in}{0.773588in}}%
\pgfpathlineto{\pgfqpoint{1.481163in}{0.773588in}}%
\pgfpathlineto{\pgfqpoint{1.468660in}{0.773588in}}%
\pgfpathlineto{\pgfqpoint{1.456188in}{0.773588in}}%
\pgfpathlineto{\pgfqpoint{1.443756in}{0.773588in}}%
\pgfpathlineto{\pgfqpoint{1.431382in}{0.773588in}}%
\pgfpathlineto{\pgfqpoint{1.418893in}{0.773588in}}%
\pgfpathlineto{\pgfqpoint{1.406401in}{0.773588in}}%
\pgfpathlineto{\pgfqpoint{1.393906in}{0.773588in}}%
\pgfpathlineto{\pgfqpoint{1.381448in}{0.773588in}}%
\pgfpathlineto{\pgfqpoint{1.368992in}{0.773588in}}%
\pgfpathlineto{\pgfqpoint{1.356458in}{0.773588in}}%
\pgfpathlineto{\pgfqpoint{1.343969in}{0.773588in}}%
\pgfpathlineto{\pgfqpoint{1.331552in}{0.773588in}}%
\pgfpathlineto{\pgfqpoint{1.319073in}{0.773588in}}%
\pgfpathlineto{\pgfqpoint{1.306603in}{0.773588in}}%
\pgfpathlineto{\pgfqpoint{1.294165in}{0.773588in}}%
\pgfpathlineto{\pgfqpoint{1.281753in}{0.773588in}}%
\pgfpathlineto{\pgfqpoint{1.269255in}{0.773588in}}%
\pgfpathlineto{\pgfqpoint{1.256724in}{0.773588in}}%
\pgfpathlineto{\pgfqpoint{1.244104in}{0.773588in}}%
\pgfpathlineto{\pgfqpoint{1.231699in}{0.773588in}}%
\pgfpathlineto{\pgfqpoint{1.219183in}{0.773588in}}%
\pgfpathlineto{\pgfqpoint{1.206622in}{0.773588in}}%
\pgfpathlineto{\pgfqpoint{1.194024in}{0.773588in}}%
\pgfpathlineto{\pgfqpoint{1.181503in}{0.773588in}}%
\pgfpathlineto{\pgfqpoint{1.168924in}{0.773588in}}%
\pgfpathlineto{\pgfqpoint{1.156226in}{0.773588in}}%
\pgfpathlineto{\pgfqpoint{1.143621in}{0.773588in}}%
\pgfpathlineto{\pgfqpoint{1.131014in}{0.773588in}}%
\pgfpathlineto{\pgfqpoint{1.118303in}{0.773588in}}%
\pgfpathlineto{\pgfqpoint{1.105655in}{0.773588in}}%
\pgfpathlineto{\pgfqpoint{1.092853in}{0.773588in}}%
\pgfpathlineto{\pgfqpoint{1.079972in}{0.773588in}}%
\pgfpathlineto{\pgfqpoint{1.067143in}{0.773588in}}%
\pgfpathlineto{\pgfqpoint{1.054264in}{0.773588in}}%
\pgfpathlineto{\pgfqpoint{1.041322in}{0.773588in}}%
\pgfpathlineto{\pgfqpoint{1.028303in}{0.773588in}}%
\pgfpathlineto{\pgfqpoint{1.015150in}{0.773588in}}%
\pgfpathlineto{\pgfqpoint{1.001616in}{0.773588in}}%
\pgfpathclose%
\pgfusepath{fill}%
\end{pgfscope}%
\begin{pgfscope}%
\pgfpathrectangle{\pgfqpoint{0.781402in}{0.773588in}}{\pgfqpoint{4.844695in}{5.415119in}}%
\pgfusepath{clip}%
\pgfsetbuttcap%
\pgfsetroundjoin%
\definecolor{currentfill}{rgb}{1.000000,0.498039,0.054902}%
\pgfsetfillcolor{currentfill}%
\pgfsetlinewidth{0.000000pt}%
\definecolor{currentstroke}{rgb}{0.000000,0.000000,0.000000}%
\pgfsetstrokecolor{currentstroke}%
\pgfsetdash{}{0pt}%
\pgfpathmoveto{\pgfqpoint{1.001616in}{0.773588in}}%
\pgfpathlineto{\pgfqpoint{1.001616in}{0.773588in}}%
\pgfpathlineto{\pgfqpoint{1.015150in}{0.773588in}}%
\pgfpathlineto{\pgfqpoint{1.028303in}{0.773588in}}%
\pgfpathlineto{\pgfqpoint{1.041322in}{0.773588in}}%
\pgfpathlineto{\pgfqpoint{1.054264in}{0.773588in}}%
\pgfpathlineto{\pgfqpoint{1.067143in}{0.773588in}}%
\pgfpathlineto{\pgfqpoint{1.079972in}{0.773588in}}%
\pgfpathlineto{\pgfqpoint{1.092853in}{0.773588in}}%
\pgfpathlineto{\pgfqpoint{1.105655in}{0.773588in}}%
\pgfpathlineto{\pgfqpoint{1.118303in}{0.773588in}}%
\pgfpathlineto{\pgfqpoint{1.131014in}{0.773588in}}%
\pgfpathlineto{\pgfqpoint{1.143621in}{0.773588in}}%
\pgfpathlineto{\pgfqpoint{1.156226in}{0.773588in}}%
\pgfpathlineto{\pgfqpoint{1.168924in}{0.773588in}}%
\pgfpathlineto{\pgfqpoint{1.181503in}{0.773588in}}%
\pgfpathlineto{\pgfqpoint{1.194024in}{0.773588in}}%
\pgfpathlineto{\pgfqpoint{1.206622in}{0.773588in}}%
\pgfpathlineto{\pgfqpoint{1.219183in}{0.773588in}}%
\pgfpathlineto{\pgfqpoint{1.231699in}{0.773588in}}%
\pgfpathlineto{\pgfqpoint{1.244104in}{0.773588in}}%
\pgfpathlineto{\pgfqpoint{1.256724in}{0.773588in}}%
\pgfpathlineto{\pgfqpoint{1.269255in}{0.773588in}}%
\pgfpathlineto{\pgfqpoint{1.281753in}{0.773588in}}%
\pgfpathlineto{\pgfqpoint{1.294165in}{0.773588in}}%
\pgfpathlineto{\pgfqpoint{1.306603in}{0.773588in}}%
\pgfpathlineto{\pgfqpoint{1.319073in}{0.773588in}}%
\pgfpathlineto{\pgfqpoint{1.331552in}{0.773588in}}%
\pgfpathlineto{\pgfqpoint{1.343969in}{0.773588in}}%
\pgfpathlineto{\pgfqpoint{1.356458in}{0.773588in}}%
\pgfpathlineto{\pgfqpoint{1.368992in}{0.773588in}}%
\pgfpathlineto{\pgfqpoint{1.381448in}{0.773588in}}%
\pgfpathlineto{\pgfqpoint{1.393906in}{0.773588in}}%
\pgfpathlineto{\pgfqpoint{1.406401in}{0.773588in}}%
\pgfpathlineto{\pgfqpoint{1.418893in}{0.773588in}}%
\pgfpathlineto{\pgfqpoint{1.431382in}{0.773588in}}%
\pgfpathlineto{\pgfqpoint{1.443756in}{0.773588in}}%
\pgfpathlineto{\pgfqpoint{1.456188in}{0.773588in}}%
\pgfpathlineto{\pgfqpoint{1.468660in}{0.773588in}}%
\pgfpathlineto{\pgfqpoint{1.481163in}{0.773588in}}%
\pgfpathlineto{\pgfqpoint{1.493578in}{0.773588in}}%
\pgfpathlineto{\pgfqpoint{1.506021in}{0.773588in}}%
\pgfpathlineto{\pgfqpoint{1.519060in}{0.773588in}}%
\pgfpathlineto{\pgfqpoint{1.531474in}{0.773588in}}%
\pgfpathlineto{\pgfqpoint{1.543839in}{0.773588in}}%
\pgfpathlineto{\pgfqpoint{1.556290in}{0.773588in}}%
\pgfpathlineto{\pgfqpoint{1.568672in}{0.773588in}}%
\pgfpathlineto{\pgfqpoint{1.581070in}{0.773588in}}%
\pgfpathlineto{\pgfqpoint{1.593430in}{0.773588in}}%
\pgfpathlineto{\pgfqpoint{1.605869in}{0.773588in}}%
\pgfpathlineto{\pgfqpoint{1.618336in}{0.773588in}}%
\pgfpathlineto{\pgfqpoint{1.630788in}{0.773588in}}%
\pgfpathlineto{\pgfqpoint{1.643249in}{0.773588in}}%
\pgfpathlineto{\pgfqpoint{1.655673in}{0.773588in}}%
\pgfpathlineto{\pgfqpoint{1.668187in}{0.773588in}}%
\pgfpathlineto{\pgfqpoint{1.680578in}{0.773588in}}%
\pgfpathlineto{\pgfqpoint{1.693005in}{0.773588in}}%
\pgfpathlineto{\pgfqpoint{1.705498in}{0.773588in}}%
\pgfpathlineto{\pgfqpoint{1.717832in}{0.773588in}}%
\pgfpathlineto{\pgfqpoint{1.730200in}{0.773588in}}%
\pgfpathlineto{\pgfqpoint{1.742593in}{0.773588in}}%
\pgfpathlineto{\pgfqpoint{1.755070in}{0.773588in}}%
\pgfpathlineto{\pgfqpoint{1.767480in}{0.773588in}}%
\pgfpathlineto{\pgfqpoint{1.779913in}{0.773588in}}%
\pgfpathlineto{\pgfqpoint{1.792355in}{0.773588in}}%
\pgfpathlineto{\pgfqpoint{1.804716in}{0.773588in}}%
\pgfpathlineto{\pgfqpoint{1.817148in}{0.773588in}}%
\pgfpathlineto{\pgfqpoint{1.829628in}{0.773588in}}%
\pgfpathlineto{\pgfqpoint{1.842065in}{0.773588in}}%
\pgfpathlineto{\pgfqpoint{1.854433in}{0.773588in}}%
\pgfpathlineto{\pgfqpoint{1.866747in}{0.773588in}}%
\pgfpathlineto{\pgfqpoint{1.879180in}{0.773588in}}%
\pgfpathlineto{\pgfqpoint{1.891620in}{0.773588in}}%
\pgfpathlineto{\pgfqpoint{1.904130in}{0.773588in}}%
\pgfpathlineto{\pgfqpoint{1.916611in}{0.773588in}}%
\pgfpathlineto{\pgfqpoint{1.929118in}{0.773588in}}%
\pgfpathlineto{\pgfqpoint{1.941620in}{0.773588in}}%
\pgfpathlineto{\pgfqpoint{1.954108in}{0.773588in}}%
\pgfpathlineto{\pgfqpoint{1.966664in}{0.773588in}}%
\pgfpathlineto{\pgfqpoint{1.979161in}{0.773588in}}%
\pgfpathlineto{\pgfqpoint{1.991646in}{0.773588in}}%
\pgfpathlineto{\pgfqpoint{2.004212in}{0.773588in}}%
\pgfpathlineto{\pgfqpoint{2.016889in}{0.773588in}}%
\pgfpathlineto{\pgfqpoint{2.030840in}{0.773588in}}%
\pgfpathlineto{\pgfqpoint{2.043350in}{0.773588in}}%
\pgfpathlineto{\pgfqpoint{2.055871in}{0.773588in}}%
\pgfpathlineto{\pgfqpoint{2.068360in}{0.773588in}}%
\pgfpathlineto{\pgfqpoint{2.080914in}{0.773588in}}%
\pgfpathlineto{\pgfqpoint{2.093434in}{0.773588in}}%
\pgfpathlineto{\pgfqpoint{2.106029in}{0.773588in}}%
\pgfpathlineto{\pgfqpoint{2.118500in}{0.773588in}}%
\pgfpathlineto{\pgfqpoint{2.130987in}{0.773588in}}%
\pgfpathlineto{\pgfqpoint{2.143512in}{0.773588in}}%
\pgfpathlineto{\pgfqpoint{2.156064in}{0.773588in}}%
\pgfpathlineto{\pgfqpoint{2.168659in}{0.773588in}}%
\pgfpathlineto{\pgfqpoint{2.181172in}{0.773588in}}%
\pgfpathlineto{\pgfqpoint{2.193673in}{0.773588in}}%
\pgfpathlineto{\pgfqpoint{2.206174in}{0.773588in}}%
\pgfpathlineto{\pgfqpoint{2.218726in}{0.773588in}}%
\pgfpathlineto{\pgfqpoint{2.231301in}{0.773588in}}%
\pgfpathlineto{\pgfqpoint{2.243791in}{0.773588in}}%
\pgfpathlineto{\pgfqpoint{2.256287in}{0.773588in}}%
\pgfpathlineto{\pgfqpoint{2.268761in}{0.773588in}}%
\pgfpathlineto{\pgfqpoint{2.281255in}{0.773588in}}%
\pgfpathlineto{\pgfqpoint{2.293873in}{0.773588in}}%
\pgfpathlineto{\pgfqpoint{2.306399in}{0.773588in}}%
\pgfpathlineto{\pgfqpoint{2.318891in}{0.773588in}}%
\pgfpathlineto{\pgfqpoint{2.331395in}{0.773588in}}%
\pgfpathlineto{\pgfqpoint{2.343909in}{0.773588in}}%
\pgfpathlineto{\pgfqpoint{2.356417in}{0.773588in}}%
\pgfpathlineto{\pgfqpoint{2.368880in}{0.773588in}}%
\pgfpathlineto{\pgfqpoint{2.381372in}{0.773588in}}%
\pgfpathlineto{\pgfqpoint{2.393733in}{0.773588in}}%
\pgfpathlineto{\pgfqpoint{2.406207in}{0.773588in}}%
\pgfpathlineto{\pgfqpoint{2.418633in}{0.773588in}}%
\pgfpathlineto{\pgfqpoint{2.433282in}{0.773588in}}%
\pgfpathlineto{\pgfqpoint{2.457197in}{0.773588in}}%
\pgfpathlineto{\pgfqpoint{2.477723in}{0.773588in}}%
\pgfpathlineto{\pgfqpoint{2.494737in}{0.773588in}}%
\pgfpathlineto{\pgfqpoint{2.515773in}{0.773588in}}%
\pgfpathlineto{\pgfqpoint{2.538352in}{0.773588in}}%
\pgfpathlineto{\pgfqpoint{2.558886in}{0.773588in}}%
\pgfpathlineto{\pgfqpoint{2.578521in}{0.773588in}}%
\pgfpathlineto{\pgfqpoint{2.598064in}{0.773588in}}%
\pgfpathlineto{\pgfqpoint{2.616149in}{0.773588in}}%
\pgfpathlineto{\pgfqpoint{2.636069in}{0.773588in}}%
\pgfpathlineto{\pgfqpoint{2.657274in}{0.773588in}}%
\pgfpathlineto{\pgfqpoint{2.676022in}{0.773588in}}%
\pgfpathlineto{\pgfqpoint{2.693540in}{0.773588in}}%
\pgfpathlineto{\pgfqpoint{2.710992in}{0.773588in}}%
\pgfpathlineto{\pgfqpoint{2.727722in}{0.773588in}}%
\pgfpathlineto{\pgfqpoint{2.744840in}{0.773588in}}%
\pgfpathlineto{\pgfqpoint{2.763118in}{0.773588in}}%
\pgfpathlineto{\pgfqpoint{2.780862in}{0.773588in}}%
\pgfpathlineto{\pgfqpoint{2.798917in}{0.773588in}}%
\pgfpathlineto{\pgfqpoint{2.815750in}{0.773588in}}%
\pgfpathlineto{\pgfqpoint{2.831972in}{0.773588in}}%
\pgfpathlineto{\pgfqpoint{2.848081in}{0.773588in}}%
\pgfpathlineto{\pgfqpoint{2.864455in}{0.773588in}}%
\pgfpathlineto{\pgfqpoint{2.880921in}{0.773588in}}%
\pgfpathlineto{\pgfqpoint{2.897563in}{0.773588in}}%
\pgfpathlineto{\pgfqpoint{2.914048in}{0.773588in}}%
\pgfpathlineto{\pgfqpoint{2.930775in}{0.773588in}}%
\pgfpathlineto{\pgfqpoint{2.949130in}{0.773588in}}%
\pgfpathlineto{\pgfqpoint{2.968433in}{0.773588in}}%
\pgfpathlineto{\pgfqpoint{2.985853in}{0.773588in}}%
\pgfpathlineto{\pgfqpoint{3.001921in}{0.773588in}}%
\pgfpathlineto{\pgfqpoint{3.018968in}{0.773588in}}%
\pgfpathlineto{\pgfqpoint{3.036276in}{0.773588in}}%
\pgfpathlineto{\pgfqpoint{3.052708in}{0.773588in}}%
\pgfpathlineto{\pgfqpoint{3.069003in}{0.773588in}}%
\pgfpathlineto{\pgfqpoint{3.085686in}{0.773588in}}%
\pgfpathlineto{\pgfqpoint{3.102316in}{0.773588in}}%
\pgfpathlineto{\pgfqpoint{3.118692in}{0.773588in}}%
\pgfpathlineto{\pgfqpoint{3.135319in}{0.773588in}}%
\pgfpathlineto{\pgfqpoint{3.152584in}{0.773588in}}%
\pgfpathlineto{\pgfqpoint{3.169651in}{0.773588in}}%
\pgfpathlineto{\pgfqpoint{3.185752in}{0.773588in}}%
\pgfpathlineto{\pgfqpoint{3.201602in}{0.773588in}}%
\pgfpathlineto{\pgfqpoint{3.217213in}{0.773588in}}%
\pgfpathlineto{\pgfqpoint{3.232763in}{0.773588in}}%
\pgfpathlineto{\pgfqpoint{3.248207in}{0.773588in}}%
\pgfpathlineto{\pgfqpoint{3.263722in}{0.773588in}}%
\pgfpathlineto{\pgfqpoint{3.279179in}{0.773588in}}%
\pgfpathlineto{\pgfqpoint{3.295281in}{0.773588in}}%
\pgfpathlineto{\pgfqpoint{3.313329in}{0.773588in}}%
\pgfpathlineto{\pgfqpoint{3.328928in}{0.773588in}}%
\pgfpathlineto{\pgfqpoint{3.344594in}{0.773588in}}%
\pgfpathlineto{\pgfqpoint{3.359992in}{0.773588in}}%
\pgfpathlineto{\pgfqpoint{3.375355in}{0.773588in}}%
\pgfpathlineto{\pgfqpoint{3.390722in}{0.773588in}}%
\pgfpathlineto{\pgfqpoint{3.405908in}{0.773588in}}%
\pgfpathlineto{\pgfqpoint{3.421182in}{0.773588in}}%
\pgfpathlineto{\pgfqpoint{3.436431in}{0.773588in}}%
\pgfpathlineto{\pgfqpoint{3.451777in}{0.773588in}}%
\pgfpathlineto{\pgfqpoint{3.467311in}{0.773588in}}%
\pgfpathlineto{\pgfqpoint{3.483019in}{0.773588in}}%
\pgfpathlineto{\pgfqpoint{3.498190in}{0.773588in}}%
\pgfpathlineto{\pgfqpoint{3.513323in}{0.773588in}}%
\pgfpathlineto{\pgfqpoint{3.528239in}{0.773588in}}%
\pgfpathlineto{\pgfqpoint{3.542041in}{0.773588in}}%
\pgfpathlineto{\pgfqpoint{3.554208in}{0.773588in}}%
\pgfpathlineto{\pgfqpoint{3.566335in}{0.773588in}}%
\pgfpathlineto{\pgfqpoint{3.578260in}{0.773588in}}%
\pgfpathlineto{\pgfqpoint{3.590359in}{0.773588in}}%
\pgfpathlineto{\pgfqpoint{3.602917in}{0.773588in}}%
\pgfpathlineto{\pgfqpoint{3.615064in}{0.773588in}}%
\pgfpathlineto{\pgfqpoint{3.627108in}{0.773588in}}%
\pgfpathlineto{\pgfqpoint{3.639052in}{0.773588in}}%
\pgfpathlineto{\pgfqpoint{3.651005in}{0.773588in}}%
\pgfpathlineto{\pgfqpoint{3.663361in}{0.773588in}}%
\pgfpathlineto{\pgfqpoint{3.675639in}{0.773588in}}%
\pgfpathlineto{\pgfqpoint{3.687677in}{0.773588in}}%
\pgfpathlineto{\pgfqpoint{3.699766in}{0.773588in}}%
\pgfpathlineto{\pgfqpoint{3.711790in}{0.773588in}}%
\pgfpathlineto{\pgfqpoint{3.723925in}{0.773588in}}%
\pgfpathlineto{\pgfqpoint{3.736218in}{0.773588in}}%
\pgfpathlineto{\pgfqpoint{3.748246in}{0.773588in}}%
\pgfpathlineto{\pgfqpoint{3.760252in}{0.773588in}}%
\pgfpathlineto{\pgfqpoint{3.772194in}{0.773588in}}%
\pgfpathlineto{\pgfqpoint{3.784213in}{0.773588in}}%
\pgfpathlineto{\pgfqpoint{3.796179in}{0.773588in}}%
\pgfpathlineto{\pgfqpoint{3.807700in}{0.773588in}}%
\pgfpathlineto{\pgfqpoint{3.819207in}{0.773588in}}%
\pgfpathlineto{\pgfqpoint{3.830748in}{0.773588in}}%
\pgfpathlineto{\pgfqpoint{3.842268in}{0.773588in}}%
\pgfpathlineto{\pgfqpoint{3.853773in}{0.773588in}}%
\pgfpathlineto{\pgfqpoint{3.865394in}{0.773588in}}%
\pgfpathlineto{\pgfqpoint{3.876999in}{0.773588in}}%
\pgfpathlineto{\pgfqpoint{3.888612in}{0.773588in}}%
\pgfpathlineto{\pgfqpoint{3.900120in}{0.773588in}}%
\pgfpathlineto{\pgfqpoint{3.911548in}{0.773588in}}%
\pgfpathlineto{\pgfqpoint{3.923007in}{0.773588in}}%
\pgfpathlineto{\pgfqpoint{3.934526in}{0.773588in}}%
\pgfpathlineto{\pgfqpoint{3.945995in}{0.773588in}}%
\pgfpathlineto{\pgfqpoint{3.957345in}{0.773588in}}%
\pgfpathlineto{\pgfqpoint{3.968800in}{0.773588in}}%
\pgfpathlineto{\pgfqpoint{3.980165in}{0.773588in}}%
\pgfpathlineto{\pgfqpoint{3.991607in}{0.773588in}}%
\pgfpathlineto{\pgfqpoint{4.003035in}{0.773588in}}%
\pgfpathlineto{\pgfqpoint{4.014312in}{0.773588in}}%
\pgfpathlineto{\pgfqpoint{4.025578in}{0.780138in}}%
\pgfpathlineto{\pgfqpoint{4.036918in}{0.781674in}}%
\pgfpathlineto{\pgfqpoint{4.048157in}{0.784374in}}%
\pgfpathlineto{\pgfqpoint{4.059512in}{0.793411in}}%
\pgfpathlineto{\pgfqpoint{4.070994in}{0.793986in}}%
\pgfpathlineto{\pgfqpoint{4.082264in}{0.797517in}}%
\pgfpathlineto{\pgfqpoint{4.093537in}{0.797517in}}%
\pgfpathlineto{\pgfqpoint{4.104772in}{0.795244in}}%
\pgfpathlineto{\pgfqpoint{4.116014in}{0.793664in}}%
\pgfpathlineto{\pgfqpoint{4.127322in}{0.795009in}}%
\pgfpathlineto{\pgfqpoint{4.138598in}{0.797025in}}%
\pgfpathlineto{\pgfqpoint{4.149813in}{0.797653in}}%
\pgfpathlineto{\pgfqpoint{4.160983in}{0.805638in}}%
\pgfpathlineto{\pgfqpoint{4.172101in}{0.902780in}}%
\pgfpathlineto{\pgfqpoint{4.183285in}{0.858315in}}%
\pgfpathlineto{\pgfqpoint{4.194400in}{0.827586in}}%
\pgfpathlineto{\pgfqpoint{4.205594in}{0.820220in}}%
\pgfpathlineto{\pgfqpoint{4.216619in}{0.834565in}}%
\pgfpathlineto{\pgfqpoint{4.227703in}{0.908748in}}%
\pgfpathlineto{\pgfqpoint{4.238844in}{0.874109in}}%
\pgfpathlineto{\pgfqpoint{4.249866in}{0.923295in}}%
\pgfpathlineto{\pgfqpoint{4.260400in}{1.712068in}}%
\pgfpathlineto{\pgfqpoint{4.270972in}{1.366989in}}%
\pgfpathlineto{\pgfqpoint{4.280743in}{1.858238in}}%
\pgfpathlineto{\pgfqpoint{4.290581in}{1.840707in}}%
\pgfpathlineto{\pgfqpoint{4.300486in}{1.905912in}}%
\pgfpathlineto{\pgfqpoint{4.310022in}{1.842558in}}%
\pgfpathlineto{\pgfqpoint{4.319388in}{1.885959in}}%
\pgfpathlineto{\pgfqpoint{4.328717in}{1.840369in}}%
\pgfpathlineto{\pgfqpoint{4.337933in}{1.902762in}}%
\pgfpathlineto{\pgfqpoint{4.346774in}{1.927275in}}%
\pgfpathlineto{\pgfqpoint{4.355507in}{1.934075in}}%
\pgfpathlineto{\pgfqpoint{4.364219in}{1.927117in}}%
\pgfpathlineto{\pgfqpoint{4.373001in}{1.913286in}}%
\pgfpathlineto{\pgfqpoint{4.381765in}{1.976384in}}%
\pgfpathlineto{\pgfqpoint{4.390451in}{1.949235in}}%
\pgfpathlineto{\pgfqpoint{4.399156in}{1.959803in}}%
\pgfpathlineto{\pgfqpoint{4.407811in}{1.957148in}}%
\pgfpathlineto{\pgfqpoint{4.416522in}{1.958386in}}%
\pgfpathlineto{\pgfqpoint{4.425248in}{1.939798in}}%
\pgfpathlineto{\pgfqpoint{4.433894in}{1.966871in}}%
\pgfpathlineto{\pgfqpoint{4.442591in}{1.963719in}}%
\pgfpathlineto{\pgfqpoint{4.451305in}{1.951847in}}%
\pgfpathlineto{\pgfqpoint{4.460011in}{1.975687in}}%
\pgfpathlineto{\pgfqpoint{4.468771in}{1.959528in}}%
\pgfpathlineto{\pgfqpoint{4.477466in}{1.976629in}}%
\pgfpathlineto{\pgfqpoint{4.486071in}{1.980098in}}%
\pgfpathlineto{\pgfqpoint{4.494752in}{1.961054in}}%
\pgfpathlineto{\pgfqpoint{4.503421in}{1.972271in}}%
\pgfpathlineto{\pgfqpoint{4.512084in}{1.961443in}}%
\pgfpathlineto{\pgfqpoint{4.520678in}{1.983090in}}%
\pgfpathlineto{\pgfqpoint{4.529276in}{1.962817in}}%
\pgfpathlineto{\pgfqpoint{4.537876in}{1.991824in}}%
\pgfpathlineto{\pgfqpoint{4.546439in}{1.990194in}}%
\pgfpathlineto{\pgfqpoint{4.554980in}{1.981305in}}%
\pgfpathlineto{\pgfqpoint{4.563550in}{1.976320in}}%
\pgfpathlineto{\pgfqpoint{4.572097in}{1.977375in}}%
\pgfpathlineto{\pgfqpoint{4.580727in}{1.987710in}}%
\pgfpathlineto{\pgfqpoint{4.589255in}{1.996773in}}%
\pgfpathlineto{\pgfqpoint{4.597719in}{2.003030in}}%
\pgfpathlineto{\pgfqpoint{4.606211in}{1.975203in}}%
\pgfpathlineto{\pgfqpoint{4.614783in}{1.991947in}}%
\pgfpathlineto{\pgfqpoint{4.623282in}{2.003276in}}%
\pgfpathlineto{\pgfqpoint{4.631733in}{1.996046in}}%
\pgfpathlineto{\pgfqpoint{4.640221in}{1.967260in}}%
\pgfpathlineto{\pgfqpoint{4.648847in}{1.949582in}}%
\pgfpathlineto{\pgfqpoint{4.657349in}{2.000882in}}%
\pgfpathlineto{\pgfqpoint{4.665825in}{1.968337in}}%
\pgfpathlineto{\pgfqpoint{4.674255in}{1.990279in}}%
\pgfpathlineto{\pgfqpoint{4.682643in}{2.003883in}}%
\pgfpathlineto{\pgfqpoint{4.691071in}{2.016009in}}%
\pgfpathlineto{\pgfqpoint{4.699465in}{2.021339in}}%
\pgfpathlineto{\pgfqpoint{4.707865in}{2.005723in}}%
\pgfpathlineto{\pgfqpoint{4.716247in}{2.016652in}}%
\pgfpathlineto{\pgfqpoint{4.724666in}{1.995905in}}%
\pgfpathlineto{\pgfqpoint{4.733049in}{2.015841in}}%
\pgfpathlineto{\pgfqpoint{4.741378in}{1.994153in}}%
\pgfpathlineto{\pgfqpoint{4.749750in}{2.007352in}}%
\pgfpathlineto{\pgfqpoint{4.758119in}{2.005010in}}%
\pgfpathlineto{\pgfqpoint{4.766528in}{2.008808in}}%
\pgfpathlineto{\pgfqpoint{4.774881in}{2.016107in}}%
\pgfpathlineto{\pgfqpoint{4.783272in}{2.005253in}}%
\pgfpathlineto{\pgfqpoint{4.791573in}{2.008402in}}%
\pgfpathlineto{\pgfqpoint{4.799850in}{2.012329in}}%
\pgfpathlineto{\pgfqpoint{4.808134in}{2.012887in}}%
\pgfpathlineto{\pgfqpoint{4.816444in}{2.007052in}}%
\pgfpathlineto{\pgfqpoint{4.824772in}{2.010632in}}%
\pgfpathlineto{\pgfqpoint{4.833061in}{2.010650in}}%
\pgfpathlineto{\pgfqpoint{4.841325in}{2.028311in}}%
\pgfpathlineto{\pgfqpoint{4.849575in}{2.001721in}}%
\pgfpathlineto{\pgfqpoint{4.857830in}{2.029836in}}%
\pgfpathlineto{\pgfqpoint{4.866069in}{2.020263in}}%
\pgfpathlineto{\pgfqpoint{4.874381in}{2.005115in}}%
\pgfpathlineto{\pgfqpoint{4.882658in}{2.008392in}}%
\pgfpathlineto{\pgfqpoint{4.890980in}{2.022830in}}%
\pgfpathlineto{\pgfqpoint{4.899205in}{2.033603in}}%
\pgfpathlineto{\pgfqpoint{4.907476in}{2.008683in}}%
\pgfpathlineto{\pgfqpoint{4.915736in}{2.024372in}}%
\pgfpathlineto{\pgfqpoint{4.924041in}{2.002932in}}%
\pgfpathlineto{\pgfqpoint{4.932347in}{2.019887in}}%
\pgfpathlineto{\pgfqpoint{4.940566in}{2.001328in}}%
\pgfpathlineto{\pgfqpoint{4.948774in}{2.012384in}}%
\pgfpathlineto{\pgfqpoint{4.956992in}{2.006114in}}%
\pgfpathlineto{\pgfqpoint{4.965138in}{1.984777in}}%
\pgfpathlineto{\pgfqpoint{4.973428in}{2.002030in}}%
\pgfpathlineto{\pgfqpoint{4.981642in}{1.996711in}}%
\pgfpathlineto{\pgfqpoint{4.989794in}{2.006792in}}%
\pgfpathlineto{\pgfqpoint{4.997948in}{2.023285in}}%
\pgfpathlineto{\pgfqpoint{5.006134in}{2.000368in}}%
\pgfpathlineto{\pgfqpoint{5.014317in}{1.997604in}}%
\pgfpathlineto{\pgfqpoint{5.022451in}{2.017700in}}%
\pgfpathlineto{\pgfqpoint{5.030602in}{2.014069in}}%
\pgfpathlineto{\pgfqpoint{5.038759in}{2.006790in}}%
\pgfpathlineto{\pgfqpoint{5.046876in}{1.994952in}}%
\pgfpathlineto{\pgfqpoint{5.055031in}{2.019743in}}%
\pgfpathlineto{\pgfqpoint{5.063216in}{2.010850in}}%
\pgfpathlineto{\pgfqpoint{5.071391in}{1.995421in}}%
\pgfpathlineto{\pgfqpoint{5.079506in}{1.994744in}}%
\pgfpathlineto{\pgfqpoint{5.087675in}{2.014413in}}%
\pgfpathlineto{\pgfqpoint{5.095819in}{2.018341in}}%
\pgfpathlineto{\pgfqpoint{5.103859in}{2.013410in}}%
\pgfpathlineto{\pgfqpoint{5.111982in}{2.006399in}}%
\pgfpathlineto{\pgfqpoint{5.120128in}{2.016296in}}%
\pgfpathlineto{\pgfqpoint{5.128231in}{2.001004in}}%
\pgfpathlineto{\pgfqpoint{5.136344in}{2.024205in}}%
\pgfpathlineto{\pgfqpoint{5.144370in}{1.987290in}}%
\pgfpathlineto{\pgfqpoint{5.152426in}{1.988997in}}%
\pgfpathlineto{\pgfqpoint{5.160447in}{2.011538in}}%
\pgfpathlineto{\pgfqpoint{5.168547in}{2.008400in}}%
\pgfpathlineto{\pgfqpoint{5.176647in}{1.992237in}}%
\pgfpathlineto{\pgfqpoint{5.184747in}{2.024057in}}%
\pgfpathlineto{\pgfqpoint{5.192761in}{2.016126in}}%
\pgfpathlineto{\pgfqpoint{5.200810in}{2.003568in}}%
\pgfpathlineto{\pgfqpoint{5.208810in}{2.025945in}}%
\pgfpathlineto{\pgfqpoint{5.216856in}{2.017889in}}%
\pgfpathlineto{\pgfqpoint{5.224894in}{2.006818in}}%
\pgfpathlineto{\pgfqpoint{5.233040in}{2.017139in}}%
\pgfpathlineto{\pgfqpoint{5.245361in}{2.019855in}}%
\pgfpathlineto{\pgfqpoint{5.253575in}{1.997690in}}%
\pgfpathlineto{\pgfqpoint{5.261632in}{2.004198in}}%
\pgfpathlineto{\pgfqpoint{5.269666in}{1.992287in}}%
\pgfpathlineto{\pgfqpoint{5.277683in}{2.012220in}}%
\pgfpathlineto{\pgfqpoint{5.285748in}{2.011043in}}%
\pgfpathlineto{\pgfqpoint{5.293795in}{2.002070in}}%
\pgfpathlineto{\pgfqpoint{5.301824in}{2.021439in}}%
\pgfpathlineto{\pgfqpoint{5.309888in}{2.004593in}}%
\pgfpathlineto{\pgfqpoint{5.317933in}{2.013298in}}%
\pgfpathlineto{\pgfqpoint{5.327487in}{2.078279in}}%
\pgfpathlineto{\pgfqpoint{5.338668in}{2.078279in}}%
\pgfpathlineto{\pgfqpoint{5.349820in}{2.078279in}}%
\pgfpathlineto{\pgfqpoint{5.360970in}{2.078279in}}%
\pgfpathlineto{\pgfqpoint{5.372247in}{2.078279in}}%
\pgfpathlineto{\pgfqpoint{5.383480in}{2.078279in}}%
\pgfpathlineto{\pgfqpoint{5.394653in}{2.078279in}}%
\pgfpathlineto{\pgfqpoint{5.405885in}{2.078279in}}%
\pgfpathlineto{\pgfqpoint{5.405885in}{3.343828in}}%
\pgfpathlineto{\pgfqpoint{5.405885in}{3.343828in}}%
\pgfpathlineto{\pgfqpoint{5.394653in}{3.343828in}}%
\pgfpathlineto{\pgfqpoint{5.383480in}{3.343828in}}%
\pgfpathlineto{\pgfqpoint{5.372247in}{3.343828in}}%
\pgfpathlineto{\pgfqpoint{5.360970in}{3.343828in}}%
\pgfpathlineto{\pgfqpoint{5.349820in}{3.343828in}}%
\pgfpathlineto{\pgfqpoint{5.338668in}{3.343828in}}%
\pgfpathlineto{\pgfqpoint{5.327487in}{3.343828in}}%
\pgfpathlineto{\pgfqpoint{5.317933in}{3.215817in}}%
\pgfpathlineto{\pgfqpoint{5.309888in}{3.198667in}}%
\pgfpathlineto{\pgfqpoint{5.301824in}{3.231854in}}%
\pgfpathlineto{\pgfqpoint{5.293795in}{3.193697in}}%
\pgfpathlineto{\pgfqpoint{5.285748in}{3.211374in}}%
\pgfpathlineto{\pgfqpoint{5.277683in}{3.213693in}}%
\pgfpathlineto{\pgfqpoint{5.269666in}{3.174426in}}%
\pgfpathlineto{\pgfqpoint{5.261632in}{3.197890in}}%
\pgfpathlineto{\pgfqpoint{5.253575in}{3.185068in}}%
\pgfpathlineto{\pgfqpoint{5.245361in}{3.228733in}}%
\pgfpathlineto{\pgfqpoint{5.233040in}{3.223384in}}%
\pgfpathlineto{\pgfqpoint{5.224894in}{3.203050in}}%
\pgfpathlineto{\pgfqpoint{5.216856in}{3.224860in}}%
\pgfpathlineto{\pgfqpoint{5.208810in}{3.240732in}}%
\pgfpathlineto{\pgfqpoint{5.200810in}{3.196648in}}%
\pgfpathlineto{\pgfqpoint{5.192761in}{3.221387in}}%
\pgfpathlineto{\pgfqpoint{5.184747in}{3.237012in}}%
\pgfpathlineto{\pgfqpoint{5.176647in}{3.174325in}}%
\pgfpathlineto{\pgfqpoint{5.168547in}{3.206168in}}%
\pgfpathlineto{\pgfqpoint{5.160447in}{3.212349in}}%
\pgfpathlineto{\pgfqpoint{5.152426in}{3.167944in}}%
\pgfpathlineto{\pgfqpoint{5.144370in}{3.164581in}}%
\pgfpathlineto{\pgfqpoint{5.136344in}{3.237303in}}%
\pgfpathlineto{\pgfqpoint{5.128231in}{3.191598in}}%
\pgfpathlineto{\pgfqpoint{5.120128in}{3.221722in}}%
\pgfpathlineto{\pgfqpoint{5.111982in}{3.202225in}}%
\pgfpathlineto{\pgfqpoint{5.103859in}{3.216037in}}%
\pgfpathlineto{\pgfqpoint{5.095819in}{3.225752in}}%
\pgfpathlineto{\pgfqpoint{5.087675in}{3.218012in}}%
\pgfpathlineto{\pgfqpoint{5.079506in}{3.179265in}}%
\pgfpathlineto{\pgfqpoint{5.071391in}{3.180598in}}%
\pgfpathlineto{\pgfqpoint{5.063216in}{3.210993in}}%
\pgfpathlineto{\pgfqpoint{5.055031in}{3.228513in}}%
\pgfpathlineto{\pgfqpoint{5.046876in}{3.179675in}}%
\pgfpathlineto{\pgfqpoint{5.038759in}{3.202995in}}%
\pgfpathlineto{\pgfqpoint{5.030602in}{3.217335in}}%
\pgfpathlineto{\pgfqpoint{5.022451in}{3.224489in}}%
\pgfpathlineto{\pgfqpoint{5.014317in}{3.184898in}}%
\pgfpathlineto{\pgfqpoint{5.006134in}{3.190344in}}%
\pgfpathlineto{\pgfqpoint{4.997948in}{3.235491in}}%
\pgfpathlineto{\pgfqpoint{4.989794in}{3.202999in}}%
\pgfpathlineto{\pgfqpoint{4.981642in}{3.183139in}}%
\pgfpathlineto{\pgfqpoint{4.973428in}{3.193617in}}%
\pgfpathlineto{\pgfqpoint{4.965138in}{3.159630in}}%
\pgfpathlineto{\pgfqpoint{4.956992in}{3.201664in}}%
\pgfpathlineto{\pgfqpoint{4.948774in}{3.214016in}}%
\pgfpathlineto{\pgfqpoint{4.940566in}{3.192235in}}%
\pgfpathlineto{\pgfqpoint{4.932347in}{3.228796in}}%
\pgfpathlineto{\pgfqpoint{4.924041in}{3.195395in}}%
\pgfpathlineto{\pgfqpoint{4.915736in}{3.237631in}}%
\pgfpathlineto{\pgfqpoint{4.907476in}{3.206725in}}%
\pgfpathlineto{\pgfqpoint{4.899205in}{3.255818in}}%
\pgfpathlineto{\pgfqpoint{4.890980in}{3.234595in}}%
\pgfpathlineto{\pgfqpoint{4.882658in}{3.206152in}}%
\pgfpathlineto{\pgfqpoint{4.874381in}{3.199696in}}%
\pgfpathlineto{\pgfqpoint{4.866069in}{3.229537in}}%
\pgfpathlineto{\pgfqpoint{4.857830in}{3.248397in}}%
\pgfpathlineto{\pgfqpoint{4.849575in}{3.193009in}}%
\pgfpathlineto{\pgfqpoint{4.841325in}{3.245392in}}%
\pgfpathlineto{\pgfqpoint{4.833061in}{3.210599in}}%
\pgfpathlineto{\pgfqpoint{4.824772in}{3.210565in}}%
\pgfpathlineto{\pgfqpoint{4.816444in}{3.203512in}}%
\pgfpathlineto{\pgfqpoint{4.808134in}{3.215006in}}%
\pgfpathlineto{\pgfqpoint{4.799850in}{3.213907in}}%
\pgfpathlineto{\pgfqpoint{4.791573in}{3.206171in}}%
\pgfpathlineto{\pgfqpoint{4.783272in}{3.199967in}}%
\pgfpathlineto{\pgfqpoint{4.774881in}{3.221349in}}%
\pgfpathlineto{\pgfqpoint{4.766528in}{3.206972in}}%
\pgfpathlineto{\pgfqpoint{4.758119in}{3.199488in}}%
\pgfpathlineto{\pgfqpoint{4.749750in}{3.204103in}}%
\pgfpathlineto{\pgfqpoint{4.741378in}{3.178101in}}%
\pgfpathlineto{\pgfqpoint{4.733049in}{3.220826in}}%
\pgfpathlineto{\pgfqpoint{4.724666in}{3.181551in}}%
\pgfpathlineto{\pgfqpoint{4.716247in}{3.222424in}}%
\pgfpathlineto{\pgfqpoint{4.707865in}{3.200894in}}%
\pgfpathlineto{\pgfqpoint{4.699465in}{3.231656in}}%
\pgfpathlineto{\pgfqpoint{4.691071in}{3.221156in}}%
\pgfpathlineto{\pgfqpoint{4.682643in}{3.197268in}}%
\pgfpathlineto{\pgfqpoint{4.674255in}{3.170469in}}%
\pgfpathlineto{\pgfqpoint{4.665825in}{3.127243in}}%
\pgfpathlineto{\pgfqpoint{4.657349in}{3.191356in}}%
\pgfpathlineto{\pgfqpoint{4.648847in}{3.090296in}}%
\pgfpathlineto{\pgfqpoint{4.640221in}{3.125121in}}%
\pgfpathlineto{\pgfqpoint{4.631733in}{3.181830in}}%
\pgfpathlineto{\pgfqpoint{4.623282in}{3.196073in}}%
\pgfpathlineto{\pgfqpoint{4.614783in}{3.173756in}}%
\pgfpathlineto{\pgfqpoint{4.606211in}{3.140769in}}%
\pgfpathlineto{\pgfqpoint{4.597719in}{3.195589in}}%
\pgfpathlineto{\pgfqpoint{4.589255in}{3.183262in}}%
\pgfpathlineto{\pgfqpoint{4.580727in}{3.165409in}}%
\pgfpathlineto{\pgfqpoint{4.572097in}{3.145048in}}%
\pgfpathlineto{\pgfqpoint{4.563550in}{3.142969in}}%
\pgfpathlineto{\pgfqpoint{4.554980in}{3.152790in}}%
\pgfpathlineto{\pgfqpoint{4.546439in}{3.170301in}}%
\pgfpathlineto{\pgfqpoint{4.537876in}{3.173512in}}%
\pgfpathlineto{\pgfqpoint{4.529276in}{3.116368in}}%
\pgfpathlineto{\pgfqpoint{4.520678in}{3.156306in}}%
\pgfpathlineto{\pgfqpoint{4.512084in}{3.113662in}}%
\pgfpathlineto{\pgfqpoint{4.503421in}{3.134994in}}%
\pgfpathlineto{\pgfqpoint{4.494752in}{3.112896in}}%
\pgfpathlineto{\pgfqpoint{4.486071in}{3.150413in}}%
\pgfpathlineto{\pgfqpoint{4.477466in}{3.143579in}}%
\pgfpathlineto{\pgfqpoint{4.468771in}{3.109889in}}%
\pgfpathlineto{\pgfqpoint{4.460011in}{3.141723in}}%
\pgfpathlineto{\pgfqpoint{4.451305in}{3.094758in}}%
\pgfpathlineto{\pgfqpoint{4.442591in}{3.118146in}}%
\pgfpathlineto{\pgfqpoint{4.433894in}{3.124355in}}%
\pgfpathlineto{\pgfqpoint{4.425248in}{3.071021in}}%
\pgfpathlineto{\pgfqpoint{4.416522in}{3.107639in}}%
\pgfpathlineto{\pgfqpoint{4.407811in}{3.105200in}}%
\pgfpathlineto{\pgfqpoint{4.399156in}{3.110432in}}%
\pgfpathlineto{\pgfqpoint{4.390451in}{3.089611in}}%
\pgfpathlineto{\pgfqpoint{4.381765in}{3.143095in}}%
\pgfpathlineto{\pgfqpoint{4.373001in}{3.018792in}}%
\pgfpathlineto{\pgfqpoint{4.364219in}{3.046040in}}%
\pgfpathlineto{\pgfqpoint{4.355507in}{3.059746in}}%
\pgfpathlineto{\pgfqpoint{4.346774in}{2.677129in}}%
\pgfpathlineto{\pgfqpoint{4.337933in}{2.066631in}}%
\pgfpathlineto{\pgfqpoint{4.328717in}{2.020191in}}%
\pgfpathlineto{\pgfqpoint{4.319388in}{1.956940in}}%
\pgfpathlineto{\pgfqpoint{4.310022in}{1.901955in}}%
\pgfpathlineto{\pgfqpoint{4.300486in}{1.974786in}}%
\pgfpathlineto{\pgfqpoint{4.290581in}{1.942847in}}%
\pgfpathlineto{\pgfqpoint{4.280743in}{1.935681in}}%
\pgfpathlineto{\pgfqpoint{4.270972in}{1.383893in}}%
\pgfpathlineto{\pgfqpoint{4.260400in}{1.726232in}}%
\pgfpathlineto{\pgfqpoint{4.249866in}{0.934646in}}%
\pgfpathlineto{\pgfqpoint{4.238844in}{0.883505in}}%
\pgfpathlineto{\pgfqpoint{4.227703in}{0.918144in}}%
\pgfpathlineto{\pgfqpoint{4.216619in}{0.843961in}}%
\pgfpathlineto{\pgfqpoint{4.205594in}{0.826004in}}%
\pgfpathlineto{\pgfqpoint{4.194400in}{0.833098in}}%
\pgfpathlineto{\pgfqpoint{4.183285in}{0.863827in}}%
\pgfpathlineto{\pgfqpoint{4.172101in}{0.908291in}}%
\pgfpathlineto{\pgfqpoint{4.160983in}{0.811150in}}%
\pgfpathlineto{\pgfqpoint{4.149813in}{0.800795in}}%
\pgfpathlineto{\pgfqpoint{4.138598in}{0.797025in}}%
\pgfpathlineto{\pgfqpoint{4.127322in}{0.795009in}}%
\pgfpathlineto{\pgfqpoint{4.116014in}{0.793664in}}%
\pgfpathlineto{\pgfqpoint{4.104772in}{0.795244in}}%
\pgfpathlineto{\pgfqpoint{4.093537in}{0.797517in}}%
\pgfpathlineto{\pgfqpoint{4.082264in}{0.797517in}}%
\pgfpathlineto{\pgfqpoint{4.070994in}{0.793986in}}%
\pgfpathlineto{\pgfqpoint{4.059512in}{0.793411in}}%
\pgfpathlineto{\pgfqpoint{4.048157in}{0.784374in}}%
\pgfpathlineto{\pgfqpoint{4.036918in}{0.781674in}}%
\pgfpathlineto{\pgfqpoint{4.025578in}{0.780138in}}%
\pgfpathlineto{\pgfqpoint{4.014312in}{0.773588in}}%
\pgfpathlineto{\pgfqpoint{4.003035in}{0.773588in}}%
\pgfpathlineto{\pgfqpoint{3.991607in}{0.773588in}}%
\pgfpathlineto{\pgfqpoint{3.980165in}{0.773588in}}%
\pgfpathlineto{\pgfqpoint{3.968800in}{0.773588in}}%
\pgfpathlineto{\pgfqpoint{3.957345in}{0.773588in}}%
\pgfpathlineto{\pgfqpoint{3.945995in}{0.773588in}}%
\pgfpathlineto{\pgfqpoint{3.934526in}{0.773588in}}%
\pgfpathlineto{\pgfqpoint{3.923007in}{0.773588in}}%
\pgfpathlineto{\pgfqpoint{3.911548in}{0.773588in}}%
\pgfpathlineto{\pgfqpoint{3.900120in}{0.773588in}}%
\pgfpathlineto{\pgfqpoint{3.888612in}{0.773588in}}%
\pgfpathlineto{\pgfqpoint{3.876999in}{0.773588in}}%
\pgfpathlineto{\pgfqpoint{3.865394in}{0.773588in}}%
\pgfpathlineto{\pgfqpoint{3.853773in}{0.773588in}}%
\pgfpathlineto{\pgfqpoint{3.842268in}{0.773588in}}%
\pgfpathlineto{\pgfqpoint{3.830748in}{0.773588in}}%
\pgfpathlineto{\pgfqpoint{3.819207in}{0.773588in}}%
\pgfpathlineto{\pgfqpoint{3.807700in}{0.773588in}}%
\pgfpathlineto{\pgfqpoint{3.796179in}{0.773588in}}%
\pgfpathlineto{\pgfqpoint{3.784213in}{0.773588in}}%
\pgfpathlineto{\pgfqpoint{3.772194in}{0.773588in}}%
\pgfpathlineto{\pgfqpoint{3.760252in}{0.773588in}}%
\pgfpathlineto{\pgfqpoint{3.748246in}{0.773588in}}%
\pgfpathlineto{\pgfqpoint{3.736218in}{0.773588in}}%
\pgfpathlineto{\pgfqpoint{3.723925in}{0.773588in}}%
\pgfpathlineto{\pgfqpoint{3.711790in}{0.773588in}}%
\pgfpathlineto{\pgfqpoint{3.699766in}{0.773588in}}%
\pgfpathlineto{\pgfqpoint{3.687677in}{0.773588in}}%
\pgfpathlineto{\pgfqpoint{3.675639in}{0.773588in}}%
\pgfpathlineto{\pgfqpoint{3.663361in}{0.773588in}}%
\pgfpathlineto{\pgfqpoint{3.651005in}{0.773588in}}%
\pgfpathlineto{\pgfqpoint{3.639052in}{0.773588in}}%
\pgfpathlineto{\pgfqpoint{3.627108in}{0.773588in}}%
\pgfpathlineto{\pgfqpoint{3.615064in}{0.773588in}}%
\pgfpathlineto{\pgfqpoint{3.602917in}{0.773588in}}%
\pgfpathlineto{\pgfqpoint{3.590359in}{0.773588in}}%
\pgfpathlineto{\pgfqpoint{3.578260in}{0.773588in}}%
\pgfpathlineto{\pgfqpoint{3.566335in}{0.773588in}}%
\pgfpathlineto{\pgfqpoint{3.554208in}{0.773588in}}%
\pgfpathlineto{\pgfqpoint{3.542041in}{0.773588in}}%
\pgfpathlineto{\pgfqpoint{3.528239in}{0.773588in}}%
\pgfpathlineto{\pgfqpoint{3.513323in}{0.773588in}}%
\pgfpathlineto{\pgfqpoint{3.498190in}{0.773588in}}%
\pgfpathlineto{\pgfqpoint{3.483019in}{0.773588in}}%
\pgfpathlineto{\pgfqpoint{3.467311in}{0.773588in}}%
\pgfpathlineto{\pgfqpoint{3.451777in}{0.773588in}}%
\pgfpathlineto{\pgfqpoint{3.436431in}{0.773588in}}%
\pgfpathlineto{\pgfqpoint{3.421182in}{0.773588in}}%
\pgfpathlineto{\pgfqpoint{3.405908in}{0.773588in}}%
\pgfpathlineto{\pgfqpoint{3.390722in}{0.773588in}}%
\pgfpathlineto{\pgfqpoint{3.375355in}{0.773588in}}%
\pgfpathlineto{\pgfqpoint{3.359992in}{0.773588in}}%
\pgfpathlineto{\pgfqpoint{3.344594in}{0.773588in}}%
\pgfpathlineto{\pgfqpoint{3.328928in}{0.773588in}}%
\pgfpathlineto{\pgfqpoint{3.313329in}{0.773588in}}%
\pgfpathlineto{\pgfqpoint{3.295281in}{0.773588in}}%
\pgfpathlineto{\pgfqpoint{3.279179in}{0.773588in}}%
\pgfpathlineto{\pgfqpoint{3.263722in}{0.773588in}}%
\pgfpathlineto{\pgfqpoint{3.248207in}{0.773588in}}%
\pgfpathlineto{\pgfqpoint{3.232763in}{0.773588in}}%
\pgfpathlineto{\pgfqpoint{3.217213in}{0.773588in}}%
\pgfpathlineto{\pgfqpoint{3.201602in}{0.773588in}}%
\pgfpathlineto{\pgfqpoint{3.185752in}{0.773588in}}%
\pgfpathlineto{\pgfqpoint{3.169651in}{0.773588in}}%
\pgfpathlineto{\pgfqpoint{3.152584in}{0.773588in}}%
\pgfpathlineto{\pgfqpoint{3.135319in}{0.773588in}}%
\pgfpathlineto{\pgfqpoint{3.118692in}{0.773588in}}%
\pgfpathlineto{\pgfqpoint{3.102316in}{0.773588in}}%
\pgfpathlineto{\pgfqpoint{3.085686in}{0.773588in}}%
\pgfpathlineto{\pgfqpoint{3.069003in}{0.773588in}}%
\pgfpathlineto{\pgfqpoint{3.052708in}{0.773588in}}%
\pgfpathlineto{\pgfqpoint{3.036276in}{0.773588in}}%
\pgfpathlineto{\pgfqpoint{3.018968in}{0.773588in}}%
\pgfpathlineto{\pgfqpoint{3.001921in}{0.773588in}}%
\pgfpathlineto{\pgfqpoint{2.985853in}{0.773588in}}%
\pgfpathlineto{\pgfqpoint{2.968433in}{0.773588in}}%
\pgfpathlineto{\pgfqpoint{2.949130in}{0.773588in}}%
\pgfpathlineto{\pgfqpoint{2.930775in}{0.773588in}}%
\pgfpathlineto{\pgfqpoint{2.914048in}{0.773588in}}%
\pgfpathlineto{\pgfqpoint{2.897563in}{0.773588in}}%
\pgfpathlineto{\pgfqpoint{2.880921in}{0.773588in}}%
\pgfpathlineto{\pgfqpoint{2.864455in}{0.773588in}}%
\pgfpathlineto{\pgfqpoint{2.848081in}{0.773588in}}%
\pgfpathlineto{\pgfqpoint{2.831972in}{0.773588in}}%
\pgfpathlineto{\pgfqpoint{2.815750in}{0.773588in}}%
\pgfpathlineto{\pgfqpoint{2.798917in}{0.773588in}}%
\pgfpathlineto{\pgfqpoint{2.780862in}{0.773588in}}%
\pgfpathlineto{\pgfqpoint{2.763118in}{0.773588in}}%
\pgfpathlineto{\pgfqpoint{2.744840in}{0.773588in}}%
\pgfpathlineto{\pgfqpoint{2.727722in}{0.773588in}}%
\pgfpathlineto{\pgfqpoint{2.710992in}{0.773588in}}%
\pgfpathlineto{\pgfqpoint{2.693540in}{0.773588in}}%
\pgfpathlineto{\pgfqpoint{2.676022in}{0.773588in}}%
\pgfpathlineto{\pgfqpoint{2.657274in}{0.773588in}}%
\pgfpathlineto{\pgfqpoint{2.636069in}{0.773588in}}%
\pgfpathlineto{\pgfqpoint{2.616149in}{0.773588in}}%
\pgfpathlineto{\pgfqpoint{2.598064in}{0.773588in}}%
\pgfpathlineto{\pgfqpoint{2.578521in}{0.773588in}}%
\pgfpathlineto{\pgfqpoint{2.558886in}{0.773588in}}%
\pgfpathlineto{\pgfqpoint{2.538352in}{0.773588in}}%
\pgfpathlineto{\pgfqpoint{2.515773in}{0.773588in}}%
\pgfpathlineto{\pgfqpoint{2.494737in}{0.773588in}}%
\pgfpathlineto{\pgfqpoint{2.477723in}{0.773588in}}%
\pgfpathlineto{\pgfqpoint{2.457197in}{0.773588in}}%
\pgfpathlineto{\pgfqpoint{2.433282in}{0.773588in}}%
\pgfpathlineto{\pgfqpoint{2.418633in}{0.773588in}}%
\pgfpathlineto{\pgfqpoint{2.406207in}{0.773588in}}%
\pgfpathlineto{\pgfqpoint{2.393733in}{0.773588in}}%
\pgfpathlineto{\pgfqpoint{2.381372in}{0.773588in}}%
\pgfpathlineto{\pgfqpoint{2.368880in}{0.773588in}}%
\pgfpathlineto{\pgfqpoint{2.356417in}{0.773588in}}%
\pgfpathlineto{\pgfqpoint{2.343909in}{0.773588in}}%
\pgfpathlineto{\pgfqpoint{2.331395in}{0.773588in}}%
\pgfpathlineto{\pgfqpoint{2.318891in}{0.773588in}}%
\pgfpathlineto{\pgfqpoint{2.306399in}{0.773588in}}%
\pgfpathlineto{\pgfqpoint{2.293873in}{0.773588in}}%
\pgfpathlineto{\pgfqpoint{2.281255in}{0.773588in}}%
\pgfpathlineto{\pgfqpoint{2.268761in}{0.773588in}}%
\pgfpathlineto{\pgfqpoint{2.256287in}{0.773588in}}%
\pgfpathlineto{\pgfqpoint{2.243791in}{0.773588in}}%
\pgfpathlineto{\pgfqpoint{2.231301in}{0.773588in}}%
\pgfpathlineto{\pgfqpoint{2.218726in}{0.773588in}}%
\pgfpathlineto{\pgfqpoint{2.206174in}{0.773588in}}%
\pgfpathlineto{\pgfqpoint{2.193673in}{0.773588in}}%
\pgfpathlineto{\pgfqpoint{2.181172in}{0.773588in}}%
\pgfpathlineto{\pgfqpoint{2.168659in}{0.773588in}}%
\pgfpathlineto{\pgfqpoint{2.156064in}{0.773588in}}%
\pgfpathlineto{\pgfqpoint{2.143512in}{0.773588in}}%
\pgfpathlineto{\pgfqpoint{2.130987in}{0.773588in}}%
\pgfpathlineto{\pgfqpoint{2.118500in}{0.773588in}}%
\pgfpathlineto{\pgfqpoint{2.106029in}{0.773588in}}%
\pgfpathlineto{\pgfqpoint{2.093434in}{0.773588in}}%
\pgfpathlineto{\pgfqpoint{2.080914in}{0.773588in}}%
\pgfpathlineto{\pgfqpoint{2.068360in}{0.773588in}}%
\pgfpathlineto{\pgfqpoint{2.055871in}{0.773588in}}%
\pgfpathlineto{\pgfqpoint{2.043350in}{0.773588in}}%
\pgfpathlineto{\pgfqpoint{2.030840in}{0.773588in}}%
\pgfpathlineto{\pgfqpoint{2.016889in}{0.773588in}}%
\pgfpathlineto{\pgfqpoint{2.004212in}{0.773588in}}%
\pgfpathlineto{\pgfqpoint{1.991646in}{0.773588in}}%
\pgfpathlineto{\pgfqpoint{1.979161in}{0.773588in}}%
\pgfpathlineto{\pgfqpoint{1.966664in}{0.773588in}}%
\pgfpathlineto{\pgfqpoint{1.954108in}{0.773588in}}%
\pgfpathlineto{\pgfqpoint{1.941620in}{0.773588in}}%
\pgfpathlineto{\pgfqpoint{1.929118in}{0.773588in}}%
\pgfpathlineto{\pgfqpoint{1.916611in}{0.773588in}}%
\pgfpathlineto{\pgfqpoint{1.904130in}{0.773588in}}%
\pgfpathlineto{\pgfqpoint{1.891620in}{0.773588in}}%
\pgfpathlineto{\pgfqpoint{1.879180in}{0.773588in}}%
\pgfpathlineto{\pgfqpoint{1.866747in}{0.773588in}}%
\pgfpathlineto{\pgfqpoint{1.854433in}{0.773588in}}%
\pgfpathlineto{\pgfqpoint{1.842065in}{0.773588in}}%
\pgfpathlineto{\pgfqpoint{1.829628in}{0.773588in}}%
\pgfpathlineto{\pgfqpoint{1.817148in}{0.773588in}}%
\pgfpathlineto{\pgfqpoint{1.804716in}{0.773588in}}%
\pgfpathlineto{\pgfqpoint{1.792355in}{0.773588in}}%
\pgfpathlineto{\pgfqpoint{1.779913in}{0.773588in}}%
\pgfpathlineto{\pgfqpoint{1.767480in}{0.773588in}}%
\pgfpathlineto{\pgfqpoint{1.755070in}{0.773588in}}%
\pgfpathlineto{\pgfqpoint{1.742593in}{0.773588in}}%
\pgfpathlineto{\pgfqpoint{1.730200in}{0.773588in}}%
\pgfpathlineto{\pgfqpoint{1.717832in}{0.773588in}}%
\pgfpathlineto{\pgfqpoint{1.705498in}{0.773588in}}%
\pgfpathlineto{\pgfqpoint{1.693005in}{0.773588in}}%
\pgfpathlineto{\pgfqpoint{1.680578in}{0.773588in}}%
\pgfpathlineto{\pgfqpoint{1.668187in}{0.773588in}}%
\pgfpathlineto{\pgfqpoint{1.655673in}{0.773588in}}%
\pgfpathlineto{\pgfqpoint{1.643249in}{0.773588in}}%
\pgfpathlineto{\pgfqpoint{1.630788in}{0.773588in}}%
\pgfpathlineto{\pgfqpoint{1.618336in}{0.773588in}}%
\pgfpathlineto{\pgfqpoint{1.605869in}{0.773588in}}%
\pgfpathlineto{\pgfqpoint{1.593430in}{0.773588in}}%
\pgfpathlineto{\pgfqpoint{1.581070in}{0.773588in}}%
\pgfpathlineto{\pgfqpoint{1.568672in}{0.773588in}}%
\pgfpathlineto{\pgfqpoint{1.556290in}{0.773588in}}%
\pgfpathlineto{\pgfqpoint{1.543839in}{0.773588in}}%
\pgfpathlineto{\pgfqpoint{1.531474in}{0.773588in}}%
\pgfpathlineto{\pgfqpoint{1.519060in}{0.773588in}}%
\pgfpathlineto{\pgfqpoint{1.506021in}{0.773588in}}%
\pgfpathlineto{\pgfqpoint{1.493578in}{0.773588in}}%
\pgfpathlineto{\pgfqpoint{1.481163in}{0.773588in}}%
\pgfpathlineto{\pgfqpoint{1.468660in}{0.773588in}}%
\pgfpathlineto{\pgfqpoint{1.456188in}{0.773588in}}%
\pgfpathlineto{\pgfqpoint{1.443756in}{0.773588in}}%
\pgfpathlineto{\pgfqpoint{1.431382in}{0.773588in}}%
\pgfpathlineto{\pgfqpoint{1.418893in}{0.773588in}}%
\pgfpathlineto{\pgfqpoint{1.406401in}{0.773588in}}%
\pgfpathlineto{\pgfqpoint{1.393906in}{0.773588in}}%
\pgfpathlineto{\pgfqpoint{1.381448in}{0.773588in}}%
\pgfpathlineto{\pgfqpoint{1.368992in}{0.773588in}}%
\pgfpathlineto{\pgfqpoint{1.356458in}{0.773588in}}%
\pgfpathlineto{\pgfqpoint{1.343969in}{0.773588in}}%
\pgfpathlineto{\pgfqpoint{1.331552in}{0.773588in}}%
\pgfpathlineto{\pgfqpoint{1.319073in}{0.773588in}}%
\pgfpathlineto{\pgfqpoint{1.306603in}{0.773588in}}%
\pgfpathlineto{\pgfqpoint{1.294165in}{0.773588in}}%
\pgfpathlineto{\pgfqpoint{1.281753in}{0.773588in}}%
\pgfpathlineto{\pgfqpoint{1.269255in}{0.773588in}}%
\pgfpathlineto{\pgfqpoint{1.256724in}{0.773588in}}%
\pgfpathlineto{\pgfqpoint{1.244104in}{0.773588in}}%
\pgfpathlineto{\pgfqpoint{1.231699in}{0.773588in}}%
\pgfpathlineto{\pgfqpoint{1.219183in}{0.773588in}}%
\pgfpathlineto{\pgfqpoint{1.206622in}{0.773588in}}%
\pgfpathlineto{\pgfqpoint{1.194024in}{0.773588in}}%
\pgfpathlineto{\pgfqpoint{1.181503in}{0.773588in}}%
\pgfpathlineto{\pgfqpoint{1.168924in}{0.773588in}}%
\pgfpathlineto{\pgfqpoint{1.156226in}{0.773588in}}%
\pgfpathlineto{\pgfqpoint{1.143621in}{0.773588in}}%
\pgfpathlineto{\pgfqpoint{1.131014in}{0.773588in}}%
\pgfpathlineto{\pgfqpoint{1.118303in}{0.773588in}}%
\pgfpathlineto{\pgfqpoint{1.105655in}{0.773588in}}%
\pgfpathlineto{\pgfqpoint{1.092853in}{0.773588in}}%
\pgfpathlineto{\pgfqpoint{1.079972in}{0.773588in}}%
\pgfpathlineto{\pgfqpoint{1.067143in}{0.773588in}}%
\pgfpathlineto{\pgfqpoint{1.054264in}{0.773588in}}%
\pgfpathlineto{\pgfqpoint{1.041322in}{0.773588in}}%
\pgfpathlineto{\pgfqpoint{1.028303in}{0.773588in}}%
\pgfpathlineto{\pgfqpoint{1.015150in}{0.773588in}}%
\pgfpathlineto{\pgfqpoint{1.001616in}{0.773588in}}%
\pgfpathclose%
\pgfusepath{fill}%
\end{pgfscope}%
\begin{pgfscope}%
\pgfpathrectangle{\pgfqpoint{0.781402in}{0.773588in}}{\pgfqpoint{4.844695in}{5.415119in}}%
\pgfusepath{clip}%
\pgfsetbuttcap%
\pgfsetroundjoin%
\definecolor{currentfill}{rgb}{0.172549,0.627451,0.172549}%
\pgfsetfillcolor{currentfill}%
\pgfsetlinewidth{0.000000pt}%
\definecolor{currentstroke}{rgb}{0.000000,0.000000,0.000000}%
\pgfsetstrokecolor{currentstroke}%
\pgfsetdash{}{0pt}%
\pgfpathmoveto{\pgfqpoint{1.001616in}{1.301014in}}%
\pgfpathlineto{\pgfqpoint{1.001616in}{0.773588in}}%
\pgfpathlineto{\pgfqpoint{1.015150in}{0.773588in}}%
\pgfpathlineto{\pgfqpoint{1.028303in}{0.773588in}}%
\pgfpathlineto{\pgfqpoint{1.041322in}{0.773588in}}%
\pgfpathlineto{\pgfqpoint{1.054264in}{0.773588in}}%
\pgfpathlineto{\pgfqpoint{1.067143in}{0.773588in}}%
\pgfpathlineto{\pgfqpoint{1.079972in}{0.773588in}}%
\pgfpathlineto{\pgfqpoint{1.092853in}{0.773588in}}%
\pgfpathlineto{\pgfqpoint{1.105655in}{0.773588in}}%
\pgfpathlineto{\pgfqpoint{1.118303in}{0.773588in}}%
\pgfpathlineto{\pgfqpoint{1.131014in}{0.773588in}}%
\pgfpathlineto{\pgfqpoint{1.143621in}{0.773588in}}%
\pgfpathlineto{\pgfqpoint{1.156226in}{0.773588in}}%
\pgfpathlineto{\pgfqpoint{1.168924in}{0.773588in}}%
\pgfpathlineto{\pgfqpoint{1.181503in}{0.773588in}}%
\pgfpathlineto{\pgfqpoint{1.194024in}{0.773588in}}%
\pgfpathlineto{\pgfqpoint{1.206622in}{0.773588in}}%
\pgfpathlineto{\pgfqpoint{1.219183in}{0.773588in}}%
\pgfpathlineto{\pgfqpoint{1.231699in}{0.773588in}}%
\pgfpathlineto{\pgfqpoint{1.244104in}{0.773588in}}%
\pgfpathlineto{\pgfqpoint{1.256724in}{0.773588in}}%
\pgfpathlineto{\pgfqpoint{1.269255in}{0.773588in}}%
\pgfpathlineto{\pgfqpoint{1.281753in}{0.773588in}}%
\pgfpathlineto{\pgfqpoint{1.294165in}{0.773588in}}%
\pgfpathlineto{\pgfqpoint{1.306603in}{0.773588in}}%
\pgfpathlineto{\pgfqpoint{1.319073in}{0.773588in}}%
\pgfpathlineto{\pgfqpoint{1.331552in}{0.773588in}}%
\pgfpathlineto{\pgfqpoint{1.343969in}{0.773588in}}%
\pgfpathlineto{\pgfqpoint{1.356458in}{0.773588in}}%
\pgfpathlineto{\pgfqpoint{1.368992in}{0.773588in}}%
\pgfpathlineto{\pgfqpoint{1.381448in}{0.773588in}}%
\pgfpathlineto{\pgfqpoint{1.393906in}{0.773588in}}%
\pgfpathlineto{\pgfqpoint{1.406401in}{0.773588in}}%
\pgfpathlineto{\pgfqpoint{1.418893in}{0.773588in}}%
\pgfpathlineto{\pgfqpoint{1.431382in}{0.773588in}}%
\pgfpathlineto{\pgfqpoint{1.443756in}{0.773588in}}%
\pgfpathlineto{\pgfqpoint{1.456188in}{0.773588in}}%
\pgfpathlineto{\pgfqpoint{1.468660in}{0.773588in}}%
\pgfpathlineto{\pgfqpoint{1.481163in}{0.773588in}}%
\pgfpathlineto{\pgfqpoint{1.493578in}{0.773588in}}%
\pgfpathlineto{\pgfqpoint{1.506021in}{0.773588in}}%
\pgfpathlineto{\pgfqpoint{1.519060in}{0.773588in}}%
\pgfpathlineto{\pgfqpoint{1.531474in}{0.773588in}}%
\pgfpathlineto{\pgfqpoint{1.543839in}{0.773588in}}%
\pgfpathlineto{\pgfqpoint{1.556290in}{0.773588in}}%
\pgfpathlineto{\pgfqpoint{1.568672in}{0.773588in}}%
\pgfpathlineto{\pgfqpoint{1.581070in}{0.773588in}}%
\pgfpathlineto{\pgfqpoint{1.593430in}{0.773588in}}%
\pgfpathlineto{\pgfqpoint{1.605869in}{0.773588in}}%
\pgfpathlineto{\pgfqpoint{1.618336in}{0.773588in}}%
\pgfpathlineto{\pgfqpoint{1.630788in}{0.773588in}}%
\pgfpathlineto{\pgfqpoint{1.643249in}{0.773588in}}%
\pgfpathlineto{\pgfqpoint{1.655673in}{0.773588in}}%
\pgfpathlineto{\pgfqpoint{1.668187in}{0.773588in}}%
\pgfpathlineto{\pgfqpoint{1.680578in}{0.773588in}}%
\pgfpathlineto{\pgfqpoint{1.693005in}{0.773588in}}%
\pgfpathlineto{\pgfqpoint{1.705498in}{0.773588in}}%
\pgfpathlineto{\pgfqpoint{1.717832in}{0.773588in}}%
\pgfpathlineto{\pgfqpoint{1.730200in}{0.773588in}}%
\pgfpathlineto{\pgfqpoint{1.742593in}{0.773588in}}%
\pgfpathlineto{\pgfqpoint{1.755070in}{0.773588in}}%
\pgfpathlineto{\pgfqpoint{1.767480in}{0.773588in}}%
\pgfpathlineto{\pgfqpoint{1.779913in}{0.773588in}}%
\pgfpathlineto{\pgfqpoint{1.792355in}{0.773588in}}%
\pgfpathlineto{\pgfqpoint{1.804716in}{0.773588in}}%
\pgfpathlineto{\pgfqpoint{1.817148in}{0.773588in}}%
\pgfpathlineto{\pgfqpoint{1.829628in}{0.773588in}}%
\pgfpathlineto{\pgfqpoint{1.842065in}{0.773588in}}%
\pgfpathlineto{\pgfqpoint{1.854433in}{0.773588in}}%
\pgfpathlineto{\pgfqpoint{1.866747in}{0.773588in}}%
\pgfpathlineto{\pgfqpoint{1.879180in}{0.773588in}}%
\pgfpathlineto{\pgfqpoint{1.891620in}{0.773588in}}%
\pgfpathlineto{\pgfqpoint{1.904130in}{0.773588in}}%
\pgfpathlineto{\pgfqpoint{1.916611in}{0.773588in}}%
\pgfpathlineto{\pgfqpoint{1.929118in}{0.773588in}}%
\pgfpathlineto{\pgfqpoint{1.941620in}{0.773588in}}%
\pgfpathlineto{\pgfqpoint{1.954108in}{0.773588in}}%
\pgfpathlineto{\pgfqpoint{1.966664in}{0.773588in}}%
\pgfpathlineto{\pgfqpoint{1.979161in}{0.773588in}}%
\pgfpathlineto{\pgfqpoint{1.991646in}{0.773588in}}%
\pgfpathlineto{\pgfqpoint{2.004212in}{0.773588in}}%
\pgfpathlineto{\pgfqpoint{2.016889in}{0.773588in}}%
\pgfpathlineto{\pgfqpoint{2.030840in}{0.773588in}}%
\pgfpathlineto{\pgfqpoint{2.043350in}{0.773588in}}%
\pgfpathlineto{\pgfqpoint{2.055871in}{0.773588in}}%
\pgfpathlineto{\pgfqpoint{2.068360in}{0.773588in}}%
\pgfpathlineto{\pgfqpoint{2.080914in}{0.773588in}}%
\pgfpathlineto{\pgfqpoint{2.093434in}{0.773588in}}%
\pgfpathlineto{\pgfqpoint{2.106029in}{0.773588in}}%
\pgfpathlineto{\pgfqpoint{2.118500in}{0.773588in}}%
\pgfpathlineto{\pgfqpoint{2.130987in}{0.773588in}}%
\pgfpathlineto{\pgfqpoint{2.143512in}{0.773588in}}%
\pgfpathlineto{\pgfqpoint{2.156064in}{0.773588in}}%
\pgfpathlineto{\pgfqpoint{2.168659in}{0.773588in}}%
\pgfpathlineto{\pgfqpoint{2.181172in}{0.773588in}}%
\pgfpathlineto{\pgfqpoint{2.193673in}{0.773588in}}%
\pgfpathlineto{\pgfqpoint{2.206174in}{0.773588in}}%
\pgfpathlineto{\pgfqpoint{2.218726in}{0.773588in}}%
\pgfpathlineto{\pgfqpoint{2.231301in}{0.773588in}}%
\pgfpathlineto{\pgfqpoint{2.243791in}{0.773588in}}%
\pgfpathlineto{\pgfqpoint{2.256287in}{0.773588in}}%
\pgfpathlineto{\pgfqpoint{2.268761in}{0.773588in}}%
\pgfpathlineto{\pgfqpoint{2.281255in}{0.773588in}}%
\pgfpathlineto{\pgfqpoint{2.293873in}{0.773588in}}%
\pgfpathlineto{\pgfqpoint{2.306399in}{0.773588in}}%
\pgfpathlineto{\pgfqpoint{2.318891in}{0.773588in}}%
\pgfpathlineto{\pgfqpoint{2.331395in}{0.773588in}}%
\pgfpathlineto{\pgfqpoint{2.343909in}{0.773588in}}%
\pgfpathlineto{\pgfqpoint{2.356417in}{0.773588in}}%
\pgfpathlineto{\pgfqpoint{2.368880in}{0.773588in}}%
\pgfpathlineto{\pgfqpoint{2.381372in}{0.773588in}}%
\pgfpathlineto{\pgfqpoint{2.393733in}{0.773588in}}%
\pgfpathlineto{\pgfqpoint{2.406207in}{0.773588in}}%
\pgfpathlineto{\pgfqpoint{2.418633in}{0.773588in}}%
\pgfpathlineto{\pgfqpoint{2.433282in}{0.773588in}}%
\pgfpathlineto{\pgfqpoint{2.457197in}{0.773588in}}%
\pgfpathlineto{\pgfqpoint{2.477723in}{0.773588in}}%
\pgfpathlineto{\pgfqpoint{2.494737in}{0.773588in}}%
\pgfpathlineto{\pgfqpoint{2.515773in}{0.773588in}}%
\pgfpathlineto{\pgfqpoint{2.538352in}{0.773588in}}%
\pgfpathlineto{\pgfqpoint{2.558886in}{0.773588in}}%
\pgfpathlineto{\pgfqpoint{2.578521in}{0.773588in}}%
\pgfpathlineto{\pgfqpoint{2.598064in}{0.773588in}}%
\pgfpathlineto{\pgfqpoint{2.616149in}{0.773588in}}%
\pgfpathlineto{\pgfqpoint{2.636069in}{0.773588in}}%
\pgfpathlineto{\pgfqpoint{2.657274in}{0.773588in}}%
\pgfpathlineto{\pgfqpoint{2.676022in}{0.773588in}}%
\pgfpathlineto{\pgfqpoint{2.693540in}{0.773588in}}%
\pgfpathlineto{\pgfqpoint{2.710992in}{0.773588in}}%
\pgfpathlineto{\pgfqpoint{2.727722in}{0.773588in}}%
\pgfpathlineto{\pgfqpoint{2.744840in}{0.773588in}}%
\pgfpathlineto{\pgfqpoint{2.763118in}{0.773588in}}%
\pgfpathlineto{\pgfqpoint{2.780862in}{0.773588in}}%
\pgfpathlineto{\pgfqpoint{2.798917in}{0.773588in}}%
\pgfpathlineto{\pgfqpoint{2.815750in}{0.773588in}}%
\pgfpathlineto{\pgfqpoint{2.831972in}{0.773588in}}%
\pgfpathlineto{\pgfqpoint{2.848081in}{0.773588in}}%
\pgfpathlineto{\pgfqpoint{2.864455in}{0.773588in}}%
\pgfpathlineto{\pgfqpoint{2.880921in}{0.773588in}}%
\pgfpathlineto{\pgfqpoint{2.897563in}{0.773588in}}%
\pgfpathlineto{\pgfqpoint{2.914048in}{0.773588in}}%
\pgfpathlineto{\pgfqpoint{2.930775in}{0.773588in}}%
\pgfpathlineto{\pgfqpoint{2.949130in}{0.773588in}}%
\pgfpathlineto{\pgfqpoint{2.968433in}{0.773588in}}%
\pgfpathlineto{\pgfqpoint{2.985853in}{0.773588in}}%
\pgfpathlineto{\pgfqpoint{3.001921in}{0.773588in}}%
\pgfpathlineto{\pgfqpoint{3.018968in}{0.773588in}}%
\pgfpathlineto{\pgfqpoint{3.036276in}{0.773588in}}%
\pgfpathlineto{\pgfqpoint{3.052708in}{0.773588in}}%
\pgfpathlineto{\pgfqpoint{3.069003in}{0.773588in}}%
\pgfpathlineto{\pgfqpoint{3.085686in}{0.773588in}}%
\pgfpathlineto{\pgfqpoint{3.102316in}{0.773588in}}%
\pgfpathlineto{\pgfqpoint{3.118692in}{0.773588in}}%
\pgfpathlineto{\pgfqpoint{3.135319in}{0.773588in}}%
\pgfpathlineto{\pgfqpoint{3.152584in}{0.773588in}}%
\pgfpathlineto{\pgfqpoint{3.169651in}{0.773588in}}%
\pgfpathlineto{\pgfqpoint{3.185752in}{0.773588in}}%
\pgfpathlineto{\pgfqpoint{3.201602in}{0.773588in}}%
\pgfpathlineto{\pgfqpoint{3.217213in}{0.773588in}}%
\pgfpathlineto{\pgfqpoint{3.232763in}{0.773588in}}%
\pgfpathlineto{\pgfqpoint{3.248207in}{0.773588in}}%
\pgfpathlineto{\pgfqpoint{3.263722in}{0.773588in}}%
\pgfpathlineto{\pgfqpoint{3.279179in}{0.773588in}}%
\pgfpathlineto{\pgfqpoint{3.295281in}{0.773588in}}%
\pgfpathlineto{\pgfqpoint{3.313329in}{0.773588in}}%
\pgfpathlineto{\pgfqpoint{3.328928in}{0.773588in}}%
\pgfpathlineto{\pgfqpoint{3.344594in}{0.773588in}}%
\pgfpathlineto{\pgfqpoint{3.359992in}{0.773588in}}%
\pgfpathlineto{\pgfqpoint{3.375355in}{0.773588in}}%
\pgfpathlineto{\pgfqpoint{3.390722in}{0.773588in}}%
\pgfpathlineto{\pgfqpoint{3.405908in}{0.773588in}}%
\pgfpathlineto{\pgfqpoint{3.421182in}{0.773588in}}%
\pgfpathlineto{\pgfqpoint{3.436431in}{0.773588in}}%
\pgfpathlineto{\pgfqpoint{3.451777in}{0.773588in}}%
\pgfpathlineto{\pgfqpoint{3.467311in}{0.773588in}}%
\pgfpathlineto{\pgfqpoint{3.483019in}{0.773588in}}%
\pgfpathlineto{\pgfqpoint{3.498190in}{0.773588in}}%
\pgfpathlineto{\pgfqpoint{3.513323in}{0.773588in}}%
\pgfpathlineto{\pgfqpoint{3.528239in}{0.773588in}}%
\pgfpathlineto{\pgfqpoint{3.542041in}{0.773588in}}%
\pgfpathlineto{\pgfqpoint{3.554208in}{0.773588in}}%
\pgfpathlineto{\pgfqpoint{3.566335in}{0.773588in}}%
\pgfpathlineto{\pgfqpoint{3.578260in}{0.773588in}}%
\pgfpathlineto{\pgfqpoint{3.590359in}{0.773588in}}%
\pgfpathlineto{\pgfqpoint{3.602917in}{0.773588in}}%
\pgfpathlineto{\pgfqpoint{3.615064in}{0.773588in}}%
\pgfpathlineto{\pgfqpoint{3.627108in}{0.773588in}}%
\pgfpathlineto{\pgfqpoint{3.639052in}{0.773588in}}%
\pgfpathlineto{\pgfqpoint{3.651005in}{0.773588in}}%
\pgfpathlineto{\pgfqpoint{3.663361in}{0.773588in}}%
\pgfpathlineto{\pgfqpoint{3.675639in}{0.773588in}}%
\pgfpathlineto{\pgfqpoint{3.687677in}{0.773588in}}%
\pgfpathlineto{\pgfqpoint{3.699766in}{0.773588in}}%
\pgfpathlineto{\pgfqpoint{3.711790in}{0.773588in}}%
\pgfpathlineto{\pgfqpoint{3.723925in}{0.773588in}}%
\pgfpathlineto{\pgfqpoint{3.736218in}{0.773588in}}%
\pgfpathlineto{\pgfqpoint{3.748246in}{0.773588in}}%
\pgfpathlineto{\pgfqpoint{3.760252in}{0.773588in}}%
\pgfpathlineto{\pgfqpoint{3.772194in}{0.773588in}}%
\pgfpathlineto{\pgfqpoint{3.784213in}{0.773588in}}%
\pgfpathlineto{\pgfqpoint{3.796179in}{0.773588in}}%
\pgfpathlineto{\pgfqpoint{3.807700in}{0.773588in}}%
\pgfpathlineto{\pgfqpoint{3.819207in}{0.773588in}}%
\pgfpathlineto{\pgfqpoint{3.830748in}{0.773588in}}%
\pgfpathlineto{\pgfqpoint{3.842268in}{0.773588in}}%
\pgfpathlineto{\pgfqpoint{3.853773in}{0.773588in}}%
\pgfpathlineto{\pgfqpoint{3.865394in}{0.773588in}}%
\pgfpathlineto{\pgfqpoint{3.876999in}{0.773588in}}%
\pgfpathlineto{\pgfqpoint{3.888612in}{0.773588in}}%
\pgfpathlineto{\pgfqpoint{3.900120in}{0.773588in}}%
\pgfpathlineto{\pgfqpoint{3.911548in}{0.773588in}}%
\pgfpathlineto{\pgfqpoint{3.923007in}{0.773588in}}%
\pgfpathlineto{\pgfqpoint{3.934526in}{0.773588in}}%
\pgfpathlineto{\pgfqpoint{3.945995in}{0.773588in}}%
\pgfpathlineto{\pgfqpoint{3.957345in}{0.773588in}}%
\pgfpathlineto{\pgfqpoint{3.968800in}{0.773588in}}%
\pgfpathlineto{\pgfqpoint{3.980165in}{0.773588in}}%
\pgfpathlineto{\pgfqpoint{3.991607in}{0.773588in}}%
\pgfpathlineto{\pgfqpoint{4.003035in}{0.773588in}}%
\pgfpathlineto{\pgfqpoint{4.014312in}{0.773588in}}%
\pgfpathlineto{\pgfqpoint{4.025578in}{0.780138in}}%
\pgfpathlineto{\pgfqpoint{4.036918in}{0.781674in}}%
\pgfpathlineto{\pgfqpoint{4.048157in}{0.784374in}}%
\pgfpathlineto{\pgfqpoint{4.059512in}{0.793411in}}%
\pgfpathlineto{\pgfqpoint{4.070994in}{0.793986in}}%
\pgfpathlineto{\pgfqpoint{4.082264in}{0.797517in}}%
\pgfpathlineto{\pgfqpoint{4.093537in}{0.797517in}}%
\pgfpathlineto{\pgfqpoint{4.104772in}{0.795244in}}%
\pgfpathlineto{\pgfqpoint{4.116014in}{0.793664in}}%
\pgfpathlineto{\pgfqpoint{4.127322in}{0.795009in}}%
\pgfpathlineto{\pgfqpoint{4.138598in}{0.797025in}}%
\pgfpathlineto{\pgfqpoint{4.149813in}{0.800795in}}%
\pgfpathlineto{\pgfqpoint{4.160983in}{0.811150in}}%
\pgfpathlineto{\pgfqpoint{4.172101in}{0.908291in}}%
\pgfpathlineto{\pgfqpoint{4.183285in}{0.863827in}}%
\pgfpathlineto{\pgfqpoint{4.194400in}{0.833098in}}%
\pgfpathlineto{\pgfqpoint{4.205594in}{0.826004in}}%
\pgfpathlineto{\pgfqpoint{4.216619in}{0.843961in}}%
\pgfpathlineto{\pgfqpoint{4.227703in}{0.918144in}}%
\pgfpathlineto{\pgfqpoint{4.238844in}{0.883505in}}%
\pgfpathlineto{\pgfqpoint{4.249866in}{0.934646in}}%
\pgfpathlineto{\pgfqpoint{4.260400in}{1.726232in}}%
\pgfpathlineto{\pgfqpoint{4.270972in}{1.383893in}}%
\pgfpathlineto{\pgfqpoint{4.280743in}{1.935681in}}%
\pgfpathlineto{\pgfqpoint{4.290581in}{1.942847in}}%
\pgfpathlineto{\pgfqpoint{4.300486in}{1.974786in}}%
\pgfpathlineto{\pgfqpoint{4.310022in}{1.901955in}}%
\pgfpathlineto{\pgfqpoint{4.319388in}{1.956940in}}%
\pgfpathlineto{\pgfqpoint{4.328717in}{2.020191in}}%
\pgfpathlineto{\pgfqpoint{4.337933in}{2.066631in}}%
\pgfpathlineto{\pgfqpoint{4.346774in}{2.677129in}}%
\pgfpathlineto{\pgfqpoint{4.355507in}{3.059746in}}%
\pgfpathlineto{\pgfqpoint{4.364219in}{3.046040in}}%
\pgfpathlineto{\pgfqpoint{4.373001in}{3.018792in}}%
\pgfpathlineto{\pgfqpoint{4.381765in}{3.143095in}}%
\pgfpathlineto{\pgfqpoint{4.390451in}{3.089611in}}%
\pgfpathlineto{\pgfqpoint{4.399156in}{3.110432in}}%
\pgfpathlineto{\pgfqpoint{4.407811in}{3.105200in}}%
\pgfpathlineto{\pgfqpoint{4.416522in}{3.107639in}}%
\pgfpathlineto{\pgfqpoint{4.425248in}{3.071021in}}%
\pgfpathlineto{\pgfqpoint{4.433894in}{3.124355in}}%
\pgfpathlineto{\pgfqpoint{4.442591in}{3.118146in}}%
\pgfpathlineto{\pgfqpoint{4.451305in}{3.094758in}}%
\pgfpathlineto{\pgfqpoint{4.460011in}{3.141723in}}%
\pgfpathlineto{\pgfqpoint{4.468771in}{3.109889in}}%
\pgfpathlineto{\pgfqpoint{4.477466in}{3.143579in}}%
\pgfpathlineto{\pgfqpoint{4.486071in}{3.150413in}}%
\pgfpathlineto{\pgfqpoint{4.494752in}{3.112896in}}%
\pgfpathlineto{\pgfqpoint{4.503421in}{3.134994in}}%
\pgfpathlineto{\pgfqpoint{4.512084in}{3.113662in}}%
\pgfpathlineto{\pgfqpoint{4.520678in}{3.156306in}}%
\pgfpathlineto{\pgfqpoint{4.529276in}{3.116368in}}%
\pgfpathlineto{\pgfqpoint{4.537876in}{3.173512in}}%
\pgfpathlineto{\pgfqpoint{4.546439in}{3.170301in}}%
\pgfpathlineto{\pgfqpoint{4.554980in}{3.152790in}}%
\pgfpathlineto{\pgfqpoint{4.563550in}{3.142969in}}%
\pgfpathlineto{\pgfqpoint{4.572097in}{3.145048in}}%
\pgfpathlineto{\pgfqpoint{4.580727in}{3.165409in}}%
\pgfpathlineto{\pgfqpoint{4.589255in}{3.183262in}}%
\pgfpathlineto{\pgfqpoint{4.597719in}{3.195589in}}%
\pgfpathlineto{\pgfqpoint{4.606211in}{3.140769in}}%
\pgfpathlineto{\pgfqpoint{4.614783in}{3.173756in}}%
\pgfpathlineto{\pgfqpoint{4.623282in}{3.196073in}}%
\pgfpathlineto{\pgfqpoint{4.631733in}{3.181830in}}%
\pgfpathlineto{\pgfqpoint{4.640221in}{3.125121in}}%
\pgfpathlineto{\pgfqpoint{4.648847in}{3.090296in}}%
\pgfpathlineto{\pgfqpoint{4.657349in}{3.191356in}}%
\pgfpathlineto{\pgfqpoint{4.665825in}{3.127243in}}%
\pgfpathlineto{\pgfqpoint{4.674255in}{3.170469in}}%
\pgfpathlineto{\pgfqpoint{4.682643in}{3.197268in}}%
\pgfpathlineto{\pgfqpoint{4.691071in}{3.221156in}}%
\pgfpathlineto{\pgfqpoint{4.699465in}{3.231656in}}%
\pgfpathlineto{\pgfqpoint{4.707865in}{3.200894in}}%
\pgfpathlineto{\pgfqpoint{4.716247in}{3.222424in}}%
\pgfpathlineto{\pgfqpoint{4.724666in}{3.181551in}}%
\pgfpathlineto{\pgfqpoint{4.733049in}{3.220826in}}%
\pgfpathlineto{\pgfqpoint{4.741378in}{3.178101in}}%
\pgfpathlineto{\pgfqpoint{4.749750in}{3.204103in}}%
\pgfpathlineto{\pgfqpoint{4.758119in}{3.199488in}}%
\pgfpathlineto{\pgfqpoint{4.766528in}{3.206972in}}%
\pgfpathlineto{\pgfqpoint{4.774881in}{3.221349in}}%
\pgfpathlineto{\pgfqpoint{4.783272in}{3.199967in}}%
\pgfpathlineto{\pgfqpoint{4.791573in}{3.206171in}}%
\pgfpathlineto{\pgfqpoint{4.799850in}{3.213907in}}%
\pgfpathlineto{\pgfqpoint{4.808134in}{3.215006in}}%
\pgfpathlineto{\pgfqpoint{4.816444in}{3.203512in}}%
\pgfpathlineto{\pgfqpoint{4.824772in}{3.210565in}}%
\pgfpathlineto{\pgfqpoint{4.833061in}{3.210599in}}%
\pgfpathlineto{\pgfqpoint{4.841325in}{3.245392in}}%
\pgfpathlineto{\pgfqpoint{4.849575in}{3.193009in}}%
\pgfpathlineto{\pgfqpoint{4.857830in}{3.248397in}}%
\pgfpathlineto{\pgfqpoint{4.866069in}{3.229537in}}%
\pgfpathlineto{\pgfqpoint{4.874381in}{3.199696in}}%
\pgfpathlineto{\pgfqpoint{4.882658in}{3.206152in}}%
\pgfpathlineto{\pgfqpoint{4.890980in}{3.234595in}}%
\pgfpathlineto{\pgfqpoint{4.899205in}{3.255818in}}%
\pgfpathlineto{\pgfqpoint{4.907476in}{3.206725in}}%
\pgfpathlineto{\pgfqpoint{4.915736in}{3.237631in}}%
\pgfpathlineto{\pgfqpoint{4.924041in}{3.195395in}}%
\pgfpathlineto{\pgfqpoint{4.932347in}{3.228796in}}%
\pgfpathlineto{\pgfqpoint{4.940566in}{3.192235in}}%
\pgfpathlineto{\pgfqpoint{4.948774in}{3.214016in}}%
\pgfpathlineto{\pgfqpoint{4.956992in}{3.201664in}}%
\pgfpathlineto{\pgfqpoint{4.965138in}{3.159630in}}%
\pgfpathlineto{\pgfqpoint{4.973428in}{3.193617in}}%
\pgfpathlineto{\pgfqpoint{4.981642in}{3.183139in}}%
\pgfpathlineto{\pgfqpoint{4.989794in}{3.202999in}}%
\pgfpathlineto{\pgfqpoint{4.997948in}{3.235491in}}%
\pgfpathlineto{\pgfqpoint{5.006134in}{3.190344in}}%
\pgfpathlineto{\pgfqpoint{5.014317in}{3.184898in}}%
\pgfpathlineto{\pgfqpoint{5.022451in}{3.224489in}}%
\pgfpathlineto{\pgfqpoint{5.030602in}{3.217335in}}%
\pgfpathlineto{\pgfqpoint{5.038759in}{3.202995in}}%
\pgfpathlineto{\pgfqpoint{5.046876in}{3.179675in}}%
\pgfpathlineto{\pgfqpoint{5.055031in}{3.228513in}}%
\pgfpathlineto{\pgfqpoint{5.063216in}{3.210993in}}%
\pgfpathlineto{\pgfqpoint{5.071391in}{3.180598in}}%
\pgfpathlineto{\pgfqpoint{5.079506in}{3.179265in}}%
\pgfpathlineto{\pgfqpoint{5.087675in}{3.218012in}}%
\pgfpathlineto{\pgfqpoint{5.095819in}{3.225752in}}%
\pgfpathlineto{\pgfqpoint{5.103859in}{3.216037in}}%
\pgfpathlineto{\pgfqpoint{5.111982in}{3.202225in}}%
\pgfpathlineto{\pgfqpoint{5.120128in}{3.221722in}}%
\pgfpathlineto{\pgfqpoint{5.128231in}{3.191598in}}%
\pgfpathlineto{\pgfqpoint{5.136344in}{3.237303in}}%
\pgfpathlineto{\pgfqpoint{5.144370in}{3.164581in}}%
\pgfpathlineto{\pgfqpoint{5.152426in}{3.167944in}}%
\pgfpathlineto{\pgfqpoint{5.160447in}{3.212349in}}%
\pgfpathlineto{\pgfqpoint{5.168547in}{3.206168in}}%
\pgfpathlineto{\pgfqpoint{5.176647in}{3.174325in}}%
\pgfpathlineto{\pgfqpoint{5.184747in}{3.237012in}}%
\pgfpathlineto{\pgfqpoint{5.192761in}{3.221387in}}%
\pgfpathlineto{\pgfqpoint{5.200810in}{3.196648in}}%
\pgfpathlineto{\pgfqpoint{5.208810in}{3.240732in}}%
\pgfpathlineto{\pgfqpoint{5.216856in}{3.224860in}}%
\pgfpathlineto{\pgfqpoint{5.224894in}{3.203050in}}%
\pgfpathlineto{\pgfqpoint{5.233040in}{3.223384in}}%
\pgfpathlineto{\pgfqpoint{5.245361in}{3.228733in}}%
\pgfpathlineto{\pgfqpoint{5.253575in}{3.185068in}}%
\pgfpathlineto{\pgfqpoint{5.261632in}{3.197890in}}%
\pgfpathlineto{\pgfqpoint{5.269666in}{3.174426in}}%
\pgfpathlineto{\pgfqpoint{5.277683in}{3.213693in}}%
\pgfpathlineto{\pgfqpoint{5.285748in}{3.211374in}}%
\pgfpathlineto{\pgfqpoint{5.293795in}{3.193697in}}%
\pgfpathlineto{\pgfqpoint{5.301824in}{3.231854in}}%
\pgfpathlineto{\pgfqpoint{5.309888in}{3.198667in}}%
\pgfpathlineto{\pgfqpoint{5.317933in}{3.215817in}}%
\pgfpathlineto{\pgfqpoint{5.327487in}{3.343828in}}%
\pgfpathlineto{\pgfqpoint{5.338668in}{3.343828in}}%
\pgfpathlineto{\pgfqpoint{5.349820in}{3.343828in}}%
\pgfpathlineto{\pgfqpoint{5.360970in}{3.343828in}}%
\pgfpathlineto{\pgfqpoint{5.372247in}{3.343828in}}%
\pgfpathlineto{\pgfqpoint{5.383480in}{3.343828in}}%
\pgfpathlineto{\pgfqpoint{5.394653in}{3.343828in}}%
\pgfpathlineto{\pgfqpoint{5.405885in}{3.343828in}}%
\pgfpathlineto{\pgfqpoint{5.405885in}{4.661008in}}%
\pgfpathlineto{\pgfqpoint{5.405885in}{4.661008in}}%
\pgfpathlineto{\pgfqpoint{5.394653in}{4.655725in}}%
\pgfpathlineto{\pgfqpoint{5.383480in}{4.676348in}}%
\pgfpathlineto{\pgfqpoint{5.372247in}{4.632160in}}%
\pgfpathlineto{\pgfqpoint{5.360970in}{4.641700in}}%
\pgfpathlineto{\pgfqpoint{5.349820in}{4.648972in}}%
\pgfpathlineto{\pgfqpoint{5.338668in}{4.645754in}}%
\pgfpathlineto{\pgfqpoint{5.327487in}{4.664334in}}%
\pgfpathlineto{\pgfqpoint{5.317933in}{4.519542in}}%
\pgfpathlineto{\pgfqpoint{5.309888in}{4.507169in}}%
\pgfpathlineto{\pgfqpoint{5.301824in}{4.554984in}}%
\pgfpathlineto{\pgfqpoint{5.293795in}{4.499284in}}%
\pgfpathlineto{\pgfqpoint{5.285748in}{4.533202in}}%
\pgfpathlineto{\pgfqpoint{5.277683in}{4.540375in}}%
\pgfpathlineto{\pgfqpoint{5.269666in}{4.477633in}}%
\pgfpathlineto{\pgfqpoint{5.261632in}{4.497161in}}%
\pgfpathlineto{\pgfqpoint{5.253575in}{4.503870in}}%
\pgfpathlineto{\pgfqpoint{5.245361in}{4.460963in}}%
\pgfpathlineto{\pgfqpoint{5.233040in}{4.561931in}}%
\pgfpathlineto{\pgfqpoint{5.224894in}{4.525459in}}%
\pgfpathlineto{\pgfqpoint{5.216856in}{4.529105in}}%
\pgfpathlineto{\pgfqpoint{5.208810in}{4.563601in}}%
\pgfpathlineto{\pgfqpoint{5.200810in}{4.513131in}}%
\pgfpathlineto{\pgfqpoint{5.192761in}{4.531508in}}%
\pgfpathlineto{\pgfqpoint{5.184747in}{4.559068in}}%
\pgfpathlineto{\pgfqpoint{5.176647in}{4.498693in}}%
\pgfpathlineto{\pgfqpoint{5.168547in}{4.531713in}}%
\pgfpathlineto{\pgfqpoint{5.160447in}{4.530772in}}%
\pgfpathlineto{\pgfqpoint{5.152426in}{4.481819in}}%
\pgfpathlineto{\pgfqpoint{5.144370in}{4.478541in}}%
\pgfpathlineto{\pgfqpoint{5.136344in}{4.550108in}}%
\pgfpathlineto{\pgfqpoint{5.128231in}{4.496253in}}%
\pgfpathlineto{\pgfqpoint{5.120128in}{4.532680in}}%
\pgfpathlineto{\pgfqpoint{5.111982in}{4.509880in}}%
\pgfpathlineto{\pgfqpoint{5.103859in}{4.531602in}}%
\pgfpathlineto{\pgfqpoint{5.095819in}{4.548270in}}%
\pgfpathlineto{\pgfqpoint{5.087675in}{4.519543in}}%
\pgfpathlineto{\pgfqpoint{5.079506in}{4.500618in}}%
\pgfpathlineto{\pgfqpoint{5.071391in}{4.484053in}}%
\pgfpathlineto{\pgfqpoint{5.063216in}{4.504166in}}%
\pgfpathlineto{\pgfqpoint{5.055031in}{4.513457in}}%
\pgfpathlineto{\pgfqpoint{5.046876in}{4.488292in}}%
\pgfpathlineto{\pgfqpoint{5.038759in}{4.517689in}}%
\pgfpathlineto{\pgfqpoint{5.030602in}{4.520239in}}%
\pgfpathlineto{\pgfqpoint{5.022451in}{4.541427in}}%
\pgfpathlineto{\pgfqpoint{5.014317in}{4.484651in}}%
\pgfpathlineto{\pgfqpoint{5.006134in}{4.496526in}}%
\pgfpathlineto{\pgfqpoint{4.997948in}{4.542287in}}%
\pgfpathlineto{\pgfqpoint{4.989794in}{4.523808in}}%
\pgfpathlineto{\pgfqpoint{4.981642in}{4.472070in}}%
\pgfpathlineto{\pgfqpoint{4.973428in}{4.495593in}}%
\pgfpathlineto{\pgfqpoint{4.965138in}{4.485935in}}%
\pgfpathlineto{\pgfqpoint{4.956992in}{4.505455in}}%
\pgfpathlineto{\pgfqpoint{4.948774in}{4.518747in}}%
\pgfpathlineto{\pgfqpoint{4.940566in}{4.488688in}}%
\pgfpathlineto{\pgfqpoint{4.932347in}{4.518769in}}%
\pgfpathlineto{\pgfqpoint{4.924041in}{4.478303in}}%
\pgfpathlineto{\pgfqpoint{4.915736in}{4.522807in}}%
\pgfpathlineto{\pgfqpoint{4.907476in}{4.498419in}}%
\pgfpathlineto{\pgfqpoint{4.899205in}{4.550643in}}%
\pgfpathlineto{\pgfqpoint{4.890980in}{4.529827in}}%
\pgfpathlineto{\pgfqpoint{4.882658in}{4.501882in}}%
\pgfpathlineto{\pgfqpoint{4.874381in}{4.471462in}}%
\pgfpathlineto{\pgfqpoint{4.866069in}{4.523872in}}%
\pgfpathlineto{\pgfqpoint{4.857830in}{4.575147in}}%
\pgfpathlineto{\pgfqpoint{4.849575in}{4.495651in}}%
\pgfpathlineto{\pgfqpoint{4.841325in}{4.554246in}}%
\pgfpathlineto{\pgfqpoint{4.833061in}{4.522476in}}%
\pgfpathlineto{\pgfqpoint{4.824772in}{4.478807in}}%
\pgfpathlineto{\pgfqpoint{4.816444in}{4.504794in}}%
\pgfpathlineto{\pgfqpoint{4.808134in}{4.526214in}}%
\pgfpathlineto{\pgfqpoint{4.799850in}{4.527335in}}%
\pgfpathlineto{\pgfqpoint{4.791573in}{4.490675in}}%
\pgfpathlineto{\pgfqpoint{4.783272in}{4.496865in}}%
\pgfpathlineto{\pgfqpoint{4.774881in}{4.522610in}}%
\pgfpathlineto{\pgfqpoint{4.766528in}{4.506900in}}%
\pgfpathlineto{\pgfqpoint{4.758119in}{4.480792in}}%
\pgfpathlineto{\pgfqpoint{4.749750in}{4.499415in}}%
\pgfpathlineto{\pgfqpoint{4.741378in}{4.484931in}}%
\pgfpathlineto{\pgfqpoint{4.733049in}{4.503171in}}%
\pgfpathlineto{\pgfqpoint{4.724666in}{4.474830in}}%
\pgfpathlineto{\pgfqpoint{4.716247in}{4.532645in}}%
\pgfpathlineto{\pgfqpoint{4.707865in}{4.480363in}}%
\pgfpathlineto{\pgfqpoint{4.699465in}{4.523338in}}%
\pgfpathlineto{\pgfqpoint{4.691071in}{4.512363in}}%
\pgfpathlineto{\pgfqpoint{4.682643in}{4.507740in}}%
\pgfpathlineto{\pgfqpoint{4.674255in}{4.483212in}}%
\pgfpathlineto{\pgfqpoint{4.665825in}{4.401763in}}%
\pgfpathlineto{\pgfqpoint{4.657349in}{4.473574in}}%
\pgfpathlineto{\pgfqpoint{4.648847in}{4.351666in}}%
\pgfpathlineto{\pgfqpoint{4.640221in}{4.405975in}}%
\pgfpathlineto{\pgfqpoint{4.631733in}{4.476753in}}%
\pgfpathlineto{\pgfqpoint{4.623282in}{4.487030in}}%
\pgfpathlineto{\pgfqpoint{4.614783in}{4.461635in}}%
\pgfpathlineto{\pgfqpoint{4.606211in}{4.429654in}}%
\pgfpathlineto{\pgfqpoint{4.597719in}{4.487330in}}%
\pgfpathlineto{\pgfqpoint{4.589255in}{4.484205in}}%
\pgfpathlineto{\pgfqpoint{4.580727in}{4.453627in}}%
\pgfpathlineto{\pgfqpoint{4.572097in}{4.395429in}}%
\pgfpathlineto{\pgfqpoint{4.563550in}{4.433481in}}%
\pgfpathlineto{\pgfqpoint{4.554980in}{4.433835in}}%
\pgfpathlineto{\pgfqpoint{4.546439in}{4.466243in}}%
\pgfpathlineto{\pgfqpoint{4.537876in}{4.471253in}}%
\pgfpathlineto{\pgfqpoint{4.529276in}{4.386438in}}%
\pgfpathlineto{\pgfqpoint{4.520678in}{4.445725in}}%
\pgfpathlineto{\pgfqpoint{4.512084in}{4.394031in}}%
\pgfpathlineto{\pgfqpoint{4.503421in}{4.419340in}}%
\pgfpathlineto{\pgfqpoint{4.494752in}{4.395369in}}%
\pgfpathlineto{\pgfqpoint{4.486071in}{4.433707in}}%
\pgfpathlineto{\pgfqpoint{4.477466in}{4.419552in}}%
\pgfpathlineto{\pgfqpoint{4.468771in}{4.376455in}}%
\pgfpathlineto{\pgfqpoint{4.460011in}{4.437742in}}%
\pgfpathlineto{\pgfqpoint{4.451305in}{4.388010in}}%
\pgfpathlineto{\pgfqpoint{4.442591in}{4.407372in}}%
\pgfpathlineto{\pgfqpoint{4.433894in}{4.417144in}}%
\pgfpathlineto{\pgfqpoint{4.425248in}{4.364245in}}%
\pgfpathlineto{\pgfqpoint{4.416522in}{4.384920in}}%
\pgfpathlineto{\pgfqpoint{4.407811in}{4.406652in}}%
\pgfpathlineto{\pgfqpoint{4.399156in}{4.411705in}}%
\pgfpathlineto{\pgfqpoint{4.390451in}{4.397536in}}%
\pgfpathlineto{\pgfqpoint{4.381765in}{4.452882in}}%
\pgfpathlineto{\pgfqpoint{4.373001in}{4.310014in}}%
\pgfpathlineto{\pgfqpoint{4.364219in}{4.345503in}}%
\pgfpathlineto{\pgfqpoint{4.355507in}{4.368981in}}%
\pgfpathlineto{\pgfqpoint{4.346774in}{3.980293in}}%
\pgfpathlineto{\pgfqpoint{4.337933in}{3.369348in}}%
\pgfpathlineto{\pgfqpoint{4.328717in}{3.313535in}}%
\pgfpathlineto{\pgfqpoint{4.319388in}{3.258960in}}%
\pgfpathlineto{\pgfqpoint{4.310022in}{3.205789in}}%
\pgfpathlineto{\pgfqpoint{4.300486in}{3.257171in}}%
\pgfpathlineto{\pgfqpoint{4.290581in}{3.249945in}}%
\pgfpathlineto{\pgfqpoint{4.280743in}{3.235203in}}%
\pgfpathlineto{\pgfqpoint{4.270972in}{2.690719in}}%
\pgfpathlineto{\pgfqpoint{4.260400in}{3.046886in}}%
\pgfpathlineto{\pgfqpoint{4.249866in}{2.228596in}}%
\pgfpathlineto{\pgfqpoint{4.238844in}{2.184649in}}%
\pgfpathlineto{\pgfqpoint{4.227703in}{2.212286in}}%
\pgfpathlineto{\pgfqpoint{4.216619in}{2.148412in}}%
\pgfpathlineto{\pgfqpoint{4.205594in}{2.130544in}}%
\pgfpathlineto{\pgfqpoint{4.194400in}{2.142830in}}%
\pgfpathlineto{\pgfqpoint{4.183285in}{2.168150in}}%
\pgfpathlineto{\pgfqpoint{4.172101in}{2.201788in}}%
\pgfpathlineto{\pgfqpoint{4.160983in}{2.118389in}}%
\pgfpathlineto{\pgfqpoint{4.149813in}{2.093443in}}%
\pgfpathlineto{\pgfqpoint{4.138598in}{2.086671in}}%
\pgfpathlineto{\pgfqpoint{4.127322in}{2.085539in}}%
\pgfpathlineto{\pgfqpoint{4.116014in}{2.086163in}}%
\pgfpathlineto{\pgfqpoint{4.104772in}{2.095661in}}%
\pgfpathlineto{\pgfqpoint{4.093537in}{2.100947in}}%
\pgfpathlineto{\pgfqpoint{4.082264in}{2.096791in}}%
\pgfpathlineto{\pgfqpoint{4.070994in}{2.075125in}}%
\pgfpathlineto{\pgfqpoint{4.059512in}{2.067073in}}%
\pgfpathlineto{\pgfqpoint{4.048157in}{2.096607in}}%
\pgfpathlineto{\pgfqpoint{4.036918in}{2.080054in}}%
\pgfpathlineto{\pgfqpoint{4.025578in}{2.079994in}}%
\pgfpathlineto{\pgfqpoint{4.014312in}{2.075191in}}%
\pgfpathlineto{\pgfqpoint{4.003035in}{2.062644in}}%
\pgfpathlineto{\pgfqpoint{3.991607in}{2.055093in}}%
\pgfpathlineto{\pgfqpoint{3.980165in}{2.048888in}}%
\pgfpathlineto{\pgfqpoint{3.968800in}{2.072648in}}%
\pgfpathlineto{\pgfqpoint{3.957345in}{2.041536in}}%
\pgfpathlineto{\pgfqpoint{3.945995in}{2.061962in}}%
\pgfpathlineto{\pgfqpoint{3.934526in}{2.053257in}}%
\pgfpathlineto{\pgfqpoint{3.923007in}{2.046423in}}%
\pgfpathlineto{\pgfqpoint{3.911548in}{2.054666in}}%
\pgfpathlineto{\pgfqpoint{3.900120in}{2.061516in}}%
\pgfpathlineto{\pgfqpoint{3.888612in}{2.037096in}}%
\pgfpathlineto{\pgfqpoint{3.876999in}{2.040263in}}%
\pgfpathlineto{\pgfqpoint{3.865394in}{2.054799in}}%
\pgfpathlineto{\pgfqpoint{3.853773in}{2.034475in}}%
\pgfpathlineto{\pgfqpoint{3.842268in}{2.043136in}}%
\pgfpathlineto{\pgfqpoint{3.830748in}{2.052897in}}%
\pgfpathlineto{\pgfqpoint{3.819207in}{2.050608in}}%
\pgfpathlineto{\pgfqpoint{3.807700in}{2.051993in}}%
\pgfpathlineto{\pgfqpoint{3.796179in}{1.897704in}}%
\pgfpathlineto{\pgfqpoint{3.784213in}{1.352262in}}%
\pgfpathlineto{\pgfqpoint{3.772194in}{1.361164in}}%
\pgfpathlineto{\pgfqpoint{3.760252in}{1.370177in}}%
\pgfpathlineto{\pgfqpoint{3.748246in}{1.359429in}}%
\pgfpathlineto{\pgfqpoint{3.736218in}{1.351867in}}%
\pgfpathlineto{\pgfqpoint{3.723925in}{1.333234in}}%
\pgfpathlineto{\pgfqpoint{3.711790in}{1.351455in}}%
\pgfpathlineto{\pgfqpoint{3.699766in}{1.368872in}}%
\pgfpathlineto{\pgfqpoint{3.687677in}{1.360204in}}%
\pgfpathlineto{\pgfqpoint{3.675639in}{1.358095in}}%
\pgfpathlineto{\pgfqpoint{3.663361in}{1.335225in}}%
\pgfpathlineto{\pgfqpoint{3.651005in}{1.347457in}}%
\pgfpathlineto{\pgfqpoint{3.639052in}{1.369263in}}%
\pgfpathlineto{\pgfqpoint{3.627108in}{1.355678in}}%
\pgfpathlineto{\pgfqpoint{3.615064in}{1.378988in}}%
\pgfpathlineto{\pgfqpoint{3.602917in}{1.351287in}}%
\pgfpathlineto{\pgfqpoint{3.590359in}{1.349011in}}%
\pgfpathlineto{\pgfqpoint{3.578260in}{1.373712in}}%
\pgfpathlineto{\pgfqpoint{3.566335in}{1.373643in}}%
\pgfpathlineto{\pgfqpoint{3.554208in}{1.373811in}}%
\pgfpathlineto{\pgfqpoint{3.542041in}{1.350537in}}%
\pgfpathlineto{\pgfqpoint{3.528239in}{1.330686in}}%
\pgfpathlineto{\pgfqpoint{3.513323in}{1.357402in}}%
\pgfpathlineto{\pgfqpoint{3.498190in}{1.353328in}}%
\pgfpathlineto{\pgfqpoint{3.483019in}{1.338719in}}%
\pgfpathlineto{\pgfqpoint{3.467311in}{1.344732in}}%
\pgfpathlineto{\pgfqpoint{3.451777in}{1.340485in}}%
\pgfpathlineto{\pgfqpoint{3.436431in}{1.346691in}}%
\pgfpathlineto{\pgfqpoint{3.421182in}{1.359677in}}%
\pgfpathlineto{\pgfqpoint{3.405908in}{1.330165in}}%
\pgfpathlineto{\pgfqpoint{3.390722in}{1.351128in}}%
\pgfpathlineto{\pgfqpoint{3.375355in}{1.358450in}}%
\pgfpathlineto{\pgfqpoint{3.359992in}{1.352495in}}%
\pgfpathlineto{\pgfqpoint{3.344594in}{1.350257in}}%
\pgfpathlineto{\pgfqpoint{3.328928in}{1.357661in}}%
\pgfpathlineto{\pgfqpoint{3.313329in}{1.358801in}}%
\pgfpathlineto{\pgfqpoint{3.295281in}{1.335752in}}%
\pgfpathlineto{\pgfqpoint{3.279179in}{1.357001in}}%
\pgfpathlineto{\pgfqpoint{3.263722in}{1.346089in}}%
\pgfpathlineto{\pgfqpoint{3.248207in}{1.370754in}}%
\pgfpathlineto{\pgfqpoint{3.232763in}{1.364579in}}%
\pgfpathlineto{\pgfqpoint{3.217213in}{1.355369in}}%
\pgfpathlineto{\pgfqpoint{3.201602in}{1.355112in}}%
\pgfpathlineto{\pgfqpoint{3.185752in}{1.343805in}}%
\pgfpathlineto{\pgfqpoint{3.169651in}{1.330591in}}%
\pgfpathlineto{\pgfqpoint{3.152584in}{1.339130in}}%
\pgfpathlineto{\pgfqpoint{3.135319in}{1.342138in}}%
\pgfpathlineto{\pgfqpoint{3.118692in}{1.348082in}}%
\pgfpathlineto{\pgfqpoint{3.102316in}{1.363293in}}%
\pgfpathlineto{\pgfqpoint{3.085686in}{1.352662in}}%
\pgfpathlineto{\pgfqpoint{3.069003in}{1.353297in}}%
\pgfpathlineto{\pgfqpoint{3.052708in}{1.359158in}}%
\pgfpathlineto{\pgfqpoint{3.036276in}{1.362909in}}%
\pgfpathlineto{\pgfqpoint{3.018968in}{1.367576in}}%
\pgfpathlineto{\pgfqpoint{3.001921in}{1.356753in}}%
\pgfpathlineto{\pgfqpoint{2.985853in}{1.362026in}}%
\pgfpathlineto{\pgfqpoint{2.968433in}{1.333926in}}%
\pgfpathlineto{\pgfqpoint{2.949130in}{1.348657in}}%
\pgfpathlineto{\pgfqpoint{2.930775in}{1.330985in}}%
\pgfpathlineto{\pgfqpoint{2.914048in}{1.369922in}}%
\pgfpathlineto{\pgfqpoint{2.897563in}{1.338973in}}%
\pgfpathlineto{\pgfqpoint{2.880921in}{1.348276in}}%
\pgfpathlineto{\pgfqpoint{2.864455in}{1.365172in}}%
\pgfpathlineto{\pgfqpoint{2.848081in}{1.354422in}}%
\pgfpathlineto{\pgfqpoint{2.831972in}{1.353259in}}%
\pgfpathlineto{\pgfqpoint{2.815750in}{1.342818in}}%
\pgfpathlineto{\pgfqpoint{2.798917in}{1.361071in}}%
\pgfpathlineto{\pgfqpoint{2.780862in}{1.344191in}}%
\pgfpathlineto{\pgfqpoint{2.763118in}{1.344190in}}%
\pgfpathlineto{\pgfqpoint{2.744840in}{1.369795in}}%
\pgfpathlineto{\pgfqpoint{2.727722in}{1.380333in}}%
\pgfpathlineto{\pgfqpoint{2.710992in}{1.366240in}}%
\pgfpathlineto{\pgfqpoint{2.693540in}{1.357920in}}%
\pgfpathlineto{\pgfqpoint{2.676022in}{1.357399in}}%
\pgfpathlineto{\pgfqpoint{2.657274in}{1.358110in}}%
\pgfpathlineto{\pgfqpoint{2.636069in}{1.356003in}}%
\pgfpathlineto{\pgfqpoint{2.616149in}{1.366253in}}%
\pgfpathlineto{\pgfqpoint{2.598064in}{1.375341in}}%
\pgfpathlineto{\pgfqpoint{2.578521in}{1.333764in}}%
\pgfpathlineto{\pgfqpoint{2.558886in}{1.369884in}}%
\pgfpathlineto{\pgfqpoint{2.538352in}{1.353853in}}%
\pgfpathlineto{\pgfqpoint{2.515773in}{1.372418in}}%
\pgfpathlineto{\pgfqpoint{2.494737in}{1.353639in}}%
\pgfpathlineto{\pgfqpoint{2.477723in}{1.342618in}}%
\pgfpathlineto{\pgfqpoint{2.457197in}{1.348664in}}%
\pgfpathlineto{\pgfqpoint{2.433282in}{1.356220in}}%
\pgfpathlineto{\pgfqpoint{2.418633in}{1.355174in}}%
\pgfpathlineto{\pgfqpoint{2.406207in}{1.372840in}}%
\pgfpathlineto{\pgfqpoint{2.393733in}{1.396569in}}%
\pgfpathlineto{\pgfqpoint{2.381372in}{1.371784in}}%
\pgfpathlineto{\pgfqpoint{2.368880in}{1.356000in}}%
\pgfpathlineto{\pgfqpoint{2.356417in}{1.367307in}}%
\pgfpathlineto{\pgfqpoint{2.343909in}{1.381915in}}%
\pgfpathlineto{\pgfqpoint{2.331395in}{1.358749in}}%
\pgfpathlineto{\pgfqpoint{2.318891in}{1.382455in}}%
\pgfpathlineto{\pgfqpoint{2.306399in}{1.356689in}}%
\pgfpathlineto{\pgfqpoint{2.293873in}{1.357232in}}%
\pgfpathlineto{\pgfqpoint{2.281255in}{1.341527in}}%
\pgfpathlineto{\pgfqpoint{2.268761in}{1.362718in}}%
\pgfpathlineto{\pgfqpoint{2.256287in}{1.375775in}}%
\pgfpathlineto{\pgfqpoint{2.243791in}{1.350556in}}%
\pgfpathlineto{\pgfqpoint{2.231301in}{1.363539in}}%
\pgfpathlineto{\pgfqpoint{2.218726in}{1.349283in}}%
\pgfpathlineto{\pgfqpoint{2.206174in}{1.363547in}}%
\pgfpathlineto{\pgfqpoint{2.193673in}{1.352149in}}%
\pgfpathlineto{\pgfqpoint{2.181172in}{1.361349in}}%
\pgfpathlineto{\pgfqpoint{2.168659in}{1.372426in}}%
\pgfpathlineto{\pgfqpoint{2.156064in}{1.365248in}}%
\pgfpathlineto{\pgfqpoint{2.143512in}{1.378462in}}%
\pgfpathlineto{\pgfqpoint{2.130987in}{1.366018in}}%
\pgfpathlineto{\pgfqpoint{2.118500in}{1.380068in}}%
\pgfpathlineto{\pgfqpoint{2.106029in}{1.369908in}}%
\pgfpathlineto{\pgfqpoint{2.093434in}{1.358259in}}%
\pgfpathlineto{\pgfqpoint{2.080914in}{1.366941in}}%
\pgfpathlineto{\pgfqpoint{2.068360in}{1.358866in}}%
\pgfpathlineto{\pgfqpoint{2.055871in}{1.391166in}}%
\pgfpathlineto{\pgfqpoint{2.043350in}{1.364063in}}%
\pgfpathlineto{\pgfqpoint{2.030840in}{1.361048in}}%
\pgfpathlineto{\pgfqpoint{2.016889in}{1.336453in}}%
\pgfpathlineto{\pgfqpoint{2.004212in}{1.354139in}}%
\pgfpathlineto{\pgfqpoint{1.991646in}{1.366747in}}%
\pgfpathlineto{\pgfqpoint{1.979161in}{1.357468in}}%
\pgfpathlineto{\pgfqpoint{1.966664in}{1.360568in}}%
\pgfpathlineto{\pgfqpoint{1.954108in}{1.362874in}}%
\pgfpathlineto{\pgfqpoint{1.941620in}{1.344243in}}%
\pgfpathlineto{\pgfqpoint{1.929118in}{1.354792in}}%
\pgfpathlineto{\pgfqpoint{1.916611in}{1.360805in}}%
\pgfpathlineto{\pgfqpoint{1.904130in}{1.380285in}}%
\pgfpathlineto{\pgfqpoint{1.891620in}{1.373643in}}%
\pgfpathlineto{\pgfqpoint{1.879180in}{1.358595in}}%
\pgfpathlineto{\pgfqpoint{1.866747in}{1.368202in}}%
\pgfpathlineto{\pgfqpoint{1.854433in}{1.378694in}}%
\pgfpathlineto{\pgfqpoint{1.842065in}{1.365549in}}%
\pgfpathlineto{\pgfqpoint{1.829628in}{1.361924in}}%
\pgfpathlineto{\pgfqpoint{1.817148in}{1.361193in}}%
\pgfpathlineto{\pgfqpoint{1.804716in}{1.347526in}}%
\pgfpathlineto{\pgfqpoint{1.792355in}{1.361194in}}%
\pgfpathlineto{\pgfqpoint{1.779913in}{1.371013in}}%
\pgfpathlineto{\pgfqpoint{1.767480in}{1.355551in}}%
\pgfpathlineto{\pgfqpoint{1.755070in}{1.365918in}}%
\pgfpathlineto{\pgfqpoint{1.742593in}{1.379392in}}%
\pgfpathlineto{\pgfqpoint{1.730200in}{1.361760in}}%
\pgfpathlineto{\pgfqpoint{1.717832in}{1.384399in}}%
\pgfpathlineto{\pgfqpoint{1.705498in}{1.369875in}}%
\pgfpathlineto{\pgfqpoint{1.693005in}{1.366204in}}%
\pgfpathlineto{\pgfqpoint{1.680578in}{1.373158in}}%
\pgfpathlineto{\pgfqpoint{1.668187in}{1.353482in}}%
\pgfpathlineto{\pgfqpoint{1.655673in}{1.373877in}}%
\pgfpathlineto{\pgfqpoint{1.643249in}{1.387436in}}%
\pgfpathlineto{\pgfqpoint{1.630788in}{1.376753in}}%
\pgfpathlineto{\pgfqpoint{1.618336in}{1.350427in}}%
\pgfpathlineto{\pgfqpoint{1.605869in}{1.361595in}}%
\pgfpathlineto{\pgfqpoint{1.593430in}{1.378345in}}%
\pgfpathlineto{\pgfqpoint{1.581070in}{1.362301in}}%
\pgfpathlineto{\pgfqpoint{1.568672in}{1.380745in}}%
\pgfpathlineto{\pgfqpoint{1.556290in}{1.382012in}}%
\pgfpathlineto{\pgfqpoint{1.543839in}{1.363024in}}%
\pgfpathlineto{\pgfqpoint{1.531474in}{1.364599in}}%
\pgfpathlineto{\pgfqpoint{1.519060in}{1.359927in}}%
\pgfpathlineto{\pgfqpoint{1.506021in}{1.354867in}}%
\pgfpathlineto{\pgfqpoint{1.493578in}{1.358615in}}%
\pgfpathlineto{\pgfqpoint{1.481163in}{1.372640in}}%
\pgfpathlineto{\pgfqpoint{1.468660in}{1.350126in}}%
\pgfpathlineto{\pgfqpoint{1.456188in}{1.370215in}}%
\pgfpathlineto{\pgfqpoint{1.443756in}{1.382502in}}%
\pgfpathlineto{\pgfqpoint{1.431382in}{1.385745in}}%
\pgfpathlineto{\pgfqpoint{1.418893in}{1.357081in}}%
\pgfpathlineto{\pgfqpoint{1.406401in}{1.353194in}}%
\pgfpathlineto{\pgfqpoint{1.393906in}{1.358340in}}%
\pgfpathlineto{\pgfqpoint{1.381448in}{1.383638in}}%
\pgfpathlineto{\pgfqpoint{1.368992in}{1.377727in}}%
\pgfpathlineto{\pgfqpoint{1.356458in}{1.337996in}}%
\pgfpathlineto{\pgfqpoint{1.343969in}{1.342539in}}%
\pgfpathlineto{\pgfqpoint{1.331552in}{1.382199in}}%
\pgfpathlineto{\pgfqpoint{1.319073in}{1.380862in}}%
\pgfpathlineto{\pgfqpoint{1.306603in}{1.385029in}}%
\pgfpathlineto{\pgfqpoint{1.294165in}{1.381199in}}%
\pgfpathlineto{\pgfqpoint{1.281753in}{1.374623in}}%
\pgfpathlineto{\pgfqpoint{1.269255in}{1.360513in}}%
\pgfpathlineto{\pgfqpoint{1.256724in}{1.362601in}}%
\pgfpathlineto{\pgfqpoint{1.244104in}{1.359756in}}%
\pgfpathlineto{\pgfqpoint{1.231699in}{1.354673in}}%
\pgfpathlineto{\pgfqpoint{1.219183in}{1.368184in}}%
\pgfpathlineto{\pgfqpoint{1.206622in}{1.375687in}}%
\pgfpathlineto{\pgfqpoint{1.194024in}{1.350589in}}%
\pgfpathlineto{\pgfqpoint{1.181503in}{1.371956in}}%
\pgfpathlineto{\pgfqpoint{1.168924in}{1.364595in}}%
\pgfpathlineto{\pgfqpoint{1.156226in}{1.345159in}}%
\pgfpathlineto{\pgfqpoint{1.143621in}{1.357908in}}%
\pgfpathlineto{\pgfqpoint{1.131014in}{1.356606in}}%
\pgfpathlineto{\pgfqpoint{1.118303in}{1.368090in}}%
\pgfpathlineto{\pgfqpoint{1.105655in}{1.344527in}}%
\pgfpathlineto{\pgfqpoint{1.092853in}{1.362511in}}%
\pgfpathlineto{\pgfqpoint{1.079972in}{1.344925in}}%
\pgfpathlineto{\pgfqpoint{1.067143in}{1.339557in}}%
\pgfpathlineto{\pgfqpoint{1.054264in}{1.363900in}}%
\pgfpathlineto{\pgfqpoint{1.041322in}{1.349874in}}%
\pgfpathlineto{\pgfqpoint{1.028303in}{1.352990in}}%
\pgfpathlineto{\pgfqpoint{1.015150in}{1.352330in}}%
\pgfpathlineto{\pgfqpoint{1.001616in}{1.301014in}}%
\pgfpathclose%
\pgfusepath{fill}%
\end{pgfscope}%
\begin{pgfscope}%
\pgfpathrectangle{\pgfqpoint{0.781402in}{0.773588in}}{\pgfqpoint{4.844695in}{5.415119in}}%
\pgfusepath{clip}%
\pgfsetbuttcap%
\pgfsetroundjoin%
\definecolor{currentfill}{rgb}{0.839216,0.152941,0.156863}%
\pgfsetfillcolor{currentfill}%
\pgfsetlinewidth{0.000000pt}%
\definecolor{currentstroke}{rgb}{0.000000,0.000000,0.000000}%
\pgfsetstrokecolor{currentstroke}%
\pgfsetdash{}{0pt}%
\pgfpathmoveto{\pgfqpoint{1.001616in}{1.770374in}}%
\pgfpathlineto{\pgfqpoint{1.001616in}{1.301014in}}%
\pgfpathlineto{\pgfqpoint{1.015150in}{1.352330in}}%
\pgfpathlineto{\pgfqpoint{1.028303in}{1.352990in}}%
\pgfpathlineto{\pgfqpoint{1.041322in}{1.349874in}}%
\pgfpathlineto{\pgfqpoint{1.054264in}{1.363900in}}%
\pgfpathlineto{\pgfqpoint{1.067143in}{1.339557in}}%
\pgfpathlineto{\pgfqpoint{1.079972in}{1.344925in}}%
\pgfpathlineto{\pgfqpoint{1.092853in}{1.362511in}}%
\pgfpathlineto{\pgfqpoint{1.105655in}{1.344527in}}%
\pgfpathlineto{\pgfqpoint{1.118303in}{1.368090in}}%
\pgfpathlineto{\pgfqpoint{1.131014in}{1.356606in}}%
\pgfpathlineto{\pgfqpoint{1.143621in}{1.357908in}}%
\pgfpathlineto{\pgfqpoint{1.156226in}{1.345159in}}%
\pgfpathlineto{\pgfqpoint{1.168924in}{1.364595in}}%
\pgfpathlineto{\pgfqpoint{1.181503in}{1.371956in}}%
\pgfpathlineto{\pgfqpoint{1.194024in}{1.350589in}}%
\pgfpathlineto{\pgfqpoint{1.206622in}{1.375687in}}%
\pgfpathlineto{\pgfqpoint{1.219183in}{1.368184in}}%
\pgfpathlineto{\pgfqpoint{1.231699in}{1.354673in}}%
\pgfpathlineto{\pgfqpoint{1.244104in}{1.359756in}}%
\pgfpathlineto{\pgfqpoint{1.256724in}{1.362601in}}%
\pgfpathlineto{\pgfqpoint{1.269255in}{1.360513in}}%
\pgfpathlineto{\pgfqpoint{1.281753in}{1.374623in}}%
\pgfpathlineto{\pgfqpoint{1.294165in}{1.381199in}}%
\pgfpathlineto{\pgfqpoint{1.306603in}{1.385029in}}%
\pgfpathlineto{\pgfqpoint{1.319073in}{1.380862in}}%
\pgfpathlineto{\pgfqpoint{1.331552in}{1.382199in}}%
\pgfpathlineto{\pgfqpoint{1.343969in}{1.342539in}}%
\pgfpathlineto{\pgfqpoint{1.356458in}{1.337996in}}%
\pgfpathlineto{\pgfqpoint{1.368992in}{1.377727in}}%
\pgfpathlineto{\pgfqpoint{1.381448in}{1.383638in}}%
\pgfpathlineto{\pgfqpoint{1.393906in}{1.358340in}}%
\pgfpathlineto{\pgfqpoint{1.406401in}{1.353194in}}%
\pgfpathlineto{\pgfqpoint{1.418893in}{1.357081in}}%
\pgfpathlineto{\pgfqpoint{1.431382in}{1.385745in}}%
\pgfpathlineto{\pgfqpoint{1.443756in}{1.382502in}}%
\pgfpathlineto{\pgfqpoint{1.456188in}{1.370215in}}%
\pgfpathlineto{\pgfqpoint{1.468660in}{1.350126in}}%
\pgfpathlineto{\pgfqpoint{1.481163in}{1.372640in}}%
\pgfpathlineto{\pgfqpoint{1.493578in}{1.358615in}}%
\pgfpathlineto{\pgfqpoint{1.506021in}{1.354867in}}%
\pgfpathlineto{\pgfqpoint{1.519060in}{1.359927in}}%
\pgfpathlineto{\pgfqpoint{1.531474in}{1.364599in}}%
\pgfpathlineto{\pgfqpoint{1.543839in}{1.363024in}}%
\pgfpathlineto{\pgfqpoint{1.556290in}{1.382012in}}%
\pgfpathlineto{\pgfqpoint{1.568672in}{1.380745in}}%
\pgfpathlineto{\pgfqpoint{1.581070in}{1.362301in}}%
\pgfpathlineto{\pgfqpoint{1.593430in}{1.378345in}}%
\pgfpathlineto{\pgfqpoint{1.605869in}{1.361595in}}%
\pgfpathlineto{\pgfqpoint{1.618336in}{1.350427in}}%
\pgfpathlineto{\pgfqpoint{1.630788in}{1.376753in}}%
\pgfpathlineto{\pgfqpoint{1.643249in}{1.387436in}}%
\pgfpathlineto{\pgfqpoint{1.655673in}{1.373877in}}%
\pgfpathlineto{\pgfqpoint{1.668187in}{1.353482in}}%
\pgfpathlineto{\pgfqpoint{1.680578in}{1.373158in}}%
\pgfpathlineto{\pgfqpoint{1.693005in}{1.366204in}}%
\pgfpathlineto{\pgfqpoint{1.705498in}{1.369875in}}%
\pgfpathlineto{\pgfqpoint{1.717832in}{1.384399in}}%
\pgfpathlineto{\pgfqpoint{1.730200in}{1.361760in}}%
\pgfpathlineto{\pgfqpoint{1.742593in}{1.379392in}}%
\pgfpathlineto{\pgfqpoint{1.755070in}{1.365918in}}%
\pgfpathlineto{\pgfqpoint{1.767480in}{1.355551in}}%
\pgfpathlineto{\pgfqpoint{1.779913in}{1.371013in}}%
\pgfpathlineto{\pgfqpoint{1.792355in}{1.361194in}}%
\pgfpathlineto{\pgfqpoint{1.804716in}{1.347526in}}%
\pgfpathlineto{\pgfqpoint{1.817148in}{1.361193in}}%
\pgfpathlineto{\pgfqpoint{1.829628in}{1.361924in}}%
\pgfpathlineto{\pgfqpoint{1.842065in}{1.365549in}}%
\pgfpathlineto{\pgfqpoint{1.854433in}{1.378694in}}%
\pgfpathlineto{\pgfqpoint{1.866747in}{1.368202in}}%
\pgfpathlineto{\pgfqpoint{1.879180in}{1.358595in}}%
\pgfpathlineto{\pgfqpoint{1.891620in}{1.373643in}}%
\pgfpathlineto{\pgfqpoint{1.904130in}{1.380285in}}%
\pgfpathlineto{\pgfqpoint{1.916611in}{1.360805in}}%
\pgfpathlineto{\pgfqpoint{1.929118in}{1.354792in}}%
\pgfpathlineto{\pgfqpoint{1.941620in}{1.344243in}}%
\pgfpathlineto{\pgfqpoint{1.954108in}{1.362874in}}%
\pgfpathlineto{\pgfqpoint{1.966664in}{1.360568in}}%
\pgfpathlineto{\pgfqpoint{1.979161in}{1.357468in}}%
\pgfpathlineto{\pgfqpoint{1.991646in}{1.366747in}}%
\pgfpathlineto{\pgfqpoint{2.004212in}{1.354139in}}%
\pgfpathlineto{\pgfqpoint{2.016889in}{1.336453in}}%
\pgfpathlineto{\pgfqpoint{2.030840in}{1.361048in}}%
\pgfpathlineto{\pgfqpoint{2.043350in}{1.364063in}}%
\pgfpathlineto{\pgfqpoint{2.055871in}{1.391166in}}%
\pgfpathlineto{\pgfqpoint{2.068360in}{1.358866in}}%
\pgfpathlineto{\pgfqpoint{2.080914in}{1.366941in}}%
\pgfpathlineto{\pgfqpoint{2.093434in}{1.358259in}}%
\pgfpathlineto{\pgfqpoint{2.106029in}{1.369908in}}%
\pgfpathlineto{\pgfqpoint{2.118500in}{1.380068in}}%
\pgfpathlineto{\pgfqpoint{2.130987in}{1.366018in}}%
\pgfpathlineto{\pgfqpoint{2.143512in}{1.378462in}}%
\pgfpathlineto{\pgfqpoint{2.156064in}{1.365248in}}%
\pgfpathlineto{\pgfqpoint{2.168659in}{1.372426in}}%
\pgfpathlineto{\pgfqpoint{2.181172in}{1.361349in}}%
\pgfpathlineto{\pgfqpoint{2.193673in}{1.352149in}}%
\pgfpathlineto{\pgfqpoint{2.206174in}{1.363547in}}%
\pgfpathlineto{\pgfqpoint{2.218726in}{1.349283in}}%
\pgfpathlineto{\pgfqpoint{2.231301in}{1.363539in}}%
\pgfpathlineto{\pgfqpoint{2.243791in}{1.350556in}}%
\pgfpathlineto{\pgfqpoint{2.256287in}{1.375775in}}%
\pgfpathlineto{\pgfqpoint{2.268761in}{1.362718in}}%
\pgfpathlineto{\pgfqpoint{2.281255in}{1.341527in}}%
\pgfpathlineto{\pgfqpoint{2.293873in}{1.357232in}}%
\pgfpathlineto{\pgfqpoint{2.306399in}{1.356689in}}%
\pgfpathlineto{\pgfqpoint{2.318891in}{1.382455in}}%
\pgfpathlineto{\pgfqpoint{2.331395in}{1.358749in}}%
\pgfpathlineto{\pgfqpoint{2.343909in}{1.381915in}}%
\pgfpathlineto{\pgfqpoint{2.356417in}{1.367307in}}%
\pgfpathlineto{\pgfqpoint{2.368880in}{1.356000in}}%
\pgfpathlineto{\pgfqpoint{2.381372in}{1.371784in}}%
\pgfpathlineto{\pgfqpoint{2.393733in}{1.396569in}}%
\pgfpathlineto{\pgfqpoint{2.406207in}{1.372840in}}%
\pgfpathlineto{\pgfqpoint{2.418633in}{1.355174in}}%
\pgfpathlineto{\pgfqpoint{2.433282in}{1.356220in}}%
\pgfpathlineto{\pgfqpoint{2.457197in}{1.348664in}}%
\pgfpathlineto{\pgfqpoint{2.477723in}{1.342618in}}%
\pgfpathlineto{\pgfqpoint{2.494737in}{1.353639in}}%
\pgfpathlineto{\pgfqpoint{2.515773in}{1.372418in}}%
\pgfpathlineto{\pgfqpoint{2.538352in}{1.353853in}}%
\pgfpathlineto{\pgfqpoint{2.558886in}{1.369884in}}%
\pgfpathlineto{\pgfqpoint{2.578521in}{1.333764in}}%
\pgfpathlineto{\pgfqpoint{2.598064in}{1.375341in}}%
\pgfpathlineto{\pgfqpoint{2.616149in}{1.366253in}}%
\pgfpathlineto{\pgfqpoint{2.636069in}{1.356003in}}%
\pgfpathlineto{\pgfqpoint{2.657274in}{1.358110in}}%
\pgfpathlineto{\pgfqpoint{2.676022in}{1.357399in}}%
\pgfpathlineto{\pgfqpoint{2.693540in}{1.357920in}}%
\pgfpathlineto{\pgfqpoint{2.710992in}{1.366240in}}%
\pgfpathlineto{\pgfqpoint{2.727722in}{1.380333in}}%
\pgfpathlineto{\pgfqpoint{2.744840in}{1.369795in}}%
\pgfpathlineto{\pgfqpoint{2.763118in}{1.344190in}}%
\pgfpathlineto{\pgfqpoint{2.780862in}{1.344191in}}%
\pgfpathlineto{\pgfqpoint{2.798917in}{1.361071in}}%
\pgfpathlineto{\pgfqpoint{2.815750in}{1.342818in}}%
\pgfpathlineto{\pgfqpoint{2.831972in}{1.353259in}}%
\pgfpathlineto{\pgfqpoint{2.848081in}{1.354422in}}%
\pgfpathlineto{\pgfqpoint{2.864455in}{1.365172in}}%
\pgfpathlineto{\pgfqpoint{2.880921in}{1.348276in}}%
\pgfpathlineto{\pgfqpoint{2.897563in}{1.338973in}}%
\pgfpathlineto{\pgfqpoint{2.914048in}{1.369922in}}%
\pgfpathlineto{\pgfqpoint{2.930775in}{1.330985in}}%
\pgfpathlineto{\pgfqpoint{2.949130in}{1.348657in}}%
\pgfpathlineto{\pgfqpoint{2.968433in}{1.333926in}}%
\pgfpathlineto{\pgfqpoint{2.985853in}{1.362026in}}%
\pgfpathlineto{\pgfqpoint{3.001921in}{1.356753in}}%
\pgfpathlineto{\pgfqpoint{3.018968in}{1.367576in}}%
\pgfpathlineto{\pgfqpoint{3.036276in}{1.362909in}}%
\pgfpathlineto{\pgfqpoint{3.052708in}{1.359158in}}%
\pgfpathlineto{\pgfqpoint{3.069003in}{1.353297in}}%
\pgfpathlineto{\pgfqpoint{3.085686in}{1.352662in}}%
\pgfpathlineto{\pgfqpoint{3.102316in}{1.363293in}}%
\pgfpathlineto{\pgfqpoint{3.118692in}{1.348082in}}%
\pgfpathlineto{\pgfqpoint{3.135319in}{1.342138in}}%
\pgfpathlineto{\pgfqpoint{3.152584in}{1.339130in}}%
\pgfpathlineto{\pgfqpoint{3.169651in}{1.330591in}}%
\pgfpathlineto{\pgfqpoint{3.185752in}{1.343805in}}%
\pgfpathlineto{\pgfqpoint{3.201602in}{1.355112in}}%
\pgfpathlineto{\pgfqpoint{3.217213in}{1.355369in}}%
\pgfpathlineto{\pgfqpoint{3.232763in}{1.364579in}}%
\pgfpathlineto{\pgfqpoint{3.248207in}{1.370754in}}%
\pgfpathlineto{\pgfqpoint{3.263722in}{1.346089in}}%
\pgfpathlineto{\pgfqpoint{3.279179in}{1.357001in}}%
\pgfpathlineto{\pgfqpoint{3.295281in}{1.335752in}}%
\pgfpathlineto{\pgfqpoint{3.313329in}{1.358801in}}%
\pgfpathlineto{\pgfqpoint{3.328928in}{1.357661in}}%
\pgfpathlineto{\pgfqpoint{3.344594in}{1.350257in}}%
\pgfpathlineto{\pgfqpoint{3.359992in}{1.352495in}}%
\pgfpathlineto{\pgfqpoint{3.375355in}{1.358450in}}%
\pgfpathlineto{\pgfqpoint{3.390722in}{1.351128in}}%
\pgfpathlineto{\pgfqpoint{3.405908in}{1.330165in}}%
\pgfpathlineto{\pgfqpoint{3.421182in}{1.359677in}}%
\pgfpathlineto{\pgfqpoint{3.436431in}{1.346691in}}%
\pgfpathlineto{\pgfqpoint{3.451777in}{1.340485in}}%
\pgfpathlineto{\pgfqpoint{3.467311in}{1.344732in}}%
\pgfpathlineto{\pgfqpoint{3.483019in}{1.338719in}}%
\pgfpathlineto{\pgfqpoint{3.498190in}{1.353328in}}%
\pgfpathlineto{\pgfqpoint{3.513323in}{1.357402in}}%
\pgfpathlineto{\pgfqpoint{3.528239in}{1.330686in}}%
\pgfpathlineto{\pgfqpoint{3.542041in}{1.350537in}}%
\pgfpathlineto{\pgfqpoint{3.554208in}{1.373811in}}%
\pgfpathlineto{\pgfqpoint{3.566335in}{1.373643in}}%
\pgfpathlineto{\pgfqpoint{3.578260in}{1.373712in}}%
\pgfpathlineto{\pgfqpoint{3.590359in}{1.349011in}}%
\pgfpathlineto{\pgfqpoint{3.602917in}{1.351287in}}%
\pgfpathlineto{\pgfqpoint{3.615064in}{1.378988in}}%
\pgfpathlineto{\pgfqpoint{3.627108in}{1.355678in}}%
\pgfpathlineto{\pgfqpoint{3.639052in}{1.369263in}}%
\pgfpathlineto{\pgfqpoint{3.651005in}{1.347457in}}%
\pgfpathlineto{\pgfqpoint{3.663361in}{1.335225in}}%
\pgfpathlineto{\pgfqpoint{3.675639in}{1.358095in}}%
\pgfpathlineto{\pgfqpoint{3.687677in}{1.360204in}}%
\pgfpathlineto{\pgfqpoint{3.699766in}{1.368872in}}%
\pgfpathlineto{\pgfqpoint{3.711790in}{1.351455in}}%
\pgfpathlineto{\pgfqpoint{3.723925in}{1.333234in}}%
\pgfpathlineto{\pgfqpoint{3.736218in}{1.351867in}}%
\pgfpathlineto{\pgfqpoint{3.748246in}{1.359429in}}%
\pgfpathlineto{\pgfqpoint{3.760252in}{1.370177in}}%
\pgfpathlineto{\pgfqpoint{3.772194in}{1.361164in}}%
\pgfpathlineto{\pgfqpoint{3.784213in}{1.352262in}}%
\pgfpathlineto{\pgfqpoint{3.796179in}{1.897704in}}%
\pgfpathlineto{\pgfqpoint{3.807700in}{2.051993in}}%
\pgfpathlineto{\pgfqpoint{3.819207in}{2.050608in}}%
\pgfpathlineto{\pgfqpoint{3.830748in}{2.052897in}}%
\pgfpathlineto{\pgfqpoint{3.842268in}{2.043136in}}%
\pgfpathlineto{\pgfqpoint{3.853773in}{2.034475in}}%
\pgfpathlineto{\pgfqpoint{3.865394in}{2.054799in}}%
\pgfpathlineto{\pgfqpoint{3.876999in}{2.040263in}}%
\pgfpathlineto{\pgfqpoint{3.888612in}{2.037096in}}%
\pgfpathlineto{\pgfqpoint{3.900120in}{2.061516in}}%
\pgfpathlineto{\pgfqpoint{3.911548in}{2.054666in}}%
\pgfpathlineto{\pgfqpoint{3.923007in}{2.046423in}}%
\pgfpathlineto{\pgfqpoint{3.934526in}{2.053257in}}%
\pgfpathlineto{\pgfqpoint{3.945995in}{2.061962in}}%
\pgfpathlineto{\pgfqpoint{3.957345in}{2.041536in}}%
\pgfpathlineto{\pgfqpoint{3.968800in}{2.072648in}}%
\pgfpathlineto{\pgfqpoint{3.980165in}{2.048888in}}%
\pgfpathlineto{\pgfqpoint{3.991607in}{2.055093in}}%
\pgfpathlineto{\pgfqpoint{4.003035in}{2.062644in}}%
\pgfpathlineto{\pgfqpoint{4.014312in}{2.075191in}}%
\pgfpathlineto{\pgfqpoint{4.025578in}{2.079994in}}%
\pgfpathlineto{\pgfqpoint{4.036918in}{2.080054in}}%
\pgfpathlineto{\pgfqpoint{4.048157in}{2.096607in}}%
\pgfpathlineto{\pgfqpoint{4.059512in}{2.067073in}}%
\pgfpathlineto{\pgfqpoint{4.070994in}{2.075125in}}%
\pgfpathlineto{\pgfqpoint{4.082264in}{2.096791in}}%
\pgfpathlineto{\pgfqpoint{4.093537in}{2.100947in}}%
\pgfpathlineto{\pgfqpoint{4.104772in}{2.095661in}}%
\pgfpathlineto{\pgfqpoint{4.116014in}{2.086163in}}%
\pgfpathlineto{\pgfqpoint{4.127322in}{2.085539in}}%
\pgfpathlineto{\pgfqpoint{4.138598in}{2.086671in}}%
\pgfpathlineto{\pgfqpoint{4.149813in}{2.093443in}}%
\pgfpathlineto{\pgfqpoint{4.160983in}{2.118389in}}%
\pgfpathlineto{\pgfqpoint{4.172101in}{2.201788in}}%
\pgfpathlineto{\pgfqpoint{4.183285in}{2.168150in}}%
\pgfpathlineto{\pgfqpoint{4.194400in}{2.142830in}}%
\pgfpathlineto{\pgfqpoint{4.205594in}{2.130544in}}%
\pgfpathlineto{\pgfqpoint{4.216619in}{2.148412in}}%
\pgfpathlineto{\pgfqpoint{4.227703in}{2.212286in}}%
\pgfpathlineto{\pgfqpoint{4.238844in}{2.184649in}}%
\pgfpathlineto{\pgfqpoint{4.249866in}{2.228596in}}%
\pgfpathlineto{\pgfqpoint{4.260400in}{3.046886in}}%
\pgfpathlineto{\pgfqpoint{4.270972in}{2.690719in}}%
\pgfpathlineto{\pgfqpoint{4.280743in}{3.235203in}}%
\pgfpathlineto{\pgfqpoint{4.290581in}{3.249945in}}%
\pgfpathlineto{\pgfqpoint{4.300486in}{3.257171in}}%
\pgfpathlineto{\pgfqpoint{4.310022in}{3.205789in}}%
\pgfpathlineto{\pgfqpoint{4.319388in}{3.258960in}}%
\pgfpathlineto{\pgfqpoint{4.328717in}{3.313535in}}%
\pgfpathlineto{\pgfqpoint{4.337933in}{3.369348in}}%
\pgfpathlineto{\pgfqpoint{4.346774in}{3.980293in}}%
\pgfpathlineto{\pgfqpoint{4.355507in}{4.368981in}}%
\pgfpathlineto{\pgfqpoint{4.364219in}{4.345503in}}%
\pgfpathlineto{\pgfqpoint{4.373001in}{4.310014in}}%
\pgfpathlineto{\pgfqpoint{4.381765in}{4.452882in}}%
\pgfpathlineto{\pgfqpoint{4.390451in}{4.397536in}}%
\pgfpathlineto{\pgfqpoint{4.399156in}{4.411705in}}%
\pgfpathlineto{\pgfqpoint{4.407811in}{4.406652in}}%
\pgfpathlineto{\pgfqpoint{4.416522in}{4.384920in}}%
\pgfpathlineto{\pgfqpoint{4.425248in}{4.364245in}}%
\pgfpathlineto{\pgfqpoint{4.433894in}{4.417144in}}%
\pgfpathlineto{\pgfqpoint{4.442591in}{4.407372in}}%
\pgfpathlineto{\pgfqpoint{4.451305in}{4.388010in}}%
\pgfpathlineto{\pgfqpoint{4.460011in}{4.437742in}}%
\pgfpathlineto{\pgfqpoint{4.468771in}{4.376455in}}%
\pgfpathlineto{\pgfqpoint{4.477466in}{4.419552in}}%
\pgfpathlineto{\pgfqpoint{4.486071in}{4.433707in}}%
\pgfpathlineto{\pgfqpoint{4.494752in}{4.395369in}}%
\pgfpathlineto{\pgfqpoint{4.503421in}{4.419340in}}%
\pgfpathlineto{\pgfqpoint{4.512084in}{4.394031in}}%
\pgfpathlineto{\pgfqpoint{4.520678in}{4.445725in}}%
\pgfpathlineto{\pgfqpoint{4.529276in}{4.386438in}}%
\pgfpathlineto{\pgfqpoint{4.537876in}{4.471253in}}%
\pgfpathlineto{\pgfqpoint{4.546439in}{4.466243in}}%
\pgfpathlineto{\pgfqpoint{4.554980in}{4.433835in}}%
\pgfpathlineto{\pgfqpoint{4.563550in}{4.433481in}}%
\pgfpathlineto{\pgfqpoint{4.572097in}{4.395429in}}%
\pgfpathlineto{\pgfqpoint{4.580727in}{4.453627in}}%
\pgfpathlineto{\pgfqpoint{4.589255in}{4.484205in}}%
\pgfpathlineto{\pgfqpoint{4.597719in}{4.487330in}}%
\pgfpathlineto{\pgfqpoint{4.606211in}{4.429654in}}%
\pgfpathlineto{\pgfqpoint{4.614783in}{4.461635in}}%
\pgfpathlineto{\pgfqpoint{4.623282in}{4.487030in}}%
\pgfpathlineto{\pgfqpoint{4.631733in}{4.476753in}}%
\pgfpathlineto{\pgfqpoint{4.640221in}{4.405975in}}%
\pgfpathlineto{\pgfqpoint{4.648847in}{4.351666in}}%
\pgfpathlineto{\pgfqpoint{4.657349in}{4.473574in}}%
\pgfpathlineto{\pgfqpoint{4.665825in}{4.401763in}}%
\pgfpathlineto{\pgfqpoint{4.674255in}{4.483212in}}%
\pgfpathlineto{\pgfqpoint{4.682643in}{4.507740in}}%
\pgfpathlineto{\pgfqpoint{4.691071in}{4.512363in}}%
\pgfpathlineto{\pgfqpoint{4.699465in}{4.523338in}}%
\pgfpathlineto{\pgfqpoint{4.707865in}{4.480363in}}%
\pgfpathlineto{\pgfqpoint{4.716247in}{4.532645in}}%
\pgfpathlineto{\pgfqpoint{4.724666in}{4.474830in}}%
\pgfpathlineto{\pgfqpoint{4.733049in}{4.503171in}}%
\pgfpathlineto{\pgfqpoint{4.741378in}{4.484931in}}%
\pgfpathlineto{\pgfqpoint{4.749750in}{4.499415in}}%
\pgfpathlineto{\pgfqpoint{4.758119in}{4.480792in}}%
\pgfpathlineto{\pgfqpoint{4.766528in}{4.506900in}}%
\pgfpathlineto{\pgfqpoint{4.774881in}{4.522610in}}%
\pgfpathlineto{\pgfqpoint{4.783272in}{4.496865in}}%
\pgfpathlineto{\pgfqpoint{4.791573in}{4.490675in}}%
\pgfpathlineto{\pgfqpoint{4.799850in}{4.527335in}}%
\pgfpathlineto{\pgfqpoint{4.808134in}{4.526214in}}%
\pgfpathlineto{\pgfqpoint{4.816444in}{4.504794in}}%
\pgfpathlineto{\pgfqpoint{4.824772in}{4.478807in}}%
\pgfpathlineto{\pgfqpoint{4.833061in}{4.522476in}}%
\pgfpathlineto{\pgfqpoint{4.841325in}{4.554246in}}%
\pgfpathlineto{\pgfqpoint{4.849575in}{4.495651in}}%
\pgfpathlineto{\pgfqpoint{4.857830in}{4.575147in}}%
\pgfpathlineto{\pgfqpoint{4.866069in}{4.523872in}}%
\pgfpathlineto{\pgfqpoint{4.874381in}{4.471462in}}%
\pgfpathlineto{\pgfqpoint{4.882658in}{4.501882in}}%
\pgfpathlineto{\pgfqpoint{4.890980in}{4.529827in}}%
\pgfpathlineto{\pgfqpoint{4.899205in}{4.550643in}}%
\pgfpathlineto{\pgfqpoint{4.907476in}{4.498419in}}%
\pgfpathlineto{\pgfqpoint{4.915736in}{4.522807in}}%
\pgfpathlineto{\pgfqpoint{4.924041in}{4.478303in}}%
\pgfpathlineto{\pgfqpoint{4.932347in}{4.518769in}}%
\pgfpathlineto{\pgfqpoint{4.940566in}{4.488688in}}%
\pgfpathlineto{\pgfqpoint{4.948774in}{4.518747in}}%
\pgfpathlineto{\pgfqpoint{4.956992in}{4.505455in}}%
\pgfpathlineto{\pgfqpoint{4.965138in}{4.485935in}}%
\pgfpathlineto{\pgfqpoint{4.973428in}{4.495593in}}%
\pgfpathlineto{\pgfqpoint{4.981642in}{4.472070in}}%
\pgfpathlineto{\pgfqpoint{4.989794in}{4.523808in}}%
\pgfpathlineto{\pgfqpoint{4.997948in}{4.542287in}}%
\pgfpathlineto{\pgfqpoint{5.006134in}{4.496526in}}%
\pgfpathlineto{\pgfqpoint{5.014317in}{4.484651in}}%
\pgfpathlineto{\pgfqpoint{5.022451in}{4.541427in}}%
\pgfpathlineto{\pgfqpoint{5.030602in}{4.520239in}}%
\pgfpathlineto{\pgfqpoint{5.038759in}{4.517689in}}%
\pgfpathlineto{\pgfqpoint{5.046876in}{4.488292in}}%
\pgfpathlineto{\pgfqpoint{5.055031in}{4.513457in}}%
\pgfpathlineto{\pgfqpoint{5.063216in}{4.504166in}}%
\pgfpathlineto{\pgfqpoint{5.071391in}{4.484053in}}%
\pgfpathlineto{\pgfqpoint{5.079506in}{4.500618in}}%
\pgfpathlineto{\pgfqpoint{5.087675in}{4.519543in}}%
\pgfpathlineto{\pgfqpoint{5.095819in}{4.548270in}}%
\pgfpathlineto{\pgfqpoint{5.103859in}{4.531602in}}%
\pgfpathlineto{\pgfqpoint{5.111982in}{4.509880in}}%
\pgfpathlineto{\pgfqpoint{5.120128in}{4.532680in}}%
\pgfpathlineto{\pgfqpoint{5.128231in}{4.496253in}}%
\pgfpathlineto{\pgfqpoint{5.136344in}{4.550108in}}%
\pgfpathlineto{\pgfqpoint{5.144370in}{4.478541in}}%
\pgfpathlineto{\pgfqpoint{5.152426in}{4.481819in}}%
\pgfpathlineto{\pgfqpoint{5.160447in}{4.530772in}}%
\pgfpathlineto{\pgfqpoint{5.168547in}{4.531713in}}%
\pgfpathlineto{\pgfqpoint{5.176647in}{4.498693in}}%
\pgfpathlineto{\pgfqpoint{5.184747in}{4.559068in}}%
\pgfpathlineto{\pgfqpoint{5.192761in}{4.531508in}}%
\pgfpathlineto{\pgfqpoint{5.200810in}{4.513131in}}%
\pgfpathlineto{\pgfqpoint{5.208810in}{4.563601in}}%
\pgfpathlineto{\pgfqpoint{5.216856in}{4.529105in}}%
\pgfpathlineto{\pgfqpoint{5.224894in}{4.525459in}}%
\pgfpathlineto{\pgfqpoint{5.233040in}{4.561931in}}%
\pgfpathlineto{\pgfqpoint{5.245361in}{4.460963in}}%
\pgfpathlineto{\pgfqpoint{5.253575in}{4.503870in}}%
\pgfpathlineto{\pgfqpoint{5.261632in}{4.497161in}}%
\pgfpathlineto{\pgfqpoint{5.269666in}{4.477633in}}%
\pgfpathlineto{\pgfqpoint{5.277683in}{4.540375in}}%
\pgfpathlineto{\pgfqpoint{5.285748in}{4.533202in}}%
\pgfpathlineto{\pgfqpoint{5.293795in}{4.499284in}}%
\pgfpathlineto{\pgfqpoint{5.301824in}{4.554984in}}%
\pgfpathlineto{\pgfqpoint{5.309888in}{4.507169in}}%
\pgfpathlineto{\pgfqpoint{5.317933in}{4.519542in}}%
\pgfpathlineto{\pgfqpoint{5.327487in}{4.664334in}}%
\pgfpathlineto{\pgfqpoint{5.338668in}{4.645754in}}%
\pgfpathlineto{\pgfqpoint{5.349820in}{4.648972in}}%
\pgfpathlineto{\pgfqpoint{5.360970in}{4.641700in}}%
\pgfpathlineto{\pgfqpoint{5.372247in}{4.632160in}}%
\pgfpathlineto{\pgfqpoint{5.383480in}{4.676348in}}%
\pgfpathlineto{\pgfqpoint{5.394653in}{4.655725in}}%
\pgfpathlineto{\pgfqpoint{5.405885in}{4.661008in}}%
\pgfpathlineto{\pgfqpoint{5.405885in}{5.895867in}}%
\pgfpathlineto{\pgfqpoint{5.405885in}{5.895867in}}%
\pgfpathlineto{\pgfqpoint{5.394653in}{5.930845in}}%
\pgfpathlineto{\pgfqpoint{5.383480in}{5.918731in}}%
\pgfpathlineto{\pgfqpoint{5.372247in}{5.893963in}}%
\pgfpathlineto{\pgfqpoint{5.360970in}{5.914700in}}%
\pgfpathlineto{\pgfqpoint{5.349820in}{5.921455in}}%
\pgfpathlineto{\pgfqpoint{5.338668in}{5.915085in}}%
\pgfpathlineto{\pgfqpoint{5.327487in}{5.921926in}}%
\pgfpathlineto{\pgfqpoint{5.317933in}{5.772575in}}%
\pgfpathlineto{\pgfqpoint{5.309888in}{5.774097in}}%
\pgfpathlineto{\pgfqpoint{5.301824in}{5.807983in}}%
\pgfpathlineto{\pgfqpoint{5.293795in}{5.771881in}}%
\pgfpathlineto{\pgfqpoint{5.285748in}{5.776071in}}%
\pgfpathlineto{\pgfqpoint{5.277683in}{5.788018in}}%
\pgfpathlineto{\pgfqpoint{5.269666in}{5.749072in}}%
\pgfpathlineto{\pgfqpoint{5.261632in}{5.770596in}}%
\pgfpathlineto{\pgfqpoint{5.253575in}{5.764168in}}%
\pgfpathlineto{\pgfqpoint{5.245361in}{5.626871in}}%
\pgfpathlineto{\pgfqpoint{5.233040in}{5.798717in}}%
\pgfpathlineto{\pgfqpoint{5.224894in}{5.800994in}}%
\pgfpathlineto{\pgfqpoint{5.216856in}{5.777751in}}%
\pgfpathlineto{\pgfqpoint{5.208810in}{5.843000in}}%
\pgfpathlineto{\pgfqpoint{5.200810in}{5.772249in}}%
\pgfpathlineto{\pgfqpoint{5.192761in}{5.800716in}}%
\pgfpathlineto{\pgfqpoint{5.184747in}{5.833704in}}%
\pgfpathlineto{\pgfqpoint{5.176647in}{5.752645in}}%
\pgfpathlineto{\pgfqpoint{5.168547in}{5.772623in}}%
\pgfpathlineto{\pgfqpoint{5.160447in}{5.791328in}}%
\pgfpathlineto{\pgfqpoint{5.152426in}{5.755535in}}%
\pgfpathlineto{\pgfqpoint{5.144370in}{5.757737in}}%
\pgfpathlineto{\pgfqpoint{5.136344in}{5.809823in}}%
\pgfpathlineto{\pgfqpoint{5.128231in}{5.778118in}}%
\pgfpathlineto{\pgfqpoint{5.120128in}{5.787803in}}%
\pgfpathlineto{\pgfqpoint{5.111982in}{5.756765in}}%
\pgfpathlineto{\pgfqpoint{5.103859in}{5.799078in}}%
\pgfpathlineto{\pgfqpoint{5.095819in}{5.810178in}}%
\pgfpathlineto{\pgfqpoint{5.087675in}{5.777296in}}%
\pgfpathlineto{\pgfqpoint{5.079506in}{5.764586in}}%
\pgfpathlineto{\pgfqpoint{5.071391in}{5.743381in}}%
\pgfpathlineto{\pgfqpoint{5.063216in}{5.751693in}}%
\pgfpathlineto{\pgfqpoint{5.055031in}{5.772021in}}%
\pgfpathlineto{\pgfqpoint{5.046876in}{5.762825in}}%
\pgfpathlineto{\pgfqpoint{5.038759in}{5.800005in}}%
\pgfpathlineto{\pgfqpoint{5.030602in}{5.779251in}}%
\pgfpathlineto{\pgfqpoint{5.022451in}{5.810378in}}%
\pgfpathlineto{\pgfqpoint{5.014317in}{5.759150in}}%
\pgfpathlineto{\pgfqpoint{5.006134in}{5.730243in}}%
\pgfpathlineto{\pgfqpoint{4.997948in}{5.791630in}}%
\pgfpathlineto{\pgfqpoint{4.989794in}{5.790295in}}%
\pgfpathlineto{\pgfqpoint{4.981642in}{5.749748in}}%
\pgfpathlineto{\pgfqpoint{4.973428in}{5.749887in}}%
\pgfpathlineto{\pgfqpoint{4.965138in}{5.735566in}}%
\pgfpathlineto{\pgfqpoint{4.956992in}{5.769930in}}%
\pgfpathlineto{\pgfqpoint{4.948774in}{5.775980in}}%
\pgfpathlineto{\pgfqpoint{4.940566in}{5.746612in}}%
\pgfpathlineto{\pgfqpoint{4.932347in}{5.781491in}}%
\pgfpathlineto{\pgfqpoint{4.924041in}{5.718479in}}%
\pgfpathlineto{\pgfqpoint{4.915736in}{5.762748in}}%
\pgfpathlineto{\pgfqpoint{4.907476in}{5.749439in}}%
\pgfpathlineto{\pgfqpoint{4.899205in}{5.808608in}}%
\pgfpathlineto{\pgfqpoint{4.890980in}{5.768961in}}%
\pgfpathlineto{\pgfqpoint{4.882658in}{5.744814in}}%
\pgfpathlineto{\pgfqpoint{4.874381in}{5.730820in}}%
\pgfpathlineto{\pgfqpoint{4.866069in}{5.771438in}}%
\pgfpathlineto{\pgfqpoint{4.857830in}{5.826175in}}%
\pgfpathlineto{\pgfqpoint{4.849575in}{5.741143in}}%
\pgfpathlineto{\pgfqpoint{4.841325in}{5.807933in}}%
\pgfpathlineto{\pgfqpoint{4.833061in}{5.772472in}}%
\pgfpathlineto{\pgfqpoint{4.824772in}{5.734477in}}%
\pgfpathlineto{\pgfqpoint{4.816444in}{5.754443in}}%
\pgfpathlineto{\pgfqpoint{4.808134in}{5.774982in}}%
\pgfpathlineto{\pgfqpoint{4.799850in}{5.772374in}}%
\pgfpathlineto{\pgfqpoint{4.791573in}{5.744669in}}%
\pgfpathlineto{\pgfqpoint{4.783272in}{5.739558in}}%
\pgfpathlineto{\pgfqpoint{4.774881in}{5.759734in}}%
\pgfpathlineto{\pgfqpoint{4.766528in}{5.731101in}}%
\pgfpathlineto{\pgfqpoint{4.758119in}{5.719410in}}%
\pgfpathlineto{\pgfqpoint{4.749750in}{5.752573in}}%
\pgfpathlineto{\pgfqpoint{4.741378in}{5.740711in}}%
\pgfpathlineto{\pgfqpoint{4.733049in}{5.755550in}}%
\pgfpathlineto{\pgfqpoint{4.724666in}{5.710039in}}%
\pgfpathlineto{\pgfqpoint{4.716247in}{5.766892in}}%
\pgfpathlineto{\pgfqpoint{4.707865in}{5.727526in}}%
\pgfpathlineto{\pgfqpoint{4.699465in}{5.767184in}}%
\pgfpathlineto{\pgfqpoint{4.691071in}{5.752686in}}%
\pgfpathlineto{\pgfqpoint{4.682643in}{5.761989in}}%
\pgfpathlineto{\pgfqpoint{4.674255in}{5.730938in}}%
\pgfpathlineto{\pgfqpoint{4.665825in}{5.647464in}}%
\pgfpathlineto{\pgfqpoint{4.657349in}{5.720245in}}%
\pgfpathlineto{\pgfqpoint{4.648847in}{5.589818in}}%
\pgfpathlineto{\pgfqpoint{4.640221in}{5.648943in}}%
\pgfpathlineto{\pgfqpoint{4.631733in}{5.716138in}}%
\pgfpathlineto{\pgfqpoint{4.623282in}{5.744681in}}%
\pgfpathlineto{\pgfqpoint{4.614783in}{5.689586in}}%
\pgfpathlineto{\pgfqpoint{4.606211in}{5.663382in}}%
\pgfpathlineto{\pgfqpoint{4.597719in}{5.726365in}}%
\pgfpathlineto{\pgfqpoint{4.589255in}{5.729893in}}%
\pgfpathlineto{\pgfqpoint{4.580727in}{5.678539in}}%
\pgfpathlineto{\pgfqpoint{4.572097in}{5.633323in}}%
\pgfpathlineto{\pgfqpoint{4.563550in}{5.668664in}}%
\pgfpathlineto{\pgfqpoint{4.554980in}{5.675694in}}%
\pgfpathlineto{\pgfqpoint{4.546439in}{5.694968in}}%
\pgfpathlineto{\pgfqpoint{4.537876in}{5.705940in}}%
\pgfpathlineto{\pgfqpoint{4.529276in}{5.617804in}}%
\pgfpathlineto{\pgfqpoint{4.520678in}{5.676077in}}%
\pgfpathlineto{\pgfqpoint{4.512084in}{5.613677in}}%
\pgfpathlineto{\pgfqpoint{4.503421in}{5.640411in}}%
\pgfpathlineto{\pgfqpoint{4.494752in}{5.627120in}}%
\pgfpathlineto{\pgfqpoint{4.486071in}{5.651740in}}%
\pgfpathlineto{\pgfqpoint{4.477466in}{5.657641in}}%
\pgfpathlineto{\pgfqpoint{4.468771in}{5.583055in}}%
\pgfpathlineto{\pgfqpoint{4.460011in}{5.648065in}}%
\pgfpathlineto{\pgfqpoint{4.451305in}{5.598857in}}%
\pgfpathlineto{\pgfqpoint{4.442591in}{5.633767in}}%
\pgfpathlineto{\pgfqpoint{4.433894in}{5.646519in}}%
\pgfpathlineto{\pgfqpoint{4.425248in}{5.593307in}}%
\pgfpathlineto{\pgfqpoint{4.416522in}{5.618644in}}%
\pgfpathlineto{\pgfqpoint{4.407811in}{5.653801in}}%
\pgfpathlineto{\pgfqpoint{4.399156in}{5.630819in}}%
\pgfpathlineto{\pgfqpoint{4.390451in}{5.644630in}}%
\pgfpathlineto{\pgfqpoint{4.381765in}{5.652077in}}%
\pgfpathlineto{\pgfqpoint{4.373001in}{5.569817in}}%
\pgfpathlineto{\pgfqpoint{4.364219in}{5.583440in}}%
\pgfpathlineto{\pgfqpoint{4.355507in}{5.612892in}}%
\pgfpathlineto{\pgfqpoint{4.346774in}{5.228899in}}%
\pgfpathlineto{\pgfqpoint{4.337933in}{4.614224in}}%
\pgfpathlineto{\pgfqpoint{4.328717in}{4.560985in}}%
\pgfpathlineto{\pgfqpoint{4.319388in}{4.531301in}}%
\pgfpathlineto{\pgfqpoint{4.310022in}{4.460577in}}%
\pgfpathlineto{\pgfqpoint{4.300486in}{4.510374in}}%
\pgfpathlineto{\pgfqpoint{4.290581in}{4.490608in}}%
\pgfpathlineto{\pgfqpoint{4.280743in}{4.467783in}}%
\pgfpathlineto{\pgfqpoint{4.270972in}{3.952584in}}%
\pgfpathlineto{\pgfqpoint{4.260400in}{4.276035in}}%
\pgfpathlineto{\pgfqpoint{4.249866in}{3.475498in}}%
\pgfpathlineto{\pgfqpoint{4.238844in}{3.416556in}}%
\pgfpathlineto{\pgfqpoint{4.227703in}{3.466682in}}%
\pgfpathlineto{\pgfqpoint{4.216619in}{3.406465in}}%
\pgfpathlineto{\pgfqpoint{4.205594in}{3.377115in}}%
\pgfpathlineto{\pgfqpoint{4.194400in}{3.395911in}}%
\pgfpathlineto{\pgfqpoint{4.183285in}{3.423197in}}%
\pgfpathlineto{\pgfqpoint{4.172101in}{3.440801in}}%
\pgfpathlineto{\pgfqpoint{4.160983in}{3.356308in}}%
\pgfpathlineto{\pgfqpoint{4.149813in}{3.346259in}}%
\pgfpathlineto{\pgfqpoint{4.138598in}{3.328045in}}%
\pgfpathlineto{\pgfqpoint{4.127322in}{3.347034in}}%
\pgfpathlineto{\pgfqpoint{4.116014in}{3.335023in}}%
\pgfpathlineto{\pgfqpoint{4.104772in}{3.347125in}}%
\pgfpathlineto{\pgfqpoint{4.093537in}{3.357141in}}%
\pgfpathlineto{\pgfqpoint{4.082264in}{3.331972in}}%
\pgfpathlineto{\pgfqpoint{4.070994in}{3.315452in}}%
\pgfpathlineto{\pgfqpoint{4.059512in}{3.288197in}}%
\pgfpathlineto{\pgfqpoint{4.048157in}{3.330196in}}%
\pgfpathlineto{\pgfqpoint{4.036918in}{3.310971in}}%
\pgfpathlineto{\pgfqpoint{4.025578in}{3.321021in}}%
\pgfpathlineto{\pgfqpoint{4.014312in}{3.321179in}}%
\pgfpathlineto{\pgfqpoint{4.003035in}{3.301529in}}%
\pgfpathlineto{\pgfqpoint{3.991607in}{3.293811in}}%
\pgfpathlineto{\pgfqpoint{3.980165in}{3.303509in}}%
\pgfpathlineto{\pgfqpoint{3.968800in}{3.304561in}}%
\pgfpathlineto{\pgfqpoint{3.957345in}{3.272861in}}%
\pgfpathlineto{\pgfqpoint{3.945995in}{3.303172in}}%
\pgfpathlineto{\pgfqpoint{3.934526in}{3.270407in}}%
\pgfpathlineto{\pgfqpoint{3.923007in}{3.279552in}}%
\pgfpathlineto{\pgfqpoint{3.911548in}{3.298420in}}%
\pgfpathlineto{\pgfqpoint{3.900120in}{3.289357in}}%
\pgfpathlineto{\pgfqpoint{3.888612in}{3.246935in}}%
\pgfpathlineto{\pgfqpoint{3.876999in}{3.257842in}}%
\pgfpathlineto{\pgfqpoint{3.865394in}{3.259136in}}%
\pgfpathlineto{\pgfqpoint{3.853773in}{3.270294in}}%
\pgfpathlineto{\pgfqpoint{3.842268in}{3.276823in}}%
\pgfpathlineto{\pgfqpoint{3.830748in}{3.270760in}}%
\pgfpathlineto{\pgfqpoint{3.819207in}{3.269204in}}%
\pgfpathlineto{\pgfqpoint{3.807700in}{3.276547in}}%
\pgfpathlineto{\pgfqpoint{3.796179in}{3.125184in}}%
\pgfpathlineto{\pgfqpoint{3.784213in}{2.593742in}}%
\pgfpathlineto{\pgfqpoint{3.772194in}{2.598661in}}%
\pgfpathlineto{\pgfqpoint{3.760252in}{2.598508in}}%
\pgfpathlineto{\pgfqpoint{3.748246in}{2.575758in}}%
\pgfpathlineto{\pgfqpoint{3.736218in}{2.577451in}}%
\pgfpathlineto{\pgfqpoint{3.723925in}{2.530157in}}%
\pgfpathlineto{\pgfqpoint{3.711790in}{2.585210in}}%
\pgfpathlineto{\pgfqpoint{3.699766in}{2.592531in}}%
\pgfpathlineto{\pgfqpoint{3.687677in}{2.585086in}}%
\pgfpathlineto{\pgfqpoint{3.675639in}{2.571801in}}%
\pgfpathlineto{\pgfqpoint{3.663361in}{2.540991in}}%
\pgfpathlineto{\pgfqpoint{3.651005in}{2.581523in}}%
\pgfpathlineto{\pgfqpoint{3.639052in}{2.598530in}}%
\pgfpathlineto{\pgfqpoint{3.627108in}{2.575256in}}%
\pgfpathlineto{\pgfqpoint{3.615064in}{2.605081in}}%
\pgfpathlineto{\pgfqpoint{3.602917in}{2.543862in}}%
\pgfpathlineto{\pgfqpoint{3.590359in}{2.545882in}}%
\pgfpathlineto{\pgfqpoint{3.578260in}{2.602945in}}%
\pgfpathlineto{\pgfqpoint{3.566335in}{2.592872in}}%
\pgfpathlineto{\pgfqpoint{3.554208in}{2.568903in}}%
\pgfpathlineto{\pgfqpoint{3.542041in}{2.390195in}}%
\pgfpathlineto{\pgfqpoint{3.528239in}{1.727882in}}%
\pgfpathlineto{\pgfqpoint{3.513323in}{1.725642in}}%
\pgfpathlineto{\pgfqpoint{3.498190in}{1.732610in}}%
\pgfpathlineto{\pgfqpoint{3.483019in}{1.712946in}}%
\pgfpathlineto{\pgfqpoint{3.467311in}{1.696086in}}%
\pgfpathlineto{\pgfqpoint{3.451777in}{1.693451in}}%
\pgfpathlineto{\pgfqpoint{3.436431in}{1.698643in}}%
\pgfpathlineto{\pgfqpoint{3.421182in}{1.730302in}}%
\pgfpathlineto{\pgfqpoint{3.405908in}{1.691564in}}%
\pgfpathlineto{\pgfqpoint{3.390722in}{1.717980in}}%
\pgfpathlineto{\pgfqpoint{3.375355in}{1.727960in}}%
\pgfpathlineto{\pgfqpoint{3.359992in}{1.709239in}}%
\pgfpathlineto{\pgfqpoint{3.344594in}{1.719344in}}%
\pgfpathlineto{\pgfqpoint{3.328928in}{1.692617in}}%
\pgfpathlineto{\pgfqpoint{3.313329in}{1.720965in}}%
\pgfpathlineto{\pgfqpoint{3.295281in}{1.659311in}}%
\pgfpathlineto{\pgfqpoint{3.279179in}{1.711362in}}%
\pgfpathlineto{\pgfqpoint{3.263722in}{1.706885in}}%
\pgfpathlineto{\pgfqpoint{3.248207in}{1.720364in}}%
\pgfpathlineto{\pgfqpoint{3.232763in}{1.711543in}}%
\pgfpathlineto{\pgfqpoint{3.217213in}{1.716683in}}%
\pgfpathlineto{\pgfqpoint{3.201602in}{1.694354in}}%
\pgfpathlineto{\pgfqpoint{3.185752in}{1.666680in}}%
\pgfpathlineto{\pgfqpoint{3.169651in}{1.649844in}}%
\pgfpathlineto{\pgfqpoint{3.152584in}{1.632309in}}%
\pgfpathlineto{\pgfqpoint{3.135319in}{1.656535in}}%
\pgfpathlineto{\pgfqpoint{3.118692in}{1.629434in}}%
\pgfpathlineto{\pgfqpoint{3.102316in}{1.661253in}}%
\pgfpathlineto{\pgfqpoint{3.085686in}{1.659915in}}%
\pgfpathlineto{\pgfqpoint{3.069003in}{1.660399in}}%
\pgfpathlineto{\pgfqpoint{3.052708in}{1.679290in}}%
\pgfpathlineto{\pgfqpoint{3.036276in}{1.662595in}}%
\pgfpathlineto{\pgfqpoint{3.018968in}{1.669173in}}%
\pgfpathlineto{\pgfqpoint{3.001921in}{1.669098in}}%
\pgfpathlineto{\pgfqpoint{2.985853in}{1.669384in}}%
\pgfpathlineto{\pgfqpoint{2.968433in}{1.588655in}}%
\pgfpathlineto{\pgfqpoint{2.949130in}{1.615946in}}%
\pgfpathlineto{\pgfqpoint{2.930775in}{1.626155in}}%
\pgfpathlineto{\pgfqpoint{2.914048in}{1.669681in}}%
\pgfpathlineto{\pgfqpoint{2.897563in}{1.668340in}}%
\pgfpathlineto{\pgfqpoint{2.880921in}{1.652682in}}%
\pgfpathlineto{\pgfqpoint{2.864455in}{1.661176in}}%
\pgfpathlineto{\pgfqpoint{2.848081in}{1.681824in}}%
\pgfpathlineto{\pgfqpoint{2.831972in}{1.672536in}}%
\pgfpathlineto{\pgfqpoint{2.815750in}{1.642009in}}%
\pgfpathlineto{\pgfqpoint{2.798917in}{1.604933in}}%
\pgfpathlineto{\pgfqpoint{2.780862in}{1.624096in}}%
\pgfpathlineto{\pgfqpoint{2.763118in}{1.644656in}}%
\pgfpathlineto{\pgfqpoint{2.744840in}{1.631636in}}%
\pgfpathlineto{\pgfqpoint{2.727722in}{1.668753in}}%
\pgfpathlineto{\pgfqpoint{2.710992in}{1.652418in}}%
\pgfpathlineto{\pgfqpoint{2.693540in}{1.610198in}}%
\pgfpathlineto{\pgfqpoint{2.676022in}{1.636396in}}%
\pgfpathlineto{\pgfqpoint{2.657274in}{1.602893in}}%
\pgfpathlineto{\pgfqpoint{2.636069in}{1.475557in}}%
\pgfpathlineto{\pgfqpoint{2.616149in}{1.661939in}}%
\pgfpathlineto{\pgfqpoint{2.598064in}{1.579268in}}%
\pgfpathlineto{\pgfqpoint{2.578521in}{1.601882in}}%
\pgfpathlineto{\pgfqpoint{2.558886in}{1.602098in}}%
\pgfpathlineto{\pgfqpoint{2.538352in}{1.471475in}}%
\pgfpathlineto{\pgfqpoint{2.515773in}{1.604114in}}%
\pgfpathlineto{\pgfqpoint{2.494737in}{1.665302in}}%
\pgfpathlineto{\pgfqpoint{2.477723in}{1.624256in}}%
\pgfpathlineto{\pgfqpoint{2.457197in}{1.469475in}}%
\pgfpathlineto{\pgfqpoint{2.433282in}{1.790336in}}%
\pgfpathlineto{\pgfqpoint{2.418633in}{1.914543in}}%
\pgfpathlineto{\pgfqpoint{2.406207in}{1.939552in}}%
\pgfpathlineto{\pgfqpoint{2.393733in}{1.944875in}}%
\pgfpathlineto{\pgfqpoint{2.381372in}{1.955963in}}%
\pgfpathlineto{\pgfqpoint{2.368880in}{1.925406in}}%
\pgfpathlineto{\pgfqpoint{2.356417in}{1.932907in}}%
\pgfpathlineto{\pgfqpoint{2.343909in}{1.917109in}}%
\pgfpathlineto{\pgfqpoint{2.331395in}{1.908842in}}%
\pgfpathlineto{\pgfqpoint{2.318891in}{1.958564in}}%
\pgfpathlineto{\pgfqpoint{2.306399in}{1.915408in}}%
\pgfpathlineto{\pgfqpoint{2.293873in}{1.922987in}}%
\pgfpathlineto{\pgfqpoint{2.281255in}{1.886079in}}%
\pgfpathlineto{\pgfqpoint{2.268761in}{1.917857in}}%
\pgfpathlineto{\pgfqpoint{2.256287in}{1.942930in}}%
\pgfpathlineto{\pgfqpoint{2.243791in}{1.909316in}}%
\pgfpathlineto{\pgfqpoint{2.231301in}{1.916497in}}%
\pgfpathlineto{\pgfqpoint{2.218726in}{1.903234in}}%
\pgfpathlineto{\pgfqpoint{2.206174in}{1.899320in}}%
\pgfpathlineto{\pgfqpoint{2.193673in}{1.912356in}}%
\pgfpathlineto{\pgfqpoint{2.181172in}{1.929017in}}%
\pgfpathlineto{\pgfqpoint{2.168659in}{1.918886in}}%
\pgfpathlineto{\pgfqpoint{2.156064in}{1.928760in}}%
\pgfpathlineto{\pgfqpoint{2.143512in}{1.955634in}}%
\pgfpathlineto{\pgfqpoint{2.130987in}{1.937450in}}%
\pgfpathlineto{\pgfqpoint{2.118500in}{1.942727in}}%
\pgfpathlineto{\pgfqpoint{2.106029in}{1.935843in}}%
\pgfpathlineto{\pgfqpoint{2.093434in}{1.928647in}}%
\pgfpathlineto{\pgfqpoint{2.080914in}{1.924213in}}%
\pgfpathlineto{\pgfqpoint{2.068360in}{1.935079in}}%
\pgfpathlineto{\pgfqpoint{2.055871in}{1.963732in}}%
\pgfpathlineto{\pgfqpoint{2.043350in}{1.923557in}}%
\pgfpathlineto{\pgfqpoint{2.030840in}{1.926915in}}%
\pgfpathlineto{\pgfqpoint{2.016889in}{1.881169in}}%
\pgfpathlineto{\pgfqpoint{2.004212in}{1.931372in}}%
\pgfpathlineto{\pgfqpoint{1.991646in}{1.937850in}}%
\pgfpathlineto{\pgfqpoint{1.979161in}{1.939602in}}%
\pgfpathlineto{\pgfqpoint{1.966664in}{1.913946in}}%
\pgfpathlineto{\pgfqpoint{1.954108in}{1.923125in}}%
\pgfpathlineto{\pgfqpoint{1.941620in}{1.915053in}}%
\pgfpathlineto{\pgfqpoint{1.929118in}{1.919890in}}%
\pgfpathlineto{\pgfqpoint{1.916611in}{1.938068in}}%
\pgfpathlineto{\pgfqpoint{1.904130in}{1.957152in}}%
\pgfpathlineto{\pgfqpoint{1.891620in}{1.943446in}}%
\pgfpathlineto{\pgfqpoint{1.879180in}{1.918392in}}%
\pgfpathlineto{\pgfqpoint{1.866747in}{1.955313in}}%
\pgfpathlineto{\pgfqpoint{1.854433in}{1.938709in}}%
\pgfpathlineto{\pgfqpoint{1.842065in}{1.929290in}}%
\pgfpathlineto{\pgfqpoint{1.829628in}{1.936697in}}%
\pgfpathlineto{\pgfqpoint{1.817148in}{1.917657in}}%
\pgfpathlineto{\pgfqpoint{1.804716in}{1.905153in}}%
\pgfpathlineto{\pgfqpoint{1.792355in}{1.950863in}}%
\pgfpathlineto{\pgfqpoint{1.779913in}{1.941308in}}%
\pgfpathlineto{\pgfqpoint{1.767480in}{1.933256in}}%
\pgfpathlineto{\pgfqpoint{1.755070in}{1.922399in}}%
\pgfpathlineto{\pgfqpoint{1.742593in}{1.944122in}}%
\pgfpathlineto{\pgfqpoint{1.730200in}{1.947889in}}%
\pgfpathlineto{\pgfqpoint{1.717832in}{1.955267in}}%
\pgfpathlineto{\pgfqpoint{1.705498in}{1.936407in}}%
\pgfpathlineto{\pgfqpoint{1.693005in}{1.943090in}}%
\pgfpathlineto{\pgfqpoint{1.680578in}{1.931054in}}%
\pgfpathlineto{\pgfqpoint{1.668187in}{1.922290in}}%
\pgfpathlineto{\pgfqpoint{1.655673in}{1.927241in}}%
\pgfpathlineto{\pgfqpoint{1.643249in}{1.941247in}}%
\pgfpathlineto{\pgfqpoint{1.630788in}{1.939007in}}%
\pgfpathlineto{\pgfqpoint{1.618336in}{1.922291in}}%
\pgfpathlineto{\pgfqpoint{1.605869in}{1.896850in}}%
\pgfpathlineto{\pgfqpoint{1.593430in}{1.951157in}}%
\pgfpathlineto{\pgfqpoint{1.581070in}{1.937729in}}%
\pgfpathlineto{\pgfqpoint{1.568672in}{1.939876in}}%
\pgfpathlineto{\pgfqpoint{1.556290in}{1.936461in}}%
\pgfpathlineto{\pgfqpoint{1.543839in}{1.939697in}}%
\pgfpathlineto{\pgfqpoint{1.531474in}{1.924340in}}%
\pgfpathlineto{\pgfqpoint{1.519060in}{1.920852in}}%
\pgfpathlineto{\pgfqpoint{1.506021in}{1.900982in}}%
\pgfpathlineto{\pgfqpoint{1.493578in}{1.915129in}}%
\pgfpathlineto{\pgfqpoint{1.481163in}{1.901372in}}%
\pgfpathlineto{\pgfqpoint{1.468660in}{1.925027in}}%
\pgfpathlineto{\pgfqpoint{1.456188in}{1.946037in}}%
\pgfpathlineto{\pgfqpoint{1.443756in}{1.952535in}}%
\pgfpathlineto{\pgfqpoint{1.431382in}{1.951306in}}%
\pgfpathlineto{\pgfqpoint{1.418893in}{1.891786in}}%
\pgfpathlineto{\pgfqpoint{1.406401in}{1.934768in}}%
\pgfpathlineto{\pgfqpoint{1.393906in}{1.951000in}}%
\pgfpathlineto{\pgfqpoint{1.381448in}{1.951022in}}%
\pgfpathlineto{\pgfqpoint{1.368992in}{1.938178in}}%
\pgfpathlineto{\pgfqpoint{1.356458in}{1.903454in}}%
\pgfpathlineto{\pgfqpoint{1.343969in}{1.903425in}}%
\pgfpathlineto{\pgfqpoint{1.331552in}{1.943959in}}%
\pgfpathlineto{\pgfqpoint{1.319073in}{1.917673in}}%
\pgfpathlineto{\pgfqpoint{1.306603in}{1.931091in}}%
\pgfpathlineto{\pgfqpoint{1.294165in}{1.946348in}}%
\pgfpathlineto{\pgfqpoint{1.281753in}{1.957243in}}%
\pgfpathlineto{\pgfqpoint{1.269255in}{1.913596in}}%
\pgfpathlineto{\pgfqpoint{1.256724in}{1.916419in}}%
\pgfpathlineto{\pgfqpoint{1.244104in}{1.919205in}}%
\pgfpathlineto{\pgfqpoint{1.231699in}{1.933755in}}%
\pgfpathlineto{\pgfqpoint{1.219183in}{1.938086in}}%
\pgfpathlineto{\pgfqpoint{1.206622in}{1.916431in}}%
\pgfpathlineto{\pgfqpoint{1.194024in}{1.900637in}}%
\pgfpathlineto{\pgfqpoint{1.181503in}{1.946726in}}%
\pgfpathlineto{\pgfqpoint{1.168924in}{1.917406in}}%
\pgfpathlineto{\pgfqpoint{1.156226in}{1.904964in}}%
\pgfpathlineto{\pgfqpoint{1.143621in}{1.914901in}}%
\pgfpathlineto{\pgfqpoint{1.131014in}{1.898634in}}%
\pgfpathlineto{\pgfqpoint{1.118303in}{1.903227in}}%
\pgfpathlineto{\pgfqpoint{1.105655in}{1.886147in}}%
\pgfpathlineto{\pgfqpoint{1.092853in}{1.909080in}}%
\pgfpathlineto{\pgfqpoint{1.079972in}{1.910422in}}%
\pgfpathlineto{\pgfqpoint{1.067143in}{1.878003in}}%
\pgfpathlineto{\pgfqpoint{1.054264in}{1.898389in}}%
\pgfpathlineto{\pgfqpoint{1.041322in}{1.876378in}}%
\pgfpathlineto{\pgfqpoint{1.028303in}{1.861608in}}%
\pgfpathlineto{\pgfqpoint{1.015150in}{1.869985in}}%
\pgfpathlineto{\pgfqpoint{1.001616in}{1.770374in}}%
\pgfpathclose%
\pgfusepath{fill}%
\end{pgfscope}%
\begin{pgfscope}%
\pgfpathrectangle{\pgfqpoint{0.781402in}{0.773588in}}{\pgfqpoint{4.844695in}{5.415119in}}%
\pgfusepath{clip}%
\pgfsetbuttcap%
\pgfsetroundjoin%
\definecolor{currentfill}{rgb}{0.580392,0.403922,0.741176}%
\pgfsetfillcolor{currentfill}%
\pgfsetlinewidth{0.000000pt}%
\definecolor{currentstroke}{rgb}{0.000000,0.000000,0.000000}%
\pgfsetstrokecolor{currentstroke}%
\pgfsetdash{}{0pt}%
\pgfpathmoveto{\pgfqpoint{1.001616in}{2.273755in}}%
\pgfpathlineto{\pgfqpoint{1.001616in}{1.770374in}}%
\pgfpathlineto{\pgfqpoint{1.015150in}{1.869985in}}%
\pgfpathlineto{\pgfqpoint{1.028303in}{1.861608in}}%
\pgfpathlineto{\pgfqpoint{1.041322in}{1.876378in}}%
\pgfpathlineto{\pgfqpoint{1.054264in}{1.898389in}}%
\pgfpathlineto{\pgfqpoint{1.067143in}{1.878003in}}%
\pgfpathlineto{\pgfqpoint{1.079972in}{1.910422in}}%
\pgfpathlineto{\pgfqpoint{1.092853in}{1.909080in}}%
\pgfpathlineto{\pgfqpoint{1.105655in}{1.886147in}}%
\pgfpathlineto{\pgfqpoint{1.118303in}{1.903227in}}%
\pgfpathlineto{\pgfqpoint{1.131014in}{1.898634in}}%
\pgfpathlineto{\pgfqpoint{1.143621in}{1.914901in}}%
\pgfpathlineto{\pgfqpoint{1.156226in}{1.904964in}}%
\pgfpathlineto{\pgfqpoint{1.168924in}{1.917406in}}%
\pgfpathlineto{\pgfqpoint{1.181503in}{1.946726in}}%
\pgfpathlineto{\pgfqpoint{1.194024in}{1.900637in}}%
\pgfpathlineto{\pgfqpoint{1.206622in}{1.916431in}}%
\pgfpathlineto{\pgfqpoint{1.219183in}{1.938086in}}%
\pgfpathlineto{\pgfqpoint{1.231699in}{1.933755in}}%
\pgfpathlineto{\pgfqpoint{1.244104in}{1.919205in}}%
\pgfpathlineto{\pgfqpoint{1.256724in}{1.916419in}}%
\pgfpathlineto{\pgfqpoint{1.269255in}{1.913596in}}%
\pgfpathlineto{\pgfqpoint{1.281753in}{1.957243in}}%
\pgfpathlineto{\pgfqpoint{1.294165in}{1.946348in}}%
\pgfpathlineto{\pgfqpoint{1.306603in}{1.931091in}}%
\pgfpathlineto{\pgfqpoint{1.319073in}{1.917673in}}%
\pgfpathlineto{\pgfqpoint{1.331552in}{1.943959in}}%
\pgfpathlineto{\pgfqpoint{1.343969in}{1.903425in}}%
\pgfpathlineto{\pgfqpoint{1.356458in}{1.903454in}}%
\pgfpathlineto{\pgfqpoint{1.368992in}{1.938178in}}%
\pgfpathlineto{\pgfqpoint{1.381448in}{1.951022in}}%
\pgfpathlineto{\pgfqpoint{1.393906in}{1.951000in}}%
\pgfpathlineto{\pgfqpoint{1.406401in}{1.934768in}}%
\pgfpathlineto{\pgfqpoint{1.418893in}{1.891786in}}%
\pgfpathlineto{\pgfqpoint{1.431382in}{1.951306in}}%
\pgfpathlineto{\pgfqpoint{1.443756in}{1.952535in}}%
\pgfpathlineto{\pgfqpoint{1.456188in}{1.946037in}}%
\pgfpathlineto{\pgfqpoint{1.468660in}{1.925027in}}%
\pgfpathlineto{\pgfqpoint{1.481163in}{1.901372in}}%
\pgfpathlineto{\pgfqpoint{1.493578in}{1.915129in}}%
\pgfpathlineto{\pgfqpoint{1.506021in}{1.900982in}}%
\pgfpathlineto{\pgfqpoint{1.519060in}{1.920852in}}%
\pgfpathlineto{\pgfqpoint{1.531474in}{1.924340in}}%
\pgfpathlineto{\pgfqpoint{1.543839in}{1.939697in}}%
\pgfpathlineto{\pgfqpoint{1.556290in}{1.936461in}}%
\pgfpathlineto{\pgfqpoint{1.568672in}{1.939876in}}%
\pgfpathlineto{\pgfqpoint{1.581070in}{1.937729in}}%
\pgfpathlineto{\pgfqpoint{1.593430in}{1.951157in}}%
\pgfpathlineto{\pgfqpoint{1.605869in}{1.896850in}}%
\pgfpathlineto{\pgfqpoint{1.618336in}{1.922291in}}%
\pgfpathlineto{\pgfqpoint{1.630788in}{1.939007in}}%
\pgfpathlineto{\pgfqpoint{1.643249in}{1.941247in}}%
\pgfpathlineto{\pgfqpoint{1.655673in}{1.927241in}}%
\pgfpathlineto{\pgfqpoint{1.668187in}{1.922290in}}%
\pgfpathlineto{\pgfqpoint{1.680578in}{1.931054in}}%
\pgfpathlineto{\pgfqpoint{1.693005in}{1.943090in}}%
\pgfpathlineto{\pgfqpoint{1.705498in}{1.936407in}}%
\pgfpathlineto{\pgfqpoint{1.717832in}{1.955267in}}%
\pgfpathlineto{\pgfqpoint{1.730200in}{1.947889in}}%
\pgfpathlineto{\pgfqpoint{1.742593in}{1.944122in}}%
\pgfpathlineto{\pgfqpoint{1.755070in}{1.922399in}}%
\pgfpathlineto{\pgfqpoint{1.767480in}{1.933256in}}%
\pgfpathlineto{\pgfqpoint{1.779913in}{1.941308in}}%
\pgfpathlineto{\pgfqpoint{1.792355in}{1.950863in}}%
\pgfpathlineto{\pgfqpoint{1.804716in}{1.905153in}}%
\pgfpathlineto{\pgfqpoint{1.817148in}{1.917657in}}%
\pgfpathlineto{\pgfqpoint{1.829628in}{1.936697in}}%
\pgfpathlineto{\pgfqpoint{1.842065in}{1.929290in}}%
\pgfpathlineto{\pgfqpoint{1.854433in}{1.938709in}}%
\pgfpathlineto{\pgfqpoint{1.866747in}{1.955313in}}%
\pgfpathlineto{\pgfqpoint{1.879180in}{1.918392in}}%
\pgfpathlineto{\pgfqpoint{1.891620in}{1.943446in}}%
\pgfpathlineto{\pgfqpoint{1.904130in}{1.957152in}}%
\pgfpathlineto{\pgfqpoint{1.916611in}{1.938068in}}%
\pgfpathlineto{\pgfqpoint{1.929118in}{1.919890in}}%
\pgfpathlineto{\pgfqpoint{1.941620in}{1.915053in}}%
\pgfpathlineto{\pgfqpoint{1.954108in}{1.923125in}}%
\pgfpathlineto{\pgfqpoint{1.966664in}{1.913946in}}%
\pgfpathlineto{\pgfqpoint{1.979161in}{1.939602in}}%
\pgfpathlineto{\pgfqpoint{1.991646in}{1.937850in}}%
\pgfpathlineto{\pgfqpoint{2.004212in}{1.931372in}}%
\pgfpathlineto{\pgfqpoint{2.016889in}{1.881169in}}%
\pgfpathlineto{\pgfqpoint{2.030840in}{1.926915in}}%
\pgfpathlineto{\pgfqpoint{2.043350in}{1.923557in}}%
\pgfpathlineto{\pgfqpoint{2.055871in}{1.963732in}}%
\pgfpathlineto{\pgfqpoint{2.068360in}{1.935079in}}%
\pgfpathlineto{\pgfqpoint{2.080914in}{1.924213in}}%
\pgfpathlineto{\pgfqpoint{2.093434in}{1.928647in}}%
\pgfpathlineto{\pgfqpoint{2.106029in}{1.935843in}}%
\pgfpathlineto{\pgfqpoint{2.118500in}{1.942727in}}%
\pgfpathlineto{\pgfqpoint{2.130987in}{1.937450in}}%
\pgfpathlineto{\pgfqpoint{2.143512in}{1.955634in}}%
\pgfpathlineto{\pgfqpoint{2.156064in}{1.928760in}}%
\pgfpathlineto{\pgfqpoint{2.168659in}{1.918886in}}%
\pgfpathlineto{\pgfqpoint{2.181172in}{1.929017in}}%
\pgfpathlineto{\pgfqpoint{2.193673in}{1.912356in}}%
\pgfpathlineto{\pgfqpoint{2.206174in}{1.899320in}}%
\pgfpathlineto{\pgfqpoint{2.218726in}{1.903234in}}%
\pgfpathlineto{\pgfqpoint{2.231301in}{1.916497in}}%
\pgfpathlineto{\pgfqpoint{2.243791in}{1.909316in}}%
\pgfpathlineto{\pgfqpoint{2.256287in}{1.942930in}}%
\pgfpathlineto{\pgfqpoint{2.268761in}{1.917857in}}%
\pgfpathlineto{\pgfqpoint{2.281255in}{1.886079in}}%
\pgfpathlineto{\pgfqpoint{2.293873in}{1.922987in}}%
\pgfpathlineto{\pgfqpoint{2.306399in}{1.915408in}}%
\pgfpathlineto{\pgfqpoint{2.318891in}{1.958564in}}%
\pgfpathlineto{\pgfqpoint{2.331395in}{1.908842in}}%
\pgfpathlineto{\pgfqpoint{2.343909in}{1.917109in}}%
\pgfpathlineto{\pgfqpoint{2.356417in}{1.932907in}}%
\pgfpathlineto{\pgfqpoint{2.368880in}{1.925406in}}%
\pgfpathlineto{\pgfqpoint{2.381372in}{1.955963in}}%
\pgfpathlineto{\pgfqpoint{2.393733in}{1.944875in}}%
\pgfpathlineto{\pgfqpoint{2.406207in}{1.939552in}}%
\pgfpathlineto{\pgfqpoint{2.418633in}{1.914543in}}%
\pgfpathlineto{\pgfqpoint{2.433282in}{1.790336in}}%
\pgfpathlineto{\pgfqpoint{2.457197in}{1.469475in}}%
\pgfpathlineto{\pgfqpoint{2.477723in}{1.624256in}}%
\pgfpathlineto{\pgfqpoint{2.494737in}{1.665302in}}%
\pgfpathlineto{\pgfqpoint{2.515773in}{1.604114in}}%
\pgfpathlineto{\pgfqpoint{2.538352in}{1.471475in}}%
\pgfpathlineto{\pgfqpoint{2.558886in}{1.602098in}}%
\pgfpathlineto{\pgfqpoint{2.578521in}{1.601882in}}%
\pgfpathlineto{\pgfqpoint{2.598064in}{1.579268in}}%
\pgfpathlineto{\pgfqpoint{2.616149in}{1.661939in}}%
\pgfpathlineto{\pgfqpoint{2.636069in}{1.475557in}}%
\pgfpathlineto{\pgfqpoint{2.657274in}{1.602893in}}%
\pgfpathlineto{\pgfqpoint{2.676022in}{1.636396in}}%
\pgfpathlineto{\pgfqpoint{2.693540in}{1.610198in}}%
\pgfpathlineto{\pgfqpoint{2.710992in}{1.652418in}}%
\pgfpathlineto{\pgfqpoint{2.727722in}{1.668753in}}%
\pgfpathlineto{\pgfqpoint{2.744840in}{1.631636in}}%
\pgfpathlineto{\pgfqpoint{2.763118in}{1.644656in}}%
\pgfpathlineto{\pgfqpoint{2.780862in}{1.624096in}}%
\pgfpathlineto{\pgfqpoint{2.798917in}{1.604933in}}%
\pgfpathlineto{\pgfqpoint{2.815750in}{1.642009in}}%
\pgfpathlineto{\pgfqpoint{2.831972in}{1.672536in}}%
\pgfpathlineto{\pgfqpoint{2.848081in}{1.681824in}}%
\pgfpathlineto{\pgfqpoint{2.864455in}{1.661176in}}%
\pgfpathlineto{\pgfqpoint{2.880921in}{1.652682in}}%
\pgfpathlineto{\pgfqpoint{2.897563in}{1.668340in}}%
\pgfpathlineto{\pgfqpoint{2.914048in}{1.669681in}}%
\pgfpathlineto{\pgfqpoint{2.930775in}{1.626155in}}%
\pgfpathlineto{\pgfqpoint{2.949130in}{1.615946in}}%
\pgfpathlineto{\pgfqpoint{2.968433in}{1.588655in}}%
\pgfpathlineto{\pgfqpoint{2.985853in}{1.669384in}}%
\pgfpathlineto{\pgfqpoint{3.001921in}{1.669098in}}%
\pgfpathlineto{\pgfqpoint{3.018968in}{1.669173in}}%
\pgfpathlineto{\pgfqpoint{3.036276in}{1.662595in}}%
\pgfpathlineto{\pgfqpoint{3.052708in}{1.679290in}}%
\pgfpathlineto{\pgfqpoint{3.069003in}{1.660399in}}%
\pgfpathlineto{\pgfqpoint{3.085686in}{1.659915in}}%
\pgfpathlineto{\pgfqpoint{3.102316in}{1.661253in}}%
\pgfpathlineto{\pgfqpoint{3.118692in}{1.629434in}}%
\pgfpathlineto{\pgfqpoint{3.135319in}{1.656535in}}%
\pgfpathlineto{\pgfqpoint{3.152584in}{1.632309in}}%
\pgfpathlineto{\pgfqpoint{3.169651in}{1.649844in}}%
\pgfpathlineto{\pgfqpoint{3.185752in}{1.666680in}}%
\pgfpathlineto{\pgfqpoint{3.201602in}{1.694354in}}%
\pgfpathlineto{\pgfqpoint{3.217213in}{1.716683in}}%
\pgfpathlineto{\pgfqpoint{3.232763in}{1.711543in}}%
\pgfpathlineto{\pgfqpoint{3.248207in}{1.720364in}}%
\pgfpathlineto{\pgfqpoint{3.263722in}{1.706885in}}%
\pgfpathlineto{\pgfqpoint{3.279179in}{1.711362in}}%
\pgfpathlineto{\pgfqpoint{3.295281in}{1.659311in}}%
\pgfpathlineto{\pgfqpoint{3.313329in}{1.720965in}}%
\pgfpathlineto{\pgfqpoint{3.328928in}{1.692617in}}%
\pgfpathlineto{\pgfqpoint{3.344594in}{1.719344in}}%
\pgfpathlineto{\pgfqpoint{3.359992in}{1.709239in}}%
\pgfpathlineto{\pgfqpoint{3.375355in}{1.727960in}}%
\pgfpathlineto{\pgfqpoint{3.390722in}{1.717980in}}%
\pgfpathlineto{\pgfqpoint{3.405908in}{1.691564in}}%
\pgfpathlineto{\pgfqpoint{3.421182in}{1.730302in}}%
\pgfpathlineto{\pgfqpoint{3.436431in}{1.698643in}}%
\pgfpathlineto{\pgfqpoint{3.451777in}{1.693451in}}%
\pgfpathlineto{\pgfqpoint{3.467311in}{1.696086in}}%
\pgfpathlineto{\pgfqpoint{3.483019in}{1.712946in}}%
\pgfpathlineto{\pgfqpoint{3.498190in}{1.732610in}}%
\pgfpathlineto{\pgfqpoint{3.513323in}{1.725642in}}%
\pgfpathlineto{\pgfqpoint{3.528239in}{1.727882in}}%
\pgfpathlineto{\pgfqpoint{3.542041in}{2.390195in}}%
\pgfpathlineto{\pgfqpoint{3.554208in}{2.568903in}}%
\pgfpathlineto{\pgfqpoint{3.566335in}{2.592872in}}%
\pgfpathlineto{\pgfqpoint{3.578260in}{2.602945in}}%
\pgfpathlineto{\pgfqpoint{3.590359in}{2.545882in}}%
\pgfpathlineto{\pgfqpoint{3.602917in}{2.543862in}}%
\pgfpathlineto{\pgfqpoint{3.615064in}{2.605081in}}%
\pgfpathlineto{\pgfqpoint{3.627108in}{2.575256in}}%
\pgfpathlineto{\pgfqpoint{3.639052in}{2.598530in}}%
\pgfpathlineto{\pgfqpoint{3.651005in}{2.581523in}}%
\pgfpathlineto{\pgfqpoint{3.663361in}{2.540991in}}%
\pgfpathlineto{\pgfqpoint{3.675639in}{2.571801in}}%
\pgfpathlineto{\pgfqpoint{3.687677in}{2.585086in}}%
\pgfpathlineto{\pgfqpoint{3.699766in}{2.592531in}}%
\pgfpathlineto{\pgfqpoint{3.711790in}{2.585210in}}%
\pgfpathlineto{\pgfqpoint{3.723925in}{2.530157in}}%
\pgfpathlineto{\pgfqpoint{3.736218in}{2.577451in}}%
\pgfpathlineto{\pgfqpoint{3.748246in}{2.575758in}}%
\pgfpathlineto{\pgfqpoint{3.760252in}{2.598508in}}%
\pgfpathlineto{\pgfqpoint{3.772194in}{2.598661in}}%
\pgfpathlineto{\pgfqpoint{3.784213in}{2.593742in}}%
\pgfpathlineto{\pgfqpoint{3.796179in}{3.125184in}}%
\pgfpathlineto{\pgfqpoint{3.807700in}{3.276547in}}%
\pgfpathlineto{\pgfqpoint{3.819207in}{3.269204in}}%
\pgfpathlineto{\pgfqpoint{3.830748in}{3.270760in}}%
\pgfpathlineto{\pgfqpoint{3.842268in}{3.276823in}}%
\pgfpathlineto{\pgfqpoint{3.853773in}{3.270294in}}%
\pgfpathlineto{\pgfqpoint{3.865394in}{3.259136in}}%
\pgfpathlineto{\pgfqpoint{3.876999in}{3.257842in}}%
\pgfpathlineto{\pgfqpoint{3.888612in}{3.246935in}}%
\pgfpathlineto{\pgfqpoint{3.900120in}{3.289357in}}%
\pgfpathlineto{\pgfqpoint{3.911548in}{3.298420in}}%
\pgfpathlineto{\pgfqpoint{3.923007in}{3.279552in}}%
\pgfpathlineto{\pgfqpoint{3.934526in}{3.270407in}}%
\pgfpathlineto{\pgfqpoint{3.945995in}{3.303172in}}%
\pgfpathlineto{\pgfqpoint{3.957345in}{3.272861in}}%
\pgfpathlineto{\pgfqpoint{3.968800in}{3.304561in}}%
\pgfpathlineto{\pgfqpoint{3.980165in}{3.303509in}}%
\pgfpathlineto{\pgfqpoint{3.991607in}{3.293811in}}%
\pgfpathlineto{\pgfqpoint{4.003035in}{3.301529in}}%
\pgfpathlineto{\pgfqpoint{4.014312in}{3.321179in}}%
\pgfpathlineto{\pgfqpoint{4.025578in}{3.321021in}}%
\pgfpathlineto{\pgfqpoint{4.036918in}{3.310971in}}%
\pgfpathlineto{\pgfqpoint{4.048157in}{3.330196in}}%
\pgfpathlineto{\pgfqpoint{4.059512in}{3.288197in}}%
\pgfpathlineto{\pgfqpoint{4.070994in}{3.315452in}}%
\pgfpathlineto{\pgfqpoint{4.082264in}{3.331972in}}%
\pgfpathlineto{\pgfqpoint{4.093537in}{3.357141in}}%
\pgfpathlineto{\pgfqpoint{4.104772in}{3.347125in}}%
\pgfpathlineto{\pgfqpoint{4.116014in}{3.335023in}}%
\pgfpathlineto{\pgfqpoint{4.127322in}{3.347034in}}%
\pgfpathlineto{\pgfqpoint{4.138598in}{3.328045in}}%
\pgfpathlineto{\pgfqpoint{4.149813in}{3.346259in}}%
\pgfpathlineto{\pgfqpoint{4.160983in}{3.356308in}}%
\pgfpathlineto{\pgfqpoint{4.172101in}{3.440801in}}%
\pgfpathlineto{\pgfqpoint{4.183285in}{3.423197in}}%
\pgfpathlineto{\pgfqpoint{4.194400in}{3.395911in}}%
\pgfpathlineto{\pgfqpoint{4.205594in}{3.377115in}}%
\pgfpathlineto{\pgfqpoint{4.216619in}{3.406465in}}%
\pgfpathlineto{\pgfqpoint{4.227703in}{3.466682in}}%
\pgfpathlineto{\pgfqpoint{4.238844in}{3.416556in}}%
\pgfpathlineto{\pgfqpoint{4.249866in}{3.475498in}}%
\pgfpathlineto{\pgfqpoint{4.260400in}{4.276035in}}%
\pgfpathlineto{\pgfqpoint{4.270972in}{3.952584in}}%
\pgfpathlineto{\pgfqpoint{4.280743in}{4.467783in}}%
\pgfpathlineto{\pgfqpoint{4.290581in}{4.490608in}}%
\pgfpathlineto{\pgfqpoint{4.300486in}{4.510374in}}%
\pgfpathlineto{\pgfqpoint{4.310022in}{4.460577in}}%
\pgfpathlineto{\pgfqpoint{4.319388in}{4.531301in}}%
\pgfpathlineto{\pgfqpoint{4.328717in}{4.560985in}}%
\pgfpathlineto{\pgfqpoint{4.337933in}{4.614224in}}%
\pgfpathlineto{\pgfqpoint{4.346774in}{5.228899in}}%
\pgfpathlineto{\pgfqpoint{4.355507in}{5.612892in}}%
\pgfpathlineto{\pgfqpoint{4.364219in}{5.583440in}}%
\pgfpathlineto{\pgfqpoint{4.373001in}{5.569817in}}%
\pgfpathlineto{\pgfqpoint{4.381765in}{5.652077in}}%
\pgfpathlineto{\pgfqpoint{4.390451in}{5.644630in}}%
\pgfpathlineto{\pgfqpoint{4.399156in}{5.630819in}}%
\pgfpathlineto{\pgfqpoint{4.407811in}{5.653801in}}%
\pgfpathlineto{\pgfqpoint{4.416522in}{5.618644in}}%
\pgfpathlineto{\pgfqpoint{4.425248in}{5.593307in}}%
\pgfpathlineto{\pgfqpoint{4.433894in}{5.646519in}}%
\pgfpathlineto{\pgfqpoint{4.442591in}{5.633767in}}%
\pgfpathlineto{\pgfqpoint{4.451305in}{5.598857in}}%
\pgfpathlineto{\pgfqpoint{4.460011in}{5.648065in}}%
\pgfpathlineto{\pgfqpoint{4.468771in}{5.583055in}}%
\pgfpathlineto{\pgfqpoint{4.477466in}{5.657641in}}%
\pgfpathlineto{\pgfqpoint{4.486071in}{5.651740in}}%
\pgfpathlineto{\pgfqpoint{4.494752in}{5.627120in}}%
\pgfpathlineto{\pgfqpoint{4.503421in}{5.640411in}}%
\pgfpathlineto{\pgfqpoint{4.512084in}{5.613677in}}%
\pgfpathlineto{\pgfqpoint{4.520678in}{5.676077in}}%
\pgfpathlineto{\pgfqpoint{4.529276in}{5.617804in}}%
\pgfpathlineto{\pgfqpoint{4.537876in}{5.705940in}}%
\pgfpathlineto{\pgfqpoint{4.546439in}{5.694968in}}%
\pgfpathlineto{\pgfqpoint{4.554980in}{5.675694in}}%
\pgfpathlineto{\pgfqpoint{4.563550in}{5.668664in}}%
\pgfpathlineto{\pgfqpoint{4.572097in}{5.633323in}}%
\pgfpathlineto{\pgfqpoint{4.580727in}{5.678539in}}%
\pgfpathlineto{\pgfqpoint{4.589255in}{5.729893in}}%
\pgfpathlineto{\pgfqpoint{4.597719in}{5.726365in}}%
\pgfpathlineto{\pgfqpoint{4.606211in}{5.663382in}}%
\pgfpathlineto{\pgfqpoint{4.614783in}{5.689586in}}%
\pgfpathlineto{\pgfqpoint{4.623282in}{5.744681in}}%
\pgfpathlineto{\pgfqpoint{4.631733in}{5.716138in}}%
\pgfpathlineto{\pgfqpoint{4.640221in}{5.648943in}}%
\pgfpathlineto{\pgfqpoint{4.648847in}{5.589818in}}%
\pgfpathlineto{\pgfqpoint{4.657349in}{5.720245in}}%
\pgfpathlineto{\pgfqpoint{4.665825in}{5.647464in}}%
\pgfpathlineto{\pgfqpoint{4.674255in}{5.730938in}}%
\pgfpathlineto{\pgfqpoint{4.682643in}{5.761989in}}%
\pgfpathlineto{\pgfqpoint{4.691071in}{5.752686in}}%
\pgfpathlineto{\pgfqpoint{4.699465in}{5.767184in}}%
\pgfpathlineto{\pgfqpoint{4.707865in}{5.727526in}}%
\pgfpathlineto{\pgfqpoint{4.716247in}{5.766892in}}%
\pgfpathlineto{\pgfqpoint{4.724666in}{5.710039in}}%
\pgfpathlineto{\pgfqpoint{4.733049in}{5.755550in}}%
\pgfpathlineto{\pgfqpoint{4.741378in}{5.740711in}}%
\pgfpathlineto{\pgfqpoint{4.749750in}{5.752573in}}%
\pgfpathlineto{\pgfqpoint{4.758119in}{5.719410in}}%
\pgfpathlineto{\pgfqpoint{4.766528in}{5.731101in}}%
\pgfpathlineto{\pgfqpoint{4.774881in}{5.759734in}}%
\pgfpathlineto{\pgfqpoint{4.783272in}{5.739558in}}%
\pgfpathlineto{\pgfqpoint{4.791573in}{5.744669in}}%
\pgfpathlineto{\pgfqpoint{4.799850in}{5.772374in}}%
\pgfpathlineto{\pgfqpoint{4.808134in}{5.774982in}}%
\pgfpathlineto{\pgfqpoint{4.816444in}{5.754443in}}%
\pgfpathlineto{\pgfqpoint{4.824772in}{5.734477in}}%
\pgfpathlineto{\pgfqpoint{4.833061in}{5.772472in}}%
\pgfpathlineto{\pgfqpoint{4.841325in}{5.807933in}}%
\pgfpathlineto{\pgfqpoint{4.849575in}{5.741143in}}%
\pgfpathlineto{\pgfqpoint{4.857830in}{5.826175in}}%
\pgfpathlineto{\pgfqpoint{4.866069in}{5.771438in}}%
\pgfpathlineto{\pgfqpoint{4.874381in}{5.730820in}}%
\pgfpathlineto{\pgfqpoint{4.882658in}{5.744814in}}%
\pgfpathlineto{\pgfqpoint{4.890980in}{5.768961in}}%
\pgfpathlineto{\pgfqpoint{4.899205in}{5.808608in}}%
\pgfpathlineto{\pgfqpoint{4.907476in}{5.749439in}}%
\pgfpathlineto{\pgfqpoint{4.915736in}{5.762748in}}%
\pgfpathlineto{\pgfqpoint{4.924041in}{5.718479in}}%
\pgfpathlineto{\pgfqpoint{4.932347in}{5.781491in}}%
\pgfpathlineto{\pgfqpoint{4.940566in}{5.746612in}}%
\pgfpathlineto{\pgfqpoint{4.948774in}{5.775980in}}%
\pgfpathlineto{\pgfqpoint{4.956992in}{5.769930in}}%
\pgfpathlineto{\pgfqpoint{4.965138in}{5.735566in}}%
\pgfpathlineto{\pgfqpoint{4.973428in}{5.749887in}}%
\pgfpathlineto{\pgfqpoint{4.981642in}{5.749748in}}%
\pgfpathlineto{\pgfqpoint{4.989794in}{5.790295in}}%
\pgfpathlineto{\pgfqpoint{4.997948in}{5.791630in}}%
\pgfpathlineto{\pgfqpoint{5.006134in}{5.730243in}}%
\pgfpathlineto{\pgfqpoint{5.014317in}{5.759150in}}%
\pgfpathlineto{\pgfqpoint{5.022451in}{5.810378in}}%
\pgfpathlineto{\pgfqpoint{5.030602in}{5.779251in}}%
\pgfpathlineto{\pgfqpoint{5.038759in}{5.800005in}}%
\pgfpathlineto{\pgfqpoint{5.046876in}{5.762825in}}%
\pgfpathlineto{\pgfqpoint{5.055031in}{5.772021in}}%
\pgfpathlineto{\pgfqpoint{5.063216in}{5.751693in}}%
\pgfpathlineto{\pgfqpoint{5.071391in}{5.743381in}}%
\pgfpathlineto{\pgfqpoint{5.079506in}{5.764586in}}%
\pgfpathlineto{\pgfqpoint{5.087675in}{5.777296in}}%
\pgfpathlineto{\pgfqpoint{5.095819in}{5.810178in}}%
\pgfpathlineto{\pgfqpoint{5.103859in}{5.799078in}}%
\pgfpathlineto{\pgfqpoint{5.111982in}{5.756765in}}%
\pgfpathlineto{\pgfqpoint{5.120128in}{5.787803in}}%
\pgfpathlineto{\pgfqpoint{5.128231in}{5.778118in}}%
\pgfpathlineto{\pgfqpoint{5.136344in}{5.809823in}}%
\pgfpathlineto{\pgfqpoint{5.144370in}{5.757737in}}%
\pgfpathlineto{\pgfqpoint{5.152426in}{5.755535in}}%
\pgfpathlineto{\pgfqpoint{5.160447in}{5.791328in}}%
\pgfpathlineto{\pgfqpoint{5.168547in}{5.772623in}}%
\pgfpathlineto{\pgfqpoint{5.176647in}{5.752645in}}%
\pgfpathlineto{\pgfqpoint{5.184747in}{5.833704in}}%
\pgfpathlineto{\pgfqpoint{5.192761in}{5.800716in}}%
\pgfpathlineto{\pgfqpoint{5.200810in}{5.772249in}}%
\pgfpathlineto{\pgfqpoint{5.208810in}{5.843000in}}%
\pgfpathlineto{\pgfqpoint{5.216856in}{5.777751in}}%
\pgfpathlineto{\pgfqpoint{5.224894in}{5.800994in}}%
\pgfpathlineto{\pgfqpoint{5.233040in}{5.798717in}}%
\pgfpathlineto{\pgfqpoint{5.245361in}{5.626871in}}%
\pgfpathlineto{\pgfqpoint{5.253575in}{5.764168in}}%
\pgfpathlineto{\pgfqpoint{5.261632in}{5.770596in}}%
\pgfpathlineto{\pgfqpoint{5.269666in}{5.749072in}}%
\pgfpathlineto{\pgfqpoint{5.277683in}{5.788018in}}%
\pgfpathlineto{\pgfqpoint{5.285748in}{5.776071in}}%
\pgfpathlineto{\pgfqpoint{5.293795in}{5.771881in}}%
\pgfpathlineto{\pgfqpoint{5.301824in}{5.807983in}}%
\pgfpathlineto{\pgfqpoint{5.309888in}{5.774097in}}%
\pgfpathlineto{\pgfqpoint{5.317933in}{5.772575in}}%
\pgfpathlineto{\pgfqpoint{5.327487in}{5.921926in}}%
\pgfpathlineto{\pgfqpoint{5.338668in}{5.915085in}}%
\pgfpathlineto{\pgfqpoint{5.349820in}{5.921455in}}%
\pgfpathlineto{\pgfqpoint{5.360970in}{5.914700in}}%
\pgfpathlineto{\pgfqpoint{5.372247in}{5.893963in}}%
\pgfpathlineto{\pgfqpoint{5.383480in}{5.918731in}}%
\pgfpathlineto{\pgfqpoint{5.394653in}{5.930845in}}%
\pgfpathlineto{\pgfqpoint{5.405885in}{5.895867in}}%
\pgfpathlineto{\pgfqpoint{5.405885in}{5.895867in}}%
\pgfpathlineto{\pgfqpoint{5.405885in}{5.895867in}}%
\pgfpathlineto{\pgfqpoint{5.394653in}{5.930845in}}%
\pgfpathlineto{\pgfqpoint{5.383480in}{5.918731in}}%
\pgfpathlineto{\pgfqpoint{5.372247in}{5.893963in}}%
\pgfpathlineto{\pgfqpoint{5.360970in}{5.914700in}}%
\pgfpathlineto{\pgfqpoint{5.349820in}{5.921455in}}%
\pgfpathlineto{\pgfqpoint{5.338668in}{5.915085in}}%
\pgfpathlineto{\pgfqpoint{5.327487in}{5.921926in}}%
\pgfpathlineto{\pgfqpoint{5.317933in}{5.772575in}}%
\pgfpathlineto{\pgfqpoint{5.309888in}{5.774097in}}%
\pgfpathlineto{\pgfqpoint{5.301824in}{5.807983in}}%
\pgfpathlineto{\pgfqpoint{5.293795in}{5.771881in}}%
\pgfpathlineto{\pgfqpoint{5.285748in}{5.776071in}}%
\pgfpathlineto{\pgfqpoint{5.277683in}{5.788018in}}%
\pgfpathlineto{\pgfqpoint{5.269666in}{5.749072in}}%
\pgfpathlineto{\pgfqpoint{5.261632in}{5.770596in}}%
\pgfpathlineto{\pgfqpoint{5.253575in}{5.764168in}}%
\pgfpathlineto{\pgfqpoint{5.245361in}{5.626871in}}%
\pgfpathlineto{\pgfqpoint{5.233040in}{5.798717in}}%
\pgfpathlineto{\pgfqpoint{5.224894in}{5.800994in}}%
\pgfpathlineto{\pgfqpoint{5.216856in}{5.777751in}}%
\pgfpathlineto{\pgfqpoint{5.208810in}{5.843000in}}%
\pgfpathlineto{\pgfqpoint{5.200810in}{5.772249in}}%
\pgfpathlineto{\pgfqpoint{5.192761in}{5.800716in}}%
\pgfpathlineto{\pgfqpoint{5.184747in}{5.833704in}}%
\pgfpathlineto{\pgfqpoint{5.176647in}{5.752645in}}%
\pgfpathlineto{\pgfqpoint{5.168547in}{5.772623in}}%
\pgfpathlineto{\pgfqpoint{5.160447in}{5.791328in}}%
\pgfpathlineto{\pgfqpoint{5.152426in}{5.755535in}}%
\pgfpathlineto{\pgfqpoint{5.144370in}{5.757737in}}%
\pgfpathlineto{\pgfqpoint{5.136344in}{5.809823in}}%
\pgfpathlineto{\pgfqpoint{5.128231in}{5.778118in}}%
\pgfpathlineto{\pgfqpoint{5.120128in}{5.787803in}}%
\pgfpathlineto{\pgfqpoint{5.111982in}{5.756765in}}%
\pgfpathlineto{\pgfqpoint{5.103859in}{5.799078in}}%
\pgfpathlineto{\pgfqpoint{5.095819in}{5.810178in}}%
\pgfpathlineto{\pgfqpoint{5.087675in}{5.777296in}}%
\pgfpathlineto{\pgfqpoint{5.079506in}{5.764586in}}%
\pgfpathlineto{\pgfqpoint{5.071391in}{5.743381in}}%
\pgfpathlineto{\pgfqpoint{5.063216in}{5.751693in}}%
\pgfpathlineto{\pgfqpoint{5.055031in}{5.772021in}}%
\pgfpathlineto{\pgfqpoint{5.046876in}{5.762825in}}%
\pgfpathlineto{\pgfqpoint{5.038759in}{5.800005in}}%
\pgfpathlineto{\pgfqpoint{5.030602in}{5.779251in}}%
\pgfpathlineto{\pgfqpoint{5.022451in}{5.810378in}}%
\pgfpathlineto{\pgfqpoint{5.014317in}{5.759150in}}%
\pgfpathlineto{\pgfqpoint{5.006134in}{5.730243in}}%
\pgfpathlineto{\pgfqpoint{4.997948in}{5.791630in}}%
\pgfpathlineto{\pgfqpoint{4.989794in}{5.790295in}}%
\pgfpathlineto{\pgfqpoint{4.981642in}{5.749748in}}%
\pgfpathlineto{\pgfqpoint{4.973428in}{5.749887in}}%
\pgfpathlineto{\pgfqpoint{4.965138in}{5.735566in}}%
\pgfpathlineto{\pgfqpoint{4.956992in}{5.769930in}}%
\pgfpathlineto{\pgfqpoint{4.948774in}{5.775980in}}%
\pgfpathlineto{\pgfqpoint{4.940566in}{5.746612in}}%
\pgfpathlineto{\pgfqpoint{4.932347in}{5.781491in}}%
\pgfpathlineto{\pgfqpoint{4.924041in}{5.718479in}}%
\pgfpathlineto{\pgfqpoint{4.915736in}{5.762748in}}%
\pgfpathlineto{\pgfqpoint{4.907476in}{5.749439in}}%
\pgfpathlineto{\pgfqpoint{4.899205in}{5.808608in}}%
\pgfpathlineto{\pgfqpoint{4.890980in}{5.768961in}}%
\pgfpathlineto{\pgfqpoint{4.882658in}{5.744814in}}%
\pgfpathlineto{\pgfqpoint{4.874381in}{5.730820in}}%
\pgfpathlineto{\pgfqpoint{4.866069in}{5.771438in}}%
\pgfpathlineto{\pgfqpoint{4.857830in}{5.826175in}}%
\pgfpathlineto{\pgfqpoint{4.849575in}{5.741143in}}%
\pgfpathlineto{\pgfqpoint{4.841325in}{5.807933in}}%
\pgfpathlineto{\pgfqpoint{4.833061in}{5.772472in}}%
\pgfpathlineto{\pgfqpoint{4.824772in}{5.734477in}}%
\pgfpathlineto{\pgfqpoint{4.816444in}{5.754443in}}%
\pgfpathlineto{\pgfqpoint{4.808134in}{5.774982in}}%
\pgfpathlineto{\pgfqpoint{4.799850in}{5.772374in}}%
\pgfpathlineto{\pgfqpoint{4.791573in}{5.744669in}}%
\pgfpathlineto{\pgfqpoint{4.783272in}{5.739558in}}%
\pgfpathlineto{\pgfqpoint{4.774881in}{5.759734in}}%
\pgfpathlineto{\pgfqpoint{4.766528in}{5.731101in}}%
\pgfpathlineto{\pgfqpoint{4.758119in}{5.719410in}}%
\pgfpathlineto{\pgfqpoint{4.749750in}{5.752573in}}%
\pgfpathlineto{\pgfqpoint{4.741378in}{5.740711in}}%
\pgfpathlineto{\pgfqpoint{4.733049in}{5.755550in}}%
\pgfpathlineto{\pgfqpoint{4.724666in}{5.710039in}}%
\pgfpathlineto{\pgfqpoint{4.716247in}{5.766892in}}%
\pgfpathlineto{\pgfqpoint{4.707865in}{5.727526in}}%
\pgfpathlineto{\pgfqpoint{4.699465in}{5.767184in}}%
\pgfpathlineto{\pgfqpoint{4.691071in}{5.752686in}}%
\pgfpathlineto{\pgfqpoint{4.682643in}{5.761989in}}%
\pgfpathlineto{\pgfqpoint{4.674255in}{5.730938in}}%
\pgfpathlineto{\pgfqpoint{4.665825in}{5.647464in}}%
\pgfpathlineto{\pgfqpoint{4.657349in}{5.720245in}}%
\pgfpathlineto{\pgfqpoint{4.648847in}{5.589818in}}%
\pgfpathlineto{\pgfqpoint{4.640221in}{5.648943in}}%
\pgfpathlineto{\pgfqpoint{4.631733in}{5.716138in}}%
\pgfpathlineto{\pgfqpoint{4.623282in}{5.744681in}}%
\pgfpathlineto{\pgfqpoint{4.614783in}{5.689586in}}%
\pgfpathlineto{\pgfqpoint{4.606211in}{5.663382in}}%
\pgfpathlineto{\pgfqpoint{4.597719in}{5.726365in}}%
\pgfpathlineto{\pgfqpoint{4.589255in}{5.729893in}}%
\pgfpathlineto{\pgfqpoint{4.580727in}{5.678539in}}%
\pgfpathlineto{\pgfqpoint{4.572097in}{5.633323in}}%
\pgfpathlineto{\pgfqpoint{4.563550in}{5.668664in}}%
\pgfpathlineto{\pgfqpoint{4.554980in}{5.675694in}}%
\pgfpathlineto{\pgfqpoint{4.546439in}{5.694968in}}%
\pgfpathlineto{\pgfqpoint{4.537876in}{5.705940in}}%
\pgfpathlineto{\pgfqpoint{4.529276in}{5.617804in}}%
\pgfpathlineto{\pgfqpoint{4.520678in}{5.676077in}}%
\pgfpathlineto{\pgfqpoint{4.512084in}{5.613677in}}%
\pgfpathlineto{\pgfqpoint{4.503421in}{5.640411in}}%
\pgfpathlineto{\pgfqpoint{4.494752in}{5.627120in}}%
\pgfpathlineto{\pgfqpoint{4.486071in}{5.651740in}}%
\pgfpathlineto{\pgfqpoint{4.477466in}{5.657641in}}%
\pgfpathlineto{\pgfqpoint{4.468771in}{5.583055in}}%
\pgfpathlineto{\pgfqpoint{4.460011in}{5.648065in}}%
\pgfpathlineto{\pgfqpoint{4.451305in}{5.598857in}}%
\pgfpathlineto{\pgfqpoint{4.442591in}{5.633767in}}%
\pgfpathlineto{\pgfqpoint{4.433894in}{5.646519in}}%
\pgfpathlineto{\pgfqpoint{4.425248in}{5.593307in}}%
\pgfpathlineto{\pgfqpoint{4.416522in}{5.618644in}}%
\pgfpathlineto{\pgfqpoint{4.407811in}{5.653801in}}%
\pgfpathlineto{\pgfqpoint{4.399156in}{5.630819in}}%
\pgfpathlineto{\pgfqpoint{4.390451in}{5.644630in}}%
\pgfpathlineto{\pgfqpoint{4.381765in}{5.652077in}}%
\pgfpathlineto{\pgfqpoint{4.373001in}{5.569817in}}%
\pgfpathlineto{\pgfqpoint{4.364219in}{5.583440in}}%
\pgfpathlineto{\pgfqpoint{4.355507in}{5.612892in}}%
\pgfpathlineto{\pgfqpoint{4.346774in}{5.228899in}}%
\pgfpathlineto{\pgfqpoint{4.337933in}{4.614224in}}%
\pgfpathlineto{\pgfqpoint{4.328717in}{4.560985in}}%
\pgfpathlineto{\pgfqpoint{4.319388in}{4.531301in}}%
\pgfpathlineto{\pgfqpoint{4.310022in}{4.460577in}}%
\pgfpathlineto{\pgfqpoint{4.300486in}{4.510374in}}%
\pgfpathlineto{\pgfqpoint{4.290581in}{4.490608in}}%
\pgfpathlineto{\pgfqpoint{4.280743in}{4.467783in}}%
\pgfpathlineto{\pgfqpoint{4.270972in}{3.952584in}}%
\pgfpathlineto{\pgfqpoint{4.260400in}{4.276035in}}%
\pgfpathlineto{\pgfqpoint{4.249866in}{3.475498in}}%
\pgfpathlineto{\pgfqpoint{4.238844in}{3.416556in}}%
\pgfpathlineto{\pgfqpoint{4.227703in}{3.466682in}}%
\pgfpathlineto{\pgfqpoint{4.216619in}{3.406465in}}%
\pgfpathlineto{\pgfqpoint{4.205594in}{3.377115in}}%
\pgfpathlineto{\pgfqpoint{4.194400in}{3.395911in}}%
\pgfpathlineto{\pgfqpoint{4.183285in}{3.423197in}}%
\pgfpathlineto{\pgfqpoint{4.172101in}{3.440801in}}%
\pgfpathlineto{\pgfqpoint{4.160983in}{3.356308in}}%
\pgfpathlineto{\pgfqpoint{4.149813in}{3.346259in}}%
\pgfpathlineto{\pgfqpoint{4.138598in}{3.328045in}}%
\pgfpathlineto{\pgfqpoint{4.127322in}{3.347034in}}%
\pgfpathlineto{\pgfqpoint{4.116014in}{3.335023in}}%
\pgfpathlineto{\pgfqpoint{4.104772in}{3.347125in}}%
\pgfpathlineto{\pgfqpoint{4.093537in}{3.357141in}}%
\pgfpathlineto{\pgfqpoint{4.082264in}{3.331972in}}%
\pgfpathlineto{\pgfqpoint{4.070994in}{3.315452in}}%
\pgfpathlineto{\pgfqpoint{4.059512in}{3.288197in}}%
\pgfpathlineto{\pgfqpoint{4.048157in}{3.330196in}}%
\pgfpathlineto{\pgfqpoint{4.036918in}{3.310971in}}%
\pgfpathlineto{\pgfqpoint{4.025578in}{3.321021in}}%
\pgfpathlineto{\pgfqpoint{4.014312in}{3.321179in}}%
\pgfpathlineto{\pgfqpoint{4.003035in}{3.301529in}}%
\pgfpathlineto{\pgfqpoint{3.991607in}{3.293811in}}%
\pgfpathlineto{\pgfqpoint{3.980165in}{3.303509in}}%
\pgfpathlineto{\pgfqpoint{3.968800in}{3.304561in}}%
\pgfpathlineto{\pgfqpoint{3.957345in}{3.272861in}}%
\pgfpathlineto{\pgfqpoint{3.945995in}{3.303172in}}%
\pgfpathlineto{\pgfqpoint{3.934526in}{3.270407in}}%
\pgfpathlineto{\pgfqpoint{3.923007in}{3.279552in}}%
\pgfpathlineto{\pgfqpoint{3.911548in}{3.298420in}}%
\pgfpathlineto{\pgfqpoint{3.900120in}{3.289357in}}%
\pgfpathlineto{\pgfqpoint{3.888612in}{3.246935in}}%
\pgfpathlineto{\pgfqpoint{3.876999in}{3.257842in}}%
\pgfpathlineto{\pgfqpoint{3.865394in}{3.259136in}}%
\pgfpathlineto{\pgfqpoint{3.853773in}{3.270294in}}%
\pgfpathlineto{\pgfqpoint{3.842268in}{3.276823in}}%
\pgfpathlineto{\pgfqpoint{3.830748in}{3.270760in}}%
\pgfpathlineto{\pgfqpoint{3.819207in}{3.269204in}}%
\pgfpathlineto{\pgfqpoint{3.807700in}{3.276547in}}%
\pgfpathlineto{\pgfqpoint{3.796179in}{3.264953in}}%
\pgfpathlineto{\pgfqpoint{3.784213in}{3.154244in}}%
\pgfpathlineto{\pgfqpoint{3.772194in}{3.186262in}}%
\pgfpathlineto{\pgfqpoint{3.760252in}{3.189335in}}%
\pgfpathlineto{\pgfqpoint{3.748246in}{3.153175in}}%
\pgfpathlineto{\pgfqpoint{3.736218in}{3.171180in}}%
\pgfpathlineto{\pgfqpoint{3.723925in}{3.112946in}}%
\pgfpathlineto{\pgfqpoint{3.711790in}{3.178747in}}%
\pgfpathlineto{\pgfqpoint{3.699766in}{3.158953in}}%
\pgfpathlineto{\pgfqpoint{3.687677in}{3.164516in}}%
\pgfpathlineto{\pgfqpoint{3.675639in}{3.154565in}}%
\pgfpathlineto{\pgfqpoint{3.663361in}{3.109027in}}%
\pgfpathlineto{\pgfqpoint{3.651005in}{3.178837in}}%
\pgfpathlineto{\pgfqpoint{3.639052in}{3.184797in}}%
\pgfpathlineto{\pgfqpoint{3.627108in}{3.171976in}}%
\pgfpathlineto{\pgfqpoint{3.615064in}{3.164137in}}%
\pgfpathlineto{\pgfqpoint{3.602917in}{3.101917in}}%
\pgfpathlineto{\pgfqpoint{3.590359in}{3.132484in}}%
\pgfpathlineto{\pgfqpoint{3.578260in}{3.191036in}}%
\pgfpathlineto{\pgfqpoint{3.566335in}{3.171755in}}%
\pgfpathlineto{\pgfqpoint{3.554208in}{3.143971in}}%
\pgfpathlineto{\pgfqpoint{3.542041in}{2.980222in}}%
\pgfpathlineto{\pgfqpoint{3.528239in}{2.341823in}}%
\pgfpathlineto{\pgfqpoint{3.513323in}{2.307482in}}%
\pgfpathlineto{\pgfqpoint{3.498190in}{2.309947in}}%
\pgfpathlineto{\pgfqpoint{3.483019in}{2.298765in}}%
\pgfpathlineto{\pgfqpoint{3.467311in}{2.274944in}}%
\pgfpathlineto{\pgfqpoint{3.451777in}{2.274327in}}%
\pgfpathlineto{\pgfqpoint{3.436431in}{2.283494in}}%
\pgfpathlineto{\pgfqpoint{3.421182in}{2.304627in}}%
\pgfpathlineto{\pgfqpoint{3.405908in}{2.298830in}}%
\pgfpathlineto{\pgfqpoint{3.390722in}{2.294135in}}%
\pgfpathlineto{\pgfqpoint{3.375355in}{2.302227in}}%
\pgfpathlineto{\pgfqpoint{3.359992in}{2.292557in}}%
\pgfpathlineto{\pgfqpoint{3.344594in}{2.293764in}}%
\pgfpathlineto{\pgfqpoint{3.328928in}{2.259904in}}%
\pgfpathlineto{\pgfqpoint{3.313329in}{2.299032in}}%
\pgfpathlineto{\pgfqpoint{3.295281in}{2.230481in}}%
\pgfpathlineto{\pgfqpoint{3.279179in}{2.290491in}}%
\pgfpathlineto{\pgfqpoint{3.263722in}{2.301460in}}%
\pgfpathlineto{\pgfqpoint{3.248207in}{2.289001in}}%
\pgfpathlineto{\pgfqpoint{3.232763in}{2.285410in}}%
\pgfpathlineto{\pgfqpoint{3.217213in}{2.293744in}}%
\pgfpathlineto{\pgfqpoint{3.201602in}{2.268956in}}%
\pgfpathlineto{\pgfqpoint{3.185752in}{2.262808in}}%
\pgfpathlineto{\pgfqpoint{3.169651in}{2.246420in}}%
\pgfpathlineto{\pgfqpoint{3.152584in}{2.203378in}}%
\pgfpathlineto{\pgfqpoint{3.135319in}{2.257412in}}%
\pgfpathlineto{\pgfqpoint{3.118692in}{2.222289in}}%
\pgfpathlineto{\pgfqpoint{3.102316in}{2.245207in}}%
\pgfpathlineto{\pgfqpoint{3.085686in}{2.230972in}}%
\pgfpathlineto{\pgfqpoint{3.069003in}{2.247565in}}%
\pgfpathlineto{\pgfqpoint{3.052708in}{2.247815in}}%
\pgfpathlineto{\pgfqpoint{3.036276in}{2.236141in}}%
\pgfpathlineto{\pgfqpoint{3.018968in}{2.239617in}}%
\pgfpathlineto{\pgfqpoint{3.001921in}{2.251431in}}%
\pgfpathlineto{\pgfqpoint{2.985853in}{2.257603in}}%
\pgfpathlineto{\pgfqpoint{2.968433in}{2.190289in}}%
\pgfpathlineto{\pgfqpoint{2.949130in}{2.212297in}}%
\pgfpathlineto{\pgfqpoint{2.930775in}{2.218497in}}%
\pgfpathlineto{\pgfqpoint{2.914048in}{2.257602in}}%
\pgfpathlineto{\pgfqpoint{2.897563in}{2.257597in}}%
\pgfpathlineto{\pgfqpoint{2.880921in}{2.236897in}}%
\pgfpathlineto{\pgfqpoint{2.864455in}{2.244601in}}%
\pgfpathlineto{\pgfqpoint{2.848081in}{2.270577in}}%
\pgfpathlineto{\pgfqpoint{2.831972in}{2.255034in}}%
\pgfpathlineto{\pgfqpoint{2.815750in}{2.233667in}}%
\pgfpathlineto{\pgfqpoint{2.798917in}{2.202482in}}%
\pgfpathlineto{\pgfqpoint{2.780862in}{2.217486in}}%
\pgfpathlineto{\pgfqpoint{2.763118in}{2.221674in}}%
\pgfpathlineto{\pgfqpoint{2.744840in}{2.204178in}}%
\pgfpathlineto{\pgfqpoint{2.727722in}{2.235055in}}%
\pgfpathlineto{\pgfqpoint{2.710992in}{2.230720in}}%
\pgfpathlineto{\pgfqpoint{2.693540in}{2.192560in}}%
\pgfpathlineto{\pgfqpoint{2.676022in}{2.218484in}}%
\pgfpathlineto{\pgfqpoint{2.657274in}{2.178541in}}%
\pgfpathlineto{\pgfqpoint{2.636069in}{2.048756in}}%
\pgfpathlineto{\pgfqpoint{2.616149in}{2.240378in}}%
\pgfpathlineto{\pgfqpoint{2.598064in}{2.141244in}}%
\pgfpathlineto{\pgfqpoint{2.578521in}{2.183930in}}%
\pgfpathlineto{\pgfqpoint{2.558886in}{2.174556in}}%
\pgfpathlineto{\pgfqpoint{2.538352in}{2.047995in}}%
\pgfpathlineto{\pgfqpoint{2.515773in}{2.166202in}}%
\pgfpathlineto{\pgfqpoint{2.494737in}{2.246066in}}%
\pgfpathlineto{\pgfqpoint{2.477723in}{2.228325in}}%
\pgfpathlineto{\pgfqpoint{2.457197in}{2.046907in}}%
\pgfpathlineto{\pgfqpoint{2.433282in}{2.359339in}}%
\pgfpathlineto{\pgfqpoint{2.418633in}{2.517704in}}%
\pgfpathlineto{\pgfqpoint{2.406207in}{2.525644in}}%
\pgfpathlineto{\pgfqpoint{2.393733in}{2.510468in}}%
\pgfpathlineto{\pgfqpoint{2.381372in}{2.553574in}}%
\pgfpathlineto{\pgfqpoint{2.368880in}{2.527400in}}%
\pgfpathlineto{\pgfqpoint{2.356417in}{2.519764in}}%
\pgfpathlineto{\pgfqpoint{2.343909in}{2.498989in}}%
\pgfpathlineto{\pgfqpoint{2.331395in}{2.508717in}}%
\pgfpathlineto{\pgfqpoint{2.318891in}{2.527911in}}%
\pgfpathlineto{\pgfqpoint{2.306399in}{2.515719in}}%
\pgfpathlineto{\pgfqpoint{2.293873in}{2.523973in}}%
\pgfpathlineto{\pgfqpoint{2.281255in}{2.491159in}}%
\pgfpathlineto{\pgfqpoint{2.268761in}{2.517744in}}%
\pgfpathlineto{\pgfqpoint{2.256287in}{2.533948in}}%
\pgfpathlineto{\pgfqpoint{2.243791in}{2.516462in}}%
\pgfpathlineto{\pgfqpoint{2.231301in}{2.508450in}}%
\pgfpathlineto{\pgfqpoint{2.218726in}{2.500476in}}%
\pgfpathlineto{\pgfqpoint{2.206174in}{2.508283in}}%
\pgfpathlineto{\pgfqpoint{2.193673in}{2.506948in}}%
\pgfpathlineto{\pgfqpoint{2.181172in}{2.518373in}}%
\pgfpathlineto{\pgfqpoint{2.168659in}{2.497139in}}%
\pgfpathlineto{\pgfqpoint{2.156064in}{2.518408in}}%
\pgfpathlineto{\pgfqpoint{2.143512in}{2.521256in}}%
\pgfpathlineto{\pgfqpoint{2.130987in}{2.525620in}}%
\pgfpathlineto{\pgfqpoint{2.118500in}{2.519678in}}%
\pgfpathlineto{\pgfqpoint{2.106029in}{2.510840in}}%
\pgfpathlineto{\pgfqpoint{2.093434in}{2.510944in}}%
\pgfpathlineto{\pgfqpoint{2.080914in}{2.511169in}}%
\pgfpathlineto{\pgfqpoint{2.068360in}{2.524887in}}%
\pgfpathlineto{\pgfqpoint{2.055871in}{2.529646in}}%
\pgfpathlineto{\pgfqpoint{2.043350in}{2.514377in}}%
\pgfpathlineto{\pgfqpoint{2.030840in}{2.519333in}}%
\pgfpathlineto{\pgfqpoint{2.016889in}{2.458242in}}%
\pgfpathlineto{\pgfqpoint{2.004212in}{2.525404in}}%
\pgfpathlineto{\pgfqpoint{1.991646in}{2.519661in}}%
\pgfpathlineto{\pgfqpoint{1.979161in}{2.541870in}}%
\pgfpathlineto{\pgfqpoint{1.966664in}{2.502231in}}%
\pgfpathlineto{\pgfqpoint{1.954108in}{2.513919in}}%
\pgfpathlineto{\pgfqpoint{1.941620in}{2.533266in}}%
\pgfpathlineto{\pgfqpoint{1.929118in}{2.518382in}}%
\pgfpathlineto{\pgfqpoint{1.916611in}{2.526294in}}%
\pgfpathlineto{\pgfqpoint{1.904130in}{2.538661in}}%
\pgfpathlineto{\pgfqpoint{1.891620in}{2.528246in}}%
\pgfpathlineto{\pgfqpoint{1.879180in}{2.519769in}}%
\pgfpathlineto{\pgfqpoint{1.866747in}{2.548325in}}%
\pgfpathlineto{\pgfqpoint{1.854433in}{2.539294in}}%
\pgfpathlineto{\pgfqpoint{1.842065in}{2.521929in}}%
\pgfpathlineto{\pgfqpoint{1.829628in}{2.520543in}}%
\pgfpathlineto{\pgfqpoint{1.817148in}{2.508238in}}%
\pgfpathlineto{\pgfqpoint{1.804716in}{2.507641in}}%
\pgfpathlineto{\pgfqpoint{1.792355in}{2.557789in}}%
\pgfpathlineto{\pgfqpoint{1.779913in}{2.528944in}}%
\pgfpathlineto{\pgfqpoint{1.767480in}{2.541586in}}%
\pgfpathlineto{\pgfqpoint{1.755070in}{2.516721in}}%
\pgfpathlineto{\pgfqpoint{1.742593in}{2.539570in}}%
\pgfpathlineto{\pgfqpoint{1.730200in}{2.543342in}}%
\pgfpathlineto{\pgfqpoint{1.717832in}{2.559057in}}%
\pgfpathlineto{\pgfqpoint{1.705498in}{2.528611in}}%
\pgfpathlineto{\pgfqpoint{1.693005in}{2.534686in}}%
\pgfpathlineto{\pgfqpoint{1.680578in}{2.534534in}}%
\pgfpathlineto{\pgfqpoint{1.668187in}{2.519315in}}%
\pgfpathlineto{\pgfqpoint{1.655673in}{2.502944in}}%
\pgfpathlineto{\pgfqpoint{1.643249in}{2.520345in}}%
\pgfpathlineto{\pgfqpoint{1.630788in}{2.519549in}}%
\pgfpathlineto{\pgfqpoint{1.618336in}{2.528328in}}%
\pgfpathlineto{\pgfqpoint{1.605869in}{2.495738in}}%
\pgfpathlineto{\pgfqpoint{1.593430in}{2.534480in}}%
\pgfpathlineto{\pgfqpoint{1.581070in}{2.536392in}}%
\pgfpathlineto{\pgfqpoint{1.568672in}{2.519569in}}%
\pgfpathlineto{\pgfqpoint{1.556290in}{2.520067in}}%
\pgfpathlineto{\pgfqpoint{1.543839in}{2.530272in}}%
\pgfpathlineto{\pgfqpoint{1.531474in}{2.534239in}}%
\pgfpathlineto{\pgfqpoint{1.519060in}{2.516628in}}%
\pgfpathlineto{\pgfqpoint{1.506021in}{2.475360in}}%
\pgfpathlineto{\pgfqpoint{1.493578in}{2.516006in}}%
\pgfpathlineto{\pgfqpoint{1.481163in}{2.491356in}}%
\pgfpathlineto{\pgfqpoint{1.468660in}{2.535284in}}%
\pgfpathlineto{\pgfqpoint{1.456188in}{2.524415in}}%
\pgfpathlineto{\pgfqpoint{1.443756in}{2.533257in}}%
\pgfpathlineto{\pgfqpoint{1.431382in}{2.540364in}}%
\pgfpathlineto{\pgfqpoint{1.418893in}{2.474782in}}%
\pgfpathlineto{\pgfqpoint{1.406401in}{2.536403in}}%
\pgfpathlineto{\pgfqpoint{1.393906in}{2.547855in}}%
\pgfpathlineto{\pgfqpoint{1.381448in}{2.520947in}}%
\pgfpathlineto{\pgfqpoint{1.368992in}{2.531567in}}%
\pgfpathlineto{\pgfqpoint{1.356458in}{2.510036in}}%
\pgfpathlineto{\pgfqpoint{1.343969in}{2.531120in}}%
\pgfpathlineto{\pgfqpoint{1.331552in}{2.529602in}}%
\pgfpathlineto{\pgfqpoint{1.319073in}{2.495518in}}%
\pgfpathlineto{\pgfqpoint{1.306603in}{2.502368in}}%
\pgfpathlineto{\pgfqpoint{1.294165in}{2.544169in}}%
\pgfpathlineto{\pgfqpoint{1.281753in}{2.540137in}}%
\pgfpathlineto{\pgfqpoint{1.269255in}{2.519551in}}%
\pgfpathlineto{\pgfqpoint{1.256724in}{2.509949in}}%
\pgfpathlineto{\pgfqpoint{1.244104in}{2.519849in}}%
\pgfpathlineto{\pgfqpoint{1.231699in}{2.546485in}}%
\pgfpathlineto{\pgfqpoint{1.219183in}{2.514693in}}%
\pgfpathlineto{\pgfqpoint{1.206622in}{2.496858in}}%
\pgfpathlineto{\pgfqpoint{1.194024in}{2.490995in}}%
\pgfpathlineto{\pgfqpoint{1.181503in}{2.534963in}}%
\pgfpathlineto{\pgfqpoint{1.168924in}{2.494313in}}%
\pgfpathlineto{\pgfqpoint{1.156226in}{2.493470in}}%
\pgfpathlineto{\pgfqpoint{1.143621in}{2.519421in}}%
\pgfpathlineto{\pgfqpoint{1.131014in}{2.493182in}}%
\pgfpathlineto{\pgfqpoint{1.118303in}{2.478406in}}%
\pgfpathlineto{\pgfqpoint{1.105655in}{2.483031in}}%
\pgfpathlineto{\pgfqpoint{1.092853in}{2.467335in}}%
\pgfpathlineto{\pgfqpoint{1.079972in}{2.502499in}}%
\pgfpathlineto{\pgfqpoint{1.067143in}{2.465066in}}%
\pgfpathlineto{\pgfqpoint{1.054264in}{2.468884in}}%
\pgfpathlineto{\pgfqpoint{1.041322in}{2.448055in}}%
\pgfpathlineto{\pgfqpoint{1.028303in}{2.423999in}}%
\pgfpathlineto{\pgfqpoint{1.015150in}{2.409674in}}%
\pgfpathlineto{\pgfqpoint{1.001616in}{2.273755in}}%
\pgfpathclose%
\pgfusepath{fill}%
\end{pgfscope}%
\begin{pgfscope}%
\pgfpathrectangle{\pgfqpoint{0.781402in}{0.773588in}}{\pgfqpoint{4.844695in}{5.415119in}}%
\pgfusepath{clip}%
\pgfsetbuttcap%
\pgfsetroundjoin%
\definecolor{currentfill}{rgb}{0.549020,0.337255,0.294118}%
\pgfsetfillcolor{currentfill}%
\pgfsetlinewidth{0.000000pt}%
\definecolor{currentstroke}{rgb}{0.000000,0.000000,0.000000}%
\pgfsetstrokecolor{currentstroke}%
\pgfsetdash{}{0pt}%
\pgfpathmoveto{\pgfqpoint{1.001616in}{2.736196in}}%
\pgfpathlineto{\pgfqpoint{1.001616in}{2.273755in}}%
\pgfpathlineto{\pgfqpoint{1.015150in}{2.409674in}}%
\pgfpathlineto{\pgfqpoint{1.028303in}{2.423999in}}%
\pgfpathlineto{\pgfqpoint{1.041322in}{2.448055in}}%
\pgfpathlineto{\pgfqpoint{1.054264in}{2.468884in}}%
\pgfpathlineto{\pgfqpoint{1.067143in}{2.465066in}}%
\pgfpathlineto{\pgfqpoint{1.079972in}{2.502499in}}%
\pgfpathlineto{\pgfqpoint{1.092853in}{2.467335in}}%
\pgfpathlineto{\pgfqpoint{1.105655in}{2.483031in}}%
\pgfpathlineto{\pgfqpoint{1.118303in}{2.478406in}}%
\pgfpathlineto{\pgfqpoint{1.131014in}{2.493182in}}%
\pgfpathlineto{\pgfqpoint{1.143621in}{2.519421in}}%
\pgfpathlineto{\pgfqpoint{1.156226in}{2.493470in}}%
\pgfpathlineto{\pgfqpoint{1.168924in}{2.494313in}}%
\pgfpathlineto{\pgfqpoint{1.181503in}{2.534963in}}%
\pgfpathlineto{\pgfqpoint{1.194024in}{2.490995in}}%
\pgfpathlineto{\pgfqpoint{1.206622in}{2.496858in}}%
\pgfpathlineto{\pgfqpoint{1.219183in}{2.514693in}}%
\pgfpathlineto{\pgfqpoint{1.231699in}{2.546485in}}%
\pgfpathlineto{\pgfqpoint{1.244104in}{2.519849in}}%
\pgfpathlineto{\pgfqpoint{1.256724in}{2.509949in}}%
\pgfpathlineto{\pgfqpoint{1.269255in}{2.519551in}}%
\pgfpathlineto{\pgfqpoint{1.281753in}{2.540137in}}%
\pgfpathlineto{\pgfqpoint{1.294165in}{2.544169in}}%
\pgfpathlineto{\pgfqpoint{1.306603in}{2.502368in}}%
\pgfpathlineto{\pgfqpoint{1.319073in}{2.495518in}}%
\pgfpathlineto{\pgfqpoint{1.331552in}{2.529602in}}%
\pgfpathlineto{\pgfqpoint{1.343969in}{2.531120in}}%
\pgfpathlineto{\pgfqpoint{1.356458in}{2.510036in}}%
\pgfpathlineto{\pgfqpoint{1.368992in}{2.531567in}}%
\pgfpathlineto{\pgfqpoint{1.381448in}{2.520947in}}%
\pgfpathlineto{\pgfqpoint{1.393906in}{2.547855in}}%
\pgfpathlineto{\pgfqpoint{1.406401in}{2.536403in}}%
\pgfpathlineto{\pgfqpoint{1.418893in}{2.474782in}}%
\pgfpathlineto{\pgfqpoint{1.431382in}{2.540364in}}%
\pgfpathlineto{\pgfqpoint{1.443756in}{2.533257in}}%
\pgfpathlineto{\pgfqpoint{1.456188in}{2.524415in}}%
\pgfpathlineto{\pgfqpoint{1.468660in}{2.535284in}}%
\pgfpathlineto{\pgfqpoint{1.481163in}{2.491356in}}%
\pgfpathlineto{\pgfqpoint{1.493578in}{2.516006in}}%
\pgfpathlineto{\pgfqpoint{1.506021in}{2.475360in}}%
\pgfpathlineto{\pgfqpoint{1.519060in}{2.516628in}}%
\pgfpathlineto{\pgfqpoint{1.531474in}{2.534239in}}%
\pgfpathlineto{\pgfqpoint{1.543839in}{2.530272in}}%
\pgfpathlineto{\pgfqpoint{1.556290in}{2.520067in}}%
\pgfpathlineto{\pgfqpoint{1.568672in}{2.519569in}}%
\pgfpathlineto{\pgfqpoint{1.581070in}{2.536392in}}%
\pgfpathlineto{\pgfqpoint{1.593430in}{2.534480in}}%
\pgfpathlineto{\pgfqpoint{1.605869in}{2.495738in}}%
\pgfpathlineto{\pgfqpoint{1.618336in}{2.528328in}}%
\pgfpathlineto{\pgfqpoint{1.630788in}{2.519549in}}%
\pgfpathlineto{\pgfqpoint{1.643249in}{2.520345in}}%
\pgfpathlineto{\pgfqpoint{1.655673in}{2.502944in}}%
\pgfpathlineto{\pgfqpoint{1.668187in}{2.519315in}}%
\pgfpathlineto{\pgfqpoint{1.680578in}{2.534534in}}%
\pgfpathlineto{\pgfqpoint{1.693005in}{2.534686in}}%
\pgfpathlineto{\pgfqpoint{1.705498in}{2.528611in}}%
\pgfpathlineto{\pgfqpoint{1.717832in}{2.559057in}}%
\pgfpathlineto{\pgfqpoint{1.730200in}{2.543342in}}%
\pgfpathlineto{\pgfqpoint{1.742593in}{2.539570in}}%
\pgfpathlineto{\pgfqpoint{1.755070in}{2.516721in}}%
\pgfpathlineto{\pgfqpoint{1.767480in}{2.541586in}}%
\pgfpathlineto{\pgfqpoint{1.779913in}{2.528944in}}%
\pgfpathlineto{\pgfqpoint{1.792355in}{2.557789in}}%
\pgfpathlineto{\pgfqpoint{1.804716in}{2.507641in}}%
\pgfpathlineto{\pgfqpoint{1.817148in}{2.508238in}}%
\pgfpathlineto{\pgfqpoint{1.829628in}{2.520543in}}%
\pgfpathlineto{\pgfqpoint{1.842065in}{2.521929in}}%
\pgfpathlineto{\pgfqpoint{1.854433in}{2.539294in}}%
\pgfpathlineto{\pgfqpoint{1.866747in}{2.548325in}}%
\pgfpathlineto{\pgfqpoint{1.879180in}{2.519769in}}%
\pgfpathlineto{\pgfqpoint{1.891620in}{2.528246in}}%
\pgfpathlineto{\pgfqpoint{1.904130in}{2.538661in}}%
\pgfpathlineto{\pgfqpoint{1.916611in}{2.526294in}}%
\pgfpathlineto{\pgfqpoint{1.929118in}{2.518382in}}%
\pgfpathlineto{\pgfqpoint{1.941620in}{2.533266in}}%
\pgfpathlineto{\pgfqpoint{1.954108in}{2.513919in}}%
\pgfpathlineto{\pgfqpoint{1.966664in}{2.502231in}}%
\pgfpathlineto{\pgfqpoint{1.979161in}{2.541870in}}%
\pgfpathlineto{\pgfqpoint{1.991646in}{2.519661in}}%
\pgfpathlineto{\pgfqpoint{2.004212in}{2.525404in}}%
\pgfpathlineto{\pgfqpoint{2.016889in}{2.458242in}}%
\pgfpathlineto{\pgfqpoint{2.030840in}{2.519333in}}%
\pgfpathlineto{\pgfqpoint{2.043350in}{2.514377in}}%
\pgfpathlineto{\pgfqpoint{2.055871in}{2.529646in}}%
\pgfpathlineto{\pgfqpoint{2.068360in}{2.524887in}}%
\pgfpathlineto{\pgfqpoint{2.080914in}{2.511169in}}%
\pgfpathlineto{\pgfqpoint{2.093434in}{2.510944in}}%
\pgfpathlineto{\pgfqpoint{2.106029in}{2.510840in}}%
\pgfpathlineto{\pgfqpoint{2.118500in}{2.519678in}}%
\pgfpathlineto{\pgfqpoint{2.130987in}{2.525620in}}%
\pgfpathlineto{\pgfqpoint{2.143512in}{2.521256in}}%
\pgfpathlineto{\pgfqpoint{2.156064in}{2.518408in}}%
\pgfpathlineto{\pgfqpoint{2.168659in}{2.497139in}}%
\pgfpathlineto{\pgfqpoint{2.181172in}{2.518373in}}%
\pgfpathlineto{\pgfqpoint{2.193673in}{2.506948in}}%
\pgfpathlineto{\pgfqpoint{2.206174in}{2.508283in}}%
\pgfpathlineto{\pgfqpoint{2.218726in}{2.500476in}}%
\pgfpathlineto{\pgfqpoint{2.231301in}{2.508450in}}%
\pgfpathlineto{\pgfqpoint{2.243791in}{2.516462in}}%
\pgfpathlineto{\pgfqpoint{2.256287in}{2.533948in}}%
\pgfpathlineto{\pgfqpoint{2.268761in}{2.517744in}}%
\pgfpathlineto{\pgfqpoint{2.281255in}{2.491159in}}%
\pgfpathlineto{\pgfqpoint{2.293873in}{2.523973in}}%
\pgfpathlineto{\pgfqpoint{2.306399in}{2.515719in}}%
\pgfpathlineto{\pgfqpoint{2.318891in}{2.527911in}}%
\pgfpathlineto{\pgfqpoint{2.331395in}{2.508717in}}%
\pgfpathlineto{\pgfqpoint{2.343909in}{2.498989in}}%
\pgfpathlineto{\pgfqpoint{2.356417in}{2.519764in}}%
\pgfpathlineto{\pgfqpoint{2.368880in}{2.527400in}}%
\pgfpathlineto{\pgfqpoint{2.381372in}{2.553574in}}%
\pgfpathlineto{\pgfqpoint{2.393733in}{2.510468in}}%
\pgfpathlineto{\pgfqpoint{2.406207in}{2.525644in}}%
\pgfpathlineto{\pgfqpoint{2.418633in}{2.517704in}}%
\pgfpathlineto{\pgfqpoint{2.433282in}{2.359339in}}%
\pgfpathlineto{\pgfqpoint{2.457197in}{2.046907in}}%
\pgfpathlineto{\pgfqpoint{2.477723in}{2.228325in}}%
\pgfpathlineto{\pgfqpoint{2.494737in}{2.246066in}}%
\pgfpathlineto{\pgfqpoint{2.515773in}{2.166202in}}%
\pgfpathlineto{\pgfqpoint{2.538352in}{2.047995in}}%
\pgfpathlineto{\pgfqpoint{2.558886in}{2.174556in}}%
\pgfpathlineto{\pgfqpoint{2.578521in}{2.183930in}}%
\pgfpathlineto{\pgfqpoint{2.598064in}{2.141244in}}%
\pgfpathlineto{\pgfqpoint{2.616149in}{2.240378in}}%
\pgfpathlineto{\pgfqpoint{2.636069in}{2.048756in}}%
\pgfpathlineto{\pgfqpoint{2.657274in}{2.178541in}}%
\pgfpathlineto{\pgfqpoint{2.676022in}{2.218484in}}%
\pgfpathlineto{\pgfqpoint{2.693540in}{2.192560in}}%
\pgfpathlineto{\pgfqpoint{2.710992in}{2.230720in}}%
\pgfpathlineto{\pgfqpoint{2.727722in}{2.235055in}}%
\pgfpathlineto{\pgfqpoint{2.744840in}{2.204178in}}%
\pgfpathlineto{\pgfqpoint{2.763118in}{2.221674in}}%
\pgfpathlineto{\pgfqpoint{2.780862in}{2.217486in}}%
\pgfpathlineto{\pgfqpoint{2.798917in}{2.202482in}}%
\pgfpathlineto{\pgfqpoint{2.815750in}{2.233667in}}%
\pgfpathlineto{\pgfqpoint{2.831972in}{2.255034in}}%
\pgfpathlineto{\pgfqpoint{2.848081in}{2.270577in}}%
\pgfpathlineto{\pgfqpoint{2.864455in}{2.244601in}}%
\pgfpathlineto{\pgfqpoint{2.880921in}{2.236897in}}%
\pgfpathlineto{\pgfqpoint{2.897563in}{2.257597in}}%
\pgfpathlineto{\pgfqpoint{2.914048in}{2.257602in}}%
\pgfpathlineto{\pgfqpoint{2.930775in}{2.218497in}}%
\pgfpathlineto{\pgfqpoint{2.949130in}{2.212297in}}%
\pgfpathlineto{\pgfqpoint{2.968433in}{2.190289in}}%
\pgfpathlineto{\pgfqpoint{2.985853in}{2.257603in}}%
\pgfpathlineto{\pgfqpoint{3.001921in}{2.251431in}}%
\pgfpathlineto{\pgfqpoint{3.018968in}{2.239617in}}%
\pgfpathlineto{\pgfqpoint{3.036276in}{2.236141in}}%
\pgfpathlineto{\pgfqpoint{3.052708in}{2.247815in}}%
\pgfpathlineto{\pgfqpoint{3.069003in}{2.247565in}}%
\pgfpathlineto{\pgfqpoint{3.085686in}{2.230972in}}%
\pgfpathlineto{\pgfqpoint{3.102316in}{2.245207in}}%
\pgfpathlineto{\pgfqpoint{3.118692in}{2.222289in}}%
\pgfpathlineto{\pgfqpoint{3.135319in}{2.257412in}}%
\pgfpathlineto{\pgfqpoint{3.152584in}{2.203378in}}%
\pgfpathlineto{\pgfqpoint{3.169651in}{2.246420in}}%
\pgfpathlineto{\pgfqpoint{3.185752in}{2.262808in}}%
\pgfpathlineto{\pgfqpoint{3.201602in}{2.268956in}}%
\pgfpathlineto{\pgfqpoint{3.217213in}{2.293744in}}%
\pgfpathlineto{\pgfqpoint{3.232763in}{2.285410in}}%
\pgfpathlineto{\pgfqpoint{3.248207in}{2.289001in}}%
\pgfpathlineto{\pgfqpoint{3.263722in}{2.301460in}}%
\pgfpathlineto{\pgfqpoint{3.279179in}{2.290491in}}%
\pgfpathlineto{\pgfqpoint{3.295281in}{2.230481in}}%
\pgfpathlineto{\pgfqpoint{3.313329in}{2.299032in}}%
\pgfpathlineto{\pgfqpoint{3.328928in}{2.259904in}}%
\pgfpathlineto{\pgfqpoint{3.344594in}{2.293764in}}%
\pgfpathlineto{\pgfqpoint{3.359992in}{2.292557in}}%
\pgfpathlineto{\pgfqpoint{3.375355in}{2.302227in}}%
\pgfpathlineto{\pgfqpoint{3.390722in}{2.294135in}}%
\pgfpathlineto{\pgfqpoint{3.405908in}{2.298830in}}%
\pgfpathlineto{\pgfqpoint{3.421182in}{2.304627in}}%
\pgfpathlineto{\pgfqpoint{3.436431in}{2.283494in}}%
\pgfpathlineto{\pgfqpoint{3.451777in}{2.274327in}}%
\pgfpathlineto{\pgfqpoint{3.467311in}{2.274944in}}%
\pgfpathlineto{\pgfqpoint{3.483019in}{2.298765in}}%
\pgfpathlineto{\pgfqpoint{3.498190in}{2.309947in}}%
\pgfpathlineto{\pgfqpoint{3.513323in}{2.307482in}}%
\pgfpathlineto{\pgfqpoint{3.528239in}{2.341823in}}%
\pgfpathlineto{\pgfqpoint{3.542041in}{2.980222in}}%
\pgfpathlineto{\pgfqpoint{3.554208in}{3.143971in}}%
\pgfpathlineto{\pgfqpoint{3.566335in}{3.171755in}}%
\pgfpathlineto{\pgfqpoint{3.578260in}{3.191036in}}%
\pgfpathlineto{\pgfqpoint{3.590359in}{3.132484in}}%
\pgfpathlineto{\pgfqpoint{3.602917in}{3.101917in}}%
\pgfpathlineto{\pgfqpoint{3.615064in}{3.164137in}}%
\pgfpathlineto{\pgfqpoint{3.627108in}{3.171976in}}%
\pgfpathlineto{\pgfqpoint{3.639052in}{3.184797in}}%
\pgfpathlineto{\pgfqpoint{3.651005in}{3.178837in}}%
\pgfpathlineto{\pgfqpoint{3.663361in}{3.109027in}}%
\pgfpathlineto{\pgfqpoint{3.675639in}{3.154565in}}%
\pgfpathlineto{\pgfqpoint{3.687677in}{3.164516in}}%
\pgfpathlineto{\pgfqpoint{3.699766in}{3.158953in}}%
\pgfpathlineto{\pgfqpoint{3.711790in}{3.178747in}}%
\pgfpathlineto{\pgfqpoint{3.723925in}{3.112946in}}%
\pgfpathlineto{\pgfqpoint{3.736218in}{3.171180in}}%
\pgfpathlineto{\pgfqpoint{3.748246in}{3.153175in}}%
\pgfpathlineto{\pgfqpoint{3.760252in}{3.189335in}}%
\pgfpathlineto{\pgfqpoint{3.772194in}{3.186262in}}%
\pgfpathlineto{\pgfqpoint{3.784213in}{3.154244in}}%
\pgfpathlineto{\pgfqpoint{3.796179in}{3.264953in}}%
\pgfpathlineto{\pgfqpoint{3.807700in}{3.276547in}}%
\pgfpathlineto{\pgfqpoint{3.819207in}{3.269204in}}%
\pgfpathlineto{\pgfqpoint{3.830748in}{3.270760in}}%
\pgfpathlineto{\pgfqpoint{3.842268in}{3.276823in}}%
\pgfpathlineto{\pgfqpoint{3.853773in}{3.270294in}}%
\pgfpathlineto{\pgfqpoint{3.865394in}{3.259136in}}%
\pgfpathlineto{\pgfqpoint{3.876999in}{3.257842in}}%
\pgfpathlineto{\pgfqpoint{3.888612in}{3.246935in}}%
\pgfpathlineto{\pgfqpoint{3.900120in}{3.289357in}}%
\pgfpathlineto{\pgfqpoint{3.911548in}{3.298420in}}%
\pgfpathlineto{\pgfqpoint{3.923007in}{3.279552in}}%
\pgfpathlineto{\pgfqpoint{3.934526in}{3.270407in}}%
\pgfpathlineto{\pgfqpoint{3.945995in}{3.303172in}}%
\pgfpathlineto{\pgfqpoint{3.957345in}{3.272861in}}%
\pgfpathlineto{\pgfqpoint{3.968800in}{3.304561in}}%
\pgfpathlineto{\pgfqpoint{3.980165in}{3.303509in}}%
\pgfpathlineto{\pgfqpoint{3.991607in}{3.293811in}}%
\pgfpathlineto{\pgfqpoint{4.003035in}{3.301529in}}%
\pgfpathlineto{\pgfqpoint{4.014312in}{3.321179in}}%
\pgfpathlineto{\pgfqpoint{4.025578in}{3.321021in}}%
\pgfpathlineto{\pgfqpoint{4.036918in}{3.310971in}}%
\pgfpathlineto{\pgfqpoint{4.048157in}{3.330196in}}%
\pgfpathlineto{\pgfqpoint{4.059512in}{3.288197in}}%
\pgfpathlineto{\pgfqpoint{4.070994in}{3.315452in}}%
\pgfpathlineto{\pgfqpoint{4.082264in}{3.331972in}}%
\pgfpathlineto{\pgfqpoint{4.093537in}{3.357141in}}%
\pgfpathlineto{\pgfqpoint{4.104772in}{3.347125in}}%
\pgfpathlineto{\pgfqpoint{4.116014in}{3.335023in}}%
\pgfpathlineto{\pgfqpoint{4.127322in}{3.347034in}}%
\pgfpathlineto{\pgfqpoint{4.138598in}{3.328045in}}%
\pgfpathlineto{\pgfqpoint{4.149813in}{3.346259in}}%
\pgfpathlineto{\pgfqpoint{4.160983in}{3.356308in}}%
\pgfpathlineto{\pgfqpoint{4.172101in}{3.440801in}}%
\pgfpathlineto{\pgfqpoint{4.183285in}{3.423197in}}%
\pgfpathlineto{\pgfqpoint{4.194400in}{3.395911in}}%
\pgfpathlineto{\pgfqpoint{4.205594in}{3.377115in}}%
\pgfpathlineto{\pgfqpoint{4.216619in}{3.406465in}}%
\pgfpathlineto{\pgfqpoint{4.227703in}{3.466682in}}%
\pgfpathlineto{\pgfqpoint{4.238844in}{3.416556in}}%
\pgfpathlineto{\pgfqpoint{4.249866in}{3.475498in}}%
\pgfpathlineto{\pgfqpoint{4.260400in}{4.276035in}}%
\pgfpathlineto{\pgfqpoint{4.270972in}{3.952584in}}%
\pgfpathlineto{\pgfqpoint{4.280743in}{4.467783in}}%
\pgfpathlineto{\pgfqpoint{4.290581in}{4.490608in}}%
\pgfpathlineto{\pgfqpoint{4.300486in}{4.510374in}}%
\pgfpathlineto{\pgfqpoint{4.310022in}{4.460577in}}%
\pgfpathlineto{\pgfqpoint{4.319388in}{4.531301in}}%
\pgfpathlineto{\pgfqpoint{4.328717in}{4.560985in}}%
\pgfpathlineto{\pgfqpoint{4.337933in}{4.614224in}}%
\pgfpathlineto{\pgfqpoint{4.346774in}{5.228899in}}%
\pgfpathlineto{\pgfqpoint{4.355507in}{5.612892in}}%
\pgfpathlineto{\pgfqpoint{4.364219in}{5.583440in}}%
\pgfpathlineto{\pgfqpoint{4.373001in}{5.569817in}}%
\pgfpathlineto{\pgfqpoint{4.381765in}{5.652077in}}%
\pgfpathlineto{\pgfqpoint{4.390451in}{5.644630in}}%
\pgfpathlineto{\pgfqpoint{4.399156in}{5.630819in}}%
\pgfpathlineto{\pgfqpoint{4.407811in}{5.653801in}}%
\pgfpathlineto{\pgfqpoint{4.416522in}{5.618644in}}%
\pgfpathlineto{\pgfqpoint{4.425248in}{5.593307in}}%
\pgfpathlineto{\pgfqpoint{4.433894in}{5.646519in}}%
\pgfpathlineto{\pgfqpoint{4.442591in}{5.633767in}}%
\pgfpathlineto{\pgfqpoint{4.451305in}{5.598857in}}%
\pgfpathlineto{\pgfqpoint{4.460011in}{5.648065in}}%
\pgfpathlineto{\pgfqpoint{4.468771in}{5.583055in}}%
\pgfpathlineto{\pgfqpoint{4.477466in}{5.657641in}}%
\pgfpathlineto{\pgfqpoint{4.486071in}{5.651740in}}%
\pgfpathlineto{\pgfqpoint{4.494752in}{5.627120in}}%
\pgfpathlineto{\pgfqpoint{4.503421in}{5.640411in}}%
\pgfpathlineto{\pgfqpoint{4.512084in}{5.613677in}}%
\pgfpathlineto{\pgfqpoint{4.520678in}{5.676077in}}%
\pgfpathlineto{\pgfqpoint{4.529276in}{5.617804in}}%
\pgfpathlineto{\pgfqpoint{4.537876in}{5.705940in}}%
\pgfpathlineto{\pgfqpoint{4.546439in}{5.694968in}}%
\pgfpathlineto{\pgfqpoint{4.554980in}{5.675694in}}%
\pgfpathlineto{\pgfqpoint{4.563550in}{5.668664in}}%
\pgfpathlineto{\pgfqpoint{4.572097in}{5.633323in}}%
\pgfpathlineto{\pgfqpoint{4.580727in}{5.678539in}}%
\pgfpathlineto{\pgfqpoint{4.589255in}{5.729893in}}%
\pgfpathlineto{\pgfqpoint{4.597719in}{5.726365in}}%
\pgfpathlineto{\pgfqpoint{4.606211in}{5.663382in}}%
\pgfpathlineto{\pgfqpoint{4.614783in}{5.689586in}}%
\pgfpathlineto{\pgfqpoint{4.623282in}{5.744681in}}%
\pgfpathlineto{\pgfqpoint{4.631733in}{5.716138in}}%
\pgfpathlineto{\pgfqpoint{4.640221in}{5.648943in}}%
\pgfpathlineto{\pgfqpoint{4.648847in}{5.589818in}}%
\pgfpathlineto{\pgfqpoint{4.657349in}{5.720245in}}%
\pgfpathlineto{\pgfqpoint{4.665825in}{5.647464in}}%
\pgfpathlineto{\pgfqpoint{4.674255in}{5.730938in}}%
\pgfpathlineto{\pgfqpoint{4.682643in}{5.761989in}}%
\pgfpathlineto{\pgfqpoint{4.691071in}{5.752686in}}%
\pgfpathlineto{\pgfqpoint{4.699465in}{5.767184in}}%
\pgfpathlineto{\pgfqpoint{4.707865in}{5.727526in}}%
\pgfpathlineto{\pgfqpoint{4.716247in}{5.766892in}}%
\pgfpathlineto{\pgfqpoint{4.724666in}{5.710039in}}%
\pgfpathlineto{\pgfqpoint{4.733049in}{5.755550in}}%
\pgfpathlineto{\pgfqpoint{4.741378in}{5.740711in}}%
\pgfpathlineto{\pgfqpoint{4.749750in}{5.752573in}}%
\pgfpathlineto{\pgfqpoint{4.758119in}{5.719410in}}%
\pgfpathlineto{\pgfqpoint{4.766528in}{5.731101in}}%
\pgfpathlineto{\pgfqpoint{4.774881in}{5.759734in}}%
\pgfpathlineto{\pgfqpoint{4.783272in}{5.739558in}}%
\pgfpathlineto{\pgfqpoint{4.791573in}{5.744669in}}%
\pgfpathlineto{\pgfqpoint{4.799850in}{5.772374in}}%
\pgfpathlineto{\pgfqpoint{4.808134in}{5.774982in}}%
\pgfpathlineto{\pgfqpoint{4.816444in}{5.754443in}}%
\pgfpathlineto{\pgfqpoint{4.824772in}{5.734477in}}%
\pgfpathlineto{\pgfqpoint{4.833061in}{5.772472in}}%
\pgfpathlineto{\pgfqpoint{4.841325in}{5.807933in}}%
\pgfpathlineto{\pgfqpoint{4.849575in}{5.741143in}}%
\pgfpathlineto{\pgfqpoint{4.857830in}{5.826175in}}%
\pgfpathlineto{\pgfqpoint{4.866069in}{5.771438in}}%
\pgfpathlineto{\pgfqpoint{4.874381in}{5.730820in}}%
\pgfpathlineto{\pgfqpoint{4.882658in}{5.744814in}}%
\pgfpathlineto{\pgfqpoint{4.890980in}{5.768961in}}%
\pgfpathlineto{\pgfqpoint{4.899205in}{5.808608in}}%
\pgfpathlineto{\pgfqpoint{4.907476in}{5.749439in}}%
\pgfpathlineto{\pgfqpoint{4.915736in}{5.762748in}}%
\pgfpathlineto{\pgfqpoint{4.924041in}{5.718479in}}%
\pgfpathlineto{\pgfqpoint{4.932347in}{5.781491in}}%
\pgfpathlineto{\pgfqpoint{4.940566in}{5.746612in}}%
\pgfpathlineto{\pgfqpoint{4.948774in}{5.775980in}}%
\pgfpathlineto{\pgfqpoint{4.956992in}{5.769930in}}%
\pgfpathlineto{\pgfqpoint{4.965138in}{5.735566in}}%
\pgfpathlineto{\pgfqpoint{4.973428in}{5.749887in}}%
\pgfpathlineto{\pgfqpoint{4.981642in}{5.749748in}}%
\pgfpathlineto{\pgfqpoint{4.989794in}{5.790295in}}%
\pgfpathlineto{\pgfqpoint{4.997948in}{5.791630in}}%
\pgfpathlineto{\pgfqpoint{5.006134in}{5.730243in}}%
\pgfpathlineto{\pgfqpoint{5.014317in}{5.759150in}}%
\pgfpathlineto{\pgfqpoint{5.022451in}{5.810378in}}%
\pgfpathlineto{\pgfqpoint{5.030602in}{5.779251in}}%
\pgfpathlineto{\pgfqpoint{5.038759in}{5.800005in}}%
\pgfpathlineto{\pgfqpoint{5.046876in}{5.762825in}}%
\pgfpathlineto{\pgfqpoint{5.055031in}{5.772021in}}%
\pgfpathlineto{\pgfqpoint{5.063216in}{5.751693in}}%
\pgfpathlineto{\pgfqpoint{5.071391in}{5.743381in}}%
\pgfpathlineto{\pgfqpoint{5.079506in}{5.764586in}}%
\pgfpathlineto{\pgfqpoint{5.087675in}{5.777296in}}%
\pgfpathlineto{\pgfqpoint{5.095819in}{5.810178in}}%
\pgfpathlineto{\pgfqpoint{5.103859in}{5.799078in}}%
\pgfpathlineto{\pgfqpoint{5.111982in}{5.756765in}}%
\pgfpathlineto{\pgfqpoint{5.120128in}{5.787803in}}%
\pgfpathlineto{\pgfqpoint{5.128231in}{5.778118in}}%
\pgfpathlineto{\pgfqpoint{5.136344in}{5.809823in}}%
\pgfpathlineto{\pgfqpoint{5.144370in}{5.757737in}}%
\pgfpathlineto{\pgfqpoint{5.152426in}{5.755535in}}%
\pgfpathlineto{\pgfqpoint{5.160447in}{5.791328in}}%
\pgfpathlineto{\pgfqpoint{5.168547in}{5.772623in}}%
\pgfpathlineto{\pgfqpoint{5.176647in}{5.752645in}}%
\pgfpathlineto{\pgfqpoint{5.184747in}{5.833704in}}%
\pgfpathlineto{\pgfqpoint{5.192761in}{5.800716in}}%
\pgfpathlineto{\pgfqpoint{5.200810in}{5.772249in}}%
\pgfpathlineto{\pgfqpoint{5.208810in}{5.843000in}}%
\pgfpathlineto{\pgfqpoint{5.216856in}{5.777751in}}%
\pgfpathlineto{\pgfqpoint{5.224894in}{5.800994in}}%
\pgfpathlineto{\pgfqpoint{5.233040in}{5.798717in}}%
\pgfpathlineto{\pgfqpoint{5.245361in}{5.626871in}}%
\pgfpathlineto{\pgfqpoint{5.253575in}{5.764168in}}%
\pgfpathlineto{\pgfqpoint{5.261632in}{5.770596in}}%
\pgfpathlineto{\pgfqpoint{5.269666in}{5.749072in}}%
\pgfpathlineto{\pgfqpoint{5.277683in}{5.788018in}}%
\pgfpathlineto{\pgfqpoint{5.285748in}{5.776071in}}%
\pgfpathlineto{\pgfqpoint{5.293795in}{5.771881in}}%
\pgfpathlineto{\pgfqpoint{5.301824in}{5.807983in}}%
\pgfpathlineto{\pgfqpoint{5.309888in}{5.774097in}}%
\pgfpathlineto{\pgfqpoint{5.317933in}{5.772575in}}%
\pgfpathlineto{\pgfqpoint{5.327487in}{5.921926in}}%
\pgfpathlineto{\pgfqpoint{5.338668in}{5.915085in}}%
\pgfpathlineto{\pgfqpoint{5.349820in}{5.921455in}}%
\pgfpathlineto{\pgfqpoint{5.360970in}{5.914700in}}%
\pgfpathlineto{\pgfqpoint{5.372247in}{5.893963in}}%
\pgfpathlineto{\pgfqpoint{5.383480in}{5.918731in}}%
\pgfpathlineto{\pgfqpoint{5.394653in}{5.930845in}}%
\pgfpathlineto{\pgfqpoint{5.405885in}{5.895867in}}%
\pgfpathlineto{\pgfqpoint{5.405885in}{5.895867in}}%
\pgfpathlineto{\pgfqpoint{5.405885in}{5.895867in}}%
\pgfpathlineto{\pgfqpoint{5.394653in}{5.930845in}}%
\pgfpathlineto{\pgfqpoint{5.383480in}{5.918731in}}%
\pgfpathlineto{\pgfqpoint{5.372247in}{5.893963in}}%
\pgfpathlineto{\pgfqpoint{5.360970in}{5.914700in}}%
\pgfpathlineto{\pgfqpoint{5.349820in}{5.921455in}}%
\pgfpathlineto{\pgfqpoint{5.338668in}{5.915085in}}%
\pgfpathlineto{\pgfqpoint{5.327487in}{5.921926in}}%
\pgfpathlineto{\pgfqpoint{5.317933in}{5.772575in}}%
\pgfpathlineto{\pgfqpoint{5.309888in}{5.774097in}}%
\pgfpathlineto{\pgfqpoint{5.301824in}{5.807983in}}%
\pgfpathlineto{\pgfqpoint{5.293795in}{5.771881in}}%
\pgfpathlineto{\pgfqpoint{5.285748in}{5.776071in}}%
\pgfpathlineto{\pgfqpoint{5.277683in}{5.788018in}}%
\pgfpathlineto{\pgfqpoint{5.269666in}{5.749072in}}%
\pgfpathlineto{\pgfqpoint{5.261632in}{5.770596in}}%
\pgfpathlineto{\pgfqpoint{5.253575in}{5.764168in}}%
\pgfpathlineto{\pgfqpoint{5.245361in}{5.626871in}}%
\pgfpathlineto{\pgfqpoint{5.233040in}{5.798717in}}%
\pgfpathlineto{\pgfqpoint{5.224894in}{5.800994in}}%
\pgfpathlineto{\pgfqpoint{5.216856in}{5.777751in}}%
\pgfpathlineto{\pgfqpoint{5.208810in}{5.843000in}}%
\pgfpathlineto{\pgfqpoint{5.200810in}{5.772249in}}%
\pgfpathlineto{\pgfqpoint{5.192761in}{5.800716in}}%
\pgfpathlineto{\pgfqpoint{5.184747in}{5.833704in}}%
\pgfpathlineto{\pgfqpoint{5.176647in}{5.752645in}}%
\pgfpathlineto{\pgfqpoint{5.168547in}{5.772623in}}%
\pgfpathlineto{\pgfqpoint{5.160447in}{5.791328in}}%
\pgfpathlineto{\pgfqpoint{5.152426in}{5.755535in}}%
\pgfpathlineto{\pgfqpoint{5.144370in}{5.757737in}}%
\pgfpathlineto{\pgfqpoint{5.136344in}{5.809823in}}%
\pgfpathlineto{\pgfqpoint{5.128231in}{5.778118in}}%
\pgfpathlineto{\pgfqpoint{5.120128in}{5.787803in}}%
\pgfpathlineto{\pgfqpoint{5.111982in}{5.756765in}}%
\pgfpathlineto{\pgfqpoint{5.103859in}{5.799078in}}%
\pgfpathlineto{\pgfqpoint{5.095819in}{5.810178in}}%
\pgfpathlineto{\pgfqpoint{5.087675in}{5.777296in}}%
\pgfpathlineto{\pgfqpoint{5.079506in}{5.764586in}}%
\pgfpathlineto{\pgfqpoint{5.071391in}{5.743381in}}%
\pgfpathlineto{\pgfqpoint{5.063216in}{5.751693in}}%
\pgfpathlineto{\pgfqpoint{5.055031in}{5.772021in}}%
\pgfpathlineto{\pgfqpoint{5.046876in}{5.762825in}}%
\pgfpathlineto{\pgfqpoint{5.038759in}{5.800005in}}%
\pgfpathlineto{\pgfqpoint{5.030602in}{5.779251in}}%
\pgfpathlineto{\pgfqpoint{5.022451in}{5.810378in}}%
\pgfpathlineto{\pgfqpoint{5.014317in}{5.759150in}}%
\pgfpathlineto{\pgfqpoint{5.006134in}{5.730243in}}%
\pgfpathlineto{\pgfqpoint{4.997948in}{5.791630in}}%
\pgfpathlineto{\pgfqpoint{4.989794in}{5.790295in}}%
\pgfpathlineto{\pgfqpoint{4.981642in}{5.749748in}}%
\pgfpathlineto{\pgfqpoint{4.973428in}{5.749887in}}%
\pgfpathlineto{\pgfqpoint{4.965138in}{5.735566in}}%
\pgfpathlineto{\pgfqpoint{4.956992in}{5.769930in}}%
\pgfpathlineto{\pgfqpoint{4.948774in}{5.775980in}}%
\pgfpathlineto{\pgfqpoint{4.940566in}{5.746612in}}%
\pgfpathlineto{\pgfqpoint{4.932347in}{5.781491in}}%
\pgfpathlineto{\pgfqpoint{4.924041in}{5.718479in}}%
\pgfpathlineto{\pgfqpoint{4.915736in}{5.762748in}}%
\pgfpathlineto{\pgfqpoint{4.907476in}{5.749439in}}%
\pgfpathlineto{\pgfqpoint{4.899205in}{5.808608in}}%
\pgfpathlineto{\pgfqpoint{4.890980in}{5.768961in}}%
\pgfpathlineto{\pgfqpoint{4.882658in}{5.744814in}}%
\pgfpathlineto{\pgfqpoint{4.874381in}{5.730820in}}%
\pgfpathlineto{\pgfqpoint{4.866069in}{5.771438in}}%
\pgfpathlineto{\pgfqpoint{4.857830in}{5.826175in}}%
\pgfpathlineto{\pgfqpoint{4.849575in}{5.741143in}}%
\pgfpathlineto{\pgfqpoint{4.841325in}{5.807933in}}%
\pgfpathlineto{\pgfqpoint{4.833061in}{5.772472in}}%
\pgfpathlineto{\pgfqpoint{4.824772in}{5.734477in}}%
\pgfpathlineto{\pgfqpoint{4.816444in}{5.754443in}}%
\pgfpathlineto{\pgfqpoint{4.808134in}{5.774982in}}%
\pgfpathlineto{\pgfqpoint{4.799850in}{5.772374in}}%
\pgfpathlineto{\pgfqpoint{4.791573in}{5.744669in}}%
\pgfpathlineto{\pgfqpoint{4.783272in}{5.739558in}}%
\pgfpathlineto{\pgfqpoint{4.774881in}{5.759734in}}%
\pgfpathlineto{\pgfqpoint{4.766528in}{5.731101in}}%
\pgfpathlineto{\pgfqpoint{4.758119in}{5.719410in}}%
\pgfpathlineto{\pgfqpoint{4.749750in}{5.752573in}}%
\pgfpathlineto{\pgfqpoint{4.741378in}{5.740711in}}%
\pgfpathlineto{\pgfqpoint{4.733049in}{5.755550in}}%
\pgfpathlineto{\pgfqpoint{4.724666in}{5.710039in}}%
\pgfpathlineto{\pgfqpoint{4.716247in}{5.766892in}}%
\pgfpathlineto{\pgfqpoint{4.707865in}{5.727526in}}%
\pgfpathlineto{\pgfqpoint{4.699465in}{5.767184in}}%
\pgfpathlineto{\pgfqpoint{4.691071in}{5.752686in}}%
\pgfpathlineto{\pgfqpoint{4.682643in}{5.761989in}}%
\pgfpathlineto{\pgfqpoint{4.674255in}{5.730938in}}%
\pgfpathlineto{\pgfqpoint{4.665825in}{5.647464in}}%
\pgfpathlineto{\pgfqpoint{4.657349in}{5.720245in}}%
\pgfpathlineto{\pgfqpoint{4.648847in}{5.589818in}}%
\pgfpathlineto{\pgfqpoint{4.640221in}{5.648943in}}%
\pgfpathlineto{\pgfqpoint{4.631733in}{5.716138in}}%
\pgfpathlineto{\pgfqpoint{4.623282in}{5.744681in}}%
\pgfpathlineto{\pgfqpoint{4.614783in}{5.689586in}}%
\pgfpathlineto{\pgfqpoint{4.606211in}{5.663382in}}%
\pgfpathlineto{\pgfqpoint{4.597719in}{5.726365in}}%
\pgfpathlineto{\pgfqpoint{4.589255in}{5.729893in}}%
\pgfpathlineto{\pgfqpoint{4.580727in}{5.678539in}}%
\pgfpathlineto{\pgfqpoint{4.572097in}{5.633323in}}%
\pgfpathlineto{\pgfqpoint{4.563550in}{5.668664in}}%
\pgfpathlineto{\pgfqpoint{4.554980in}{5.675694in}}%
\pgfpathlineto{\pgfqpoint{4.546439in}{5.694968in}}%
\pgfpathlineto{\pgfqpoint{4.537876in}{5.705940in}}%
\pgfpathlineto{\pgfqpoint{4.529276in}{5.617804in}}%
\pgfpathlineto{\pgfqpoint{4.520678in}{5.676077in}}%
\pgfpathlineto{\pgfqpoint{4.512084in}{5.613677in}}%
\pgfpathlineto{\pgfqpoint{4.503421in}{5.640411in}}%
\pgfpathlineto{\pgfqpoint{4.494752in}{5.627120in}}%
\pgfpathlineto{\pgfqpoint{4.486071in}{5.651740in}}%
\pgfpathlineto{\pgfqpoint{4.477466in}{5.657641in}}%
\pgfpathlineto{\pgfqpoint{4.468771in}{5.583055in}}%
\pgfpathlineto{\pgfqpoint{4.460011in}{5.648065in}}%
\pgfpathlineto{\pgfqpoint{4.451305in}{5.598857in}}%
\pgfpathlineto{\pgfqpoint{4.442591in}{5.633767in}}%
\pgfpathlineto{\pgfqpoint{4.433894in}{5.646519in}}%
\pgfpathlineto{\pgfqpoint{4.425248in}{5.593307in}}%
\pgfpathlineto{\pgfqpoint{4.416522in}{5.618644in}}%
\pgfpathlineto{\pgfqpoint{4.407811in}{5.653801in}}%
\pgfpathlineto{\pgfqpoint{4.399156in}{5.630819in}}%
\pgfpathlineto{\pgfqpoint{4.390451in}{5.644630in}}%
\pgfpathlineto{\pgfqpoint{4.381765in}{5.652077in}}%
\pgfpathlineto{\pgfqpoint{4.373001in}{5.569817in}}%
\pgfpathlineto{\pgfqpoint{4.364219in}{5.583440in}}%
\pgfpathlineto{\pgfqpoint{4.355507in}{5.612892in}}%
\pgfpathlineto{\pgfqpoint{4.346774in}{5.228899in}}%
\pgfpathlineto{\pgfqpoint{4.337933in}{4.614224in}}%
\pgfpathlineto{\pgfqpoint{4.328717in}{4.560985in}}%
\pgfpathlineto{\pgfqpoint{4.319388in}{4.531301in}}%
\pgfpathlineto{\pgfqpoint{4.310022in}{4.460577in}}%
\pgfpathlineto{\pgfqpoint{4.300486in}{4.510374in}}%
\pgfpathlineto{\pgfqpoint{4.290581in}{4.490608in}}%
\pgfpathlineto{\pgfqpoint{4.280743in}{4.467783in}}%
\pgfpathlineto{\pgfqpoint{4.270972in}{3.952584in}}%
\pgfpathlineto{\pgfqpoint{4.260400in}{4.276035in}}%
\pgfpathlineto{\pgfqpoint{4.249866in}{3.475498in}}%
\pgfpathlineto{\pgfqpoint{4.238844in}{3.416556in}}%
\pgfpathlineto{\pgfqpoint{4.227703in}{3.466682in}}%
\pgfpathlineto{\pgfqpoint{4.216619in}{3.406465in}}%
\pgfpathlineto{\pgfqpoint{4.205594in}{3.377115in}}%
\pgfpathlineto{\pgfqpoint{4.194400in}{3.395911in}}%
\pgfpathlineto{\pgfqpoint{4.183285in}{3.423197in}}%
\pgfpathlineto{\pgfqpoint{4.172101in}{3.440801in}}%
\pgfpathlineto{\pgfqpoint{4.160983in}{3.356308in}}%
\pgfpathlineto{\pgfqpoint{4.149813in}{3.346259in}}%
\pgfpathlineto{\pgfqpoint{4.138598in}{3.328045in}}%
\pgfpathlineto{\pgfqpoint{4.127322in}{3.347034in}}%
\pgfpathlineto{\pgfqpoint{4.116014in}{3.335023in}}%
\pgfpathlineto{\pgfqpoint{4.104772in}{3.347125in}}%
\pgfpathlineto{\pgfqpoint{4.093537in}{3.357141in}}%
\pgfpathlineto{\pgfqpoint{4.082264in}{3.331972in}}%
\pgfpathlineto{\pgfqpoint{4.070994in}{3.315452in}}%
\pgfpathlineto{\pgfqpoint{4.059512in}{3.288197in}}%
\pgfpathlineto{\pgfqpoint{4.048157in}{3.330196in}}%
\pgfpathlineto{\pgfqpoint{4.036918in}{3.310971in}}%
\pgfpathlineto{\pgfqpoint{4.025578in}{3.321021in}}%
\pgfpathlineto{\pgfqpoint{4.014312in}{3.321179in}}%
\pgfpathlineto{\pgfqpoint{4.003035in}{3.301529in}}%
\pgfpathlineto{\pgfqpoint{3.991607in}{3.293811in}}%
\pgfpathlineto{\pgfqpoint{3.980165in}{3.303509in}}%
\pgfpathlineto{\pgfqpoint{3.968800in}{3.304561in}}%
\pgfpathlineto{\pgfqpoint{3.957345in}{3.272861in}}%
\pgfpathlineto{\pgfqpoint{3.945995in}{3.303172in}}%
\pgfpathlineto{\pgfqpoint{3.934526in}{3.270407in}}%
\pgfpathlineto{\pgfqpoint{3.923007in}{3.279552in}}%
\pgfpathlineto{\pgfqpoint{3.911548in}{3.298420in}}%
\pgfpathlineto{\pgfqpoint{3.900120in}{3.289357in}}%
\pgfpathlineto{\pgfqpoint{3.888612in}{3.246935in}}%
\pgfpathlineto{\pgfqpoint{3.876999in}{3.257842in}}%
\pgfpathlineto{\pgfqpoint{3.865394in}{3.259136in}}%
\pgfpathlineto{\pgfqpoint{3.853773in}{3.270294in}}%
\pgfpathlineto{\pgfqpoint{3.842268in}{3.276823in}}%
\pgfpathlineto{\pgfqpoint{3.830748in}{3.270760in}}%
\pgfpathlineto{\pgfqpoint{3.819207in}{3.269204in}}%
\pgfpathlineto{\pgfqpoint{3.807700in}{3.276547in}}%
\pgfpathlineto{\pgfqpoint{3.796179in}{3.264953in}}%
\pgfpathlineto{\pgfqpoint{3.784213in}{3.154244in}}%
\pgfpathlineto{\pgfqpoint{3.772194in}{3.186262in}}%
\pgfpathlineto{\pgfqpoint{3.760252in}{3.189335in}}%
\pgfpathlineto{\pgfqpoint{3.748246in}{3.153175in}}%
\pgfpathlineto{\pgfqpoint{3.736218in}{3.171180in}}%
\pgfpathlineto{\pgfqpoint{3.723925in}{3.112946in}}%
\pgfpathlineto{\pgfqpoint{3.711790in}{3.178747in}}%
\pgfpathlineto{\pgfqpoint{3.699766in}{3.158953in}}%
\pgfpathlineto{\pgfqpoint{3.687677in}{3.164516in}}%
\pgfpathlineto{\pgfqpoint{3.675639in}{3.154565in}}%
\pgfpathlineto{\pgfqpoint{3.663361in}{3.109027in}}%
\pgfpathlineto{\pgfqpoint{3.651005in}{3.178837in}}%
\pgfpathlineto{\pgfqpoint{3.639052in}{3.184797in}}%
\pgfpathlineto{\pgfqpoint{3.627108in}{3.171976in}}%
\pgfpathlineto{\pgfqpoint{3.615064in}{3.164137in}}%
\pgfpathlineto{\pgfqpoint{3.602917in}{3.101917in}}%
\pgfpathlineto{\pgfqpoint{3.590359in}{3.132484in}}%
\pgfpathlineto{\pgfqpoint{3.578260in}{3.191036in}}%
\pgfpathlineto{\pgfqpoint{3.566335in}{3.171755in}}%
\pgfpathlineto{\pgfqpoint{3.554208in}{3.143971in}}%
\pgfpathlineto{\pgfqpoint{3.542041in}{3.066544in}}%
\pgfpathlineto{\pgfqpoint{3.528239in}{2.725352in}}%
\pgfpathlineto{\pgfqpoint{3.513323in}{2.689748in}}%
\pgfpathlineto{\pgfqpoint{3.498190in}{2.680501in}}%
\pgfpathlineto{\pgfqpoint{3.483019in}{2.687970in}}%
\pgfpathlineto{\pgfqpoint{3.467311in}{2.630315in}}%
\pgfpathlineto{\pgfqpoint{3.451777in}{2.651649in}}%
\pgfpathlineto{\pgfqpoint{3.436431in}{2.655963in}}%
\pgfpathlineto{\pgfqpoint{3.421182in}{2.666227in}}%
\pgfpathlineto{\pgfqpoint{3.405908in}{2.677180in}}%
\pgfpathlineto{\pgfqpoint{3.390722in}{2.660912in}}%
\pgfpathlineto{\pgfqpoint{3.375355in}{2.654920in}}%
\pgfpathlineto{\pgfqpoint{3.359992in}{2.651150in}}%
\pgfpathlineto{\pgfqpoint{3.344594in}{2.649711in}}%
\pgfpathlineto{\pgfqpoint{3.328928in}{2.613954in}}%
\pgfpathlineto{\pgfqpoint{3.313329in}{2.646419in}}%
\pgfpathlineto{\pgfqpoint{3.295281in}{2.584690in}}%
\pgfpathlineto{\pgfqpoint{3.279179in}{2.634390in}}%
\pgfpathlineto{\pgfqpoint{3.263722in}{2.652192in}}%
\pgfpathlineto{\pgfqpoint{3.248207in}{2.630758in}}%
\pgfpathlineto{\pgfqpoint{3.232763in}{2.639798in}}%
\pgfpathlineto{\pgfqpoint{3.217213in}{2.637163in}}%
\pgfpathlineto{\pgfqpoint{3.201602in}{2.607003in}}%
\pgfpathlineto{\pgfqpoint{3.185752in}{2.585390in}}%
\pgfpathlineto{\pgfqpoint{3.169651in}{2.543878in}}%
\pgfpathlineto{\pgfqpoint{3.152584in}{2.509450in}}%
\pgfpathlineto{\pgfqpoint{3.135319in}{2.559873in}}%
\pgfpathlineto{\pgfqpoint{3.118692in}{2.545337in}}%
\pgfpathlineto{\pgfqpoint{3.102316in}{2.563645in}}%
\pgfpathlineto{\pgfqpoint{3.085686in}{2.513817in}}%
\pgfpathlineto{\pgfqpoint{3.069003in}{2.568646in}}%
\pgfpathlineto{\pgfqpoint{3.052708in}{2.554806in}}%
\pgfpathlineto{\pgfqpoint{3.036276in}{2.535531in}}%
\pgfpathlineto{\pgfqpoint{3.018968in}{2.538470in}}%
\pgfpathlineto{\pgfqpoint{3.001921in}{2.562471in}}%
\pgfpathlineto{\pgfqpoint{2.985853in}{2.564093in}}%
\pgfpathlineto{\pgfqpoint{2.968433in}{2.429272in}}%
\pgfpathlineto{\pgfqpoint{2.949130in}{2.484437in}}%
\pgfpathlineto{\pgfqpoint{2.930775in}{2.511588in}}%
\pgfpathlineto{\pgfqpoint{2.914048in}{2.553501in}}%
\pgfpathlineto{\pgfqpoint{2.897563in}{2.582492in}}%
\pgfpathlineto{\pgfqpoint{2.880921in}{2.534069in}}%
\pgfpathlineto{\pgfqpoint{2.864455in}{2.550864in}}%
\pgfpathlineto{\pgfqpoint{2.848081in}{2.601750in}}%
\pgfpathlineto{\pgfqpoint{2.831972in}{2.557395in}}%
\pgfpathlineto{\pgfqpoint{2.815750in}{2.559164in}}%
\pgfpathlineto{\pgfqpoint{2.798917in}{2.458376in}}%
\pgfpathlineto{\pgfqpoint{2.780862in}{2.498080in}}%
\pgfpathlineto{\pgfqpoint{2.763118in}{2.507874in}}%
\pgfpathlineto{\pgfqpoint{2.744840in}{2.487325in}}%
\pgfpathlineto{\pgfqpoint{2.727722in}{2.528269in}}%
\pgfpathlineto{\pgfqpoint{2.710992in}{2.518833in}}%
\pgfpathlineto{\pgfqpoint{2.693540in}{2.464405in}}%
\pgfpathlineto{\pgfqpoint{2.676022in}{2.520883in}}%
\pgfpathlineto{\pgfqpoint{2.657274in}{2.411114in}}%
\pgfpathlineto{\pgfqpoint{2.636069in}{2.190707in}}%
\pgfpathlineto{\pgfqpoint{2.616149in}{2.537159in}}%
\pgfpathlineto{\pgfqpoint{2.598064in}{2.377907in}}%
\pgfpathlineto{\pgfqpoint{2.578521in}{2.439983in}}%
\pgfpathlineto{\pgfqpoint{2.558886in}{2.404478in}}%
\pgfpathlineto{\pgfqpoint{2.538352in}{2.186154in}}%
\pgfpathlineto{\pgfqpoint{2.515773in}{2.474736in}}%
\pgfpathlineto{\pgfqpoint{2.494737in}{2.510436in}}%
\pgfpathlineto{\pgfqpoint{2.477723in}{2.493295in}}%
\pgfpathlineto{\pgfqpoint{2.457197in}{2.177929in}}%
\pgfpathlineto{\pgfqpoint{2.433282in}{2.810953in}}%
\pgfpathlineto{\pgfqpoint{2.418633in}{3.092874in}}%
\pgfpathlineto{\pgfqpoint{2.406207in}{3.077467in}}%
\pgfpathlineto{\pgfqpoint{2.393733in}{3.099295in}}%
\pgfpathlineto{\pgfqpoint{2.381372in}{3.091239in}}%
\pgfpathlineto{\pgfqpoint{2.368880in}{3.077858in}}%
\pgfpathlineto{\pgfqpoint{2.356417in}{3.084966in}}%
\pgfpathlineto{\pgfqpoint{2.343909in}{3.083779in}}%
\pgfpathlineto{\pgfqpoint{2.331395in}{3.060141in}}%
\pgfpathlineto{\pgfqpoint{2.318891in}{3.088815in}}%
\pgfpathlineto{\pgfqpoint{2.306399in}{3.067690in}}%
\pgfpathlineto{\pgfqpoint{2.293873in}{3.071742in}}%
\pgfpathlineto{\pgfqpoint{2.281255in}{3.065402in}}%
\pgfpathlineto{\pgfqpoint{2.268761in}{3.083359in}}%
\pgfpathlineto{\pgfqpoint{2.256287in}{3.087617in}}%
\pgfpathlineto{\pgfqpoint{2.243791in}{3.076867in}}%
\pgfpathlineto{\pgfqpoint{2.231301in}{3.072971in}}%
\pgfpathlineto{\pgfqpoint{2.218726in}{3.059201in}}%
\pgfpathlineto{\pgfqpoint{2.206174in}{3.085509in}}%
\pgfpathlineto{\pgfqpoint{2.193673in}{3.071975in}}%
\pgfpathlineto{\pgfqpoint{2.181172in}{3.072762in}}%
\pgfpathlineto{\pgfqpoint{2.168659in}{3.074861in}}%
\pgfpathlineto{\pgfqpoint{2.156064in}{3.052149in}}%
\pgfpathlineto{\pgfqpoint{2.143512in}{3.075691in}}%
\pgfpathlineto{\pgfqpoint{2.130987in}{3.074455in}}%
\pgfpathlineto{\pgfqpoint{2.118500in}{3.089357in}}%
\pgfpathlineto{\pgfqpoint{2.106029in}{3.075366in}}%
\pgfpathlineto{\pgfqpoint{2.093434in}{3.052445in}}%
\pgfpathlineto{\pgfqpoint{2.080914in}{3.084696in}}%
\pgfpathlineto{\pgfqpoint{2.068360in}{3.072865in}}%
\pgfpathlineto{\pgfqpoint{2.055871in}{3.072921in}}%
\pgfpathlineto{\pgfqpoint{2.043350in}{3.074720in}}%
\pgfpathlineto{\pgfqpoint{2.030840in}{3.079317in}}%
\pgfpathlineto{\pgfqpoint{2.016889in}{2.993655in}}%
\pgfpathlineto{\pgfqpoint{2.004212in}{3.080138in}}%
\pgfpathlineto{\pgfqpoint{1.991646in}{3.068062in}}%
\pgfpathlineto{\pgfqpoint{1.979161in}{3.095799in}}%
\pgfpathlineto{\pgfqpoint{1.966664in}{3.071031in}}%
\pgfpathlineto{\pgfqpoint{1.954108in}{3.069292in}}%
\pgfpathlineto{\pgfqpoint{1.941620in}{3.076580in}}%
\pgfpathlineto{\pgfqpoint{1.929118in}{3.081472in}}%
\pgfpathlineto{\pgfqpoint{1.916611in}{3.068231in}}%
\pgfpathlineto{\pgfqpoint{1.904130in}{3.094658in}}%
\pgfpathlineto{\pgfqpoint{1.891620in}{3.075455in}}%
\pgfpathlineto{\pgfqpoint{1.879180in}{3.089930in}}%
\pgfpathlineto{\pgfqpoint{1.866747in}{3.095490in}}%
\pgfpathlineto{\pgfqpoint{1.854433in}{3.130882in}}%
\pgfpathlineto{\pgfqpoint{1.842065in}{3.084228in}}%
\pgfpathlineto{\pgfqpoint{1.829628in}{3.083817in}}%
\pgfpathlineto{\pgfqpoint{1.817148in}{3.087165in}}%
\pgfpathlineto{\pgfqpoint{1.804716in}{3.090860in}}%
\pgfpathlineto{\pgfqpoint{1.792355in}{3.106927in}}%
\pgfpathlineto{\pgfqpoint{1.779913in}{3.083303in}}%
\pgfpathlineto{\pgfqpoint{1.767480in}{3.091703in}}%
\pgfpathlineto{\pgfqpoint{1.755070in}{3.095331in}}%
\pgfpathlineto{\pgfqpoint{1.742593in}{3.091146in}}%
\pgfpathlineto{\pgfqpoint{1.730200in}{3.096001in}}%
\pgfpathlineto{\pgfqpoint{1.717832in}{3.122498in}}%
\pgfpathlineto{\pgfqpoint{1.705498in}{3.082828in}}%
\pgfpathlineto{\pgfqpoint{1.693005in}{3.080581in}}%
\pgfpathlineto{\pgfqpoint{1.680578in}{3.101244in}}%
\pgfpathlineto{\pgfqpoint{1.668187in}{3.084845in}}%
\pgfpathlineto{\pgfqpoint{1.655673in}{3.083480in}}%
\pgfpathlineto{\pgfqpoint{1.643249in}{3.087852in}}%
\pgfpathlineto{\pgfqpoint{1.630788in}{3.090462in}}%
\pgfpathlineto{\pgfqpoint{1.618336in}{3.086191in}}%
\pgfpathlineto{\pgfqpoint{1.605869in}{3.077817in}}%
\pgfpathlineto{\pgfqpoint{1.593430in}{3.108918in}}%
\pgfpathlineto{\pgfqpoint{1.581070in}{3.096682in}}%
\pgfpathlineto{\pgfqpoint{1.568672in}{3.102427in}}%
\pgfpathlineto{\pgfqpoint{1.556290in}{3.088360in}}%
\pgfpathlineto{\pgfqpoint{1.543839in}{3.091981in}}%
\pgfpathlineto{\pgfqpoint{1.531474in}{3.110975in}}%
\pgfpathlineto{\pgfqpoint{1.519060in}{3.083687in}}%
\pgfpathlineto{\pgfqpoint{1.506021in}{3.039025in}}%
\pgfpathlineto{\pgfqpoint{1.493578in}{3.096414in}}%
\pgfpathlineto{\pgfqpoint{1.481163in}{3.080651in}}%
\pgfpathlineto{\pgfqpoint{1.468660in}{3.079528in}}%
\pgfpathlineto{\pgfqpoint{1.456188in}{3.086896in}}%
\pgfpathlineto{\pgfqpoint{1.443756in}{3.094507in}}%
\pgfpathlineto{\pgfqpoint{1.431382in}{3.101402in}}%
\pgfpathlineto{\pgfqpoint{1.418893in}{3.064844in}}%
\pgfpathlineto{\pgfqpoint{1.406401in}{3.089108in}}%
\pgfpathlineto{\pgfqpoint{1.393906in}{3.085073in}}%
\pgfpathlineto{\pgfqpoint{1.381448in}{3.075557in}}%
\pgfpathlineto{\pgfqpoint{1.368992in}{3.094899in}}%
\pgfpathlineto{\pgfqpoint{1.356458in}{3.068695in}}%
\pgfpathlineto{\pgfqpoint{1.343969in}{3.091770in}}%
\pgfpathlineto{\pgfqpoint{1.331552in}{3.093626in}}%
\pgfpathlineto{\pgfqpoint{1.319073in}{3.074177in}}%
\pgfpathlineto{\pgfqpoint{1.306603in}{3.078489in}}%
\pgfpathlineto{\pgfqpoint{1.294165in}{3.105292in}}%
\pgfpathlineto{\pgfqpoint{1.281753in}{3.081736in}}%
\pgfpathlineto{\pgfqpoint{1.269255in}{3.083805in}}%
\pgfpathlineto{\pgfqpoint{1.256724in}{3.054659in}}%
\pgfpathlineto{\pgfqpoint{1.244104in}{3.091190in}}%
\pgfpathlineto{\pgfqpoint{1.231699in}{3.095557in}}%
\pgfpathlineto{\pgfqpoint{1.219183in}{3.059090in}}%
\pgfpathlineto{\pgfqpoint{1.206622in}{3.067068in}}%
\pgfpathlineto{\pgfqpoint{1.194024in}{3.067018in}}%
\pgfpathlineto{\pgfqpoint{1.181503in}{3.072174in}}%
\pgfpathlineto{\pgfqpoint{1.168924in}{3.049177in}}%
\pgfpathlineto{\pgfqpoint{1.156226in}{3.043227in}}%
\pgfpathlineto{\pgfqpoint{1.143621in}{3.068982in}}%
\pgfpathlineto{\pgfqpoint{1.131014in}{3.042428in}}%
\pgfpathlineto{\pgfqpoint{1.118303in}{3.045253in}}%
\pgfpathlineto{\pgfqpoint{1.105655in}{3.051896in}}%
\pgfpathlineto{\pgfqpoint{1.092853in}{2.996437in}}%
\pgfpathlineto{\pgfqpoint{1.079972in}{3.026816in}}%
\pgfpathlineto{\pgfqpoint{1.067143in}{3.007848in}}%
\pgfpathlineto{\pgfqpoint{1.054264in}{3.007543in}}%
\pgfpathlineto{\pgfqpoint{1.041322in}{2.998528in}}%
\pgfpathlineto{\pgfqpoint{1.028303in}{2.978046in}}%
\pgfpathlineto{\pgfqpoint{1.015150in}{2.926401in}}%
\pgfpathlineto{\pgfqpoint{1.001616in}{2.736196in}}%
\pgfpathclose%
\pgfusepath{fill}%
\end{pgfscope}%
\begin{pgfscope}%
\pgfsetbuttcap%
\pgfsetroundjoin%
\definecolor{currentfill}{rgb}{0.000000,0.000000,0.000000}%
\pgfsetfillcolor{currentfill}%
\pgfsetlinewidth{0.803000pt}%
\definecolor{currentstroke}{rgb}{0.000000,0.000000,0.000000}%
\pgfsetstrokecolor{currentstroke}%
\pgfsetdash{}{0pt}%
\pgfsys@defobject{currentmarker}{\pgfqpoint{0.000000in}{-0.048611in}}{\pgfqpoint{0.000000in}{0.000000in}}{%
\pgfpathmoveto{\pgfqpoint{0.000000in}{0.000000in}}%
\pgfpathlineto{\pgfqpoint{0.000000in}{-0.048611in}}%
\pgfusepath{stroke,fill}%
}%
\begin{pgfscope}%
\pgfsys@transformshift{0.994752in}{0.773588in}%
\pgfsys@useobject{currentmarker}{}%
\end{pgfscope}%
\end{pgfscope}%
\begin{pgfscope}%
\definecolor{textcolor}{rgb}{0.000000,0.000000,0.000000}%
\pgfsetstrokecolor{textcolor}%
\pgfsetfillcolor{textcolor}%
\pgftext[x=0.994752in,y=0.676366in,,top]{\color{textcolor}\rmfamily\fontsize{10.000000}{12.000000}\selectfont \(\displaystyle {0}\)}%
\end{pgfscope}%
\begin{pgfscope}%
\pgfsetbuttcap%
\pgfsetroundjoin%
\definecolor{currentfill}{rgb}{0.000000,0.000000,0.000000}%
\pgfsetfillcolor{currentfill}%
\pgfsetlinewidth{0.803000pt}%
\definecolor{currentstroke}{rgb}{0.000000,0.000000,0.000000}%
\pgfsetstrokecolor{currentstroke}%
\pgfsetdash{}{0pt}%
\pgfsys@defobject{currentmarker}{\pgfqpoint{0.000000in}{-0.048611in}}{\pgfqpoint{0.000000in}{0.000000in}}{%
\pgfpathmoveto{\pgfqpoint{0.000000in}{0.000000in}}%
\pgfpathlineto{\pgfqpoint{0.000000in}{-0.048611in}}%
\pgfusepath{stroke,fill}%
}%
\begin{pgfscope}%
\pgfsys@transformshift{1.962413in}{0.773588in}%
\pgfsys@useobject{currentmarker}{}%
\end{pgfscope}%
\end{pgfscope}%
\begin{pgfscope}%
\definecolor{textcolor}{rgb}{0.000000,0.000000,0.000000}%
\pgfsetstrokecolor{textcolor}%
\pgfsetfillcolor{textcolor}%
\pgftext[x=1.962413in,y=0.676366in,,top]{\color{textcolor}\rmfamily\fontsize{10.000000}{12.000000}\selectfont \(\displaystyle {100}\)}%
\end{pgfscope}%
\begin{pgfscope}%
\pgfsetbuttcap%
\pgfsetroundjoin%
\definecolor{currentfill}{rgb}{0.000000,0.000000,0.000000}%
\pgfsetfillcolor{currentfill}%
\pgfsetlinewidth{0.803000pt}%
\definecolor{currentstroke}{rgb}{0.000000,0.000000,0.000000}%
\pgfsetstrokecolor{currentstroke}%
\pgfsetdash{}{0pt}%
\pgfsys@defobject{currentmarker}{\pgfqpoint{0.000000in}{-0.048611in}}{\pgfqpoint{0.000000in}{0.000000in}}{%
\pgfpathmoveto{\pgfqpoint{0.000000in}{0.000000in}}%
\pgfpathlineto{\pgfqpoint{0.000000in}{-0.048611in}}%
\pgfusepath{stroke,fill}%
}%
\begin{pgfscope}%
\pgfsys@transformshift{2.930074in}{0.773588in}%
\pgfsys@useobject{currentmarker}{}%
\end{pgfscope}%
\end{pgfscope}%
\begin{pgfscope}%
\definecolor{textcolor}{rgb}{0.000000,0.000000,0.000000}%
\pgfsetstrokecolor{textcolor}%
\pgfsetfillcolor{textcolor}%
\pgftext[x=2.930074in,y=0.676366in,,top]{\color{textcolor}\rmfamily\fontsize{10.000000}{12.000000}\selectfont \(\displaystyle {200}\)}%
\end{pgfscope}%
\begin{pgfscope}%
\pgfsetbuttcap%
\pgfsetroundjoin%
\definecolor{currentfill}{rgb}{0.000000,0.000000,0.000000}%
\pgfsetfillcolor{currentfill}%
\pgfsetlinewidth{0.803000pt}%
\definecolor{currentstroke}{rgb}{0.000000,0.000000,0.000000}%
\pgfsetstrokecolor{currentstroke}%
\pgfsetdash{}{0pt}%
\pgfsys@defobject{currentmarker}{\pgfqpoint{0.000000in}{-0.048611in}}{\pgfqpoint{0.000000in}{0.000000in}}{%
\pgfpathmoveto{\pgfqpoint{0.000000in}{0.000000in}}%
\pgfpathlineto{\pgfqpoint{0.000000in}{-0.048611in}}%
\pgfusepath{stroke,fill}%
}%
\begin{pgfscope}%
\pgfsys@transformshift{3.897735in}{0.773588in}%
\pgfsys@useobject{currentmarker}{}%
\end{pgfscope}%
\end{pgfscope}%
\begin{pgfscope}%
\definecolor{textcolor}{rgb}{0.000000,0.000000,0.000000}%
\pgfsetstrokecolor{textcolor}%
\pgfsetfillcolor{textcolor}%
\pgftext[x=3.897735in,y=0.676366in,,top]{\color{textcolor}\rmfamily\fontsize{10.000000}{12.000000}\selectfont \(\displaystyle {300}\)}%
\end{pgfscope}%
\begin{pgfscope}%
\pgfsetbuttcap%
\pgfsetroundjoin%
\definecolor{currentfill}{rgb}{0.000000,0.000000,0.000000}%
\pgfsetfillcolor{currentfill}%
\pgfsetlinewidth{0.803000pt}%
\definecolor{currentstroke}{rgb}{0.000000,0.000000,0.000000}%
\pgfsetstrokecolor{currentstroke}%
\pgfsetdash{}{0pt}%
\pgfsys@defobject{currentmarker}{\pgfqpoint{0.000000in}{-0.048611in}}{\pgfqpoint{0.000000in}{0.000000in}}{%
\pgfpathmoveto{\pgfqpoint{0.000000in}{0.000000in}}%
\pgfpathlineto{\pgfqpoint{0.000000in}{-0.048611in}}%
\pgfusepath{stroke,fill}%
}%
\begin{pgfscope}%
\pgfsys@transformshift{4.865395in}{0.773588in}%
\pgfsys@useobject{currentmarker}{}%
\end{pgfscope}%
\end{pgfscope}%
\begin{pgfscope}%
\definecolor{textcolor}{rgb}{0.000000,0.000000,0.000000}%
\pgfsetstrokecolor{textcolor}%
\pgfsetfillcolor{textcolor}%
\pgftext[x=4.865395in,y=0.676366in,,top]{\color{textcolor}\rmfamily\fontsize{10.000000}{12.000000}\selectfont \(\displaystyle {400}\)}%
\end{pgfscope}%
\begin{pgfscope}%
\definecolor{textcolor}{rgb}{0.000000,0.000000,0.000000}%
\pgfsetstrokecolor{textcolor}%
\pgfsetfillcolor{textcolor}%
\pgftext[x=3.203750in,y=0.497354in,,top]{\color{textcolor}\rmfamily\fontsize{10.000000}{12.000000}\selectfont Time (milliseconds)}%
\end{pgfscope}%
\begin{pgfscope}%
\pgfsetbuttcap%
\pgfsetroundjoin%
\definecolor{currentfill}{rgb}{0.000000,0.000000,0.000000}%
\pgfsetfillcolor{currentfill}%
\pgfsetlinewidth{0.803000pt}%
\definecolor{currentstroke}{rgb}{0.000000,0.000000,0.000000}%
\pgfsetstrokecolor{currentstroke}%
\pgfsetdash{}{0pt}%
\pgfsys@defobject{currentmarker}{\pgfqpoint{-0.048611in}{0.000000in}}{\pgfqpoint{-0.000000in}{0.000000in}}{%
\pgfpathmoveto{\pgfqpoint{-0.000000in}{0.000000in}}%
\pgfpathlineto{\pgfqpoint{-0.048611in}{0.000000in}}%
\pgfusepath{stroke,fill}%
}%
\begin{pgfscope}%
\pgfsys@transformshift{0.781402in}{0.773588in}%
\pgfsys@useobject{currentmarker}{}%
\end{pgfscope}%
\end{pgfscope}%
\begin{pgfscope}%
\definecolor{textcolor}{rgb}{0.000000,0.000000,0.000000}%
\pgfsetstrokecolor{textcolor}%
\pgfsetfillcolor{textcolor}%
\pgftext[x=0.614736in, y=0.725363in, left, base]{\color{textcolor}\rmfamily\fontsize{10.000000}{12.000000}\selectfont \(\displaystyle {0}\)}%
\end{pgfscope}%
\begin{pgfscope}%
\pgfsetbuttcap%
\pgfsetroundjoin%
\definecolor{currentfill}{rgb}{0.000000,0.000000,0.000000}%
\pgfsetfillcolor{currentfill}%
\pgfsetlinewidth{0.803000pt}%
\definecolor{currentstroke}{rgb}{0.000000,0.000000,0.000000}%
\pgfsetstrokecolor{currentstroke}%
\pgfsetdash{}{0pt}%
\pgfsys@defobject{currentmarker}{\pgfqpoint{-0.048611in}{0.000000in}}{\pgfqpoint{-0.000000in}{0.000000in}}{%
\pgfpathmoveto{\pgfqpoint{-0.000000in}{0.000000in}}%
\pgfpathlineto{\pgfqpoint{-0.048611in}{0.000000in}}%
\pgfusepath{stroke,fill}%
}%
\begin{pgfscope}%
\pgfsys@transformshift{0.781402in}{1.961848in}%
\pgfsys@useobject{currentmarker}{}%
\end{pgfscope}%
\end{pgfscope}%
\begin{pgfscope}%
\definecolor{textcolor}{rgb}{0.000000,0.000000,0.000000}%
\pgfsetstrokecolor{textcolor}%
\pgfsetfillcolor{textcolor}%
\pgftext[x=0.614736in, y=1.913623in, left, base]{\color{textcolor}\rmfamily\fontsize{10.000000}{12.000000}\selectfont \(\displaystyle {2}\)}%
\end{pgfscope}%
\begin{pgfscope}%
\pgfsetbuttcap%
\pgfsetroundjoin%
\definecolor{currentfill}{rgb}{0.000000,0.000000,0.000000}%
\pgfsetfillcolor{currentfill}%
\pgfsetlinewidth{0.803000pt}%
\definecolor{currentstroke}{rgb}{0.000000,0.000000,0.000000}%
\pgfsetstrokecolor{currentstroke}%
\pgfsetdash{}{0pt}%
\pgfsys@defobject{currentmarker}{\pgfqpoint{-0.048611in}{0.000000in}}{\pgfqpoint{-0.000000in}{0.000000in}}{%
\pgfpathmoveto{\pgfqpoint{-0.000000in}{0.000000in}}%
\pgfpathlineto{\pgfqpoint{-0.048611in}{0.000000in}}%
\pgfusepath{stroke,fill}%
}%
\begin{pgfscope}%
\pgfsys@transformshift{0.781402in}{3.150108in}%
\pgfsys@useobject{currentmarker}{}%
\end{pgfscope}%
\end{pgfscope}%
\begin{pgfscope}%
\definecolor{textcolor}{rgb}{0.000000,0.000000,0.000000}%
\pgfsetstrokecolor{textcolor}%
\pgfsetfillcolor{textcolor}%
\pgftext[x=0.614736in, y=3.101883in, left, base]{\color{textcolor}\rmfamily\fontsize{10.000000}{12.000000}\selectfont \(\displaystyle {4}\)}%
\end{pgfscope}%
\begin{pgfscope}%
\pgfsetbuttcap%
\pgfsetroundjoin%
\definecolor{currentfill}{rgb}{0.000000,0.000000,0.000000}%
\pgfsetfillcolor{currentfill}%
\pgfsetlinewidth{0.803000pt}%
\definecolor{currentstroke}{rgb}{0.000000,0.000000,0.000000}%
\pgfsetstrokecolor{currentstroke}%
\pgfsetdash{}{0pt}%
\pgfsys@defobject{currentmarker}{\pgfqpoint{-0.048611in}{0.000000in}}{\pgfqpoint{-0.000000in}{0.000000in}}{%
\pgfpathmoveto{\pgfqpoint{-0.000000in}{0.000000in}}%
\pgfpathlineto{\pgfqpoint{-0.048611in}{0.000000in}}%
\pgfusepath{stroke,fill}%
}%
\begin{pgfscope}%
\pgfsys@transformshift{0.781402in}{4.338368in}%
\pgfsys@useobject{currentmarker}{}%
\end{pgfscope}%
\end{pgfscope}%
\begin{pgfscope}%
\definecolor{textcolor}{rgb}{0.000000,0.000000,0.000000}%
\pgfsetstrokecolor{textcolor}%
\pgfsetfillcolor{textcolor}%
\pgftext[x=0.614736in, y=4.290142in, left, base]{\color{textcolor}\rmfamily\fontsize{10.000000}{12.000000}\selectfont \(\displaystyle {6}\)}%
\end{pgfscope}%
\begin{pgfscope}%
\pgfsetbuttcap%
\pgfsetroundjoin%
\definecolor{currentfill}{rgb}{0.000000,0.000000,0.000000}%
\pgfsetfillcolor{currentfill}%
\pgfsetlinewidth{0.803000pt}%
\definecolor{currentstroke}{rgb}{0.000000,0.000000,0.000000}%
\pgfsetstrokecolor{currentstroke}%
\pgfsetdash{}{0pt}%
\pgfsys@defobject{currentmarker}{\pgfqpoint{-0.048611in}{0.000000in}}{\pgfqpoint{-0.000000in}{0.000000in}}{%
\pgfpathmoveto{\pgfqpoint{-0.000000in}{0.000000in}}%
\pgfpathlineto{\pgfqpoint{-0.048611in}{0.000000in}}%
\pgfusepath{stroke,fill}%
}%
\begin{pgfscope}%
\pgfsys@transformshift{0.781402in}{5.526627in}%
\pgfsys@useobject{currentmarker}{}%
\end{pgfscope}%
\end{pgfscope}%
\begin{pgfscope}%
\definecolor{textcolor}{rgb}{0.000000,0.000000,0.000000}%
\pgfsetstrokecolor{textcolor}%
\pgfsetfillcolor{textcolor}%
\pgftext[x=0.614736in, y=5.478402in, left, base]{\color{textcolor}\rmfamily\fontsize{10.000000}{12.000000}\selectfont \(\displaystyle {8}\)}%
\end{pgfscope}%
\begin{pgfscope}%
\definecolor{textcolor}{rgb}{0.000000,0.000000,0.000000}%
\pgfsetstrokecolor{textcolor}%
\pgfsetfillcolor{textcolor}%
\pgftext[x=0.559180in,y=3.481148in,,bottom,rotate=90.000000]{\color{textcolor}\rmfamily\fontsize{10.000000}{12.000000}\selectfont Throughput (million operations/second)}%
\end{pgfscope}%
\begin{pgfscope}%
\pgfpathrectangle{\pgfqpoint{0.781402in}{0.773588in}}{\pgfqpoint{4.844695in}{5.415119in}}%
\pgfusepath{clip}%
\pgfsetrectcap%
\pgfsetroundjoin%
\pgfsetlinewidth{1.505625pt}%
\definecolor{currentstroke}{rgb}{0.000000,0.000000,1.000000}%
\pgfsetstrokecolor{currentstroke}%
\pgfsetdash{}{0pt}%
\pgfpathmoveto{\pgfqpoint{1.091273in}{0.773588in}}%
\pgfpathlineto{\pgfqpoint{1.091273in}{6.188708in}}%
\pgfusepath{stroke}%
\end{pgfscope}%
\begin{pgfscope}%
\pgfpathrectangle{\pgfqpoint{0.781402in}{0.773588in}}{\pgfqpoint{4.844695in}{5.415119in}}%
\pgfusepath{clip}%
\pgfsetrectcap%
\pgfsetroundjoin%
\pgfsetlinewidth{1.505625pt}%
\definecolor{currentstroke}{rgb}{0.750000,0.750000,0.000000}%
\pgfsetstrokecolor{currentstroke}%
\pgfsetdash{}{0pt}%
\pgfpathmoveto{\pgfqpoint{2.424291in}{0.773588in}}%
\pgfpathlineto{\pgfqpoint{2.424291in}{6.188708in}}%
\pgfusepath{stroke}%
\end{pgfscope}%
\begin{pgfscope}%
\pgfpathrectangle{\pgfqpoint{0.781402in}{0.773588in}}{\pgfqpoint{4.844695in}{5.415119in}}%
\pgfusepath{clip}%
\pgfsetrectcap%
\pgfsetroundjoin%
\pgfsetlinewidth{1.505625pt}%
\definecolor{currentstroke}{rgb}{0.750000,0.000000,0.750000}%
\pgfsetstrokecolor{currentstroke}%
\pgfsetdash{}{0pt}%
\pgfpathmoveto{\pgfqpoint{3.529163in}{0.773588in}}%
\pgfpathlineto{\pgfqpoint{3.529163in}{6.188708in}}%
\pgfusepath{stroke}%
\end{pgfscope}%
\begin{pgfscope}%
\pgfpathrectangle{\pgfqpoint{0.781402in}{0.773588in}}{\pgfqpoint{4.844695in}{5.415119in}}%
\pgfusepath{clip}%
\pgfsetrectcap%
\pgfsetroundjoin%
\pgfsetlinewidth{1.505625pt}%
\definecolor{currentstroke}{rgb}{1.000000,0.000000,0.000000}%
\pgfsetstrokecolor{currentstroke}%
\pgfsetdash{}{0pt}%
\pgfpathmoveto{\pgfqpoint{3.802748in}{0.773588in}}%
\pgfpathlineto{\pgfqpoint{3.802748in}{6.188708in}}%
\pgfusepath{stroke}%
\end{pgfscope}%
\begin{pgfscope}%
\pgfpathrectangle{\pgfqpoint{0.781402in}{0.773588in}}{\pgfqpoint{4.844695in}{5.415119in}}%
\pgfusepath{clip}%
\pgfsetrectcap%
\pgfsetroundjoin%
\pgfsetlinewidth{1.505625pt}%
\definecolor{currentstroke}{rgb}{0.000000,0.500000,0.000000}%
\pgfsetstrokecolor{currentstroke}%
\pgfsetdash{}{0pt}%
\pgfpathmoveto{\pgfqpoint{3.818529in}{0.773588in}}%
\pgfpathlineto{\pgfqpoint{3.818529in}{6.188708in}}%
\pgfusepath{stroke}%
\end{pgfscope}%
\begin{pgfscope}%
\pgfpathrectangle{\pgfqpoint{0.781402in}{0.773588in}}{\pgfqpoint{4.844695in}{5.415119in}}%
\pgfusepath{clip}%
\pgfsetrectcap%
\pgfsetroundjoin%
\pgfsetlinewidth{1.505625pt}%
\definecolor{currentstroke}{rgb}{0.000000,0.750000,0.750000}%
\pgfsetstrokecolor{currentstroke}%
\pgfsetdash{}{0pt}%
\pgfpathmoveto{\pgfqpoint{4.346570in}{0.773588in}}%
\pgfpathlineto{\pgfqpoint{4.346570in}{6.188708in}}%
\pgfusepath{stroke}%
\end{pgfscope}%
\begin{pgfscope}%
\pgfpathrectangle{\pgfqpoint{0.781402in}{0.773588in}}{\pgfqpoint{4.844695in}{5.415119in}}%
\pgfusepath{clip}%
\pgfsetrectcap%
\pgfsetroundjoin%
\pgfsetlinewidth{1.505625pt}%
\definecolor{currentstroke}{rgb}{0.750000,0.000000,0.750000}%
\pgfsetstrokecolor{currentstroke}%
\pgfsetdash{}{0pt}%
\pgfpathmoveto{\pgfqpoint{4.346623in}{0.773588in}}%
\pgfpathlineto{\pgfqpoint{4.346623in}{6.188708in}}%
\pgfusepath{stroke}%
\end{pgfscope}%
\begin{pgfscope}%
\pgfsetrectcap%
\pgfsetmiterjoin%
\pgfsetlinewidth{0.803000pt}%
\definecolor{currentstroke}{rgb}{0.000000,0.000000,0.000000}%
\pgfsetstrokecolor{currentstroke}%
\pgfsetdash{}{0pt}%
\pgfpathmoveto{\pgfqpoint{0.781402in}{0.773588in}}%
\pgfpathlineto{\pgfqpoint{0.781402in}{6.188708in}}%
\pgfusepath{stroke}%
\end{pgfscope}%
\begin{pgfscope}%
\pgfsetrectcap%
\pgfsetmiterjoin%
\pgfsetlinewidth{0.803000pt}%
\definecolor{currentstroke}{rgb}{0.000000,0.000000,0.000000}%
\pgfsetstrokecolor{currentstroke}%
\pgfsetdash{}{0pt}%
\pgfpathmoveto{\pgfqpoint{5.626098in}{0.773588in}}%
\pgfpathlineto{\pgfqpoint{5.626098in}{6.188708in}}%
\pgfusepath{stroke}%
\end{pgfscope}%
\begin{pgfscope}%
\pgfsetrectcap%
\pgfsetmiterjoin%
\pgfsetlinewidth{0.803000pt}%
\definecolor{currentstroke}{rgb}{0.000000,0.000000,0.000000}%
\pgfsetstrokecolor{currentstroke}%
\pgfsetdash{}{0pt}%
\pgfpathmoveto{\pgfqpoint{0.781402in}{0.773588in}}%
\pgfpathlineto{\pgfqpoint{5.626098in}{0.773588in}}%
\pgfusepath{stroke}%
\end{pgfscope}%
\begin{pgfscope}%
\pgfsetrectcap%
\pgfsetmiterjoin%
\pgfsetlinewidth{0.803000pt}%
\definecolor{currentstroke}{rgb}{0.000000,0.000000,0.000000}%
\pgfsetstrokecolor{currentstroke}%
\pgfsetdash{}{0pt}%
\pgfpathmoveto{\pgfqpoint{0.781402in}{6.188708in}}%
\pgfpathlineto{\pgfqpoint{5.626098in}{6.188708in}}%
\pgfusepath{stroke}%
\end{pgfscope}%
\begin{pgfscope}%
\pgfsetbuttcap%
\pgfsetmiterjoin%
\definecolor{currentfill}{rgb}{1.000000,1.000000,1.000000}%
\pgfsetfillcolor{currentfill}%
\pgfsetfillopacity{0.800000}%
\pgfsetlinewidth{1.003750pt}%
\definecolor{currentstroke}{rgb}{0.800000,0.800000,0.800000}%
\pgfsetstrokecolor{currentstroke}%
\pgfsetstrokeopacity{0.800000}%
\pgfsetdash{}{0pt}%
\pgfpathmoveto{\pgfqpoint{0.878625in}{3.559851in}}%
\pgfpathlineto{\pgfqpoint{3.799925in}{3.559851in}}%
\pgfpathquadraticcurveto{\pgfqpoint{3.827703in}{3.559851in}}{\pgfqpoint{3.827703in}{3.587628in}}%
\pgfpathlineto{\pgfqpoint{3.827703in}{6.091486in}}%
\pgfpathquadraticcurveto{\pgfqpoint{3.827703in}{6.119263in}}{\pgfqpoint{3.799925in}{6.119263in}}%
\pgfpathlineto{\pgfqpoint{0.878625in}{6.119263in}}%
\pgfpathquadraticcurveto{\pgfqpoint{0.850847in}{6.119263in}}{\pgfqpoint{0.850847in}{6.091486in}}%
\pgfpathlineto{\pgfqpoint{0.850847in}{3.587628in}}%
\pgfpathquadraticcurveto{\pgfqpoint{0.850847in}{3.559851in}}{\pgfqpoint{0.878625in}{3.559851in}}%
\pgfpathclose%
\pgfusepath{stroke,fill}%
\end{pgfscope}%
\begin{pgfscope}%
\pgfsetrectcap%
\pgfsetroundjoin%
\pgfsetlinewidth{1.505625pt}%
\definecolor{currentstroke}{rgb}{0.000000,0.000000,1.000000}%
\pgfsetstrokecolor{currentstroke}%
\pgfsetdash{}{0pt}%
\pgfpathmoveto{\pgfqpoint{0.906402in}{6.015097in}}%
\pgfpathlineto{\pgfqpoint{1.184180in}{6.015097in}}%
\pgfusepath{stroke}%
\end{pgfscope}%
\begin{pgfscope}%
\definecolor{textcolor}{rgb}{0.000000,0.000000,0.000000}%
\pgfsetstrokecolor{textcolor}%
\pgfsetfillcolor{textcolor}%
\pgftext[x=1.295291in,y=5.966486in,left,base]{\color{textcolor}\rmfamily\fontsize{10.000000}{12.000000}\selectfont Started migration}%
\end{pgfscope}%
\begin{pgfscope}%
\pgfsetrectcap%
\pgfsetroundjoin%
\pgfsetlinewidth{1.505625pt}%
\definecolor{currentstroke}{rgb}{0.750000,0.750000,0.000000}%
\pgfsetstrokecolor{currentstroke}%
\pgfsetdash{}{0pt}%
\pgfpathmoveto{\pgfqpoint{0.906402in}{5.821424in}}%
\pgfpathlineto{\pgfqpoint{1.184180in}{5.821424in}}%
\pgfusepath{stroke}%
\end{pgfscope}%
\begin{pgfscope}%
\definecolor{textcolor}{rgb}{0.000000,0.000000,0.000000}%
\pgfsetstrokecolor{textcolor}%
\pgfsetfillcolor{textcolor}%
\pgftext[x=1.295291in,y=5.772813in,left,base]{\color{textcolor}\rmfamily\fontsize{10.000000}{12.000000}\selectfont Started prefill writes}%
\end{pgfscope}%
\begin{pgfscope}%
\pgfsetrectcap%
\pgfsetroundjoin%
\pgfsetlinewidth{1.505625pt}%
\definecolor{currentstroke}{rgb}{0.750000,0.000000,0.750000}%
\pgfsetstrokecolor{currentstroke}%
\pgfsetdash{}{0pt}%
\pgfpathmoveto{\pgfqpoint{0.906402in}{5.627751in}}%
\pgfpathlineto{\pgfqpoint{1.184180in}{5.627751in}}%
\pgfusepath{stroke}%
\end{pgfscope}%
\begin{pgfscope}%
\definecolor{textcolor}{rgb}{0.000000,0.000000,0.000000}%
\pgfsetstrokecolor{textcolor}%
\pgfsetfillcolor{textcolor}%
\pgftext[x=1.295291in,y=5.579140in,left,base]{\color{textcolor}\rmfamily\fontsize{10.000000}{12.000000}\selectfont Finished prefill writes}%
\end{pgfscope}%
\begin{pgfscope}%
\pgfsetrectcap%
\pgfsetroundjoin%
\pgfsetlinewidth{1.505625pt}%
\definecolor{currentstroke}{rgb}{1.000000,0.000000,0.000000}%
\pgfsetstrokecolor{currentstroke}%
\pgfsetdash{}{0pt}%
\pgfpathmoveto{\pgfqpoint{0.906402in}{5.434078in}}%
\pgfpathlineto{\pgfqpoint{1.184180in}{5.434078in}}%
\pgfusepath{stroke}%
\end{pgfscope}%
\begin{pgfscope}%
\definecolor{textcolor}{rgb}{0.000000,0.000000,0.000000}%
\pgfsetstrokecolor{textcolor}%
\pgfsetfillcolor{textcolor}%
\pgftext[x=1.295291in,y=5.385467in,left,base]{\color{textcolor}\rmfamily\fontsize{10.000000}{12.000000}\selectfont Transferred ownership to the destination}%
\end{pgfscope}%
\begin{pgfscope}%
\pgfsetrectcap%
\pgfsetroundjoin%
\pgfsetlinewidth{1.505625pt}%
\definecolor{currentstroke}{rgb}{0.000000,0.500000,0.000000}%
\pgfsetstrokecolor{currentstroke}%
\pgfsetdash{}{0pt}%
\pgfpathmoveto{\pgfqpoint{0.906402in}{5.240406in}}%
\pgfpathlineto{\pgfqpoint{1.184180in}{5.240406in}}%
\pgfusepath{stroke}%
\end{pgfscope}%
\begin{pgfscope}%
\definecolor{textcolor}{rgb}{0.000000,0.000000,0.000000}%
\pgfsetstrokecolor{textcolor}%
\pgfsetfillcolor{textcolor}%
\pgftext[x=1.295291in,y=5.191794in,left,base]{\color{textcolor}\rmfamily\fontsize{10.000000}{12.000000}\selectfont Started reading dirty pages}%
\end{pgfscope}%
\begin{pgfscope}%
\pgfsetrectcap%
\pgfsetroundjoin%
\pgfsetlinewidth{1.505625pt}%
\definecolor{currentstroke}{rgb}{0.000000,0.750000,0.750000}%
\pgfsetstrokecolor{currentstroke}%
\pgfsetdash{}{0pt}%
\pgfpathmoveto{\pgfqpoint{0.906402in}{5.046733in}}%
\pgfpathlineto{\pgfqpoint{1.184180in}{5.046733in}}%
\pgfusepath{stroke}%
\end{pgfscope}%
\begin{pgfscope}%
\definecolor{textcolor}{rgb}{0.000000,0.000000,0.000000}%
\pgfsetstrokecolor{textcolor}%
\pgfsetfillcolor{textcolor}%
\pgftext[x=1.295291in,y=4.998122in,left,base]{\color{textcolor}\rmfamily\fontsize{10.000000}{12.000000}\selectfont Finished reading dirty pages}%
\end{pgfscope}%
\begin{pgfscope}%
\pgfsetrectcap%
\pgfsetroundjoin%
\pgfsetlinewidth{1.505625pt}%
\definecolor{currentstroke}{rgb}{0.750000,0.000000,0.750000}%
\pgfsetstrokecolor{currentstroke}%
\pgfsetdash{}{0pt}%
\pgfpathmoveto{\pgfqpoint{0.906402in}{4.853060in}}%
\pgfpathlineto{\pgfqpoint{1.184180in}{4.853060in}}%
\pgfusepath{stroke}%
\end{pgfscope}%
\begin{pgfscope}%
\definecolor{textcolor}{rgb}{0.000000,0.000000,0.000000}%
\pgfsetstrokecolor{textcolor}%
\pgfsetfillcolor{textcolor}%
\pgftext[x=1.295291in,y=4.804449in,left,base]{\color{textcolor}\rmfamily\fontsize{10.000000}{12.000000}\selectfont Finished migration}%
\end{pgfscope}%
\begin{pgfscope}%
\pgfsetbuttcap%
\pgfsetmiterjoin%
\definecolor{currentfill}{rgb}{0.121569,0.466667,0.705882}%
\pgfsetfillcolor{currentfill}%
\pgfsetlinewidth{0.000000pt}%
\definecolor{currentstroke}{rgb}{0.000000,0.000000,0.000000}%
\pgfsetstrokecolor{currentstroke}%
\pgfsetstrokeopacity{0.000000}%
\pgfsetdash{}{0pt}%
\pgfpathmoveto{\pgfqpoint{0.906402in}{4.610776in}}%
\pgfpathlineto{\pgfqpoint{1.184180in}{4.610776in}}%
\pgfpathlineto{\pgfqpoint{1.184180in}{4.707998in}}%
\pgfpathlineto{\pgfqpoint{0.906402in}{4.707998in}}%
\pgfpathclose%
\pgfusepath{fill}%
\end{pgfscope}%
\begin{pgfscope}%
\definecolor{textcolor}{rgb}{0.000000,0.000000,0.000000}%
\pgfsetstrokecolor{textcolor}%
\pgfsetfillcolor{textcolor}%
\pgftext[x=1.295291in,y=4.610776in,left,base]{\color{textcolor}\rmfamily\fontsize{10.000000}{12.000000}\selectfont BF1 read at destination}%
\end{pgfscope}%
\begin{pgfscope}%
\pgfsetbuttcap%
\pgfsetmiterjoin%
\definecolor{currentfill}{rgb}{1.000000,0.498039,0.054902}%
\pgfsetfillcolor{currentfill}%
\pgfsetlinewidth{0.000000pt}%
\definecolor{currentstroke}{rgb}{0.000000,0.000000,0.000000}%
\pgfsetstrokecolor{currentstroke}%
\pgfsetstrokeopacity{0.000000}%
\pgfsetdash{}{0pt}%
\pgfpathmoveto{\pgfqpoint{0.906402in}{4.417103in}}%
\pgfpathlineto{\pgfqpoint{1.184180in}{4.417103in}}%
\pgfpathlineto{\pgfqpoint{1.184180in}{4.514326in}}%
\pgfpathlineto{\pgfqpoint{0.906402in}{4.514326in}}%
\pgfpathclose%
\pgfusepath{fill}%
\end{pgfscope}%
\begin{pgfscope}%
\definecolor{textcolor}{rgb}{0.000000,0.000000,0.000000}%
\pgfsetstrokecolor{textcolor}%
\pgfsetfillcolor{textcolor}%
\pgftext[x=1.295291in,y=4.417103in,left,base]{\color{textcolor}\rmfamily\fontsize{10.000000}{12.000000}\selectfont BF1 write at destination}%
\end{pgfscope}%
\begin{pgfscope}%
\pgfsetbuttcap%
\pgfsetmiterjoin%
\definecolor{currentfill}{rgb}{0.172549,0.627451,0.172549}%
\pgfsetfillcolor{currentfill}%
\pgfsetlinewidth{0.000000pt}%
\definecolor{currentstroke}{rgb}{0.000000,0.000000,0.000000}%
\pgfsetstrokecolor{currentstroke}%
\pgfsetstrokeopacity{0.000000}%
\pgfsetdash{}{0pt}%
\pgfpathmoveto{\pgfqpoint{0.906402in}{4.223431in}}%
\pgfpathlineto{\pgfqpoint{1.184180in}{4.223431in}}%
\pgfpathlineto{\pgfqpoint{1.184180in}{4.320653in}}%
\pgfpathlineto{\pgfqpoint{0.906402in}{4.320653in}}%
\pgfpathclose%
\pgfusepath{fill}%
\end{pgfscope}%
\begin{pgfscope}%
\definecolor{textcolor}{rgb}{0.000000,0.000000,0.000000}%
\pgfsetstrokecolor{textcolor}%
\pgfsetfillcolor{textcolor}%
\pgftext[x=1.295291in,y=4.223431in,left,base]{\color{textcolor}\rmfamily\fontsize{10.000000}{12.000000}\selectfont BF2 read at source}%
\end{pgfscope}%
\begin{pgfscope}%
\pgfsetbuttcap%
\pgfsetmiterjoin%
\definecolor{currentfill}{rgb}{0.839216,0.152941,0.156863}%
\pgfsetfillcolor{currentfill}%
\pgfsetlinewidth{0.000000pt}%
\definecolor{currentstroke}{rgb}{0.000000,0.000000,0.000000}%
\pgfsetstrokecolor{currentstroke}%
\pgfsetstrokeopacity{0.000000}%
\pgfsetdash{}{0pt}%
\pgfpathmoveto{\pgfqpoint{0.906402in}{4.029758in}}%
\pgfpathlineto{\pgfqpoint{1.184180in}{4.029758in}}%
\pgfpathlineto{\pgfqpoint{1.184180in}{4.126980in}}%
\pgfpathlineto{\pgfqpoint{0.906402in}{4.126980in}}%
\pgfpathclose%
\pgfusepath{fill}%
\end{pgfscope}%
\begin{pgfscope}%
\definecolor{textcolor}{rgb}{0.000000,0.000000,0.000000}%
\pgfsetstrokecolor{textcolor}%
\pgfsetfillcolor{textcolor}%
\pgftext[x=1.295291in,y=4.029758in,left,base]{\color{textcolor}\rmfamily\fontsize{10.000000}{12.000000}\selectfont BF2 write at source}%
\end{pgfscope}%
\begin{pgfscope}%
\pgfsetbuttcap%
\pgfsetmiterjoin%
\definecolor{currentfill}{rgb}{0.580392,0.403922,0.741176}%
\pgfsetfillcolor{currentfill}%
\pgfsetlinewidth{0.000000pt}%
\definecolor{currentstroke}{rgb}{0.000000,0.000000,0.000000}%
\pgfsetstrokecolor{currentstroke}%
\pgfsetstrokeopacity{0.000000}%
\pgfsetdash{}{0pt}%
\pgfpathmoveto{\pgfqpoint{0.906402in}{3.836085in}}%
\pgfpathlineto{\pgfqpoint{1.184180in}{3.836085in}}%
\pgfpathlineto{\pgfqpoint{1.184180in}{3.933307in}}%
\pgfpathlineto{\pgfqpoint{0.906402in}{3.933307in}}%
\pgfpathclose%
\pgfusepath{fill}%
\end{pgfscope}%
\begin{pgfscope}%
\definecolor{textcolor}{rgb}{0.000000,0.000000,0.000000}%
\pgfsetstrokecolor{textcolor}%
\pgfsetfillcolor{textcolor}%
\pgftext[x=1.295291in,y=3.836085in,left,base]{\color{textcolor}\rmfamily\fontsize{10.000000}{12.000000}\selectfont BF1 read at source}%
\end{pgfscope}%
\begin{pgfscope}%
\pgfsetbuttcap%
\pgfsetmiterjoin%
\definecolor{currentfill}{rgb}{0.549020,0.337255,0.294118}%
\pgfsetfillcolor{currentfill}%
\pgfsetlinewidth{0.000000pt}%
\definecolor{currentstroke}{rgb}{0.000000,0.000000,0.000000}%
\pgfsetstrokecolor{currentstroke}%
\pgfsetstrokeopacity{0.000000}%
\pgfsetdash{}{0pt}%
\pgfpathmoveto{\pgfqpoint{0.906402in}{3.642412in}}%
\pgfpathlineto{\pgfqpoint{1.184180in}{3.642412in}}%
\pgfpathlineto{\pgfqpoint{1.184180in}{3.739634in}}%
\pgfpathlineto{\pgfqpoint{0.906402in}{3.739634in}}%
\pgfpathclose%
\pgfusepath{fill}%
\end{pgfscope}%
\begin{pgfscope}%
\definecolor{textcolor}{rgb}{0.000000,0.000000,0.000000}%
\pgfsetstrokecolor{textcolor}%
\pgfsetfillcolor{textcolor}%
\pgftext[x=1.295291in,y=3.642412in,left,base]{\color{textcolor}\rmfamily\fontsize{10.000000}{12.000000}\selectfont BF1 write at source}%
\end{pgfscope}%
\end{pgfpicture}%
\makeatother%
\endgroup%

    \end{center}
    \caption{Migration timeline of a bloom filter (2MB huge pages)}
    \label{fig:bloomfilterhp}
\end{figure}


In this section we discuss the case for migrating a bloom filter while it is
being queried. A bloom filter is a migration friendly object because
no operation on the bloom filter causes a memory allocation after we initialize
the object. The supported operations (put, get) only read or modify memory
locations.

At the start of the benchmark, two bloom filter objects, BF1
and BF2, are on the source machine. There is no bloom filter on the destination,
making the source machine overloaded in this scenario. On each machine there
are two threads, a reader and a writer, each of which uniformly chooses among
the available bloom filter objects available on the local machine to send their
respective query (get, put) to. In this benchmark the size of each of the bloom
filters is close to 800MB.

\autoref{fig:bloomfilter} shows the timeline of the events and the throughput
of each of the operations on the two bloom filter objects present in the system
on either of the machines as we migrate BF1. We are using 4KB pages in this
case and BF1 consists of 200000 pages. Around 60\% of those were dirtied during
the migration.

During the prefill period (~2200ms to ~7000ms), both objects see a decrease in
the number of writes. As the writer thread gets blocked inside the signal
handler while writing to the BF1 object, both BF1 and BF2 writes are equally
impacted since they equally share the writer thread. On the other hand, read
throughput on both objects stay the same as the reader thread never gets
blocked.

At around 7000ms we take away the write access to BF1 from the source, but
reads continue unimpacted until around 8700ms, where we take away the read
access too. At 8700ms the final transfer phase starts and until 9000ms,
the destination prepares its memory locations by calling into the operating
system and the memory allocator for each page that is being transferred.

At around 9300ms, the first read and write operations on the destination
succeed as the transfer of underlying memory for outstanding read/write
operations is prioritized. At around 10700ms, the migration is complete and
each of the BF objects is on a separate machine, doubling the overall
throughput.


\autoref{fig:bloomfilterhp} shows the same timeline in the case of using
huge pages. 383 huge pages where transferred and 256 of them were dirtied
by writes during the prefill phase. The overall flow of operations is similar
to the previous case, but
the application enjoys much shorter times across all metrics including the
end to end latency and read and write unavailability periods.

\section{Case study: Hash table partition}
\label{sec:evalgenericobj}
In this benchmark we discuss the case of migrating a hash table partition. We
use \texttt{std::map} which is migration unfriendly because of the possibility
of frequent memory allocations.


\begin{figure}[tp]
    \begin{center}
        %% Creator: Matplotlib, PGF backend
%%
%% To include the figure in your LaTeX document, write
%%   \input{<filename>.pgf}
%%
%% Make sure the required packages are loaded in your preamble
%%   \usepackage{pgf}
%%
%% and, on pdftex
%%   \usepackage[utf8]{inputenc}\DeclareUnicodeCharacter{2212}{-}
%%
%% or, on luatex and xetex
%%   \usepackage{unicode-math}
%%
%% Figures using additional raster images can only be included by \input if
%% they are in the same directory as the main LaTeX file. For loading figures
%% from other directories you can use the `import` package
%%   \usepackage{import}
%%
%% and then include the figures with
%%   \import{<path to file>}{<filename>.pgf}
%%
%% Matplotlib used the following preamble
%%
\begingroup%
\makeatletter%
\begin{pgfpicture}%
\pgfpathrectangle{\pgfpointorigin}{\pgfqpoint{6.251220in}{7.032623in}}%
\pgfusepath{use as bounding box, clip}%
\begin{pgfscope}%
\pgfsetbuttcap%
\pgfsetmiterjoin%
\definecolor{currentfill}{rgb}{1.000000,1.000000,1.000000}%
\pgfsetfillcolor{currentfill}%
\pgfsetlinewidth{0.000000pt}%
\definecolor{currentstroke}{rgb}{1.000000,1.000000,1.000000}%
\pgfsetstrokecolor{currentstroke}%
\pgfsetdash{}{0pt}%
\pgfpathmoveto{\pgfqpoint{0.000000in}{0.000000in}}%
\pgfpathlineto{\pgfqpoint{6.251220in}{0.000000in}}%
\pgfpathlineto{\pgfqpoint{6.251220in}{7.032623in}}%
\pgfpathlineto{\pgfqpoint{0.000000in}{7.032623in}}%
\pgfpathclose%
\pgfusepath{fill}%
\end{pgfscope}%
\begin{pgfscope}%
\pgfsetbuttcap%
\pgfsetmiterjoin%
\definecolor{currentfill}{rgb}{1.000000,1.000000,1.000000}%
\pgfsetfillcolor{currentfill}%
\pgfsetlinewidth{0.000000pt}%
\definecolor{currentstroke}{rgb}{0.000000,0.000000,0.000000}%
\pgfsetstrokecolor{currentstroke}%
\pgfsetstrokeopacity{0.000000}%
\pgfsetdash{}{0pt}%
\pgfpathmoveto{\pgfqpoint{0.781402in}{0.773588in}}%
\pgfpathlineto{\pgfqpoint{2.221647in}{0.773588in}}%
\pgfpathlineto{\pgfqpoint{2.221647in}{6.188708in}}%
\pgfpathlineto{\pgfqpoint{0.781402in}{6.188708in}}%
\pgfpathclose%
\pgfusepath{fill}%
\end{pgfscope}%
\begin{pgfscope}%
\pgfpathrectangle{\pgfqpoint{0.781402in}{0.773588in}}{\pgfqpoint{1.440244in}{5.415119in}}%
\pgfusepath{clip}%
\pgfsetbuttcap%
\pgfsetroundjoin%
\definecolor{currentfill}{rgb}{0.121569,0.466667,0.705882}%
\pgfsetfillcolor{currentfill}%
\pgfsetlinewidth{0.000000pt}%
\definecolor{currentstroke}{rgb}{0.000000,0.000000,0.000000}%
\pgfsetstrokecolor{currentstroke}%
\pgfsetdash{}{0pt}%
\pgfpathmoveto{\pgfqpoint{0.797895in}{0.773588in}}%
\pgfpathlineto{\pgfqpoint{0.797895in}{0.773588in}}%
\pgfpathlineto{\pgfqpoint{0.842612in}{0.773588in}}%
\pgfpathlineto{\pgfqpoint{0.887244in}{0.773588in}}%
\pgfpathlineto{\pgfqpoint{0.933783in}{0.773588in}}%
\pgfpathlineto{\pgfqpoint{0.978015in}{0.773588in}}%
\pgfpathlineto{\pgfqpoint{1.021908in}{0.773588in}}%
\pgfpathlineto{\pgfqpoint{1.067773in}{0.773588in}}%
\pgfpathlineto{\pgfqpoint{1.111790in}{0.773588in}}%
\pgfpathlineto{\pgfqpoint{1.155171in}{0.773588in}}%
\pgfpathlineto{\pgfqpoint{1.200192in}{0.773588in}}%
\pgfpathlineto{\pgfqpoint{1.244609in}{0.773588in}}%
\pgfpathlineto{\pgfqpoint{1.289074in}{0.773588in}}%
\pgfpathlineto{\pgfqpoint{1.334262in}{0.773588in}}%
\pgfpathlineto{\pgfqpoint{1.378287in}{0.773588in}}%
\pgfpathlineto{\pgfqpoint{1.422957in}{0.773588in}}%
\pgfpathlineto{\pgfqpoint{1.468334in}{0.773588in}}%
\pgfpathlineto{\pgfqpoint{1.511740in}{0.773588in}}%
\pgfpathlineto{\pgfqpoint{1.555718in}{0.773588in}}%
\pgfpathlineto{\pgfqpoint{1.600816in}{0.773588in}}%
\pgfpathlineto{\pgfqpoint{1.645274in}{0.773588in}}%
\pgfpathlineto{\pgfqpoint{1.689326in}{0.773588in}}%
\pgfpathlineto{\pgfqpoint{1.734500in}{0.773588in}}%
\pgfpathlineto{\pgfqpoint{1.778099in}{0.773588in}}%
\pgfpathlineto{\pgfqpoint{1.822804in}{0.773588in}}%
\pgfpathlineto{\pgfqpoint{1.869493in}{0.773588in}}%
\pgfpathlineto{\pgfqpoint{1.915908in}{0.773588in}}%
\pgfpathlineto{\pgfqpoint{1.963476in}{0.773588in}}%
\pgfpathlineto{\pgfqpoint{2.011209in}{0.773588in}}%
\pgfpathlineto{\pgfqpoint{2.057441in}{0.773588in}}%
\pgfpathlineto{\pgfqpoint{2.104325in}{0.773588in}}%
\pgfpathlineto{\pgfqpoint{2.153284in}{0.773588in}}%
\pgfpathlineto{\pgfqpoint{2.201242in}{0.773588in}}%
\pgfpathlineto{\pgfqpoint{2.249804in}{0.773588in}}%
\pgfpathlineto{\pgfqpoint{2.299591in}{0.773588in}}%
\pgfpathlineto{\pgfqpoint{2.348431in}{0.773588in}}%
\pgfpathlineto{\pgfqpoint{2.397147in}{0.773588in}}%
\pgfpathlineto{\pgfqpoint{2.452591in}{0.773588in}}%
\pgfpathlineto{\pgfqpoint{2.505534in}{0.773588in}}%
\pgfpathlineto{\pgfqpoint{2.555498in}{0.773588in}}%
\pgfpathlineto{\pgfqpoint{2.604306in}{0.773588in}}%
\pgfpathlineto{\pgfqpoint{2.651003in}{0.773588in}}%
\pgfpathlineto{\pgfqpoint{2.698034in}{0.773588in}}%
\pgfpathlineto{\pgfqpoint{2.745668in}{0.773588in}}%
\pgfpathlineto{\pgfqpoint{2.790736in}{0.773588in}}%
\pgfpathlineto{\pgfqpoint{2.836381in}{0.773588in}}%
\pgfpathlineto{\pgfqpoint{2.883134in}{0.773588in}}%
\pgfpathlineto{\pgfqpoint{2.927413in}{0.773588in}}%
\pgfpathlineto{\pgfqpoint{2.971917in}{0.773588in}}%
\pgfpathlineto{\pgfqpoint{3.017486in}{0.773588in}}%
\pgfpathlineto{\pgfqpoint{3.062761in}{0.773588in}}%
\pgfpathlineto{\pgfqpoint{3.108180in}{0.773588in}}%
\pgfpathlineto{\pgfqpoint{3.155580in}{0.773588in}}%
\pgfpathlineto{\pgfqpoint{3.200519in}{0.773588in}}%
\pgfpathlineto{\pgfqpoint{3.245458in}{0.773588in}}%
\pgfpathlineto{\pgfqpoint{3.290748in}{0.773588in}}%
\pgfpathlineto{\pgfqpoint{3.336184in}{0.773588in}}%
\pgfpathlineto{\pgfqpoint{3.381274in}{0.773588in}}%
\pgfpathlineto{\pgfqpoint{3.428526in}{0.773588in}}%
\pgfpathlineto{\pgfqpoint{3.473135in}{0.773588in}}%
\pgfpathlineto{\pgfqpoint{3.517815in}{0.773588in}}%
\pgfpathlineto{\pgfqpoint{3.564489in}{0.773588in}}%
\pgfpathlineto{\pgfqpoint{3.609556in}{0.773588in}}%
\pgfpathlineto{\pgfqpoint{3.654909in}{0.773588in}}%
\pgfpathlineto{\pgfqpoint{3.701949in}{0.773588in}}%
\pgfpathlineto{\pgfqpoint{3.747249in}{0.773588in}}%
\pgfpathlineto{\pgfqpoint{3.793040in}{0.773588in}}%
\pgfpathlineto{\pgfqpoint{3.839775in}{0.773588in}}%
\pgfpathlineto{\pgfqpoint{3.884614in}{0.773588in}}%
\pgfpathlineto{\pgfqpoint{3.930307in}{0.773588in}}%
\pgfpathlineto{\pgfqpoint{3.976868in}{0.773588in}}%
\pgfpathlineto{\pgfqpoint{4.022115in}{0.773588in}}%
\pgfpathlineto{\pgfqpoint{4.067507in}{0.773588in}}%
\pgfpathlineto{\pgfqpoint{4.114362in}{0.773588in}}%
\pgfpathlineto{\pgfqpoint{4.159078in}{0.773588in}}%
\pgfpathlineto{\pgfqpoint{4.204543in}{0.773588in}}%
\pgfpathlineto{\pgfqpoint{4.252389in}{0.773588in}}%
\pgfpathlineto{\pgfqpoint{4.297354in}{0.773588in}}%
\pgfpathlineto{\pgfqpoint{4.343118in}{0.773588in}}%
\pgfpathlineto{\pgfqpoint{4.390187in}{0.773588in}}%
\pgfpathlineto{\pgfqpoint{4.435284in}{0.773588in}}%
\pgfpathlineto{\pgfqpoint{4.480684in}{0.773588in}}%
\pgfpathlineto{\pgfqpoint{4.527424in}{0.773588in}}%
\pgfpathlineto{\pgfqpoint{4.572708in}{0.773588in}}%
\pgfpathlineto{\pgfqpoint{4.617918in}{0.773588in}}%
\pgfpathlineto{\pgfqpoint{4.663622in}{0.773588in}}%
\pgfpathlineto{\pgfqpoint{4.708228in}{0.773588in}}%
\pgfpathlineto{\pgfqpoint{4.753449in}{0.773588in}}%
\pgfpathlineto{\pgfqpoint{4.799887in}{0.773588in}}%
\pgfpathlineto{\pgfqpoint{4.845429in}{0.773588in}}%
\pgfpathlineto{\pgfqpoint{4.891298in}{0.773588in}}%
\pgfpathlineto{\pgfqpoint{4.939032in}{0.773588in}}%
\pgfpathlineto{\pgfqpoint{4.984973in}{0.773588in}}%
\pgfpathlineto{\pgfqpoint{5.030678in}{0.773588in}}%
\pgfpathlineto{\pgfqpoint{5.078074in}{0.773588in}}%
\pgfpathlineto{\pgfqpoint{5.124036in}{0.773588in}}%
\pgfpathlineto{\pgfqpoint{5.169192in}{0.773588in}}%
\pgfpathlineto{\pgfqpoint{5.216291in}{0.773588in}}%
\pgfpathlineto{\pgfqpoint{5.262146in}{0.773588in}}%
\pgfpathlineto{\pgfqpoint{5.307244in}{0.773588in}}%
\pgfpathlineto{\pgfqpoint{5.353614in}{0.773588in}}%
\pgfpathlineto{\pgfqpoint{5.399334in}{0.773588in}}%
\pgfpathlineto{\pgfqpoint{5.444526in}{0.773588in}}%
\pgfpathlineto{\pgfqpoint{5.491128in}{0.773588in}}%
\pgfpathlineto{\pgfqpoint{5.536203in}{0.773588in}}%
\pgfpathlineto{\pgfqpoint{5.581803in}{0.773588in}}%
\pgfpathlineto{\pgfqpoint{5.628857in}{0.773588in}}%
\pgfpathlineto{\pgfqpoint{5.674262in}{0.773588in}}%
\pgfpathlineto{\pgfqpoint{5.719569in}{0.773588in}}%
\pgfpathlineto{\pgfqpoint{5.765813in}{0.773588in}}%
\pgfpathlineto{\pgfqpoint{5.810269in}{0.773588in}}%
\pgfpathlineto{\pgfqpoint{5.856181in}{0.773588in}}%
\pgfpathlineto{\pgfqpoint{5.903366in}{0.773588in}}%
\pgfpathlineto{\pgfqpoint{5.948991in}{0.773588in}}%
\pgfpathlineto{\pgfqpoint{5.994686in}{0.773588in}}%
\pgfpathlineto{\pgfqpoint{6.041630in}{0.773588in}}%
\pgfpathlineto{\pgfqpoint{6.087229in}{0.773588in}}%
\pgfpathlineto{\pgfqpoint{6.133235in}{0.773588in}}%
\pgfpathlineto{\pgfqpoint{6.180802in}{0.773588in}}%
\pgfpathlineto{\pgfqpoint{6.227112in}{0.773588in}}%
\pgfpathlineto{\pgfqpoint{6.273584in}{0.773588in}}%
\pgfpathlineto{\pgfqpoint{6.321070in}{0.773588in}}%
\pgfpathlineto{\pgfqpoint{6.367508in}{0.773588in}}%
\pgfpathlineto{\pgfqpoint{6.413399in}{0.773588in}}%
\pgfpathlineto{\pgfqpoint{6.461324in}{0.773588in}}%
\pgfpathlineto{\pgfqpoint{6.507651in}{0.773588in}}%
\pgfpathlineto{\pgfqpoint{6.553117in}{0.773588in}}%
\pgfpathlineto{\pgfqpoint{6.599302in}{0.773588in}}%
\pgfpathlineto{\pgfqpoint{6.643960in}{0.773588in}}%
\pgfpathlineto{\pgfqpoint{6.688504in}{0.773588in}}%
\pgfpathlineto{\pgfqpoint{6.734887in}{0.773588in}}%
\pgfpathlineto{\pgfqpoint{6.779295in}{0.773588in}}%
\pgfpathlineto{\pgfqpoint{6.824012in}{0.773588in}}%
\pgfpathlineto{\pgfqpoint{6.869544in}{0.773588in}}%
\pgfpathlineto{\pgfqpoint{6.914194in}{0.773588in}}%
\pgfpathlineto{\pgfqpoint{6.958763in}{0.773588in}}%
\pgfpathlineto{\pgfqpoint{7.005149in}{0.773588in}}%
\pgfpathlineto{\pgfqpoint{7.050071in}{0.773588in}}%
\pgfpathlineto{\pgfqpoint{7.094205in}{0.773588in}}%
\pgfpathlineto{\pgfqpoint{7.140134in}{0.773588in}}%
\pgfpathlineto{\pgfqpoint{7.184277in}{0.773588in}}%
\pgfpathlineto{\pgfqpoint{7.228120in}{0.773588in}}%
\pgfpathlineto{\pgfqpoint{7.273914in}{0.773588in}}%
\pgfpathlineto{\pgfqpoint{7.318484in}{0.773588in}}%
\pgfpathlineto{\pgfqpoint{7.363749in}{0.773588in}}%
\pgfpathlineto{\pgfqpoint{7.409977in}{0.773588in}}%
\pgfpathlineto{\pgfqpoint{7.455548in}{0.773588in}}%
\pgfpathlineto{\pgfqpoint{7.500683in}{0.773588in}}%
\pgfpathlineto{\pgfqpoint{7.546892in}{0.773588in}}%
\pgfpathlineto{\pgfqpoint{7.591890in}{0.773588in}}%
\pgfpathlineto{\pgfqpoint{7.636592in}{0.773588in}}%
\pgfpathlineto{\pgfqpoint{7.682978in}{0.773588in}}%
\pgfpathlineto{\pgfqpoint{7.728803in}{0.773588in}}%
\pgfpathlineto{\pgfqpoint{7.773226in}{0.773588in}}%
\pgfpathlineto{\pgfqpoint{7.819947in}{0.773588in}}%
\pgfpathlineto{\pgfqpoint{7.865263in}{0.773588in}}%
\pgfpathlineto{\pgfqpoint{7.910161in}{0.773588in}}%
\pgfpathlineto{\pgfqpoint{7.955879in}{0.773588in}}%
\pgfpathlineto{\pgfqpoint{8.000573in}{0.773588in}}%
\pgfpathlineto{\pgfqpoint{8.045278in}{0.773588in}}%
\pgfpathlineto{\pgfqpoint{8.091881in}{0.773588in}}%
\pgfpathlineto{\pgfqpoint{8.136842in}{0.773588in}}%
\pgfpathlineto{\pgfqpoint{8.181791in}{0.773588in}}%
\pgfpathlineto{\pgfqpoint{8.227930in}{0.773588in}}%
\pgfpathlineto{\pgfqpoint{8.273006in}{0.773588in}}%
\pgfpathlineto{\pgfqpoint{8.318789in}{0.773588in}}%
\pgfpathlineto{\pgfqpoint{8.364636in}{0.773588in}}%
\pgfpathlineto{\pgfqpoint{8.409767in}{0.773588in}}%
\pgfpathlineto{\pgfqpoint{8.453780in}{0.773588in}}%
\pgfpathlineto{\pgfqpoint{8.499298in}{0.773588in}}%
\pgfpathlineto{\pgfqpoint{8.543976in}{0.773588in}}%
\pgfpathlineto{\pgfqpoint{8.589124in}{0.773588in}}%
\pgfpathlineto{\pgfqpoint{8.635315in}{0.773588in}}%
\pgfpathlineto{\pgfqpoint{8.681080in}{0.773588in}}%
\pgfpathlineto{\pgfqpoint{8.727404in}{0.773588in}}%
\pgfpathlineto{\pgfqpoint{8.774902in}{0.773588in}}%
\pgfpathlineto{\pgfqpoint{8.821068in}{0.773588in}}%
\pgfpathlineto{\pgfqpoint{8.866242in}{0.773588in}}%
\pgfpathlineto{\pgfqpoint{8.912131in}{0.773588in}}%
\pgfpathlineto{\pgfqpoint{8.957924in}{0.773588in}}%
\pgfpathlineto{\pgfqpoint{9.003947in}{0.773588in}}%
\pgfpathlineto{\pgfqpoint{9.050165in}{0.773588in}}%
\pgfpathlineto{\pgfqpoint{9.095727in}{0.773588in}}%
\pgfpathlineto{\pgfqpoint{9.141270in}{0.773588in}}%
\pgfpathlineto{\pgfqpoint{9.188332in}{0.773588in}}%
\pgfpathlineto{\pgfqpoint{9.233107in}{0.773588in}}%
\pgfpathlineto{\pgfqpoint{9.278439in}{0.773588in}}%
\pgfpathlineto{\pgfqpoint{9.324423in}{0.773588in}}%
\pgfpathlineto{\pgfqpoint{9.369639in}{0.773588in}}%
\pgfpathlineto{\pgfqpoint{9.415178in}{0.773588in}}%
\pgfpathlineto{\pgfqpoint{9.462006in}{0.773588in}}%
\pgfpathlineto{\pgfqpoint{9.507583in}{0.773588in}}%
\pgfpathlineto{\pgfqpoint{9.552197in}{0.773588in}}%
\pgfpathlineto{\pgfqpoint{9.598338in}{0.773588in}}%
\pgfpathlineto{\pgfqpoint{9.644121in}{0.773588in}}%
\pgfpathlineto{\pgfqpoint{9.689497in}{0.773588in}}%
\pgfpathlineto{\pgfqpoint{9.735785in}{0.773588in}}%
\pgfpathlineto{\pgfqpoint{9.781321in}{0.773588in}}%
\pgfpathlineto{\pgfqpoint{9.826599in}{0.773588in}}%
\pgfpathlineto{\pgfqpoint{9.873173in}{0.773588in}}%
\pgfpathlineto{\pgfqpoint{9.918842in}{0.773588in}}%
\pgfpathlineto{\pgfqpoint{9.964562in}{0.773588in}}%
\pgfpathlineto{\pgfqpoint{10.011915in}{0.773588in}}%
\pgfpathlineto{\pgfqpoint{10.057613in}{0.773588in}}%
\pgfpathlineto{\pgfqpoint{10.103098in}{0.773588in}}%
\pgfpathlineto{\pgfqpoint{10.151348in}{0.773588in}}%
\pgfpathlineto{\pgfqpoint{10.197247in}{0.773588in}}%
\pgfpathlineto{\pgfqpoint{10.242601in}{0.773588in}}%
\pgfpathlineto{\pgfqpoint{10.289084in}{0.773588in}}%
\pgfpathlineto{\pgfqpoint{10.335334in}{0.773588in}}%
\pgfpathlineto{\pgfqpoint{10.381570in}{0.773588in}}%
\pgfpathlineto{\pgfqpoint{10.429362in}{0.773588in}}%
\pgfpathlineto{\pgfqpoint{10.475642in}{0.773588in}}%
\pgfpathlineto{\pgfqpoint{10.521105in}{0.773588in}}%
\pgfpathlineto{\pgfqpoint{10.568154in}{0.773588in}}%
\pgfpathlineto{\pgfqpoint{10.613886in}{0.773588in}}%
\pgfpathlineto{\pgfqpoint{10.660112in}{0.773588in}}%
\pgfpathlineto{\pgfqpoint{10.707461in}{0.773588in}}%
\pgfpathlineto{\pgfqpoint{10.753168in}{0.773588in}}%
\pgfpathlineto{\pgfqpoint{10.798768in}{0.773588in}}%
\pgfpathlineto{\pgfqpoint{10.845790in}{0.773588in}}%
\pgfpathlineto{\pgfqpoint{10.891801in}{0.773588in}}%
\pgfpathlineto{\pgfqpoint{10.937381in}{0.773588in}}%
\pgfpathlineto{\pgfqpoint{10.984145in}{0.773588in}}%
\pgfpathlineto{\pgfqpoint{11.029398in}{0.773588in}}%
\pgfpathlineto{\pgfqpoint{11.074633in}{0.773588in}}%
\pgfpathlineto{\pgfqpoint{11.121978in}{0.773588in}}%
\pgfpathlineto{\pgfqpoint{11.167471in}{0.773588in}}%
\pgfpathlineto{\pgfqpoint{11.213355in}{0.773588in}}%
\pgfpathlineto{\pgfqpoint{11.261350in}{0.773588in}}%
\pgfpathlineto{\pgfqpoint{11.307361in}{0.773588in}}%
\pgfpathlineto{\pgfqpoint{11.353756in}{0.773588in}}%
\pgfpathlineto{\pgfqpoint{11.402044in}{0.773588in}}%
\pgfpathlineto{\pgfqpoint{11.448531in}{0.773588in}}%
\pgfpathlineto{\pgfqpoint{11.494874in}{0.773588in}}%
\pgfpathlineto{\pgfqpoint{11.542665in}{0.773588in}}%
\pgfpathlineto{\pgfqpoint{11.588799in}{0.773588in}}%
\pgfpathlineto{\pgfqpoint{11.634589in}{0.773588in}}%
\pgfpathlineto{\pgfqpoint{11.682053in}{0.773588in}}%
\pgfpathlineto{\pgfqpoint{11.727607in}{0.773588in}}%
\pgfpathlineto{\pgfqpoint{11.772894in}{0.773588in}}%
\pgfpathlineto{\pgfqpoint{11.819406in}{0.773588in}}%
\pgfpathlineto{\pgfqpoint{11.864862in}{0.773588in}}%
\pgfpathlineto{\pgfqpoint{11.911501in}{0.773588in}}%
\pgfpathlineto{\pgfqpoint{11.959143in}{0.773588in}}%
\pgfpathlineto{\pgfqpoint{12.004414in}{0.773588in}}%
\pgfpathlineto{\pgfqpoint{12.049869in}{0.773588in}}%
\pgfpathlineto{\pgfqpoint{12.096623in}{0.773588in}}%
\pgfpathlineto{\pgfqpoint{12.141887in}{0.773588in}}%
\pgfpathlineto{\pgfqpoint{12.187925in}{0.773588in}}%
\pgfpathlineto{\pgfqpoint{12.235985in}{0.773588in}}%
\pgfpathlineto{\pgfqpoint{12.282271in}{0.773588in}}%
\pgfpathlineto{\pgfqpoint{12.328136in}{0.773588in}}%
\pgfpathlineto{\pgfqpoint{12.375261in}{0.773588in}}%
\pgfpathlineto{\pgfqpoint{12.421372in}{0.773588in}}%
\pgfpathlineto{\pgfqpoint{12.467453in}{0.773588in}}%
\pgfpathlineto{\pgfqpoint{12.514481in}{0.773588in}}%
\pgfpathlineto{\pgfqpoint{12.560711in}{0.773588in}}%
\pgfpathlineto{\pgfqpoint{12.607615in}{0.773588in}}%
\pgfpathlineto{\pgfqpoint{12.655424in}{0.773588in}}%
\pgfpathlineto{\pgfqpoint{12.701835in}{0.773588in}}%
\pgfpathlineto{\pgfqpoint{12.749025in}{0.773588in}}%
\pgfpathlineto{\pgfqpoint{12.796907in}{0.773588in}}%
\pgfpathlineto{\pgfqpoint{12.843155in}{0.773588in}}%
\pgfpathlineto{\pgfqpoint{12.889988in}{0.773588in}}%
\pgfpathlineto{\pgfqpoint{12.936799in}{0.773588in}}%
\pgfpathlineto{\pgfqpoint{12.983596in}{0.773588in}}%
\pgfpathlineto{\pgfqpoint{13.030499in}{0.773588in}}%
\pgfpathlineto{\pgfqpoint{13.077275in}{0.773588in}}%
\pgfpathlineto{\pgfqpoint{13.123339in}{0.773588in}}%
\pgfpathlineto{\pgfqpoint{13.169075in}{0.773588in}}%
\pgfpathlineto{\pgfqpoint{13.216719in}{0.773588in}}%
\pgfpathlineto{\pgfqpoint{13.262980in}{0.773588in}}%
\pgfpathlineto{\pgfqpoint{13.308974in}{0.773588in}}%
\pgfpathlineto{\pgfqpoint{13.356518in}{0.773588in}}%
\pgfpathlineto{\pgfqpoint{13.401859in}{0.773588in}}%
\pgfpathlineto{\pgfqpoint{13.448069in}{0.773588in}}%
\pgfpathlineto{\pgfqpoint{13.495727in}{0.773588in}}%
\pgfpathlineto{\pgfqpoint{13.541646in}{0.773588in}}%
\pgfpathlineto{\pgfqpoint{13.587249in}{0.773588in}}%
\pgfpathlineto{\pgfqpoint{13.635346in}{0.773588in}}%
\pgfpathlineto{\pgfqpoint{13.682151in}{0.773588in}}%
\pgfpathlineto{\pgfqpoint{13.729143in}{0.773588in}}%
\pgfpathlineto{\pgfqpoint{13.777441in}{0.773588in}}%
\pgfpathlineto{\pgfqpoint{13.825297in}{0.773588in}}%
\pgfpathlineto{\pgfqpoint{13.873610in}{0.773588in}}%
\pgfpathlineto{\pgfqpoint{13.922606in}{0.773588in}}%
\pgfpathlineto{\pgfqpoint{13.969353in}{0.773588in}}%
\pgfpathlineto{\pgfqpoint{14.016344in}{0.773588in}}%
\pgfpathlineto{\pgfqpoint{14.064951in}{0.773588in}}%
\pgfpathlineto{\pgfqpoint{14.112147in}{0.773588in}}%
\pgfpathlineto{\pgfqpoint{14.160147in}{0.773588in}}%
\pgfpathlineto{\pgfqpoint{14.209376in}{0.773588in}}%
\pgfpathlineto{\pgfqpoint{14.255593in}{0.773588in}}%
\pgfpathlineto{\pgfqpoint{14.302418in}{0.773588in}}%
\pgfpathlineto{\pgfqpoint{14.350224in}{0.773588in}}%
\pgfpathlineto{\pgfqpoint{14.397126in}{0.773588in}}%
\pgfpathlineto{\pgfqpoint{14.443729in}{0.773588in}}%
\pgfpathlineto{\pgfqpoint{14.491726in}{0.773588in}}%
\pgfpathlineto{\pgfqpoint{14.538630in}{0.773588in}}%
\pgfpathlineto{\pgfqpoint{14.585477in}{0.773588in}}%
\pgfpathlineto{\pgfqpoint{14.633992in}{0.773588in}}%
\pgfpathlineto{\pgfqpoint{14.680226in}{0.773588in}}%
\pgfpathlineto{\pgfqpoint{14.726757in}{0.773588in}}%
\pgfpathlineto{\pgfqpoint{14.774752in}{0.773588in}}%
\pgfpathlineto{\pgfqpoint{14.821474in}{0.773588in}}%
\pgfpathlineto{\pgfqpoint{14.868242in}{0.773588in}}%
\pgfpathlineto{\pgfqpoint{14.916239in}{0.773588in}}%
\pgfpathlineto{\pgfqpoint{14.963167in}{0.773588in}}%
\pgfpathlineto{\pgfqpoint{15.010214in}{0.773588in}}%
\pgfpathlineto{\pgfqpoint{15.058774in}{0.773588in}}%
\pgfpathlineto{\pgfqpoint{15.105478in}{0.773588in}}%
\pgfpathlineto{\pgfqpoint{15.152658in}{0.773588in}}%
\pgfpathlineto{\pgfqpoint{15.201456in}{0.773588in}}%
\pgfpathlineto{\pgfqpoint{15.248863in}{0.773588in}}%
\pgfpathlineto{\pgfqpoint{15.295866in}{0.773588in}}%
\pgfpathlineto{\pgfqpoint{15.343883in}{0.773588in}}%
\pgfpathlineto{\pgfqpoint{15.389896in}{0.773588in}}%
\pgfpathlineto{\pgfqpoint{15.436052in}{0.773588in}}%
\pgfpathlineto{\pgfqpoint{15.484858in}{0.773588in}}%
\pgfpathlineto{\pgfqpoint{15.532329in}{0.773588in}}%
\pgfpathlineto{\pgfqpoint{15.579815in}{0.773588in}}%
\pgfpathlineto{\pgfqpoint{15.628660in}{0.773588in}}%
\pgfpathlineto{\pgfqpoint{15.676019in}{0.773588in}}%
\pgfpathlineto{\pgfqpoint{15.722814in}{0.773588in}}%
\pgfpathlineto{\pgfqpoint{15.770154in}{0.773588in}}%
\pgfpathlineto{\pgfqpoint{15.817273in}{0.773588in}}%
\pgfpathlineto{\pgfqpoint{15.863592in}{0.773588in}}%
\pgfpathlineto{\pgfqpoint{15.911811in}{0.773588in}}%
\pgfpathlineto{\pgfqpoint{15.957799in}{0.773588in}}%
\pgfpathlineto{\pgfqpoint{16.003926in}{0.773588in}}%
\pgfpathlineto{\pgfqpoint{16.051820in}{0.773588in}}%
\pgfpathlineto{\pgfqpoint{16.098697in}{0.773588in}}%
\pgfpathlineto{\pgfqpoint{16.146338in}{0.773588in}}%
\pgfpathlineto{\pgfqpoint{16.195175in}{0.773588in}}%
\pgfpathlineto{\pgfqpoint{16.242078in}{0.773588in}}%
\pgfpathlineto{\pgfqpoint{16.289817in}{0.773588in}}%
\pgfpathlineto{\pgfqpoint{16.339537in}{0.773588in}}%
\pgfpathlineto{\pgfqpoint{16.387666in}{0.773588in}}%
\pgfpathlineto{\pgfqpoint{16.436622in}{0.773588in}}%
\pgfpathlineto{\pgfqpoint{16.486560in}{0.773588in}}%
\pgfpathlineto{\pgfqpoint{16.533619in}{0.773588in}}%
\pgfpathlineto{\pgfqpoint{16.581135in}{0.773588in}}%
\pgfpathlineto{\pgfqpoint{16.630280in}{0.773588in}}%
\pgfpathlineto{\pgfqpoint{16.677566in}{0.773588in}}%
\pgfpathlineto{\pgfqpoint{16.724568in}{0.773588in}}%
\pgfpathlineto{\pgfqpoint{16.773713in}{0.773588in}}%
\pgfpathlineto{\pgfqpoint{16.821038in}{0.773588in}}%
\pgfpathlineto{\pgfqpoint{16.868266in}{0.773588in}}%
\pgfpathlineto{\pgfqpoint{16.916302in}{0.773588in}}%
\pgfpathlineto{\pgfqpoint{16.963533in}{0.773588in}}%
\pgfpathlineto{\pgfqpoint{17.010585in}{0.773588in}}%
\pgfpathlineto{\pgfqpoint{17.059518in}{0.773588in}}%
\pgfpathlineto{\pgfqpoint{17.107470in}{0.773588in}}%
\pgfpathlineto{\pgfqpoint{17.154507in}{0.773588in}}%
\pgfpathlineto{\pgfqpoint{17.203674in}{0.773588in}}%
\pgfpathlineto{\pgfqpoint{17.250860in}{0.773588in}}%
\pgfpathlineto{\pgfqpoint{17.298941in}{0.773588in}}%
\pgfpathlineto{\pgfqpoint{17.347576in}{0.773588in}}%
\pgfpathlineto{\pgfqpoint{17.394714in}{0.773588in}}%
\pgfpathlineto{\pgfqpoint{17.442362in}{0.773588in}}%
\pgfpathlineto{\pgfqpoint{17.491125in}{0.773588in}}%
\pgfpathlineto{\pgfqpoint{17.538409in}{0.773588in}}%
\pgfpathlineto{\pgfqpoint{17.585742in}{0.773588in}}%
\pgfpathlineto{\pgfqpoint{17.634653in}{0.773588in}}%
\pgfpathlineto{\pgfqpoint{17.681914in}{0.773588in}}%
\pgfpathlineto{\pgfqpoint{17.729727in}{0.773588in}}%
\pgfpathlineto{\pgfqpoint{17.779014in}{0.773588in}}%
\pgfpathlineto{\pgfqpoint{17.826809in}{0.773588in}}%
\pgfpathlineto{\pgfqpoint{17.874600in}{0.773588in}}%
\pgfpathlineto{\pgfqpoint{17.922885in}{0.773588in}}%
\pgfpathlineto{\pgfqpoint{17.970910in}{0.773588in}}%
\pgfpathlineto{\pgfqpoint{18.020026in}{0.773588in}}%
\pgfpathlineto{\pgfqpoint{18.069524in}{0.773588in}}%
\pgfpathlineto{\pgfqpoint{18.117307in}{0.773588in}}%
\pgfpathlineto{\pgfqpoint{18.164405in}{0.773588in}}%
\pgfpathlineto{\pgfqpoint{18.213333in}{0.773588in}}%
\pgfpathlineto{\pgfqpoint{18.260871in}{0.773588in}}%
\pgfpathlineto{\pgfqpoint{18.308441in}{0.773588in}}%
\pgfpathlineto{\pgfqpoint{18.357538in}{0.773588in}}%
\pgfpathlineto{\pgfqpoint{18.405242in}{0.773588in}}%
\pgfpathlineto{\pgfqpoint{18.452284in}{0.773588in}}%
\pgfpathlineto{\pgfqpoint{18.500915in}{0.773588in}}%
\pgfpathlineto{\pgfqpoint{18.548460in}{0.773588in}}%
\pgfpathlineto{\pgfqpoint{18.595724in}{0.773588in}}%
\pgfpathlineto{\pgfqpoint{18.644639in}{0.773588in}}%
\pgfpathlineto{\pgfqpoint{18.693122in}{0.773588in}}%
\pgfpathlineto{\pgfqpoint{18.741665in}{0.773588in}}%
\pgfpathlineto{\pgfqpoint{18.791212in}{0.773588in}}%
\pgfpathlineto{\pgfqpoint{18.839312in}{0.773588in}}%
\pgfpathlineto{\pgfqpoint{18.888188in}{0.773588in}}%
\pgfpathlineto{\pgfqpoint{18.937671in}{0.773588in}}%
\pgfpathlineto{\pgfqpoint{18.985002in}{0.773588in}}%
\pgfpathlineto{\pgfqpoint{19.033794in}{0.773588in}}%
\pgfpathlineto{\pgfqpoint{19.084207in}{0.773588in}}%
\pgfpathlineto{\pgfqpoint{19.132724in}{0.773588in}}%
\pgfpathlineto{\pgfqpoint{19.181302in}{0.773588in}}%
\pgfpathlineto{\pgfqpoint{19.231826in}{0.773588in}}%
\pgfpathlineto{\pgfqpoint{19.279742in}{0.773588in}}%
\pgfpathlineto{\pgfqpoint{19.327829in}{0.773588in}}%
\pgfpathlineto{\pgfqpoint{19.376890in}{0.773588in}}%
\pgfpathlineto{\pgfqpoint{19.425178in}{0.773588in}}%
\pgfpathlineto{\pgfqpoint{19.473154in}{0.773588in}}%
\pgfpathlineto{\pgfqpoint{19.522292in}{0.773588in}}%
\pgfpathlineto{\pgfqpoint{19.569747in}{0.773588in}}%
\pgfpathlineto{\pgfqpoint{19.617707in}{0.773588in}}%
\pgfpathlineto{\pgfqpoint{19.666940in}{0.773588in}}%
\pgfpathlineto{\pgfqpoint{19.714265in}{0.773588in}}%
\pgfpathlineto{\pgfqpoint{19.761678in}{0.773588in}}%
\pgfpathlineto{\pgfqpoint{19.810712in}{0.773588in}}%
\pgfpathlineto{\pgfqpoint{19.858933in}{0.773588in}}%
\pgfpathlineto{\pgfqpoint{19.906615in}{0.773588in}}%
\pgfpathlineto{\pgfqpoint{19.955252in}{0.773588in}}%
\pgfpathlineto{\pgfqpoint{20.003692in}{0.773588in}}%
\pgfpathlineto{\pgfqpoint{20.053312in}{0.773588in}}%
\pgfpathlineto{\pgfqpoint{20.104350in}{0.773588in}}%
\pgfpathlineto{\pgfqpoint{20.153679in}{0.773588in}}%
\pgfpathlineto{\pgfqpoint{20.203615in}{0.773588in}}%
\pgfpathlineto{\pgfqpoint{20.255593in}{0.773588in}}%
\pgfpathlineto{\pgfqpoint{20.305924in}{0.773588in}}%
\pgfpathlineto{\pgfqpoint{20.355462in}{0.773588in}}%
\pgfpathlineto{\pgfqpoint{20.406414in}{0.773588in}}%
\pgfpathlineto{\pgfqpoint{20.455604in}{0.773588in}}%
\pgfpathlineto{\pgfqpoint{20.505208in}{0.773588in}}%
\pgfpathlineto{\pgfqpoint{20.556290in}{0.773588in}}%
\pgfpathlineto{\pgfqpoint{20.605567in}{0.773588in}}%
\pgfpathlineto{\pgfqpoint{20.654668in}{0.773588in}}%
\pgfpathlineto{\pgfqpoint{20.704616in}{0.773588in}}%
\pgfpathlineto{\pgfqpoint{20.753558in}{0.773588in}}%
\pgfpathlineto{\pgfqpoint{20.802965in}{0.773588in}}%
\pgfpathlineto{\pgfqpoint{20.854118in}{0.773588in}}%
\pgfpathlineto{\pgfqpoint{20.904164in}{0.773588in}}%
\pgfpathlineto{\pgfqpoint{20.953585in}{0.773588in}}%
\pgfpathlineto{\pgfqpoint{21.003693in}{0.773588in}}%
\pgfpathlineto{\pgfqpoint{21.053515in}{0.773588in}}%
\pgfpathlineto{\pgfqpoint{21.102370in}{0.773588in}}%
\pgfpathlineto{\pgfqpoint{21.152719in}{0.773588in}}%
\pgfpathlineto{\pgfqpoint{21.201683in}{0.773588in}}%
\pgfpathlineto{\pgfqpoint{21.250655in}{0.773588in}}%
\pgfpathlineto{\pgfqpoint{21.300955in}{0.773588in}}%
\pgfpathlineto{\pgfqpoint{21.350353in}{0.773588in}}%
\pgfpathlineto{\pgfqpoint{21.399733in}{0.773588in}}%
\pgfpathlineto{\pgfqpoint{21.450830in}{0.773588in}}%
\pgfpathlineto{\pgfqpoint{21.501023in}{0.773588in}}%
\pgfpathlineto{\pgfqpoint{21.550957in}{0.773588in}}%
\pgfpathlineto{\pgfqpoint{21.602538in}{0.773588in}}%
\pgfpathlineto{\pgfqpoint{21.652276in}{0.773588in}}%
\pgfpathlineto{\pgfqpoint{21.700864in}{0.773588in}}%
\pgfpathlineto{\pgfqpoint{21.751624in}{0.773588in}}%
\pgfpathlineto{\pgfqpoint{21.800474in}{0.773588in}}%
\pgfpathlineto{\pgfqpoint{21.849580in}{0.773588in}}%
\pgfpathlineto{\pgfqpoint{21.900834in}{0.773588in}}%
\pgfpathlineto{\pgfqpoint{21.950544in}{0.773588in}}%
\pgfpathlineto{\pgfqpoint{21.999803in}{0.773588in}}%
\pgfpathlineto{\pgfqpoint{22.050354in}{0.773588in}}%
\pgfpathlineto{\pgfqpoint{22.099907in}{0.773588in}}%
\pgfpathlineto{\pgfqpoint{22.148817in}{0.773588in}}%
\pgfpathlineto{\pgfqpoint{22.200587in}{0.773588in}}%
\pgfpathlineto{\pgfqpoint{22.250634in}{0.773588in}}%
\pgfpathlineto{\pgfqpoint{22.300933in}{0.773588in}}%
\pgfpathlineto{\pgfqpoint{22.352827in}{0.773588in}}%
\pgfpathlineto{\pgfqpoint{22.402084in}{0.773588in}}%
\pgfpathlineto{\pgfqpoint{22.451976in}{0.773588in}}%
\pgfpathlineto{\pgfqpoint{22.503552in}{0.773588in}}%
\pgfpathlineto{\pgfqpoint{22.554073in}{0.773588in}}%
\pgfpathlineto{\pgfqpoint{22.603657in}{0.773588in}}%
\pgfpathlineto{\pgfqpoint{22.654822in}{0.773588in}}%
\pgfpathlineto{\pgfqpoint{22.703398in}{0.773588in}}%
\pgfpathlineto{\pgfqpoint{22.751700in}{0.773588in}}%
\pgfpathlineto{\pgfqpoint{22.803829in}{0.773588in}}%
\pgfpathlineto{\pgfqpoint{22.853966in}{0.773588in}}%
\pgfpathlineto{\pgfqpoint{22.904236in}{0.773588in}}%
\pgfpathlineto{\pgfqpoint{22.956359in}{0.773588in}}%
\pgfpathlineto{\pgfqpoint{23.006022in}{0.773588in}}%
\pgfpathlineto{\pgfqpoint{23.055187in}{0.773588in}}%
\pgfpathlineto{\pgfqpoint{23.107134in}{0.773588in}}%
\pgfpathlineto{\pgfqpoint{23.157756in}{0.773588in}}%
\pgfpathlineto{\pgfqpoint{23.208694in}{0.773588in}}%
\pgfpathlineto{\pgfqpoint{23.260726in}{0.773588in}}%
\pgfpathlineto{\pgfqpoint{23.310918in}{0.773588in}}%
\pgfpathlineto{\pgfqpoint{23.361310in}{0.773588in}}%
\pgfpathlineto{\pgfqpoint{23.411909in}{0.773588in}}%
\pgfpathlineto{\pgfqpoint{23.462017in}{0.773588in}}%
\pgfpathlineto{\pgfqpoint{23.511867in}{0.773588in}}%
\pgfpathlineto{\pgfqpoint{23.563429in}{0.773588in}}%
\pgfpathlineto{\pgfqpoint{23.611923in}{0.773588in}}%
\pgfpathlineto{\pgfqpoint{23.661575in}{0.773588in}}%
\pgfpathlineto{\pgfqpoint{23.713074in}{0.773588in}}%
\pgfpathlineto{\pgfqpoint{23.762329in}{0.773588in}}%
\pgfpathlineto{\pgfqpoint{23.811512in}{0.773588in}}%
\pgfpathlineto{\pgfqpoint{23.862673in}{0.773588in}}%
\pgfpathlineto{\pgfqpoint{23.913088in}{0.773588in}}%
\pgfpathlineto{\pgfqpoint{23.964514in}{0.773588in}}%
\pgfpathlineto{\pgfqpoint{24.016442in}{0.773588in}}%
\pgfpathlineto{\pgfqpoint{24.066933in}{0.773588in}}%
\pgfpathlineto{\pgfqpoint{24.118637in}{0.773588in}}%
\pgfpathlineto{\pgfqpoint{24.170873in}{0.773588in}}%
\pgfpathlineto{\pgfqpoint{24.221010in}{0.773588in}}%
\pgfpathlineto{\pgfqpoint{24.271277in}{0.773588in}}%
\pgfpathlineto{\pgfqpoint{24.321907in}{0.773588in}}%
\pgfpathlineto{\pgfqpoint{24.371890in}{0.773588in}}%
\pgfpathlineto{\pgfqpoint{24.422475in}{0.773588in}}%
\pgfpathlineto{\pgfqpoint{24.474381in}{0.773588in}}%
\pgfpathlineto{\pgfqpoint{24.524258in}{0.773588in}}%
\pgfpathlineto{\pgfqpoint{24.573998in}{0.773588in}}%
\pgfpathlineto{\pgfqpoint{24.626556in}{0.773588in}}%
\pgfpathlineto{\pgfqpoint{24.676389in}{0.773588in}}%
\pgfpathlineto{\pgfqpoint{24.726499in}{0.773588in}}%
\pgfpathlineto{\pgfqpoint{24.778304in}{0.773588in}}%
\pgfpathlineto{\pgfqpoint{24.828140in}{0.773588in}}%
\pgfpathlineto{\pgfqpoint{24.878143in}{0.773588in}}%
\pgfpathlineto{\pgfqpoint{24.928869in}{0.773588in}}%
\pgfpathlineto{\pgfqpoint{24.978211in}{0.773588in}}%
\pgfpathlineto{\pgfqpoint{25.028630in}{0.773588in}}%
\pgfpathlineto{\pgfqpoint{25.080539in}{0.773588in}}%
\pgfpathlineto{\pgfqpoint{25.131043in}{0.773588in}}%
\pgfpathlineto{\pgfqpoint{25.181554in}{0.773588in}}%
\pgfpathlineto{\pgfqpoint{25.232944in}{0.773588in}}%
\pgfpathlineto{\pgfqpoint{25.282222in}{0.773588in}}%
\pgfpathlineto{\pgfqpoint{25.332608in}{0.773588in}}%
\pgfpathlineto{\pgfqpoint{25.383622in}{0.773588in}}%
\pgfpathlineto{\pgfqpoint{25.433954in}{0.773588in}}%
\pgfpathlineto{\pgfqpoint{25.483381in}{0.773588in}}%
\pgfpathlineto{\pgfqpoint{25.535558in}{0.773588in}}%
\pgfpathlineto{\pgfqpoint{25.586387in}{0.773588in}}%
\pgfpathlineto{\pgfqpoint{25.637513in}{0.773588in}}%
\pgfpathlineto{\pgfqpoint{25.689152in}{0.773588in}}%
\pgfpathlineto{\pgfqpoint{25.740073in}{0.773588in}}%
\pgfpathlineto{\pgfqpoint{25.790287in}{0.773588in}}%
\pgfpathlineto{\pgfqpoint{25.841969in}{0.773588in}}%
\pgfpathlineto{\pgfqpoint{25.891991in}{0.773588in}}%
\pgfpathlineto{\pgfqpoint{25.942005in}{0.773588in}}%
\pgfpathlineto{\pgfqpoint{25.994130in}{0.773588in}}%
\pgfpathlineto{\pgfqpoint{26.044216in}{0.773588in}}%
\pgfpathlineto{\pgfqpoint{26.093895in}{0.773588in}}%
\pgfpathlineto{\pgfqpoint{26.144837in}{0.773588in}}%
\pgfpathlineto{\pgfqpoint{26.195263in}{0.773588in}}%
\pgfpathlineto{\pgfqpoint{26.245717in}{0.773588in}}%
\pgfpathlineto{\pgfqpoint{26.297518in}{0.773588in}}%
\pgfpathlineto{\pgfqpoint{26.346588in}{0.773588in}}%
\pgfpathlineto{\pgfqpoint{26.395829in}{0.773588in}}%
\pgfpathlineto{\pgfqpoint{26.447655in}{0.773588in}}%
\pgfpathlineto{\pgfqpoint{26.499064in}{0.773588in}}%
\pgfpathlineto{\pgfqpoint{26.549446in}{0.773588in}}%
\pgfpathlineto{\pgfqpoint{26.601868in}{0.773588in}}%
\pgfpathlineto{\pgfqpoint{26.653161in}{0.773588in}}%
\pgfpathlineto{\pgfqpoint{26.704176in}{0.773588in}}%
\pgfpathlineto{\pgfqpoint{26.756448in}{0.773588in}}%
\pgfpathlineto{\pgfqpoint{26.807173in}{0.773588in}}%
\pgfpathlineto{\pgfqpoint{26.857706in}{0.773588in}}%
\pgfpathlineto{\pgfqpoint{26.909530in}{0.773588in}}%
\pgfpathlineto{\pgfqpoint{26.960748in}{0.773588in}}%
\pgfpathlineto{\pgfqpoint{27.010539in}{0.773588in}}%
\pgfpathlineto{\pgfqpoint{27.063062in}{0.773588in}}%
\pgfpathlineto{\pgfqpoint{27.113950in}{0.773588in}}%
\pgfpathlineto{\pgfqpoint{27.165274in}{0.773588in}}%
\pgfpathlineto{\pgfqpoint{27.218587in}{0.773588in}}%
\pgfpathlineto{\pgfqpoint{27.269679in}{0.773588in}}%
\pgfpathlineto{\pgfqpoint{27.321160in}{0.773588in}}%
\pgfpathlineto{\pgfqpoint{27.372863in}{0.773588in}}%
\pgfpathlineto{\pgfqpoint{27.423066in}{0.773588in}}%
\pgfpathlineto{\pgfqpoint{27.473173in}{0.773588in}}%
\pgfpathlineto{\pgfqpoint{27.525849in}{0.773588in}}%
\pgfpathlineto{\pgfqpoint{27.576178in}{0.773588in}}%
\pgfpathlineto{\pgfqpoint{27.626352in}{0.773588in}}%
\pgfpathlineto{\pgfqpoint{27.678294in}{0.773588in}}%
\pgfpathlineto{\pgfqpoint{27.728571in}{0.773588in}}%
\pgfpathlineto{\pgfqpoint{27.779331in}{0.773588in}}%
\pgfpathlineto{\pgfqpoint{27.833067in}{0.773588in}}%
\pgfpathlineto{\pgfqpoint{27.883786in}{0.773588in}}%
\pgfpathlineto{\pgfqpoint{27.935127in}{0.773588in}}%
\pgfpathlineto{\pgfqpoint{27.988562in}{0.773588in}}%
\pgfpathlineto{\pgfqpoint{28.039571in}{0.773588in}}%
\pgfpathlineto{\pgfqpoint{28.090650in}{0.773588in}}%
\pgfpathlineto{\pgfqpoint{28.143944in}{0.773588in}}%
\pgfpathlineto{\pgfqpoint{28.195034in}{0.773588in}}%
\pgfpathlineto{\pgfqpoint{28.245320in}{0.773588in}}%
\pgfpathlineto{\pgfqpoint{28.297462in}{0.773588in}}%
\pgfpathlineto{\pgfqpoint{28.347853in}{0.773588in}}%
\pgfpathlineto{\pgfqpoint{28.399138in}{0.773588in}}%
\pgfpathlineto{\pgfqpoint{28.451952in}{0.773588in}}%
\pgfpathlineto{\pgfqpoint{28.502707in}{0.773588in}}%
\pgfpathlineto{\pgfqpoint{28.553749in}{0.773588in}}%
\pgfpathlineto{\pgfqpoint{28.605351in}{0.773588in}}%
\pgfpathlineto{\pgfqpoint{28.656185in}{0.773588in}}%
\pgfpathlineto{\pgfqpoint{28.706300in}{0.773588in}}%
\pgfpathlineto{\pgfqpoint{28.758074in}{0.773588in}}%
\pgfpathlineto{\pgfqpoint{28.808587in}{0.773588in}}%
\pgfpathlineto{\pgfqpoint{28.860347in}{0.773588in}}%
\pgfpathlineto{\pgfqpoint{28.913538in}{0.773588in}}%
\pgfpathlineto{\pgfqpoint{28.965063in}{0.773588in}}%
\pgfpathlineto{\pgfqpoint{29.015651in}{0.773588in}}%
\pgfpathlineto{\pgfqpoint{29.068863in}{0.773588in}}%
\pgfpathlineto{\pgfqpoint{29.121029in}{0.773588in}}%
\pgfpathlineto{\pgfqpoint{29.173700in}{0.773588in}}%
\pgfpathlineto{\pgfqpoint{29.226818in}{0.773588in}}%
\pgfpathlineto{\pgfqpoint{29.279399in}{0.773588in}}%
\pgfpathlineto{\pgfqpoint{29.332261in}{0.773588in}}%
\pgfpathlineto{\pgfqpoint{29.386173in}{0.773588in}}%
\pgfpathlineto{\pgfqpoint{29.438335in}{0.773588in}}%
\pgfpathlineto{\pgfqpoint{29.490075in}{0.773588in}}%
\pgfpathlineto{\pgfqpoint{29.543760in}{0.773588in}}%
\pgfpathlineto{\pgfqpoint{29.596078in}{0.773588in}}%
\pgfpathlineto{\pgfqpoint{29.647988in}{0.773588in}}%
\pgfpathlineto{\pgfqpoint{29.701409in}{0.773588in}}%
\pgfpathlineto{\pgfqpoint{29.753093in}{0.773588in}}%
\pgfpathlineto{\pgfqpoint{29.805208in}{0.773588in}}%
\pgfpathlineto{\pgfqpoint{29.858913in}{0.773588in}}%
\pgfpathlineto{\pgfqpoint{29.910558in}{0.773588in}}%
\pgfpathlineto{\pgfqpoint{29.962066in}{0.773588in}}%
\pgfpathlineto{\pgfqpoint{30.014968in}{0.773588in}}%
\pgfpathlineto{\pgfqpoint{30.066247in}{0.773588in}}%
\pgfpathlineto{\pgfqpoint{30.117082in}{0.773588in}}%
\pgfpathlineto{\pgfqpoint{30.169497in}{0.773588in}}%
\pgfpathlineto{\pgfqpoint{30.221783in}{0.773588in}}%
\pgfpathlineto{\pgfqpoint{30.276704in}{0.773588in}}%
\pgfpathlineto{\pgfqpoint{30.337041in}{0.773588in}}%
\pgfpathlineto{\pgfqpoint{30.397563in}{0.773588in}}%
\pgfpathlineto{\pgfqpoint{30.457369in}{0.773588in}}%
\pgfpathlineto{\pgfqpoint{30.519963in}{0.773588in}}%
\pgfpathlineto{\pgfqpoint{30.582999in}{0.773588in}}%
\pgfpathlineto{\pgfqpoint{30.648377in}{0.773588in}}%
\pgfpathlineto{\pgfqpoint{30.717993in}{0.773588in}}%
\pgfpathlineto{\pgfqpoint{30.786068in}{0.773588in}}%
\pgfpathlineto{\pgfqpoint{30.855684in}{0.773588in}}%
\pgfpathlineto{\pgfqpoint{30.928370in}{0.773588in}}%
\pgfpathlineto{\pgfqpoint{31.000738in}{0.773588in}}%
\pgfpathlineto{\pgfqpoint{31.073353in}{0.773588in}}%
\pgfpathlineto{\pgfqpoint{31.150292in}{0.773588in}}%
\pgfpathlineto{\pgfqpoint{31.225476in}{0.773588in}}%
\pgfpathlineto{\pgfqpoint{31.304546in}{0.773588in}}%
\pgfpathlineto{\pgfqpoint{31.385546in}{0.773588in}}%
\pgfpathlineto{\pgfqpoint{31.463397in}{0.773588in}}%
\pgfpathlineto{\pgfqpoint{31.543273in}{0.773588in}}%
\pgfpathlineto{\pgfqpoint{31.629527in}{0.773588in}}%
\pgfpathlineto{\pgfqpoint{31.715049in}{0.773588in}}%
\pgfpathlineto{\pgfqpoint{31.802724in}{0.773588in}}%
\pgfpathlineto{\pgfqpoint{31.890397in}{0.773588in}}%
\pgfpathlineto{\pgfqpoint{31.975639in}{0.773588in}}%
\pgfpathlineto{\pgfqpoint{32.062211in}{0.773588in}}%
\pgfpathlineto{\pgfqpoint{32.153206in}{0.773588in}}%
\pgfpathlineto{\pgfqpoint{32.244846in}{0.773588in}}%
\pgfpathlineto{\pgfqpoint{32.334387in}{0.773588in}}%
\pgfpathlineto{\pgfqpoint{32.429213in}{0.773588in}}%
\pgfpathlineto{\pgfqpoint{32.518604in}{0.773588in}}%
\pgfpathlineto{\pgfqpoint{32.584565in}{0.773588in}}%
\pgfpathlineto{\pgfqpoint{32.638735in}{0.773588in}}%
\pgfpathlineto{\pgfqpoint{32.691213in}{0.773588in}}%
\pgfpathlineto{\pgfqpoint{32.743153in}{0.773588in}}%
\pgfpathlineto{\pgfqpoint{32.797230in}{0.773588in}}%
\pgfpathlineto{\pgfqpoint{32.850576in}{0.773588in}}%
\pgfpathlineto{\pgfqpoint{32.903667in}{0.773588in}}%
\pgfpathlineto{\pgfqpoint{32.957868in}{0.773588in}}%
\pgfpathlineto{\pgfqpoint{33.010153in}{0.773588in}}%
\pgfpathlineto{\pgfqpoint{33.062733in}{0.773588in}}%
\pgfpathlineto{\pgfqpoint{33.115518in}{0.773588in}}%
\pgfpathlineto{\pgfqpoint{33.167699in}{0.773588in}}%
\pgfpathlineto{\pgfqpoint{33.219830in}{0.773588in}}%
\pgfpathlineto{\pgfqpoint{33.273785in}{0.773588in}}%
\pgfpathlineto{\pgfqpoint{33.326652in}{0.773588in}}%
\pgfpathlineto{\pgfqpoint{33.379482in}{0.773588in}}%
\pgfpathlineto{\pgfqpoint{33.433283in}{0.773588in}}%
\pgfpathlineto{\pgfqpoint{33.484775in}{0.773588in}}%
\pgfpathlineto{\pgfqpoint{33.536879in}{0.773588in}}%
\pgfpathlineto{\pgfqpoint{33.590505in}{0.773588in}}%
\pgfpathlineto{\pgfqpoint{33.642373in}{0.773588in}}%
\pgfpathlineto{\pgfqpoint{33.693680in}{0.773588in}}%
\pgfpathlineto{\pgfqpoint{33.746200in}{0.773588in}}%
\pgfpathlineto{\pgfqpoint{33.784450in}{0.773588in}}%
\pgfpathlineto{\pgfqpoint{33.832264in}{0.773588in}}%
\pgfpathlineto{\pgfqpoint{33.870742in}{0.773588in}}%
\pgfpathlineto{\pgfqpoint{33.912991in}{0.773588in}}%
\pgfpathlineto{\pgfqpoint{33.952831in}{0.773588in}}%
\pgfpathlineto{\pgfqpoint{33.989641in}{0.773588in}}%
\pgfpathlineto{\pgfqpoint{34.021637in}{0.773588in}}%
\pgfpathlineto{\pgfqpoint{34.052161in}{0.773588in}}%
\pgfpathlineto{\pgfqpoint{34.076513in}{0.773588in}}%
\pgfpathlineto{\pgfqpoint{34.101659in}{0.773588in}}%
\pgfpathlineto{\pgfqpoint{34.125436in}{0.773588in}}%
\pgfpathlineto{\pgfqpoint{34.150180in}{0.773588in}}%
\pgfpathlineto{\pgfqpoint{34.173862in}{0.773588in}}%
\pgfpathlineto{\pgfqpoint{34.197440in}{0.773588in}}%
\pgfpathlineto{\pgfqpoint{34.222832in}{0.773588in}}%
\pgfpathlineto{\pgfqpoint{34.246364in}{0.773588in}}%
\pgfpathlineto{\pgfqpoint{34.270997in}{0.773588in}}%
\pgfpathlineto{\pgfqpoint{34.293786in}{0.773588in}}%
\pgfpathlineto{\pgfqpoint{34.318035in}{0.773588in}}%
\pgfpathlineto{\pgfqpoint{34.340984in}{0.773588in}}%
\pgfpathlineto{\pgfqpoint{34.366319in}{0.773588in}}%
\pgfpathlineto{\pgfqpoint{34.388883in}{0.773588in}}%
\pgfpathlineto{\pgfqpoint{34.412956in}{0.773588in}}%
\pgfpathlineto{\pgfqpoint{34.437757in}{0.773588in}}%
\pgfpathlineto{\pgfqpoint{34.460943in}{0.773588in}}%
\pgfpathlineto{\pgfqpoint{34.484122in}{0.773588in}}%
\pgfpathlineto{\pgfqpoint{34.508593in}{0.773588in}}%
\pgfpathlineto{\pgfqpoint{34.531364in}{0.773588in}}%
\pgfpathlineto{\pgfqpoint{34.554213in}{0.773588in}}%
\pgfpathlineto{\pgfqpoint{34.578241in}{0.773588in}}%
\pgfpathlineto{\pgfqpoint{34.601042in}{0.773588in}}%
\pgfpathlineto{\pgfqpoint{34.623987in}{0.773588in}}%
\pgfpathlineto{\pgfqpoint{34.648006in}{0.773588in}}%
\pgfpathlineto{\pgfqpoint{34.671232in}{0.773588in}}%
\pgfpathlineto{\pgfqpoint{34.693969in}{0.773588in}}%
\pgfpathlineto{\pgfqpoint{34.717979in}{0.773588in}}%
\pgfpathlineto{\pgfqpoint{34.741399in}{0.773588in}}%
\pgfpathlineto{\pgfqpoint{34.763884in}{0.773588in}}%
\pgfpathlineto{\pgfqpoint{34.788312in}{0.773588in}}%
\pgfpathlineto{\pgfqpoint{34.810332in}{0.773588in}}%
\pgfpathlineto{\pgfqpoint{34.834146in}{0.773588in}}%
\pgfpathlineto{\pgfqpoint{34.856576in}{0.773588in}}%
\pgfpathlineto{\pgfqpoint{34.880395in}{0.773588in}}%
\pgfpathlineto{\pgfqpoint{34.902821in}{0.773588in}}%
\pgfpathlineto{\pgfqpoint{34.926805in}{0.773588in}}%
\pgfpathlineto{\pgfqpoint{34.948980in}{0.773588in}}%
\pgfpathlineto{\pgfqpoint{34.973284in}{0.773588in}}%
\pgfpathlineto{\pgfqpoint{34.996094in}{0.773588in}}%
\pgfpathlineto{\pgfqpoint{35.020184in}{0.773588in}}%
\pgfpathlineto{\pgfqpoint{35.047391in}{0.773588in}}%
\pgfpathlineto{\pgfqpoint{35.098205in}{0.773588in}}%
\pgfpathlineto{\pgfqpoint{35.149293in}{0.773588in}}%
\pgfpathlineto{\pgfqpoint{35.201170in}{0.773588in}}%
\pgfpathlineto{\pgfqpoint{35.254065in}{0.773588in}}%
\pgfpathlineto{\pgfqpoint{35.305776in}{0.773588in}}%
\pgfpathlineto{\pgfqpoint{35.357710in}{0.773588in}}%
\pgfpathlineto{\pgfqpoint{35.411910in}{0.773588in}}%
\pgfpathlineto{\pgfqpoint{35.464942in}{0.773588in}}%
\pgfpathlineto{\pgfqpoint{35.464942in}{2.413621in}}%
\pgfpathlineto{\pgfqpoint{35.464942in}{2.413621in}}%
\pgfpathlineto{\pgfqpoint{35.411910in}{2.413621in}}%
\pgfpathlineto{\pgfqpoint{35.357710in}{2.413621in}}%
\pgfpathlineto{\pgfqpoint{35.305776in}{2.413621in}}%
\pgfpathlineto{\pgfqpoint{35.254065in}{2.413621in}}%
\pgfpathlineto{\pgfqpoint{35.201170in}{2.413621in}}%
\pgfpathlineto{\pgfqpoint{35.149293in}{2.413621in}}%
\pgfpathlineto{\pgfqpoint{35.098205in}{2.413621in}}%
\pgfpathlineto{\pgfqpoint{35.047391in}{2.423078in}}%
\pgfpathlineto{\pgfqpoint{35.020184in}{2.470556in}}%
\pgfpathlineto{\pgfqpoint{34.996094in}{2.424295in}}%
\pgfpathlineto{\pgfqpoint{34.973284in}{2.403966in}}%
\pgfpathlineto{\pgfqpoint{34.948980in}{2.587240in}}%
\pgfpathlineto{\pgfqpoint{34.926805in}{2.468813in}}%
\pgfpathlineto{\pgfqpoint{34.902821in}{2.477526in}}%
\pgfpathlineto{\pgfqpoint{34.880395in}{2.522929in}}%
\pgfpathlineto{\pgfqpoint{34.856576in}{2.570004in}}%
\pgfpathlineto{\pgfqpoint{34.834146in}{2.475980in}}%
\pgfpathlineto{\pgfqpoint{34.810332in}{2.469111in}}%
\pgfpathlineto{\pgfqpoint{34.788312in}{2.457377in}}%
\pgfpathlineto{\pgfqpoint{34.763884in}{2.581556in}}%
\pgfpathlineto{\pgfqpoint{34.741399in}{2.431345in}}%
\pgfpathlineto{\pgfqpoint{34.717979in}{2.431018in}}%
\pgfpathlineto{\pgfqpoint{34.693969in}{2.578334in}}%
\pgfpathlineto{\pgfqpoint{34.671232in}{2.380016in}}%
\pgfpathlineto{\pgfqpoint{34.648006in}{2.409123in}}%
\pgfpathlineto{\pgfqpoint{34.623987in}{2.434375in}}%
\pgfpathlineto{\pgfqpoint{34.601042in}{2.526820in}}%
\pgfpathlineto{\pgfqpoint{34.578241in}{2.553834in}}%
\pgfpathlineto{\pgfqpoint{34.554213in}{2.434395in}}%
\pgfpathlineto{\pgfqpoint{34.531364in}{2.411536in}}%
\pgfpathlineto{\pgfqpoint{34.508593in}{2.282667in}}%
\pgfpathlineto{\pgfqpoint{34.484122in}{2.433623in}}%
\pgfpathlineto{\pgfqpoint{34.460943in}{2.469518in}}%
\pgfpathlineto{\pgfqpoint{34.437757in}{2.338107in}}%
\pgfpathlineto{\pgfqpoint{34.412956in}{2.232666in}}%
\pgfpathlineto{\pgfqpoint{34.388883in}{2.287340in}}%
\pgfpathlineto{\pgfqpoint{34.366319in}{2.271817in}}%
\pgfpathlineto{\pgfqpoint{34.340984in}{2.394574in}}%
\pgfpathlineto{\pgfqpoint{34.318035in}{2.449950in}}%
\pgfpathlineto{\pgfqpoint{34.293786in}{2.361353in}}%
\pgfpathlineto{\pgfqpoint{34.270997in}{2.267974in}}%
\pgfpathlineto{\pgfqpoint{34.246364in}{2.316562in}}%
\pgfpathlineto{\pgfqpoint{34.222832in}{2.257032in}}%
\pgfpathlineto{\pgfqpoint{34.197440in}{2.332444in}}%
\pgfpathlineto{\pgfqpoint{34.173862in}{2.339022in}}%
\pgfpathlineto{\pgfqpoint{34.150180in}{2.446672in}}%
\pgfpathlineto{\pgfqpoint{34.125436in}{2.290107in}}%
\pgfpathlineto{\pgfqpoint{34.101659in}{2.218952in}}%
\pgfpathlineto{\pgfqpoint{34.076513in}{2.293678in}}%
\pgfpathlineto{\pgfqpoint{34.052161in}{1.799827in}}%
\pgfpathlineto{\pgfqpoint{34.021637in}{1.503451in}}%
\pgfpathlineto{\pgfqpoint{33.989641in}{1.243236in}}%
\pgfpathlineto{\pgfqpoint{33.952831in}{1.158641in}}%
\pgfpathlineto{\pgfqpoint{33.912991in}{1.089794in}}%
\pgfpathlineto{\pgfqpoint{33.870742in}{1.025897in}}%
\pgfpathlineto{\pgfqpoint{33.832264in}{0.773588in}}%
\pgfpathlineto{\pgfqpoint{33.784450in}{0.773588in}}%
\pgfpathlineto{\pgfqpoint{33.746200in}{0.773588in}}%
\pgfpathlineto{\pgfqpoint{33.693680in}{0.773588in}}%
\pgfpathlineto{\pgfqpoint{33.642373in}{0.773588in}}%
\pgfpathlineto{\pgfqpoint{33.590505in}{0.773588in}}%
\pgfpathlineto{\pgfqpoint{33.536879in}{0.773588in}}%
\pgfpathlineto{\pgfqpoint{33.484775in}{0.773588in}}%
\pgfpathlineto{\pgfqpoint{33.433283in}{0.773588in}}%
\pgfpathlineto{\pgfqpoint{33.379482in}{0.773588in}}%
\pgfpathlineto{\pgfqpoint{33.326652in}{0.773588in}}%
\pgfpathlineto{\pgfqpoint{33.273785in}{0.773588in}}%
\pgfpathlineto{\pgfqpoint{33.219830in}{0.773588in}}%
\pgfpathlineto{\pgfqpoint{33.167699in}{0.773588in}}%
\pgfpathlineto{\pgfqpoint{33.115518in}{0.773588in}}%
\pgfpathlineto{\pgfqpoint{33.062733in}{0.773588in}}%
\pgfpathlineto{\pgfqpoint{33.010153in}{0.773588in}}%
\pgfpathlineto{\pgfqpoint{32.957868in}{0.773588in}}%
\pgfpathlineto{\pgfqpoint{32.903667in}{0.773588in}}%
\pgfpathlineto{\pgfqpoint{32.850576in}{0.773588in}}%
\pgfpathlineto{\pgfqpoint{32.797230in}{0.773588in}}%
\pgfpathlineto{\pgfqpoint{32.743153in}{0.773588in}}%
\pgfpathlineto{\pgfqpoint{32.691213in}{0.773588in}}%
\pgfpathlineto{\pgfqpoint{32.638735in}{0.773588in}}%
\pgfpathlineto{\pgfqpoint{32.584565in}{0.773588in}}%
\pgfpathlineto{\pgfqpoint{32.518604in}{0.773588in}}%
\pgfpathlineto{\pgfqpoint{32.429213in}{0.773588in}}%
\pgfpathlineto{\pgfqpoint{32.334387in}{0.773588in}}%
\pgfpathlineto{\pgfqpoint{32.244846in}{0.773588in}}%
\pgfpathlineto{\pgfqpoint{32.153206in}{0.773588in}}%
\pgfpathlineto{\pgfqpoint{32.062211in}{0.773588in}}%
\pgfpathlineto{\pgfqpoint{31.975639in}{0.773588in}}%
\pgfpathlineto{\pgfqpoint{31.890397in}{0.773588in}}%
\pgfpathlineto{\pgfqpoint{31.802724in}{0.773588in}}%
\pgfpathlineto{\pgfqpoint{31.715049in}{0.773588in}}%
\pgfpathlineto{\pgfqpoint{31.629527in}{0.773588in}}%
\pgfpathlineto{\pgfqpoint{31.543273in}{0.773588in}}%
\pgfpathlineto{\pgfqpoint{31.463397in}{0.773588in}}%
\pgfpathlineto{\pgfqpoint{31.385546in}{0.773588in}}%
\pgfpathlineto{\pgfqpoint{31.304546in}{0.773588in}}%
\pgfpathlineto{\pgfqpoint{31.225476in}{0.773588in}}%
\pgfpathlineto{\pgfqpoint{31.150292in}{0.773588in}}%
\pgfpathlineto{\pgfqpoint{31.073353in}{0.773588in}}%
\pgfpathlineto{\pgfqpoint{31.000738in}{0.773588in}}%
\pgfpathlineto{\pgfqpoint{30.928370in}{0.773588in}}%
\pgfpathlineto{\pgfqpoint{30.855684in}{0.773588in}}%
\pgfpathlineto{\pgfqpoint{30.786068in}{0.773588in}}%
\pgfpathlineto{\pgfqpoint{30.717993in}{0.773588in}}%
\pgfpathlineto{\pgfqpoint{30.648377in}{0.773588in}}%
\pgfpathlineto{\pgfqpoint{30.582999in}{0.773588in}}%
\pgfpathlineto{\pgfqpoint{30.519963in}{0.773588in}}%
\pgfpathlineto{\pgfqpoint{30.457369in}{0.773588in}}%
\pgfpathlineto{\pgfqpoint{30.397563in}{0.773588in}}%
\pgfpathlineto{\pgfqpoint{30.337041in}{0.773588in}}%
\pgfpathlineto{\pgfqpoint{30.276704in}{0.773588in}}%
\pgfpathlineto{\pgfqpoint{30.221783in}{0.773588in}}%
\pgfpathlineto{\pgfqpoint{30.169497in}{0.773588in}}%
\pgfpathlineto{\pgfqpoint{30.117082in}{0.773588in}}%
\pgfpathlineto{\pgfqpoint{30.066247in}{0.773588in}}%
\pgfpathlineto{\pgfqpoint{30.014968in}{0.773588in}}%
\pgfpathlineto{\pgfqpoint{29.962066in}{0.773588in}}%
\pgfpathlineto{\pgfqpoint{29.910558in}{0.773588in}}%
\pgfpathlineto{\pgfqpoint{29.858913in}{0.773588in}}%
\pgfpathlineto{\pgfqpoint{29.805208in}{0.773588in}}%
\pgfpathlineto{\pgfqpoint{29.753093in}{0.773588in}}%
\pgfpathlineto{\pgfqpoint{29.701409in}{0.773588in}}%
\pgfpathlineto{\pgfqpoint{29.647988in}{0.773588in}}%
\pgfpathlineto{\pgfqpoint{29.596078in}{0.773588in}}%
\pgfpathlineto{\pgfqpoint{29.543760in}{0.773588in}}%
\pgfpathlineto{\pgfqpoint{29.490075in}{0.773588in}}%
\pgfpathlineto{\pgfqpoint{29.438335in}{0.773588in}}%
\pgfpathlineto{\pgfqpoint{29.386173in}{0.773588in}}%
\pgfpathlineto{\pgfqpoint{29.332261in}{0.773588in}}%
\pgfpathlineto{\pgfqpoint{29.279399in}{0.773588in}}%
\pgfpathlineto{\pgfqpoint{29.226818in}{0.773588in}}%
\pgfpathlineto{\pgfqpoint{29.173700in}{0.773588in}}%
\pgfpathlineto{\pgfqpoint{29.121029in}{0.773588in}}%
\pgfpathlineto{\pgfqpoint{29.068863in}{0.773588in}}%
\pgfpathlineto{\pgfqpoint{29.015651in}{0.773588in}}%
\pgfpathlineto{\pgfqpoint{28.965063in}{0.773588in}}%
\pgfpathlineto{\pgfqpoint{28.913538in}{0.773588in}}%
\pgfpathlineto{\pgfqpoint{28.860347in}{0.773588in}}%
\pgfpathlineto{\pgfqpoint{28.808587in}{0.773588in}}%
\pgfpathlineto{\pgfqpoint{28.758074in}{0.773588in}}%
\pgfpathlineto{\pgfqpoint{28.706300in}{0.773588in}}%
\pgfpathlineto{\pgfqpoint{28.656185in}{0.773588in}}%
\pgfpathlineto{\pgfqpoint{28.605351in}{0.773588in}}%
\pgfpathlineto{\pgfqpoint{28.553749in}{0.773588in}}%
\pgfpathlineto{\pgfqpoint{28.502707in}{0.773588in}}%
\pgfpathlineto{\pgfqpoint{28.451952in}{0.773588in}}%
\pgfpathlineto{\pgfqpoint{28.399138in}{0.773588in}}%
\pgfpathlineto{\pgfqpoint{28.347853in}{0.773588in}}%
\pgfpathlineto{\pgfqpoint{28.297462in}{0.773588in}}%
\pgfpathlineto{\pgfqpoint{28.245320in}{0.773588in}}%
\pgfpathlineto{\pgfqpoint{28.195034in}{0.773588in}}%
\pgfpathlineto{\pgfqpoint{28.143944in}{0.773588in}}%
\pgfpathlineto{\pgfqpoint{28.090650in}{0.773588in}}%
\pgfpathlineto{\pgfqpoint{28.039571in}{0.773588in}}%
\pgfpathlineto{\pgfqpoint{27.988562in}{0.773588in}}%
\pgfpathlineto{\pgfqpoint{27.935127in}{0.773588in}}%
\pgfpathlineto{\pgfqpoint{27.883786in}{0.773588in}}%
\pgfpathlineto{\pgfqpoint{27.833067in}{0.773588in}}%
\pgfpathlineto{\pgfqpoint{27.779331in}{0.773588in}}%
\pgfpathlineto{\pgfqpoint{27.728571in}{0.773588in}}%
\pgfpathlineto{\pgfqpoint{27.678294in}{0.773588in}}%
\pgfpathlineto{\pgfqpoint{27.626352in}{0.773588in}}%
\pgfpathlineto{\pgfqpoint{27.576178in}{0.773588in}}%
\pgfpathlineto{\pgfqpoint{27.525849in}{0.773588in}}%
\pgfpathlineto{\pgfqpoint{27.473173in}{0.773588in}}%
\pgfpathlineto{\pgfqpoint{27.423066in}{0.773588in}}%
\pgfpathlineto{\pgfqpoint{27.372863in}{0.773588in}}%
\pgfpathlineto{\pgfqpoint{27.321160in}{0.773588in}}%
\pgfpathlineto{\pgfqpoint{27.269679in}{0.773588in}}%
\pgfpathlineto{\pgfqpoint{27.218587in}{0.773588in}}%
\pgfpathlineto{\pgfqpoint{27.165274in}{0.773588in}}%
\pgfpathlineto{\pgfqpoint{27.113950in}{0.773588in}}%
\pgfpathlineto{\pgfqpoint{27.063062in}{0.773588in}}%
\pgfpathlineto{\pgfqpoint{27.010539in}{0.773588in}}%
\pgfpathlineto{\pgfqpoint{26.960748in}{0.773588in}}%
\pgfpathlineto{\pgfqpoint{26.909530in}{0.773588in}}%
\pgfpathlineto{\pgfqpoint{26.857706in}{0.773588in}}%
\pgfpathlineto{\pgfqpoint{26.807173in}{0.773588in}}%
\pgfpathlineto{\pgfqpoint{26.756448in}{0.773588in}}%
\pgfpathlineto{\pgfqpoint{26.704176in}{0.773588in}}%
\pgfpathlineto{\pgfqpoint{26.653161in}{0.773588in}}%
\pgfpathlineto{\pgfqpoint{26.601868in}{0.773588in}}%
\pgfpathlineto{\pgfqpoint{26.549446in}{0.773588in}}%
\pgfpathlineto{\pgfqpoint{26.499064in}{0.773588in}}%
\pgfpathlineto{\pgfqpoint{26.447655in}{0.773588in}}%
\pgfpathlineto{\pgfqpoint{26.395829in}{0.773588in}}%
\pgfpathlineto{\pgfqpoint{26.346588in}{0.773588in}}%
\pgfpathlineto{\pgfqpoint{26.297518in}{0.773588in}}%
\pgfpathlineto{\pgfqpoint{26.245717in}{0.773588in}}%
\pgfpathlineto{\pgfqpoint{26.195263in}{0.773588in}}%
\pgfpathlineto{\pgfqpoint{26.144837in}{0.773588in}}%
\pgfpathlineto{\pgfqpoint{26.093895in}{0.773588in}}%
\pgfpathlineto{\pgfqpoint{26.044216in}{0.773588in}}%
\pgfpathlineto{\pgfqpoint{25.994130in}{0.773588in}}%
\pgfpathlineto{\pgfqpoint{25.942005in}{0.773588in}}%
\pgfpathlineto{\pgfqpoint{25.891991in}{0.773588in}}%
\pgfpathlineto{\pgfqpoint{25.841969in}{0.773588in}}%
\pgfpathlineto{\pgfqpoint{25.790287in}{0.773588in}}%
\pgfpathlineto{\pgfqpoint{25.740073in}{0.773588in}}%
\pgfpathlineto{\pgfqpoint{25.689152in}{0.773588in}}%
\pgfpathlineto{\pgfqpoint{25.637513in}{0.773588in}}%
\pgfpathlineto{\pgfqpoint{25.586387in}{0.773588in}}%
\pgfpathlineto{\pgfqpoint{25.535558in}{0.773588in}}%
\pgfpathlineto{\pgfqpoint{25.483381in}{0.773588in}}%
\pgfpathlineto{\pgfqpoint{25.433954in}{0.773588in}}%
\pgfpathlineto{\pgfqpoint{25.383622in}{0.773588in}}%
\pgfpathlineto{\pgfqpoint{25.332608in}{0.773588in}}%
\pgfpathlineto{\pgfqpoint{25.282222in}{0.773588in}}%
\pgfpathlineto{\pgfqpoint{25.232944in}{0.773588in}}%
\pgfpathlineto{\pgfqpoint{25.181554in}{0.773588in}}%
\pgfpathlineto{\pgfqpoint{25.131043in}{0.773588in}}%
\pgfpathlineto{\pgfqpoint{25.080539in}{0.773588in}}%
\pgfpathlineto{\pgfqpoint{25.028630in}{0.773588in}}%
\pgfpathlineto{\pgfqpoint{24.978211in}{0.773588in}}%
\pgfpathlineto{\pgfqpoint{24.928869in}{0.773588in}}%
\pgfpathlineto{\pgfqpoint{24.878143in}{0.773588in}}%
\pgfpathlineto{\pgfqpoint{24.828140in}{0.773588in}}%
\pgfpathlineto{\pgfqpoint{24.778304in}{0.773588in}}%
\pgfpathlineto{\pgfqpoint{24.726499in}{0.773588in}}%
\pgfpathlineto{\pgfqpoint{24.676389in}{0.773588in}}%
\pgfpathlineto{\pgfqpoint{24.626556in}{0.773588in}}%
\pgfpathlineto{\pgfqpoint{24.573998in}{0.773588in}}%
\pgfpathlineto{\pgfqpoint{24.524258in}{0.773588in}}%
\pgfpathlineto{\pgfqpoint{24.474381in}{0.773588in}}%
\pgfpathlineto{\pgfqpoint{24.422475in}{0.773588in}}%
\pgfpathlineto{\pgfqpoint{24.371890in}{0.773588in}}%
\pgfpathlineto{\pgfqpoint{24.321907in}{0.773588in}}%
\pgfpathlineto{\pgfqpoint{24.271277in}{0.773588in}}%
\pgfpathlineto{\pgfqpoint{24.221010in}{0.773588in}}%
\pgfpathlineto{\pgfqpoint{24.170873in}{0.773588in}}%
\pgfpathlineto{\pgfqpoint{24.118637in}{0.773588in}}%
\pgfpathlineto{\pgfqpoint{24.066933in}{0.773588in}}%
\pgfpathlineto{\pgfqpoint{24.016442in}{0.773588in}}%
\pgfpathlineto{\pgfqpoint{23.964514in}{0.773588in}}%
\pgfpathlineto{\pgfqpoint{23.913088in}{0.773588in}}%
\pgfpathlineto{\pgfqpoint{23.862673in}{0.773588in}}%
\pgfpathlineto{\pgfqpoint{23.811512in}{0.773588in}}%
\pgfpathlineto{\pgfqpoint{23.762329in}{0.773588in}}%
\pgfpathlineto{\pgfqpoint{23.713074in}{0.773588in}}%
\pgfpathlineto{\pgfqpoint{23.661575in}{0.773588in}}%
\pgfpathlineto{\pgfqpoint{23.611923in}{0.773588in}}%
\pgfpathlineto{\pgfqpoint{23.563429in}{0.773588in}}%
\pgfpathlineto{\pgfqpoint{23.511867in}{0.773588in}}%
\pgfpathlineto{\pgfqpoint{23.462017in}{0.773588in}}%
\pgfpathlineto{\pgfqpoint{23.411909in}{0.773588in}}%
\pgfpathlineto{\pgfqpoint{23.361310in}{0.773588in}}%
\pgfpathlineto{\pgfqpoint{23.310918in}{0.773588in}}%
\pgfpathlineto{\pgfqpoint{23.260726in}{0.773588in}}%
\pgfpathlineto{\pgfqpoint{23.208694in}{0.773588in}}%
\pgfpathlineto{\pgfqpoint{23.157756in}{0.773588in}}%
\pgfpathlineto{\pgfqpoint{23.107134in}{0.773588in}}%
\pgfpathlineto{\pgfqpoint{23.055187in}{0.773588in}}%
\pgfpathlineto{\pgfqpoint{23.006022in}{0.773588in}}%
\pgfpathlineto{\pgfqpoint{22.956359in}{0.773588in}}%
\pgfpathlineto{\pgfqpoint{22.904236in}{0.773588in}}%
\pgfpathlineto{\pgfqpoint{22.853966in}{0.773588in}}%
\pgfpathlineto{\pgfqpoint{22.803829in}{0.773588in}}%
\pgfpathlineto{\pgfqpoint{22.751700in}{0.773588in}}%
\pgfpathlineto{\pgfqpoint{22.703398in}{0.773588in}}%
\pgfpathlineto{\pgfqpoint{22.654822in}{0.773588in}}%
\pgfpathlineto{\pgfqpoint{22.603657in}{0.773588in}}%
\pgfpathlineto{\pgfqpoint{22.554073in}{0.773588in}}%
\pgfpathlineto{\pgfqpoint{22.503552in}{0.773588in}}%
\pgfpathlineto{\pgfqpoint{22.451976in}{0.773588in}}%
\pgfpathlineto{\pgfqpoint{22.402084in}{0.773588in}}%
\pgfpathlineto{\pgfqpoint{22.352827in}{0.773588in}}%
\pgfpathlineto{\pgfqpoint{22.300933in}{0.773588in}}%
\pgfpathlineto{\pgfqpoint{22.250634in}{0.773588in}}%
\pgfpathlineto{\pgfqpoint{22.200587in}{0.773588in}}%
\pgfpathlineto{\pgfqpoint{22.148817in}{0.773588in}}%
\pgfpathlineto{\pgfqpoint{22.099907in}{0.773588in}}%
\pgfpathlineto{\pgfqpoint{22.050354in}{0.773588in}}%
\pgfpathlineto{\pgfqpoint{21.999803in}{0.773588in}}%
\pgfpathlineto{\pgfqpoint{21.950544in}{0.773588in}}%
\pgfpathlineto{\pgfqpoint{21.900834in}{0.773588in}}%
\pgfpathlineto{\pgfqpoint{21.849580in}{0.773588in}}%
\pgfpathlineto{\pgfqpoint{21.800474in}{0.773588in}}%
\pgfpathlineto{\pgfqpoint{21.751624in}{0.773588in}}%
\pgfpathlineto{\pgfqpoint{21.700864in}{0.773588in}}%
\pgfpathlineto{\pgfqpoint{21.652276in}{0.773588in}}%
\pgfpathlineto{\pgfqpoint{21.602538in}{0.773588in}}%
\pgfpathlineto{\pgfqpoint{21.550957in}{0.773588in}}%
\pgfpathlineto{\pgfqpoint{21.501023in}{0.773588in}}%
\pgfpathlineto{\pgfqpoint{21.450830in}{0.773588in}}%
\pgfpathlineto{\pgfqpoint{21.399733in}{0.773588in}}%
\pgfpathlineto{\pgfqpoint{21.350353in}{0.773588in}}%
\pgfpathlineto{\pgfqpoint{21.300955in}{0.773588in}}%
\pgfpathlineto{\pgfqpoint{21.250655in}{0.773588in}}%
\pgfpathlineto{\pgfqpoint{21.201683in}{0.773588in}}%
\pgfpathlineto{\pgfqpoint{21.152719in}{0.773588in}}%
\pgfpathlineto{\pgfqpoint{21.102370in}{0.773588in}}%
\pgfpathlineto{\pgfqpoint{21.053515in}{0.773588in}}%
\pgfpathlineto{\pgfqpoint{21.003693in}{0.773588in}}%
\pgfpathlineto{\pgfqpoint{20.953585in}{0.773588in}}%
\pgfpathlineto{\pgfqpoint{20.904164in}{0.773588in}}%
\pgfpathlineto{\pgfqpoint{20.854118in}{0.773588in}}%
\pgfpathlineto{\pgfqpoint{20.802965in}{0.773588in}}%
\pgfpathlineto{\pgfqpoint{20.753558in}{0.773588in}}%
\pgfpathlineto{\pgfqpoint{20.704616in}{0.773588in}}%
\pgfpathlineto{\pgfqpoint{20.654668in}{0.773588in}}%
\pgfpathlineto{\pgfqpoint{20.605567in}{0.773588in}}%
\pgfpathlineto{\pgfqpoint{20.556290in}{0.773588in}}%
\pgfpathlineto{\pgfqpoint{20.505208in}{0.773588in}}%
\pgfpathlineto{\pgfqpoint{20.455604in}{0.773588in}}%
\pgfpathlineto{\pgfqpoint{20.406414in}{0.773588in}}%
\pgfpathlineto{\pgfqpoint{20.355462in}{0.773588in}}%
\pgfpathlineto{\pgfqpoint{20.305924in}{0.773588in}}%
\pgfpathlineto{\pgfqpoint{20.255593in}{0.773588in}}%
\pgfpathlineto{\pgfqpoint{20.203615in}{0.773588in}}%
\pgfpathlineto{\pgfqpoint{20.153679in}{0.773588in}}%
\pgfpathlineto{\pgfqpoint{20.104350in}{0.773588in}}%
\pgfpathlineto{\pgfqpoint{20.053312in}{0.773588in}}%
\pgfpathlineto{\pgfqpoint{20.003692in}{0.773588in}}%
\pgfpathlineto{\pgfqpoint{19.955252in}{0.773588in}}%
\pgfpathlineto{\pgfqpoint{19.906615in}{0.773588in}}%
\pgfpathlineto{\pgfqpoint{19.858933in}{0.773588in}}%
\pgfpathlineto{\pgfqpoint{19.810712in}{0.773588in}}%
\pgfpathlineto{\pgfqpoint{19.761678in}{0.773588in}}%
\pgfpathlineto{\pgfqpoint{19.714265in}{0.773588in}}%
\pgfpathlineto{\pgfqpoint{19.666940in}{0.773588in}}%
\pgfpathlineto{\pgfqpoint{19.617707in}{0.773588in}}%
\pgfpathlineto{\pgfqpoint{19.569747in}{0.773588in}}%
\pgfpathlineto{\pgfqpoint{19.522292in}{0.773588in}}%
\pgfpathlineto{\pgfqpoint{19.473154in}{0.773588in}}%
\pgfpathlineto{\pgfqpoint{19.425178in}{0.773588in}}%
\pgfpathlineto{\pgfqpoint{19.376890in}{0.773588in}}%
\pgfpathlineto{\pgfqpoint{19.327829in}{0.773588in}}%
\pgfpathlineto{\pgfqpoint{19.279742in}{0.773588in}}%
\pgfpathlineto{\pgfqpoint{19.231826in}{0.773588in}}%
\pgfpathlineto{\pgfqpoint{19.181302in}{0.773588in}}%
\pgfpathlineto{\pgfqpoint{19.132724in}{0.773588in}}%
\pgfpathlineto{\pgfqpoint{19.084207in}{0.773588in}}%
\pgfpathlineto{\pgfqpoint{19.033794in}{0.773588in}}%
\pgfpathlineto{\pgfqpoint{18.985002in}{0.773588in}}%
\pgfpathlineto{\pgfqpoint{18.937671in}{0.773588in}}%
\pgfpathlineto{\pgfqpoint{18.888188in}{0.773588in}}%
\pgfpathlineto{\pgfqpoint{18.839312in}{0.773588in}}%
\pgfpathlineto{\pgfqpoint{18.791212in}{0.773588in}}%
\pgfpathlineto{\pgfqpoint{18.741665in}{0.773588in}}%
\pgfpathlineto{\pgfqpoint{18.693122in}{0.773588in}}%
\pgfpathlineto{\pgfqpoint{18.644639in}{0.773588in}}%
\pgfpathlineto{\pgfqpoint{18.595724in}{0.773588in}}%
\pgfpathlineto{\pgfqpoint{18.548460in}{0.773588in}}%
\pgfpathlineto{\pgfqpoint{18.500915in}{0.773588in}}%
\pgfpathlineto{\pgfqpoint{18.452284in}{0.773588in}}%
\pgfpathlineto{\pgfqpoint{18.405242in}{0.773588in}}%
\pgfpathlineto{\pgfqpoint{18.357538in}{0.773588in}}%
\pgfpathlineto{\pgfqpoint{18.308441in}{0.773588in}}%
\pgfpathlineto{\pgfqpoint{18.260871in}{0.773588in}}%
\pgfpathlineto{\pgfqpoint{18.213333in}{0.773588in}}%
\pgfpathlineto{\pgfqpoint{18.164405in}{0.773588in}}%
\pgfpathlineto{\pgfqpoint{18.117307in}{0.773588in}}%
\pgfpathlineto{\pgfqpoint{18.069524in}{0.773588in}}%
\pgfpathlineto{\pgfqpoint{18.020026in}{0.773588in}}%
\pgfpathlineto{\pgfqpoint{17.970910in}{0.773588in}}%
\pgfpathlineto{\pgfqpoint{17.922885in}{0.773588in}}%
\pgfpathlineto{\pgfqpoint{17.874600in}{0.773588in}}%
\pgfpathlineto{\pgfqpoint{17.826809in}{0.773588in}}%
\pgfpathlineto{\pgfqpoint{17.779014in}{0.773588in}}%
\pgfpathlineto{\pgfqpoint{17.729727in}{0.773588in}}%
\pgfpathlineto{\pgfqpoint{17.681914in}{0.773588in}}%
\pgfpathlineto{\pgfqpoint{17.634653in}{0.773588in}}%
\pgfpathlineto{\pgfqpoint{17.585742in}{0.773588in}}%
\pgfpathlineto{\pgfqpoint{17.538409in}{0.773588in}}%
\pgfpathlineto{\pgfqpoint{17.491125in}{0.773588in}}%
\pgfpathlineto{\pgfqpoint{17.442362in}{0.773588in}}%
\pgfpathlineto{\pgfqpoint{17.394714in}{0.773588in}}%
\pgfpathlineto{\pgfqpoint{17.347576in}{0.773588in}}%
\pgfpathlineto{\pgfqpoint{17.298941in}{0.773588in}}%
\pgfpathlineto{\pgfqpoint{17.250860in}{0.773588in}}%
\pgfpathlineto{\pgfqpoint{17.203674in}{0.773588in}}%
\pgfpathlineto{\pgfqpoint{17.154507in}{0.773588in}}%
\pgfpathlineto{\pgfqpoint{17.107470in}{0.773588in}}%
\pgfpathlineto{\pgfqpoint{17.059518in}{0.773588in}}%
\pgfpathlineto{\pgfqpoint{17.010585in}{0.773588in}}%
\pgfpathlineto{\pgfqpoint{16.963533in}{0.773588in}}%
\pgfpathlineto{\pgfqpoint{16.916302in}{0.773588in}}%
\pgfpathlineto{\pgfqpoint{16.868266in}{0.773588in}}%
\pgfpathlineto{\pgfqpoint{16.821038in}{0.773588in}}%
\pgfpathlineto{\pgfqpoint{16.773713in}{0.773588in}}%
\pgfpathlineto{\pgfqpoint{16.724568in}{0.773588in}}%
\pgfpathlineto{\pgfqpoint{16.677566in}{0.773588in}}%
\pgfpathlineto{\pgfqpoint{16.630280in}{0.773588in}}%
\pgfpathlineto{\pgfqpoint{16.581135in}{0.773588in}}%
\pgfpathlineto{\pgfqpoint{16.533619in}{0.773588in}}%
\pgfpathlineto{\pgfqpoint{16.486560in}{0.773588in}}%
\pgfpathlineto{\pgfqpoint{16.436622in}{0.773588in}}%
\pgfpathlineto{\pgfqpoint{16.387666in}{0.773588in}}%
\pgfpathlineto{\pgfqpoint{16.339537in}{0.773588in}}%
\pgfpathlineto{\pgfqpoint{16.289817in}{0.773588in}}%
\pgfpathlineto{\pgfqpoint{16.242078in}{0.773588in}}%
\pgfpathlineto{\pgfqpoint{16.195175in}{0.773588in}}%
\pgfpathlineto{\pgfqpoint{16.146338in}{0.773588in}}%
\pgfpathlineto{\pgfqpoint{16.098697in}{0.773588in}}%
\pgfpathlineto{\pgfqpoint{16.051820in}{0.773588in}}%
\pgfpathlineto{\pgfqpoint{16.003926in}{0.773588in}}%
\pgfpathlineto{\pgfqpoint{15.957799in}{0.773588in}}%
\pgfpathlineto{\pgfqpoint{15.911811in}{0.773588in}}%
\pgfpathlineto{\pgfqpoint{15.863592in}{0.773588in}}%
\pgfpathlineto{\pgfqpoint{15.817273in}{0.773588in}}%
\pgfpathlineto{\pgfqpoint{15.770154in}{0.773588in}}%
\pgfpathlineto{\pgfqpoint{15.722814in}{0.773588in}}%
\pgfpathlineto{\pgfqpoint{15.676019in}{0.773588in}}%
\pgfpathlineto{\pgfqpoint{15.628660in}{0.773588in}}%
\pgfpathlineto{\pgfqpoint{15.579815in}{0.773588in}}%
\pgfpathlineto{\pgfqpoint{15.532329in}{0.773588in}}%
\pgfpathlineto{\pgfqpoint{15.484858in}{0.773588in}}%
\pgfpathlineto{\pgfqpoint{15.436052in}{0.773588in}}%
\pgfpathlineto{\pgfqpoint{15.389896in}{0.773588in}}%
\pgfpathlineto{\pgfqpoint{15.343883in}{0.773588in}}%
\pgfpathlineto{\pgfqpoint{15.295866in}{0.773588in}}%
\pgfpathlineto{\pgfqpoint{15.248863in}{0.773588in}}%
\pgfpathlineto{\pgfqpoint{15.201456in}{0.773588in}}%
\pgfpathlineto{\pgfqpoint{15.152658in}{0.773588in}}%
\pgfpathlineto{\pgfqpoint{15.105478in}{0.773588in}}%
\pgfpathlineto{\pgfqpoint{15.058774in}{0.773588in}}%
\pgfpathlineto{\pgfqpoint{15.010214in}{0.773588in}}%
\pgfpathlineto{\pgfqpoint{14.963167in}{0.773588in}}%
\pgfpathlineto{\pgfqpoint{14.916239in}{0.773588in}}%
\pgfpathlineto{\pgfqpoint{14.868242in}{0.773588in}}%
\pgfpathlineto{\pgfqpoint{14.821474in}{0.773588in}}%
\pgfpathlineto{\pgfqpoint{14.774752in}{0.773588in}}%
\pgfpathlineto{\pgfqpoint{14.726757in}{0.773588in}}%
\pgfpathlineto{\pgfqpoint{14.680226in}{0.773588in}}%
\pgfpathlineto{\pgfqpoint{14.633992in}{0.773588in}}%
\pgfpathlineto{\pgfqpoint{14.585477in}{0.773588in}}%
\pgfpathlineto{\pgfqpoint{14.538630in}{0.773588in}}%
\pgfpathlineto{\pgfqpoint{14.491726in}{0.773588in}}%
\pgfpathlineto{\pgfqpoint{14.443729in}{0.773588in}}%
\pgfpathlineto{\pgfqpoint{14.397126in}{0.773588in}}%
\pgfpathlineto{\pgfqpoint{14.350224in}{0.773588in}}%
\pgfpathlineto{\pgfqpoint{14.302418in}{0.773588in}}%
\pgfpathlineto{\pgfqpoint{14.255593in}{0.773588in}}%
\pgfpathlineto{\pgfqpoint{14.209376in}{0.773588in}}%
\pgfpathlineto{\pgfqpoint{14.160147in}{0.773588in}}%
\pgfpathlineto{\pgfqpoint{14.112147in}{0.773588in}}%
\pgfpathlineto{\pgfqpoint{14.064951in}{0.773588in}}%
\pgfpathlineto{\pgfqpoint{14.016344in}{0.773588in}}%
\pgfpathlineto{\pgfqpoint{13.969353in}{0.773588in}}%
\pgfpathlineto{\pgfqpoint{13.922606in}{0.773588in}}%
\pgfpathlineto{\pgfqpoint{13.873610in}{0.773588in}}%
\pgfpathlineto{\pgfqpoint{13.825297in}{0.773588in}}%
\pgfpathlineto{\pgfqpoint{13.777441in}{0.773588in}}%
\pgfpathlineto{\pgfqpoint{13.729143in}{0.773588in}}%
\pgfpathlineto{\pgfqpoint{13.682151in}{0.773588in}}%
\pgfpathlineto{\pgfqpoint{13.635346in}{0.773588in}}%
\pgfpathlineto{\pgfqpoint{13.587249in}{0.773588in}}%
\pgfpathlineto{\pgfqpoint{13.541646in}{0.773588in}}%
\pgfpathlineto{\pgfqpoint{13.495727in}{0.773588in}}%
\pgfpathlineto{\pgfqpoint{13.448069in}{0.773588in}}%
\pgfpathlineto{\pgfqpoint{13.401859in}{0.773588in}}%
\pgfpathlineto{\pgfqpoint{13.356518in}{0.773588in}}%
\pgfpathlineto{\pgfqpoint{13.308974in}{0.773588in}}%
\pgfpathlineto{\pgfqpoint{13.262980in}{0.773588in}}%
\pgfpathlineto{\pgfqpoint{13.216719in}{0.773588in}}%
\pgfpathlineto{\pgfqpoint{13.169075in}{0.773588in}}%
\pgfpathlineto{\pgfqpoint{13.123339in}{0.773588in}}%
\pgfpathlineto{\pgfqpoint{13.077275in}{0.773588in}}%
\pgfpathlineto{\pgfqpoint{13.030499in}{0.773588in}}%
\pgfpathlineto{\pgfqpoint{12.983596in}{0.773588in}}%
\pgfpathlineto{\pgfqpoint{12.936799in}{0.773588in}}%
\pgfpathlineto{\pgfqpoint{12.889988in}{0.773588in}}%
\pgfpathlineto{\pgfqpoint{12.843155in}{0.773588in}}%
\pgfpathlineto{\pgfqpoint{12.796907in}{0.773588in}}%
\pgfpathlineto{\pgfqpoint{12.749025in}{0.773588in}}%
\pgfpathlineto{\pgfqpoint{12.701835in}{0.773588in}}%
\pgfpathlineto{\pgfqpoint{12.655424in}{0.773588in}}%
\pgfpathlineto{\pgfqpoint{12.607615in}{0.773588in}}%
\pgfpathlineto{\pgfqpoint{12.560711in}{0.773588in}}%
\pgfpathlineto{\pgfqpoint{12.514481in}{0.773588in}}%
\pgfpathlineto{\pgfqpoint{12.467453in}{0.773588in}}%
\pgfpathlineto{\pgfqpoint{12.421372in}{0.773588in}}%
\pgfpathlineto{\pgfqpoint{12.375261in}{0.773588in}}%
\pgfpathlineto{\pgfqpoint{12.328136in}{0.773588in}}%
\pgfpathlineto{\pgfqpoint{12.282271in}{0.773588in}}%
\pgfpathlineto{\pgfqpoint{12.235985in}{0.773588in}}%
\pgfpathlineto{\pgfqpoint{12.187925in}{0.773588in}}%
\pgfpathlineto{\pgfqpoint{12.141887in}{0.773588in}}%
\pgfpathlineto{\pgfqpoint{12.096623in}{0.773588in}}%
\pgfpathlineto{\pgfqpoint{12.049869in}{0.773588in}}%
\pgfpathlineto{\pgfqpoint{12.004414in}{0.773588in}}%
\pgfpathlineto{\pgfqpoint{11.959143in}{0.773588in}}%
\pgfpathlineto{\pgfqpoint{11.911501in}{0.773588in}}%
\pgfpathlineto{\pgfqpoint{11.864862in}{0.773588in}}%
\pgfpathlineto{\pgfqpoint{11.819406in}{0.773588in}}%
\pgfpathlineto{\pgfqpoint{11.772894in}{0.773588in}}%
\pgfpathlineto{\pgfqpoint{11.727607in}{0.773588in}}%
\pgfpathlineto{\pgfqpoint{11.682053in}{0.773588in}}%
\pgfpathlineto{\pgfqpoint{11.634589in}{0.773588in}}%
\pgfpathlineto{\pgfqpoint{11.588799in}{0.773588in}}%
\pgfpathlineto{\pgfqpoint{11.542665in}{0.773588in}}%
\pgfpathlineto{\pgfqpoint{11.494874in}{0.773588in}}%
\pgfpathlineto{\pgfqpoint{11.448531in}{0.773588in}}%
\pgfpathlineto{\pgfqpoint{11.402044in}{0.773588in}}%
\pgfpathlineto{\pgfqpoint{11.353756in}{0.773588in}}%
\pgfpathlineto{\pgfqpoint{11.307361in}{0.773588in}}%
\pgfpathlineto{\pgfqpoint{11.261350in}{0.773588in}}%
\pgfpathlineto{\pgfqpoint{11.213355in}{0.773588in}}%
\pgfpathlineto{\pgfqpoint{11.167471in}{0.773588in}}%
\pgfpathlineto{\pgfqpoint{11.121978in}{0.773588in}}%
\pgfpathlineto{\pgfqpoint{11.074633in}{0.773588in}}%
\pgfpathlineto{\pgfqpoint{11.029398in}{0.773588in}}%
\pgfpathlineto{\pgfqpoint{10.984145in}{0.773588in}}%
\pgfpathlineto{\pgfqpoint{10.937381in}{0.773588in}}%
\pgfpathlineto{\pgfqpoint{10.891801in}{0.773588in}}%
\pgfpathlineto{\pgfqpoint{10.845790in}{0.773588in}}%
\pgfpathlineto{\pgfqpoint{10.798768in}{0.773588in}}%
\pgfpathlineto{\pgfqpoint{10.753168in}{0.773588in}}%
\pgfpathlineto{\pgfqpoint{10.707461in}{0.773588in}}%
\pgfpathlineto{\pgfqpoint{10.660112in}{0.773588in}}%
\pgfpathlineto{\pgfqpoint{10.613886in}{0.773588in}}%
\pgfpathlineto{\pgfqpoint{10.568154in}{0.773588in}}%
\pgfpathlineto{\pgfqpoint{10.521105in}{0.773588in}}%
\pgfpathlineto{\pgfqpoint{10.475642in}{0.773588in}}%
\pgfpathlineto{\pgfqpoint{10.429362in}{0.773588in}}%
\pgfpathlineto{\pgfqpoint{10.381570in}{0.773588in}}%
\pgfpathlineto{\pgfqpoint{10.335334in}{0.773588in}}%
\pgfpathlineto{\pgfqpoint{10.289084in}{0.773588in}}%
\pgfpathlineto{\pgfqpoint{10.242601in}{0.773588in}}%
\pgfpathlineto{\pgfqpoint{10.197247in}{0.773588in}}%
\pgfpathlineto{\pgfqpoint{10.151348in}{0.773588in}}%
\pgfpathlineto{\pgfqpoint{10.103098in}{0.773588in}}%
\pgfpathlineto{\pgfqpoint{10.057613in}{0.773588in}}%
\pgfpathlineto{\pgfqpoint{10.011915in}{0.773588in}}%
\pgfpathlineto{\pgfqpoint{9.964562in}{0.773588in}}%
\pgfpathlineto{\pgfqpoint{9.918842in}{0.773588in}}%
\pgfpathlineto{\pgfqpoint{9.873173in}{0.773588in}}%
\pgfpathlineto{\pgfqpoint{9.826599in}{0.773588in}}%
\pgfpathlineto{\pgfqpoint{9.781321in}{0.773588in}}%
\pgfpathlineto{\pgfqpoint{9.735785in}{0.773588in}}%
\pgfpathlineto{\pgfqpoint{9.689497in}{0.773588in}}%
\pgfpathlineto{\pgfqpoint{9.644121in}{0.773588in}}%
\pgfpathlineto{\pgfqpoint{9.598338in}{0.773588in}}%
\pgfpathlineto{\pgfqpoint{9.552197in}{0.773588in}}%
\pgfpathlineto{\pgfqpoint{9.507583in}{0.773588in}}%
\pgfpathlineto{\pgfqpoint{9.462006in}{0.773588in}}%
\pgfpathlineto{\pgfqpoint{9.415178in}{0.773588in}}%
\pgfpathlineto{\pgfqpoint{9.369639in}{0.773588in}}%
\pgfpathlineto{\pgfqpoint{9.324423in}{0.773588in}}%
\pgfpathlineto{\pgfqpoint{9.278439in}{0.773588in}}%
\pgfpathlineto{\pgfqpoint{9.233107in}{0.773588in}}%
\pgfpathlineto{\pgfqpoint{9.188332in}{0.773588in}}%
\pgfpathlineto{\pgfqpoint{9.141270in}{0.773588in}}%
\pgfpathlineto{\pgfqpoint{9.095727in}{0.773588in}}%
\pgfpathlineto{\pgfqpoint{9.050165in}{0.773588in}}%
\pgfpathlineto{\pgfqpoint{9.003947in}{0.773588in}}%
\pgfpathlineto{\pgfqpoint{8.957924in}{0.773588in}}%
\pgfpathlineto{\pgfqpoint{8.912131in}{0.773588in}}%
\pgfpathlineto{\pgfqpoint{8.866242in}{0.773588in}}%
\pgfpathlineto{\pgfqpoint{8.821068in}{0.773588in}}%
\pgfpathlineto{\pgfqpoint{8.774902in}{0.773588in}}%
\pgfpathlineto{\pgfqpoint{8.727404in}{0.773588in}}%
\pgfpathlineto{\pgfqpoint{8.681080in}{0.773588in}}%
\pgfpathlineto{\pgfqpoint{8.635315in}{0.773588in}}%
\pgfpathlineto{\pgfqpoint{8.589124in}{0.773588in}}%
\pgfpathlineto{\pgfqpoint{8.543976in}{0.773588in}}%
\pgfpathlineto{\pgfqpoint{8.499298in}{0.773588in}}%
\pgfpathlineto{\pgfqpoint{8.453780in}{0.773588in}}%
\pgfpathlineto{\pgfqpoint{8.409767in}{0.773588in}}%
\pgfpathlineto{\pgfqpoint{8.364636in}{0.773588in}}%
\pgfpathlineto{\pgfqpoint{8.318789in}{0.773588in}}%
\pgfpathlineto{\pgfqpoint{8.273006in}{0.773588in}}%
\pgfpathlineto{\pgfqpoint{8.227930in}{0.773588in}}%
\pgfpathlineto{\pgfqpoint{8.181791in}{0.773588in}}%
\pgfpathlineto{\pgfqpoint{8.136842in}{0.773588in}}%
\pgfpathlineto{\pgfqpoint{8.091881in}{0.773588in}}%
\pgfpathlineto{\pgfqpoint{8.045278in}{0.773588in}}%
\pgfpathlineto{\pgfqpoint{8.000573in}{0.773588in}}%
\pgfpathlineto{\pgfqpoint{7.955879in}{0.773588in}}%
\pgfpathlineto{\pgfqpoint{7.910161in}{0.773588in}}%
\pgfpathlineto{\pgfqpoint{7.865263in}{0.773588in}}%
\pgfpathlineto{\pgfqpoint{7.819947in}{0.773588in}}%
\pgfpathlineto{\pgfqpoint{7.773226in}{0.773588in}}%
\pgfpathlineto{\pgfqpoint{7.728803in}{0.773588in}}%
\pgfpathlineto{\pgfqpoint{7.682978in}{0.773588in}}%
\pgfpathlineto{\pgfqpoint{7.636592in}{0.773588in}}%
\pgfpathlineto{\pgfqpoint{7.591890in}{0.773588in}}%
\pgfpathlineto{\pgfqpoint{7.546892in}{0.773588in}}%
\pgfpathlineto{\pgfqpoint{7.500683in}{0.773588in}}%
\pgfpathlineto{\pgfqpoint{7.455548in}{0.773588in}}%
\pgfpathlineto{\pgfqpoint{7.409977in}{0.773588in}}%
\pgfpathlineto{\pgfqpoint{7.363749in}{0.773588in}}%
\pgfpathlineto{\pgfqpoint{7.318484in}{0.773588in}}%
\pgfpathlineto{\pgfqpoint{7.273914in}{0.773588in}}%
\pgfpathlineto{\pgfqpoint{7.228120in}{0.773588in}}%
\pgfpathlineto{\pgfqpoint{7.184277in}{0.773588in}}%
\pgfpathlineto{\pgfqpoint{7.140134in}{0.773588in}}%
\pgfpathlineto{\pgfqpoint{7.094205in}{0.773588in}}%
\pgfpathlineto{\pgfqpoint{7.050071in}{0.773588in}}%
\pgfpathlineto{\pgfqpoint{7.005149in}{0.773588in}}%
\pgfpathlineto{\pgfqpoint{6.958763in}{0.773588in}}%
\pgfpathlineto{\pgfqpoint{6.914194in}{0.773588in}}%
\pgfpathlineto{\pgfqpoint{6.869544in}{0.773588in}}%
\pgfpathlineto{\pgfqpoint{6.824012in}{0.773588in}}%
\pgfpathlineto{\pgfqpoint{6.779295in}{0.773588in}}%
\pgfpathlineto{\pgfqpoint{6.734887in}{0.773588in}}%
\pgfpathlineto{\pgfqpoint{6.688504in}{0.773588in}}%
\pgfpathlineto{\pgfqpoint{6.643960in}{0.773588in}}%
\pgfpathlineto{\pgfqpoint{6.599302in}{0.773588in}}%
\pgfpathlineto{\pgfqpoint{6.553117in}{0.773588in}}%
\pgfpathlineto{\pgfqpoint{6.507651in}{0.773588in}}%
\pgfpathlineto{\pgfqpoint{6.461324in}{0.773588in}}%
\pgfpathlineto{\pgfqpoint{6.413399in}{0.773588in}}%
\pgfpathlineto{\pgfqpoint{6.367508in}{0.773588in}}%
\pgfpathlineto{\pgfqpoint{6.321070in}{0.773588in}}%
\pgfpathlineto{\pgfqpoint{6.273584in}{0.773588in}}%
\pgfpathlineto{\pgfqpoint{6.227112in}{0.773588in}}%
\pgfpathlineto{\pgfqpoint{6.180802in}{0.773588in}}%
\pgfpathlineto{\pgfqpoint{6.133235in}{0.773588in}}%
\pgfpathlineto{\pgfqpoint{6.087229in}{0.773588in}}%
\pgfpathlineto{\pgfqpoint{6.041630in}{0.773588in}}%
\pgfpathlineto{\pgfqpoint{5.994686in}{0.773588in}}%
\pgfpathlineto{\pgfqpoint{5.948991in}{0.773588in}}%
\pgfpathlineto{\pgfqpoint{5.903366in}{0.773588in}}%
\pgfpathlineto{\pgfqpoint{5.856181in}{0.773588in}}%
\pgfpathlineto{\pgfqpoint{5.810269in}{0.773588in}}%
\pgfpathlineto{\pgfqpoint{5.765813in}{0.773588in}}%
\pgfpathlineto{\pgfqpoint{5.719569in}{0.773588in}}%
\pgfpathlineto{\pgfqpoint{5.674262in}{0.773588in}}%
\pgfpathlineto{\pgfqpoint{5.628857in}{0.773588in}}%
\pgfpathlineto{\pgfqpoint{5.581803in}{0.773588in}}%
\pgfpathlineto{\pgfqpoint{5.536203in}{0.773588in}}%
\pgfpathlineto{\pgfqpoint{5.491128in}{0.773588in}}%
\pgfpathlineto{\pgfqpoint{5.444526in}{0.773588in}}%
\pgfpathlineto{\pgfqpoint{5.399334in}{0.773588in}}%
\pgfpathlineto{\pgfqpoint{5.353614in}{0.773588in}}%
\pgfpathlineto{\pgfqpoint{5.307244in}{0.773588in}}%
\pgfpathlineto{\pgfqpoint{5.262146in}{0.773588in}}%
\pgfpathlineto{\pgfqpoint{5.216291in}{0.773588in}}%
\pgfpathlineto{\pgfqpoint{5.169192in}{0.773588in}}%
\pgfpathlineto{\pgfqpoint{5.124036in}{0.773588in}}%
\pgfpathlineto{\pgfqpoint{5.078074in}{0.773588in}}%
\pgfpathlineto{\pgfqpoint{5.030678in}{0.773588in}}%
\pgfpathlineto{\pgfqpoint{4.984973in}{0.773588in}}%
\pgfpathlineto{\pgfqpoint{4.939032in}{0.773588in}}%
\pgfpathlineto{\pgfqpoint{4.891298in}{0.773588in}}%
\pgfpathlineto{\pgfqpoint{4.845429in}{0.773588in}}%
\pgfpathlineto{\pgfqpoint{4.799887in}{0.773588in}}%
\pgfpathlineto{\pgfqpoint{4.753449in}{0.773588in}}%
\pgfpathlineto{\pgfqpoint{4.708228in}{0.773588in}}%
\pgfpathlineto{\pgfqpoint{4.663622in}{0.773588in}}%
\pgfpathlineto{\pgfqpoint{4.617918in}{0.773588in}}%
\pgfpathlineto{\pgfqpoint{4.572708in}{0.773588in}}%
\pgfpathlineto{\pgfqpoint{4.527424in}{0.773588in}}%
\pgfpathlineto{\pgfqpoint{4.480684in}{0.773588in}}%
\pgfpathlineto{\pgfqpoint{4.435284in}{0.773588in}}%
\pgfpathlineto{\pgfqpoint{4.390187in}{0.773588in}}%
\pgfpathlineto{\pgfqpoint{4.343118in}{0.773588in}}%
\pgfpathlineto{\pgfqpoint{4.297354in}{0.773588in}}%
\pgfpathlineto{\pgfqpoint{4.252389in}{0.773588in}}%
\pgfpathlineto{\pgfqpoint{4.204543in}{0.773588in}}%
\pgfpathlineto{\pgfqpoint{4.159078in}{0.773588in}}%
\pgfpathlineto{\pgfqpoint{4.114362in}{0.773588in}}%
\pgfpathlineto{\pgfqpoint{4.067507in}{0.773588in}}%
\pgfpathlineto{\pgfqpoint{4.022115in}{0.773588in}}%
\pgfpathlineto{\pgfqpoint{3.976868in}{0.773588in}}%
\pgfpathlineto{\pgfqpoint{3.930307in}{0.773588in}}%
\pgfpathlineto{\pgfqpoint{3.884614in}{0.773588in}}%
\pgfpathlineto{\pgfqpoint{3.839775in}{0.773588in}}%
\pgfpathlineto{\pgfqpoint{3.793040in}{0.773588in}}%
\pgfpathlineto{\pgfqpoint{3.747249in}{0.773588in}}%
\pgfpathlineto{\pgfqpoint{3.701949in}{0.773588in}}%
\pgfpathlineto{\pgfqpoint{3.654909in}{0.773588in}}%
\pgfpathlineto{\pgfqpoint{3.609556in}{0.773588in}}%
\pgfpathlineto{\pgfqpoint{3.564489in}{0.773588in}}%
\pgfpathlineto{\pgfqpoint{3.517815in}{0.773588in}}%
\pgfpathlineto{\pgfqpoint{3.473135in}{0.773588in}}%
\pgfpathlineto{\pgfqpoint{3.428526in}{0.773588in}}%
\pgfpathlineto{\pgfqpoint{3.381274in}{0.773588in}}%
\pgfpathlineto{\pgfqpoint{3.336184in}{0.773588in}}%
\pgfpathlineto{\pgfqpoint{3.290748in}{0.773588in}}%
\pgfpathlineto{\pgfqpoint{3.245458in}{0.773588in}}%
\pgfpathlineto{\pgfqpoint{3.200519in}{0.773588in}}%
\pgfpathlineto{\pgfqpoint{3.155580in}{0.773588in}}%
\pgfpathlineto{\pgfqpoint{3.108180in}{0.773588in}}%
\pgfpathlineto{\pgfqpoint{3.062761in}{0.773588in}}%
\pgfpathlineto{\pgfqpoint{3.017486in}{0.773588in}}%
\pgfpathlineto{\pgfqpoint{2.971917in}{0.773588in}}%
\pgfpathlineto{\pgfqpoint{2.927413in}{0.773588in}}%
\pgfpathlineto{\pgfqpoint{2.883134in}{0.773588in}}%
\pgfpathlineto{\pgfqpoint{2.836381in}{0.773588in}}%
\pgfpathlineto{\pgfqpoint{2.790736in}{0.773588in}}%
\pgfpathlineto{\pgfqpoint{2.745668in}{0.773588in}}%
\pgfpathlineto{\pgfqpoint{2.698034in}{0.773588in}}%
\pgfpathlineto{\pgfqpoint{2.651003in}{0.773588in}}%
\pgfpathlineto{\pgfqpoint{2.604306in}{0.773588in}}%
\pgfpathlineto{\pgfqpoint{2.555498in}{0.773588in}}%
\pgfpathlineto{\pgfqpoint{2.505534in}{0.773588in}}%
\pgfpathlineto{\pgfqpoint{2.452591in}{0.773588in}}%
\pgfpathlineto{\pgfqpoint{2.397147in}{0.773588in}}%
\pgfpathlineto{\pgfqpoint{2.348431in}{0.773588in}}%
\pgfpathlineto{\pgfqpoint{2.299591in}{0.773588in}}%
\pgfpathlineto{\pgfqpoint{2.249804in}{0.773588in}}%
\pgfpathlineto{\pgfqpoint{2.201242in}{0.773588in}}%
\pgfpathlineto{\pgfqpoint{2.153284in}{0.773588in}}%
\pgfpathlineto{\pgfqpoint{2.104325in}{0.773588in}}%
\pgfpathlineto{\pgfqpoint{2.057441in}{0.773588in}}%
\pgfpathlineto{\pgfqpoint{2.011209in}{0.773588in}}%
\pgfpathlineto{\pgfqpoint{1.963476in}{0.773588in}}%
\pgfpathlineto{\pgfqpoint{1.915908in}{0.773588in}}%
\pgfpathlineto{\pgfqpoint{1.869493in}{0.773588in}}%
\pgfpathlineto{\pgfqpoint{1.822804in}{0.773588in}}%
\pgfpathlineto{\pgfqpoint{1.778099in}{0.773588in}}%
\pgfpathlineto{\pgfqpoint{1.734500in}{0.773588in}}%
\pgfpathlineto{\pgfqpoint{1.689326in}{0.773588in}}%
\pgfpathlineto{\pgfqpoint{1.645274in}{0.773588in}}%
\pgfpathlineto{\pgfqpoint{1.600816in}{0.773588in}}%
\pgfpathlineto{\pgfqpoint{1.555718in}{0.773588in}}%
\pgfpathlineto{\pgfqpoint{1.511740in}{0.773588in}}%
\pgfpathlineto{\pgfqpoint{1.468334in}{0.773588in}}%
\pgfpathlineto{\pgfqpoint{1.422957in}{0.773588in}}%
\pgfpathlineto{\pgfqpoint{1.378287in}{0.773588in}}%
\pgfpathlineto{\pgfqpoint{1.334262in}{0.773588in}}%
\pgfpathlineto{\pgfqpoint{1.289074in}{0.773588in}}%
\pgfpathlineto{\pgfqpoint{1.244609in}{0.773588in}}%
\pgfpathlineto{\pgfqpoint{1.200192in}{0.773588in}}%
\pgfpathlineto{\pgfqpoint{1.155171in}{0.773588in}}%
\pgfpathlineto{\pgfqpoint{1.111790in}{0.773588in}}%
\pgfpathlineto{\pgfqpoint{1.067773in}{0.773588in}}%
\pgfpathlineto{\pgfqpoint{1.021908in}{0.773588in}}%
\pgfpathlineto{\pgfqpoint{0.978015in}{0.773588in}}%
\pgfpathlineto{\pgfqpoint{0.933783in}{0.773588in}}%
\pgfpathlineto{\pgfqpoint{0.887244in}{0.773588in}}%
\pgfpathlineto{\pgfqpoint{0.842612in}{0.773588in}}%
\pgfpathlineto{\pgfqpoint{0.797895in}{0.773588in}}%
\pgfpathclose%
\pgfusepath{fill}%
\end{pgfscope}%
\begin{pgfscope}%
\pgfpathrectangle{\pgfqpoint{0.781402in}{0.773588in}}{\pgfqpoint{1.440244in}{5.415119in}}%
\pgfusepath{clip}%
\pgfsetbuttcap%
\pgfsetroundjoin%
\definecolor{currentfill}{rgb}{1.000000,0.498039,0.054902}%
\pgfsetfillcolor{currentfill}%
\pgfsetlinewidth{0.000000pt}%
\definecolor{currentstroke}{rgb}{0.000000,0.000000,0.000000}%
\pgfsetstrokecolor{currentstroke}%
\pgfsetdash{}{0pt}%
\pgfpathmoveto{\pgfqpoint{0.797895in}{0.773588in}}%
\pgfpathlineto{\pgfqpoint{0.797895in}{0.773588in}}%
\pgfpathlineto{\pgfqpoint{0.842612in}{0.773588in}}%
\pgfpathlineto{\pgfqpoint{0.887244in}{0.773588in}}%
\pgfpathlineto{\pgfqpoint{0.933783in}{0.773588in}}%
\pgfpathlineto{\pgfqpoint{0.978015in}{0.773588in}}%
\pgfpathlineto{\pgfqpoint{1.021908in}{0.773588in}}%
\pgfpathlineto{\pgfqpoint{1.067773in}{0.773588in}}%
\pgfpathlineto{\pgfqpoint{1.111790in}{0.773588in}}%
\pgfpathlineto{\pgfqpoint{1.155171in}{0.773588in}}%
\pgfpathlineto{\pgfqpoint{1.200192in}{0.773588in}}%
\pgfpathlineto{\pgfqpoint{1.244609in}{0.773588in}}%
\pgfpathlineto{\pgfqpoint{1.289074in}{0.773588in}}%
\pgfpathlineto{\pgfqpoint{1.334262in}{0.773588in}}%
\pgfpathlineto{\pgfqpoint{1.378287in}{0.773588in}}%
\pgfpathlineto{\pgfqpoint{1.422957in}{0.773588in}}%
\pgfpathlineto{\pgfqpoint{1.468334in}{0.773588in}}%
\pgfpathlineto{\pgfqpoint{1.511740in}{0.773588in}}%
\pgfpathlineto{\pgfqpoint{1.555718in}{0.773588in}}%
\pgfpathlineto{\pgfqpoint{1.600816in}{0.773588in}}%
\pgfpathlineto{\pgfqpoint{1.645274in}{0.773588in}}%
\pgfpathlineto{\pgfqpoint{1.689326in}{0.773588in}}%
\pgfpathlineto{\pgfqpoint{1.734500in}{0.773588in}}%
\pgfpathlineto{\pgfqpoint{1.778099in}{0.773588in}}%
\pgfpathlineto{\pgfqpoint{1.822804in}{0.773588in}}%
\pgfpathlineto{\pgfqpoint{1.869493in}{0.773588in}}%
\pgfpathlineto{\pgfqpoint{1.915908in}{0.773588in}}%
\pgfpathlineto{\pgfqpoint{1.963476in}{0.773588in}}%
\pgfpathlineto{\pgfqpoint{2.011209in}{0.773588in}}%
\pgfpathlineto{\pgfqpoint{2.057441in}{0.773588in}}%
\pgfpathlineto{\pgfqpoint{2.104325in}{0.773588in}}%
\pgfpathlineto{\pgfqpoint{2.153284in}{0.773588in}}%
\pgfpathlineto{\pgfqpoint{2.201242in}{0.773588in}}%
\pgfpathlineto{\pgfqpoint{2.249804in}{0.773588in}}%
\pgfpathlineto{\pgfqpoint{2.299591in}{0.773588in}}%
\pgfpathlineto{\pgfqpoint{2.348431in}{0.773588in}}%
\pgfpathlineto{\pgfqpoint{2.397147in}{0.773588in}}%
\pgfpathlineto{\pgfqpoint{2.452591in}{0.773588in}}%
\pgfpathlineto{\pgfqpoint{2.505534in}{0.773588in}}%
\pgfpathlineto{\pgfqpoint{2.555498in}{0.773588in}}%
\pgfpathlineto{\pgfqpoint{2.604306in}{0.773588in}}%
\pgfpathlineto{\pgfqpoint{2.651003in}{0.773588in}}%
\pgfpathlineto{\pgfqpoint{2.698034in}{0.773588in}}%
\pgfpathlineto{\pgfqpoint{2.745668in}{0.773588in}}%
\pgfpathlineto{\pgfqpoint{2.790736in}{0.773588in}}%
\pgfpathlineto{\pgfqpoint{2.836381in}{0.773588in}}%
\pgfpathlineto{\pgfqpoint{2.883134in}{0.773588in}}%
\pgfpathlineto{\pgfqpoint{2.927413in}{0.773588in}}%
\pgfpathlineto{\pgfqpoint{2.971917in}{0.773588in}}%
\pgfpathlineto{\pgfqpoint{3.017486in}{0.773588in}}%
\pgfpathlineto{\pgfqpoint{3.062761in}{0.773588in}}%
\pgfpathlineto{\pgfqpoint{3.108180in}{0.773588in}}%
\pgfpathlineto{\pgfqpoint{3.155580in}{0.773588in}}%
\pgfpathlineto{\pgfqpoint{3.200519in}{0.773588in}}%
\pgfpathlineto{\pgfqpoint{3.245458in}{0.773588in}}%
\pgfpathlineto{\pgfqpoint{3.290748in}{0.773588in}}%
\pgfpathlineto{\pgfqpoint{3.336184in}{0.773588in}}%
\pgfpathlineto{\pgfqpoint{3.381274in}{0.773588in}}%
\pgfpathlineto{\pgfqpoint{3.428526in}{0.773588in}}%
\pgfpathlineto{\pgfqpoint{3.473135in}{0.773588in}}%
\pgfpathlineto{\pgfqpoint{3.517815in}{0.773588in}}%
\pgfpathlineto{\pgfqpoint{3.564489in}{0.773588in}}%
\pgfpathlineto{\pgfqpoint{3.609556in}{0.773588in}}%
\pgfpathlineto{\pgfqpoint{3.654909in}{0.773588in}}%
\pgfpathlineto{\pgfqpoint{3.701949in}{0.773588in}}%
\pgfpathlineto{\pgfqpoint{3.747249in}{0.773588in}}%
\pgfpathlineto{\pgfqpoint{3.793040in}{0.773588in}}%
\pgfpathlineto{\pgfqpoint{3.839775in}{0.773588in}}%
\pgfpathlineto{\pgfqpoint{3.884614in}{0.773588in}}%
\pgfpathlineto{\pgfqpoint{3.930307in}{0.773588in}}%
\pgfpathlineto{\pgfqpoint{3.976868in}{0.773588in}}%
\pgfpathlineto{\pgfqpoint{4.022115in}{0.773588in}}%
\pgfpathlineto{\pgfqpoint{4.067507in}{0.773588in}}%
\pgfpathlineto{\pgfqpoint{4.114362in}{0.773588in}}%
\pgfpathlineto{\pgfqpoint{4.159078in}{0.773588in}}%
\pgfpathlineto{\pgfqpoint{4.204543in}{0.773588in}}%
\pgfpathlineto{\pgfqpoint{4.252389in}{0.773588in}}%
\pgfpathlineto{\pgfqpoint{4.297354in}{0.773588in}}%
\pgfpathlineto{\pgfqpoint{4.343118in}{0.773588in}}%
\pgfpathlineto{\pgfqpoint{4.390187in}{0.773588in}}%
\pgfpathlineto{\pgfqpoint{4.435284in}{0.773588in}}%
\pgfpathlineto{\pgfqpoint{4.480684in}{0.773588in}}%
\pgfpathlineto{\pgfqpoint{4.527424in}{0.773588in}}%
\pgfpathlineto{\pgfqpoint{4.572708in}{0.773588in}}%
\pgfpathlineto{\pgfqpoint{4.617918in}{0.773588in}}%
\pgfpathlineto{\pgfqpoint{4.663622in}{0.773588in}}%
\pgfpathlineto{\pgfqpoint{4.708228in}{0.773588in}}%
\pgfpathlineto{\pgfqpoint{4.753449in}{0.773588in}}%
\pgfpathlineto{\pgfqpoint{4.799887in}{0.773588in}}%
\pgfpathlineto{\pgfqpoint{4.845429in}{0.773588in}}%
\pgfpathlineto{\pgfqpoint{4.891298in}{0.773588in}}%
\pgfpathlineto{\pgfqpoint{4.939032in}{0.773588in}}%
\pgfpathlineto{\pgfqpoint{4.984973in}{0.773588in}}%
\pgfpathlineto{\pgfqpoint{5.030678in}{0.773588in}}%
\pgfpathlineto{\pgfqpoint{5.078074in}{0.773588in}}%
\pgfpathlineto{\pgfqpoint{5.124036in}{0.773588in}}%
\pgfpathlineto{\pgfqpoint{5.169192in}{0.773588in}}%
\pgfpathlineto{\pgfqpoint{5.216291in}{0.773588in}}%
\pgfpathlineto{\pgfqpoint{5.262146in}{0.773588in}}%
\pgfpathlineto{\pgfqpoint{5.307244in}{0.773588in}}%
\pgfpathlineto{\pgfqpoint{5.353614in}{0.773588in}}%
\pgfpathlineto{\pgfqpoint{5.399334in}{0.773588in}}%
\pgfpathlineto{\pgfqpoint{5.444526in}{0.773588in}}%
\pgfpathlineto{\pgfqpoint{5.491128in}{0.773588in}}%
\pgfpathlineto{\pgfqpoint{5.536203in}{0.773588in}}%
\pgfpathlineto{\pgfqpoint{5.581803in}{0.773588in}}%
\pgfpathlineto{\pgfqpoint{5.628857in}{0.773588in}}%
\pgfpathlineto{\pgfqpoint{5.674262in}{0.773588in}}%
\pgfpathlineto{\pgfqpoint{5.719569in}{0.773588in}}%
\pgfpathlineto{\pgfqpoint{5.765813in}{0.773588in}}%
\pgfpathlineto{\pgfqpoint{5.810269in}{0.773588in}}%
\pgfpathlineto{\pgfqpoint{5.856181in}{0.773588in}}%
\pgfpathlineto{\pgfqpoint{5.903366in}{0.773588in}}%
\pgfpathlineto{\pgfqpoint{5.948991in}{0.773588in}}%
\pgfpathlineto{\pgfqpoint{5.994686in}{0.773588in}}%
\pgfpathlineto{\pgfqpoint{6.041630in}{0.773588in}}%
\pgfpathlineto{\pgfqpoint{6.087229in}{0.773588in}}%
\pgfpathlineto{\pgfqpoint{6.133235in}{0.773588in}}%
\pgfpathlineto{\pgfqpoint{6.180802in}{0.773588in}}%
\pgfpathlineto{\pgfqpoint{6.227112in}{0.773588in}}%
\pgfpathlineto{\pgfqpoint{6.273584in}{0.773588in}}%
\pgfpathlineto{\pgfqpoint{6.321070in}{0.773588in}}%
\pgfpathlineto{\pgfqpoint{6.367508in}{0.773588in}}%
\pgfpathlineto{\pgfqpoint{6.413399in}{0.773588in}}%
\pgfpathlineto{\pgfqpoint{6.461324in}{0.773588in}}%
\pgfpathlineto{\pgfqpoint{6.507651in}{0.773588in}}%
\pgfpathlineto{\pgfqpoint{6.553117in}{0.773588in}}%
\pgfpathlineto{\pgfqpoint{6.599302in}{0.773588in}}%
\pgfpathlineto{\pgfqpoint{6.643960in}{0.773588in}}%
\pgfpathlineto{\pgfqpoint{6.688504in}{0.773588in}}%
\pgfpathlineto{\pgfqpoint{6.734887in}{0.773588in}}%
\pgfpathlineto{\pgfqpoint{6.779295in}{0.773588in}}%
\pgfpathlineto{\pgfqpoint{6.824012in}{0.773588in}}%
\pgfpathlineto{\pgfqpoint{6.869544in}{0.773588in}}%
\pgfpathlineto{\pgfqpoint{6.914194in}{0.773588in}}%
\pgfpathlineto{\pgfqpoint{6.958763in}{0.773588in}}%
\pgfpathlineto{\pgfqpoint{7.005149in}{0.773588in}}%
\pgfpathlineto{\pgfqpoint{7.050071in}{0.773588in}}%
\pgfpathlineto{\pgfqpoint{7.094205in}{0.773588in}}%
\pgfpathlineto{\pgfqpoint{7.140134in}{0.773588in}}%
\pgfpathlineto{\pgfqpoint{7.184277in}{0.773588in}}%
\pgfpathlineto{\pgfqpoint{7.228120in}{0.773588in}}%
\pgfpathlineto{\pgfqpoint{7.273914in}{0.773588in}}%
\pgfpathlineto{\pgfqpoint{7.318484in}{0.773588in}}%
\pgfpathlineto{\pgfqpoint{7.363749in}{0.773588in}}%
\pgfpathlineto{\pgfqpoint{7.409977in}{0.773588in}}%
\pgfpathlineto{\pgfqpoint{7.455548in}{0.773588in}}%
\pgfpathlineto{\pgfqpoint{7.500683in}{0.773588in}}%
\pgfpathlineto{\pgfqpoint{7.546892in}{0.773588in}}%
\pgfpathlineto{\pgfqpoint{7.591890in}{0.773588in}}%
\pgfpathlineto{\pgfqpoint{7.636592in}{0.773588in}}%
\pgfpathlineto{\pgfqpoint{7.682978in}{0.773588in}}%
\pgfpathlineto{\pgfqpoint{7.728803in}{0.773588in}}%
\pgfpathlineto{\pgfqpoint{7.773226in}{0.773588in}}%
\pgfpathlineto{\pgfqpoint{7.819947in}{0.773588in}}%
\pgfpathlineto{\pgfqpoint{7.865263in}{0.773588in}}%
\pgfpathlineto{\pgfqpoint{7.910161in}{0.773588in}}%
\pgfpathlineto{\pgfqpoint{7.955879in}{0.773588in}}%
\pgfpathlineto{\pgfqpoint{8.000573in}{0.773588in}}%
\pgfpathlineto{\pgfqpoint{8.045278in}{0.773588in}}%
\pgfpathlineto{\pgfqpoint{8.091881in}{0.773588in}}%
\pgfpathlineto{\pgfqpoint{8.136842in}{0.773588in}}%
\pgfpathlineto{\pgfqpoint{8.181791in}{0.773588in}}%
\pgfpathlineto{\pgfqpoint{8.227930in}{0.773588in}}%
\pgfpathlineto{\pgfqpoint{8.273006in}{0.773588in}}%
\pgfpathlineto{\pgfqpoint{8.318789in}{0.773588in}}%
\pgfpathlineto{\pgfqpoint{8.364636in}{0.773588in}}%
\pgfpathlineto{\pgfqpoint{8.409767in}{0.773588in}}%
\pgfpathlineto{\pgfqpoint{8.453780in}{0.773588in}}%
\pgfpathlineto{\pgfqpoint{8.499298in}{0.773588in}}%
\pgfpathlineto{\pgfqpoint{8.543976in}{0.773588in}}%
\pgfpathlineto{\pgfqpoint{8.589124in}{0.773588in}}%
\pgfpathlineto{\pgfqpoint{8.635315in}{0.773588in}}%
\pgfpathlineto{\pgfqpoint{8.681080in}{0.773588in}}%
\pgfpathlineto{\pgfqpoint{8.727404in}{0.773588in}}%
\pgfpathlineto{\pgfqpoint{8.774902in}{0.773588in}}%
\pgfpathlineto{\pgfqpoint{8.821068in}{0.773588in}}%
\pgfpathlineto{\pgfqpoint{8.866242in}{0.773588in}}%
\pgfpathlineto{\pgfqpoint{8.912131in}{0.773588in}}%
\pgfpathlineto{\pgfqpoint{8.957924in}{0.773588in}}%
\pgfpathlineto{\pgfqpoint{9.003947in}{0.773588in}}%
\pgfpathlineto{\pgfqpoint{9.050165in}{0.773588in}}%
\pgfpathlineto{\pgfqpoint{9.095727in}{0.773588in}}%
\pgfpathlineto{\pgfqpoint{9.141270in}{0.773588in}}%
\pgfpathlineto{\pgfqpoint{9.188332in}{0.773588in}}%
\pgfpathlineto{\pgfqpoint{9.233107in}{0.773588in}}%
\pgfpathlineto{\pgfqpoint{9.278439in}{0.773588in}}%
\pgfpathlineto{\pgfqpoint{9.324423in}{0.773588in}}%
\pgfpathlineto{\pgfqpoint{9.369639in}{0.773588in}}%
\pgfpathlineto{\pgfqpoint{9.415178in}{0.773588in}}%
\pgfpathlineto{\pgfqpoint{9.462006in}{0.773588in}}%
\pgfpathlineto{\pgfqpoint{9.507583in}{0.773588in}}%
\pgfpathlineto{\pgfqpoint{9.552197in}{0.773588in}}%
\pgfpathlineto{\pgfqpoint{9.598338in}{0.773588in}}%
\pgfpathlineto{\pgfqpoint{9.644121in}{0.773588in}}%
\pgfpathlineto{\pgfqpoint{9.689497in}{0.773588in}}%
\pgfpathlineto{\pgfqpoint{9.735785in}{0.773588in}}%
\pgfpathlineto{\pgfqpoint{9.781321in}{0.773588in}}%
\pgfpathlineto{\pgfqpoint{9.826599in}{0.773588in}}%
\pgfpathlineto{\pgfqpoint{9.873173in}{0.773588in}}%
\pgfpathlineto{\pgfqpoint{9.918842in}{0.773588in}}%
\pgfpathlineto{\pgfqpoint{9.964562in}{0.773588in}}%
\pgfpathlineto{\pgfqpoint{10.011915in}{0.773588in}}%
\pgfpathlineto{\pgfqpoint{10.057613in}{0.773588in}}%
\pgfpathlineto{\pgfqpoint{10.103098in}{0.773588in}}%
\pgfpathlineto{\pgfqpoint{10.151348in}{0.773588in}}%
\pgfpathlineto{\pgfqpoint{10.197247in}{0.773588in}}%
\pgfpathlineto{\pgfqpoint{10.242601in}{0.773588in}}%
\pgfpathlineto{\pgfqpoint{10.289084in}{0.773588in}}%
\pgfpathlineto{\pgfqpoint{10.335334in}{0.773588in}}%
\pgfpathlineto{\pgfqpoint{10.381570in}{0.773588in}}%
\pgfpathlineto{\pgfqpoint{10.429362in}{0.773588in}}%
\pgfpathlineto{\pgfqpoint{10.475642in}{0.773588in}}%
\pgfpathlineto{\pgfqpoint{10.521105in}{0.773588in}}%
\pgfpathlineto{\pgfqpoint{10.568154in}{0.773588in}}%
\pgfpathlineto{\pgfqpoint{10.613886in}{0.773588in}}%
\pgfpathlineto{\pgfqpoint{10.660112in}{0.773588in}}%
\pgfpathlineto{\pgfqpoint{10.707461in}{0.773588in}}%
\pgfpathlineto{\pgfqpoint{10.753168in}{0.773588in}}%
\pgfpathlineto{\pgfqpoint{10.798768in}{0.773588in}}%
\pgfpathlineto{\pgfqpoint{10.845790in}{0.773588in}}%
\pgfpathlineto{\pgfqpoint{10.891801in}{0.773588in}}%
\pgfpathlineto{\pgfqpoint{10.937381in}{0.773588in}}%
\pgfpathlineto{\pgfqpoint{10.984145in}{0.773588in}}%
\pgfpathlineto{\pgfqpoint{11.029398in}{0.773588in}}%
\pgfpathlineto{\pgfqpoint{11.074633in}{0.773588in}}%
\pgfpathlineto{\pgfqpoint{11.121978in}{0.773588in}}%
\pgfpathlineto{\pgfqpoint{11.167471in}{0.773588in}}%
\pgfpathlineto{\pgfqpoint{11.213355in}{0.773588in}}%
\pgfpathlineto{\pgfqpoint{11.261350in}{0.773588in}}%
\pgfpathlineto{\pgfqpoint{11.307361in}{0.773588in}}%
\pgfpathlineto{\pgfqpoint{11.353756in}{0.773588in}}%
\pgfpathlineto{\pgfqpoint{11.402044in}{0.773588in}}%
\pgfpathlineto{\pgfqpoint{11.448531in}{0.773588in}}%
\pgfpathlineto{\pgfqpoint{11.494874in}{0.773588in}}%
\pgfpathlineto{\pgfqpoint{11.542665in}{0.773588in}}%
\pgfpathlineto{\pgfqpoint{11.588799in}{0.773588in}}%
\pgfpathlineto{\pgfqpoint{11.634589in}{0.773588in}}%
\pgfpathlineto{\pgfqpoint{11.682053in}{0.773588in}}%
\pgfpathlineto{\pgfqpoint{11.727607in}{0.773588in}}%
\pgfpathlineto{\pgfqpoint{11.772894in}{0.773588in}}%
\pgfpathlineto{\pgfqpoint{11.819406in}{0.773588in}}%
\pgfpathlineto{\pgfqpoint{11.864862in}{0.773588in}}%
\pgfpathlineto{\pgfqpoint{11.911501in}{0.773588in}}%
\pgfpathlineto{\pgfqpoint{11.959143in}{0.773588in}}%
\pgfpathlineto{\pgfqpoint{12.004414in}{0.773588in}}%
\pgfpathlineto{\pgfqpoint{12.049869in}{0.773588in}}%
\pgfpathlineto{\pgfqpoint{12.096623in}{0.773588in}}%
\pgfpathlineto{\pgfqpoint{12.141887in}{0.773588in}}%
\pgfpathlineto{\pgfqpoint{12.187925in}{0.773588in}}%
\pgfpathlineto{\pgfqpoint{12.235985in}{0.773588in}}%
\pgfpathlineto{\pgfqpoint{12.282271in}{0.773588in}}%
\pgfpathlineto{\pgfqpoint{12.328136in}{0.773588in}}%
\pgfpathlineto{\pgfqpoint{12.375261in}{0.773588in}}%
\pgfpathlineto{\pgfqpoint{12.421372in}{0.773588in}}%
\pgfpathlineto{\pgfqpoint{12.467453in}{0.773588in}}%
\pgfpathlineto{\pgfqpoint{12.514481in}{0.773588in}}%
\pgfpathlineto{\pgfqpoint{12.560711in}{0.773588in}}%
\pgfpathlineto{\pgfqpoint{12.607615in}{0.773588in}}%
\pgfpathlineto{\pgfqpoint{12.655424in}{0.773588in}}%
\pgfpathlineto{\pgfqpoint{12.701835in}{0.773588in}}%
\pgfpathlineto{\pgfqpoint{12.749025in}{0.773588in}}%
\pgfpathlineto{\pgfqpoint{12.796907in}{0.773588in}}%
\pgfpathlineto{\pgfqpoint{12.843155in}{0.773588in}}%
\pgfpathlineto{\pgfqpoint{12.889988in}{0.773588in}}%
\pgfpathlineto{\pgfqpoint{12.936799in}{0.773588in}}%
\pgfpathlineto{\pgfqpoint{12.983596in}{0.773588in}}%
\pgfpathlineto{\pgfqpoint{13.030499in}{0.773588in}}%
\pgfpathlineto{\pgfqpoint{13.077275in}{0.773588in}}%
\pgfpathlineto{\pgfqpoint{13.123339in}{0.773588in}}%
\pgfpathlineto{\pgfqpoint{13.169075in}{0.773588in}}%
\pgfpathlineto{\pgfqpoint{13.216719in}{0.773588in}}%
\pgfpathlineto{\pgfqpoint{13.262980in}{0.773588in}}%
\pgfpathlineto{\pgfqpoint{13.308974in}{0.773588in}}%
\pgfpathlineto{\pgfqpoint{13.356518in}{0.773588in}}%
\pgfpathlineto{\pgfqpoint{13.401859in}{0.773588in}}%
\pgfpathlineto{\pgfqpoint{13.448069in}{0.773588in}}%
\pgfpathlineto{\pgfqpoint{13.495727in}{0.773588in}}%
\pgfpathlineto{\pgfqpoint{13.541646in}{0.773588in}}%
\pgfpathlineto{\pgfqpoint{13.587249in}{0.773588in}}%
\pgfpathlineto{\pgfqpoint{13.635346in}{0.773588in}}%
\pgfpathlineto{\pgfqpoint{13.682151in}{0.773588in}}%
\pgfpathlineto{\pgfqpoint{13.729143in}{0.773588in}}%
\pgfpathlineto{\pgfqpoint{13.777441in}{0.773588in}}%
\pgfpathlineto{\pgfqpoint{13.825297in}{0.773588in}}%
\pgfpathlineto{\pgfqpoint{13.873610in}{0.773588in}}%
\pgfpathlineto{\pgfqpoint{13.922606in}{0.773588in}}%
\pgfpathlineto{\pgfqpoint{13.969353in}{0.773588in}}%
\pgfpathlineto{\pgfqpoint{14.016344in}{0.773588in}}%
\pgfpathlineto{\pgfqpoint{14.064951in}{0.773588in}}%
\pgfpathlineto{\pgfqpoint{14.112147in}{0.773588in}}%
\pgfpathlineto{\pgfqpoint{14.160147in}{0.773588in}}%
\pgfpathlineto{\pgfqpoint{14.209376in}{0.773588in}}%
\pgfpathlineto{\pgfqpoint{14.255593in}{0.773588in}}%
\pgfpathlineto{\pgfqpoint{14.302418in}{0.773588in}}%
\pgfpathlineto{\pgfqpoint{14.350224in}{0.773588in}}%
\pgfpathlineto{\pgfqpoint{14.397126in}{0.773588in}}%
\pgfpathlineto{\pgfqpoint{14.443729in}{0.773588in}}%
\pgfpathlineto{\pgfqpoint{14.491726in}{0.773588in}}%
\pgfpathlineto{\pgfqpoint{14.538630in}{0.773588in}}%
\pgfpathlineto{\pgfqpoint{14.585477in}{0.773588in}}%
\pgfpathlineto{\pgfqpoint{14.633992in}{0.773588in}}%
\pgfpathlineto{\pgfqpoint{14.680226in}{0.773588in}}%
\pgfpathlineto{\pgfqpoint{14.726757in}{0.773588in}}%
\pgfpathlineto{\pgfqpoint{14.774752in}{0.773588in}}%
\pgfpathlineto{\pgfqpoint{14.821474in}{0.773588in}}%
\pgfpathlineto{\pgfqpoint{14.868242in}{0.773588in}}%
\pgfpathlineto{\pgfqpoint{14.916239in}{0.773588in}}%
\pgfpathlineto{\pgfqpoint{14.963167in}{0.773588in}}%
\pgfpathlineto{\pgfqpoint{15.010214in}{0.773588in}}%
\pgfpathlineto{\pgfqpoint{15.058774in}{0.773588in}}%
\pgfpathlineto{\pgfqpoint{15.105478in}{0.773588in}}%
\pgfpathlineto{\pgfqpoint{15.152658in}{0.773588in}}%
\pgfpathlineto{\pgfqpoint{15.201456in}{0.773588in}}%
\pgfpathlineto{\pgfqpoint{15.248863in}{0.773588in}}%
\pgfpathlineto{\pgfqpoint{15.295866in}{0.773588in}}%
\pgfpathlineto{\pgfqpoint{15.343883in}{0.773588in}}%
\pgfpathlineto{\pgfqpoint{15.389896in}{0.773588in}}%
\pgfpathlineto{\pgfqpoint{15.436052in}{0.773588in}}%
\pgfpathlineto{\pgfqpoint{15.484858in}{0.773588in}}%
\pgfpathlineto{\pgfqpoint{15.532329in}{0.773588in}}%
\pgfpathlineto{\pgfqpoint{15.579815in}{0.773588in}}%
\pgfpathlineto{\pgfqpoint{15.628660in}{0.773588in}}%
\pgfpathlineto{\pgfqpoint{15.676019in}{0.773588in}}%
\pgfpathlineto{\pgfqpoint{15.722814in}{0.773588in}}%
\pgfpathlineto{\pgfqpoint{15.770154in}{0.773588in}}%
\pgfpathlineto{\pgfqpoint{15.817273in}{0.773588in}}%
\pgfpathlineto{\pgfqpoint{15.863592in}{0.773588in}}%
\pgfpathlineto{\pgfqpoint{15.911811in}{0.773588in}}%
\pgfpathlineto{\pgfqpoint{15.957799in}{0.773588in}}%
\pgfpathlineto{\pgfqpoint{16.003926in}{0.773588in}}%
\pgfpathlineto{\pgfqpoint{16.051820in}{0.773588in}}%
\pgfpathlineto{\pgfqpoint{16.098697in}{0.773588in}}%
\pgfpathlineto{\pgfqpoint{16.146338in}{0.773588in}}%
\pgfpathlineto{\pgfqpoint{16.195175in}{0.773588in}}%
\pgfpathlineto{\pgfqpoint{16.242078in}{0.773588in}}%
\pgfpathlineto{\pgfqpoint{16.289817in}{0.773588in}}%
\pgfpathlineto{\pgfqpoint{16.339537in}{0.773588in}}%
\pgfpathlineto{\pgfqpoint{16.387666in}{0.773588in}}%
\pgfpathlineto{\pgfqpoint{16.436622in}{0.773588in}}%
\pgfpathlineto{\pgfqpoint{16.486560in}{0.773588in}}%
\pgfpathlineto{\pgfqpoint{16.533619in}{0.773588in}}%
\pgfpathlineto{\pgfqpoint{16.581135in}{0.773588in}}%
\pgfpathlineto{\pgfqpoint{16.630280in}{0.773588in}}%
\pgfpathlineto{\pgfqpoint{16.677566in}{0.773588in}}%
\pgfpathlineto{\pgfqpoint{16.724568in}{0.773588in}}%
\pgfpathlineto{\pgfqpoint{16.773713in}{0.773588in}}%
\pgfpathlineto{\pgfqpoint{16.821038in}{0.773588in}}%
\pgfpathlineto{\pgfqpoint{16.868266in}{0.773588in}}%
\pgfpathlineto{\pgfqpoint{16.916302in}{0.773588in}}%
\pgfpathlineto{\pgfqpoint{16.963533in}{0.773588in}}%
\pgfpathlineto{\pgfqpoint{17.010585in}{0.773588in}}%
\pgfpathlineto{\pgfqpoint{17.059518in}{0.773588in}}%
\pgfpathlineto{\pgfqpoint{17.107470in}{0.773588in}}%
\pgfpathlineto{\pgfqpoint{17.154507in}{0.773588in}}%
\pgfpathlineto{\pgfqpoint{17.203674in}{0.773588in}}%
\pgfpathlineto{\pgfqpoint{17.250860in}{0.773588in}}%
\pgfpathlineto{\pgfqpoint{17.298941in}{0.773588in}}%
\pgfpathlineto{\pgfqpoint{17.347576in}{0.773588in}}%
\pgfpathlineto{\pgfqpoint{17.394714in}{0.773588in}}%
\pgfpathlineto{\pgfqpoint{17.442362in}{0.773588in}}%
\pgfpathlineto{\pgfqpoint{17.491125in}{0.773588in}}%
\pgfpathlineto{\pgfqpoint{17.538409in}{0.773588in}}%
\pgfpathlineto{\pgfqpoint{17.585742in}{0.773588in}}%
\pgfpathlineto{\pgfqpoint{17.634653in}{0.773588in}}%
\pgfpathlineto{\pgfqpoint{17.681914in}{0.773588in}}%
\pgfpathlineto{\pgfqpoint{17.729727in}{0.773588in}}%
\pgfpathlineto{\pgfqpoint{17.779014in}{0.773588in}}%
\pgfpathlineto{\pgfqpoint{17.826809in}{0.773588in}}%
\pgfpathlineto{\pgfqpoint{17.874600in}{0.773588in}}%
\pgfpathlineto{\pgfqpoint{17.922885in}{0.773588in}}%
\pgfpathlineto{\pgfqpoint{17.970910in}{0.773588in}}%
\pgfpathlineto{\pgfqpoint{18.020026in}{0.773588in}}%
\pgfpathlineto{\pgfqpoint{18.069524in}{0.773588in}}%
\pgfpathlineto{\pgfqpoint{18.117307in}{0.773588in}}%
\pgfpathlineto{\pgfqpoint{18.164405in}{0.773588in}}%
\pgfpathlineto{\pgfqpoint{18.213333in}{0.773588in}}%
\pgfpathlineto{\pgfqpoint{18.260871in}{0.773588in}}%
\pgfpathlineto{\pgfqpoint{18.308441in}{0.773588in}}%
\pgfpathlineto{\pgfqpoint{18.357538in}{0.773588in}}%
\pgfpathlineto{\pgfqpoint{18.405242in}{0.773588in}}%
\pgfpathlineto{\pgfqpoint{18.452284in}{0.773588in}}%
\pgfpathlineto{\pgfqpoint{18.500915in}{0.773588in}}%
\pgfpathlineto{\pgfqpoint{18.548460in}{0.773588in}}%
\pgfpathlineto{\pgfqpoint{18.595724in}{0.773588in}}%
\pgfpathlineto{\pgfqpoint{18.644639in}{0.773588in}}%
\pgfpathlineto{\pgfqpoint{18.693122in}{0.773588in}}%
\pgfpathlineto{\pgfqpoint{18.741665in}{0.773588in}}%
\pgfpathlineto{\pgfqpoint{18.791212in}{0.773588in}}%
\pgfpathlineto{\pgfqpoint{18.839312in}{0.773588in}}%
\pgfpathlineto{\pgfqpoint{18.888188in}{0.773588in}}%
\pgfpathlineto{\pgfqpoint{18.937671in}{0.773588in}}%
\pgfpathlineto{\pgfqpoint{18.985002in}{0.773588in}}%
\pgfpathlineto{\pgfqpoint{19.033794in}{0.773588in}}%
\pgfpathlineto{\pgfqpoint{19.084207in}{0.773588in}}%
\pgfpathlineto{\pgfqpoint{19.132724in}{0.773588in}}%
\pgfpathlineto{\pgfqpoint{19.181302in}{0.773588in}}%
\pgfpathlineto{\pgfqpoint{19.231826in}{0.773588in}}%
\pgfpathlineto{\pgfqpoint{19.279742in}{0.773588in}}%
\pgfpathlineto{\pgfqpoint{19.327829in}{0.773588in}}%
\pgfpathlineto{\pgfqpoint{19.376890in}{0.773588in}}%
\pgfpathlineto{\pgfqpoint{19.425178in}{0.773588in}}%
\pgfpathlineto{\pgfqpoint{19.473154in}{0.773588in}}%
\pgfpathlineto{\pgfqpoint{19.522292in}{0.773588in}}%
\pgfpathlineto{\pgfqpoint{19.569747in}{0.773588in}}%
\pgfpathlineto{\pgfqpoint{19.617707in}{0.773588in}}%
\pgfpathlineto{\pgfqpoint{19.666940in}{0.773588in}}%
\pgfpathlineto{\pgfqpoint{19.714265in}{0.773588in}}%
\pgfpathlineto{\pgfqpoint{19.761678in}{0.773588in}}%
\pgfpathlineto{\pgfqpoint{19.810712in}{0.773588in}}%
\pgfpathlineto{\pgfqpoint{19.858933in}{0.773588in}}%
\pgfpathlineto{\pgfqpoint{19.906615in}{0.773588in}}%
\pgfpathlineto{\pgfqpoint{19.955252in}{0.773588in}}%
\pgfpathlineto{\pgfqpoint{20.003692in}{0.773588in}}%
\pgfpathlineto{\pgfqpoint{20.053312in}{0.773588in}}%
\pgfpathlineto{\pgfqpoint{20.104350in}{0.773588in}}%
\pgfpathlineto{\pgfqpoint{20.153679in}{0.773588in}}%
\pgfpathlineto{\pgfqpoint{20.203615in}{0.773588in}}%
\pgfpathlineto{\pgfqpoint{20.255593in}{0.773588in}}%
\pgfpathlineto{\pgfqpoint{20.305924in}{0.773588in}}%
\pgfpathlineto{\pgfqpoint{20.355462in}{0.773588in}}%
\pgfpathlineto{\pgfqpoint{20.406414in}{0.773588in}}%
\pgfpathlineto{\pgfqpoint{20.455604in}{0.773588in}}%
\pgfpathlineto{\pgfqpoint{20.505208in}{0.773588in}}%
\pgfpathlineto{\pgfqpoint{20.556290in}{0.773588in}}%
\pgfpathlineto{\pgfqpoint{20.605567in}{0.773588in}}%
\pgfpathlineto{\pgfqpoint{20.654668in}{0.773588in}}%
\pgfpathlineto{\pgfqpoint{20.704616in}{0.773588in}}%
\pgfpathlineto{\pgfqpoint{20.753558in}{0.773588in}}%
\pgfpathlineto{\pgfqpoint{20.802965in}{0.773588in}}%
\pgfpathlineto{\pgfqpoint{20.854118in}{0.773588in}}%
\pgfpathlineto{\pgfqpoint{20.904164in}{0.773588in}}%
\pgfpathlineto{\pgfqpoint{20.953585in}{0.773588in}}%
\pgfpathlineto{\pgfqpoint{21.003693in}{0.773588in}}%
\pgfpathlineto{\pgfqpoint{21.053515in}{0.773588in}}%
\pgfpathlineto{\pgfqpoint{21.102370in}{0.773588in}}%
\pgfpathlineto{\pgfqpoint{21.152719in}{0.773588in}}%
\pgfpathlineto{\pgfqpoint{21.201683in}{0.773588in}}%
\pgfpathlineto{\pgfqpoint{21.250655in}{0.773588in}}%
\pgfpathlineto{\pgfqpoint{21.300955in}{0.773588in}}%
\pgfpathlineto{\pgfqpoint{21.350353in}{0.773588in}}%
\pgfpathlineto{\pgfqpoint{21.399733in}{0.773588in}}%
\pgfpathlineto{\pgfqpoint{21.450830in}{0.773588in}}%
\pgfpathlineto{\pgfqpoint{21.501023in}{0.773588in}}%
\pgfpathlineto{\pgfqpoint{21.550957in}{0.773588in}}%
\pgfpathlineto{\pgfqpoint{21.602538in}{0.773588in}}%
\pgfpathlineto{\pgfqpoint{21.652276in}{0.773588in}}%
\pgfpathlineto{\pgfqpoint{21.700864in}{0.773588in}}%
\pgfpathlineto{\pgfqpoint{21.751624in}{0.773588in}}%
\pgfpathlineto{\pgfqpoint{21.800474in}{0.773588in}}%
\pgfpathlineto{\pgfqpoint{21.849580in}{0.773588in}}%
\pgfpathlineto{\pgfqpoint{21.900834in}{0.773588in}}%
\pgfpathlineto{\pgfqpoint{21.950544in}{0.773588in}}%
\pgfpathlineto{\pgfqpoint{21.999803in}{0.773588in}}%
\pgfpathlineto{\pgfqpoint{22.050354in}{0.773588in}}%
\pgfpathlineto{\pgfqpoint{22.099907in}{0.773588in}}%
\pgfpathlineto{\pgfqpoint{22.148817in}{0.773588in}}%
\pgfpathlineto{\pgfqpoint{22.200587in}{0.773588in}}%
\pgfpathlineto{\pgfqpoint{22.250634in}{0.773588in}}%
\pgfpathlineto{\pgfqpoint{22.300933in}{0.773588in}}%
\pgfpathlineto{\pgfqpoint{22.352827in}{0.773588in}}%
\pgfpathlineto{\pgfqpoint{22.402084in}{0.773588in}}%
\pgfpathlineto{\pgfqpoint{22.451976in}{0.773588in}}%
\pgfpathlineto{\pgfqpoint{22.503552in}{0.773588in}}%
\pgfpathlineto{\pgfqpoint{22.554073in}{0.773588in}}%
\pgfpathlineto{\pgfqpoint{22.603657in}{0.773588in}}%
\pgfpathlineto{\pgfqpoint{22.654822in}{0.773588in}}%
\pgfpathlineto{\pgfqpoint{22.703398in}{0.773588in}}%
\pgfpathlineto{\pgfqpoint{22.751700in}{0.773588in}}%
\pgfpathlineto{\pgfqpoint{22.803829in}{0.773588in}}%
\pgfpathlineto{\pgfqpoint{22.853966in}{0.773588in}}%
\pgfpathlineto{\pgfqpoint{22.904236in}{0.773588in}}%
\pgfpathlineto{\pgfqpoint{22.956359in}{0.773588in}}%
\pgfpathlineto{\pgfqpoint{23.006022in}{0.773588in}}%
\pgfpathlineto{\pgfqpoint{23.055187in}{0.773588in}}%
\pgfpathlineto{\pgfqpoint{23.107134in}{0.773588in}}%
\pgfpathlineto{\pgfqpoint{23.157756in}{0.773588in}}%
\pgfpathlineto{\pgfqpoint{23.208694in}{0.773588in}}%
\pgfpathlineto{\pgfqpoint{23.260726in}{0.773588in}}%
\pgfpathlineto{\pgfqpoint{23.310918in}{0.773588in}}%
\pgfpathlineto{\pgfqpoint{23.361310in}{0.773588in}}%
\pgfpathlineto{\pgfqpoint{23.411909in}{0.773588in}}%
\pgfpathlineto{\pgfqpoint{23.462017in}{0.773588in}}%
\pgfpathlineto{\pgfqpoint{23.511867in}{0.773588in}}%
\pgfpathlineto{\pgfqpoint{23.563429in}{0.773588in}}%
\pgfpathlineto{\pgfqpoint{23.611923in}{0.773588in}}%
\pgfpathlineto{\pgfqpoint{23.661575in}{0.773588in}}%
\pgfpathlineto{\pgfqpoint{23.713074in}{0.773588in}}%
\pgfpathlineto{\pgfqpoint{23.762329in}{0.773588in}}%
\pgfpathlineto{\pgfqpoint{23.811512in}{0.773588in}}%
\pgfpathlineto{\pgfqpoint{23.862673in}{0.773588in}}%
\pgfpathlineto{\pgfqpoint{23.913088in}{0.773588in}}%
\pgfpathlineto{\pgfqpoint{23.964514in}{0.773588in}}%
\pgfpathlineto{\pgfqpoint{24.016442in}{0.773588in}}%
\pgfpathlineto{\pgfqpoint{24.066933in}{0.773588in}}%
\pgfpathlineto{\pgfqpoint{24.118637in}{0.773588in}}%
\pgfpathlineto{\pgfqpoint{24.170873in}{0.773588in}}%
\pgfpathlineto{\pgfqpoint{24.221010in}{0.773588in}}%
\pgfpathlineto{\pgfqpoint{24.271277in}{0.773588in}}%
\pgfpathlineto{\pgfqpoint{24.321907in}{0.773588in}}%
\pgfpathlineto{\pgfqpoint{24.371890in}{0.773588in}}%
\pgfpathlineto{\pgfqpoint{24.422475in}{0.773588in}}%
\pgfpathlineto{\pgfqpoint{24.474381in}{0.773588in}}%
\pgfpathlineto{\pgfqpoint{24.524258in}{0.773588in}}%
\pgfpathlineto{\pgfqpoint{24.573998in}{0.773588in}}%
\pgfpathlineto{\pgfqpoint{24.626556in}{0.773588in}}%
\pgfpathlineto{\pgfqpoint{24.676389in}{0.773588in}}%
\pgfpathlineto{\pgfqpoint{24.726499in}{0.773588in}}%
\pgfpathlineto{\pgfqpoint{24.778304in}{0.773588in}}%
\pgfpathlineto{\pgfqpoint{24.828140in}{0.773588in}}%
\pgfpathlineto{\pgfqpoint{24.878143in}{0.773588in}}%
\pgfpathlineto{\pgfqpoint{24.928869in}{0.773588in}}%
\pgfpathlineto{\pgfqpoint{24.978211in}{0.773588in}}%
\pgfpathlineto{\pgfqpoint{25.028630in}{0.773588in}}%
\pgfpathlineto{\pgfqpoint{25.080539in}{0.773588in}}%
\pgfpathlineto{\pgfqpoint{25.131043in}{0.773588in}}%
\pgfpathlineto{\pgfqpoint{25.181554in}{0.773588in}}%
\pgfpathlineto{\pgfqpoint{25.232944in}{0.773588in}}%
\pgfpathlineto{\pgfqpoint{25.282222in}{0.773588in}}%
\pgfpathlineto{\pgfqpoint{25.332608in}{0.773588in}}%
\pgfpathlineto{\pgfqpoint{25.383622in}{0.773588in}}%
\pgfpathlineto{\pgfqpoint{25.433954in}{0.773588in}}%
\pgfpathlineto{\pgfqpoint{25.483381in}{0.773588in}}%
\pgfpathlineto{\pgfqpoint{25.535558in}{0.773588in}}%
\pgfpathlineto{\pgfqpoint{25.586387in}{0.773588in}}%
\pgfpathlineto{\pgfqpoint{25.637513in}{0.773588in}}%
\pgfpathlineto{\pgfqpoint{25.689152in}{0.773588in}}%
\pgfpathlineto{\pgfqpoint{25.740073in}{0.773588in}}%
\pgfpathlineto{\pgfqpoint{25.790287in}{0.773588in}}%
\pgfpathlineto{\pgfqpoint{25.841969in}{0.773588in}}%
\pgfpathlineto{\pgfqpoint{25.891991in}{0.773588in}}%
\pgfpathlineto{\pgfqpoint{25.942005in}{0.773588in}}%
\pgfpathlineto{\pgfqpoint{25.994130in}{0.773588in}}%
\pgfpathlineto{\pgfqpoint{26.044216in}{0.773588in}}%
\pgfpathlineto{\pgfqpoint{26.093895in}{0.773588in}}%
\pgfpathlineto{\pgfqpoint{26.144837in}{0.773588in}}%
\pgfpathlineto{\pgfqpoint{26.195263in}{0.773588in}}%
\pgfpathlineto{\pgfqpoint{26.245717in}{0.773588in}}%
\pgfpathlineto{\pgfqpoint{26.297518in}{0.773588in}}%
\pgfpathlineto{\pgfqpoint{26.346588in}{0.773588in}}%
\pgfpathlineto{\pgfqpoint{26.395829in}{0.773588in}}%
\pgfpathlineto{\pgfqpoint{26.447655in}{0.773588in}}%
\pgfpathlineto{\pgfqpoint{26.499064in}{0.773588in}}%
\pgfpathlineto{\pgfqpoint{26.549446in}{0.773588in}}%
\pgfpathlineto{\pgfqpoint{26.601868in}{0.773588in}}%
\pgfpathlineto{\pgfqpoint{26.653161in}{0.773588in}}%
\pgfpathlineto{\pgfqpoint{26.704176in}{0.773588in}}%
\pgfpathlineto{\pgfqpoint{26.756448in}{0.773588in}}%
\pgfpathlineto{\pgfqpoint{26.807173in}{0.773588in}}%
\pgfpathlineto{\pgfqpoint{26.857706in}{0.773588in}}%
\pgfpathlineto{\pgfqpoint{26.909530in}{0.773588in}}%
\pgfpathlineto{\pgfqpoint{26.960748in}{0.773588in}}%
\pgfpathlineto{\pgfqpoint{27.010539in}{0.773588in}}%
\pgfpathlineto{\pgfqpoint{27.063062in}{0.773588in}}%
\pgfpathlineto{\pgfqpoint{27.113950in}{0.773588in}}%
\pgfpathlineto{\pgfqpoint{27.165274in}{0.773588in}}%
\pgfpathlineto{\pgfqpoint{27.218587in}{0.773588in}}%
\pgfpathlineto{\pgfqpoint{27.269679in}{0.773588in}}%
\pgfpathlineto{\pgfqpoint{27.321160in}{0.773588in}}%
\pgfpathlineto{\pgfqpoint{27.372863in}{0.773588in}}%
\pgfpathlineto{\pgfqpoint{27.423066in}{0.773588in}}%
\pgfpathlineto{\pgfqpoint{27.473173in}{0.773588in}}%
\pgfpathlineto{\pgfqpoint{27.525849in}{0.773588in}}%
\pgfpathlineto{\pgfqpoint{27.576178in}{0.773588in}}%
\pgfpathlineto{\pgfqpoint{27.626352in}{0.773588in}}%
\pgfpathlineto{\pgfqpoint{27.678294in}{0.773588in}}%
\pgfpathlineto{\pgfqpoint{27.728571in}{0.773588in}}%
\pgfpathlineto{\pgfqpoint{27.779331in}{0.773588in}}%
\pgfpathlineto{\pgfqpoint{27.833067in}{0.773588in}}%
\pgfpathlineto{\pgfqpoint{27.883786in}{0.773588in}}%
\pgfpathlineto{\pgfqpoint{27.935127in}{0.773588in}}%
\pgfpathlineto{\pgfqpoint{27.988562in}{0.773588in}}%
\pgfpathlineto{\pgfqpoint{28.039571in}{0.773588in}}%
\pgfpathlineto{\pgfqpoint{28.090650in}{0.773588in}}%
\pgfpathlineto{\pgfqpoint{28.143944in}{0.773588in}}%
\pgfpathlineto{\pgfqpoint{28.195034in}{0.773588in}}%
\pgfpathlineto{\pgfqpoint{28.245320in}{0.773588in}}%
\pgfpathlineto{\pgfqpoint{28.297462in}{0.773588in}}%
\pgfpathlineto{\pgfqpoint{28.347853in}{0.773588in}}%
\pgfpathlineto{\pgfqpoint{28.399138in}{0.773588in}}%
\pgfpathlineto{\pgfqpoint{28.451952in}{0.773588in}}%
\pgfpathlineto{\pgfqpoint{28.502707in}{0.773588in}}%
\pgfpathlineto{\pgfqpoint{28.553749in}{0.773588in}}%
\pgfpathlineto{\pgfqpoint{28.605351in}{0.773588in}}%
\pgfpathlineto{\pgfqpoint{28.656185in}{0.773588in}}%
\pgfpathlineto{\pgfqpoint{28.706300in}{0.773588in}}%
\pgfpathlineto{\pgfqpoint{28.758074in}{0.773588in}}%
\pgfpathlineto{\pgfqpoint{28.808587in}{0.773588in}}%
\pgfpathlineto{\pgfqpoint{28.860347in}{0.773588in}}%
\pgfpathlineto{\pgfqpoint{28.913538in}{0.773588in}}%
\pgfpathlineto{\pgfqpoint{28.965063in}{0.773588in}}%
\pgfpathlineto{\pgfqpoint{29.015651in}{0.773588in}}%
\pgfpathlineto{\pgfqpoint{29.068863in}{0.773588in}}%
\pgfpathlineto{\pgfqpoint{29.121029in}{0.773588in}}%
\pgfpathlineto{\pgfqpoint{29.173700in}{0.773588in}}%
\pgfpathlineto{\pgfqpoint{29.226818in}{0.773588in}}%
\pgfpathlineto{\pgfqpoint{29.279399in}{0.773588in}}%
\pgfpathlineto{\pgfqpoint{29.332261in}{0.773588in}}%
\pgfpathlineto{\pgfqpoint{29.386173in}{0.773588in}}%
\pgfpathlineto{\pgfqpoint{29.438335in}{0.773588in}}%
\pgfpathlineto{\pgfqpoint{29.490075in}{0.773588in}}%
\pgfpathlineto{\pgfqpoint{29.543760in}{0.773588in}}%
\pgfpathlineto{\pgfqpoint{29.596078in}{0.773588in}}%
\pgfpathlineto{\pgfqpoint{29.647988in}{0.773588in}}%
\pgfpathlineto{\pgfqpoint{29.701409in}{0.773588in}}%
\pgfpathlineto{\pgfqpoint{29.753093in}{0.773588in}}%
\pgfpathlineto{\pgfqpoint{29.805208in}{0.773588in}}%
\pgfpathlineto{\pgfqpoint{29.858913in}{0.773588in}}%
\pgfpathlineto{\pgfqpoint{29.910558in}{0.773588in}}%
\pgfpathlineto{\pgfqpoint{29.962066in}{0.773588in}}%
\pgfpathlineto{\pgfqpoint{30.014968in}{0.773588in}}%
\pgfpathlineto{\pgfqpoint{30.066247in}{0.773588in}}%
\pgfpathlineto{\pgfqpoint{30.117082in}{0.773588in}}%
\pgfpathlineto{\pgfqpoint{30.169497in}{0.773588in}}%
\pgfpathlineto{\pgfqpoint{30.221783in}{0.773588in}}%
\pgfpathlineto{\pgfqpoint{30.276704in}{0.773588in}}%
\pgfpathlineto{\pgfqpoint{30.337041in}{0.773588in}}%
\pgfpathlineto{\pgfqpoint{30.397563in}{0.773588in}}%
\pgfpathlineto{\pgfqpoint{30.457369in}{0.773588in}}%
\pgfpathlineto{\pgfqpoint{30.519963in}{0.773588in}}%
\pgfpathlineto{\pgfqpoint{30.582999in}{0.773588in}}%
\pgfpathlineto{\pgfqpoint{30.648377in}{0.773588in}}%
\pgfpathlineto{\pgfqpoint{30.717993in}{0.773588in}}%
\pgfpathlineto{\pgfqpoint{30.786068in}{0.773588in}}%
\pgfpathlineto{\pgfqpoint{30.855684in}{0.773588in}}%
\pgfpathlineto{\pgfqpoint{30.928370in}{0.773588in}}%
\pgfpathlineto{\pgfqpoint{31.000738in}{0.773588in}}%
\pgfpathlineto{\pgfqpoint{31.073353in}{0.773588in}}%
\pgfpathlineto{\pgfqpoint{31.150292in}{0.773588in}}%
\pgfpathlineto{\pgfqpoint{31.225476in}{0.773588in}}%
\pgfpathlineto{\pgfqpoint{31.304546in}{0.773588in}}%
\pgfpathlineto{\pgfqpoint{31.385546in}{0.773588in}}%
\pgfpathlineto{\pgfqpoint{31.463397in}{0.773588in}}%
\pgfpathlineto{\pgfqpoint{31.543273in}{0.773588in}}%
\pgfpathlineto{\pgfqpoint{31.629527in}{0.773588in}}%
\pgfpathlineto{\pgfqpoint{31.715049in}{0.773588in}}%
\pgfpathlineto{\pgfqpoint{31.802724in}{0.773588in}}%
\pgfpathlineto{\pgfqpoint{31.890397in}{0.773588in}}%
\pgfpathlineto{\pgfqpoint{31.975639in}{0.773588in}}%
\pgfpathlineto{\pgfqpoint{32.062211in}{0.773588in}}%
\pgfpathlineto{\pgfqpoint{32.153206in}{0.773588in}}%
\pgfpathlineto{\pgfqpoint{32.244846in}{0.773588in}}%
\pgfpathlineto{\pgfqpoint{32.334387in}{0.773588in}}%
\pgfpathlineto{\pgfqpoint{32.429213in}{0.773588in}}%
\pgfpathlineto{\pgfqpoint{32.518604in}{0.773588in}}%
\pgfpathlineto{\pgfqpoint{32.584565in}{0.773588in}}%
\pgfpathlineto{\pgfqpoint{32.638735in}{0.773588in}}%
\pgfpathlineto{\pgfqpoint{32.691213in}{0.773588in}}%
\pgfpathlineto{\pgfqpoint{32.743153in}{0.773588in}}%
\pgfpathlineto{\pgfqpoint{32.797230in}{0.773588in}}%
\pgfpathlineto{\pgfqpoint{32.850576in}{0.773588in}}%
\pgfpathlineto{\pgfqpoint{32.903667in}{0.773588in}}%
\pgfpathlineto{\pgfqpoint{32.957868in}{0.773588in}}%
\pgfpathlineto{\pgfqpoint{33.010153in}{0.773588in}}%
\pgfpathlineto{\pgfqpoint{33.062733in}{0.773588in}}%
\pgfpathlineto{\pgfqpoint{33.115518in}{0.773588in}}%
\pgfpathlineto{\pgfqpoint{33.167699in}{0.773588in}}%
\pgfpathlineto{\pgfqpoint{33.219830in}{0.773588in}}%
\pgfpathlineto{\pgfqpoint{33.273785in}{0.773588in}}%
\pgfpathlineto{\pgfqpoint{33.326652in}{0.773588in}}%
\pgfpathlineto{\pgfqpoint{33.379482in}{0.773588in}}%
\pgfpathlineto{\pgfqpoint{33.433283in}{0.773588in}}%
\pgfpathlineto{\pgfqpoint{33.484775in}{0.773588in}}%
\pgfpathlineto{\pgfqpoint{33.536879in}{0.773588in}}%
\pgfpathlineto{\pgfqpoint{33.590505in}{0.773588in}}%
\pgfpathlineto{\pgfqpoint{33.642373in}{0.773588in}}%
\pgfpathlineto{\pgfqpoint{33.693680in}{0.773588in}}%
\pgfpathlineto{\pgfqpoint{33.746200in}{0.773588in}}%
\pgfpathlineto{\pgfqpoint{33.784450in}{0.773588in}}%
\pgfpathlineto{\pgfqpoint{33.832264in}{0.773588in}}%
\pgfpathlineto{\pgfqpoint{33.870742in}{1.025897in}}%
\pgfpathlineto{\pgfqpoint{33.912991in}{1.089794in}}%
\pgfpathlineto{\pgfqpoint{33.952831in}{1.158641in}}%
\pgfpathlineto{\pgfqpoint{33.989641in}{1.243236in}}%
\pgfpathlineto{\pgfqpoint{34.021637in}{1.503451in}}%
\pgfpathlineto{\pgfqpoint{34.052161in}{1.799827in}}%
\pgfpathlineto{\pgfqpoint{34.076513in}{2.293678in}}%
\pgfpathlineto{\pgfqpoint{34.101659in}{2.218952in}}%
\pgfpathlineto{\pgfqpoint{34.125436in}{2.290107in}}%
\pgfpathlineto{\pgfqpoint{34.150180in}{2.446672in}}%
\pgfpathlineto{\pgfqpoint{34.173862in}{2.339022in}}%
\pgfpathlineto{\pgfqpoint{34.197440in}{2.332444in}}%
\pgfpathlineto{\pgfqpoint{34.222832in}{2.257032in}}%
\pgfpathlineto{\pgfqpoint{34.246364in}{2.316562in}}%
\pgfpathlineto{\pgfqpoint{34.270997in}{2.267974in}}%
\pgfpathlineto{\pgfqpoint{34.293786in}{2.361353in}}%
\pgfpathlineto{\pgfqpoint{34.318035in}{2.449950in}}%
\pgfpathlineto{\pgfqpoint{34.340984in}{2.394574in}}%
\pgfpathlineto{\pgfqpoint{34.366319in}{2.271817in}}%
\pgfpathlineto{\pgfqpoint{34.388883in}{2.287340in}}%
\pgfpathlineto{\pgfqpoint{34.412956in}{2.232666in}}%
\pgfpathlineto{\pgfqpoint{34.437757in}{2.338107in}}%
\pgfpathlineto{\pgfqpoint{34.460943in}{2.469518in}}%
\pgfpathlineto{\pgfqpoint{34.484122in}{2.433623in}}%
\pgfpathlineto{\pgfqpoint{34.508593in}{2.282667in}}%
\pgfpathlineto{\pgfqpoint{34.531364in}{2.411536in}}%
\pgfpathlineto{\pgfqpoint{34.554213in}{2.434395in}}%
\pgfpathlineto{\pgfqpoint{34.578241in}{2.553834in}}%
\pgfpathlineto{\pgfqpoint{34.601042in}{2.526820in}}%
\pgfpathlineto{\pgfqpoint{34.623987in}{2.434375in}}%
\pgfpathlineto{\pgfqpoint{34.648006in}{2.409123in}}%
\pgfpathlineto{\pgfqpoint{34.671232in}{2.380016in}}%
\pgfpathlineto{\pgfqpoint{34.693969in}{2.578334in}}%
\pgfpathlineto{\pgfqpoint{34.717979in}{2.431018in}}%
\pgfpathlineto{\pgfqpoint{34.741399in}{2.431345in}}%
\pgfpathlineto{\pgfqpoint{34.763884in}{2.581556in}}%
\pgfpathlineto{\pgfqpoint{34.788312in}{2.457377in}}%
\pgfpathlineto{\pgfqpoint{34.810332in}{2.469111in}}%
\pgfpathlineto{\pgfqpoint{34.834146in}{2.475980in}}%
\pgfpathlineto{\pgfqpoint{34.856576in}{2.570004in}}%
\pgfpathlineto{\pgfqpoint{34.880395in}{2.522929in}}%
\pgfpathlineto{\pgfqpoint{34.902821in}{2.477526in}}%
\pgfpathlineto{\pgfqpoint{34.926805in}{2.468813in}}%
\pgfpathlineto{\pgfqpoint{34.948980in}{2.587240in}}%
\pgfpathlineto{\pgfqpoint{34.973284in}{2.403966in}}%
\pgfpathlineto{\pgfqpoint{34.996094in}{2.424295in}}%
\pgfpathlineto{\pgfqpoint{35.020184in}{2.470556in}}%
\pgfpathlineto{\pgfqpoint{35.047391in}{2.423078in}}%
\pgfpathlineto{\pgfqpoint{35.098205in}{2.413621in}}%
\pgfpathlineto{\pgfqpoint{35.149293in}{2.413621in}}%
\pgfpathlineto{\pgfqpoint{35.201170in}{2.413621in}}%
\pgfpathlineto{\pgfqpoint{35.254065in}{2.413621in}}%
\pgfpathlineto{\pgfqpoint{35.305776in}{2.413621in}}%
\pgfpathlineto{\pgfqpoint{35.357710in}{2.413621in}}%
\pgfpathlineto{\pgfqpoint{35.411910in}{2.413621in}}%
\pgfpathlineto{\pgfqpoint{35.464942in}{2.413621in}}%
\pgfpathlineto{\pgfqpoint{35.464942in}{3.993842in}}%
\pgfpathlineto{\pgfqpoint{35.464942in}{3.993842in}}%
\pgfpathlineto{\pgfqpoint{35.411910in}{3.993842in}}%
\pgfpathlineto{\pgfqpoint{35.357710in}{3.993842in}}%
\pgfpathlineto{\pgfqpoint{35.305776in}{3.993842in}}%
\pgfpathlineto{\pgfqpoint{35.254065in}{3.993842in}}%
\pgfpathlineto{\pgfqpoint{35.201170in}{3.993842in}}%
\pgfpathlineto{\pgfqpoint{35.149293in}{3.993842in}}%
\pgfpathlineto{\pgfqpoint{35.098205in}{3.993842in}}%
\pgfpathlineto{\pgfqpoint{35.047391in}{4.038936in}}%
\pgfpathlineto{\pgfqpoint{35.020184in}{4.182959in}}%
\pgfpathlineto{\pgfqpoint{34.996094in}{4.101072in}}%
\pgfpathlineto{\pgfqpoint{34.973284in}{3.989480in}}%
\pgfpathlineto{\pgfqpoint{34.948980in}{4.239053in}}%
\pgfpathlineto{\pgfqpoint{34.926805in}{4.231403in}}%
\pgfpathlineto{\pgfqpoint{34.902821in}{4.198446in}}%
\pgfpathlineto{\pgfqpoint{34.880395in}{4.201901in}}%
\pgfpathlineto{\pgfqpoint{34.856576in}{4.232916in}}%
\pgfpathlineto{\pgfqpoint{34.834146in}{4.201329in}}%
\pgfpathlineto{\pgfqpoint{34.810332in}{4.192246in}}%
\pgfpathlineto{\pgfqpoint{34.788312in}{4.142544in}}%
\pgfpathlineto{\pgfqpoint{34.763884in}{4.251407in}}%
\pgfpathlineto{\pgfqpoint{34.741399in}{4.195680in}}%
\pgfpathlineto{\pgfqpoint{34.717979in}{4.168563in}}%
\pgfpathlineto{\pgfqpoint{34.693969in}{4.254313in}}%
\pgfpathlineto{\pgfqpoint{34.671232in}{3.997581in}}%
\pgfpathlineto{\pgfqpoint{34.648006in}{4.071453in}}%
\pgfpathlineto{\pgfqpoint{34.623987in}{4.124102in}}%
\pgfpathlineto{\pgfqpoint{34.601042in}{4.217950in}}%
\pgfpathlineto{\pgfqpoint{34.578241in}{4.194563in}}%
\pgfpathlineto{\pgfqpoint{34.554213in}{4.127126in}}%
\pgfpathlineto{\pgfqpoint{34.531364in}{4.171934in}}%
\pgfpathlineto{\pgfqpoint{34.508593in}{3.944352in}}%
\pgfpathlineto{\pgfqpoint{34.484122in}{4.134872in}}%
\pgfpathlineto{\pgfqpoint{34.460943in}{4.217964in}}%
\pgfpathlineto{\pgfqpoint{34.437757in}{3.898493in}}%
\pgfpathlineto{\pgfqpoint{34.412956in}{3.769623in}}%
\pgfpathlineto{\pgfqpoint{34.388883in}{3.916080in}}%
\pgfpathlineto{\pgfqpoint{34.366319in}{3.888388in}}%
\pgfpathlineto{\pgfqpoint{34.340984in}{3.972416in}}%
\pgfpathlineto{\pgfqpoint{34.318035in}{4.075356in}}%
\pgfpathlineto{\pgfqpoint{34.293786in}{3.930055in}}%
\pgfpathlineto{\pgfqpoint{34.270997in}{3.877862in}}%
\pgfpathlineto{\pgfqpoint{34.246364in}{4.040966in}}%
\pgfpathlineto{\pgfqpoint{34.222832in}{3.988507in}}%
\pgfpathlineto{\pgfqpoint{34.197440in}{3.869449in}}%
\pgfpathlineto{\pgfqpoint{34.173862in}{3.942672in}}%
\pgfpathlineto{\pgfqpoint{34.150180in}{3.941303in}}%
\pgfpathlineto{\pgfqpoint{34.125436in}{3.898067in}}%
\pgfpathlineto{\pgfqpoint{34.101659in}{3.612521in}}%
\pgfpathlineto{\pgfqpoint{34.076513in}{3.753732in}}%
\pgfpathlineto{\pgfqpoint{34.052161in}{2.790032in}}%
\pgfpathlineto{\pgfqpoint{34.021637in}{2.208863in}}%
\pgfpathlineto{\pgfqpoint{33.989641in}{1.768315in}}%
\pgfpathlineto{\pgfqpoint{33.952831in}{1.533299in}}%
\pgfpathlineto{\pgfqpoint{33.912991in}{1.377266in}}%
\pgfpathlineto{\pgfqpoint{33.870742in}{1.246842in}}%
\pgfpathlineto{\pgfqpoint{33.832264in}{0.773588in}}%
\pgfpathlineto{\pgfqpoint{33.784450in}{0.773588in}}%
\pgfpathlineto{\pgfqpoint{33.746200in}{0.773588in}}%
\pgfpathlineto{\pgfqpoint{33.693680in}{0.773588in}}%
\pgfpathlineto{\pgfqpoint{33.642373in}{0.773588in}}%
\pgfpathlineto{\pgfqpoint{33.590505in}{0.773588in}}%
\pgfpathlineto{\pgfqpoint{33.536879in}{0.773588in}}%
\pgfpathlineto{\pgfqpoint{33.484775in}{0.773588in}}%
\pgfpathlineto{\pgfqpoint{33.433283in}{0.773588in}}%
\pgfpathlineto{\pgfqpoint{33.379482in}{0.773588in}}%
\pgfpathlineto{\pgfqpoint{33.326652in}{0.773588in}}%
\pgfpathlineto{\pgfqpoint{33.273785in}{0.773588in}}%
\pgfpathlineto{\pgfqpoint{33.219830in}{0.773588in}}%
\pgfpathlineto{\pgfqpoint{33.167699in}{0.773588in}}%
\pgfpathlineto{\pgfqpoint{33.115518in}{0.773588in}}%
\pgfpathlineto{\pgfqpoint{33.062733in}{0.773588in}}%
\pgfpathlineto{\pgfqpoint{33.010153in}{0.773588in}}%
\pgfpathlineto{\pgfqpoint{32.957868in}{0.773588in}}%
\pgfpathlineto{\pgfqpoint{32.903667in}{0.773588in}}%
\pgfpathlineto{\pgfqpoint{32.850576in}{0.773588in}}%
\pgfpathlineto{\pgfqpoint{32.797230in}{0.773588in}}%
\pgfpathlineto{\pgfqpoint{32.743153in}{0.773588in}}%
\pgfpathlineto{\pgfqpoint{32.691213in}{0.773588in}}%
\pgfpathlineto{\pgfqpoint{32.638735in}{0.773588in}}%
\pgfpathlineto{\pgfqpoint{32.584565in}{0.773588in}}%
\pgfpathlineto{\pgfqpoint{32.518604in}{0.773588in}}%
\pgfpathlineto{\pgfqpoint{32.429213in}{0.773588in}}%
\pgfpathlineto{\pgfqpoint{32.334387in}{0.773588in}}%
\pgfpathlineto{\pgfqpoint{32.244846in}{0.773588in}}%
\pgfpathlineto{\pgfqpoint{32.153206in}{0.773588in}}%
\pgfpathlineto{\pgfqpoint{32.062211in}{0.773588in}}%
\pgfpathlineto{\pgfqpoint{31.975639in}{0.773588in}}%
\pgfpathlineto{\pgfqpoint{31.890397in}{0.773588in}}%
\pgfpathlineto{\pgfqpoint{31.802724in}{0.773588in}}%
\pgfpathlineto{\pgfqpoint{31.715049in}{0.773588in}}%
\pgfpathlineto{\pgfqpoint{31.629527in}{0.773588in}}%
\pgfpathlineto{\pgfqpoint{31.543273in}{0.773588in}}%
\pgfpathlineto{\pgfqpoint{31.463397in}{0.773588in}}%
\pgfpathlineto{\pgfqpoint{31.385546in}{0.773588in}}%
\pgfpathlineto{\pgfqpoint{31.304546in}{0.773588in}}%
\pgfpathlineto{\pgfqpoint{31.225476in}{0.773588in}}%
\pgfpathlineto{\pgfqpoint{31.150292in}{0.773588in}}%
\pgfpathlineto{\pgfqpoint{31.073353in}{0.773588in}}%
\pgfpathlineto{\pgfqpoint{31.000738in}{0.773588in}}%
\pgfpathlineto{\pgfqpoint{30.928370in}{0.773588in}}%
\pgfpathlineto{\pgfqpoint{30.855684in}{0.773588in}}%
\pgfpathlineto{\pgfqpoint{30.786068in}{0.773588in}}%
\pgfpathlineto{\pgfqpoint{30.717993in}{0.773588in}}%
\pgfpathlineto{\pgfqpoint{30.648377in}{0.773588in}}%
\pgfpathlineto{\pgfqpoint{30.582999in}{0.773588in}}%
\pgfpathlineto{\pgfqpoint{30.519963in}{0.773588in}}%
\pgfpathlineto{\pgfqpoint{30.457369in}{0.773588in}}%
\pgfpathlineto{\pgfqpoint{30.397563in}{0.773588in}}%
\pgfpathlineto{\pgfqpoint{30.337041in}{0.773588in}}%
\pgfpathlineto{\pgfqpoint{30.276704in}{0.773588in}}%
\pgfpathlineto{\pgfqpoint{30.221783in}{0.773588in}}%
\pgfpathlineto{\pgfqpoint{30.169497in}{0.773588in}}%
\pgfpathlineto{\pgfqpoint{30.117082in}{0.773588in}}%
\pgfpathlineto{\pgfqpoint{30.066247in}{0.773588in}}%
\pgfpathlineto{\pgfqpoint{30.014968in}{0.773588in}}%
\pgfpathlineto{\pgfqpoint{29.962066in}{0.773588in}}%
\pgfpathlineto{\pgfqpoint{29.910558in}{0.773588in}}%
\pgfpathlineto{\pgfqpoint{29.858913in}{0.773588in}}%
\pgfpathlineto{\pgfqpoint{29.805208in}{0.773588in}}%
\pgfpathlineto{\pgfqpoint{29.753093in}{0.773588in}}%
\pgfpathlineto{\pgfqpoint{29.701409in}{0.773588in}}%
\pgfpathlineto{\pgfqpoint{29.647988in}{0.773588in}}%
\pgfpathlineto{\pgfqpoint{29.596078in}{0.773588in}}%
\pgfpathlineto{\pgfqpoint{29.543760in}{0.773588in}}%
\pgfpathlineto{\pgfqpoint{29.490075in}{0.773588in}}%
\pgfpathlineto{\pgfqpoint{29.438335in}{0.773588in}}%
\pgfpathlineto{\pgfqpoint{29.386173in}{0.773588in}}%
\pgfpathlineto{\pgfqpoint{29.332261in}{0.773588in}}%
\pgfpathlineto{\pgfqpoint{29.279399in}{0.773588in}}%
\pgfpathlineto{\pgfqpoint{29.226818in}{0.773588in}}%
\pgfpathlineto{\pgfqpoint{29.173700in}{0.773588in}}%
\pgfpathlineto{\pgfqpoint{29.121029in}{0.773588in}}%
\pgfpathlineto{\pgfqpoint{29.068863in}{0.773588in}}%
\pgfpathlineto{\pgfqpoint{29.015651in}{0.773588in}}%
\pgfpathlineto{\pgfqpoint{28.965063in}{0.773588in}}%
\pgfpathlineto{\pgfqpoint{28.913538in}{0.773588in}}%
\pgfpathlineto{\pgfqpoint{28.860347in}{0.773588in}}%
\pgfpathlineto{\pgfqpoint{28.808587in}{0.773588in}}%
\pgfpathlineto{\pgfqpoint{28.758074in}{0.773588in}}%
\pgfpathlineto{\pgfqpoint{28.706300in}{0.773588in}}%
\pgfpathlineto{\pgfqpoint{28.656185in}{0.773588in}}%
\pgfpathlineto{\pgfqpoint{28.605351in}{0.773588in}}%
\pgfpathlineto{\pgfqpoint{28.553749in}{0.773588in}}%
\pgfpathlineto{\pgfqpoint{28.502707in}{0.773588in}}%
\pgfpathlineto{\pgfqpoint{28.451952in}{0.773588in}}%
\pgfpathlineto{\pgfqpoint{28.399138in}{0.773588in}}%
\pgfpathlineto{\pgfqpoint{28.347853in}{0.773588in}}%
\pgfpathlineto{\pgfqpoint{28.297462in}{0.773588in}}%
\pgfpathlineto{\pgfqpoint{28.245320in}{0.773588in}}%
\pgfpathlineto{\pgfqpoint{28.195034in}{0.773588in}}%
\pgfpathlineto{\pgfqpoint{28.143944in}{0.773588in}}%
\pgfpathlineto{\pgfqpoint{28.090650in}{0.773588in}}%
\pgfpathlineto{\pgfqpoint{28.039571in}{0.773588in}}%
\pgfpathlineto{\pgfqpoint{27.988562in}{0.773588in}}%
\pgfpathlineto{\pgfqpoint{27.935127in}{0.773588in}}%
\pgfpathlineto{\pgfqpoint{27.883786in}{0.773588in}}%
\pgfpathlineto{\pgfqpoint{27.833067in}{0.773588in}}%
\pgfpathlineto{\pgfqpoint{27.779331in}{0.773588in}}%
\pgfpathlineto{\pgfqpoint{27.728571in}{0.773588in}}%
\pgfpathlineto{\pgfqpoint{27.678294in}{0.773588in}}%
\pgfpathlineto{\pgfqpoint{27.626352in}{0.773588in}}%
\pgfpathlineto{\pgfqpoint{27.576178in}{0.773588in}}%
\pgfpathlineto{\pgfqpoint{27.525849in}{0.773588in}}%
\pgfpathlineto{\pgfqpoint{27.473173in}{0.773588in}}%
\pgfpathlineto{\pgfqpoint{27.423066in}{0.773588in}}%
\pgfpathlineto{\pgfqpoint{27.372863in}{0.773588in}}%
\pgfpathlineto{\pgfqpoint{27.321160in}{0.773588in}}%
\pgfpathlineto{\pgfqpoint{27.269679in}{0.773588in}}%
\pgfpathlineto{\pgfqpoint{27.218587in}{0.773588in}}%
\pgfpathlineto{\pgfqpoint{27.165274in}{0.773588in}}%
\pgfpathlineto{\pgfqpoint{27.113950in}{0.773588in}}%
\pgfpathlineto{\pgfqpoint{27.063062in}{0.773588in}}%
\pgfpathlineto{\pgfqpoint{27.010539in}{0.773588in}}%
\pgfpathlineto{\pgfqpoint{26.960748in}{0.773588in}}%
\pgfpathlineto{\pgfqpoint{26.909530in}{0.773588in}}%
\pgfpathlineto{\pgfqpoint{26.857706in}{0.773588in}}%
\pgfpathlineto{\pgfqpoint{26.807173in}{0.773588in}}%
\pgfpathlineto{\pgfqpoint{26.756448in}{0.773588in}}%
\pgfpathlineto{\pgfqpoint{26.704176in}{0.773588in}}%
\pgfpathlineto{\pgfqpoint{26.653161in}{0.773588in}}%
\pgfpathlineto{\pgfqpoint{26.601868in}{0.773588in}}%
\pgfpathlineto{\pgfqpoint{26.549446in}{0.773588in}}%
\pgfpathlineto{\pgfqpoint{26.499064in}{0.773588in}}%
\pgfpathlineto{\pgfqpoint{26.447655in}{0.773588in}}%
\pgfpathlineto{\pgfqpoint{26.395829in}{0.773588in}}%
\pgfpathlineto{\pgfqpoint{26.346588in}{0.773588in}}%
\pgfpathlineto{\pgfqpoint{26.297518in}{0.773588in}}%
\pgfpathlineto{\pgfqpoint{26.245717in}{0.773588in}}%
\pgfpathlineto{\pgfqpoint{26.195263in}{0.773588in}}%
\pgfpathlineto{\pgfqpoint{26.144837in}{0.773588in}}%
\pgfpathlineto{\pgfqpoint{26.093895in}{0.773588in}}%
\pgfpathlineto{\pgfqpoint{26.044216in}{0.773588in}}%
\pgfpathlineto{\pgfqpoint{25.994130in}{0.773588in}}%
\pgfpathlineto{\pgfqpoint{25.942005in}{0.773588in}}%
\pgfpathlineto{\pgfqpoint{25.891991in}{0.773588in}}%
\pgfpathlineto{\pgfqpoint{25.841969in}{0.773588in}}%
\pgfpathlineto{\pgfqpoint{25.790287in}{0.773588in}}%
\pgfpathlineto{\pgfqpoint{25.740073in}{0.773588in}}%
\pgfpathlineto{\pgfqpoint{25.689152in}{0.773588in}}%
\pgfpathlineto{\pgfqpoint{25.637513in}{0.773588in}}%
\pgfpathlineto{\pgfqpoint{25.586387in}{0.773588in}}%
\pgfpathlineto{\pgfqpoint{25.535558in}{0.773588in}}%
\pgfpathlineto{\pgfqpoint{25.483381in}{0.773588in}}%
\pgfpathlineto{\pgfqpoint{25.433954in}{0.773588in}}%
\pgfpathlineto{\pgfqpoint{25.383622in}{0.773588in}}%
\pgfpathlineto{\pgfqpoint{25.332608in}{0.773588in}}%
\pgfpathlineto{\pgfqpoint{25.282222in}{0.773588in}}%
\pgfpathlineto{\pgfqpoint{25.232944in}{0.773588in}}%
\pgfpathlineto{\pgfqpoint{25.181554in}{0.773588in}}%
\pgfpathlineto{\pgfqpoint{25.131043in}{0.773588in}}%
\pgfpathlineto{\pgfqpoint{25.080539in}{0.773588in}}%
\pgfpathlineto{\pgfqpoint{25.028630in}{0.773588in}}%
\pgfpathlineto{\pgfqpoint{24.978211in}{0.773588in}}%
\pgfpathlineto{\pgfqpoint{24.928869in}{0.773588in}}%
\pgfpathlineto{\pgfqpoint{24.878143in}{0.773588in}}%
\pgfpathlineto{\pgfqpoint{24.828140in}{0.773588in}}%
\pgfpathlineto{\pgfqpoint{24.778304in}{0.773588in}}%
\pgfpathlineto{\pgfqpoint{24.726499in}{0.773588in}}%
\pgfpathlineto{\pgfqpoint{24.676389in}{0.773588in}}%
\pgfpathlineto{\pgfqpoint{24.626556in}{0.773588in}}%
\pgfpathlineto{\pgfqpoint{24.573998in}{0.773588in}}%
\pgfpathlineto{\pgfqpoint{24.524258in}{0.773588in}}%
\pgfpathlineto{\pgfqpoint{24.474381in}{0.773588in}}%
\pgfpathlineto{\pgfqpoint{24.422475in}{0.773588in}}%
\pgfpathlineto{\pgfqpoint{24.371890in}{0.773588in}}%
\pgfpathlineto{\pgfqpoint{24.321907in}{0.773588in}}%
\pgfpathlineto{\pgfqpoint{24.271277in}{0.773588in}}%
\pgfpathlineto{\pgfqpoint{24.221010in}{0.773588in}}%
\pgfpathlineto{\pgfqpoint{24.170873in}{0.773588in}}%
\pgfpathlineto{\pgfqpoint{24.118637in}{0.773588in}}%
\pgfpathlineto{\pgfqpoint{24.066933in}{0.773588in}}%
\pgfpathlineto{\pgfqpoint{24.016442in}{0.773588in}}%
\pgfpathlineto{\pgfqpoint{23.964514in}{0.773588in}}%
\pgfpathlineto{\pgfqpoint{23.913088in}{0.773588in}}%
\pgfpathlineto{\pgfqpoint{23.862673in}{0.773588in}}%
\pgfpathlineto{\pgfqpoint{23.811512in}{0.773588in}}%
\pgfpathlineto{\pgfqpoint{23.762329in}{0.773588in}}%
\pgfpathlineto{\pgfqpoint{23.713074in}{0.773588in}}%
\pgfpathlineto{\pgfqpoint{23.661575in}{0.773588in}}%
\pgfpathlineto{\pgfqpoint{23.611923in}{0.773588in}}%
\pgfpathlineto{\pgfqpoint{23.563429in}{0.773588in}}%
\pgfpathlineto{\pgfqpoint{23.511867in}{0.773588in}}%
\pgfpathlineto{\pgfqpoint{23.462017in}{0.773588in}}%
\pgfpathlineto{\pgfqpoint{23.411909in}{0.773588in}}%
\pgfpathlineto{\pgfqpoint{23.361310in}{0.773588in}}%
\pgfpathlineto{\pgfqpoint{23.310918in}{0.773588in}}%
\pgfpathlineto{\pgfqpoint{23.260726in}{0.773588in}}%
\pgfpathlineto{\pgfqpoint{23.208694in}{0.773588in}}%
\pgfpathlineto{\pgfqpoint{23.157756in}{0.773588in}}%
\pgfpathlineto{\pgfqpoint{23.107134in}{0.773588in}}%
\pgfpathlineto{\pgfqpoint{23.055187in}{0.773588in}}%
\pgfpathlineto{\pgfqpoint{23.006022in}{0.773588in}}%
\pgfpathlineto{\pgfqpoint{22.956359in}{0.773588in}}%
\pgfpathlineto{\pgfqpoint{22.904236in}{0.773588in}}%
\pgfpathlineto{\pgfqpoint{22.853966in}{0.773588in}}%
\pgfpathlineto{\pgfqpoint{22.803829in}{0.773588in}}%
\pgfpathlineto{\pgfqpoint{22.751700in}{0.773588in}}%
\pgfpathlineto{\pgfqpoint{22.703398in}{0.773588in}}%
\pgfpathlineto{\pgfqpoint{22.654822in}{0.773588in}}%
\pgfpathlineto{\pgfqpoint{22.603657in}{0.773588in}}%
\pgfpathlineto{\pgfqpoint{22.554073in}{0.773588in}}%
\pgfpathlineto{\pgfqpoint{22.503552in}{0.773588in}}%
\pgfpathlineto{\pgfqpoint{22.451976in}{0.773588in}}%
\pgfpathlineto{\pgfqpoint{22.402084in}{0.773588in}}%
\pgfpathlineto{\pgfqpoint{22.352827in}{0.773588in}}%
\pgfpathlineto{\pgfqpoint{22.300933in}{0.773588in}}%
\pgfpathlineto{\pgfqpoint{22.250634in}{0.773588in}}%
\pgfpathlineto{\pgfqpoint{22.200587in}{0.773588in}}%
\pgfpathlineto{\pgfqpoint{22.148817in}{0.773588in}}%
\pgfpathlineto{\pgfqpoint{22.099907in}{0.773588in}}%
\pgfpathlineto{\pgfqpoint{22.050354in}{0.773588in}}%
\pgfpathlineto{\pgfqpoint{21.999803in}{0.773588in}}%
\pgfpathlineto{\pgfqpoint{21.950544in}{0.773588in}}%
\pgfpathlineto{\pgfqpoint{21.900834in}{0.773588in}}%
\pgfpathlineto{\pgfqpoint{21.849580in}{0.773588in}}%
\pgfpathlineto{\pgfqpoint{21.800474in}{0.773588in}}%
\pgfpathlineto{\pgfqpoint{21.751624in}{0.773588in}}%
\pgfpathlineto{\pgfqpoint{21.700864in}{0.773588in}}%
\pgfpathlineto{\pgfqpoint{21.652276in}{0.773588in}}%
\pgfpathlineto{\pgfqpoint{21.602538in}{0.773588in}}%
\pgfpathlineto{\pgfqpoint{21.550957in}{0.773588in}}%
\pgfpathlineto{\pgfqpoint{21.501023in}{0.773588in}}%
\pgfpathlineto{\pgfqpoint{21.450830in}{0.773588in}}%
\pgfpathlineto{\pgfqpoint{21.399733in}{0.773588in}}%
\pgfpathlineto{\pgfqpoint{21.350353in}{0.773588in}}%
\pgfpathlineto{\pgfqpoint{21.300955in}{0.773588in}}%
\pgfpathlineto{\pgfqpoint{21.250655in}{0.773588in}}%
\pgfpathlineto{\pgfqpoint{21.201683in}{0.773588in}}%
\pgfpathlineto{\pgfqpoint{21.152719in}{0.773588in}}%
\pgfpathlineto{\pgfqpoint{21.102370in}{0.773588in}}%
\pgfpathlineto{\pgfqpoint{21.053515in}{0.773588in}}%
\pgfpathlineto{\pgfqpoint{21.003693in}{0.773588in}}%
\pgfpathlineto{\pgfqpoint{20.953585in}{0.773588in}}%
\pgfpathlineto{\pgfqpoint{20.904164in}{0.773588in}}%
\pgfpathlineto{\pgfqpoint{20.854118in}{0.773588in}}%
\pgfpathlineto{\pgfqpoint{20.802965in}{0.773588in}}%
\pgfpathlineto{\pgfqpoint{20.753558in}{0.773588in}}%
\pgfpathlineto{\pgfqpoint{20.704616in}{0.773588in}}%
\pgfpathlineto{\pgfqpoint{20.654668in}{0.773588in}}%
\pgfpathlineto{\pgfqpoint{20.605567in}{0.773588in}}%
\pgfpathlineto{\pgfqpoint{20.556290in}{0.773588in}}%
\pgfpathlineto{\pgfqpoint{20.505208in}{0.773588in}}%
\pgfpathlineto{\pgfqpoint{20.455604in}{0.773588in}}%
\pgfpathlineto{\pgfqpoint{20.406414in}{0.773588in}}%
\pgfpathlineto{\pgfqpoint{20.355462in}{0.773588in}}%
\pgfpathlineto{\pgfqpoint{20.305924in}{0.773588in}}%
\pgfpathlineto{\pgfqpoint{20.255593in}{0.773588in}}%
\pgfpathlineto{\pgfqpoint{20.203615in}{0.773588in}}%
\pgfpathlineto{\pgfqpoint{20.153679in}{0.773588in}}%
\pgfpathlineto{\pgfqpoint{20.104350in}{0.773588in}}%
\pgfpathlineto{\pgfqpoint{20.053312in}{0.773588in}}%
\pgfpathlineto{\pgfqpoint{20.003692in}{0.773588in}}%
\pgfpathlineto{\pgfqpoint{19.955252in}{0.773588in}}%
\pgfpathlineto{\pgfqpoint{19.906615in}{0.773588in}}%
\pgfpathlineto{\pgfqpoint{19.858933in}{0.773588in}}%
\pgfpathlineto{\pgfqpoint{19.810712in}{0.773588in}}%
\pgfpathlineto{\pgfqpoint{19.761678in}{0.773588in}}%
\pgfpathlineto{\pgfqpoint{19.714265in}{0.773588in}}%
\pgfpathlineto{\pgfqpoint{19.666940in}{0.773588in}}%
\pgfpathlineto{\pgfqpoint{19.617707in}{0.773588in}}%
\pgfpathlineto{\pgfqpoint{19.569747in}{0.773588in}}%
\pgfpathlineto{\pgfqpoint{19.522292in}{0.773588in}}%
\pgfpathlineto{\pgfqpoint{19.473154in}{0.773588in}}%
\pgfpathlineto{\pgfqpoint{19.425178in}{0.773588in}}%
\pgfpathlineto{\pgfqpoint{19.376890in}{0.773588in}}%
\pgfpathlineto{\pgfqpoint{19.327829in}{0.773588in}}%
\pgfpathlineto{\pgfqpoint{19.279742in}{0.773588in}}%
\pgfpathlineto{\pgfqpoint{19.231826in}{0.773588in}}%
\pgfpathlineto{\pgfqpoint{19.181302in}{0.773588in}}%
\pgfpathlineto{\pgfqpoint{19.132724in}{0.773588in}}%
\pgfpathlineto{\pgfqpoint{19.084207in}{0.773588in}}%
\pgfpathlineto{\pgfqpoint{19.033794in}{0.773588in}}%
\pgfpathlineto{\pgfqpoint{18.985002in}{0.773588in}}%
\pgfpathlineto{\pgfqpoint{18.937671in}{0.773588in}}%
\pgfpathlineto{\pgfqpoint{18.888188in}{0.773588in}}%
\pgfpathlineto{\pgfqpoint{18.839312in}{0.773588in}}%
\pgfpathlineto{\pgfqpoint{18.791212in}{0.773588in}}%
\pgfpathlineto{\pgfqpoint{18.741665in}{0.773588in}}%
\pgfpathlineto{\pgfqpoint{18.693122in}{0.773588in}}%
\pgfpathlineto{\pgfqpoint{18.644639in}{0.773588in}}%
\pgfpathlineto{\pgfqpoint{18.595724in}{0.773588in}}%
\pgfpathlineto{\pgfqpoint{18.548460in}{0.773588in}}%
\pgfpathlineto{\pgfqpoint{18.500915in}{0.773588in}}%
\pgfpathlineto{\pgfqpoint{18.452284in}{0.773588in}}%
\pgfpathlineto{\pgfqpoint{18.405242in}{0.773588in}}%
\pgfpathlineto{\pgfqpoint{18.357538in}{0.773588in}}%
\pgfpathlineto{\pgfqpoint{18.308441in}{0.773588in}}%
\pgfpathlineto{\pgfqpoint{18.260871in}{0.773588in}}%
\pgfpathlineto{\pgfqpoint{18.213333in}{0.773588in}}%
\pgfpathlineto{\pgfqpoint{18.164405in}{0.773588in}}%
\pgfpathlineto{\pgfqpoint{18.117307in}{0.773588in}}%
\pgfpathlineto{\pgfqpoint{18.069524in}{0.773588in}}%
\pgfpathlineto{\pgfqpoint{18.020026in}{0.773588in}}%
\pgfpathlineto{\pgfqpoint{17.970910in}{0.773588in}}%
\pgfpathlineto{\pgfqpoint{17.922885in}{0.773588in}}%
\pgfpathlineto{\pgfqpoint{17.874600in}{0.773588in}}%
\pgfpathlineto{\pgfqpoint{17.826809in}{0.773588in}}%
\pgfpathlineto{\pgfqpoint{17.779014in}{0.773588in}}%
\pgfpathlineto{\pgfqpoint{17.729727in}{0.773588in}}%
\pgfpathlineto{\pgfqpoint{17.681914in}{0.773588in}}%
\pgfpathlineto{\pgfqpoint{17.634653in}{0.773588in}}%
\pgfpathlineto{\pgfqpoint{17.585742in}{0.773588in}}%
\pgfpathlineto{\pgfqpoint{17.538409in}{0.773588in}}%
\pgfpathlineto{\pgfqpoint{17.491125in}{0.773588in}}%
\pgfpathlineto{\pgfqpoint{17.442362in}{0.773588in}}%
\pgfpathlineto{\pgfqpoint{17.394714in}{0.773588in}}%
\pgfpathlineto{\pgfqpoint{17.347576in}{0.773588in}}%
\pgfpathlineto{\pgfqpoint{17.298941in}{0.773588in}}%
\pgfpathlineto{\pgfqpoint{17.250860in}{0.773588in}}%
\pgfpathlineto{\pgfqpoint{17.203674in}{0.773588in}}%
\pgfpathlineto{\pgfqpoint{17.154507in}{0.773588in}}%
\pgfpathlineto{\pgfqpoint{17.107470in}{0.773588in}}%
\pgfpathlineto{\pgfqpoint{17.059518in}{0.773588in}}%
\pgfpathlineto{\pgfqpoint{17.010585in}{0.773588in}}%
\pgfpathlineto{\pgfqpoint{16.963533in}{0.773588in}}%
\pgfpathlineto{\pgfqpoint{16.916302in}{0.773588in}}%
\pgfpathlineto{\pgfqpoint{16.868266in}{0.773588in}}%
\pgfpathlineto{\pgfqpoint{16.821038in}{0.773588in}}%
\pgfpathlineto{\pgfqpoint{16.773713in}{0.773588in}}%
\pgfpathlineto{\pgfqpoint{16.724568in}{0.773588in}}%
\pgfpathlineto{\pgfqpoint{16.677566in}{0.773588in}}%
\pgfpathlineto{\pgfqpoint{16.630280in}{0.773588in}}%
\pgfpathlineto{\pgfqpoint{16.581135in}{0.773588in}}%
\pgfpathlineto{\pgfqpoint{16.533619in}{0.773588in}}%
\pgfpathlineto{\pgfqpoint{16.486560in}{0.773588in}}%
\pgfpathlineto{\pgfqpoint{16.436622in}{0.773588in}}%
\pgfpathlineto{\pgfqpoint{16.387666in}{0.773588in}}%
\pgfpathlineto{\pgfqpoint{16.339537in}{0.773588in}}%
\pgfpathlineto{\pgfqpoint{16.289817in}{0.773588in}}%
\pgfpathlineto{\pgfqpoint{16.242078in}{0.773588in}}%
\pgfpathlineto{\pgfqpoint{16.195175in}{0.773588in}}%
\pgfpathlineto{\pgfqpoint{16.146338in}{0.773588in}}%
\pgfpathlineto{\pgfqpoint{16.098697in}{0.773588in}}%
\pgfpathlineto{\pgfqpoint{16.051820in}{0.773588in}}%
\pgfpathlineto{\pgfqpoint{16.003926in}{0.773588in}}%
\pgfpathlineto{\pgfqpoint{15.957799in}{0.773588in}}%
\pgfpathlineto{\pgfqpoint{15.911811in}{0.773588in}}%
\pgfpathlineto{\pgfqpoint{15.863592in}{0.773588in}}%
\pgfpathlineto{\pgfqpoint{15.817273in}{0.773588in}}%
\pgfpathlineto{\pgfqpoint{15.770154in}{0.773588in}}%
\pgfpathlineto{\pgfqpoint{15.722814in}{0.773588in}}%
\pgfpathlineto{\pgfqpoint{15.676019in}{0.773588in}}%
\pgfpathlineto{\pgfqpoint{15.628660in}{0.773588in}}%
\pgfpathlineto{\pgfqpoint{15.579815in}{0.773588in}}%
\pgfpathlineto{\pgfqpoint{15.532329in}{0.773588in}}%
\pgfpathlineto{\pgfqpoint{15.484858in}{0.773588in}}%
\pgfpathlineto{\pgfqpoint{15.436052in}{0.773588in}}%
\pgfpathlineto{\pgfqpoint{15.389896in}{0.773588in}}%
\pgfpathlineto{\pgfqpoint{15.343883in}{0.773588in}}%
\pgfpathlineto{\pgfqpoint{15.295866in}{0.773588in}}%
\pgfpathlineto{\pgfqpoint{15.248863in}{0.773588in}}%
\pgfpathlineto{\pgfqpoint{15.201456in}{0.773588in}}%
\pgfpathlineto{\pgfqpoint{15.152658in}{0.773588in}}%
\pgfpathlineto{\pgfqpoint{15.105478in}{0.773588in}}%
\pgfpathlineto{\pgfqpoint{15.058774in}{0.773588in}}%
\pgfpathlineto{\pgfqpoint{15.010214in}{0.773588in}}%
\pgfpathlineto{\pgfqpoint{14.963167in}{0.773588in}}%
\pgfpathlineto{\pgfqpoint{14.916239in}{0.773588in}}%
\pgfpathlineto{\pgfqpoint{14.868242in}{0.773588in}}%
\pgfpathlineto{\pgfqpoint{14.821474in}{0.773588in}}%
\pgfpathlineto{\pgfqpoint{14.774752in}{0.773588in}}%
\pgfpathlineto{\pgfqpoint{14.726757in}{0.773588in}}%
\pgfpathlineto{\pgfqpoint{14.680226in}{0.773588in}}%
\pgfpathlineto{\pgfqpoint{14.633992in}{0.773588in}}%
\pgfpathlineto{\pgfqpoint{14.585477in}{0.773588in}}%
\pgfpathlineto{\pgfqpoint{14.538630in}{0.773588in}}%
\pgfpathlineto{\pgfqpoint{14.491726in}{0.773588in}}%
\pgfpathlineto{\pgfqpoint{14.443729in}{0.773588in}}%
\pgfpathlineto{\pgfqpoint{14.397126in}{0.773588in}}%
\pgfpathlineto{\pgfqpoint{14.350224in}{0.773588in}}%
\pgfpathlineto{\pgfqpoint{14.302418in}{0.773588in}}%
\pgfpathlineto{\pgfqpoint{14.255593in}{0.773588in}}%
\pgfpathlineto{\pgfqpoint{14.209376in}{0.773588in}}%
\pgfpathlineto{\pgfqpoint{14.160147in}{0.773588in}}%
\pgfpathlineto{\pgfqpoint{14.112147in}{0.773588in}}%
\pgfpathlineto{\pgfqpoint{14.064951in}{0.773588in}}%
\pgfpathlineto{\pgfqpoint{14.016344in}{0.773588in}}%
\pgfpathlineto{\pgfqpoint{13.969353in}{0.773588in}}%
\pgfpathlineto{\pgfqpoint{13.922606in}{0.773588in}}%
\pgfpathlineto{\pgfqpoint{13.873610in}{0.773588in}}%
\pgfpathlineto{\pgfqpoint{13.825297in}{0.773588in}}%
\pgfpathlineto{\pgfqpoint{13.777441in}{0.773588in}}%
\pgfpathlineto{\pgfqpoint{13.729143in}{0.773588in}}%
\pgfpathlineto{\pgfqpoint{13.682151in}{0.773588in}}%
\pgfpathlineto{\pgfqpoint{13.635346in}{0.773588in}}%
\pgfpathlineto{\pgfqpoint{13.587249in}{0.773588in}}%
\pgfpathlineto{\pgfqpoint{13.541646in}{0.773588in}}%
\pgfpathlineto{\pgfqpoint{13.495727in}{0.773588in}}%
\pgfpathlineto{\pgfqpoint{13.448069in}{0.773588in}}%
\pgfpathlineto{\pgfqpoint{13.401859in}{0.773588in}}%
\pgfpathlineto{\pgfqpoint{13.356518in}{0.773588in}}%
\pgfpathlineto{\pgfqpoint{13.308974in}{0.773588in}}%
\pgfpathlineto{\pgfqpoint{13.262980in}{0.773588in}}%
\pgfpathlineto{\pgfqpoint{13.216719in}{0.773588in}}%
\pgfpathlineto{\pgfqpoint{13.169075in}{0.773588in}}%
\pgfpathlineto{\pgfqpoint{13.123339in}{0.773588in}}%
\pgfpathlineto{\pgfqpoint{13.077275in}{0.773588in}}%
\pgfpathlineto{\pgfqpoint{13.030499in}{0.773588in}}%
\pgfpathlineto{\pgfqpoint{12.983596in}{0.773588in}}%
\pgfpathlineto{\pgfqpoint{12.936799in}{0.773588in}}%
\pgfpathlineto{\pgfqpoint{12.889988in}{0.773588in}}%
\pgfpathlineto{\pgfqpoint{12.843155in}{0.773588in}}%
\pgfpathlineto{\pgfqpoint{12.796907in}{0.773588in}}%
\pgfpathlineto{\pgfqpoint{12.749025in}{0.773588in}}%
\pgfpathlineto{\pgfqpoint{12.701835in}{0.773588in}}%
\pgfpathlineto{\pgfqpoint{12.655424in}{0.773588in}}%
\pgfpathlineto{\pgfqpoint{12.607615in}{0.773588in}}%
\pgfpathlineto{\pgfqpoint{12.560711in}{0.773588in}}%
\pgfpathlineto{\pgfqpoint{12.514481in}{0.773588in}}%
\pgfpathlineto{\pgfqpoint{12.467453in}{0.773588in}}%
\pgfpathlineto{\pgfqpoint{12.421372in}{0.773588in}}%
\pgfpathlineto{\pgfqpoint{12.375261in}{0.773588in}}%
\pgfpathlineto{\pgfqpoint{12.328136in}{0.773588in}}%
\pgfpathlineto{\pgfqpoint{12.282271in}{0.773588in}}%
\pgfpathlineto{\pgfqpoint{12.235985in}{0.773588in}}%
\pgfpathlineto{\pgfqpoint{12.187925in}{0.773588in}}%
\pgfpathlineto{\pgfqpoint{12.141887in}{0.773588in}}%
\pgfpathlineto{\pgfqpoint{12.096623in}{0.773588in}}%
\pgfpathlineto{\pgfqpoint{12.049869in}{0.773588in}}%
\pgfpathlineto{\pgfqpoint{12.004414in}{0.773588in}}%
\pgfpathlineto{\pgfqpoint{11.959143in}{0.773588in}}%
\pgfpathlineto{\pgfqpoint{11.911501in}{0.773588in}}%
\pgfpathlineto{\pgfqpoint{11.864862in}{0.773588in}}%
\pgfpathlineto{\pgfqpoint{11.819406in}{0.773588in}}%
\pgfpathlineto{\pgfqpoint{11.772894in}{0.773588in}}%
\pgfpathlineto{\pgfqpoint{11.727607in}{0.773588in}}%
\pgfpathlineto{\pgfqpoint{11.682053in}{0.773588in}}%
\pgfpathlineto{\pgfqpoint{11.634589in}{0.773588in}}%
\pgfpathlineto{\pgfqpoint{11.588799in}{0.773588in}}%
\pgfpathlineto{\pgfqpoint{11.542665in}{0.773588in}}%
\pgfpathlineto{\pgfqpoint{11.494874in}{0.773588in}}%
\pgfpathlineto{\pgfqpoint{11.448531in}{0.773588in}}%
\pgfpathlineto{\pgfqpoint{11.402044in}{0.773588in}}%
\pgfpathlineto{\pgfqpoint{11.353756in}{0.773588in}}%
\pgfpathlineto{\pgfqpoint{11.307361in}{0.773588in}}%
\pgfpathlineto{\pgfqpoint{11.261350in}{0.773588in}}%
\pgfpathlineto{\pgfqpoint{11.213355in}{0.773588in}}%
\pgfpathlineto{\pgfqpoint{11.167471in}{0.773588in}}%
\pgfpathlineto{\pgfqpoint{11.121978in}{0.773588in}}%
\pgfpathlineto{\pgfqpoint{11.074633in}{0.773588in}}%
\pgfpathlineto{\pgfqpoint{11.029398in}{0.773588in}}%
\pgfpathlineto{\pgfqpoint{10.984145in}{0.773588in}}%
\pgfpathlineto{\pgfqpoint{10.937381in}{0.773588in}}%
\pgfpathlineto{\pgfqpoint{10.891801in}{0.773588in}}%
\pgfpathlineto{\pgfqpoint{10.845790in}{0.773588in}}%
\pgfpathlineto{\pgfqpoint{10.798768in}{0.773588in}}%
\pgfpathlineto{\pgfqpoint{10.753168in}{0.773588in}}%
\pgfpathlineto{\pgfqpoint{10.707461in}{0.773588in}}%
\pgfpathlineto{\pgfqpoint{10.660112in}{0.773588in}}%
\pgfpathlineto{\pgfqpoint{10.613886in}{0.773588in}}%
\pgfpathlineto{\pgfqpoint{10.568154in}{0.773588in}}%
\pgfpathlineto{\pgfqpoint{10.521105in}{0.773588in}}%
\pgfpathlineto{\pgfqpoint{10.475642in}{0.773588in}}%
\pgfpathlineto{\pgfqpoint{10.429362in}{0.773588in}}%
\pgfpathlineto{\pgfqpoint{10.381570in}{0.773588in}}%
\pgfpathlineto{\pgfqpoint{10.335334in}{0.773588in}}%
\pgfpathlineto{\pgfqpoint{10.289084in}{0.773588in}}%
\pgfpathlineto{\pgfqpoint{10.242601in}{0.773588in}}%
\pgfpathlineto{\pgfqpoint{10.197247in}{0.773588in}}%
\pgfpathlineto{\pgfqpoint{10.151348in}{0.773588in}}%
\pgfpathlineto{\pgfqpoint{10.103098in}{0.773588in}}%
\pgfpathlineto{\pgfqpoint{10.057613in}{0.773588in}}%
\pgfpathlineto{\pgfqpoint{10.011915in}{0.773588in}}%
\pgfpathlineto{\pgfqpoint{9.964562in}{0.773588in}}%
\pgfpathlineto{\pgfqpoint{9.918842in}{0.773588in}}%
\pgfpathlineto{\pgfqpoint{9.873173in}{0.773588in}}%
\pgfpathlineto{\pgfqpoint{9.826599in}{0.773588in}}%
\pgfpathlineto{\pgfqpoint{9.781321in}{0.773588in}}%
\pgfpathlineto{\pgfqpoint{9.735785in}{0.773588in}}%
\pgfpathlineto{\pgfqpoint{9.689497in}{0.773588in}}%
\pgfpathlineto{\pgfqpoint{9.644121in}{0.773588in}}%
\pgfpathlineto{\pgfqpoint{9.598338in}{0.773588in}}%
\pgfpathlineto{\pgfqpoint{9.552197in}{0.773588in}}%
\pgfpathlineto{\pgfqpoint{9.507583in}{0.773588in}}%
\pgfpathlineto{\pgfqpoint{9.462006in}{0.773588in}}%
\pgfpathlineto{\pgfqpoint{9.415178in}{0.773588in}}%
\pgfpathlineto{\pgfqpoint{9.369639in}{0.773588in}}%
\pgfpathlineto{\pgfqpoint{9.324423in}{0.773588in}}%
\pgfpathlineto{\pgfqpoint{9.278439in}{0.773588in}}%
\pgfpathlineto{\pgfqpoint{9.233107in}{0.773588in}}%
\pgfpathlineto{\pgfqpoint{9.188332in}{0.773588in}}%
\pgfpathlineto{\pgfqpoint{9.141270in}{0.773588in}}%
\pgfpathlineto{\pgfqpoint{9.095727in}{0.773588in}}%
\pgfpathlineto{\pgfqpoint{9.050165in}{0.773588in}}%
\pgfpathlineto{\pgfqpoint{9.003947in}{0.773588in}}%
\pgfpathlineto{\pgfqpoint{8.957924in}{0.773588in}}%
\pgfpathlineto{\pgfqpoint{8.912131in}{0.773588in}}%
\pgfpathlineto{\pgfqpoint{8.866242in}{0.773588in}}%
\pgfpathlineto{\pgfqpoint{8.821068in}{0.773588in}}%
\pgfpathlineto{\pgfqpoint{8.774902in}{0.773588in}}%
\pgfpathlineto{\pgfqpoint{8.727404in}{0.773588in}}%
\pgfpathlineto{\pgfqpoint{8.681080in}{0.773588in}}%
\pgfpathlineto{\pgfqpoint{8.635315in}{0.773588in}}%
\pgfpathlineto{\pgfqpoint{8.589124in}{0.773588in}}%
\pgfpathlineto{\pgfqpoint{8.543976in}{0.773588in}}%
\pgfpathlineto{\pgfqpoint{8.499298in}{0.773588in}}%
\pgfpathlineto{\pgfqpoint{8.453780in}{0.773588in}}%
\pgfpathlineto{\pgfqpoint{8.409767in}{0.773588in}}%
\pgfpathlineto{\pgfqpoint{8.364636in}{0.773588in}}%
\pgfpathlineto{\pgfqpoint{8.318789in}{0.773588in}}%
\pgfpathlineto{\pgfqpoint{8.273006in}{0.773588in}}%
\pgfpathlineto{\pgfqpoint{8.227930in}{0.773588in}}%
\pgfpathlineto{\pgfqpoint{8.181791in}{0.773588in}}%
\pgfpathlineto{\pgfqpoint{8.136842in}{0.773588in}}%
\pgfpathlineto{\pgfqpoint{8.091881in}{0.773588in}}%
\pgfpathlineto{\pgfqpoint{8.045278in}{0.773588in}}%
\pgfpathlineto{\pgfqpoint{8.000573in}{0.773588in}}%
\pgfpathlineto{\pgfqpoint{7.955879in}{0.773588in}}%
\pgfpathlineto{\pgfqpoint{7.910161in}{0.773588in}}%
\pgfpathlineto{\pgfqpoint{7.865263in}{0.773588in}}%
\pgfpathlineto{\pgfqpoint{7.819947in}{0.773588in}}%
\pgfpathlineto{\pgfqpoint{7.773226in}{0.773588in}}%
\pgfpathlineto{\pgfqpoint{7.728803in}{0.773588in}}%
\pgfpathlineto{\pgfqpoint{7.682978in}{0.773588in}}%
\pgfpathlineto{\pgfqpoint{7.636592in}{0.773588in}}%
\pgfpathlineto{\pgfqpoint{7.591890in}{0.773588in}}%
\pgfpathlineto{\pgfqpoint{7.546892in}{0.773588in}}%
\pgfpathlineto{\pgfqpoint{7.500683in}{0.773588in}}%
\pgfpathlineto{\pgfqpoint{7.455548in}{0.773588in}}%
\pgfpathlineto{\pgfqpoint{7.409977in}{0.773588in}}%
\pgfpathlineto{\pgfqpoint{7.363749in}{0.773588in}}%
\pgfpathlineto{\pgfqpoint{7.318484in}{0.773588in}}%
\pgfpathlineto{\pgfqpoint{7.273914in}{0.773588in}}%
\pgfpathlineto{\pgfqpoint{7.228120in}{0.773588in}}%
\pgfpathlineto{\pgfqpoint{7.184277in}{0.773588in}}%
\pgfpathlineto{\pgfqpoint{7.140134in}{0.773588in}}%
\pgfpathlineto{\pgfqpoint{7.094205in}{0.773588in}}%
\pgfpathlineto{\pgfqpoint{7.050071in}{0.773588in}}%
\pgfpathlineto{\pgfqpoint{7.005149in}{0.773588in}}%
\pgfpathlineto{\pgfqpoint{6.958763in}{0.773588in}}%
\pgfpathlineto{\pgfqpoint{6.914194in}{0.773588in}}%
\pgfpathlineto{\pgfqpoint{6.869544in}{0.773588in}}%
\pgfpathlineto{\pgfqpoint{6.824012in}{0.773588in}}%
\pgfpathlineto{\pgfqpoint{6.779295in}{0.773588in}}%
\pgfpathlineto{\pgfqpoint{6.734887in}{0.773588in}}%
\pgfpathlineto{\pgfqpoint{6.688504in}{0.773588in}}%
\pgfpathlineto{\pgfqpoint{6.643960in}{0.773588in}}%
\pgfpathlineto{\pgfqpoint{6.599302in}{0.773588in}}%
\pgfpathlineto{\pgfqpoint{6.553117in}{0.773588in}}%
\pgfpathlineto{\pgfqpoint{6.507651in}{0.773588in}}%
\pgfpathlineto{\pgfqpoint{6.461324in}{0.773588in}}%
\pgfpathlineto{\pgfqpoint{6.413399in}{0.773588in}}%
\pgfpathlineto{\pgfqpoint{6.367508in}{0.773588in}}%
\pgfpathlineto{\pgfqpoint{6.321070in}{0.773588in}}%
\pgfpathlineto{\pgfqpoint{6.273584in}{0.773588in}}%
\pgfpathlineto{\pgfqpoint{6.227112in}{0.773588in}}%
\pgfpathlineto{\pgfqpoint{6.180802in}{0.773588in}}%
\pgfpathlineto{\pgfqpoint{6.133235in}{0.773588in}}%
\pgfpathlineto{\pgfqpoint{6.087229in}{0.773588in}}%
\pgfpathlineto{\pgfqpoint{6.041630in}{0.773588in}}%
\pgfpathlineto{\pgfqpoint{5.994686in}{0.773588in}}%
\pgfpathlineto{\pgfqpoint{5.948991in}{0.773588in}}%
\pgfpathlineto{\pgfqpoint{5.903366in}{0.773588in}}%
\pgfpathlineto{\pgfqpoint{5.856181in}{0.773588in}}%
\pgfpathlineto{\pgfqpoint{5.810269in}{0.773588in}}%
\pgfpathlineto{\pgfqpoint{5.765813in}{0.773588in}}%
\pgfpathlineto{\pgfqpoint{5.719569in}{0.773588in}}%
\pgfpathlineto{\pgfqpoint{5.674262in}{0.773588in}}%
\pgfpathlineto{\pgfqpoint{5.628857in}{0.773588in}}%
\pgfpathlineto{\pgfqpoint{5.581803in}{0.773588in}}%
\pgfpathlineto{\pgfqpoint{5.536203in}{0.773588in}}%
\pgfpathlineto{\pgfqpoint{5.491128in}{0.773588in}}%
\pgfpathlineto{\pgfqpoint{5.444526in}{0.773588in}}%
\pgfpathlineto{\pgfqpoint{5.399334in}{0.773588in}}%
\pgfpathlineto{\pgfqpoint{5.353614in}{0.773588in}}%
\pgfpathlineto{\pgfqpoint{5.307244in}{0.773588in}}%
\pgfpathlineto{\pgfqpoint{5.262146in}{0.773588in}}%
\pgfpathlineto{\pgfqpoint{5.216291in}{0.773588in}}%
\pgfpathlineto{\pgfqpoint{5.169192in}{0.773588in}}%
\pgfpathlineto{\pgfqpoint{5.124036in}{0.773588in}}%
\pgfpathlineto{\pgfqpoint{5.078074in}{0.773588in}}%
\pgfpathlineto{\pgfqpoint{5.030678in}{0.773588in}}%
\pgfpathlineto{\pgfqpoint{4.984973in}{0.773588in}}%
\pgfpathlineto{\pgfqpoint{4.939032in}{0.773588in}}%
\pgfpathlineto{\pgfqpoint{4.891298in}{0.773588in}}%
\pgfpathlineto{\pgfqpoint{4.845429in}{0.773588in}}%
\pgfpathlineto{\pgfqpoint{4.799887in}{0.773588in}}%
\pgfpathlineto{\pgfqpoint{4.753449in}{0.773588in}}%
\pgfpathlineto{\pgfqpoint{4.708228in}{0.773588in}}%
\pgfpathlineto{\pgfqpoint{4.663622in}{0.773588in}}%
\pgfpathlineto{\pgfqpoint{4.617918in}{0.773588in}}%
\pgfpathlineto{\pgfqpoint{4.572708in}{0.773588in}}%
\pgfpathlineto{\pgfqpoint{4.527424in}{0.773588in}}%
\pgfpathlineto{\pgfqpoint{4.480684in}{0.773588in}}%
\pgfpathlineto{\pgfqpoint{4.435284in}{0.773588in}}%
\pgfpathlineto{\pgfqpoint{4.390187in}{0.773588in}}%
\pgfpathlineto{\pgfqpoint{4.343118in}{0.773588in}}%
\pgfpathlineto{\pgfqpoint{4.297354in}{0.773588in}}%
\pgfpathlineto{\pgfqpoint{4.252389in}{0.773588in}}%
\pgfpathlineto{\pgfqpoint{4.204543in}{0.773588in}}%
\pgfpathlineto{\pgfqpoint{4.159078in}{0.773588in}}%
\pgfpathlineto{\pgfqpoint{4.114362in}{0.773588in}}%
\pgfpathlineto{\pgfqpoint{4.067507in}{0.773588in}}%
\pgfpathlineto{\pgfqpoint{4.022115in}{0.773588in}}%
\pgfpathlineto{\pgfqpoint{3.976868in}{0.773588in}}%
\pgfpathlineto{\pgfqpoint{3.930307in}{0.773588in}}%
\pgfpathlineto{\pgfqpoint{3.884614in}{0.773588in}}%
\pgfpathlineto{\pgfqpoint{3.839775in}{0.773588in}}%
\pgfpathlineto{\pgfqpoint{3.793040in}{0.773588in}}%
\pgfpathlineto{\pgfqpoint{3.747249in}{0.773588in}}%
\pgfpathlineto{\pgfqpoint{3.701949in}{0.773588in}}%
\pgfpathlineto{\pgfqpoint{3.654909in}{0.773588in}}%
\pgfpathlineto{\pgfqpoint{3.609556in}{0.773588in}}%
\pgfpathlineto{\pgfqpoint{3.564489in}{0.773588in}}%
\pgfpathlineto{\pgfqpoint{3.517815in}{0.773588in}}%
\pgfpathlineto{\pgfqpoint{3.473135in}{0.773588in}}%
\pgfpathlineto{\pgfqpoint{3.428526in}{0.773588in}}%
\pgfpathlineto{\pgfqpoint{3.381274in}{0.773588in}}%
\pgfpathlineto{\pgfqpoint{3.336184in}{0.773588in}}%
\pgfpathlineto{\pgfqpoint{3.290748in}{0.773588in}}%
\pgfpathlineto{\pgfqpoint{3.245458in}{0.773588in}}%
\pgfpathlineto{\pgfqpoint{3.200519in}{0.773588in}}%
\pgfpathlineto{\pgfqpoint{3.155580in}{0.773588in}}%
\pgfpathlineto{\pgfqpoint{3.108180in}{0.773588in}}%
\pgfpathlineto{\pgfqpoint{3.062761in}{0.773588in}}%
\pgfpathlineto{\pgfqpoint{3.017486in}{0.773588in}}%
\pgfpathlineto{\pgfqpoint{2.971917in}{0.773588in}}%
\pgfpathlineto{\pgfqpoint{2.927413in}{0.773588in}}%
\pgfpathlineto{\pgfqpoint{2.883134in}{0.773588in}}%
\pgfpathlineto{\pgfqpoint{2.836381in}{0.773588in}}%
\pgfpathlineto{\pgfqpoint{2.790736in}{0.773588in}}%
\pgfpathlineto{\pgfqpoint{2.745668in}{0.773588in}}%
\pgfpathlineto{\pgfqpoint{2.698034in}{0.773588in}}%
\pgfpathlineto{\pgfqpoint{2.651003in}{0.773588in}}%
\pgfpathlineto{\pgfqpoint{2.604306in}{0.773588in}}%
\pgfpathlineto{\pgfqpoint{2.555498in}{0.773588in}}%
\pgfpathlineto{\pgfqpoint{2.505534in}{0.773588in}}%
\pgfpathlineto{\pgfqpoint{2.452591in}{0.773588in}}%
\pgfpathlineto{\pgfqpoint{2.397147in}{0.773588in}}%
\pgfpathlineto{\pgfqpoint{2.348431in}{0.773588in}}%
\pgfpathlineto{\pgfqpoint{2.299591in}{0.773588in}}%
\pgfpathlineto{\pgfqpoint{2.249804in}{0.773588in}}%
\pgfpathlineto{\pgfqpoint{2.201242in}{0.773588in}}%
\pgfpathlineto{\pgfqpoint{2.153284in}{0.773588in}}%
\pgfpathlineto{\pgfqpoint{2.104325in}{0.773588in}}%
\pgfpathlineto{\pgfqpoint{2.057441in}{0.773588in}}%
\pgfpathlineto{\pgfqpoint{2.011209in}{0.773588in}}%
\pgfpathlineto{\pgfqpoint{1.963476in}{0.773588in}}%
\pgfpathlineto{\pgfqpoint{1.915908in}{0.773588in}}%
\pgfpathlineto{\pgfqpoint{1.869493in}{0.773588in}}%
\pgfpathlineto{\pgfqpoint{1.822804in}{0.773588in}}%
\pgfpathlineto{\pgfqpoint{1.778099in}{0.773588in}}%
\pgfpathlineto{\pgfqpoint{1.734500in}{0.773588in}}%
\pgfpathlineto{\pgfqpoint{1.689326in}{0.773588in}}%
\pgfpathlineto{\pgfqpoint{1.645274in}{0.773588in}}%
\pgfpathlineto{\pgfqpoint{1.600816in}{0.773588in}}%
\pgfpathlineto{\pgfqpoint{1.555718in}{0.773588in}}%
\pgfpathlineto{\pgfqpoint{1.511740in}{0.773588in}}%
\pgfpathlineto{\pgfqpoint{1.468334in}{0.773588in}}%
\pgfpathlineto{\pgfqpoint{1.422957in}{0.773588in}}%
\pgfpathlineto{\pgfqpoint{1.378287in}{0.773588in}}%
\pgfpathlineto{\pgfqpoint{1.334262in}{0.773588in}}%
\pgfpathlineto{\pgfqpoint{1.289074in}{0.773588in}}%
\pgfpathlineto{\pgfqpoint{1.244609in}{0.773588in}}%
\pgfpathlineto{\pgfqpoint{1.200192in}{0.773588in}}%
\pgfpathlineto{\pgfqpoint{1.155171in}{0.773588in}}%
\pgfpathlineto{\pgfqpoint{1.111790in}{0.773588in}}%
\pgfpathlineto{\pgfqpoint{1.067773in}{0.773588in}}%
\pgfpathlineto{\pgfqpoint{1.021908in}{0.773588in}}%
\pgfpathlineto{\pgfqpoint{0.978015in}{0.773588in}}%
\pgfpathlineto{\pgfqpoint{0.933783in}{0.773588in}}%
\pgfpathlineto{\pgfqpoint{0.887244in}{0.773588in}}%
\pgfpathlineto{\pgfqpoint{0.842612in}{0.773588in}}%
\pgfpathlineto{\pgfqpoint{0.797895in}{0.773588in}}%
\pgfpathclose%
\pgfusepath{fill}%
\end{pgfscope}%
\begin{pgfscope}%
\pgfpathrectangle{\pgfqpoint{0.781402in}{0.773588in}}{\pgfqpoint{1.440244in}{5.415119in}}%
\pgfusepath{clip}%
\pgfsetbuttcap%
\pgfsetroundjoin%
\definecolor{currentfill}{rgb}{0.172549,0.627451,0.172549}%
\pgfsetfillcolor{currentfill}%
\pgfsetlinewidth{0.000000pt}%
\definecolor{currentstroke}{rgb}{0.000000,0.000000,0.000000}%
\pgfsetstrokecolor{currentstroke}%
\pgfsetdash{}{0pt}%
\pgfpathmoveto{\pgfqpoint{0.797895in}{0.773588in}}%
\pgfpathlineto{\pgfqpoint{0.797895in}{0.773588in}}%
\pgfpathlineto{\pgfqpoint{0.842612in}{0.773588in}}%
\pgfpathlineto{\pgfqpoint{0.887244in}{0.773588in}}%
\pgfpathlineto{\pgfqpoint{0.933783in}{0.773588in}}%
\pgfpathlineto{\pgfqpoint{0.978015in}{0.773588in}}%
\pgfpathlineto{\pgfqpoint{1.021908in}{0.773588in}}%
\pgfpathlineto{\pgfqpoint{1.067773in}{0.773588in}}%
\pgfpathlineto{\pgfqpoint{1.111790in}{0.773588in}}%
\pgfpathlineto{\pgfqpoint{1.155171in}{0.773588in}}%
\pgfpathlineto{\pgfqpoint{1.200192in}{0.773588in}}%
\pgfpathlineto{\pgfqpoint{1.244609in}{0.773588in}}%
\pgfpathlineto{\pgfqpoint{1.289074in}{0.773588in}}%
\pgfpathlineto{\pgfqpoint{1.334262in}{0.773588in}}%
\pgfpathlineto{\pgfqpoint{1.378287in}{0.773588in}}%
\pgfpathlineto{\pgfqpoint{1.422957in}{0.773588in}}%
\pgfpathlineto{\pgfqpoint{1.468334in}{0.773588in}}%
\pgfpathlineto{\pgfqpoint{1.511740in}{0.773588in}}%
\pgfpathlineto{\pgfqpoint{1.555718in}{0.773588in}}%
\pgfpathlineto{\pgfqpoint{1.600816in}{0.773588in}}%
\pgfpathlineto{\pgfqpoint{1.645274in}{0.773588in}}%
\pgfpathlineto{\pgfqpoint{1.689326in}{0.773588in}}%
\pgfpathlineto{\pgfqpoint{1.734500in}{0.773588in}}%
\pgfpathlineto{\pgfqpoint{1.778099in}{0.773588in}}%
\pgfpathlineto{\pgfqpoint{1.822804in}{0.773588in}}%
\pgfpathlineto{\pgfqpoint{1.869493in}{0.773588in}}%
\pgfpathlineto{\pgfqpoint{1.915908in}{0.773588in}}%
\pgfpathlineto{\pgfqpoint{1.963476in}{0.773588in}}%
\pgfpathlineto{\pgfqpoint{2.011209in}{0.773588in}}%
\pgfpathlineto{\pgfqpoint{2.057441in}{0.773588in}}%
\pgfpathlineto{\pgfqpoint{2.104325in}{0.773588in}}%
\pgfpathlineto{\pgfqpoint{2.153284in}{0.773588in}}%
\pgfpathlineto{\pgfqpoint{2.201242in}{0.773588in}}%
\pgfpathlineto{\pgfqpoint{2.249804in}{0.773588in}}%
\pgfpathlineto{\pgfqpoint{2.299591in}{0.773588in}}%
\pgfpathlineto{\pgfqpoint{2.348431in}{0.773588in}}%
\pgfpathlineto{\pgfqpoint{2.397147in}{0.773588in}}%
\pgfpathlineto{\pgfqpoint{2.452591in}{0.773588in}}%
\pgfpathlineto{\pgfqpoint{2.505534in}{0.773588in}}%
\pgfpathlineto{\pgfqpoint{2.555498in}{0.773588in}}%
\pgfpathlineto{\pgfqpoint{2.604306in}{0.773588in}}%
\pgfpathlineto{\pgfqpoint{2.651003in}{0.773588in}}%
\pgfpathlineto{\pgfqpoint{2.698034in}{0.773588in}}%
\pgfpathlineto{\pgfqpoint{2.745668in}{0.773588in}}%
\pgfpathlineto{\pgfqpoint{2.790736in}{0.773588in}}%
\pgfpathlineto{\pgfqpoint{2.836381in}{0.773588in}}%
\pgfpathlineto{\pgfqpoint{2.883134in}{0.773588in}}%
\pgfpathlineto{\pgfqpoint{2.927413in}{0.773588in}}%
\pgfpathlineto{\pgfqpoint{2.971917in}{0.773588in}}%
\pgfpathlineto{\pgfqpoint{3.017486in}{0.773588in}}%
\pgfpathlineto{\pgfqpoint{3.062761in}{0.773588in}}%
\pgfpathlineto{\pgfqpoint{3.108180in}{0.773588in}}%
\pgfpathlineto{\pgfqpoint{3.155580in}{0.773588in}}%
\pgfpathlineto{\pgfqpoint{3.200519in}{0.773588in}}%
\pgfpathlineto{\pgfqpoint{3.245458in}{0.773588in}}%
\pgfpathlineto{\pgfqpoint{3.290748in}{0.773588in}}%
\pgfpathlineto{\pgfqpoint{3.336184in}{0.773588in}}%
\pgfpathlineto{\pgfqpoint{3.381274in}{0.773588in}}%
\pgfpathlineto{\pgfqpoint{3.428526in}{0.773588in}}%
\pgfpathlineto{\pgfqpoint{3.473135in}{0.773588in}}%
\pgfpathlineto{\pgfqpoint{3.517815in}{0.773588in}}%
\pgfpathlineto{\pgfqpoint{3.564489in}{0.773588in}}%
\pgfpathlineto{\pgfqpoint{3.609556in}{0.773588in}}%
\pgfpathlineto{\pgfqpoint{3.654909in}{0.773588in}}%
\pgfpathlineto{\pgfqpoint{3.701949in}{0.773588in}}%
\pgfpathlineto{\pgfqpoint{3.747249in}{0.773588in}}%
\pgfpathlineto{\pgfqpoint{3.793040in}{0.773588in}}%
\pgfpathlineto{\pgfqpoint{3.839775in}{0.773588in}}%
\pgfpathlineto{\pgfqpoint{3.884614in}{0.773588in}}%
\pgfpathlineto{\pgfqpoint{3.930307in}{0.773588in}}%
\pgfpathlineto{\pgfqpoint{3.976868in}{0.773588in}}%
\pgfpathlineto{\pgfqpoint{4.022115in}{0.773588in}}%
\pgfpathlineto{\pgfqpoint{4.067507in}{0.773588in}}%
\pgfpathlineto{\pgfqpoint{4.114362in}{0.773588in}}%
\pgfpathlineto{\pgfqpoint{4.159078in}{0.773588in}}%
\pgfpathlineto{\pgfqpoint{4.204543in}{0.773588in}}%
\pgfpathlineto{\pgfqpoint{4.252389in}{0.773588in}}%
\pgfpathlineto{\pgfqpoint{4.297354in}{0.773588in}}%
\pgfpathlineto{\pgfqpoint{4.343118in}{0.773588in}}%
\pgfpathlineto{\pgfqpoint{4.390187in}{0.773588in}}%
\pgfpathlineto{\pgfqpoint{4.435284in}{0.773588in}}%
\pgfpathlineto{\pgfqpoint{4.480684in}{0.773588in}}%
\pgfpathlineto{\pgfqpoint{4.527424in}{0.773588in}}%
\pgfpathlineto{\pgfqpoint{4.572708in}{0.773588in}}%
\pgfpathlineto{\pgfqpoint{4.617918in}{0.773588in}}%
\pgfpathlineto{\pgfqpoint{4.663622in}{0.773588in}}%
\pgfpathlineto{\pgfqpoint{4.708228in}{0.773588in}}%
\pgfpathlineto{\pgfqpoint{4.753449in}{0.773588in}}%
\pgfpathlineto{\pgfqpoint{4.799887in}{0.773588in}}%
\pgfpathlineto{\pgfqpoint{4.845429in}{0.773588in}}%
\pgfpathlineto{\pgfqpoint{4.891298in}{0.773588in}}%
\pgfpathlineto{\pgfqpoint{4.939032in}{0.773588in}}%
\pgfpathlineto{\pgfqpoint{4.984973in}{0.773588in}}%
\pgfpathlineto{\pgfqpoint{5.030678in}{0.773588in}}%
\pgfpathlineto{\pgfqpoint{5.078074in}{0.773588in}}%
\pgfpathlineto{\pgfqpoint{5.124036in}{0.773588in}}%
\pgfpathlineto{\pgfqpoint{5.169192in}{0.773588in}}%
\pgfpathlineto{\pgfqpoint{5.216291in}{0.773588in}}%
\pgfpathlineto{\pgfqpoint{5.262146in}{0.773588in}}%
\pgfpathlineto{\pgfqpoint{5.307244in}{0.773588in}}%
\pgfpathlineto{\pgfqpoint{5.353614in}{0.773588in}}%
\pgfpathlineto{\pgfqpoint{5.399334in}{0.773588in}}%
\pgfpathlineto{\pgfqpoint{5.444526in}{0.773588in}}%
\pgfpathlineto{\pgfqpoint{5.491128in}{0.773588in}}%
\pgfpathlineto{\pgfqpoint{5.536203in}{0.773588in}}%
\pgfpathlineto{\pgfqpoint{5.581803in}{0.773588in}}%
\pgfpathlineto{\pgfqpoint{5.628857in}{0.773588in}}%
\pgfpathlineto{\pgfqpoint{5.674262in}{0.773588in}}%
\pgfpathlineto{\pgfqpoint{5.719569in}{0.773588in}}%
\pgfpathlineto{\pgfqpoint{5.765813in}{0.773588in}}%
\pgfpathlineto{\pgfqpoint{5.810269in}{0.773588in}}%
\pgfpathlineto{\pgfqpoint{5.856181in}{0.773588in}}%
\pgfpathlineto{\pgfqpoint{5.903366in}{0.773588in}}%
\pgfpathlineto{\pgfqpoint{5.948991in}{0.773588in}}%
\pgfpathlineto{\pgfqpoint{5.994686in}{0.773588in}}%
\pgfpathlineto{\pgfqpoint{6.041630in}{0.773588in}}%
\pgfpathlineto{\pgfqpoint{6.087229in}{0.773588in}}%
\pgfpathlineto{\pgfqpoint{6.133235in}{0.773588in}}%
\pgfpathlineto{\pgfqpoint{6.180802in}{0.773588in}}%
\pgfpathlineto{\pgfqpoint{6.227112in}{0.773588in}}%
\pgfpathlineto{\pgfqpoint{6.273584in}{0.773588in}}%
\pgfpathlineto{\pgfqpoint{6.321070in}{0.773588in}}%
\pgfpathlineto{\pgfqpoint{6.367508in}{0.773588in}}%
\pgfpathlineto{\pgfqpoint{6.413399in}{0.773588in}}%
\pgfpathlineto{\pgfqpoint{6.461324in}{0.773588in}}%
\pgfpathlineto{\pgfqpoint{6.507651in}{0.773588in}}%
\pgfpathlineto{\pgfqpoint{6.553117in}{0.773588in}}%
\pgfpathlineto{\pgfqpoint{6.599302in}{0.773588in}}%
\pgfpathlineto{\pgfqpoint{6.643960in}{0.773588in}}%
\pgfpathlineto{\pgfqpoint{6.688504in}{0.773588in}}%
\pgfpathlineto{\pgfqpoint{6.734887in}{0.773588in}}%
\pgfpathlineto{\pgfqpoint{6.779295in}{0.773588in}}%
\pgfpathlineto{\pgfqpoint{6.824012in}{0.773588in}}%
\pgfpathlineto{\pgfqpoint{6.869544in}{0.773588in}}%
\pgfpathlineto{\pgfqpoint{6.914194in}{0.773588in}}%
\pgfpathlineto{\pgfqpoint{6.958763in}{0.773588in}}%
\pgfpathlineto{\pgfqpoint{7.005149in}{0.773588in}}%
\pgfpathlineto{\pgfqpoint{7.050071in}{0.773588in}}%
\pgfpathlineto{\pgfqpoint{7.094205in}{0.773588in}}%
\pgfpathlineto{\pgfqpoint{7.140134in}{0.773588in}}%
\pgfpathlineto{\pgfqpoint{7.184277in}{0.773588in}}%
\pgfpathlineto{\pgfqpoint{7.228120in}{0.773588in}}%
\pgfpathlineto{\pgfqpoint{7.273914in}{0.773588in}}%
\pgfpathlineto{\pgfqpoint{7.318484in}{0.773588in}}%
\pgfpathlineto{\pgfqpoint{7.363749in}{0.773588in}}%
\pgfpathlineto{\pgfqpoint{7.409977in}{0.773588in}}%
\pgfpathlineto{\pgfqpoint{7.455548in}{0.773588in}}%
\pgfpathlineto{\pgfqpoint{7.500683in}{0.773588in}}%
\pgfpathlineto{\pgfqpoint{7.546892in}{0.773588in}}%
\pgfpathlineto{\pgfqpoint{7.591890in}{0.773588in}}%
\pgfpathlineto{\pgfqpoint{7.636592in}{0.773588in}}%
\pgfpathlineto{\pgfqpoint{7.682978in}{0.773588in}}%
\pgfpathlineto{\pgfqpoint{7.728803in}{0.773588in}}%
\pgfpathlineto{\pgfqpoint{7.773226in}{0.773588in}}%
\pgfpathlineto{\pgfqpoint{7.819947in}{0.773588in}}%
\pgfpathlineto{\pgfqpoint{7.865263in}{0.773588in}}%
\pgfpathlineto{\pgfqpoint{7.910161in}{0.773588in}}%
\pgfpathlineto{\pgfqpoint{7.955879in}{0.773588in}}%
\pgfpathlineto{\pgfqpoint{8.000573in}{0.773588in}}%
\pgfpathlineto{\pgfqpoint{8.045278in}{0.773588in}}%
\pgfpathlineto{\pgfqpoint{8.091881in}{0.773588in}}%
\pgfpathlineto{\pgfqpoint{8.136842in}{0.773588in}}%
\pgfpathlineto{\pgfqpoint{8.181791in}{0.773588in}}%
\pgfpathlineto{\pgfqpoint{8.227930in}{0.773588in}}%
\pgfpathlineto{\pgfqpoint{8.273006in}{0.773588in}}%
\pgfpathlineto{\pgfqpoint{8.318789in}{0.773588in}}%
\pgfpathlineto{\pgfqpoint{8.364636in}{0.773588in}}%
\pgfpathlineto{\pgfqpoint{8.409767in}{0.773588in}}%
\pgfpathlineto{\pgfqpoint{8.453780in}{0.773588in}}%
\pgfpathlineto{\pgfqpoint{8.499298in}{0.773588in}}%
\pgfpathlineto{\pgfqpoint{8.543976in}{0.773588in}}%
\pgfpathlineto{\pgfqpoint{8.589124in}{0.773588in}}%
\pgfpathlineto{\pgfqpoint{8.635315in}{0.773588in}}%
\pgfpathlineto{\pgfqpoint{8.681080in}{0.773588in}}%
\pgfpathlineto{\pgfqpoint{8.727404in}{0.773588in}}%
\pgfpathlineto{\pgfqpoint{8.774902in}{0.773588in}}%
\pgfpathlineto{\pgfqpoint{8.821068in}{0.773588in}}%
\pgfpathlineto{\pgfqpoint{8.866242in}{0.773588in}}%
\pgfpathlineto{\pgfqpoint{8.912131in}{0.773588in}}%
\pgfpathlineto{\pgfqpoint{8.957924in}{0.773588in}}%
\pgfpathlineto{\pgfqpoint{9.003947in}{0.773588in}}%
\pgfpathlineto{\pgfqpoint{9.050165in}{0.773588in}}%
\pgfpathlineto{\pgfqpoint{9.095727in}{0.773588in}}%
\pgfpathlineto{\pgfqpoint{9.141270in}{0.773588in}}%
\pgfpathlineto{\pgfqpoint{9.188332in}{0.773588in}}%
\pgfpathlineto{\pgfqpoint{9.233107in}{0.773588in}}%
\pgfpathlineto{\pgfqpoint{9.278439in}{0.773588in}}%
\pgfpathlineto{\pgfqpoint{9.324423in}{0.773588in}}%
\pgfpathlineto{\pgfqpoint{9.369639in}{0.773588in}}%
\pgfpathlineto{\pgfqpoint{9.415178in}{0.773588in}}%
\pgfpathlineto{\pgfqpoint{9.462006in}{0.773588in}}%
\pgfpathlineto{\pgfqpoint{9.507583in}{0.773588in}}%
\pgfpathlineto{\pgfqpoint{9.552197in}{0.773588in}}%
\pgfpathlineto{\pgfqpoint{9.598338in}{0.773588in}}%
\pgfpathlineto{\pgfqpoint{9.644121in}{0.773588in}}%
\pgfpathlineto{\pgfqpoint{9.689497in}{0.773588in}}%
\pgfpathlineto{\pgfqpoint{9.735785in}{0.773588in}}%
\pgfpathlineto{\pgfqpoint{9.781321in}{0.773588in}}%
\pgfpathlineto{\pgfqpoint{9.826599in}{0.773588in}}%
\pgfpathlineto{\pgfqpoint{9.873173in}{0.773588in}}%
\pgfpathlineto{\pgfqpoint{9.918842in}{0.773588in}}%
\pgfpathlineto{\pgfqpoint{9.964562in}{0.773588in}}%
\pgfpathlineto{\pgfqpoint{10.011915in}{0.773588in}}%
\pgfpathlineto{\pgfqpoint{10.057613in}{0.773588in}}%
\pgfpathlineto{\pgfqpoint{10.103098in}{0.773588in}}%
\pgfpathlineto{\pgfqpoint{10.151348in}{0.773588in}}%
\pgfpathlineto{\pgfqpoint{10.197247in}{0.773588in}}%
\pgfpathlineto{\pgfqpoint{10.242601in}{0.773588in}}%
\pgfpathlineto{\pgfqpoint{10.289084in}{0.773588in}}%
\pgfpathlineto{\pgfqpoint{10.335334in}{0.773588in}}%
\pgfpathlineto{\pgfqpoint{10.381570in}{0.773588in}}%
\pgfpathlineto{\pgfqpoint{10.429362in}{0.773588in}}%
\pgfpathlineto{\pgfqpoint{10.475642in}{0.773588in}}%
\pgfpathlineto{\pgfqpoint{10.521105in}{0.773588in}}%
\pgfpathlineto{\pgfqpoint{10.568154in}{0.773588in}}%
\pgfpathlineto{\pgfqpoint{10.613886in}{0.773588in}}%
\pgfpathlineto{\pgfqpoint{10.660112in}{0.773588in}}%
\pgfpathlineto{\pgfqpoint{10.707461in}{0.773588in}}%
\pgfpathlineto{\pgfqpoint{10.753168in}{0.773588in}}%
\pgfpathlineto{\pgfqpoint{10.798768in}{0.773588in}}%
\pgfpathlineto{\pgfqpoint{10.845790in}{0.773588in}}%
\pgfpathlineto{\pgfqpoint{10.891801in}{0.773588in}}%
\pgfpathlineto{\pgfqpoint{10.937381in}{0.773588in}}%
\pgfpathlineto{\pgfqpoint{10.984145in}{0.773588in}}%
\pgfpathlineto{\pgfqpoint{11.029398in}{0.773588in}}%
\pgfpathlineto{\pgfqpoint{11.074633in}{0.773588in}}%
\pgfpathlineto{\pgfqpoint{11.121978in}{0.773588in}}%
\pgfpathlineto{\pgfqpoint{11.167471in}{0.773588in}}%
\pgfpathlineto{\pgfqpoint{11.213355in}{0.773588in}}%
\pgfpathlineto{\pgfqpoint{11.261350in}{0.773588in}}%
\pgfpathlineto{\pgfqpoint{11.307361in}{0.773588in}}%
\pgfpathlineto{\pgfqpoint{11.353756in}{0.773588in}}%
\pgfpathlineto{\pgfqpoint{11.402044in}{0.773588in}}%
\pgfpathlineto{\pgfqpoint{11.448531in}{0.773588in}}%
\pgfpathlineto{\pgfqpoint{11.494874in}{0.773588in}}%
\pgfpathlineto{\pgfqpoint{11.542665in}{0.773588in}}%
\pgfpathlineto{\pgfqpoint{11.588799in}{0.773588in}}%
\pgfpathlineto{\pgfqpoint{11.634589in}{0.773588in}}%
\pgfpathlineto{\pgfqpoint{11.682053in}{0.773588in}}%
\pgfpathlineto{\pgfqpoint{11.727607in}{0.773588in}}%
\pgfpathlineto{\pgfqpoint{11.772894in}{0.773588in}}%
\pgfpathlineto{\pgfqpoint{11.819406in}{0.773588in}}%
\pgfpathlineto{\pgfqpoint{11.864862in}{0.773588in}}%
\pgfpathlineto{\pgfqpoint{11.911501in}{0.773588in}}%
\pgfpathlineto{\pgfqpoint{11.959143in}{0.773588in}}%
\pgfpathlineto{\pgfqpoint{12.004414in}{0.773588in}}%
\pgfpathlineto{\pgfqpoint{12.049869in}{0.773588in}}%
\pgfpathlineto{\pgfqpoint{12.096623in}{0.773588in}}%
\pgfpathlineto{\pgfqpoint{12.141887in}{0.773588in}}%
\pgfpathlineto{\pgfqpoint{12.187925in}{0.773588in}}%
\pgfpathlineto{\pgfqpoint{12.235985in}{0.773588in}}%
\pgfpathlineto{\pgfqpoint{12.282271in}{0.773588in}}%
\pgfpathlineto{\pgfqpoint{12.328136in}{0.773588in}}%
\pgfpathlineto{\pgfqpoint{12.375261in}{0.773588in}}%
\pgfpathlineto{\pgfqpoint{12.421372in}{0.773588in}}%
\pgfpathlineto{\pgfqpoint{12.467453in}{0.773588in}}%
\pgfpathlineto{\pgfqpoint{12.514481in}{0.773588in}}%
\pgfpathlineto{\pgfqpoint{12.560711in}{0.773588in}}%
\pgfpathlineto{\pgfqpoint{12.607615in}{0.773588in}}%
\pgfpathlineto{\pgfqpoint{12.655424in}{0.773588in}}%
\pgfpathlineto{\pgfqpoint{12.701835in}{0.773588in}}%
\pgfpathlineto{\pgfqpoint{12.749025in}{0.773588in}}%
\pgfpathlineto{\pgfqpoint{12.796907in}{0.773588in}}%
\pgfpathlineto{\pgfqpoint{12.843155in}{0.773588in}}%
\pgfpathlineto{\pgfqpoint{12.889988in}{0.773588in}}%
\pgfpathlineto{\pgfqpoint{12.936799in}{0.773588in}}%
\pgfpathlineto{\pgfqpoint{12.983596in}{0.773588in}}%
\pgfpathlineto{\pgfqpoint{13.030499in}{0.773588in}}%
\pgfpathlineto{\pgfqpoint{13.077275in}{0.773588in}}%
\pgfpathlineto{\pgfqpoint{13.123339in}{0.773588in}}%
\pgfpathlineto{\pgfqpoint{13.169075in}{0.773588in}}%
\pgfpathlineto{\pgfqpoint{13.216719in}{0.773588in}}%
\pgfpathlineto{\pgfqpoint{13.262980in}{0.773588in}}%
\pgfpathlineto{\pgfqpoint{13.308974in}{0.773588in}}%
\pgfpathlineto{\pgfqpoint{13.356518in}{0.773588in}}%
\pgfpathlineto{\pgfqpoint{13.401859in}{0.773588in}}%
\pgfpathlineto{\pgfqpoint{13.448069in}{0.773588in}}%
\pgfpathlineto{\pgfqpoint{13.495727in}{0.773588in}}%
\pgfpathlineto{\pgfqpoint{13.541646in}{0.773588in}}%
\pgfpathlineto{\pgfqpoint{13.587249in}{0.773588in}}%
\pgfpathlineto{\pgfqpoint{13.635346in}{0.773588in}}%
\pgfpathlineto{\pgfqpoint{13.682151in}{0.773588in}}%
\pgfpathlineto{\pgfqpoint{13.729143in}{0.773588in}}%
\pgfpathlineto{\pgfqpoint{13.777441in}{0.773588in}}%
\pgfpathlineto{\pgfqpoint{13.825297in}{0.773588in}}%
\pgfpathlineto{\pgfqpoint{13.873610in}{0.773588in}}%
\pgfpathlineto{\pgfqpoint{13.922606in}{0.773588in}}%
\pgfpathlineto{\pgfqpoint{13.969353in}{0.773588in}}%
\pgfpathlineto{\pgfqpoint{14.016344in}{0.773588in}}%
\pgfpathlineto{\pgfqpoint{14.064951in}{0.773588in}}%
\pgfpathlineto{\pgfqpoint{14.112147in}{0.773588in}}%
\pgfpathlineto{\pgfqpoint{14.160147in}{0.773588in}}%
\pgfpathlineto{\pgfqpoint{14.209376in}{0.773588in}}%
\pgfpathlineto{\pgfqpoint{14.255593in}{0.773588in}}%
\pgfpathlineto{\pgfqpoint{14.302418in}{0.773588in}}%
\pgfpathlineto{\pgfqpoint{14.350224in}{0.773588in}}%
\pgfpathlineto{\pgfqpoint{14.397126in}{0.773588in}}%
\pgfpathlineto{\pgfqpoint{14.443729in}{0.773588in}}%
\pgfpathlineto{\pgfqpoint{14.491726in}{0.773588in}}%
\pgfpathlineto{\pgfqpoint{14.538630in}{0.773588in}}%
\pgfpathlineto{\pgfqpoint{14.585477in}{0.773588in}}%
\pgfpathlineto{\pgfqpoint{14.633992in}{0.773588in}}%
\pgfpathlineto{\pgfqpoint{14.680226in}{0.773588in}}%
\pgfpathlineto{\pgfqpoint{14.726757in}{0.773588in}}%
\pgfpathlineto{\pgfqpoint{14.774752in}{0.773588in}}%
\pgfpathlineto{\pgfqpoint{14.821474in}{0.773588in}}%
\pgfpathlineto{\pgfqpoint{14.868242in}{0.773588in}}%
\pgfpathlineto{\pgfqpoint{14.916239in}{0.773588in}}%
\pgfpathlineto{\pgfqpoint{14.963167in}{0.773588in}}%
\pgfpathlineto{\pgfqpoint{15.010214in}{0.773588in}}%
\pgfpathlineto{\pgfqpoint{15.058774in}{0.773588in}}%
\pgfpathlineto{\pgfqpoint{15.105478in}{0.773588in}}%
\pgfpathlineto{\pgfqpoint{15.152658in}{0.773588in}}%
\pgfpathlineto{\pgfqpoint{15.201456in}{0.773588in}}%
\pgfpathlineto{\pgfqpoint{15.248863in}{0.773588in}}%
\pgfpathlineto{\pgfqpoint{15.295866in}{0.773588in}}%
\pgfpathlineto{\pgfqpoint{15.343883in}{0.773588in}}%
\pgfpathlineto{\pgfqpoint{15.389896in}{0.773588in}}%
\pgfpathlineto{\pgfqpoint{15.436052in}{0.773588in}}%
\pgfpathlineto{\pgfqpoint{15.484858in}{0.773588in}}%
\pgfpathlineto{\pgfqpoint{15.532329in}{0.773588in}}%
\pgfpathlineto{\pgfqpoint{15.579815in}{0.773588in}}%
\pgfpathlineto{\pgfqpoint{15.628660in}{0.773588in}}%
\pgfpathlineto{\pgfqpoint{15.676019in}{0.773588in}}%
\pgfpathlineto{\pgfqpoint{15.722814in}{0.773588in}}%
\pgfpathlineto{\pgfqpoint{15.770154in}{0.773588in}}%
\pgfpathlineto{\pgfqpoint{15.817273in}{0.773588in}}%
\pgfpathlineto{\pgfqpoint{15.863592in}{0.773588in}}%
\pgfpathlineto{\pgfqpoint{15.911811in}{0.773588in}}%
\pgfpathlineto{\pgfqpoint{15.957799in}{0.773588in}}%
\pgfpathlineto{\pgfqpoint{16.003926in}{0.773588in}}%
\pgfpathlineto{\pgfqpoint{16.051820in}{0.773588in}}%
\pgfpathlineto{\pgfqpoint{16.098697in}{0.773588in}}%
\pgfpathlineto{\pgfqpoint{16.146338in}{0.773588in}}%
\pgfpathlineto{\pgfqpoint{16.195175in}{0.773588in}}%
\pgfpathlineto{\pgfqpoint{16.242078in}{0.773588in}}%
\pgfpathlineto{\pgfqpoint{16.289817in}{0.773588in}}%
\pgfpathlineto{\pgfqpoint{16.339537in}{0.773588in}}%
\pgfpathlineto{\pgfqpoint{16.387666in}{0.773588in}}%
\pgfpathlineto{\pgfqpoint{16.436622in}{0.773588in}}%
\pgfpathlineto{\pgfqpoint{16.486560in}{0.773588in}}%
\pgfpathlineto{\pgfqpoint{16.533619in}{0.773588in}}%
\pgfpathlineto{\pgfqpoint{16.581135in}{0.773588in}}%
\pgfpathlineto{\pgfqpoint{16.630280in}{0.773588in}}%
\pgfpathlineto{\pgfqpoint{16.677566in}{0.773588in}}%
\pgfpathlineto{\pgfqpoint{16.724568in}{0.773588in}}%
\pgfpathlineto{\pgfqpoint{16.773713in}{0.773588in}}%
\pgfpathlineto{\pgfqpoint{16.821038in}{0.773588in}}%
\pgfpathlineto{\pgfqpoint{16.868266in}{0.773588in}}%
\pgfpathlineto{\pgfqpoint{16.916302in}{0.773588in}}%
\pgfpathlineto{\pgfqpoint{16.963533in}{0.773588in}}%
\pgfpathlineto{\pgfqpoint{17.010585in}{0.773588in}}%
\pgfpathlineto{\pgfqpoint{17.059518in}{0.773588in}}%
\pgfpathlineto{\pgfqpoint{17.107470in}{0.773588in}}%
\pgfpathlineto{\pgfqpoint{17.154507in}{0.773588in}}%
\pgfpathlineto{\pgfqpoint{17.203674in}{0.773588in}}%
\pgfpathlineto{\pgfqpoint{17.250860in}{0.773588in}}%
\pgfpathlineto{\pgfqpoint{17.298941in}{0.773588in}}%
\pgfpathlineto{\pgfqpoint{17.347576in}{0.773588in}}%
\pgfpathlineto{\pgfqpoint{17.394714in}{0.773588in}}%
\pgfpathlineto{\pgfqpoint{17.442362in}{0.773588in}}%
\pgfpathlineto{\pgfqpoint{17.491125in}{0.773588in}}%
\pgfpathlineto{\pgfqpoint{17.538409in}{0.773588in}}%
\pgfpathlineto{\pgfqpoint{17.585742in}{0.773588in}}%
\pgfpathlineto{\pgfqpoint{17.634653in}{0.773588in}}%
\pgfpathlineto{\pgfqpoint{17.681914in}{0.773588in}}%
\pgfpathlineto{\pgfqpoint{17.729727in}{0.773588in}}%
\pgfpathlineto{\pgfqpoint{17.779014in}{0.773588in}}%
\pgfpathlineto{\pgfqpoint{17.826809in}{0.773588in}}%
\pgfpathlineto{\pgfqpoint{17.874600in}{0.773588in}}%
\pgfpathlineto{\pgfqpoint{17.922885in}{0.773588in}}%
\pgfpathlineto{\pgfqpoint{17.970910in}{0.773588in}}%
\pgfpathlineto{\pgfqpoint{18.020026in}{0.773588in}}%
\pgfpathlineto{\pgfqpoint{18.069524in}{0.773588in}}%
\pgfpathlineto{\pgfqpoint{18.117307in}{0.773588in}}%
\pgfpathlineto{\pgfqpoint{18.164405in}{0.773588in}}%
\pgfpathlineto{\pgfqpoint{18.213333in}{0.773588in}}%
\pgfpathlineto{\pgfqpoint{18.260871in}{0.773588in}}%
\pgfpathlineto{\pgfqpoint{18.308441in}{0.773588in}}%
\pgfpathlineto{\pgfqpoint{18.357538in}{0.773588in}}%
\pgfpathlineto{\pgfqpoint{18.405242in}{0.773588in}}%
\pgfpathlineto{\pgfqpoint{18.452284in}{0.773588in}}%
\pgfpathlineto{\pgfqpoint{18.500915in}{0.773588in}}%
\pgfpathlineto{\pgfqpoint{18.548460in}{0.773588in}}%
\pgfpathlineto{\pgfqpoint{18.595724in}{0.773588in}}%
\pgfpathlineto{\pgfqpoint{18.644639in}{0.773588in}}%
\pgfpathlineto{\pgfqpoint{18.693122in}{0.773588in}}%
\pgfpathlineto{\pgfqpoint{18.741665in}{0.773588in}}%
\pgfpathlineto{\pgfqpoint{18.791212in}{0.773588in}}%
\pgfpathlineto{\pgfqpoint{18.839312in}{0.773588in}}%
\pgfpathlineto{\pgfqpoint{18.888188in}{0.773588in}}%
\pgfpathlineto{\pgfqpoint{18.937671in}{0.773588in}}%
\pgfpathlineto{\pgfqpoint{18.985002in}{0.773588in}}%
\pgfpathlineto{\pgfqpoint{19.033794in}{0.773588in}}%
\pgfpathlineto{\pgfqpoint{19.084207in}{0.773588in}}%
\pgfpathlineto{\pgfqpoint{19.132724in}{0.773588in}}%
\pgfpathlineto{\pgfqpoint{19.181302in}{0.773588in}}%
\pgfpathlineto{\pgfqpoint{19.231826in}{0.773588in}}%
\pgfpathlineto{\pgfqpoint{19.279742in}{0.773588in}}%
\pgfpathlineto{\pgfqpoint{19.327829in}{0.773588in}}%
\pgfpathlineto{\pgfqpoint{19.376890in}{0.773588in}}%
\pgfpathlineto{\pgfqpoint{19.425178in}{0.773588in}}%
\pgfpathlineto{\pgfqpoint{19.473154in}{0.773588in}}%
\pgfpathlineto{\pgfqpoint{19.522292in}{0.773588in}}%
\pgfpathlineto{\pgfqpoint{19.569747in}{0.773588in}}%
\pgfpathlineto{\pgfqpoint{19.617707in}{0.773588in}}%
\pgfpathlineto{\pgfqpoint{19.666940in}{0.773588in}}%
\pgfpathlineto{\pgfqpoint{19.714265in}{0.773588in}}%
\pgfpathlineto{\pgfqpoint{19.761678in}{0.773588in}}%
\pgfpathlineto{\pgfqpoint{19.810712in}{0.773588in}}%
\pgfpathlineto{\pgfqpoint{19.858933in}{0.773588in}}%
\pgfpathlineto{\pgfqpoint{19.906615in}{0.773588in}}%
\pgfpathlineto{\pgfqpoint{19.955252in}{0.773588in}}%
\pgfpathlineto{\pgfqpoint{20.003692in}{0.773588in}}%
\pgfpathlineto{\pgfqpoint{20.053312in}{0.773588in}}%
\pgfpathlineto{\pgfqpoint{20.104350in}{0.773588in}}%
\pgfpathlineto{\pgfqpoint{20.153679in}{0.773588in}}%
\pgfpathlineto{\pgfqpoint{20.203615in}{0.773588in}}%
\pgfpathlineto{\pgfqpoint{20.255593in}{0.773588in}}%
\pgfpathlineto{\pgfqpoint{20.305924in}{0.773588in}}%
\pgfpathlineto{\pgfqpoint{20.355462in}{0.773588in}}%
\pgfpathlineto{\pgfqpoint{20.406414in}{0.773588in}}%
\pgfpathlineto{\pgfqpoint{20.455604in}{0.773588in}}%
\pgfpathlineto{\pgfqpoint{20.505208in}{0.773588in}}%
\pgfpathlineto{\pgfqpoint{20.556290in}{0.773588in}}%
\pgfpathlineto{\pgfqpoint{20.605567in}{0.773588in}}%
\pgfpathlineto{\pgfqpoint{20.654668in}{0.773588in}}%
\pgfpathlineto{\pgfqpoint{20.704616in}{0.773588in}}%
\pgfpathlineto{\pgfqpoint{20.753558in}{0.773588in}}%
\pgfpathlineto{\pgfqpoint{20.802965in}{0.773588in}}%
\pgfpathlineto{\pgfqpoint{20.854118in}{0.773588in}}%
\pgfpathlineto{\pgfqpoint{20.904164in}{0.773588in}}%
\pgfpathlineto{\pgfqpoint{20.953585in}{0.773588in}}%
\pgfpathlineto{\pgfqpoint{21.003693in}{0.773588in}}%
\pgfpathlineto{\pgfqpoint{21.053515in}{0.773588in}}%
\pgfpathlineto{\pgfqpoint{21.102370in}{0.773588in}}%
\pgfpathlineto{\pgfqpoint{21.152719in}{0.773588in}}%
\pgfpathlineto{\pgfqpoint{21.201683in}{0.773588in}}%
\pgfpathlineto{\pgfqpoint{21.250655in}{0.773588in}}%
\pgfpathlineto{\pgfqpoint{21.300955in}{0.773588in}}%
\pgfpathlineto{\pgfqpoint{21.350353in}{0.773588in}}%
\pgfpathlineto{\pgfqpoint{21.399733in}{0.773588in}}%
\pgfpathlineto{\pgfqpoint{21.450830in}{0.773588in}}%
\pgfpathlineto{\pgfqpoint{21.501023in}{0.773588in}}%
\pgfpathlineto{\pgfqpoint{21.550957in}{0.773588in}}%
\pgfpathlineto{\pgfqpoint{21.602538in}{0.773588in}}%
\pgfpathlineto{\pgfqpoint{21.652276in}{0.773588in}}%
\pgfpathlineto{\pgfqpoint{21.700864in}{0.773588in}}%
\pgfpathlineto{\pgfqpoint{21.751624in}{0.773588in}}%
\pgfpathlineto{\pgfqpoint{21.800474in}{0.773588in}}%
\pgfpathlineto{\pgfqpoint{21.849580in}{0.773588in}}%
\pgfpathlineto{\pgfqpoint{21.900834in}{0.773588in}}%
\pgfpathlineto{\pgfqpoint{21.950544in}{0.773588in}}%
\pgfpathlineto{\pgfqpoint{21.999803in}{0.773588in}}%
\pgfpathlineto{\pgfqpoint{22.050354in}{0.773588in}}%
\pgfpathlineto{\pgfqpoint{22.099907in}{0.773588in}}%
\pgfpathlineto{\pgfqpoint{22.148817in}{0.773588in}}%
\pgfpathlineto{\pgfqpoint{22.200587in}{0.773588in}}%
\pgfpathlineto{\pgfqpoint{22.250634in}{0.773588in}}%
\pgfpathlineto{\pgfqpoint{22.300933in}{0.773588in}}%
\pgfpathlineto{\pgfqpoint{22.352827in}{0.773588in}}%
\pgfpathlineto{\pgfqpoint{22.402084in}{0.773588in}}%
\pgfpathlineto{\pgfqpoint{22.451976in}{0.773588in}}%
\pgfpathlineto{\pgfqpoint{22.503552in}{0.773588in}}%
\pgfpathlineto{\pgfqpoint{22.554073in}{0.773588in}}%
\pgfpathlineto{\pgfqpoint{22.603657in}{0.773588in}}%
\pgfpathlineto{\pgfqpoint{22.654822in}{0.773588in}}%
\pgfpathlineto{\pgfqpoint{22.703398in}{0.773588in}}%
\pgfpathlineto{\pgfqpoint{22.751700in}{0.773588in}}%
\pgfpathlineto{\pgfqpoint{22.803829in}{0.773588in}}%
\pgfpathlineto{\pgfqpoint{22.853966in}{0.773588in}}%
\pgfpathlineto{\pgfqpoint{22.904236in}{0.773588in}}%
\pgfpathlineto{\pgfqpoint{22.956359in}{0.773588in}}%
\pgfpathlineto{\pgfqpoint{23.006022in}{0.773588in}}%
\pgfpathlineto{\pgfqpoint{23.055187in}{0.773588in}}%
\pgfpathlineto{\pgfqpoint{23.107134in}{0.773588in}}%
\pgfpathlineto{\pgfqpoint{23.157756in}{0.773588in}}%
\pgfpathlineto{\pgfqpoint{23.208694in}{0.773588in}}%
\pgfpathlineto{\pgfqpoint{23.260726in}{0.773588in}}%
\pgfpathlineto{\pgfqpoint{23.310918in}{0.773588in}}%
\pgfpathlineto{\pgfqpoint{23.361310in}{0.773588in}}%
\pgfpathlineto{\pgfqpoint{23.411909in}{0.773588in}}%
\pgfpathlineto{\pgfqpoint{23.462017in}{0.773588in}}%
\pgfpathlineto{\pgfqpoint{23.511867in}{0.773588in}}%
\pgfpathlineto{\pgfqpoint{23.563429in}{0.773588in}}%
\pgfpathlineto{\pgfqpoint{23.611923in}{0.773588in}}%
\pgfpathlineto{\pgfqpoint{23.661575in}{0.773588in}}%
\pgfpathlineto{\pgfqpoint{23.713074in}{0.773588in}}%
\pgfpathlineto{\pgfqpoint{23.762329in}{0.773588in}}%
\pgfpathlineto{\pgfqpoint{23.811512in}{0.773588in}}%
\pgfpathlineto{\pgfqpoint{23.862673in}{0.773588in}}%
\pgfpathlineto{\pgfqpoint{23.913088in}{0.773588in}}%
\pgfpathlineto{\pgfqpoint{23.964514in}{0.773588in}}%
\pgfpathlineto{\pgfqpoint{24.016442in}{0.773588in}}%
\pgfpathlineto{\pgfqpoint{24.066933in}{0.773588in}}%
\pgfpathlineto{\pgfqpoint{24.118637in}{0.773588in}}%
\pgfpathlineto{\pgfqpoint{24.170873in}{0.773588in}}%
\pgfpathlineto{\pgfqpoint{24.221010in}{0.773588in}}%
\pgfpathlineto{\pgfqpoint{24.271277in}{0.773588in}}%
\pgfpathlineto{\pgfqpoint{24.321907in}{0.773588in}}%
\pgfpathlineto{\pgfqpoint{24.371890in}{0.773588in}}%
\pgfpathlineto{\pgfqpoint{24.422475in}{0.773588in}}%
\pgfpathlineto{\pgfqpoint{24.474381in}{0.773588in}}%
\pgfpathlineto{\pgfqpoint{24.524258in}{0.773588in}}%
\pgfpathlineto{\pgfqpoint{24.573998in}{0.773588in}}%
\pgfpathlineto{\pgfqpoint{24.626556in}{0.773588in}}%
\pgfpathlineto{\pgfqpoint{24.676389in}{0.773588in}}%
\pgfpathlineto{\pgfqpoint{24.726499in}{0.773588in}}%
\pgfpathlineto{\pgfqpoint{24.778304in}{0.773588in}}%
\pgfpathlineto{\pgfqpoint{24.828140in}{0.773588in}}%
\pgfpathlineto{\pgfqpoint{24.878143in}{0.773588in}}%
\pgfpathlineto{\pgfqpoint{24.928869in}{0.773588in}}%
\pgfpathlineto{\pgfqpoint{24.978211in}{0.773588in}}%
\pgfpathlineto{\pgfqpoint{25.028630in}{0.773588in}}%
\pgfpathlineto{\pgfqpoint{25.080539in}{0.773588in}}%
\pgfpathlineto{\pgfqpoint{25.131043in}{0.773588in}}%
\pgfpathlineto{\pgfqpoint{25.181554in}{0.773588in}}%
\pgfpathlineto{\pgfqpoint{25.232944in}{0.773588in}}%
\pgfpathlineto{\pgfqpoint{25.282222in}{0.773588in}}%
\pgfpathlineto{\pgfqpoint{25.332608in}{0.773588in}}%
\pgfpathlineto{\pgfqpoint{25.383622in}{0.773588in}}%
\pgfpathlineto{\pgfqpoint{25.433954in}{0.773588in}}%
\pgfpathlineto{\pgfqpoint{25.483381in}{0.773588in}}%
\pgfpathlineto{\pgfqpoint{25.535558in}{0.773588in}}%
\pgfpathlineto{\pgfqpoint{25.586387in}{0.773588in}}%
\pgfpathlineto{\pgfqpoint{25.637513in}{0.773588in}}%
\pgfpathlineto{\pgfqpoint{25.689152in}{0.773588in}}%
\pgfpathlineto{\pgfqpoint{25.740073in}{0.773588in}}%
\pgfpathlineto{\pgfqpoint{25.790287in}{0.773588in}}%
\pgfpathlineto{\pgfqpoint{25.841969in}{0.773588in}}%
\pgfpathlineto{\pgfqpoint{25.891991in}{0.773588in}}%
\pgfpathlineto{\pgfqpoint{25.942005in}{0.773588in}}%
\pgfpathlineto{\pgfqpoint{25.994130in}{0.773588in}}%
\pgfpathlineto{\pgfqpoint{26.044216in}{0.773588in}}%
\pgfpathlineto{\pgfqpoint{26.093895in}{0.773588in}}%
\pgfpathlineto{\pgfqpoint{26.144837in}{0.773588in}}%
\pgfpathlineto{\pgfqpoint{26.195263in}{0.773588in}}%
\pgfpathlineto{\pgfqpoint{26.245717in}{0.773588in}}%
\pgfpathlineto{\pgfqpoint{26.297518in}{0.773588in}}%
\pgfpathlineto{\pgfqpoint{26.346588in}{0.773588in}}%
\pgfpathlineto{\pgfqpoint{26.395829in}{0.773588in}}%
\pgfpathlineto{\pgfqpoint{26.447655in}{0.773588in}}%
\pgfpathlineto{\pgfqpoint{26.499064in}{0.773588in}}%
\pgfpathlineto{\pgfqpoint{26.549446in}{0.773588in}}%
\pgfpathlineto{\pgfqpoint{26.601868in}{0.773588in}}%
\pgfpathlineto{\pgfqpoint{26.653161in}{0.773588in}}%
\pgfpathlineto{\pgfqpoint{26.704176in}{0.773588in}}%
\pgfpathlineto{\pgfqpoint{26.756448in}{0.773588in}}%
\pgfpathlineto{\pgfqpoint{26.807173in}{0.773588in}}%
\pgfpathlineto{\pgfqpoint{26.857706in}{0.773588in}}%
\pgfpathlineto{\pgfqpoint{26.909530in}{0.773588in}}%
\pgfpathlineto{\pgfqpoint{26.960748in}{0.773588in}}%
\pgfpathlineto{\pgfqpoint{27.010539in}{0.773588in}}%
\pgfpathlineto{\pgfqpoint{27.063062in}{0.773588in}}%
\pgfpathlineto{\pgfqpoint{27.113950in}{0.773588in}}%
\pgfpathlineto{\pgfqpoint{27.165274in}{0.773588in}}%
\pgfpathlineto{\pgfqpoint{27.218587in}{0.773588in}}%
\pgfpathlineto{\pgfqpoint{27.269679in}{0.773588in}}%
\pgfpathlineto{\pgfqpoint{27.321160in}{0.773588in}}%
\pgfpathlineto{\pgfqpoint{27.372863in}{0.773588in}}%
\pgfpathlineto{\pgfqpoint{27.423066in}{0.773588in}}%
\pgfpathlineto{\pgfqpoint{27.473173in}{0.773588in}}%
\pgfpathlineto{\pgfqpoint{27.525849in}{0.773588in}}%
\pgfpathlineto{\pgfqpoint{27.576178in}{0.773588in}}%
\pgfpathlineto{\pgfqpoint{27.626352in}{0.773588in}}%
\pgfpathlineto{\pgfqpoint{27.678294in}{0.773588in}}%
\pgfpathlineto{\pgfqpoint{27.728571in}{0.773588in}}%
\pgfpathlineto{\pgfqpoint{27.779331in}{0.773588in}}%
\pgfpathlineto{\pgfqpoint{27.833067in}{0.773588in}}%
\pgfpathlineto{\pgfqpoint{27.883786in}{0.773588in}}%
\pgfpathlineto{\pgfqpoint{27.935127in}{0.773588in}}%
\pgfpathlineto{\pgfqpoint{27.988562in}{0.773588in}}%
\pgfpathlineto{\pgfqpoint{28.039571in}{0.773588in}}%
\pgfpathlineto{\pgfqpoint{28.090650in}{0.773588in}}%
\pgfpathlineto{\pgfqpoint{28.143944in}{0.773588in}}%
\pgfpathlineto{\pgfqpoint{28.195034in}{0.773588in}}%
\pgfpathlineto{\pgfqpoint{28.245320in}{0.773588in}}%
\pgfpathlineto{\pgfqpoint{28.297462in}{0.773588in}}%
\pgfpathlineto{\pgfqpoint{28.347853in}{0.773588in}}%
\pgfpathlineto{\pgfqpoint{28.399138in}{0.773588in}}%
\pgfpathlineto{\pgfqpoint{28.451952in}{0.773588in}}%
\pgfpathlineto{\pgfqpoint{28.502707in}{0.773588in}}%
\pgfpathlineto{\pgfqpoint{28.553749in}{0.773588in}}%
\pgfpathlineto{\pgfqpoint{28.605351in}{0.773588in}}%
\pgfpathlineto{\pgfqpoint{28.656185in}{0.773588in}}%
\pgfpathlineto{\pgfqpoint{28.706300in}{0.773588in}}%
\pgfpathlineto{\pgfqpoint{28.758074in}{0.773588in}}%
\pgfpathlineto{\pgfqpoint{28.808587in}{0.773588in}}%
\pgfpathlineto{\pgfqpoint{28.860347in}{0.773588in}}%
\pgfpathlineto{\pgfqpoint{28.913538in}{0.773588in}}%
\pgfpathlineto{\pgfqpoint{28.965063in}{0.773588in}}%
\pgfpathlineto{\pgfqpoint{29.015651in}{0.773588in}}%
\pgfpathlineto{\pgfqpoint{29.068863in}{0.773588in}}%
\pgfpathlineto{\pgfqpoint{29.121029in}{0.773588in}}%
\pgfpathlineto{\pgfqpoint{29.173700in}{0.773588in}}%
\pgfpathlineto{\pgfqpoint{29.226818in}{0.773588in}}%
\pgfpathlineto{\pgfqpoint{29.279399in}{0.773588in}}%
\pgfpathlineto{\pgfqpoint{29.332261in}{0.773588in}}%
\pgfpathlineto{\pgfqpoint{29.386173in}{0.773588in}}%
\pgfpathlineto{\pgfqpoint{29.438335in}{0.773588in}}%
\pgfpathlineto{\pgfqpoint{29.490075in}{0.773588in}}%
\pgfpathlineto{\pgfqpoint{29.543760in}{0.773588in}}%
\pgfpathlineto{\pgfqpoint{29.596078in}{0.773588in}}%
\pgfpathlineto{\pgfqpoint{29.647988in}{0.773588in}}%
\pgfpathlineto{\pgfqpoint{29.701409in}{0.773588in}}%
\pgfpathlineto{\pgfqpoint{29.753093in}{0.773588in}}%
\pgfpathlineto{\pgfqpoint{29.805208in}{0.773588in}}%
\pgfpathlineto{\pgfqpoint{29.858913in}{0.773588in}}%
\pgfpathlineto{\pgfqpoint{29.910558in}{0.773588in}}%
\pgfpathlineto{\pgfqpoint{29.962066in}{0.773588in}}%
\pgfpathlineto{\pgfqpoint{30.014968in}{0.773588in}}%
\pgfpathlineto{\pgfqpoint{30.066247in}{0.773588in}}%
\pgfpathlineto{\pgfqpoint{30.117082in}{0.773588in}}%
\pgfpathlineto{\pgfqpoint{30.169497in}{0.773588in}}%
\pgfpathlineto{\pgfqpoint{30.221783in}{0.773588in}}%
\pgfpathlineto{\pgfqpoint{30.276704in}{0.773588in}}%
\pgfpathlineto{\pgfqpoint{30.337041in}{0.773588in}}%
\pgfpathlineto{\pgfqpoint{30.397563in}{0.773588in}}%
\pgfpathlineto{\pgfqpoint{30.457369in}{0.773588in}}%
\pgfpathlineto{\pgfqpoint{30.519963in}{0.773588in}}%
\pgfpathlineto{\pgfqpoint{30.582999in}{0.773588in}}%
\pgfpathlineto{\pgfqpoint{30.648377in}{0.773588in}}%
\pgfpathlineto{\pgfqpoint{30.717993in}{0.773588in}}%
\pgfpathlineto{\pgfqpoint{30.786068in}{0.773588in}}%
\pgfpathlineto{\pgfqpoint{30.855684in}{0.773588in}}%
\pgfpathlineto{\pgfqpoint{30.928370in}{0.773588in}}%
\pgfpathlineto{\pgfqpoint{31.000738in}{0.773588in}}%
\pgfpathlineto{\pgfqpoint{31.073353in}{0.773588in}}%
\pgfpathlineto{\pgfqpoint{31.150292in}{0.773588in}}%
\pgfpathlineto{\pgfqpoint{31.225476in}{0.773588in}}%
\pgfpathlineto{\pgfqpoint{31.304546in}{0.773588in}}%
\pgfpathlineto{\pgfqpoint{31.385546in}{0.773588in}}%
\pgfpathlineto{\pgfqpoint{31.463397in}{0.773588in}}%
\pgfpathlineto{\pgfqpoint{31.543273in}{0.773588in}}%
\pgfpathlineto{\pgfqpoint{31.629527in}{0.773588in}}%
\pgfpathlineto{\pgfqpoint{31.715049in}{0.773588in}}%
\pgfpathlineto{\pgfqpoint{31.802724in}{0.773588in}}%
\pgfpathlineto{\pgfqpoint{31.890397in}{0.773588in}}%
\pgfpathlineto{\pgfqpoint{31.975639in}{0.773588in}}%
\pgfpathlineto{\pgfqpoint{32.062211in}{0.773588in}}%
\pgfpathlineto{\pgfqpoint{32.153206in}{0.773588in}}%
\pgfpathlineto{\pgfqpoint{32.244846in}{0.773588in}}%
\pgfpathlineto{\pgfqpoint{32.334387in}{0.773588in}}%
\pgfpathlineto{\pgfqpoint{32.429213in}{0.773588in}}%
\pgfpathlineto{\pgfqpoint{32.518604in}{0.773588in}}%
\pgfpathlineto{\pgfqpoint{32.584565in}{0.773588in}}%
\pgfpathlineto{\pgfqpoint{32.638735in}{0.773588in}}%
\pgfpathlineto{\pgfqpoint{32.691213in}{0.773588in}}%
\pgfpathlineto{\pgfqpoint{32.743153in}{0.773588in}}%
\pgfpathlineto{\pgfqpoint{32.797230in}{0.773588in}}%
\pgfpathlineto{\pgfqpoint{32.850576in}{0.773588in}}%
\pgfpathlineto{\pgfqpoint{32.903667in}{0.773588in}}%
\pgfpathlineto{\pgfqpoint{32.957868in}{0.773588in}}%
\pgfpathlineto{\pgfqpoint{33.010153in}{0.773588in}}%
\pgfpathlineto{\pgfqpoint{33.062733in}{0.773588in}}%
\pgfpathlineto{\pgfqpoint{33.115518in}{0.773588in}}%
\pgfpathlineto{\pgfqpoint{33.167699in}{0.773588in}}%
\pgfpathlineto{\pgfqpoint{33.219830in}{0.773588in}}%
\pgfpathlineto{\pgfqpoint{33.273785in}{0.773588in}}%
\pgfpathlineto{\pgfqpoint{33.326652in}{0.773588in}}%
\pgfpathlineto{\pgfqpoint{33.379482in}{0.773588in}}%
\pgfpathlineto{\pgfqpoint{33.433283in}{0.773588in}}%
\pgfpathlineto{\pgfqpoint{33.484775in}{0.773588in}}%
\pgfpathlineto{\pgfqpoint{33.536879in}{0.773588in}}%
\pgfpathlineto{\pgfqpoint{33.590505in}{0.773588in}}%
\pgfpathlineto{\pgfqpoint{33.642373in}{0.773588in}}%
\pgfpathlineto{\pgfqpoint{33.693680in}{0.773588in}}%
\pgfpathlineto{\pgfqpoint{33.746200in}{0.773588in}}%
\pgfpathlineto{\pgfqpoint{33.784450in}{0.773588in}}%
\pgfpathlineto{\pgfqpoint{33.832264in}{0.773588in}}%
\pgfpathlineto{\pgfqpoint{33.870742in}{1.246842in}}%
\pgfpathlineto{\pgfqpoint{33.912991in}{1.377266in}}%
\pgfpathlineto{\pgfqpoint{33.952831in}{1.533299in}}%
\pgfpathlineto{\pgfqpoint{33.989641in}{1.768315in}}%
\pgfpathlineto{\pgfqpoint{34.021637in}{2.208863in}}%
\pgfpathlineto{\pgfqpoint{34.052161in}{2.790032in}}%
\pgfpathlineto{\pgfqpoint{34.076513in}{3.753732in}}%
\pgfpathlineto{\pgfqpoint{34.101659in}{3.612521in}}%
\pgfpathlineto{\pgfqpoint{34.125436in}{3.898067in}}%
\pgfpathlineto{\pgfqpoint{34.150180in}{3.941303in}}%
\pgfpathlineto{\pgfqpoint{34.173862in}{3.942672in}}%
\pgfpathlineto{\pgfqpoint{34.197440in}{3.869449in}}%
\pgfpathlineto{\pgfqpoint{34.222832in}{3.988507in}}%
\pgfpathlineto{\pgfqpoint{34.246364in}{4.040966in}}%
\pgfpathlineto{\pgfqpoint{34.270997in}{3.877862in}}%
\pgfpathlineto{\pgfqpoint{34.293786in}{3.930055in}}%
\pgfpathlineto{\pgfqpoint{34.318035in}{4.075356in}}%
\pgfpathlineto{\pgfqpoint{34.340984in}{3.972416in}}%
\pgfpathlineto{\pgfqpoint{34.366319in}{3.888388in}}%
\pgfpathlineto{\pgfqpoint{34.388883in}{3.916080in}}%
\pgfpathlineto{\pgfqpoint{34.412956in}{3.769623in}}%
\pgfpathlineto{\pgfqpoint{34.437757in}{3.898493in}}%
\pgfpathlineto{\pgfqpoint{34.460943in}{4.217964in}}%
\pgfpathlineto{\pgfqpoint{34.484122in}{4.134872in}}%
\pgfpathlineto{\pgfqpoint{34.508593in}{3.944352in}}%
\pgfpathlineto{\pgfqpoint{34.531364in}{4.171934in}}%
\pgfpathlineto{\pgfqpoint{34.554213in}{4.127126in}}%
\pgfpathlineto{\pgfqpoint{34.578241in}{4.194563in}}%
\pgfpathlineto{\pgfqpoint{34.601042in}{4.217950in}}%
\pgfpathlineto{\pgfqpoint{34.623987in}{4.124102in}}%
\pgfpathlineto{\pgfqpoint{34.648006in}{4.071453in}}%
\pgfpathlineto{\pgfqpoint{34.671232in}{3.997581in}}%
\pgfpathlineto{\pgfqpoint{34.693969in}{4.254313in}}%
\pgfpathlineto{\pgfqpoint{34.717979in}{4.168563in}}%
\pgfpathlineto{\pgfqpoint{34.741399in}{4.195680in}}%
\pgfpathlineto{\pgfqpoint{34.763884in}{4.251407in}}%
\pgfpathlineto{\pgfqpoint{34.788312in}{4.142544in}}%
\pgfpathlineto{\pgfqpoint{34.810332in}{4.192246in}}%
\pgfpathlineto{\pgfqpoint{34.834146in}{4.201329in}}%
\pgfpathlineto{\pgfqpoint{34.856576in}{4.232916in}}%
\pgfpathlineto{\pgfqpoint{34.880395in}{4.201901in}}%
\pgfpathlineto{\pgfqpoint{34.902821in}{4.198446in}}%
\pgfpathlineto{\pgfqpoint{34.926805in}{4.231403in}}%
\pgfpathlineto{\pgfqpoint{34.948980in}{4.239053in}}%
\pgfpathlineto{\pgfqpoint{34.973284in}{3.989480in}}%
\pgfpathlineto{\pgfqpoint{34.996094in}{4.101072in}}%
\pgfpathlineto{\pgfqpoint{35.020184in}{4.182959in}}%
\pgfpathlineto{\pgfqpoint{35.047391in}{4.038936in}}%
\pgfpathlineto{\pgfqpoint{35.098205in}{3.993842in}}%
\pgfpathlineto{\pgfqpoint{35.149293in}{3.993842in}}%
\pgfpathlineto{\pgfqpoint{35.201170in}{3.993842in}}%
\pgfpathlineto{\pgfqpoint{35.254065in}{3.993842in}}%
\pgfpathlineto{\pgfqpoint{35.305776in}{3.993842in}}%
\pgfpathlineto{\pgfqpoint{35.357710in}{3.993842in}}%
\pgfpathlineto{\pgfqpoint{35.411910in}{3.993842in}}%
\pgfpathlineto{\pgfqpoint{35.464942in}{3.993842in}}%
\pgfpathlineto{\pgfqpoint{35.464942in}{5.758501in}}%
\pgfpathlineto{\pgfqpoint{35.464942in}{5.758501in}}%
\pgfpathlineto{\pgfqpoint{35.411910in}{5.758501in}}%
\pgfpathlineto{\pgfqpoint{35.357710in}{5.758501in}}%
\pgfpathlineto{\pgfqpoint{35.305776in}{5.758501in}}%
\pgfpathlineto{\pgfqpoint{35.254065in}{5.758501in}}%
\pgfpathlineto{\pgfqpoint{35.201170in}{5.758501in}}%
\pgfpathlineto{\pgfqpoint{35.149293in}{5.758501in}}%
\pgfpathlineto{\pgfqpoint{35.098205in}{5.758501in}}%
\pgfpathlineto{\pgfqpoint{35.047391in}{5.775449in}}%
\pgfpathlineto{\pgfqpoint{35.020184in}{5.865058in}}%
\pgfpathlineto{\pgfqpoint{34.996094in}{5.825752in}}%
\pgfpathlineto{\pgfqpoint{34.973284in}{5.726189in}}%
\pgfpathlineto{\pgfqpoint{34.948980in}{5.930845in}}%
\pgfpathlineto{\pgfqpoint{34.926805in}{5.899825in}}%
\pgfpathlineto{\pgfqpoint{34.902821in}{5.880249in}}%
\pgfpathlineto{\pgfqpoint{34.880395in}{5.893046in}}%
\pgfpathlineto{\pgfqpoint{34.856576in}{5.911984in}}%
\pgfpathlineto{\pgfqpoint{34.834146in}{5.857730in}}%
\pgfpathlineto{\pgfqpoint{34.810332in}{5.868196in}}%
\pgfpathlineto{\pgfqpoint{34.788312in}{5.815038in}}%
\pgfpathlineto{\pgfqpoint{34.763884in}{5.918658in}}%
\pgfpathlineto{\pgfqpoint{34.741399in}{5.854178in}}%
\pgfpathlineto{\pgfqpoint{34.717979in}{5.813609in}}%
\pgfpathlineto{\pgfqpoint{34.693969in}{5.908282in}}%
\pgfpathlineto{\pgfqpoint{34.671232in}{5.706485in}}%
\pgfpathlineto{\pgfqpoint{34.648006in}{5.744292in}}%
\pgfpathlineto{\pgfqpoint{34.623987in}{5.791091in}}%
\pgfpathlineto{\pgfqpoint{34.601042in}{5.862622in}}%
\pgfpathlineto{\pgfqpoint{34.578241in}{5.846668in}}%
\pgfpathlineto{\pgfqpoint{34.554213in}{5.745325in}}%
\pgfpathlineto{\pgfqpoint{34.531364in}{5.809148in}}%
\pgfpathlineto{\pgfqpoint{34.508593in}{5.601241in}}%
\pgfpathlineto{\pgfqpoint{34.484122in}{5.740043in}}%
\pgfpathlineto{\pgfqpoint{34.460943in}{5.834541in}}%
\pgfpathlineto{\pgfqpoint{34.437757in}{5.582839in}}%
\pgfpathlineto{\pgfqpoint{34.412956in}{5.442515in}}%
\pgfpathlineto{\pgfqpoint{34.388883in}{5.557546in}}%
\pgfpathlineto{\pgfqpoint{34.366319in}{5.504363in}}%
\pgfpathlineto{\pgfqpoint{34.340984in}{5.584654in}}%
\pgfpathlineto{\pgfqpoint{34.318035in}{5.678338in}}%
\pgfpathlineto{\pgfqpoint{34.293786in}{5.518952in}}%
\pgfpathlineto{\pgfqpoint{34.270997in}{5.460686in}}%
\pgfpathlineto{\pgfqpoint{34.246364in}{5.534473in}}%
\pgfpathlineto{\pgfqpoint{34.222832in}{5.484292in}}%
\pgfpathlineto{\pgfqpoint{34.197440in}{5.442565in}}%
\pgfpathlineto{\pgfqpoint{34.173862in}{5.446238in}}%
\pgfpathlineto{\pgfqpoint{34.150180in}{5.483715in}}%
\pgfpathlineto{\pgfqpoint{34.125436in}{5.364631in}}%
\pgfpathlineto{\pgfqpoint{34.101659in}{5.182737in}}%
\pgfpathlineto{\pgfqpoint{34.076513in}{5.235025in}}%
\pgfpathlineto{\pgfqpoint{34.052161in}{3.729137in}}%
\pgfpathlineto{\pgfqpoint{34.021637in}{2.864879in}}%
\pgfpathlineto{\pgfqpoint{33.989641in}{2.224563in}}%
\pgfpathlineto{\pgfqpoint{33.952831in}{1.906942in}}%
\pgfpathlineto{\pgfqpoint{33.912991in}{1.669618in}}%
\pgfpathlineto{\pgfqpoint{33.870742in}{1.454760in}}%
\pgfpathlineto{\pgfqpoint{33.832264in}{0.773588in}}%
\pgfpathlineto{\pgfqpoint{33.784450in}{0.773588in}}%
\pgfpathlineto{\pgfqpoint{33.746200in}{0.773588in}}%
\pgfpathlineto{\pgfqpoint{33.693680in}{0.773588in}}%
\pgfpathlineto{\pgfqpoint{33.642373in}{0.773588in}}%
\pgfpathlineto{\pgfqpoint{33.590505in}{0.773588in}}%
\pgfpathlineto{\pgfqpoint{33.536879in}{0.773588in}}%
\pgfpathlineto{\pgfqpoint{33.484775in}{0.773588in}}%
\pgfpathlineto{\pgfqpoint{33.433283in}{0.773588in}}%
\pgfpathlineto{\pgfqpoint{33.379482in}{0.773588in}}%
\pgfpathlineto{\pgfqpoint{33.326652in}{0.773588in}}%
\pgfpathlineto{\pgfqpoint{33.273785in}{0.773588in}}%
\pgfpathlineto{\pgfqpoint{33.219830in}{0.773588in}}%
\pgfpathlineto{\pgfqpoint{33.167699in}{0.773588in}}%
\pgfpathlineto{\pgfqpoint{33.115518in}{0.773588in}}%
\pgfpathlineto{\pgfqpoint{33.062733in}{0.773588in}}%
\pgfpathlineto{\pgfqpoint{33.010153in}{0.773588in}}%
\pgfpathlineto{\pgfqpoint{32.957868in}{0.773588in}}%
\pgfpathlineto{\pgfqpoint{32.903667in}{0.773588in}}%
\pgfpathlineto{\pgfqpoint{32.850576in}{0.773588in}}%
\pgfpathlineto{\pgfqpoint{32.797230in}{0.773588in}}%
\pgfpathlineto{\pgfqpoint{32.743153in}{0.773588in}}%
\pgfpathlineto{\pgfqpoint{32.691213in}{0.773588in}}%
\pgfpathlineto{\pgfqpoint{32.638735in}{0.773588in}}%
\pgfpathlineto{\pgfqpoint{32.584565in}{0.773588in}}%
\pgfpathlineto{\pgfqpoint{32.518604in}{0.773588in}}%
\pgfpathlineto{\pgfqpoint{32.429213in}{0.773588in}}%
\pgfpathlineto{\pgfqpoint{32.334387in}{0.773588in}}%
\pgfpathlineto{\pgfqpoint{32.244846in}{0.773588in}}%
\pgfpathlineto{\pgfqpoint{32.153206in}{0.773588in}}%
\pgfpathlineto{\pgfqpoint{32.062211in}{0.773588in}}%
\pgfpathlineto{\pgfqpoint{31.975639in}{0.773588in}}%
\pgfpathlineto{\pgfqpoint{31.890397in}{0.773588in}}%
\pgfpathlineto{\pgfqpoint{31.802724in}{0.773588in}}%
\pgfpathlineto{\pgfqpoint{31.715049in}{0.773588in}}%
\pgfpathlineto{\pgfqpoint{31.629527in}{0.773588in}}%
\pgfpathlineto{\pgfqpoint{31.543273in}{0.773588in}}%
\pgfpathlineto{\pgfqpoint{31.463397in}{0.773588in}}%
\pgfpathlineto{\pgfqpoint{31.385546in}{0.773588in}}%
\pgfpathlineto{\pgfqpoint{31.304546in}{0.773588in}}%
\pgfpathlineto{\pgfqpoint{31.225476in}{0.773588in}}%
\pgfpathlineto{\pgfqpoint{31.150292in}{0.773588in}}%
\pgfpathlineto{\pgfqpoint{31.073353in}{0.773588in}}%
\pgfpathlineto{\pgfqpoint{31.000738in}{0.773588in}}%
\pgfpathlineto{\pgfqpoint{30.928370in}{0.773588in}}%
\pgfpathlineto{\pgfqpoint{30.855684in}{0.773588in}}%
\pgfpathlineto{\pgfqpoint{30.786068in}{0.773588in}}%
\pgfpathlineto{\pgfqpoint{30.717993in}{0.773588in}}%
\pgfpathlineto{\pgfqpoint{30.648377in}{0.773588in}}%
\pgfpathlineto{\pgfqpoint{30.582999in}{0.773588in}}%
\pgfpathlineto{\pgfqpoint{30.519963in}{0.773588in}}%
\pgfpathlineto{\pgfqpoint{30.457369in}{0.773588in}}%
\pgfpathlineto{\pgfqpoint{30.397563in}{0.773588in}}%
\pgfpathlineto{\pgfqpoint{30.337041in}{0.773588in}}%
\pgfpathlineto{\pgfqpoint{30.276704in}{0.773588in}}%
\pgfpathlineto{\pgfqpoint{30.221783in}{0.773588in}}%
\pgfpathlineto{\pgfqpoint{30.169497in}{0.773588in}}%
\pgfpathlineto{\pgfqpoint{30.117082in}{0.773588in}}%
\pgfpathlineto{\pgfqpoint{30.066247in}{0.773588in}}%
\pgfpathlineto{\pgfqpoint{30.014968in}{0.773588in}}%
\pgfpathlineto{\pgfqpoint{29.962066in}{0.773588in}}%
\pgfpathlineto{\pgfqpoint{29.910558in}{0.773588in}}%
\pgfpathlineto{\pgfqpoint{29.858913in}{0.773588in}}%
\pgfpathlineto{\pgfqpoint{29.805208in}{0.773588in}}%
\pgfpathlineto{\pgfqpoint{29.753093in}{0.773588in}}%
\pgfpathlineto{\pgfqpoint{29.701409in}{0.773588in}}%
\pgfpathlineto{\pgfqpoint{29.647988in}{0.773588in}}%
\pgfpathlineto{\pgfqpoint{29.596078in}{0.773588in}}%
\pgfpathlineto{\pgfqpoint{29.543760in}{0.773588in}}%
\pgfpathlineto{\pgfqpoint{29.490075in}{0.773588in}}%
\pgfpathlineto{\pgfqpoint{29.438335in}{0.773588in}}%
\pgfpathlineto{\pgfqpoint{29.386173in}{0.773588in}}%
\pgfpathlineto{\pgfqpoint{29.332261in}{0.773588in}}%
\pgfpathlineto{\pgfqpoint{29.279399in}{0.773588in}}%
\pgfpathlineto{\pgfqpoint{29.226818in}{0.773588in}}%
\pgfpathlineto{\pgfqpoint{29.173700in}{0.773588in}}%
\pgfpathlineto{\pgfqpoint{29.121029in}{0.773588in}}%
\pgfpathlineto{\pgfqpoint{29.068863in}{0.773588in}}%
\pgfpathlineto{\pgfqpoint{29.015651in}{0.773588in}}%
\pgfpathlineto{\pgfqpoint{28.965063in}{0.773588in}}%
\pgfpathlineto{\pgfqpoint{28.913538in}{0.773588in}}%
\pgfpathlineto{\pgfqpoint{28.860347in}{0.773588in}}%
\pgfpathlineto{\pgfqpoint{28.808587in}{0.773588in}}%
\pgfpathlineto{\pgfqpoint{28.758074in}{0.773588in}}%
\pgfpathlineto{\pgfqpoint{28.706300in}{0.773588in}}%
\pgfpathlineto{\pgfqpoint{28.656185in}{0.773588in}}%
\pgfpathlineto{\pgfqpoint{28.605351in}{0.773588in}}%
\pgfpathlineto{\pgfqpoint{28.553749in}{0.773588in}}%
\pgfpathlineto{\pgfqpoint{28.502707in}{0.773588in}}%
\pgfpathlineto{\pgfqpoint{28.451952in}{0.773588in}}%
\pgfpathlineto{\pgfqpoint{28.399138in}{0.773588in}}%
\pgfpathlineto{\pgfqpoint{28.347853in}{0.773588in}}%
\pgfpathlineto{\pgfqpoint{28.297462in}{0.773588in}}%
\pgfpathlineto{\pgfqpoint{28.245320in}{0.773588in}}%
\pgfpathlineto{\pgfqpoint{28.195034in}{0.773588in}}%
\pgfpathlineto{\pgfqpoint{28.143944in}{0.773588in}}%
\pgfpathlineto{\pgfqpoint{28.090650in}{0.773588in}}%
\pgfpathlineto{\pgfqpoint{28.039571in}{0.773588in}}%
\pgfpathlineto{\pgfqpoint{27.988562in}{0.773588in}}%
\pgfpathlineto{\pgfqpoint{27.935127in}{0.773588in}}%
\pgfpathlineto{\pgfqpoint{27.883786in}{0.773588in}}%
\pgfpathlineto{\pgfqpoint{27.833067in}{0.773588in}}%
\pgfpathlineto{\pgfqpoint{27.779331in}{0.773588in}}%
\pgfpathlineto{\pgfqpoint{27.728571in}{0.773588in}}%
\pgfpathlineto{\pgfqpoint{27.678294in}{0.773588in}}%
\pgfpathlineto{\pgfqpoint{27.626352in}{0.773588in}}%
\pgfpathlineto{\pgfqpoint{27.576178in}{0.773588in}}%
\pgfpathlineto{\pgfqpoint{27.525849in}{0.773588in}}%
\pgfpathlineto{\pgfqpoint{27.473173in}{0.773588in}}%
\pgfpathlineto{\pgfqpoint{27.423066in}{0.773588in}}%
\pgfpathlineto{\pgfqpoint{27.372863in}{0.773588in}}%
\pgfpathlineto{\pgfqpoint{27.321160in}{0.773588in}}%
\pgfpathlineto{\pgfqpoint{27.269679in}{0.773588in}}%
\pgfpathlineto{\pgfqpoint{27.218587in}{0.773588in}}%
\pgfpathlineto{\pgfqpoint{27.165274in}{0.773588in}}%
\pgfpathlineto{\pgfqpoint{27.113950in}{0.773588in}}%
\pgfpathlineto{\pgfqpoint{27.063062in}{0.773588in}}%
\pgfpathlineto{\pgfqpoint{27.010539in}{0.773588in}}%
\pgfpathlineto{\pgfqpoint{26.960748in}{0.773588in}}%
\pgfpathlineto{\pgfqpoint{26.909530in}{0.773588in}}%
\pgfpathlineto{\pgfqpoint{26.857706in}{0.773588in}}%
\pgfpathlineto{\pgfqpoint{26.807173in}{0.773588in}}%
\pgfpathlineto{\pgfqpoint{26.756448in}{0.773588in}}%
\pgfpathlineto{\pgfqpoint{26.704176in}{0.773588in}}%
\pgfpathlineto{\pgfqpoint{26.653161in}{0.773588in}}%
\pgfpathlineto{\pgfqpoint{26.601868in}{0.773588in}}%
\pgfpathlineto{\pgfqpoint{26.549446in}{0.773588in}}%
\pgfpathlineto{\pgfqpoint{26.499064in}{0.773588in}}%
\pgfpathlineto{\pgfqpoint{26.447655in}{0.773588in}}%
\pgfpathlineto{\pgfqpoint{26.395829in}{0.773588in}}%
\pgfpathlineto{\pgfqpoint{26.346588in}{0.773588in}}%
\pgfpathlineto{\pgfqpoint{26.297518in}{0.773588in}}%
\pgfpathlineto{\pgfqpoint{26.245717in}{0.773588in}}%
\pgfpathlineto{\pgfqpoint{26.195263in}{0.773588in}}%
\pgfpathlineto{\pgfqpoint{26.144837in}{0.773588in}}%
\pgfpathlineto{\pgfqpoint{26.093895in}{0.773588in}}%
\pgfpathlineto{\pgfqpoint{26.044216in}{0.773588in}}%
\pgfpathlineto{\pgfqpoint{25.994130in}{0.773588in}}%
\pgfpathlineto{\pgfqpoint{25.942005in}{0.773588in}}%
\pgfpathlineto{\pgfqpoint{25.891991in}{0.773588in}}%
\pgfpathlineto{\pgfqpoint{25.841969in}{0.773588in}}%
\pgfpathlineto{\pgfqpoint{25.790287in}{0.773588in}}%
\pgfpathlineto{\pgfqpoint{25.740073in}{0.773588in}}%
\pgfpathlineto{\pgfqpoint{25.689152in}{0.773588in}}%
\pgfpathlineto{\pgfqpoint{25.637513in}{0.773588in}}%
\pgfpathlineto{\pgfqpoint{25.586387in}{0.773588in}}%
\pgfpathlineto{\pgfqpoint{25.535558in}{0.773588in}}%
\pgfpathlineto{\pgfqpoint{25.483381in}{0.773588in}}%
\pgfpathlineto{\pgfqpoint{25.433954in}{0.773588in}}%
\pgfpathlineto{\pgfqpoint{25.383622in}{0.773588in}}%
\pgfpathlineto{\pgfqpoint{25.332608in}{0.773588in}}%
\pgfpathlineto{\pgfqpoint{25.282222in}{0.773588in}}%
\pgfpathlineto{\pgfqpoint{25.232944in}{0.773588in}}%
\pgfpathlineto{\pgfqpoint{25.181554in}{0.773588in}}%
\pgfpathlineto{\pgfqpoint{25.131043in}{0.773588in}}%
\pgfpathlineto{\pgfqpoint{25.080539in}{0.773588in}}%
\pgfpathlineto{\pgfqpoint{25.028630in}{0.773588in}}%
\pgfpathlineto{\pgfqpoint{24.978211in}{0.773588in}}%
\pgfpathlineto{\pgfqpoint{24.928869in}{0.773588in}}%
\pgfpathlineto{\pgfqpoint{24.878143in}{0.773588in}}%
\pgfpathlineto{\pgfqpoint{24.828140in}{0.773588in}}%
\pgfpathlineto{\pgfqpoint{24.778304in}{0.773588in}}%
\pgfpathlineto{\pgfqpoint{24.726499in}{0.773588in}}%
\pgfpathlineto{\pgfqpoint{24.676389in}{0.773588in}}%
\pgfpathlineto{\pgfqpoint{24.626556in}{0.773588in}}%
\pgfpathlineto{\pgfqpoint{24.573998in}{0.773588in}}%
\pgfpathlineto{\pgfqpoint{24.524258in}{0.773588in}}%
\pgfpathlineto{\pgfqpoint{24.474381in}{0.773588in}}%
\pgfpathlineto{\pgfqpoint{24.422475in}{0.773588in}}%
\pgfpathlineto{\pgfqpoint{24.371890in}{0.773588in}}%
\pgfpathlineto{\pgfqpoint{24.321907in}{0.773588in}}%
\pgfpathlineto{\pgfqpoint{24.271277in}{0.773588in}}%
\pgfpathlineto{\pgfqpoint{24.221010in}{0.773588in}}%
\pgfpathlineto{\pgfqpoint{24.170873in}{0.773588in}}%
\pgfpathlineto{\pgfqpoint{24.118637in}{0.773588in}}%
\pgfpathlineto{\pgfqpoint{24.066933in}{0.773588in}}%
\pgfpathlineto{\pgfqpoint{24.016442in}{0.773588in}}%
\pgfpathlineto{\pgfqpoint{23.964514in}{0.773588in}}%
\pgfpathlineto{\pgfqpoint{23.913088in}{0.773588in}}%
\pgfpathlineto{\pgfqpoint{23.862673in}{0.773588in}}%
\pgfpathlineto{\pgfqpoint{23.811512in}{0.773588in}}%
\pgfpathlineto{\pgfqpoint{23.762329in}{0.773588in}}%
\pgfpathlineto{\pgfqpoint{23.713074in}{0.773588in}}%
\pgfpathlineto{\pgfqpoint{23.661575in}{0.773588in}}%
\pgfpathlineto{\pgfqpoint{23.611923in}{0.773588in}}%
\pgfpathlineto{\pgfqpoint{23.563429in}{0.773588in}}%
\pgfpathlineto{\pgfqpoint{23.511867in}{0.773588in}}%
\pgfpathlineto{\pgfqpoint{23.462017in}{0.773588in}}%
\pgfpathlineto{\pgfqpoint{23.411909in}{0.773588in}}%
\pgfpathlineto{\pgfqpoint{23.361310in}{0.773588in}}%
\pgfpathlineto{\pgfqpoint{23.310918in}{0.773588in}}%
\pgfpathlineto{\pgfqpoint{23.260726in}{0.773588in}}%
\pgfpathlineto{\pgfqpoint{23.208694in}{0.773588in}}%
\pgfpathlineto{\pgfqpoint{23.157756in}{0.773588in}}%
\pgfpathlineto{\pgfqpoint{23.107134in}{0.773588in}}%
\pgfpathlineto{\pgfqpoint{23.055187in}{0.773588in}}%
\pgfpathlineto{\pgfqpoint{23.006022in}{0.773588in}}%
\pgfpathlineto{\pgfqpoint{22.956359in}{0.773588in}}%
\pgfpathlineto{\pgfqpoint{22.904236in}{0.773588in}}%
\pgfpathlineto{\pgfqpoint{22.853966in}{0.773588in}}%
\pgfpathlineto{\pgfqpoint{22.803829in}{0.773588in}}%
\pgfpathlineto{\pgfqpoint{22.751700in}{0.773588in}}%
\pgfpathlineto{\pgfqpoint{22.703398in}{0.773588in}}%
\pgfpathlineto{\pgfqpoint{22.654822in}{0.773588in}}%
\pgfpathlineto{\pgfqpoint{22.603657in}{0.773588in}}%
\pgfpathlineto{\pgfqpoint{22.554073in}{0.773588in}}%
\pgfpathlineto{\pgfqpoint{22.503552in}{0.773588in}}%
\pgfpathlineto{\pgfqpoint{22.451976in}{0.773588in}}%
\pgfpathlineto{\pgfqpoint{22.402084in}{0.773588in}}%
\pgfpathlineto{\pgfqpoint{22.352827in}{0.773588in}}%
\pgfpathlineto{\pgfqpoint{22.300933in}{0.773588in}}%
\pgfpathlineto{\pgfqpoint{22.250634in}{0.773588in}}%
\pgfpathlineto{\pgfqpoint{22.200587in}{0.773588in}}%
\pgfpathlineto{\pgfqpoint{22.148817in}{0.773588in}}%
\pgfpathlineto{\pgfqpoint{22.099907in}{0.773588in}}%
\pgfpathlineto{\pgfqpoint{22.050354in}{0.773588in}}%
\pgfpathlineto{\pgfqpoint{21.999803in}{0.773588in}}%
\pgfpathlineto{\pgfqpoint{21.950544in}{0.773588in}}%
\pgfpathlineto{\pgfqpoint{21.900834in}{0.773588in}}%
\pgfpathlineto{\pgfqpoint{21.849580in}{0.773588in}}%
\pgfpathlineto{\pgfqpoint{21.800474in}{0.773588in}}%
\pgfpathlineto{\pgfqpoint{21.751624in}{0.773588in}}%
\pgfpathlineto{\pgfqpoint{21.700864in}{0.773588in}}%
\pgfpathlineto{\pgfqpoint{21.652276in}{0.773588in}}%
\pgfpathlineto{\pgfqpoint{21.602538in}{0.773588in}}%
\pgfpathlineto{\pgfqpoint{21.550957in}{0.773588in}}%
\pgfpathlineto{\pgfqpoint{21.501023in}{0.773588in}}%
\pgfpathlineto{\pgfqpoint{21.450830in}{0.773588in}}%
\pgfpathlineto{\pgfqpoint{21.399733in}{0.773588in}}%
\pgfpathlineto{\pgfqpoint{21.350353in}{0.773588in}}%
\pgfpathlineto{\pgfqpoint{21.300955in}{0.773588in}}%
\pgfpathlineto{\pgfqpoint{21.250655in}{0.773588in}}%
\pgfpathlineto{\pgfqpoint{21.201683in}{0.773588in}}%
\pgfpathlineto{\pgfqpoint{21.152719in}{0.773588in}}%
\pgfpathlineto{\pgfqpoint{21.102370in}{0.773588in}}%
\pgfpathlineto{\pgfqpoint{21.053515in}{0.773588in}}%
\pgfpathlineto{\pgfqpoint{21.003693in}{0.773588in}}%
\pgfpathlineto{\pgfqpoint{20.953585in}{0.773588in}}%
\pgfpathlineto{\pgfqpoint{20.904164in}{0.773588in}}%
\pgfpathlineto{\pgfqpoint{20.854118in}{0.773588in}}%
\pgfpathlineto{\pgfqpoint{20.802965in}{0.773588in}}%
\pgfpathlineto{\pgfqpoint{20.753558in}{0.773588in}}%
\pgfpathlineto{\pgfqpoint{20.704616in}{0.773588in}}%
\pgfpathlineto{\pgfqpoint{20.654668in}{0.773588in}}%
\pgfpathlineto{\pgfqpoint{20.605567in}{0.773588in}}%
\pgfpathlineto{\pgfqpoint{20.556290in}{0.773588in}}%
\pgfpathlineto{\pgfqpoint{20.505208in}{0.773588in}}%
\pgfpathlineto{\pgfqpoint{20.455604in}{0.773588in}}%
\pgfpathlineto{\pgfqpoint{20.406414in}{0.773588in}}%
\pgfpathlineto{\pgfqpoint{20.355462in}{0.773588in}}%
\pgfpathlineto{\pgfqpoint{20.305924in}{0.773588in}}%
\pgfpathlineto{\pgfqpoint{20.255593in}{0.773588in}}%
\pgfpathlineto{\pgfqpoint{20.203615in}{0.773588in}}%
\pgfpathlineto{\pgfqpoint{20.153679in}{0.773588in}}%
\pgfpathlineto{\pgfqpoint{20.104350in}{0.773588in}}%
\pgfpathlineto{\pgfqpoint{20.053312in}{0.773588in}}%
\pgfpathlineto{\pgfqpoint{20.003692in}{0.773588in}}%
\pgfpathlineto{\pgfqpoint{19.955252in}{0.773588in}}%
\pgfpathlineto{\pgfqpoint{19.906615in}{0.773588in}}%
\pgfpathlineto{\pgfqpoint{19.858933in}{0.773588in}}%
\pgfpathlineto{\pgfqpoint{19.810712in}{0.773588in}}%
\pgfpathlineto{\pgfqpoint{19.761678in}{0.773588in}}%
\pgfpathlineto{\pgfqpoint{19.714265in}{0.773588in}}%
\pgfpathlineto{\pgfqpoint{19.666940in}{0.773588in}}%
\pgfpathlineto{\pgfqpoint{19.617707in}{0.773588in}}%
\pgfpathlineto{\pgfqpoint{19.569747in}{0.773588in}}%
\pgfpathlineto{\pgfqpoint{19.522292in}{0.773588in}}%
\pgfpathlineto{\pgfqpoint{19.473154in}{0.773588in}}%
\pgfpathlineto{\pgfqpoint{19.425178in}{0.773588in}}%
\pgfpathlineto{\pgfqpoint{19.376890in}{0.773588in}}%
\pgfpathlineto{\pgfqpoint{19.327829in}{0.773588in}}%
\pgfpathlineto{\pgfqpoint{19.279742in}{0.773588in}}%
\pgfpathlineto{\pgfqpoint{19.231826in}{0.773588in}}%
\pgfpathlineto{\pgfqpoint{19.181302in}{0.773588in}}%
\pgfpathlineto{\pgfqpoint{19.132724in}{0.773588in}}%
\pgfpathlineto{\pgfqpoint{19.084207in}{0.773588in}}%
\pgfpathlineto{\pgfqpoint{19.033794in}{0.773588in}}%
\pgfpathlineto{\pgfqpoint{18.985002in}{0.773588in}}%
\pgfpathlineto{\pgfqpoint{18.937671in}{0.773588in}}%
\pgfpathlineto{\pgfqpoint{18.888188in}{0.773588in}}%
\pgfpathlineto{\pgfqpoint{18.839312in}{0.773588in}}%
\pgfpathlineto{\pgfqpoint{18.791212in}{0.773588in}}%
\pgfpathlineto{\pgfqpoint{18.741665in}{0.773588in}}%
\pgfpathlineto{\pgfqpoint{18.693122in}{0.773588in}}%
\pgfpathlineto{\pgfqpoint{18.644639in}{0.773588in}}%
\pgfpathlineto{\pgfqpoint{18.595724in}{0.773588in}}%
\pgfpathlineto{\pgfqpoint{18.548460in}{0.773588in}}%
\pgfpathlineto{\pgfqpoint{18.500915in}{0.773588in}}%
\pgfpathlineto{\pgfqpoint{18.452284in}{0.773588in}}%
\pgfpathlineto{\pgfqpoint{18.405242in}{0.773588in}}%
\pgfpathlineto{\pgfqpoint{18.357538in}{0.773588in}}%
\pgfpathlineto{\pgfqpoint{18.308441in}{0.773588in}}%
\pgfpathlineto{\pgfqpoint{18.260871in}{0.773588in}}%
\pgfpathlineto{\pgfqpoint{18.213333in}{0.773588in}}%
\pgfpathlineto{\pgfqpoint{18.164405in}{0.773588in}}%
\pgfpathlineto{\pgfqpoint{18.117307in}{0.773588in}}%
\pgfpathlineto{\pgfqpoint{18.069524in}{0.773588in}}%
\pgfpathlineto{\pgfqpoint{18.020026in}{0.773588in}}%
\pgfpathlineto{\pgfqpoint{17.970910in}{0.773588in}}%
\pgfpathlineto{\pgfqpoint{17.922885in}{0.773588in}}%
\pgfpathlineto{\pgfqpoint{17.874600in}{0.773588in}}%
\pgfpathlineto{\pgfqpoint{17.826809in}{0.773588in}}%
\pgfpathlineto{\pgfqpoint{17.779014in}{0.773588in}}%
\pgfpathlineto{\pgfqpoint{17.729727in}{0.773588in}}%
\pgfpathlineto{\pgfqpoint{17.681914in}{0.773588in}}%
\pgfpathlineto{\pgfqpoint{17.634653in}{0.773588in}}%
\pgfpathlineto{\pgfqpoint{17.585742in}{0.773588in}}%
\pgfpathlineto{\pgfqpoint{17.538409in}{0.773588in}}%
\pgfpathlineto{\pgfqpoint{17.491125in}{0.773588in}}%
\pgfpathlineto{\pgfqpoint{17.442362in}{0.773588in}}%
\pgfpathlineto{\pgfqpoint{17.394714in}{0.773588in}}%
\pgfpathlineto{\pgfqpoint{17.347576in}{0.773588in}}%
\pgfpathlineto{\pgfqpoint{17.298941in}{0.773588in}}%
\pgfpathlineto{\pgfqpoint{17.250860in}{0.773588in}}%
\pgfpathlineto{\pgfqpoint{17.203674in}{0.773588in}}%
\pgfpathlineto{\pgfqpoint{17.154507in}{0.773588in}}%
\pgfpathlineto{\pgfqpoint{17.107470in}{0.773588in}}%
\pgfpathlineto{\pgfqpoint{17.059518in}{0.773588in}}%
\pgfpathlineto{\pgfqpoint{17.010585in}{0.773588in}}%
\pgfpathlineto{\pgfqpoint{16.963533in}{0.773588in}}%
\pgfpathlineto{\pgfqpoint{16.916302in}{0.773588in}}%
\pgfpathlineto{\pgfqpoint{16.868266in}{0.773588in}}%
\pgfpathlineto{\pgfqpoint{16.821038in}{0.773588in}}%
\pgfpathlineto{\pgfqpoint{16.773713in}{0.773588in}}%
\pgfpathlineto{\pgfqpoint{16.724568in}{0.773588in}}%
\pgfpathlineto{\pgfqpoint{16.677566in}{0.773588in}}%
\pgfpathlineto{\pgfqpoint{16.630280in}{0.773588in}}%
\pgfpathlineto{\pgfqpoint{16.581135in}{0.773588in}}%
\pgfpathlineto{\pgfqpoint{16.533619in}{0.773588in}}%
\pgfpathlineto{\pgfqpoint{16.486560in}{0.773588in}}%
\pgfpathlineto{\pgfqpoint{16.436622in}{0.773588in}}%
\pgfpathlineto{\pgfqpoint{16.387666in}{0.773588in}}%
\pgfpathlineto{\pgfqpoint{16.339537in}{0.773588in}}%
\pgfpathlineto{\pgfqpoint{16.289817in}{0.773588in}}%
\pgfpathlineto{\pgfqpoint{16.242078in}{0.773588in}}%
\pgfpathlineto{\pgfqpoint{16.195175in}{0.773588in}}%
\pgfpathlineto{\pgfqpoint{16.146338in}{0.773588in}}%
\pgfpathlineto{\pgfqpoint{16.098697in}{0.773588in}}%
\pgfpathlineto{\pgfqpoint{16.051820in}{0.773588in}}%
\pgfpathlineto{\pgfqpoint{16.003926in}{0.773588in}}%
\pgfpathlineto{\pgfqpoint{15.957799in}{0.773588in}}%
\pgfpathlineto{\pgfqpoint{15.911811in}{0.773588in}}%
\pgfpathlineto{\pgfqpoint{15.863592in}{0.773588in}}%
\pgfpathlineto{\pgfqpoint{15.817273in}{0.773588in}}%
\pgfpathlineto{\pgfqpoint{15.770154in}{0.773588in}}%
\pgfpathlineto{\pgfqpoint{15.722814in}{0.773588in}}%
\pgfpathlineto{\pgfqpoint{15.676019in}{0.773588in}}%
\pgfpathlineto{\pgfqpoint{15.628660in}{0.773588in}}%
\pgfpathlineto{\pgfqpoint{15.579815in}{0.773588in}}%
\pgfpathlineto{\pgfqpoint{15.532329in}{0.773588in}}%
\pgfpathlineto{\pgfqpoint{15.484858in}{0.773588in}}%
\pgfpathlineto{\pgfqpoint{15.436052in}{0.773588in}}%
\pgfpathlineto{\pgfqpoint{15.389896in}{0.773588in}}%
\pgfpathlineto{\pgfqpoint{15.343883in}{0.773588in}}%
\pgfpathlineto{\pgfqpoint{15.295866in}{0.773588in}}%
\pgfpathlineto{\pgfqpoint{15.248863in}{0.773588in}}%
\pgfpathlineto{\pgfqpoint{15.201456in}{0.773588in}}%
\pgfpathlineto{\pgfqpoint{15.152658in}{0.773588in}}%
\pgfpathlineto{\pgfqpoint{15.105478in}{0.773588in}}%
\pgfpathlineto{\pgfqpoint{15.058774in}{0.773588in}}%
\pgfpathlineto{\pgfqpoint{15.010214in}{0.773588in}}%
\pgfpathlineto{\pgfqpoint{14.963167in}{0.773588in}}%
\pgfpathlineto{\pgfqpoint{14.916239in}{0.773588in}}%
\pgfpathlineto{\pgfqpoint{14.868242in}{0.773588in}}%
\pgfpathlineto{\pgfqpoint{14.821474in}{0.773588in}}%
\pgfpathlineto{\pgfqpoint{14.774752in}{0.773588in}}%
\pgfpathlineto{\pgfqpoint{14.726757in}{0.773588in}}%
\pgfpathlineto{\pgfqpoint{14.680226in}{0.773588in}}%
\pgfpathlineto{\pgfqpoint{14.633992in}{0.773588in}}%
\pgfpathlineto{\pgfqpoint{14.585477in}{0.773588in}}%
\pgfpathlineto{\pgfqpoint{14.538630in}{0.773588in}}%
\pgfpathlineto{\pgfqpoint{14.491726in}{0.773588in}}%
\pgfpathlineto{\pgfqpoint{14.443729in}{0.773588in}}%
\pgfpathlineto{\pgfqpoint{14.397126in}{0.773588in}}%
\pgfpathlineto{\pgfqpoint{14.350224in}{0.773588in}}%
\pgfpathlineto{\pgfqpoint{14.302418in}{0.773588in}}%
\pgfpathlineto{\pgfqpoint{14.255593in}{0.773588in}}%
\pgfpathlineto{\pgfqpoint{14.209376in}{0.773588in}}%
\pgfpathlineto{\pgfqpoint{14.160147in}{0.773588in}}%
\pgfpathlineto{\pgfqpoint{14.112147in}{0.773588in}}%
\pgfpathlineto{\pgfqpoint{14.064951in}{0.773588in}}%
\pgfpathlineto{\pgfqpoint{14.016344in}{0.773588in}}%
\pgfpathlineto{\pgfqpoint{13.969353in}{0.773588in}}%
\pgfpathlineto{\pgfqpoint{13.922606in}{0.773588in}}%
\pgfpathlineto{\pgfqpoint{13.873610in}{0.773588in}}%
\pgfpathlineto{\pgfqpoint{13.825297in}{0.773588in}}%
\pgfpathlineto{\pgfqpoint{13.777441in}{0.773588in}}%
\pgfpathlineto{\pgfqpoint{13.729143in}{0.773588in}}%
\pgfpathlineto{\pgfqpoint{13.682151in}{0.773588in}}%
\pgfpathlineto{\pgfqpoint{13.635346in}{0.773588in}}%
\pgfpathlineto{\pgfqpoint{13.587249in}{0.773588in}}%
\pgfpathlineto{\pgfqpoint{13.541646in}{0.773588in}}%
\pgfpathlineto{\pgfqpoint{13.495727in}{0.773588in}}%
\pgfpathlineto{\pgfqpoint{13.448069in}{0.773588in}}%
\pgfpathlineto{\pgfqpoint{13.401859in}{0.773588in}}%
\pgfpathlineto{\pgfqpoint{13.356518in}{0.773588in}}%
\pgfpathlineto{\pgfqpoint{13.308974in}{0.773588in}}%
\pgfpathlineto{\pgfqpoint{13.262980in}{0.773588in}}%
\pgfpathlineto{\pgfqpoint{13.216719in}{0.773588in}}%
\pgfpathlineto{\pgfqpoint{13.169075in}{0.773588in}}%
\pgfpathlineto{\pgfqpoint{13.123339in}{0.773588in}}%
\pgfpathlineto{\pgfqpoint{13.077275in}{0.773588in}}%
\pgfpathlineto{\pgfqpoint{13.030499in}{0.773588in}}%
\pgfpathlineto{\pgfqpoint{12.983596in}{0.773588in}}%
\pgfpathlineto{\pgfqpoint{12.936799in}{0.773588in}}%
\pgfpathlineto{\pgfqpoint{12.889988in}{0.773588in}}%
\pgfpathlineto{\pgfqpoint{12.843155in}{0.773588in}}%
\pgfpathlineto{\pgfqpoint{12.796907in}{0.773588in}}%
\pgfpathlineto{\pgfqpoint{12.749025in}{0.773588in}}%
\pgfpathlineto{\pgfqpoint{12.701835in}{0.773588in}}%
\pgfpathlineto{\pgfqpoint{12.655424in}{0.773588in}}%
\pgfpathlineto{\pgfqpoint{12.607615in}{0.773588in}}%
\pgfpathlineto{\pgfqpoint{12.560711in}{0.773588in}}%
\pgfpathlineto{\pgfqpoint{12.514481in}{0.773588in}}%
\pgfpathlineto{\pgfqpoint{12.467453in}{0.773588in}}%
\pgfpathlineto{\pgfqpoint{12.421372in}{0.773588in}}%
\pgfpathlineto{\pgfqpoint{12.375261in}{0.773588in}}%
\pgfpathlineto{\pgfqpoint{12.328136in}{0.773588in}}%
\pgfpathlineto{\pgfqpoint{12.282271in}{0.773588in}}%
\pgfpathlineto{\pgfqpoint{12.235985in}{0.773588in}}%
\pgfpathlineto{\pgfqpoint{12.187925in}{0.773588in}}%
\pgfpathlineto{\pgfqpoint{12.141887in}{0.773588in}}%
\pgfpathlineto{\pgfqpoint{12.096623in}{0.773588in}}%
\pgfpathlineto{\pgfqpoint{12.049869in}{0.773588in}}%
\pgfpathlineto{\pgfqpoint{12.004414in}{0.773588in}}%
\pgfpathlineto{\pgfqpoint{11.959143in}{0.773588in}}%
\pgfpathlineto{\pgfqpoint{11.911501in}{0.773588in}}%
\pgfpathlineto{\pgfqpoint{11.864862in}{0.773588in}}%
\pgfpathlineto{\pgfqpoint{11.819406in}{0.773588in}}%
\pgfpathlineto{\pgfqpoint{11.772894in}{0.773588in}}%
\pgfpathlineto{\pgfqpoint{11.727607in}{0.773588in}}%
\pgfpathlineto{\pgfqpoint{11.682053in}{0.773588in}}%
\pgfpathlineto{\pgfqpoint{11.634589in}{0.773588in}}%
\pgfpathlineto{\pgfqpoint{11.588799in}{0.773588in}}%
\pgfpathlineto{\pgfqpoint{11.542665in}{0.773588in}}%
\pgfpathlineto{\pgfqpoint{11.494874in}{0.773588in}}%
\pgfpathlineto{\pgfqpoint{11.448531in}{0.773588in}}%
\pgfpathlineto{\pgfqpoint{11.402044in}{0.773588in}}%
\pgfpathlineto{\pgfqpoint{11.353756in}{0.773588in}}%
\pgfpathlineto{\pgfqpoint{11.307361in}{0.773588in}}%
\pgfpathlineto{\pgfqpoint{11.261350in}{0.773588in}}%
\pgfpathlineto{\pgfqpoint{11.213355in}{0.773588in}}%
\pgfpathlineto{\pgfqpoint{11.167471in}{0.773588in}}%
\pgfpathlineto{\pgfqpoint{11.121978in}{0.773588in}}%
\pgfpathlineto{\pgfqpoint{11.074633in}{0.773588in}}%
\pgfpathlineto{\pgfqpoint{11.029398in}{0.773588in}}%
\pgfpathlineto{\pgfqpoint{10.984145in}{0.773588in}}%
\pgfpathlineto{\pgfqpoint{10.937381in}{0.773588in}}%
\pgfpathlineto{\pgfqpoint{10.891801in}{0.773588in}}%
\pgfpathlineto{\pgfqpoint{10.845790in}{0.773588in}}%
\pgfpathlineto{\pgfqpoint{10.798768in}{0.773588in}}%
\pgfpathlineto{\pgfqpoint{10.753168in}{0.773588in}}%
\pgfpathlineto{\pgfqpoint{10.707461in}{0.773588in}}%
\pgfpathlineto{\pgfqpoint{10.660112in}{0.773588in}}%
\pgfpathlineto{\pgfqpoint{10.613886in}{0.773588in}}%
\pgfpathlineto{\pgfqpoint{10.568154in}{0.773588in}}%
\pgfpathlineto{\pgfqpoint{10.521105in}{0.773588in}}%
\pgfpathlineto{\pgfqpoint{10.475642in}{0.773588in}}%
\pgfpathlineto{\pgfqpoint{10.429362in}{0.773588in}}%
\pgfpathlineto{\pgfqpoint{10.381570in}{0.773588in}}%
\pgfpathlineto{\pgfqpoint{10.335334in}{0.773588in}}%
\pgfpathlineto{\pgfqpoint{10.289084in}{0.773588in}}%
\pgfpathlineto{\pgfqpoint{10.242601in}{0.773588in}}%
\pgfpathlineto{\pgfqpoint{10.197247in}{0.773588in}}%
\pgfpathlineto{\pgfqpoint{10.151348in}{0.773588in}}%
\pgfpathlineto{\pgfqpoint{10.103098in}{0.773588in}}%
\pgfpathlineto{\pgfqpoint{10.057613in}{0.773588in}}%
\pgfpathlineto{\pgfqpoint{10.011915in}{0.773588in}}%
\pgfpathlineto{\pgfqpoint{9.964562in}{0.773588in}}%
\pgfpathlineto{\pgfqpoint{9.918842in}{0.773588in}}%
\pgfpathlineto{\pgfqpoint{9.873173in}{0.773588in}}%
\pgfpathlineto{\pgfqpoint{9.826599in}{0.773588in}}%
\pgfpathlineto{\pgfqpoint{9.781321in}{0.773588in}}%
\pgfpathlineto{\pgfqpoint{9.735785in}{0.773588in}}%
\pgfpathlineto{\pgfqpoint{9.689497in}{0.773588in}}%
\pgfpathlineto{\pgfqpoint{9.644121in}{0.773588in}}%
\pgfpathlineto{\pgfqpoint{9.598338in}{0.773588in}}%
\pgfpathlineto{\pgfqpoint{9.552197in}{0.773588in}}%
\pgfpathlineto{\pgfqpoint{9.507583in}{0.773588in}}%
\pgfpathlineto{\pgfqpoint{9.462006in}{0.773588in}}%
\pgfpathlineto{\pgfqpoint{9.415178in}{0.773588in}}%
\pgfpathlineto{\pgfqpoint{9.369639in}{0.773588in}}%
\pgfpathlineto{\pgfqpoint{9.324423in}{0.773588in}}%
\pgfpathlineto{\pgfqpoint{9.278439in}{0.773588in}}%
\pgfpathlineto{\pgfqpoint{9.233107in}{0.773588in}}%
\pgfpathlineto{\pgfqpoint{9.188332in}{0.773588in}}%
\pgfpathlineto{\pgfqpoint{9.141270in}{0.773588in}}%
\pgfpathlineto{\pgfqpoint{9.095727in}{0.773588in}}%
\pgfpathlineto{\pgfqpoint{9.050165in}{0.773588in}}%
\pgfpathlineto{\pgfqpoint{9.003947in}{0.773588in}}%
\pgfpathlineto{\pgfqpoint{8.957924in}{0.773588in}}%
\pgfpathlineto{\pgfqpoint{8.912131in}{0.773588in}}%
\pgfpathlineto{\pgfqpoint{8.866242in}{0.773588in}}%
\pgfpathlineto{\pgfqpoint{8.821068in}{0.773588in}}%
\pgfpathlineto{\pgfqpoint{8.774902in}{0.773588in}}%
\pgfpathlineto{\pgfqpoint{8.727404in}{0.773588in}}%
\pgfpathlineto{\pgfqpoint{8.681080in}{0.773588in}}%
\pgfpathlineto{\pgfqpoint{8.635315in}{0.773588in}}%
\pgfpathlineto{\pgfqpoint{8.589124in}{0.773588in}}%
\pgfpathlineto{\pgfqpoint{8.543976in}{0.773588in}}%
\pgfpathlineto{\pgfqpoint{8.499298in}{0.773588in}}%
\pgfpathlineto{\pgfqpoint{8.453780in}{0.773588in}}%
\pgfpathlineto{\pgfqpoint{8.409767in}{0.773588in}}%
\pgfpathlineto{\pgfqpoint{8.364636in}{0.773588in}}%
\pgfpathlineto{\pgfqpoint{8.318789in}{0.773588in}}%
\pgfpathlineto{\pgfqpoint{8.273006in}{0.773588in}}%
\pgfpathlineto{\pgfqpoint{8.227930in}{0.773588in}}%
\pgfpathlineto{\pgfqpoint{8.181791in}{0.773588in}}%
\pgfpathlineto{\pgfqpoint{8.136842in}{0.773588in}}%
\pgfpathlineto{\pgfqpoint{8.091881in}{0.773588in}}%
\pgfpathlineto{\pgfqpoint{8.045278in}{0.773588in}}%
\pgfpathlineto{\pgfqpoint{8.000573in}{0.773588in}}%
\pgfpathlineto{\pgfqpoint{7.955879in}{0.773588in}}%
\pgfpathlineto{\pgfqpoint{7.910161in}{0.773588in}}%
\pgfpathlineto{\pgfqpoint{7.865263in}{0.773588in}}%
\pgfpathlineto{\pgfqpoint{7.819947in}{0.773588in}}%
\pgfpathlineto{\pgfqpoint{7.773226in}{0.773588in}}%
\pgfpathlineto{\pgfqpoint{7.728803in}{0.773588in}}%
\pgfpathlineto{\pgfqpoint{7.682978in}{0.773588in}}%
\pgfpathlineto{\pgfqpoint{7.636592in}{0.773588in}}%
\pgfpathlineto{\pgfqpoint{7.591890in}{0.773588in}}%
\pgfpathlineto{\pgfqpoint{7.546892in}{0.773588in}}%
\pgfpathlineto{\pgfqpoint{7.500683in}{0.773588in}}%
\pgfpathlineto{\pgfqpoint{7.455548in}{0.773588in}}%
\pgfpathlineto{\pgfqpoint{7.409977in}{0.773588in}}%
\pgfpathlineto{\pgfqpoint{7.363749in}{0.773588in}}%
\pgfpathlineto{\pgfqpoint{7.318484in}{0.773588in}}%
\pgfpathlineto{\pgfqpoint{7.273914in}{0.773588in}}%
\pgfpathlineto{\pgfqpoint{7.228120in}{0.773588in}}%
\pgfpathlineto{\pgfqpoint{7.184277in}{0.773588in}}%
\pgfpathlineto{\pgfqpoint{7.140134in}{0.773588in}}%
\pgfpathlineto{\pgfqpoint{7.094205in}{0.773588in}}%
\pgfpathlineto{\pgfqpoint{7.050071in}{0.773588in}}%
\pgfpathlineto{\pgfqpoint{7.005149in}{0.773588in}}%
\pgfpathlineto{\pgfqpoint{6.958763in}{0.773588in}}%
\pgfpathlineto{\pgfqpoint{6.914194in}{0.773588in}}%
\pgfpathlineto{\pgfqpoint{6.869544in}{0.773588in}}%
\pgfpathlineto{\pgfqpoint{6.824012in}{0.773588in}}%
\pgfpathlineto{\pgfqpoint{6.779295in}{0.773588in}}%
\pgfpathlineto{\pgfqpoint{6.734887in}{0.773588in}}%
\pgfpathlineto{\pgfqpoint{6.688504in}{0.773588in}}%
\pgfpathlineto{\pgfqpoint{6.643960in}{0.773588in}}%
\pgfpathlineto{\pgfqpoint{6.599302in}{0.773588in}}%
\pgfpathlineto{\pgfqpoint{6.553117in}{0.773588in}}%
\pgfpathlineto{\pgfqpoint{6.507651in}{0.773588in}}%
\pgfpathlineto{\pgfqpoint{6.461324in}{0.773588in}}%
\pgfpathlineto{\pgfqpoint{6.413399in}{0.773588in}}%
\pgfpathlineto{\pgfqpoint{6.367508in}{0.773588in}}%
\pgfpathlineto{\pgfqpoint{6.321070in}{0.773588in}}%
\pgfpathlineto{\pgfqpoint{6.273584in}{0.773588in}}%
\pgfpathlineto{\pgfqpoint{6.227112in}{0.773588in}}%
\pgfpathlineto{\pgfqpoint{6.180802in}{0.773588in}}%
\pgfpathlineto{\pgfqpoint{6.133235in}{0.773588in}}%
\pgfpathlineto{\pgfqpoint{6.087229in}{0.773588in}}%
\pgfpathlineto{\pgfqpoint{6.041630in}{0.773588in}}%
\pgfpathlineto{\pgfqpoint{5.994686in}{0.773588in}}%
\pgfpathlineto{\pgfqpoint{5.948991in}{0.773588in}}%
\pgfpathlineto{\pgfqpoint{5.903366in}{0.773588in}}%
\pgfpathlineto{\pgfqpoint{5.856181in}{0.773588in}}%
\pgfpathlineto{\pgfqpoint{5.810269in}{0.773588in}}%
\pgfpathlineto{\pgfqpoint{5.765813in}{0.773588in}}%
\pgfpathlineto{\pgfqpoint{5.719569in}{0.773588in}}%
\pgfpathlineto{\pgfqpoint{5.674262in}{0.773588in}}%
\pgfpathlineto{\pgfqpoint{5.628857in}{0.773588in}}%
\pgfpathlineto{\pgfqpoint{5.581803in}{0.773588in}}%
\pgfpathlineto{\pgfqpoint{5.536203in}{0.773588in}}%
\pgfpathlineto{\pgfqpoint{5.491128in}{0.773588in}}%
\pgfpathlineto{\pgfqpoint{5.444526in}{0.773588in}}%
\pgfpathlineto{\pgfqpoint{5.399334in}{0.773588in}}%
\pgfpathlineto{\pgfqpoint{5.353614in}{0.773588in}}%
\pgfpathlineto{\pgfqpoint{5.307244in}{0.773588in}}%
\pgfpathlineto{\pgfqpoint{5.262146in}{0.773588in}}%
\pgfpathlineto{\pgfqpoint{5.216291in}{0.773588in}}%
\pgfpathlineto{\pgfqpoint{5.169192in}{0.773588in}}%
\pgfpathlineto{\pgfqpoint{5.124036in}{0.773588in}}%
\pgfpathlineto{\pgfqpoint{5.078074in}{0.773588in}}%
\pgfpathlineto{\pgfqpoint{5.030678in}{0.773588in}}%
\pgfpathlineto{\pgfqpoint{4.984973in}{0.773588in}}%
\pgfpathlineto{\pgfqpoint{4.939032in}{0.773588in}}%
\pgfpathlineto{\pgfqpoint{4.891298in}{0.773588in}}%
\pgfpathlineto{\pgfqpoint{4.845429in}{0.773588in}}%
\pgfpathlineto{\pgfqpoint{4.799887in}{0.773588in}}%
\pgfpathlineto{\pgfqpoint{4.753449in}{0.773588in}}%
\pgfpathlineto{\pgfqpoint{4.708228in}{0.773588in}}%
\pgfpathlineto{\pgfqpoint{4.663622in}{0.773588in}}%
\pgfpathlineto{\pgfqpoint{4.617918in}{0.773588in}}%
\pgfpathlineto{\pgfqpoint{4.572708in}{0.773588in}}%
\pgfpathlineto{\pgfqpoint{4.527424in}{0.773588in}}%
\pgfpathlineto{\pgfqpoint{4.480684in}{0.773588in}}%
\pgfpathlineto{\pgfqpoint{4.435284in}{0.773588in}}%
\pgfpathlineto{\pgfqpoint{4.390187in}{0.773588in}}%
\pgfpathlineto{\pgfqpoint{4.343118in}{0.773588in}}%
\pgfpathlineto{\pgfqpoint{4.297354in}{0.773588in}}%
\pgfpathlineto{\pgfqpoint{4.252389in}{0.773588in}}%
\pgfpathlineto{\pgfqpoint{4.204543in}{0.773588in}}%
\pgfpathlineto{\pgfqpoint{4.159078in}{0.773588in}}%
\pgfpathlineto{\pgfqpoint{4.114362in}{0.773588in}}%
\pgfpathlineto{\pgfqpoint{4.067507in}{0.773588in}}%
\pgfpathlineto{\pgfqpoint{4.022115in}{0.773588in}}%
\pgfpathlineto{\pgfqpoint{3.976868in}{0.773588in}}%
\pgfpathlineto{\pgfqpoint{3.930307in}{0.773588in}}%
\pgfpathlineto{\pgfqpoint{3.884614in}{0.773588in}}%
\pgfpathlineto{\pgfqpoint{3.839775in}{0.773588in}}%
\pgfpathlineto{\pgfqpoint{3.793040in}{0.773588in}}%
\pgfpathlineto{\pgfqpoint{3.747249in}{0.773588in}}%
\pgfpathlineto{\pgfqpoint{3.701949in}{0.773588in}}%
\pgfpathlineto{\pgfqpoint{3.654909in}{0.773588in}}%
\pgfpathlineto{\pgfqpoint{3.609556in}{0.773588in}}%
\pgfpathlineto{\pgfqpoint{3.564489in}{0.773588in}}%
\pgfpathlineto{\pgfqpoint{3.517815in}{0.773588in}}%
\pgfpathlineto{\pgfqpoint{3.473135in}{0.773588in}}%
\pgfpathlineto{\pgfqpoint{3.428526in}{0.773588in}}%
\pgfpathlineto{\pgfqpoint{3.381274in}{0.773588in}}%
\pgfpathlineto{\pgfqpoint{3.336184in}{0.773588in}}%
\pgfpathlineto{\pgfqpoint{3.290748in}{0.773588in}}%
\pgfpathlineto{\pgfqpoint{3.245458in}{0.773588in}}%
\pgfpathlineto{\pgfqpoint{3.200519in}{0.773588in}}%
\pgfpathlineto{\pgfqpoint{3.155580in}{0.773588in}}%
\pgfpathlineto{\pgfqpoint{3.108180in}{0.773588in}}%
\pgfpathlineto{\pgfqpoint{3.062761in}{0.773588in}}%
\pgfpathlineto{\pgfqpoint{3.017486in}{0.773588in}}%
\pgfpathlineto{\pgfqpoint{2.971917in}{0.773588in}}%
\pgfpathlineto{\pgfqpoint{2.927413in}{0.773588in}}%
\pgfpathlineto{\pgfqpoint{2.883134in}{0.773588in}}%
\pgfpathlineto{\pgfqpoint{2.836381in}{0.773588in}}%
\pgfpathlineto{\pgfqpoint{2.790736in}{0.773588in}}%
\pgfpathlineto{\pgfqpoint{2.745668in}{0.773588in}}%
\pgfpathlineto{\pgfqpoint{2.698034in}{0.773588in}}%
\pgfpathlineto{\pgfqpoint{2.651003in}{0.773588in}}%
\pgfpathlineto{\pgfqpoint{2.604306in}{0.773588in}}%
\pgfpathlineto{\pgfqpoint{2.555498in}{0.773588in}}%
\pgfpathlineto{\pgfqpoint{2.505534in}{0.773588in}}%
\pgfpathlineto{\pgfqpoint{2.452591in}{0.773588in}}%
\pgfpathlineto{\pgfqpoint{2.397147in}{0.773588in}}%
\pgfpathlineto{\pgfqpoint{2.348431in}{0.773588in}}%
\pgfpathlineto{\pgfqpoint{2.299591in}{0.773588in}}%
\pgfpathlineto{\pgfqpoint{2.249804in}{0.773588in}}%
\pgfpathlineto{\pgfqpoint{2.201242in}{0.773588in}}%
\pgfpathlineto{\pgfqpoint{2.153284in}{0.773588in}}%
\pgfpathlineto{\pgfqpoint{2.104325in}{0.773588in}}%
\pgfpathlineto{\pgfqpoint{2.057441in}{0.773588in}}%
\pgfpathlineto{\pgfqpoint{2.011209in}{0.773588in}}%
\pgfpathlineto{\pgfqpoint{1.963476in}{0.773588in}}%
\pgfpathlineto{\pgfqpoint{1.915908in}{0.773588in}}%
\pgfpathlineto{\pgfqpoint{1.869493in}{0.773588in}}%
\pgfpathlineto{\pgfqpoint{1.822804in}{0.773588in}}%
\pgfpathlineto{\pgfqpoint{1.778099in}{0.773588in}}%
\pgfpathlineto{\pgfqpoint{1.734500in}{0.773588in}}%
\pgfpathlineto{\pgfqpoint{1.689326in}{0.773588in}}%
\pgfpathlineto{\pgfqpoint{1.645274in}{0.773588in}}%
\pgfpathlineto{\pgfqpoint{1.600816in}{0.773588in}}%
\pgfpathlineto{\pgfqpoint{1.555718in}{0.773588in}}%
\pgfpathlineto{\pgfqpoint{1.511740in}{0.773588in}}%
\pgfpathlineto{\pgfqpoint{1.468334in}{0.773588in}}%
\pgfpathlineto{\pgfqpoint{1.422957in}{0.773588in}}%
\pgfpathlineto{\pgfqpoint{1.378287in}{0.773588in}}%
\pgfpathlineto{\pgfqpoint{1.334262in}{0.773588in}}%
\pgfpathlineto{\pgfqpoint{1.289074in}{0.773588in}}%
\pgfpathlineto{\pgfqpoint{1.244609in}{0.773588in}}%
\pgfpathlineto{\pgfqpoint{1.200192in}{0.773588in}}%
\pgfpathlineto{\pgfqpoint{1.155171in}{0.773588in}}%
\pgfpathlineto{\pgfqpoint{1.111790in}{0.773588in}}%
\pgfpathlineto{\pgfqpoint{1.067773in}{0.773588in}}%
\pgfpathlineto{\pgfqpoint{1.021908in}{0.773588in}}%
\pgfpathlineto{\pgfqpoint{0.978015in}{0.773588in}}%
\pgfpathlineto{\pgfqpoint{0.933783in}{0.773588in}}%
\pgfpathlineto{\pgfqpoint{0.887244in}{0.773588in}}%
\pgfpathlineto{\pgfqpoint{0.842612in}{0.773588in}}%
\pgfpathlineto{\pgfqpoint{0.797895in}{0.773588in}}%
\pgfpathclose%
\pgfusepath{fill}%
\end{pgfscope}%
\begin{pgfscope}%
\pgfpathrectangle{\pgfqpoint{0.781402in}{0.773588in}}{\pgfqpoint{1.440244in}{5.415119in}}%
\pgfusepath{clip}%
\pgfsetbuttcap%
\pgfsetroundjoin%
\definecolor{currentfill}{rgb}{0.839216,0.152941,0.156863}%
\pgfsetfillcolor{currentfill}%
\pgfsetlinewidth{0.000000pt}%
\definecolor{currentstroke}{rgb}{0.000000,0.000000,0.000000}%
\pgfsetstrokecolor{currentstroke}%
\pgfsetdash{}{0pt}%
\pgfpathmoveto{\pgfqpoint{0.797895in}{1.281056in}}%
\pgfpathlineto{\pgfqpoint{0.797895in}{0.773588in}}%
\pgfpathlineto{\pgfqpoint{0.842612in}{0.773588in}}%
\pgfpathlineto{\pgfqpoint{0.887244in}{0.773588in}}%
\pgfpathlineto{\pgfqpoint{0.933783in}{0.773588in}}%
\pgfpathlineto{\pgfqpoint{0.978015in}{0.773588in}}%
\pgfpathlineto{\pgfqpoint{1.021908in}{0.773588in}}%
\pgfpathlineto{\pgfqpoint{1.067773in}{0.773588in}}%
\pgfpathlineto{\pgfqpoint{1.111790in}{0.773588in}}%
\pgfpathlineto{\pgfqpoint{1.155171in}{0.773588in}}%
\pgfpathlineto{\pgfqpoint{1.200192in}{0.773588in}}%
\pgfpathlineto{\pgfqpoint{1.244609in}{0.773588in}}%
\pgfpathlineto{\pgfqpoint{1.289074in}{0.773588in}}%
\pgfpathlineto{\pgfqpoint{1.334262in}{0.773588in}}%
\pgfpathlineto{\pgfqpoint{1.378287in}{0.773588in}}%
\pgfpathlineto{\pgfqpoint{1.422957in}{0.773588in}}%
\pgfpathlineto{\pgfqpoint{1.468334in}{0.773588in}}%
\pgfpathlineto{\pgfqpoint{1.511740in}{0.773588in}}%
\pgfpathlineto{\pgfqpoint{1.555718in}{0.773588in}}%
\pgfpathlineto{\pgfqpoint{1.600816in}{0.773588in}}%
\pgfpathlineto{\pgfqpoint{1.645274in}{0.773588in}}%
\pgfpathlineto{\pgfqpoint{1.689326in}{0.773588in}}%
\pgfpathlineto{\pgfqpoint{1.734500in}{0.773588in}}%
\pgfpathlineto{\pgfqpoint{1.778099in}{0.773588in}}%
\pgfpathlineto{\pgfqpoint{1.822804in}{0.773588in}}%
\pgfpathlineto{\pgfqpoint{1.869493in}{0.773588in}}%
\pgfpathlineto{\pgfqpoint{1.915908in}{0.773588in}}%
\pgfpathlineto{\pgfqpoint{1.963476in}{0.773588in}}%
\pgfpathlineto{\pgfqpoint{2.011209in}{0.773588in}}%
\pgfpathlineto{\pgfqpoint{2.057441in}{0.773588in}}%
\pgfpathlineto{\pgfqpoint{2.104325in}{0.773588in}}%
\pgfpathlineto{\pgfqpoint{2.153284in}{0.773588in}}%
\pgfpathlineto{\pgfqpoint{2.201242in}{0.773588in}}%
\pgfpathlineto{\pgfqpoint{2.249804in}{0.773588in}}%
\pgfpathlineto{\pgfqpoint{2.299591in}{0.773588in}}%
\pgfpathlineto{\pgfqpoint{2.348431in}{0.773588in}}%
\pgfpathlineto{\pgfqpoint{2.397147in}{0.773588in}}%
\pgfpathlineto{\pgfqpoint{2.452591in}{0.773588in}}%
\pgfpathlineto{\pgfqpoint{2.505534in}{0.773588in}}%
\pgfpathlineto{\pgfqpoint{2.555498in}{0.773588in}}%
\pgfpathlineto{\pgfqpoint{2.604306in}{0.773588in}}%
\pgfpathlineto{\pgfqpoint{2.651003in}{0.773588in}}%
\pgfpathlineto{\pgfqpoint{2.698034in}{0.773588in}}%
\pgfpathlineto{\pgfqpoint{2.745668in}{0.773588in}}%
\pgfpathlineto{\pgfqpoint{2.790736in}{0.773588in}}%
\pgfpathlineto{\pgfqpoint{2.836381in}{0.773588in}}%
\pgfpathlineto{\pgfqpoint{2.883134in}{0.773588in}}%
\pgfpathlineto{\pgfqpoint{2.927413in}{0.773588in}}%
\pgfpathlineto{\pgfqpoint{2.971917in}{0.773588in}}%
\pgfpathlineto{\pgfqpoint{3.017486in}{0.773588in}}%
\pgfpathlineto{\pgfqpoint{3.062761in}{0.773588in}}%
\pgfpathlineto{\pgfqpoint{3.108180in}{0.773588in}}%
\pgfpathlineto{\pgfqpoint{3.155580in}{0.773588in}}%
\pgfpathlineto{\pgfqpoint{3.200519in}{0.773588in}}%
\pgfpathlineto{\pgfqpoint{3.245458in}{0.773588in}}%
\pgfpathlineto{\pgfqpoint{3.290748in}{0.773588in}}%
\pgfpathlineto{\pgfqpoint{3.336184in}{0.773588in}}%
\pgfpathlineto{\pgfqpoint{3.381274in}{0.773588in}}%
\pgfpathlineto{\pgfqpoint{3.428526in}{0.773588in}}%
\pgfpathlineto{\pgfqpoint{3.473135in}{0.773588in}}%
\pgfpathlineto{\pgfqpoint{3.517815in}{0.773588in}}%
\pgfpathlineto{\pgfqpoint{3.564489in}{0.773588in}}%
\pgfpathlineto{\pgfqpoint{3.609556in}{0.773588in}}%
\pgfpathlineto{\pgfqpoint{3.654909in}{0.773588in}}%
\pgfpathlineto{\pgfqpoint{3.701949in}{0.773588in}}%
\pgfpathlineto{\pgfqpoint{3.747249in}{0.773588in}}%
\pgfpathlineto{\pgfqpoint{3.793040in}{0.773588in}}%
\pgfpathlineto{\pgfqpoint{3.839775in}{0.773588in}}%
\pgfpathlineto{\pgfqpoint{3.884614in}{0.773588in}}%
\pgfpathlineto{\pgfqpoint{3.930307in}{0.773588in}}%
\pgfpathlineto{\pgfqpoint{3.976868in}{0.773588in}}%
\pgfpathlineto{\pgfqpoint{4.022115in}{0.773588in}}%
\pgfpathlineto{\pgfqpoint{4.067507in}{0.773588in}}%
\pgfpathlineto{\pgfqpoint{4.114362in}{0.773588in}}%
\pgfpathlineto{\pgfqpoint{4.159078in}{0.773588in}}%
\pgfpathlineto{\pgfqpoint{4.204543in}{0.773588in}}%
\pgfpathlineto{\pgfqpoint{4.252389in}{0.773588in}}%
\pgfpathlineto{\pgfqpoint{4.297354in}{0.773588in}}%
\pgfpathlineto{\pgfqpoint{4.343118in}{0.773588in}}%
\pgfpathlineto{\pgfqpoint{4.390187in}{0.773588in}}%
\pgfpathlineto{\pgfqpoint{4.435284in}{0.773588in}}%
\pgfpathlineto{\pgfqpoint{4.480684in}{0.773588in}}%
\pgfpathlineto{\pgfqpoint{4.527424in}{0.773588in}}%
\pgfpathlineto{\pgfqpoint{4.572708in}{0.773588in}}%
\pgfpathlineto{\pgfqpoint{4.617918in}{0.773588in}}%
\pgfpathlineto{\pgfqpoint{4.663622in}{0.773588in}}%
\pgfpathlineto{\pgfqpoint{4.708228in}{0.773588in}}%
\pgfpathlineto{\pgfqpoint{4.753449in}{0.773588in}}%
\pgfpathlineto{\pgfqpoint{4.799887in}{0.773588in}}%
\pgfpathlineto{\pgfqpoint{4.845429in}{0.773588in}}%
\pgfpathlineto{\pgfqpoint{4.891298in}{0.773588in}}%
\pgfpathlineto{\pgfqpoint{4.939032in}{0.773588in}}%
\pgfpathlineto{\pgfqpoint{4.984973in}{0.773588in}}%
\pgfpathlineto{\pgfqpoint{5.030678in}{0.773588in}}%
\pgfpathlineto{\pgfqpoint{5.078074in}{0.773588in}}%
\pgfpathlineto{\pgfqpoint{5.124036in}{0.773588in}}%
\pgfpathlineto{\pgfqpoint{5.169192in}{0.773588in}}%
\pgfpathlineto{\pgfqpoint{5.216291in}{0.773588in}}%
\pgfpathlineto{\pgfqpoint{5.262146in}{0.773588in}}%
\pgfpathlineto{\pgfqpoint{5.307244in}{0.773588in}}%
\pgfpathlineto{\pgfqpoint{5.353614in}{0.773588in}}%
\pgfpathlineto{\pgfqpoint{5.399334in}{0.773588in}}%
\pgfpathlineto{\pgfqpoint{5.444526in}{0.773588in}}%
\pgfpathlineto{\pgfqpoint{5.491128in}{0.773588in}}%
\pgfpathlineto{\pgfqpoint{5.536203in}{0.773588in}}%
\pgfpathlineto{\pgfqpoint{5.581803in}{0.773588in}}%
\pgfpathlineto{\pgfqpoint{5.628857in}{0.773588in}}%
\pgfpathlineto{\pgfqpoint{5.674262in}{0.773588in}}%
\pgfpathlineto{\pgfqpoint{5.719569in}{0.773588in}}%
\pgfpathlineto{\pgfqpoint{5.765813in}{0.773588in}}%
\pgfpathlineto{\pgfqpoint{5.810269in}{0.773588in}}%
\pgfpathlineto{\pgfqpoint{5.856181in}{0.773588in}}%
\pgfpathlineto{\pgfqpoint{5.903366in}{0.773588in}}%
\pgfpathlineto{\pgfqpoint{5.948991in}{0.773588in}}%
\pgfpathlineto{\pgfqpoint{5.994686in}{0.773588in}}%
\pgfpathlineto{\pgfqpoint{6.041630in}{0.773588in}}%
\pgfpathlineto{\pgfqpoint{6.087229in}{0.773588in}}%
\pgfpathlineto{\pgfqpoint{6.133235in}{0.773588in}}%
\pgfpathlineto{\pgfqpoint{6.180802in}{0.773588in}}%
\pgfpathlineto{\pgfqpoint{6.227112in}{0.773588in}}%
\pgfpathlineto{\pgfqpoint{6.273584in}{0.773588in}}%
\pgfpathlineto{\pgfqpoint{6.321070in}{0.773588in}}%
\pgfpathlineto{\pgfqpoint{6.367508in}{0.773588in}}%
\pgfpathlineto{\pgfqpoint{6.413399in}{0.773588in}}%
\pgfpathlineto{\pgfqpoint{6.461324in}{0.773588in}}%
\pgfpathlineto{\pgfqpoint{6.507651in}{0.773588in}}%
\pgfpathlineto{\pgfqpoint{6.553117in}{0.773588in}}%
\pgfpathlineto{\pgfqpoint{6.599302in}{0.773588in}}%
\pgfpathlineto{\pgfqpoint{6.643960in}{0.773588in}}%
\pgfpathlineto{\pgfqpoint{6.688504in}{0.773588in}}%
\pgfpathlineto{\pgfqpoint{6.734887in}{0.773588in}}%
\pgfpathlineto{\pgfqpoint{6.779295in}{0.773588in}}%
\pgfpathlineto{\pgfqpoint{6.824012in}{0.773588in}}%
\pgfpathlineto{\pgfqpoint{6.869544in}{0.773588in}}%
\pgfpathlineto{\pgfqpoint{6.914194in}{0.773588in}}%
\pgfpathlineto{\pgfqpoint{6.958763in}{0.773588in}}%
\pgfpathlineto{\pgfqpoint{7.005149in}{0.773588in}}%
\pgfpathlineto{\pgfqpoint{7.050071in}{0.773588in}}%
\pgfpathlineto{\pgfqpoint{7.094205in}{0.773588in}}%
\pgfpathlineto{\pgfqpoint{7.140134in}{0.773588in}}%
\pgfpathlineto{\pgfqpoint{7.184277in}{0.773588in}}%
\pgfpathlineto{\pgfqpoint{7.228120in}{0.773588in}}%
\pgfpathlineto{\pgfqpoint{7.273914in}{0.773588in}}%
\pgfpathlineto{\pgfqpoint{7.318484in}{0.773588in}}%
\pgfpathlineto{\pgfqpoint{7.363749in}{0.773588in}}%
\pgfpathlineto{\pgfqpoint{7.409977in}{0.773588in}}%
\pgfpathlineto{\pgfqpoint{7.455548in}{0.773588in}}%
\pgfpathlineto{\pgfqpoint{7.500683in}{0.773588in}}%
\pgfpathlineto{\pgfqpoint{7.546892in}{0.773588in}}%
\pgfpathlineto{\pgfqpoint{7.591890in}{0.773588in}}%
\pgfpathlineto{\pgfqpoint{7.636592in}{0.773588in}}%
\pgfpathlineto{\pgfqpoint{7.682978in}{0.773588in}}%
\pgfpathlineto{\pgfqpoint{7.728803in}{0.773588in}}%
\pgfpathlineto{\pgfqpoint{7.773226in}{0.773588in}}%
\pgfpathlineto{\pgfqpoint{7.819947in}{0.773588in}}%
\pgfpathlineto{\pgfqpoint{7.865263in}{0.773588in}}%
\pgfpathlineto{\pgfqpoint{7.910161in}{0.773588in}}%
\pgfpathlineto{\pgfqpoint{7.955879in}{0.773588in}}%
\pgfpathlineto{\pgfqpoint{8.000573in}{0.773588in}}%
\pgfpathlineto{\pgfqpoint{8.045278in}{0.773588in}}%
\pgfpathlineto{\pgfqpoint{8.091881in}{0.773588in}}%
\pgfpathlineto{\pgfqpoint{8.136842in}{0.773588in}}%
\pgfpathlineto{\pgfqpoint{8.181791in}{0.773588in}}%
\pgfpathlineto{\pgfqpoint{8.227930in}{0.773588in}}%
\pgfpathlineto{\pgfqpoint{8.273006in}{0.773588in}}%
\pgfpathlineto{\pgfqpoint{8.318789in}{0.773588in}}%
\pgfpathlineto{\pgfqpoint{8.364636in}{0.773588in}}%
\pgfpathlineto{\pgfqpoint{8.409767in}{0.773588in}}%
\pgfpathlineto{\pgfqpoint{8.453780in}{0.773588in}}%
\pgfpathlineto{\pgfqpoint{8.499298in}{0.773588in}}%
\pgfpathlineto{\pgfqpoint{8.543976in}{0.773588in}}%
\pgfpathlineto{\pgfqpoint{8.589124in}{0.773588in}}%
\pgfpathlineto{\pgfqpoint{8.635315in}{0.773588in}}%
\pgfpathlineto{\pgfqpoint{8.681080in}{0.773588in}}%
\pgfpathlineto{\pgfqpoint{8.727404in}{0.773588in}}%
\pgfpathlineto{\pgfqpoint{8.774902in}{0.773588in}}%
\pgfpathlineto{\pgfqpoint{8.821068in}{0.773588in}}%
\pgfpathlineto{\pgfqpoint{8.866242in}{0.773588in}}%
\pgfpathlineto{\pgfqpoint{8.912131in}{0.773588in}}%
\pgfpathlineto{\pgfqpoint{8.957924in}{0.773588in}}%
\pgfpathlineto{\pgfqpoint{9.003947in}{0.773588in}}%
\pgfpathlineto{\pgfqpoint{9.050165in}{0.773588in}}%
\pgfpathlineto{\pgfqpoint{9.095727in}{0.773588in}}%
\pgfpathlineto{\pgfqpoint{9.141270in}{0.773588in}}%
\pgfpathlineto{\pgfqpoint{9.188332in}{0.773588in}}%
\pgfpathlineto{\pgfqpoint{9.233107in}{0.773588in}}%
\pgfpathlineto{\pgfqpoint{9.278439in}{0.773588in}}%
\pgfpathlineto{\pgfqpoint{9.324423in}{0.773588in}}%
\pgfpathlineto{\pgfqpoint{9.369639in}{0.773588in}}%
\pgfpathlineto{\pgfqpoint{9.415178in}{0.773588in}}%
\pgfpathlineto{\pgfqpoint{9.462006in}{0.773588in}}%
\pgfpathlineto{\pgfqpoint{9.507583in}{0.773588in}}%
\pgfpathlineto{\pgfqpoint{9.552197in}{0.773588in}}%
\pgfpathlineto{\pgfqpoint{9.598338in}{0.773588in}}%
\pgfpathlineto{\pgfqpoint{9.644121in}{0.773588in}}%
\pgfpathlineto{\pgfqpoint{9.689497in}{0.773588in}}%
\pgfpathlineto{\pgfqpoint{9.735785in}{0.773588in}}%
\pgfpathlineto{\pgfqpoint{9.781321in}{0.773588in}}%
\pgfpathlineto{\pgfqpoint{9.826599in}{0.773588in}}%
\pgfpathlineto{\pgfqpoint{9.873173in}{0.773588in}}%
\pgfpathlineto{\pgfqpoint{9.918842in}{0.773588in}}%
\pgfpathlineto{\pgfqpoint{9.964562in}{0.773588in}}%
\pgfpathlineto{\pgfqpoint{10.011915in}{0.773588in}}%
\pgfpathlineto{\pgfqpoint{10.057613in}{0.773588in}}%
\pgfpathlineto{\pgfqpoint{10.103098in}{0.773588in}}%
\pgfpathlineto{\pgfqpoint{10.151348in}{0.773588in}}%
\pgfpathlineto{\pgfqpoint{10.197247in}{0.773588in}}%
\pgfpathlineto{\pgfqpoint{10.242601in}{0.773588in}}%
\pgfpathlineto{\pgfqpoint{10.289084in}{0.773588in}}%
\pgfpathlineto{\pgfqpoint{10.335334in}{0.773588in}}%
\pgfpathlineto{\pgfqpoint{10.381570in}{0.773588in}}%
\pgfpathlineto{\pgfqpoint{10.429362in}{0.773588in}}%
\pgfpathlineto{\pgfqpoint{10.475642in}{0.773588in}}%
\pgfpathlineto{\pgfqpoint{10.521105in}{0.773588in}}%
\pgfpathlineto{\pgfqpoint{10.568154in}{0.773588in}}%
\pgfpathlineto{\pgfqpoint{10.613886in}{0.773588in}}%
\pgfpathlineto{\pgfqpoint{10.660112in}{0.773588in}}%
\pgfpathlineto{\pgfqpoint{10.707461in}{0.773588in}}%
\pgfpathlineto{\pgfqpoint{10.753168in}{0.773588in}}%
\pgfpathlineto{\pgfqpoint{10.798768in}{0.773588in}}%
\pgfpathlineto{\pgfqpoint{10.845790in}{0.773588in}}%
\pgfpathlineto{\pgfqpoint{10.891801in}{0.773588in}}%
\pgfpathlineto{\pgfqpoint{10.937381in}{0.773588in}}%
\pgfpathlineto{\pgfqpoint{10.984145in}{0.773588in}}%
\pgfpathlineto{\pgfqpoint{11.029398in}{0.773588in}}%
\pgfpathlineto{\pgfqpoint{11.074633in}{0.773588in}}%
\pgfpathlineto{\pgfqpoint{11.121978in}{0.773588in}}%
\pgfpathlineto{\pgfqpoint{11.167471in}{0.773588in}}%
\pgfpathlineto{\pgfqpoint{11.213355in}{0.773588in}}%
\pgfpathlineto{\pgfqpoint{11.261350in}{0.773588in}}%
\pgfpathlineto{\pgfqpoint{11.307361in}{0.773588in}}%
\pgfpathlineto{\pgfqpoint{11.353756in}{0.773588in}}%
\pgfpathlineto{\pgfqpoint{11.402044in}{0.773588in}}%
\pgfpathlineto{\pgfqpoint{11.448531in}{0.773588in}}%
\pgfpathlineto{\pgfqpoint{11.494874in}{0.773588in}}%
\pgfpathlineto{\pgfqpoint{11.542665in}{0.773588in}}%
\pgfpathlineto{\pgfqpoint{11.588799in}{0.773588in}}%
\pgfpathlineto{\pgfqpoint{11.634589in}{0.773588in}}%
\pgfpathlineto{\pgfqpoint{11.682053in}{0.773588in}}%
\pgfpathlineto{\pgfqpoint{11.727607in}{0.773588in}}%
\pgfpathlineto{\pgfqpoint{11.772894in}{0.773588in}}%
\pgfpathlineto{\pgfqpoint{11.819406in}{0.773588in}}%
\pgfpathlineto{\pgfqpoint{11.864862in}{0.773588in}}%
\pgfpathlineto{\pgfqpoint{11.911501in}{0.773588in}}%
\pgfpathlineto{\pgfqpoint{11.959143in}{0.773588in}}%
\pgfpathlineto{\pgfqpoint{12.004414in}{0.773588in}}%
\pgfpathlineto{\pgfqpoint{12.049869in}{0.773588in}}%
\pgfpathlineto{\pgfqpoint{12.096623in}{0.773588in}}%
\pgfpathlineto{\pgfqpoint{12.141887in}{0.773588in}}%
\pgfpathlineto{\pgfqpoint{12.187925in}{0.773588in}}%
\pgfpathlineto{\pgfqpoint{12.235985in}{0.773588in}}%
\pgfpathlineto{\pgfqpoint{12.282271in}{0.773588in}}%
\pgfpathlineto{\pgfqpoint{12.328136in}{0.773588in}}%
\pgfpathlineto{\pgfqpoint{12.375261in}{0.773588in}}%
\pgfpathlineto{\pgfqpoint{12.421372in}{0.773588in}}%
\pgfpathlineto{\pgfqpoint{12.467453in}{0.773588in}}%
\pgfpathlineto{\pgfqpoint{12.514481in}{0.773588in}}%
\pgfpathlineto{\pgfqpoint{12.560711in}{0.773588in}}%
\pgfpathlineto{\pgfqpoint{12.607615in}{0.773588in}}%
\pgfpathlineto{\pgfqpoint{12.655424in}{0.773588in}}%
\pgfpathlineto{\pgfqpoint{12.701835in}{0.773588in}}%
\pgfpathlineto{\pgfqpoint{12.749025in}{0.773588in}}%
\pgfpathlineto{\pgfqpoint{12.796907in}{0.773588in}}%
\pgfpathlineto{\pgfqpoint{12.843155in}{0.773588in}}%
\pgfpathlineto{\pgfqpoint{12.889988in}{0.773588in}}%
\pgfpathlineto{\pgfqpoint{12.936799in}{0.773588in}}%
\pgfpathlineto{\pgfqpoint{12.983596in}{0.773588in}}%
\pgfpathlineto{\pgfqpoint{13.030499in}{0.773588in}}%
\pgfpathlineto{\pgfqpoint{13.077275in}{0.773588in}}%
\pgfpathlineto{\pgfqpoint{13.123339in}{0.773588in}}%
\pgfpathlineto{\pgfqpoint{13.169075in}{0.773588in}}%
\pgfpathlineto{\pgfqpoint{13.216719in}{0.773588in}}%
\pgfpathlineto{\pgfqpoint{13.262980in}{0.773588in}}%
\pgfpathlineto{\pgfqpoint{13.308974in}{0.773588in}}%
\pgfpathlineto{\pgfqpoint{13.356518in}{0.773588in}}%
\pgfpathlineto{\pgfqpoint{13.401859in}{0.773588in}}%
\pgfpathlineto{\pgfqpoint{13.448069in}{0.773588in}}%
\pgfpathlineto{\pgfqpoint{13.495727in}{0.773588in}}%
\pgfpathlineto{\pgfqpoint{13.541646in}{0.773588in}}%
\pgfpathlineto{\pgfqpoint{13.587249in}{0.773588in}}%
\pgfpathlineto{\pgfqpoint{13.635346in}{0.773588in}}%
\pgfpathlineto{\pgfqpoint{13.682151in}{0.773588in}}%
\pgfpathlineto{\pgfqpoint{13.729143in}{0.773588in}}%
\pgfpathlineto{\pgfqpoint{13.777441in}{0.773588in}}%
\pgfpathlineto{\pgfqpoint{13.825297in}{0.773588in}}%
\pgfpathlineto{\pgfqpoint{13.873610in}{0.773588in}}%
\pgfpathlineto{\pgfqpoint{13.922606in}{0.773588in}}%
\pgfpathlineto{\pgfqpoint{13.969353in}{0.773588in}}%
\pgfpathlineto{\pgfqpoint{14.016344in}{0.773588in}}%
\pgfpathlineto{\pgfqpoint{14.064951in}{0.773588in}}%
\pgfpathlineto{\pgfqpoint{14.112147in}{0.773588in}}%
\pgfpathlineto{\pgfqpoint{14.160147in}{0.773588in}}%
\pgfpathlineto{\pgfqpoint{14.209376in}{0.773588in}}%
\pgfpathlineto{\pgfqpoint{14.255593in}{0.773588in}}%
\pgfpathlineto{\pgfqpoint{14.302418in}{0.773588in}}%
\pgfpathlineto{\pgfqpoint{14.350224in}{0.773588in}}%
\pgfpathlineto{\pgfqpoint{14.397126in}{0.773588in}}%
\pgfpathlineto{\pgfqpoint{14.443729in}{0.773588in}}%
\pgfpathlineto{\pgfqpoint{14.491726in}{0.773588in}}%
\pgfpathlineto{\pgfqpoint{14.538630in}{0.773588in}}%
\pgfpathlineto{\pgfqpoint{14.585477in}{0.773588in}}%
\pgfpathlineto{\pgfqpoint{14.633992in}{0.773588in}}%
\pgfpathlineto{\pgfqpoint{14.680226in}{0.773588in}}%
\pgfpathlineto{\pgfqpoint{14.726757in}{0.773588in}}%
\pgfpathlineto{\pgfqpoint{14.774752in}{0.773588in}}%
\pgfpathlineto{\pgfqpoint{14.821474in}{0.773588in}}%
\pgfpathlineto{\pgfqpoint{14.868242in}{0.773588in}}%
\pgfpathlineto{\pgfqpoint{14.916239in}{0.773588in}}%
\pgfpathlineto{\pgfqpoint{14.963167in}{0.773588in}}%
\pgfpathlineto{\pgfqpoint{15.010214in}{0.773588in}}%
\pgfpathlineto{\pgfqpoint{15.058774in}{0.773588in}}%
\pgfpathlineto{\pgfqpoint{15.105478in}{0.773588in}}%
\pgfpathlineto{\pgfqpoint{15.152658in}{0.773588in}}%
\pgfpathlineto{\pgfqpoint{15.201456in}{0.773588in}}%
\pgfpathlineto{\pgfqpoint{15.248863in}{0.773588in}}%
\pgfpathlineto{\pgfqpoint{15.295866in}{0.773588in}}%
\pgfpathlineto{\pgfqpoint{15.343883in}{0.773588in}}%
\pgfpathlineto{\pgfqpoint{15.389896in}{0.773588in}}%
\pgfpathlineto{\pgfqpoint{15.436052in}{0.773588in}}%
\pgfpathlineto{\pgfqpoint{15.484858in}{0.773588in}}%
\pgfpathlineto{\pgfqpoint{15.532329in}{0.773588in}}%
\pgfpathlineto{\pgfqpoint{15.579815in}{0.773588in}}%
\pgfpathlineto{\pgfqpoint{15.628660in}{0.773588in}}%
\pgfpathlineto{\pgfqpoint{15.676019in}{0.773588in}}%
\pgfpathlineto{\pgfqpoint{15.722814in}{0.773588in}}%
\pgfpathlineto{\pgfqpoint{15.770154in}{0.773588in}}%
\pgfpathlineto{\pgfqpoint{15.817273in}{0.773588in}}%
\pgfpathlineto{\pgfqpoint{15.863592in}{0.773588in}}%
\pgfpathlineto{\pgfqpoint{15.911811in}{0.773588in}}%
\pgfpathlineto{\pgfqpoint{15.957799in}{0.773588in}}%
\pgfpathlineto{\pgfqpoint{16.003926in}{0.773588in}}%
\pgfpathlineto{\pgfqpoint{16.051820in}{0.773588in}}%
\pgfpathlineto{\pgfqpoint{16.098697in}{0.773588in}}%
\pgfpathlineto{\pgfqpoint{16.146338in}{0.773588in}}%
\pgfpathlineto{\pgfqpoint{16.195175in}{0.773588in}}%
\pgfpathlineto{\pgfqpoint{16.242078in}{0.773588in}}%
\pgfpathlineto{\pgfqpoint{16.289817in}{0.773588in}}%
\pgfpathlineto{\pgfqpoint{16.339537in}{0.773588in}}%
\pgfpathlineto{\pgfqpoint{16.387666in}{0.773588in}}%
\pgfpathlineto{\pgfqpoint{16.436622in}{0.773588in}}%
\pgfpathlineto{\pgfqpoint{16.486560in}{0.773588in}}%
\pgfpathlineto{\pgfqpoint{16.533619in}{0.773588in}}%
\pgfpathlineto{\pgfqpoint{16.581135in}{0.773588in}}%
\pgfpathlineto{\pgfqpoint{16.630280in}{0.773588in}}%
\pgfpathlineto{\pgfqpoint{16.677566in}{0.773588in}}%
\pgfpathlineto{\pgfqpoint{16.724568in}{0.773588in}}%
\pgfpathlineto{\pgfqpoint{16.773713in}{0.773588in}}%
\pgfpathlineto{\pgfqpoint{16.821038in}{0.773588in}}%
\pgfpathlineto{\pgfqpoint{16.868266in}{0.773588in}}%
\pgfpathlineto{\pgfqpoint{16.916302in}{0.773588in}}%
\pgfpathlineto{\pgfqpoint{16.963533in}{0.773588in}}%
\pgfpathlineto{\pgfqpoint{17.010585in}{0.773588in}}%
\pgfpathlineto{\pgfqpoint{17.059518in}{0.773588in}}%
\pgfpathlineto{\pgfqpoint{17.107470in}{0.773588in}}%
\pgfpathlineto{\pgfqpoint{17.154507in}{0.773588in}}%
\pgfpathlineto{\pgfqpoint{17.203674in}{0.773588in}}%
\pgfpathlineto{\pgfqpoint{17.250860in}{0.773588in}}%
\pgfpathlineto{\pgfqpoint{17.298941in}{0.773588in}}%
\pgfpathlineto{\pgfqpoint{17.347576in}{0.773588in}}%
\pgfpathlineto{\pgfqpoint{17.394714in}{0.773588in}}%
\pgfpathlineto{\pgfqpoint{17.442362in}{0.773588in}}%
\pgfpathlineto{\pgfqpoint{17.491125in}{0.773588in}}%
\pgfpathlineto{\pgfqpoint{17.538409in}{0.773588in}}%
\pgfpathlineto{\pgfqpoint{17.585742in}{0.773588in}}%
\pgfpathlineto{\pgfqpoint{17.634653in}{0.773588in}}%
\pgfpathlineto{\pgfqpoint{17.681914in}{0.773588in}}%
\pgfpathlineto{\pgfqpoint{17.729727in}{0.773588in}}%
\pgfpathlineto{\pgfqpoint{17.779014in}{0.773588in}}%
\pgfpathlineto{\pgfqpoint{17.826809in}{0.773588in}}%
\pgfpathlineto{\pgfqpoint{17.874600in}{0.773588in}}%
\pgfpathlineto{\pgfqpoint{17.922885in}{0.773588in}}%
\pgfpathlineto{\pgfqpoint{17.970910in}{0.773588in}}%
\pgfpathlineto{\pgfqpoint{18.020026in}{0.773588in}}%
\pgfpathlineto{\pgfqpoint{18.069524in}{0.773588in}}%
\pgfpathlineto{\pgfqpoint{18.117307in}{0.773588in}}%
\pgfpathlineto{\pgfqpoint{18.164405in}{0.773588in}}%
\pgfpathlineto{\pgfqpoint{18.213333in}{0.773588in}}%
\pgfpathlineto{\pgfqpoint{18.260871in}{0.773588in}}%
\pgfpathlineto{\pgfqpoint{18.308441in}{0.773588in}}%
\pgfpathlineto{\pgfqpoint{18.357538in}{0.773588in}}%
\pgfpathlineto{\pgfqpoint{18.405242in}{0.773588in}}%
\pgfpathlineto{\pgfqpoint{18.452284in}{0.773588in}}%
\pgfpathlineto{\pgfqpoint{18.500915in}{0.773588in}}%
\pgfpathlineto{\pgfqpoint{18.548460in}{0.773588in}}%
\pgfpathlineto{\pgfqpoint{18.595724in}{0.773588in}}%
\pgfpathlineto{\pgfqpoint{18.644639in}{0.773588in}}%
\pgfpathlineto{\pgfqpoint{18.693122in}{0.773588in}}%
\pgfpathlineto{\pgfqpoint{18.741665in}{0.773588in}}%
\pgfpathlineto{\pgfqpoint{18.791212in}{0.773588in}}%
\pgfpathlineto{\pgfqpoint{18.839312in}{0.773588in}}%
\pgfpathlineto{\pgfqpoint{18.888188in}{0.773588in}}%
\pgfpathlineto{\pgfqpoint{18.937671in}{0.773588in}}%
\pgfpathlineto{\pgfqpoint{18.985002in}{0.773588in}}%
\pgfpathlineto{\pgfqpoint{19.033794in}{0.773588in}}%
\pgfpathlineto{\pgfqpoint{19.084207in}{0.773588in}}%
\pgfpathlineto{\pgfqpoint{19.132724in}{0.773588in}}%
\pgfpathlineto{\pgfqpoint{19.181302in}{0.773588in}}%
\pgfpathlineto{\pgfqpoint{19.231826in}{0.773588in}}%
\pgfpathlineto{\pgfqpoint{19.279742in}{0.773588in}}%
\pgfpathlineto{\pgfqpoint{19.327829in}{0.773588in}}%
\pgfpathlineto{\pgfqpoint{19.376890in}{0.773588in}}%
\pgfpathlineto{\pgfqpoint{19.425178in}{0.773588in}}%
\pgfpathlineto{\pgfqpoint{19.473154in}{0.773588in}}%
\pgfpathlineto{\pgfqpoint{19.522292in}{0.773588in}}%
\pgfpathlineto{\pgfqpoint{19.569747in}{0.773588in}}%
\pgfpathlineto{\pgfqpoint{19.617707in}{0.773588in}}%
\pgfpathlineto{\pgfqpoint{19.666940in}{0.773588in}}%
\pgfpathlineto{\pgfqpoint{19.714265in}{0.773588in}}%
\pgfpathlineto{\pgfqpoint{19.761678in}{0.773588in}}%
\pgfpathlineto{\pgfqpoint{19.810712in}{0.773588in}}%
\pgfpathlineto{\pgfqpoint{19.858933in}{0.773588in}}%
\pgfpathlineto{\pgfqpoint{19.906615in}{0.773588in}}%
\pgfpathlineto{\pgfqpoint{19.955252in}{0.773588in}}%
\pgfpathlineto{\pgfqpoint{20.003692in}{0.773588in}}%
\pgfpathlineto{\pgfqpoint{20.053312in}{0.773588in}}%
\pgfpathlineto{\pgfqpoint{20.104350in}{0.773588in}}%
\pgfpathlineto{\pgfqpoint{20.153679in}{0.773588in}}%
\pgfpathlineto{\pgfqpoint{20.203615in}{0.773588in}}%
\pgfpathlineto{\pgfqpoint{20.255593in}{0.773588in}}%
\pgfpathlineto{\pgfqpoint{20.305924in}{0.773588in}}%
\pgfpathlineto{\pgfqpoint{20.355462in}{0.773588in}}%
\pgfpathlineto{\pgfqpoint{20.406414in}{0.773588in}}%
\pgfpathlineto{\pgfqpoint{20.455604in}{0.773588in}}%
\pgfpathlineto{\pgfqpoint{20.505208in}{0.773588in}}%
\pgfpathlineto{\pgfqpoint{20.556290in}{0.773588in}}%
\pgfpathlineto{\pgfqpoint{20.605567in}{0.773588in}}%
\pgfpathlineto{\pgfqpoint{20.654668in}{0.773588in}}%
\pgfpathlineto{\pgfqpoint{20.704616in}{0.773588in}}%
\pgfpathlineto{\pgfqpoint{20.753558in}{0.773588in}}%
\pgfpathlineto{\pgfqpoint{20.802965in}{0.773588in}}%
\pgfpathlineto{\pgfqpoint{20.854118in}{0.773588in}}%
\pgfpathlineto{\pgfqpoint{20.904164in}{0.773588in}}%
\pgfpathlineto{\pgfqpoint{20.953585in}{0.773588in}}%
\pgfpathlineto{\pgfqpoint{21.003693in}{0.773588in}}%
\pgfpathlineto{\pgfqpoint{21.053515in}{0.773588in}}%
\pgfpathlineto{\pgfqpoint{21.102370in}{0.773588in}}%
\pgfpathlineto{\pgfqpoint{21.152719in}{0.773588in}}%
\pgfpathlineto{\pgfqpoint{21.201683in}{0.773588in}}%
\pgfpathlineto{\pgfqpoint{21.250655in}{0.773588in}}%
\pgfpathlineto{\pgfqpoint{21.300955in}{0.773588in}}%
\pgfpathlineto{\pgfqpoint{21.350353in}{0.773588in}}%
\pgfpathlineto{\pgfqpoint{21.399733in}{0.773588in}}%
\pgfpathlineto{\pgfqpoint{21.450830in}{0.773588in}}%
\pgfpathlineto{\pgfqpoint{21.501023in}{0.773588in}}%
\pgfpathlineto{\pgfqpoint{21.550957in}{0.773588in}}%
\pgfpathlineto{\pgfqpoint{21.602538in}{0.773588in}}%
\pgfpathlineto{\pgfqpoint{21.652276in}{0.773588in}}%
\pgfpathlineto{\pgfqpoint{21.700864in}{0.773588in}}%
\pgfpathlineto{\pgfqpoint{21.751624in}{0.773588in}}%
\pgfpathlineto{\pgfqpoint{21.800474in}{0.773588in}}%
\pgfpathlineto{\pgfqpoint{21.849580in}{0.773588in}}%
\pgfpathlineto{\pgfqpoint{21.900834in}{0.773588in}}%
\pgfpathlineto{\pgfqpoint{21.950544in}{0.773588in}}%
\pgfpathlineto{\pgfqpoint{21.999803in}{0.773588in}}%
\pgfpathlineto{\pgfqpoint{22.050354in}{0.773588in}}%
\pgfpathlineto{\pgfqpoint{22.099907in}{0.773588in}}%
\pgfpathlineto{\pgfqpoint{22.148817in}{0.773588in}}%
\pgfpathlineto{\pgfqpoint{22.200587in}{0.773588in}}%
\pgfpathlineto{\pgfqpoint{22.250634in}{0.773588in}}%
\pgfpathlineto{\pgfqpoint{22.300933in}{0.773588in}}%
\pgfpathlineto{\pgfqpoint{22.352827in}{0.773588in}}%
\pgfpathlineto{\pgfqpoint{22.402084in}{0.773588in}}%
\pgfpathlineto{\pgfqpoint{22.451976in}{0.773588in}}%
\pgfpathlineto{\pgfqpoint{22.503552in}{0.773588in}}%
\pgfpathlineto{\pgfqpoint{22.554073in}{0.773588in}}%
\pgfpathlineto{\pgfqpoint{22.603657in}{0.773588in}}%
\pgfpathlineto{\pgfqpoint{22.654822in}{0.773588in}}%
\pgfpathlineto{\pgfqpoint{22.703398in}{0.773588in}}%
\pgfpathlineto{\pgfqpoint{22.751700in}{0.773588in}}%
\pgfpathlineto{\pgfqpoint{22.803829in}{0.773588in}}%
\pgfpathlineto{\pgfqpoint{22.853966in}{0.773588in}}%
\pgfpathlineto{\pgfqpoint{22.904236in}{0.773588in}}%
\pgfpathlineto{\pgfqpoint{22.956359in}{0.773588in}}%
\pgfpathlineto{\pgfqpoint{23.006022in}{0.773588in}}%
\pgfpathlineto{\pgfqpoint{23.055187in}{0.773588in}}%
\pgfpathlineto{\pgfqpoint{23.107134in}{0.773588in}}%
\pgfpathlineto{\pgfqpoint{23.157756in}{0.773588in}}%
\pgfpathlineto{\pgfqpoint{23.208694in}{0.773588in}}%
\pgfpathlineto{\pgfqpoint{23.260726in}{0.773588in}}%
\pgfpathlineto{\pgfqpoint{23.310918in}{0.773588in}}%
\pgfpathlineto{\pgfqpoint{23.361310in}{0.773588in}}%
\pgfpathlineto{\pgfqpoint{23.411909in}{0.773588in}}%
\pgfpathlineto{\pgfqpoint{23.462017in}{0.773588in}}%
\pgfpathlineto{\pgfqpoint{23.511867in}{0.773588in}}%
\pgfpathlineto{\pgfqpoint{23.563429in}{0.773588in}}%
\pgfpathlineto{\pgfqpoint{23.611923in}{0.773588in}}%
\pgfpathlineto{\pgfqpoint{23.661575in}{0.773588in}}%
\pgfpathlineto{\pgfqpoint{23.713074in}{0.773588in}}%
\pgfpathlineto{\pgfqpoint{23.762329in}{0.773588in}}%
\pgfpathlineto{\pgfqpoint{23.811512in}{0.773588in}}%
\pgfpathlineto{\pgfqpoint{23.862673in}{0.773588in}}%
\pgfpathlineto{\pgfqpoint{23.913088in}{0.773588in}}%
\pgfpathlineto{\pgfqpoint{23.964514in}{0.773588in}}%
\pgfpathlineto{\pgfqpoint{24.016442in}{0.773588in}}%
\pgfpathlineto{\pgfqpoint{24.066933in}{0.773588in}}%
\pgfpathlineto{\pgfqpoint{24.118637in}{0.773588in}}%
\pgfpathlineto{\pgfqpoint{24.170873in}{0.773588in}}%
\pgfpathlineto{\pgfqpoint{24.221010in}{0.773588in}}%
\pgfpathlineto{\pgfqpoint{24.271277in}{0.773588in}}%
\pgfpathlineto{\pgfqpoint{24.321907in}{0.773588in}}%
\pgfpathlineto{\pgfqpoint{24.371890in}{0.773588in}}%
\pgfpathlineto{\pgfqpoint{24.422475in}{0.773588in}}%
\pgfpathlineto{\pgfqpoint{24.474381in}{0.773588in}}%
\pgfpathlineto{\pgfqpoint{24.524258in}{0.773588in}}%
\pgfpathlineto{\pgfqpoint{24.573998in}{0.773588in}}%
\pgfpathlineto{\pgfqpoint{24.626556in}{0.773588in}}%
\pgfpathlineto{\pgfqpoint{24.676389in}{0.773588in}}%
\pgfpathlineto{\pgfqpoint{24.726499in}{0.773588in}}%
\pgfpathlineto{\pgfqpoint{24.778304in}{0.773588in}}%
\pgfpathlineto{\pgfqpoint{24.828140in}{0.773588in}}%
\pgfpathlineto{\pgfqpoint{24.878143in}{0.773588in}}%
\pgfpathlineto{\pgfqpoint{24.928869in}{0.773588in}}%
\pgfpathlineto{\pgfqpoint{24.978211in}{0.773588in}}%
\pgfpathlineto{\pgfqpoint{25.028630in}{0.773588in}}%
\pgfpathlineto{\pgfqpoint{25.080539in}{0.773588in}}%
\pgfpathlineto{\pgfqpoint{25.131043in}{0.773588in}}%
\pgfpathlineto{\pgfqpoint{25.181554in}{0.773588in}}%
\pgfpathlineto{\pgfqpoint{25.232944in}{0.773588in}}%
\pgfpathlineto{\pgfqpoint{25.282222in}{0.773588in}}%
\pgfpathlineto{\pgfqpoint{25.332608in}{0.773588in}}%
\pgfpathlineto{\pgfqpoint{25.383622in}{0.773588in}}%
\pgfpathlineto{\pgfqpoint{25.433954in}{0.773588in}}%
\pgfpathlineto{\pgfqpoint{25.483381in}{0.773588in}}%
\pgfpathlineto{\pgfqpoint{25.535558in}{0.773588in}}%
\pgfpathlineto{\pgfqpoint{25.586387in}{0.773588in}}%
\pgfpathlineto{\pgfqpoint{25.637513in}{0.773588in}}%
\pgfpathlineto{\pgfqpoint{25.689152in}{0.773588in}}%
\pgfpathlineto{\pgfqpoint{25.740073in}{0.773588in}}%
\pgfpathlineto{\pgfqpoint{25.790287in}{0.773588in}}%
\pgfpathlineto{\pgfqpoint{25.841969in}{0.773588in}}%
\pgfpathlineto{\pgfqpoint{25.891991in}{0.773588in}}%
\pgfpathlineto{\pgfqpoint{25.942005in}{0.773588in}}%
\pgfpathlineto{\pgfqpoint{25.994130in}{0.773588in}}%
\pgfpathlineto{\pgfqpoint{26.044216in}{0.773588in}}%
\pgfpathlineto{\pgfqpoint{26.093895in}{0.773588in}}%
\pgfpathlineto{\pgfqpoint{26.144837in}{0.773588in}}%
\pgfpathlineto{\pgfqpoint{26.195263in}{0.773588in}}%
\pgfpathlineto{\pgfqpoint{26.245717in}{0.773588in}}%
\pgfpathlineto{\pgfqpoint{26.297518in}{0.773588in}}%
\pgfpathlineto{\pgfqpoint{26.346588in}{0.773588in}}%
\pgfpathlineto{\pgfqpoint{26.395829in}{0.773588in}}%
\pgfpathlineto{\pgfqpoint{26.447655in}{0.773588in}}%
\pgfpathlineto{\pgfqpoint{26.499064in}{0.773588in}}%
\pgfpathlineto{\pgfqpoint{26.549446in}{0.773588in}}%
\pgfpathlineto{\pgfqpoint{26.601868in}{0.773588in}}%
\pgfpathlineto{\pgfqpoint{26.653161in}{0.773588in}}%
\pgfpathlineto{\pgfqpoint{26.704176in}{0.773588in}}%
\pgfpathlineto{\pgfqpoint{26.756448in}{0.773588in}}%
\pgfpathlineto{\pgfqpoint{26.807173in}{0.773588in}}%
\pgfpathlineto{\pgfqpoint{26.857706in}{0.773588in}}%
\pgfpathlineto{\pgfqpoint{26.909530in}{0.773588in}}%
\pgfpathlineto{\pgfqpoint{26.960748in}{0.773588in}}%
\pgfpathlineto{\pgfqpoint{27.010539in}{0.773588in}}%
\pgfpathlineto{\pgfqpoint{27.063062in}{0.773588in}}%
\pgfpathlineto{\pgfqpoint{27.113950in}{0.773588in}}%
\pgfpathlineto{\pgfqpoint{27.165274in}{0.773588in}}%
\pgfpathlineto{\pgfqpoint{27.218587in}{0.773588in}}%
\pgfpathlineto{\pgfqpoint{27.269679in}{0.773588in}}%
\pgfpathlineto{\pgfqpoint{27.321160in}{0.773588in}}%
\pgfpathlineto{\pgfqpoint{27.372863in}{0.773588in}}%
\pgfpathlineto{\pgfqpoint{27.423066in}{0.773588in}}%
\pgfpathlineto{\pgfqpoint{27.473173in}{0.773588in}}%
\pgfpathlineto{\pgfqpoint{27.525849in}{0.773588in}}%
\pgfpathlineto{\pgfqpoint{27.576178in}{0.773588in}}%
\pgfpathlineto{\pgfqpoint{27.626352in}{0.773588in}}%
\pgfpathlineto{\pgfqpoint{27.678294in}{0.773588in}}%
\pgfpathlineto{\pgfqpoint{27.728571in}{0.773588in}}%
\pgfpathlineto{\pgfqpoint{27.779331in}{0.773588in}}%
\pgfpathlineto{\pgfqpoint{27.833067in}{0.773588in}}%
\pgfpathlineto{\pgfqpoint{27.883786in}{0.773588in}}%
\pgfpathlineto{\pgfqpoint{27.935127in}{0.773588in}}%
\pgfpathlineto{\pgfqpoint{27.988562in}{0.773588in}}%
\pgfpathlineto{\pgfqpoint{28.039571in}{0.773588in}}%
\pgfpathlineto{\pgfqpoint{28.090650in}{0.773588in}}%
\pgfpathlineto{\pgfqpoint{28.143944in}{0.773588in}}%
\pgfpathlineto{\pgfqpoint{28.195034in}{0.773588in}}%
\pgfpathlineto{\pgfqpoint{28.245320in}{0.773588in}}%
\pgfpathlineto{\pgfqpoint{28.297462in}{0.773588in}}%
\pgfpathlineto{\pgfqpoint{28.347853in}{0.773588in}}%
\pgfpathlineto{\pgfqpoint{28.399138in}{0.773588in}}%
\pgfpathlineto{\pgfqpoint{28.451952in}{0.773588in}}%
\pgfpathlineto{\pgfqpoint{28.502707in}{0.773588in}}%
\pgfpathlineto{\pgfqpoint{28.553749in}{0.773588in}}%
\pgfpathlineto{\pgfqpoint{28.605351in}{0.773588in}}%
\pgfpathlineto{\pgfqpoint{28.656185in}{0.773588in}}%
\pgfpathlineto{\pgfqpoint{28.706300in}{0.773588in}}%
\pgfpathlineto{\pgfqpoint{28.758074in}{0.773588in}}%
\pgfpathlineto{\pgfqpoint{28.808587in}{0.773588in}}%
\pgfpathlineto{\pgfqpoint{28.860347in}{0.773588in}}%
\pgfpathlineto{\pgfqpoint{28.913538in}{0.773588in}}%
\pgfpathlineto{\pgfqpoint{28.965063in}{0.773588in}}%
\pgfpathlineto{\pgfqpoint{29.015651in}{0.773588in}}%
\pgfpathlineto{\pgfqpoint{29.068863in}{0.773588in}}%
\pgfpathlineto{\pgfqpoint{29.121029in}{0.773588in}}%
\pgfpathlineto{\pgfqpoint{29.173700in}{0.773588in}}%
\pgfpathlineto{\pgfqpoint{29.226818in}{0.773588in}}%
\pgfpathlineto{\pgfqpoint{29.279399in}{0.773588in}}%
\pgfpathlineto{\pgfqpoint{29.332261in}{0.773588in}}%
\pgfpathlineto{\pgfqpoint{29.386173in}{0.773588in}}%
\pgfpathlineto{\pgfqpoint{29.438335in}{0.773588in}}%
\pgfpathlineto{\pgfqpoint{29.490075in}{0.773588in}}%
\pgfpathlineto{\pgfqpoint{29.543760in}{0.773588in}}%
\pgfpathlineto{\pgfqpoint{29.596078in}{0.773588in}}%
\pgfpathlineto{\pgfqpoint{29.647988in}{0.773588in}}%
\pgfpathlineto{\pgfqpoint{29.701409in}{0.773588in}}%
\pgfpathlineto{\pgfqpoint{29.753093in}{0.773588in}}%
\pgfpathlineto{\pgfqpoint{29.805208in}{0.773588in}}%
\pgfpathlineto{\pgfqpoint{29.858913in}{0.773588in}}%
\pgfpathlineto{\pgfqpoint{29.910558in}{0.773588in}}%
\pgfpathlineto{\pgfqpoint{29.962066in}{0.773588in}}%
\pgfpathlineto{\pgfqpoint{30.014968in}{0.773588in}}%
\pgfpathlineto{\pgfqpoint{30.066247in}{0.773588in}}%
\pgfpathlineto{\pgfqpoint{30.117082in}{0.773588in}}%
\pgfpathlineto{\pgfqpoint{30.169497in}{0.773588in}}%
\pgfpathlineto{\pgfqpoint{30.221783in}{0.773588in}}%
\pgfpathlineto{\pgfqpoint{30.276704in}{0.773588in}}%
\pgfpathlineto{\pgfqpoint{30.337041in}{0.773588in}}%
\pgfpathlineto{\pgfqpoint{30.397563in}{0.773588in}}%
\pgfpathlineto{\pgfqpoint{30.457369in}{0.773588in}}%
\pgfpathlineto{\pgfqpoint{30.519963in}{0.773588in}}%
\pgfpathlineto{\pgfqpoint{30.582999in}{0.773588in}}%
\pgfpathlineto{\pgfqpoint{30.648377in}{0.773588in}}%
\pgfpathlineto{\pgfqpoint{30.717993in}{0.773588in}}%
\pgfpathlineto{\pgfqpoint{30.786068in}{0.773588in}}%
\pgfpathlineto{\pgfqpoint{30.855684in}{0.773588in}}%
\pgfpathlineto{\pgfqpoint{30.928370in}{0.773588in}}%
\pgfpathlineto{\pgfqpoint{31.000738in}{0.773588in}}%
\pgfpathlineto{\pgfqpoint{31.073353in}{0.773588in}}%
\pgfpathlineto{\pgfqpoint{31.150292in}{0.773588in}}%
\pgfpathlineto{\pgfqpoint{31.225476in}{0.773588in}}%
\pgfpathlineto{\pgfqpoint{31.304546in}{0.773588in}}%
\pgfpathlineto{\pgfqpoint{31.385546in}{0.773588in}}%
\pgfpathlineto{\pgfqpoint{31.463397in}{0.773588in}}%
\pgfpathlineto{\pgfqpoint{31.543273in}{0.773588in}}%
\pgfpathlineto{\pgfqpoint{31.629527in}{0.773588in}}%
\pgfpathlineto{\pgfqpoint{31.715049in}{0.773588in}}%
\pgfpathlineto{\pgfqpoint{31.802724in}{0.773588in}}%
\pgfpathlineto{\pgfqpoint{31.890397in}{0.773588in}}%
\pgfpathlineto{\pgfqpoint{31.975639in}{0.773588in}}%
\pgfpathlineto{\pgfqpoint{32.062211in}{0.773588in}}%
\pgfpathlineto{\pgfqpoint{32.153206in}{0.773588in}}%
\pgfpathlineto{\pgfqpoint{32.244846in}{0.773588in}}%
\pgfpathlineto{\pgfqpoint{32.334387in}{0.773588in}}%
\pgfpathlineto{\pgfqpoint{32.429213in}{0.773588in}}%
\pgfpathlineto{\pgfqpoint{32.518604in}{0.773588in}}%
\pgfpathlineto{\pgfqpoint{32.584565in}{0.773588in}}%
\pgfpathlineto{\pgfqpoint{32.638735in}{0.773588in}}%
\pgfpathlineto{\pgfqpoint{32.691213in}{0.773588in}}%
\pgfpathlineto{\pgfqpoint{32.743153in}{0.773588in}}%
\pgfpathlineto{\pgfqpoint{32.797230in}{0.773588in}}%
\pgfpathlineto{\pgfqpoint{32.850576in}{0.773588in}}%
\pgfpathlineto{\pgfqpoint{32.903667in}{0.773588in}}%
\pgfpathlineto{\pgfqpoint{32.957868in}{0.773588in}}%
\pgfpathlineto{\pgfqpoint{33.010153in}{0.773588in}}%
\pgfpathlineto{\pgfqpoint{33.062733in}{0.773588in}}%
\pgfpathlineto{\pgfqpoint{33.115518in}{0.773588in}}%
\pgfpathlineto{\pgfqpoint{33.167699in}{0.773588in}}%
\pgfpathlineto{\pgfqpoint{33.219830in}{0.773588in}}%
\pgfpathlineto{\pgfqpoint{33.273785in}{0.773588in}}%
\pgfpathlineto{\pgfqpoint{33.326652in}{0.773588in}}%
\pgfpathlineto{\pgfqpoint{33.379482in}{0.773588in}}%
\pgfpathlineto{\pgfqpoint{33.433283in}{0.773588in}}%
\pgfpathlineto{\pgfqpoint{33.484775in}{0.773588in}}%
\pgfpathlineto{\pgfqpoint{33.536879in}{0.773588in}}%
\pgfpathlineto{\pgfqpoint{33.590505in}{0.773588in}}%
\pgfpathlineto{\pgfqpoint{33.642373in}{0.773588in}}%
\pgfpathlineto{\pgfqpoint{33.693680in}{0.773588in}}%
\pgfpathlineto{\pgfqpoint{33.746200in}{0.773588in}}%
\pgfpathlineto{\pgfqpoint{33.784450in}{0.773588in}}%
\pgfpathlineto{\pgfqpoint{33.832264in}{0.773588in}}%
\pgfpathlineto{\pgfqpoint{33.870742in}{1.454760in}}%
\pgfpathlineto{\pgfqpoint{33.912991in}{1.669618in}}%
\pgfpathlineto{\pgfqpoint{33.952831in}{1.906942in}}%
\pgfpathlineto{\pgfqpoint{33.989641in}{2.224563in}}%
\pgfpathlineto{\pgfqpoint{34.021637in}{2.864879in}}%
\pgfpathlineto{\pgfqpoint{34.052161in}{3.729137in}}%
\pgfpathlineto{\pgfqpoint{34.076513in}{5.235025in}}%
\pgfpathlineto{\pgfqpoint{34.101659in}{5.182737in}}%
\pgfpathlineto{\pgfqpoint{34.125436in}{5.364631in}}%
\pgfpathlineto{\pgfqpoint{34.150180in}{5.483715in}}%
\pgfpathlineto{\pgfqpoint{34.173862in}{5.446238in}}%
\pgfpathlineto{\pgfqpoint{34.197440in}{5.442565in}}%
\pgfpathlineto{\pgfqpoint{34.222832in}{5.484292in}}%
\pgfpathlineto{\pgfqpoint{34.246364in}{5.534473in}}%
\pgfpathlineto{\pgfqpoint{34.270997in}{5.460686in}}%
\pgfpathlineto{\pgfqpoint{34.293786in}{5.518952in}}%
\pgfpathlineto{\pgfqpoint{34.318035in}{5.678338in}}%
\pgfpathlineto{\pgfqpoint{34.340984in}{5.584654in}}%
\pgfpathlineto{\pgfqpoint{34.366319in}{5.504363in}}%
\pgfpathlineto{\pgfqpoint{34.388883in}{5.557546in}}%
\pgfpathlineto{\pgfqpoint{34.412956in}{5.442515in}}%
\pgfpathlineto{\pgfqpoint{34.437757in}{5.582839in}}%
\pgfpathlineto{\pgfqpoint{34.460943in}{5.834541in}}%
\pgfpathlineto{\pgfqpoint{34.484122in}{5.740043in}}%
\pgfpathlineto{\pgfqpoint{34.508593in}{5.601241in}}%
\pgfpathlineto{\pgfqpoint{34.531364in}{5.809148in}}%
\pgfpathlineto{\pgfqpoint{34.554213in}{5.745325in}}%
\pgfpathlineto{\pgfqpoint{34.578241in}{5.846668in}}%
\pgfpathlineto{\pgfqpoint{34.601042in}{5.862622in}}%
\pgfpathlineto{\pgfqpoint{34.623987in}{5.791091in}}%
\pgfpathlineto{\pgfqpoint{34.648006in}{5.744292in}}%
\pgfpathlineto{\pgfqpoint{34.671232in}{5.706485in}}%
\pgfpathlineto{\pgfqpoint{34.693969in}{5.908282in}}%
\pgfpathlineto{\pgfqpoint{34.717979in}{5.813609in}}%
\pgfpathlineto{\pgfqpoint{34.741399in}{5.854178in}}%
\pgfpathlineto{\pgfqpoint{34.763884in}{5.918658in}}%
\pgfpathlineto{\pgfqpoint{34.788312in}{5.815038in}}%
\pgfpathlineto{\pgfqpoint{34.810332in}{5.868196in}}%
\pgfpathlineto{\pgfqpoint{34.834146in}{5.857730in}}%
\pgfpathlineto{\pgfqpoint{34.856576in}{5.911984in}}%
\pgfpathlineto{\pgfqpoint{34.880395in}{5.893046in}}%
\pgfpathlineto{\pgfqpoint{34.902821in}{5.880249in}}%
\pgfpathlineto{\pgfqpoint{34.926805in}{5.899825in}}%
\pgfpathlineto{\pgfqpoint{34.948980in}{5.930845in}}%
\pgfpathlineto{\pgfqpoint{34.973284in}{5.726189in}}%
\pgfpathlineto{\pgfqpoint{34.996094in}{5.825752in}}%
\pgfpathlineto{\pgfqpoint{35.020184in}{5.865058in}}%
\pgfpathlineto{\pgfqpoint{35.047391in}{5.775449in}}%
\pgfpathlineto{\pgfqpoint{35.098205in}{5.758501in}}%
\pgfpathlineto{\pgfqpoint{35.149293in}{5.758501in}}%
\pgfpathlineto{\pgfqpoint{35.201170in}{5.758501in}}%
\pgfpathlineto{\pgfqpoint{35.254065in}{5.758501in}}%
\pgfpathlineto{\pgfqpoint{35.305776in}{5.758501in}}%
\pgfpathlineto{\pgfqpoint{35.357710in}{5.758501in}}%
\pgfpathlineto{\pgfqpoint{35.411910in}{5.758501in}}%
\pgfpathlineto{\pgfqpoint{35.464942in}{5.758501in}}%
\pgfpathlineto{\pgfqpoint{35.464942in}{5.758501in}}%
\pgfpathlineto{\pgfqpoint{35.464942in}{5.758501in}}%
\pgfpathlineto{\pgfqpoint{35.411910in}{5.758501in}}%
\pgfpathlineto{\pgfqpoint{35.357710in}{5.758501in}}%
\pgfpathlineto{\pgfqpoint{35.305776in}{5.758501in}}%
\pgfpathlineto{\pgfqpoint{35.254065in}{5.758501in}}%
\pgfpathlineto{\pgfqpoint{35.201170in}{5.758501in}}%
\pgfpathlineto{\pgfqpoint{35.149293in}{5.758501in}}%
\pgfpathlineto{\pgfqpoint{35.098205in}{5.758501in}}%
\pgfpathlineto{\pgfqpoint{35.047391in}{5.775449in}}%
\pgfpathlineto{\pgfqpoint{35.020184in}{5.865058in}}%
\pgfpathlineto{\pgfqpoint{34.996094in}{5.825752in}}%
\pgfpathlineto{\pgfqpoint{34.973284in}{5.726189in}}%
\pgfpathlineto{\pgfqpoint{34.948980in}{5.930845in}}%
\pgfpathlineto{\pgfqpoint{34.926805in}{5.899825in}}%
\pgfpathlineto{\pgfqpoint{34.902821in}{5.880249in}}%
\pgfpathlineto{\pgfqpoint{34.880395in}{5.893046in}}%
\pgfpathlineto{\pgfqpoint{34.856576in}{5.911984in}}%
\pgfpathlineto{\pgfqpoint{34.834146in}{5.857730in}}%
\pgfpathlineto{\pgfqpoint{34.810332in}{5.868196in}}%
\pgfpathlineto{\pgfqpoint{34.788312in}{5.815038in}}%
\pgfpathlineto{\pgfqpoint{34.763884in}{5.918658in}}%
\pgfpathlineto{\pgfqpoint{34.741399in}{5.854178in}}%
\pgfpathlineto{\pgfqpoint{34.717979in}{5.813609in}}%
\pgfpathlineto{\pgfqpoint{34.693969in}{5.908282in}}%
\pgfpathlineto{\pgfqpoint{34.671232in}{5.706485in}}%
\pgfpathlineto{\pgfqpoint{34.648006in}{5.744292in}}%
\pgfpathlineto{\pgfqpoint{34.623987in}{5.791091in}}%
\pgfpathlineto{\pgfqpoint{34.601042in}{5.862622in}}%
\pgfpathlineto{\pgfqpoint{34.578241in}{5.846668in}}%
\pgfpathlineto{\pgfqpoint{34.554213in}{5.745325in}}%
\pgfpathlineto{\pgfqpoint{34.531364in}{5.809148in}}%
\pgfpathlineto{\pgfqpoint{34.508593in}{5.601241in}}%
\pgfpathlineto{\pgfqpoint{34.484122in}{5.740043in}}%
\pgfpathlineto{\pgfqpoint{34.460943in}{5.834541in}}%
\pgfpathlineto{\pgfqpoint{34.437757in}{5.582839in}}%
\pgfpathlineto{\pgfqpoint{34.412956in}{5.442515in}}%
\pgfpathlineto{\pgfqpoint{34.388883in}{5.557546in}}%
\pgfpathlineto{\pgfqpoint{34.366319in}{5.504363in}}%
\pgfpathlineto{\pgfqpoint{34.340984in}{5.584654in}}%
\pgfpathlineto{\pgfqpoint{34.318035in}{5.678338in}}%
\pgfpathlineto{\pgfqpoint{34.293786in}{5.518952in}}%
\pgfpathlineto{\pgfqpoint{34.270997in}{5.460686in}}%
\pgfpathlineto{\pgfqpoint{34.246364in}{5.534473in}}%
\pgfpathlineto{\pgfqpoint{34.222832in}{5.484292in}}%
\pgfpathlineto{\pgfqpoint{34.197440in}{5.442565in}}%
\pgfpathlineto{\pgfqpoint{34.173862in}{5.446238in}}%
\pgfpathlineto{\pgfqpoint{34.150180in}{5.483715in}}%
\pgfpathlineto{\pgfqpoint{34.125436in}{5.364631in}}%
\pgfpathlineto{\pgfqpoint{34.101659in}{5.182737in}}%
\pgfpathlineto{\pgfqpoint{34.076513in}{5.235025in}}%
\pgfpathlineto{\pgfqpoint{34.052161in}{3.729137in}}%
\pgfpathlineto{\pgfqpoint{34.021637in}{2.864879in}}%
\pgfpathlineto{\pgfqpoint{33.989641in}{2.224563in}}%
\pgfpathlineto{\pgfqpoint{33.952831in}{1.906942in}}%
\pgfpathlineto{\pgfqpoint{33.912991in}{1.669618in}}%
\pgfpathlineto{\pgfqpoint{33.870742in}{1.454760in}}%
\pgfpathlineto{\pgfqpoint{33.832264in}{0.773588in}}%
\pgfpathlineto{\pgfqpoint{33.784450in}{0.773588in}}%
\pgfpathlineto{\pgfqpoint{33.746200in}{0.773588in}}%
\pgfpathlineto{\pgfqpoint{33.693680in}{0.773588in}}%
\pgfpathlineto{\pgfqpoint{33.642373in}{0.773588in}}%
\pgfpathlineto{\pgfqpoint{33.590505in}{0.773588in}}%
\pgfpathlineto{\pgfqpoint{33.536879in}{0.773588in}}%
\pgfpathlineto{\pgfqpoint{33.484775in}{0.773588in}}%
\pgfpathlineto{\pgfqpoint{33.433283in}{0.773588in}}%
\pgfpathlineto{\pgfqpoint{33.379482in}{0.773588in}}%
\pgfpathlineto{\pgfqpoint{33.326652in}{0.773588in}}%
\pgfpathlineto{\pgfqpoint{33.273785in}{0.773588in}}%
\pgfpathlineto{\pgfqpoint{33.219830in}{1.230236in}}%
\pgfpathlineto{\pgfqpoint{33.167699in}{1.423144in}}%
\pgfpathlineto{\pgfqpoint{33.115518in}{1.488284in}}%
\pgfpathlineto{\pgfqpoint{33.062733in}{1.414323in}}%
\pgfpathlineto{\pgfqpoint{33.010153in}{1.429618in}}%
\pgfpathlineto{\pgfqpoint{32.957868in}{1.459985in}}%
\pgfpathlineto{\pgfqpoint{32.903667in}{1.442369in}}%
\pgfpathlineto{\pgfqpoint{32.850576in}{1.393664in}}%
\pgfpathlineto{\pgfqpoint{32.797230in}{1.405600in}}%
\pgfpathlineto{\pgfqpoint{32.743153in}{1.453492in}}%
\pgfpathlineto{\pgfqpoint{32.691213in}{1.465799in}}%
\pgfpathlineto{\pgfqpoint{32.638735in}{1.466197in}}%
\pgfpathlineto{\pgfqpoint{32.584565in}{1.438877in}}%
\pgfpathlineto{\pgfqpoint{32.518604in}{1.197335in}}%
\pgfpathlineto{\pgfqpoint{32.429213in}{1.153535in}}%
\pgfpathlineto{\pgfqpoint{32.334387in}{1.189047in}}%
\pgfpathlineto{\pgfqpoint{32.244846in}{1.172160in}}%
\pgfpathlineto{\pgfqpoint{32.153206in}{1.177862in}}%
\pgfpathlineto{\pgfqpoint{32.062211in}{1.165488in}}%
\pgfpathlineto{\pgfqpoint{31.975639in}{1.193878in}}%
\pgfpathlineto{\pgfqpoint{31.890397in}{1.213882in}}%
\pgfpathlineto{\pgfqpoint{31.802724in}{1.145304in}}%
\pgfpathlineto{\pgfqpoint{31.715049in}{1.194231in}}%
\pgfpathlineto{\pgfqpoint{31.629527in}{1.200709in}}%
\pgfpathlineto{\pgfqpoint{31.543273in}{1.177079in}}%
\pgfpathlineto{\pgfqpoint{31.463397in}{1.226639in}}%
\pgfpathlineto{\pgfqpoint{31.385546in}{1.228932in}}%
\pgfpathlineto{\pgfqpoint{31.304546in}{1.197516in}}%
\pgfpathlineto{\pgfqpoint{31.225476in}{1.245446in}}%
\pgfpathlineto{\pgfqpoint{31.150292in}{1.234987in}}%
\pgfpathlineto{\pgfqpoint{31.073353in}{1.239497in}}%
\pgfpathlineto{\pgfqpoint{31.000738in}{1.269004in}}%
\pgfpathlineto{\pgfqpoint{30.928370in}{1.248520in}}%
\pgfpathlineto{\pgfqpoint{30.855684in}{1.273397in}}%
\pgfpathlineto{\pgfqpoint{30.786068in}{1.299568in}}%
\pgfpathlineto{\pgfqpoint{30.717993in}{1.249212in}}%
\pgfpathlineto{\pgfqpoint{30.648377in}{1.321329in}}%
\pgfpathlineto{\pgfqpoint{30.582999in}{1.305125in}}%
\pgfpathlineto{\pgfqpoint{30.519963in}{1.333525in}}%
\pgfpathlineto{\pgfqpoint{30.457369in}{1.372779in}}%
\pgfpathlineto{\pgfqpoint{30.397563in}{1.326296in}}%
\pgfpathlineto{\pgfqpoint{30.337041in}{1.372834in}}%
\pgfpathlineto{\pgfqpoint{30.276704in}{1.374681in}}%
\pgfpathlineto{\pgfqpoint{30.221783in}{1.411341in}}%
\pgfpathlineto{\pgfqpoint{30.169497in}{1.447263in}}%
\pgfpathlineto{\pgfqpoint{30.117082in}{1.447685in}}%
\pgfpathlineto{\pgfqpoint{30.066247in}{1.464706in}}%
\pgfpathlineto{\pgfqpoint{30.014968in}{1.496976in}}%
\pgfpathlineto{\pgfqpoint{29.962066in}{1.466086in}}%
\pgfpathlineto{\pgfqpoint{29.910558in}{1.450867in}}%
\pgfpathlineto{\pgfqpoint{29.858913in}{1.411402in}}%
\pgfpathlineto{\pgfqpoint{29.805208in}{1.447695in}}%
\pgfpathlineto{\pgfqpoint{29.753093in}{1.467650in}}%
\pgfpathlineto{\pgfqpoint{29.701409in}{1.464372in}}%
\pgfpathlineto{\pgfqpoint{29.647988in}{1.476625in}}%
\pgfpathlineto{\pgfqpoint{29.596078in}{1.401838in}}%
\pgfpathlineto{\pgfqpoint{29.543760in}{1.435204in}}%
\pgfpathlineto{\pgfqpoint{29.490075in}{1.448577in}}%
\pgfpathlineto{\pgfqpoint{29.438335in}{1.450584in}}%
\pgfpathlineto{\pgfqpoint{29.386173in}{1.474166in}}%
\pgfpathlineto{\pgfqpoint{29.332261in}{1.459502in}}%
\pgfpathlineto{\pgfqpoint{29.279399in}{1.429412in}}%
\pgfpathlineto{\pgfqpoint{29.226818in}{1.480796in}}%
\pgfpathlineto{\pgfqpoint{29.173700in}{1.453336in}}%
\pgfpathlineto{\pgfqpoint{29.121029in}{1.396378in}}%
\pgfpathlineto{\pgfqpoint{29.068863in}{1.466339in}}%
\pgfpathlineto{\pgfqpoint{29.015651in}{1.498293in}}%
\pgfpathlineto{\pgfqpoint{28.965063in}{1.466028in}}%
\pgfpathlineto{\pgfqpoint{28.913538in}{1.422930in}}%
\pgfpathlineto{\pgfqpoint{28.860347in}{1.440938in}}%
\pgfpathlineto{\pgfqpoint{28.808587in}{1.493059in}}%
\pgfpathlineto{\pgfqpoint{28.758074in}{1.469018in}}%
\pgfpathlineto{\pgfqpoint{28.706300in}{1.488245in}}%
\pgfpathlineto{\pgfqpoint{28.656185in}{1.495303in}}%
\pgfpathlineto{\pgfqpoint{28.605351in}{1.454000in}}%
\pgfpathlineto{\pgfqpoint{28.553749in}{1.468684in}}%
\pgfpathlineto{\pgfqpoint{28.502707in}{1.476075in}}%
\pgfpathlineto{\pgfqpoint{28.451952in}{1.413812in}}%
\pgfpathlineto{\pgfqpoint{28.399138in}{1.435521in}}%
\pgfpathlineto{\pgfqpoint{28.347853in}{1.472000in}}%
\pgfpathlineto{\pgfqpoint{28.297462in}{1.516530in}}%
\pgfpathlineto{\pgfqpoint{28.245320in}{1.463950in}}%
\pgfpathlineto{\pgfqpoint{28.195034in}{1.465843in}}%
\pgfpathlineto{\pgfqpoint{28.143944in}{1.437680in}}%
\pgfpathlineto{\pgfqpoint{28.090650in}{1.466132in}}%
\pgfpathlineto{\pgfqpoint{28.039571in}{1.470166in}}%
\pgfpathlineto{\pgfqpoint{27.988562in}{1.463974in}}%
\pgfpathlineto{\pgfqpoint{27.935127in}{1.426934in}}%
\pgfpathlineto{\pgfqpoint{27.883786in}{1.447905in}}%
\pgfpathlineto{\pgfqpoint{27.833067in}{1.430942in}}%
\pgfpathlineto{\pgfqpoint{27.779331in}{1.453054in}}%
\pgfpathlineto{\pgfqpoint{27.728571in}{1.497292in}}%
\pgfpathlineto{\pgfqpoint{27.678294in}{1.439046in}}%
\pgfpathlineto{\pgfqpoint{27.626352in}{1.461354in}}%
\pgfpathlineto{\pgfqpoint{27.576178in}{1.467419in}}%
\pgfpathlineto{\pgfqpoint{27.525849in}{1.433163in}}%
\pgfpathlineto{\pgfqpoint{27.473173in}{1.466522in}}%
\pgfpathlineto{\pgfqpoint{27.423066in}{1.471617in}}%
\pgfpathlineto{\pgfqpoint{27.372863in}{1.514552in}}%
\pgfpathlineto{\pgfqpoint{27.321160in}{1.441672in}}%
\pgfpathlineto{\pgfqpoint{27.269679in}{1.440422in}}%
\pgfpathlineto{\pgfqpoint{27.218587in}{1.453594in}}%
\pgfpathlineto{\pgfqpoint{27.165274in}{1.437036in}}%
\pgfpathlineto{\pgfqpoint{27.113950in}{1.494852in}}%
\pgfpathlineto{\pgfqpoint{27.063062in}{1.476219in}}%
\pgfpathlineto{\pgfqpoint{27.010539in}{1.518800in}}%
\pgfpathlineto{\pgfqpoint{26.960748in}{1.481435in}}%
\pgfpathlineto{\pgfqpoint{26.909530in}{1.498993in}}%
\pgfpathlineto{\pgfqpoint{26.857706in}{1.441237in}}%
\pgfpathlineto{\pgfqpoint{26.807173in}{1.454026in}}%
\pgfpathlineto{\pgfqpoint{26.756448in}{1.466846in}}%
\pgfpathlineto{\pgfqpoint{26.704176in}{1.474687in}}%
\pgfpathlineto{\pgfqpoint{26.653161in}{1.477742in}}%
\pgfpathlineto{\pgfqpoint{26.601868in}{1.441924in}}%
\pgfpathlineto{\pgfqpoint{26.549446in}{1.488861in}}%
\pgfpathlineto{\pgfqpoint{26.499064in}{1.401061in}}%
\pgfpathlineto{\pgfqpoint{26.447655in}{1.489283in}}%
\pgfpathlineto{\pgfqpoint{26.395829in}{1.491356in}}%
\pgfpathlineto{\pgfqpoint{26.346588in}{1.496304in}}%
\pgfpathlineto{\pgfqpoint{26.297518in}{1.464632in}}%
\pgfpathlineto{\pgfqpoint{26.245717in}{1.478346in}}%
\pgfpathlineto{\pgfqpoint{26.195263in}{1.454513in}}%
\pgfpathlineto{\pgfqpoint{26.144837in}{1.485889in}}%
\pgfpathlineto{\pgfqpoint{26.093895in}{1.464323in}}%
\pgfpathlineto{\pgfqpoint{26.044216in}{1.462308in}}%
\pgfpathlineto{\pgfqpoint{25.994130in}{1.499292in}}%
\pgfpathlineto{\pgfqpoint{25.942005in}{1.461304in}}%
\pgfpathlineto{\pgfqpoint{25.891991in}{1.472900in}}%
\pgfpathlineto{\pgfqpoint{25.841969in}{1.510289in}}%
\pgfpathlineto{\pgfqpoint{25.790287in}{1.509874in}}%
\pgfpathlineto{\pgfqpoint{25.740073in}{1.416044in}}%
\pgfpathlineto{\pgfqpoint{25.689152in}{1.431142in}}%
\pgfpathlineto{\pgfqpoint{25.637513in}{1.447750in}}%
\pgfpathlineto{\pgfqpoint{25.586387in}{1.430021in}}%
\pgfpathlineto{\pgfqpoint{25.535558in}{1.460130in}}%
\pgfpathlineto{\pgfqpoint{25.483381in}{1.476461in}}%
\pgfpathlineto{\pgfqpoint{25.433954in}{1.479729in}}%
\pgfpathlineto{\pgfqpoint{25.383622in}{1.507858in}}%
\pgfpathlineto{\pgfqpoint{25.332608in}{1.455526in}}%
\pgfpathlineto{\pgfqpoint{25.282222in}{1.509514in}}%
\pgfpathlineto{\pgfqpoint{25.232944in}{1.492010in}}%
\pgfpathlineto{\pgfqpoint{25.181554in}{1.480570in}}%
\pgfpathlineto{\pgfqpoint{25.131043in}{1.483945in}}%
\pgfpathlineto{\pgfqpoint{25.080539in}{1.468689in}}%
\pgfpathlineto{\pgfqpoint{25.028630in}{1.453886in}}%
\pgfpathlineto{\pgfqpoint{24.978211in}{1.501248in}}%
\pgfpathlineto{\pgfqpoint{24.928869in}{1.511209in}}%
\pgfpathlineto{\pgfqpoint{24.878143in}{1.454865in}}%
\pgfpathlineto{\pgfqpoint{24.828140in}{1.515156in}}%
\pgfpathlineto{\pgfqpoint{24.778304in}{1.480893in}}%
\pgfpathlineto{\pgfqpoint{24.726499in}{1.497794in}}%
\pgfpathlineto{\pgfqpoint{24.676389in}{1.469206in}}%
\pgfpathlineto{\pgfqpoint{24.626556in}{1.429175in}}%
\pgfpathlineto{\pgfqpoint{24.573998in}{1.488680in}}%
\pgfpathlineto{\pgfqpoint{24.524258in}{1.479312in}}%
\pgfpathlineto{\pgfqpoint{24.474381in}{1.459339in}}%
\pgfpathlineto{\pgfqpoint{24.422475in}{1.471589in}}%
\pgfpathlineto{\pgfqpoint{24.371890in}{1.479833in}}%
\pgfpathlineto{\pgfqpoint{24.321907in}{1.496870in}}%
\pgfpathlineto{\pgfqpoint{24.271277in}{1.436571in}}%
\pgfpathlineto{\pgfqpoint{24.221010in}{1.466875in}}%
\pgfpathlineto{\pgfqpoint{24.170873in}{1.424458in}}%
\pgfpathlineto{\pgfqpoint{24.118637in}{1.404869in}}%
\pgfpathlineto{\pgfqpoint{24.066933in}{1.486397in}}%
\pgfpathlineto{\pgfqpoint{24.016442in}{1.474912in}}%
\pgfpathlineto{\pgfqpoint{23.964514in}{1.429663in}}%
\pgfpathlineto{\pgfqpoint{23.913088in}{1.471046in}}%
\pgfpathlineto{\pgfqpoint{23.862673in}{1.503511in}}%
\pgfpathlineto{\pgfqpoint{23.811512in}{1.487867in}}%
\pgfpathlineto{\pgfqpoint{23.762329in}{1.518819in}}%
\pgfpathlineto{\pgfqpoint{23.713074in}{1.492267in}}%
\pgfpathlineto{\pgfqpoint{23.661575in}{1.457956in}}%
\pgfpathlineto{\pgfqpoint{23.611923in}{1.502550in}}%
\pgfpathlineto{\pgfqpoint{23.563429in}{1.484534in}}%
\pgfpathlineto{\pgfqpoint{23.511867in}{1.483293in}}%
\pgfpathlineto{\pgfqpoint{23.462017in}{1.472289in}}%
\pgfpathlineto{\pgfqpoint{23.411909in}{1.458922in}}%
\pgfpathlineto{\pgfqpoint{23.361310in}{1.464609in}}%
\pgfpathlineto{\pgfqpoint{23.310918in}{1.478442in}}%
\pgfpathlineto{\pgfqpoint{23.260726in}{1.491507in}}%
\pgfpathlineto{\pgfqpoint{23.208694in}{1.458867in}}%
\pgfpathlineto{\pgfqpoint{23.157756in}{1.437185in}}%
\pgfpathlineto{\pgfqpoint{23.107134in}{1.431065in}}%
\pgfpathlineto{\pgfqpoint{23.055187in}{1.520456in}}%
\pgfpathlineto{\pgfqpoint{23.006022in}{1.506808in}}%
\pgfpathlineto{\pgfqpoint{22.956359in}{1.491616in}}%
\pgfpathlineto{\pgfqpoint{22.904236in}{1.432232in}}%
\pgfpathlineto{\pgfqpoint{22.853966in}{1.462154in}}%
\pgfpathlineto{\pgfqpoint{22.803829in}{1.519235in}}%
\pgfpathlineto{\pgfqpoint{22.751700in}{1.515947in}}%
\pgfpathlineto{\pgfqpoint{22.703398in}{1.564313in}}%
\pgfpathlineto{\pgfqpoint{22.654822in}{1.481284in}}%
\pgfpathlineto{\pgfqpoint{22.603657in}{1.492179in}}%
\pgfpathlineto{\pgfqpoint{22.554073in}{1.477021in}}%
\pgfpathlineto{\pgfqpoint{22.503552in}{1.489534in}}%
\pgfpathlineto{\pgfqpoint{22.451976in}{1.474127in}}%
\pgfpathlineto{\pgfqpoint{22.402084in}{1.439666in}}%
\pgfpathlineto{\pgfqpoint{22.352827in}{1.468034in}}%
\pgfpathlineto{\pgfqpoint{22.300933in}{1.455792in}}%
\pgfpathlineto{\pgfqpoint{22.250634in}{1.474893in}}%
\pgfpathlineto{\pgfqpoint{22.200587in}{1.451044in}}%
\pgfpathlineto{\pgfqpoint{22.148817in}{1.527757in}}%
\pgfpathlineto{\pgfqpoint{22.099907in}{1.528125in}}%
\pgfpathlineto{\pgfqpoint{22.050354in}{1.462006in}}%
\pgfpathlineto{\pgfqpoint{21.999803in}{1.514535in}}%
\pgfpathlineto{\pgfqpoint{21.950544in}{1.497388in}}%
\pgfpathlineto{\pgfqpoint{21.900834in}{1.484586in}}%
\pgfpathlineto{\pgfqpoint{21.849580in}{1.518978in}}%
\pgfpathlineto{\pgfqpoint{21.800474in}{1.481533in}}%
\pgfpathlineto{\pgfqpoint{21.751624in}{1.485587in}}%
\pgfpathlineto{\pgfqpoint{21.700864in}{1.492572in}}%
\pgfpathlineto{\pgfqpoint{21.652276in}{1.486753in}}%
\pgfpathlineto{\pgfqpoint{21.602538in}{1.472393in}}%
\pgfpathlineto{\pgfqpoint{21.550957in}{1.489803in}}%
\pgfpathlineto{\pgfqpoint{21.501023in}{1.472200in}}%
\pgfpathlineto{\pgfqpoint{21.450830in}{1.462158in}}%
\pgfpathlineto{\pgfqpoint{21.399733in}{1.454686in}}%
\pgfpathlineto{\pgfqpoint{21.350353in}{1.486772in}}%
\pgfpathlineto{\pgfqpoint{21.300955in}{1.450986in}}%
\pgfpathlineto{\pgfqpoint{21.250655in}{1.492178in}}%
\pgfpathlineto{\pgfqpoint{21.201683in}{1.483018in}}%
\pgfpathlineto{\pgfqpoint{21.152719in}{1.495756in}}%
\pgfpathlineto{\pgfqpoint{21.102370in}{1.514168in}}%
\pgfpathlineto{\pgfqpoint{21.053515in}{1.437617in}}%
\pgfpathlineto{\pgfqpoint{21.003693in}{1.450949in}}%
\pgfpathlineto{\pgfqpoint{20.953585in}{1.516103in}}%
\pgfpathlineto{\pgfqpoint{20.904164in}{1.466930in}}%
\pgfpathlineto{\pgfqpoint{20.854118in}{1.455757in}}%
\pgfpathlineto{\pgfqpoint{20.802965in}{1.588274in}}%
\pgfpathlineto{\pgfqpoint{20.753558in}{1.504972in}}%
\pgfpathlineto{\pgfqpoint{20.704616in}{1.507557in}}%
\pgfpathlineto{\pgfqpoint{20.654668in}{1.483979in}}%
\pgfpathlineto{\pgfqpoint{20.605567in}{1.503138in}}%
\pgfpathlineto{\pgfqpoint{20.556290in}{1.472967in}}%
\pgfpathlineto{\pgfqpoint{20.505208in}{1.523941in}}%
\pgfpathlineto{\pgfqpoint{20.455604in}{1.459393in}}%
\pgfpathlineto{\pgfqpoint{20.406414in}{1.508474in}}%
\pgfpathlineto{\pgfqpoint{20.355462in}{1.455924in}}%
\pgfpathlineto{\pgfqpoint{20.305924in}{1.483280in}}%
\pgfpathlineto{\pgfqpoint{20.255593in}{1.453003in}}%
\pgfpathlineto{\pgfqpoint{20.203615in}{1.498954in}}%
\pgfpathlineto{\pgfqpoint{20.153679in}{1.475407in}}%
\pgfpathlineto{\pgfqpoint{20.104350in}{1.450999in}}%
\pgfpathlineto{\pgfqpoint{20.053312in}{1.505370in}}%
\pgfpathlineto{\pgfqpoint{20.003692in}{1.477769in}}%
\pgfpathlineto{\pgfqpoint{19.955252in}{1.496075in}}%
\pgfpathlineto{\pgfqpoint{19.906615in}{1.532652in}}%
\pgfpathlineto{\pgfqpoint{19.858933in}{1.513257in}}%
\pgfpathlineto{\pgfqpoint{19.810712in}{1.497489in}}%
\pgfpathlineto{\pgfqpoint{19.761678in}{1.504500in}}%
\pgfpathlineto{\pgfqpoint{19.714265in}{1.525351in}}%
\pgfpathlineto{\pgfqpoint{19.666940in}{1.511122in}}%
\pgfpathlineto{\pgfqpoint{19.617707in}{1.483391in}}%
\pgfpathlineto{\pgfqpoint{19.569747in}{1.557243in}}%
\pgfpathlineto{\pgfqpoint{19.522292in}{1.513186in}}%
\pgfpathlineto{\pgfqpoint{19.473154in}{1.482732in}}%
\pgfpathlineto{\pgfqpoint{19.425178in}{1.506046in}}%
\pgfpathlineto{\pgfqpoint{19.376890in}{1.478216in}}%
\pgfpathlineto{\pgfqpoint{19.327829in}{1.473298in}}%
\pgfpathlineto{\pgfqpoint{19.279742in}{1.554138in}}%
\pgfpathlineto{\pgfqpoint{19.231826in}{1.452129in}}%
\pgfpathlineto{\pgfqpoint{19.181302in}{1.471644in}}%
\pgfpathlineto{\pgfqpoint{19.132724in}{1.498695in}}%
\pgfpathlineto{\pgfqpoint{19.084207in}{1.400470in}}%
\pgfpathlineto{\pgfqpoint{19.033794in}{1.491809in}}%
\pgfpathlineto{\pgfqpoint{18.985002in}{1.494968in}}%
\pgfpathlineto{\pgfqpoint{18.937671in}{1.506432in}}%
\pgfpathlineto{\pgfqpoint{18.888188in}{1.487581in}}%
\pgfpathlineto{\pgfqpoint{18.839312in}{1.511409in}}%
\pgfpathlineto{\pgfqpoint{18.791212in}{1.444044in}}%
\pgfpathlineto{\pgfqpoint{18.741665in}{1.505364in}}%
\pgfpathlineto{\pgfqpoint{18.693122in}{1.465243in}}%
\pgfpathlineto{\pgfqpoint{18.644639in}{1.526806in}}%
\pgfpathlineto{\pgfqpoint{18.595724in}{1.507826in}}%
\pgfpathlineto{\pgfqpoint{18.548460in}{1.506176in}}%
\pgfpathlineto{\pgfqpoint{18.500915in}{1.487375in}}%
\pgfpathlineto{\pgfqpoint{18.452284in}{1.468578in}}%
\pgfpathlineto{\pgfqpoint{18.405242in}{1.488980in}}%
\pgfpathlineto{\pgfqpoint{18.357538in}{1.461378in}}%
\pgfpathlineto{\pgfqpoint{18.308441in}{1.541367in}}%
\pgfpathlineto{\pgfqpoint{18.260871in}{1.510613in}}%
\pgfpathlineto{\pgfqpoint{18.213333in}{1.465073in}}%
\pgfpathlineto{\pgfqpoint{18.164405in}{1.513989in}}%
\pgfpathlineto{\pgfqpoint{18.117307in}{1.467506in}}%
\pgfpathlineto{\pgfqpoint{18.069524in}{1.486525in}}%
\pgfpathlineto{\pgfqpoint{18.020026in}{1.421714in}}%
\pgfpathlineto{\pgfqpoint{17.970910in}{1.489846in}}%
\pgfpathlineto{\pgfqpoint{17.922885in}{1.553495in}}%
\pgfpathlineto{\pgfqpoint{17.874600in}{1.523670in}}%
\pgfpathlineto{\pgfqpoint{17.826809in}{1.467132in}}%
\pgfpathlineto{\pgfqpoint{17.779014in}{1.444273in}}%
\pgfpathlineto{\pgfqpoint{17.729727in}{1.487892in}}%
\pgfpathlineto{\pgfqpoint{17.681914in}{1.583122in}}%
\pgfpathlineto{\pgfqpoint{17.634653in}{1.506229in}}%
\pgfpathlineto{\pgfqpoint{17.585742in}{1.600814in}}%
\pgfpathlineto{\pgfqpoint{17.538409in}{1.523829in}}%
\pgfpathlineto{\pgfqpoint{17.491125in}{1.537184in}}%
\pgfpathlineto{\pgfqpoint{17.442362in}{1.481885in}}%
\pgfpathlineto{\pgfqpoint{17.394714in}{1.477277in}}%
\pgfpathlineto{\pgfqpoint{17.347576in}{1.535034in}}%
\pgfpathlineto{\pgfqpoint{17.298941in}{1.495207in}}%
\pgfpathlineto{\pgfqpoint{17.250860in}{1.488999in}}%
\pgfpathlineto{\pgfqpoint{17.203674in}{1.516527in}}%
\pgfpathlineto{\pgfqpoint{17.154507in}{1.527718in}}%
\pgfpathlineto{\pgfqpoint{17.107470in}{1.454535in}}%
\pgfpathlineto{\pgfqpoint{17.059518in}{1.482082in}}%
\pgfpathlineto{\pgfqpoint{17.010585in}{1.515163in}}%
\pgfpathlineto{\pgfqpoint{16.963533in}{1.531321in}}%
\pgfpathlineto{\pgfqpoint{16.916302in}{1.569257in}}%
\pgfpathlineto{\pgfqpoint{16.868266in}{1.525440in}}%
\pgfpathlineto{\pgfqpoint{16.821038in}{1.543591in}}%
\pgfpathlineto{\pgfqpoint{16.773713in}{1.503750in}}%
\pgfpathlineto{\pgfqpoint{16.724568in}{1.517676in}}%
\pgfpathlineto{\pgfqpoint{16.677566in}{1.538170in}}%
\pgfpathlineto{\pgfqpoint{16.630280in}{1.488775in}}%
\pgfpathlineto{\pgfqpoint{16.581135in}{1.493295in}}%
\pgfpathlineto{\pgfqpoint{16.533619in}{1.517671in}}%
\pgfpathlineto{\pgfqpoint{16.486560in}{1.479180in}}%
\pgfpathlineto{\pgfqpoint{16.436622in}{1.435421in}}%
\pgfpathlineto{\pgfqpoint{16.387666in}{1.454859in}}%
\pgfpathlineto{\pgfqpoint{16.339537in}{1.490275in}}%
\pgfpathlineto{\pgfqpoint{16.289817in}{1.480735in}}%
\pgfpathlineto{\pgfqpoint{16.242078in}{1.524677in}}%
\pgfpathlineto{\pgfqpoint{16.195175in}{1.518044in}}%
\pgfpathlineto{\pgfqpoint{16.146338in}{1.480017in}}%
\pgfpathlineto{\pgfqpoint{16.098697in}{1.497695in}}%
\pgfpathlineto{\pgfqpoint{16.051820in}{1.553606in}}%
\pgfpathlineto{\pgfqpoint{16.003926in}{1.547024in}}%
\pgfpathlineto{\pgfqpoint{15.957799in}{1.581828in}}%
\pgfpathlineto{\pgfqpoint{15.911811in}{1.516610in}}%
\pgfpathlineto{\pgfqpoint{15.863592in}{1.514360in}}%
\pgfpathlineto{\pgfqpoint{15.817273in}{1.508377in}}%
\pgfpathlineto{\pgfqpoint{15.770154in}{1.534549in}}%
\pgfpathlineto{\pgfqpoint{15.722814in}{1.540818in}}%
\pgfpathlineto{\pgfqpoint{15.676019in}{1.540627in}}%
\pgfpathlineto{\pgfqpoint{15.628660in}{1.494626in}}%
\pgfpathlineto{\pgfqpoint{15.579815in}{1.510925in}}%
\pgfpathlineto{\pgfqpoint{15.532329in}{1.492190in}}%
\pgfpathlineto{\pgfqpoint{15.484858in}{1.500648in}}%
\pgfpathlineto{\pgfqpoint{15.436052in}{1.517126in}}%
\pgfpathlineto{\pgfqpoint{15.389896in}{1.561039in}}%
\pgfpathlineto{\pgfqpoint{15.343883in}{1.544535in}}%
\pgfpathlineto{\pgfqpoint{15.295866in}{1.546965in}}%
\pgfpathlineto{\pgfqpoint{15.248863in}{1.498469in}}%
\pgfpathlineto{\pgfqpoint{15.201456in}{1.483789in}}%
\pgfpathlineto{\pgfqpoint{15.152658in}{1.543727in}}%
\pgfpathlineto{\pgfqpoint{15.105478in}{1.565050in}}%
\pgfpathlineto{\pgfqpoint{15.058774in}{1.476105in}}%
\pgfpathlineto{\pgfqpoint{15.010214in}{1.546708in}}%
\pgfpathlineto{\pgfqpoint{14.963167in}{1.469295in}}%
\pgfpathlineto{\pgfqpoint{14.916239in}{1.491863in}}%
\pgfpathlineto{\pgfqpoint{14.868242in}{1.548620in}}%
\pgfpathlineto{\pgfqpoint{14.821474in}{1.591513in}}%
\pgfpathlineto{\pgfqpoint{14.774752in}{1.538923in}}%
\pgfpathlineto{\pgfqpoint{14.726757in}{1.507421in}}%
\pgfpathlineto{\pgfqpoint{14.680226in}{1.518840in}}%
\pgfpathlineto{\pgfqpoint{14.633992in}{1.497269in}}%
\pgfpathlineto{\pgfqpoint{14.585477in}{1.511082in}}%
\pgfpathlineto{\pgfqpoint{14.538630in}{1.531812in}}%
\pgfpathlineto{\pgfqpoint{14.491726in}{1.506448in}}%
\pgfpathlineto{\pgfqpoint{14.443729in}{1.563335in}}%
\pgfpathlineto{\pgfqpoint{14.397126in}{1.541511in}}%
\pgfpathlineto{\pgfqpoint{14.350224in}{1.535720in}}%
\pgfpathlineto{\pgfqpoint{14.302418in}{1.547427in}}%
\pgfpathlineto{\pgfqpoint{14.255593in}{1.552002in}}%
\pgfpathlineto{\pgfqpoint{14.209376in}{1.498058in}}%
\pgfpathlineto{\pgfqpoint{14.160147in}{1.488189in}}%
\pgfpathlineto{\pgfqpoint{14.112147in}{1.495848in}}%
\pgfpathlineto{\pgfqpoint{14.064951in}{1.526672in}}%
\pgfpathlineto{\pgfqpoint{14.016344in}{1.538810in}}%
\pgfpathlineto{\pgfqpoint{13.969353in}{1.492281in}}%
\pgfpathlineto{\pgfqpoint{13.922606in}{1.475342in}}%
\pgfpathlineto{\pgfqpoint{13.873610in}{1.498642in}}%
\pgfpathlineto{\pgfqpoint{13.825297in}{1.477918in}}%
\pgfpathlineto{\pgfqpoint{13.777441in}{1.505462in}}%
\pgfpathlineto{\pgfqpoint{13.729143in}{1.534486in}}%
\pgfpathlineto{\pgfqpoint{13.682151in}{1.501355in}}%
\pgfpathlineto{\pgfqpoint{13.635346in}{1.521893in}}%
\pgfpathlineto{\pgfqpoint{13.587249in}{1.524477in}}%
\pgfpathlineto{\pgfqpoint{13.541646in}{1.519941in}}%
\pgfpathlineto{\pgfqpoint{13.495727in}{1.564865in}}%
\pgfpathlineto{\pgfqpoint{13.448069in}{1.521523in}}%
\pgfpathlineto{\pgfqpoint{13.401859in}{1.547631in}}%
\pgfpathlineto{\pgfqpoint{13.356518in}{1.574697in}}%
\pgfpathlineto{\pgfqpoint{13.308974in}{1.527229in}}%
\pgfpathlineto{\pgfqpoint{13.262980in}{1.568872in}}%
\pgfpathlineto{\pgfqpoint{13.216719in}{1.470586in}}%
\pgfpathlineto{\pgfqpoint{13.169075in}{1.552921in}}%
\pgfpathlineto{\pgfqpoint{13.123339in}{1.500250in}}%
\pgfpathlineto{\pgfqpoint{13.077275in}{1.575979in}}%
\pgfpathlineto{\pgfqpoint{13.030499in}{1.505766in}}%
\pgfpathlineto{\pgfqpoint{12.983596in}{1.517211in}}%
\pgfpathlineto{\pgfqpoint{12.936799in}{1.561665in}}%
\pgfpathlineto{\pgfqpoint{12.889988in}{1.540314in}}%
\pgfpathlineto{\pgfqpoint{12.843155in}{1.490534in}}%
\pgfpathlineto{\pgfqpoint{12.796907in}{1.517135in}}%
\pgfpathlineto{\pgfqpoint{12.749025in}{1.508387in}}%
\pgfpathlineto{\pgfqpoint{12.701835in}{1.546420in}}%
\pgfpathlineto{\pgfqpoint{12.655424in}{1.506971in}}%
\pgfpathlineto{\pgfqpoint{12.607615in}{1.536917in}}%
\pgfpathlineto{\pgfqpoint{12.560711in}{1.464062in}}%
\pgfpathlineto{\pgfqpoint{12.514481in}{1.594296in}}%
\pgfpathlineto{\pgfqpoint{12.467453in}{1.508554in}}%
\pgfpathlineto{\pgfqpoint{12.421372in}{1.539521in}}%
\pgfpathlineto{\pgfqpoint{12.375261in}{1.524465in}}%
\pgfpathlineto{\pgfqpoint{12.328136in}{1.497095in}}%
\pgfpathlineto{\pgfqpoint{12.282271in}{1.510600in}}%
\pgfpathlineto{\pgfqpoint{12.235985in}{1.482695in}}%
\pgfpathlineto{\pgfqpoint{12.187925in}{1.534904in}}%
\pgfpathlineto{\pgfqpoint{12.141887in}{1.577279in}}%
\pgfpathlineto{\pgfqpoint{12.096623in}{1.537181in}}%
\pgfpathlineto{\pgfqpoint{12.049869in}{1.537005in}}%
\pgfpathlineto{\pgfqpoint{12.004414in}{1.529616in}}%
\pgfpathlineto{\pgfqpoint{11.959143in}{1.570733in}}%
\pgfpathlineto{\pgfqpoint{11.911501in}{1.512933in}}%
\pgfpathlineto{\pgfqpoint{11.864862in}{1.556197in}}%
\pgfpathlineto{\pgfqpoint{11.819406in}{1.563906in}}%
\pgfpathlineto{\pgfqpoint{11.772894in}{1.515236in}}%
\pgfpathlineto{\pgfqpoint{11.727607in}{1.565494in}}%
\pgfpathlineto{\pgfqpoint{11.682053in}{1.578824in}}%
\pgfpathlineto{\pgfqpoint{11.634589in}{1.509051in}}%
\pgfpathlineto{\pgfqpoint{11.588799in}{1.519041in}}%
\pgfpathlineto{\pgfqpoint{11.542665in}{1.579485in}}%
\pgfpathlineto{\pgfqpoint{11.494874in}{1.564506in}}%
\pgfpathlineto{\pgfqpoint{11.448531in}{1.515160in}}%
\pgfpathlineto{\pgfqpoint{11.402044in}{1.542273in}}%
\pgfpathlineto{\pgfqpoint{11.353756in}{1.545928in}}%
\pgfpathlineto{\pgfqpoint{11.307361in}{1.601051in}}%
\pgfpathlineto{\pgfqpoint{11.261350in}{1.529538in}}%
\pgfpathlineto{\pgfqpoint{11.213355in}{1.501424in}}%
\pgfpathlineto{\pgfqpoint{11.167471in}{1.582735in}}%
\pgfpathlineto{\pgfqpoint{11.121978in}{1.486227in}}%
\pgfpathlineto{\pgfqpoint{11.074633in}{1.559755in}}%
\pgfpathlineto{\pgfqpoint{11.029398in}{1.566477in}}%
\pgfpathlineto{\pgfqpoint{10.984145in}{1.564078in}}%
\pgfpathlineto{\pgfqpoint{10.937381in}{1.533498in}}%
\pgfpathlineto{\pgfqpoint{10.891801in}{1.528621in}}%
\pgfpathlineto{\pgfqpoint{10.845790in}{1.525429in}}%
\pgfpathlineto{\pgfqpoint{10.798768in}{1.525837in}}%
\pgfpathlineto{\pgfqpoint{10.753168in}{1.538547in}}%
\pgfpathlineto{\pgfqpoint{10.707461in}{1.555544in}}%
\pgfpathlineto{\pgfqpoint{10.660112in}{1.477165in}}%
\pgfpathlineto{\pgfqpoint{10.613886in}{1.487792in}}%
\pgfpathlineto{\pgfqpoint{10.568154in}{1.556578in}}%
\pgfpathlineto{\pgfqpoint{10.521105in}{1.508048in}}%
\pgfpathlineto{\pgfqpoint{10.475642in}{1.538791in}}%
\pgfpathlineto{\pgfqpoint{10.429362in}{1.541046in}}%
\pgfpathlineto{\pgfqpoint{10.381570in}{1.532981in}}%
\pgfpathlineto{\pgfqpoint{10.335334in}{1.490875in}}%
\pgfpathlineto{\pgfqpoint{10.289084in}{1.505217in}}%
\pgfpathlineto{\pgfqpoint{10.242601in}{1.498136in}}%
\pgfpathlineto{\pgfqpoint{10.197247in}{1.584503in}}%
\pgfpathlineto{\pgfqpoint{10.151348in}{1.521257in}}%
\pgfpathlineto{\pgfqpoint{10.103098in}{1.503213in}}%
\pgfpathlineto{\pgfqpoint{10.057613in}{1.520968in}}%
\pgfpathlineto{\pgfqpoint{10.011915in}{1.505468in}}%
\pgfpathlineto{\pgfqpoint{9.964562in}{1.544111in}}%
\pgfpathlineto{\pgfqpoint{9.918842in}{1.503684in}}%
\pgfpathlineto{\pgfqpoint{9.873173in}{1.536439in}}%
\pgfpathlineto{\pgfqpoint{9.826599in}{1.550244in}}%
\pgfpathlineto{\pgfqpoint{9.781321in}{1.576866in}}%
\pgfpathlineto{\pgfqpoint{9.735785in}{1.571327in}}%
\pgfpathlineto{\pgfqpoint{9.689497in}{1.563754in}}%
\pgfpathlineto{\pgfqpoint{9.644121in}{1.525747in}}%
\pgfpathlineto{\pgfqpoint{9.598338in}{1.554603in}}%
\pgfpathlineto{\pgfqpoint{9.552197in}{1.615870in}}%
\pgfpathlineto{\pgfqpoint{9.507583in}{1.530625in}}%
\pgfpathlineto{\pgfqpoint{9.462006in}{1.505764in}}%
\pgfpathlineto{\pgfqpoint{9.415178in}{1.559155in}}%
\pgfpathlineto{\pgfqpoint{9.369639in}{1.539785in}}%
\pgfpathlineto{\pgfqpoint{9.324423in}{1.592727in}}%
\pgfpathlineto{\pgfqpoint{9.278439in}{1.561329in}}%
\pgfpathlineto{\pgfqpoint{9.233107in}{1.582109in}}%
\pgfpathlineto{\pgfqpoint{9.188332in}{1.567783in}}%
\pgfpathlineto{\pgfqpoint{9.141270in}{1.506562in}}%
\pgfpathlineto{\pgfqpoint{9.095727in}{1.521941in}}%
\pgfpathlineto{\pgfqpoint{9.050165in}{1.585890in}}%
\pgfpathlineto{\pgfqpoint{9.003947in}{1.496155in}}%
\pgfpathlineto{\pgfqpoint{8.957924in}{1.566446in}}%
\pgfpathlineto{\pgfqpoint{8.912131in}{1.637496in}}%
\pgfpathlineto{\pgfqpoint{8.866242in}{1.528400in}}%
\pgfpathlineto{\pgfqpoint{8.821068in}{1.523053in}}%
\pgfpathlineto{\pgfqpoint{8.774902in}{1.505111in}}%
\pgfpathlineto{\pgfqpoint{8.727404in}{1.517910in}}%
\pgfpathlineto{\pgfqpoint{8.681080in}{1.495158in}}%
\pgfpathlineto{\pgfqpoint{8.635315in}{1.586244in}}%
\pgfpathlineto{\pgfqpoint{8.589124in}{1.511789in}}%
\pgfpathlineto{\pgfqpoint{8.543976in}{1.573395in}}%
\pgfpathlineto{\pgfqpoint{8.499298in}{1.509364in}}%
\pgfpathlineto{\pgfqpoint{8.453780in}{1.587303in}}%
\pgfpathlineto{\pgfqpoint{8.409767in}{1.569722in}}%
\pgfpathlineto{\pgfqpoint{8.364636in}{1.546337in}}%
\pgfpathlineto{\pgfqpoint{8.318789in}{1.548952in}}%
\pgfpathlineto{\pgfqpoint{8.273006in}{1.549999in}}%
\pgfpathlineto{\pgfqpoint{8.227930in}{1.552209in}}%
\pgfpathlineto{\pgfqpoint{8.181791in}{1.500494in}}%
\pgfpathlineto{\pgfqpoint{8.136842in}{1.541154in}}%
\pgfpathlineto{\pgfqpoint{8.091881in}{1.571345in}}%
\pgfpathlineto{\pgfqpoint{8.045278in}{1.601356in}}%
\pgfpathlineto{\pgfqpoint{8.000573in}{1.488688in}}%
\pgfpathlineto{\pgfqpoint{7.955879in}{1.537689in}}%
\pgfpathlineto{\pgfqpoint{7.910161in}{1.562561in}}%
\pgfpathlineto{\pgfqpoint{7.865263in}{1.525802in}}%
\pgfpathlineto{\pgfqpoint{7.819947in}{1.548381in}}%
\pgfpathlineto{\pgfqpoint{7.773226in}{1.615933in}}%
\pgfpathlineto{\pgfqpoint{7.728803in}{1.568185in}}%
\pgfpathlineto{\pgfqpoint{7.682978in}{1.581928in}}%
\pgfpathlineto{\pgfqpoint{7.636592in}{1.558451in}}%
\pgfpathlineto{\pgfqpoint{7.591890in}{1.560429in}}%
\pgfpathlineto{\pgfqpoint{7.546892in}{1.556177in}}%
\pgfpathlineto{\pgfqpoint{7.500683in}{1.584308in}}%
\pgfpathlineto{\pgfqpoint{7.455548in}{1.582574in}}%
\pgfpathlineto{\pgfqpoint{7.409977in}{1.568201in}}%
\pgfpathlineto{\pgfqpoint{7.363749in}{1.582469in}}%
\pgfpathlineto{\pgfqpoint{7.318484in}{1.554929in}}%
\pgfpathlineto{\pgfqpoint{7.273914in}{1.570827in}}%
\pgfpathlineto{\pgfqpoint{7.228120in}{1.557341in}}%
\pgfpathlineto{\pgfqpoint{7.184277in}{1.614228in}}%
\pgfpathlineto{\pgfqpoint{7.140134in}{1.583456in}}%
\pgfpathlineto{\pgfqpoint{7.094205in}{1.593184in}}%
\pgfpathlineto{\pgfqpoint{7.050071in}{1.581141in}}%
\pgfpathlineto{\pgfqpoint{7.005149in}{1.598260in}}%
\pgfpathlineto{\pgfqpoint{6.958763in}{1.576631in}}%
\pgfpathlineto{\pgfqpoint{6.914194in}{1.558499in}}%
\pgfpathlineto{\pgfqpoint{6.869544in}{1.547619in}}%
\pgfpathlineto{\pgfqpoint{6.824012in}{1.572692in}}%
\pgfpathlineto{\pgfqpoint{6.779295in}{1.574511in}}%
\pgfpathlineto{\pgfqpoint{6.734887in}{1.479787in}}%
\pgfpathlineto{\pgfqpoint{6.688504in}{1.551500in}}%
\pgfpathlineto{\pgfqpoint{6.643960in}{1.526847in}}%
\pgfpathlineto{\pgfqpoint{6.599302in}{1.570688in}}%
\pgfpathlineto{\pgfqpoint{6.553117in}{1.519916in}}%
\pgfpathlineto{\pgfqpoint{6.507651in}{1.500187in}}%
\pgfpathlineto{\pgfqpoint{6.461324in}{1.540551in}}%
\pgfpathlineto{\pgfqpoint{6.413399in}{1.533672in}}%
\pgfpathlineto{\pgfqpoint{6.367508in}{1.510727in}}%
\pgfpathlineto{\pgfqpoint{6.321070in}{1.486201in}}%
\pgfpathlineto{\pgfqpoint{6.273584in}{1.547490in}}%
\pgfpathlineto{\pgfqpoint{6.227112in}{1.515156in}}%
\pgfpathlineto{\pgfqpoint{6.180802in}{1.459558in}}%
\pgfpathlineto{\pgfqpoint{6.133235in}{1.552728in}}%
\pgfpathlineto{\pgfqpoint{6.087229in}{1.495876in}}%
\pgfpathlineto{\pgfqpoint{6.041630in}{1.591384in}}%
\pgfpathlineto{\pgfqpoint{5.994686in}{1.484192in}}%
\pgfpathlineto{\pgfqpoint{5.948991in}{1.525182in}}%
\pgfpathlineto{\pgfqpoint{5.903366in}{1.559782in}}%
\pgfpathlineto{\pgfqpoint{5.856181in}{1.516021in}}%
\pgfpathlineto{\pgfqpoint{5.810269in}{1.587129in}}%
\pgfpathlineto{\pgfqpoint{5.765813in}{1.564435in}}%
\pgfpathlineto{\pgfqpoint{5.719569in}{1.549158in}}%
\pgfpathlineto{\pgfqpoint{5.674262in}{1.567508in}}%
\pgfpathlineto{\pgfqpoint{5.628857in}{1.517186in}}%
\pgfpathlineto{\pgfqpoint{5.581803in}{1.528391in}}%
\pgfpathlineto{\pgfqpoint{5.536203in}{1.570565in}}%
\pgfpathlineto{\pgfqpoint{5.491128in}{1.607472in}}%
\pgfpathlineto{\pgfqpoint{5.444526in}{1.561463in}}%
\pgfpathlineto{\pgfqpoint{5.399334in}{1.590423in}}%
\pgfpathlineto{\pgfqpoint{5.353614in}{1.538996in}}%
\pgfpathlineto{\pgfqpoint{5.307244in}{1.578955in}}%
\pgfpathlineto{\pgfqpoint{5.262146in}{1.507684in}}%
\pgfpathlineto{\pgfqpoint{5.216291in}{1.516594in}}%
\pgfpathlineto{\pgfqpoint{5.169192in}{1.503848in}}%
\pgfpathlineto{\pgfqpoint{5.124036in}{1.496465in}}%
\pgfpathlineto{\pgfqpoint{5.078074in}{1.563684in}}%
\pgfpathlineto{\pgfqpoint{5.030678in}{1.549426in}}%
\pgfpathlineto{\pgfqpoint{4.984973in}{1.605568in}}%
\pgfpathlineto{\pgfqpoint{4.939032in}{1.498860in}}%
\pgfpathlineto{\pgfqpoint{4.891298in}{1.536251in}}%
\pgfpathlineto{\pgfqpoint{4.845429in}{1.586627in}}%
\pgfpathlineto{\pgfqpoint{4.799887in}{1.602354in}}%
\pgfpathlineto{\pgfqpoint{4.753449in}{1.570300in}}%
\pgfpathlineto{\pgfqpoint{4.708228in}{1.563118in}}%
\pgfpathlineto{\pgfqpoint{4.663622in}{1.547884in}}%
\pgfpathlineto{\pgfqpoint{4.617918in}{1.553545in}}%
\pgfpathlineto{\pgfqpoint{4.572708in}{1.484313in}}%
\pgfpathlineto{\pgfqpoint{4.527424in}{1.542123in}}%
\pgfpathlineto{\pgfqpoint{4.480684in}{1.471395in}}%
\pgfpathlineto{\pgfqpoint{4.435284in}{1.574393in}}%
\pgfpathlineto{\pgfqpoint{4.390187in}{1.572779in}}%
\pgfpathlineto{\pgfqpoint{4.343118in}{1.489350in}}%
\pgfpathlineto{\pgfqpoint{4.297354in}{1.585193in}}%
\pgfpathlineto{\pgfqpoint{4.252389in}{1.484992in}}%
\pgfpathlineto{\pgfqpoint{4.204543in}{1.546017in}}%
\pgfpathlineto{\pgfqpoint{4.159078in}{1.580009in}}%
\pgfpathlineto{\pgfqpoint{4.114362in}{1.536141in}}%
\pgfpathlineto{\pgfqpoint{4.067507in}{1.522087in}}%
\pgfpathlineto{\pgfqpoint{4.022115in}{1.551921in}}%
\pgfpathlineto{\pgfqpoint{3.976868in}{1.579625in}}%
\pgfpathlineto{\pgfqpoint{3.930307in}{1.533538in}}%
\pgfpathlineto{\pgfqpoint{3.884614in}{1.571975in}}%
\pgfpathlineto{\pgfqpoint{3.839775in}{1.560708in}}%
\pgfpathlineto{\pgfqpoint{3.793040in}{1.516540in}}%
\pgfpathlineto{\pgfqpoint{3.747249in}{1.563787in}}%
\pgfpathlineto{\pgfqpoint{3.701949in}{1.557015in}}%
\pgfpathlineto{\pgfqpoint{3.654909in}{1.568488in}}%
\pgfpathlineto{\pgfqpoint{3.609556in}{1.568303in}}%
\pgfpathlineto{\pgfqpoint{3.564489in}{1.568396in}}%
\pgfpathlineto{\pgfqpoint{3.517815in}{1.611411in}}%
\pgfpathlineto{\pgfqpoint{3.473135in}{1.528243in}}%
\pgfpathlineto{\pgfqpoint{3.428526in}{1.553507in}}%
\pgfpathlineto{\pgfqpoint{3.381274in}{1.580310in}}%
\pgfpathlineto{\pgfqpoint{3.336184in}{1.536065in}}%
\pgfpathlineto{\pgfqpoint{3.290748in}{1.578655in}}%
\pgfpathlineto{\pgfqpoint{3.245458in}{1.541177in}}%
\pgfpathlineto{\pgfqpoint{3.200519in}{1.611822in}}%
\pgfpathlineto{\pgfqpoint{3.155580in}{1.505975in}}%
\pgfpathlineto{\pgfqpoint{3.108180in}{1.547334in}}%
\pgfpathlineto{\pgfqpoint{3.062761in}{1.509831in}}%
\pgfpathlineto{\pgfqpoint{3.017486in}{1.566672in}}%
\pgfpathlineto{\pgfqpoint{2.971917in}{1.562850in}}%
\pgfpathlineto{\pgfqpoint{2.927413in}{1.605658in}}%
\pgfpathlineto{\pgfqpoint{2.883134in}{1.562943in}}%
\pgfpathlineto{\pgfqpoint{2.836381in}{1.494875in}}%
\pgfpathlineto{\pgfqpoint{2.790736in}{1.518697in}}%
\pgfpathlineto{\pgfqpoint{2.745668in}{1.534324in}}%
\pgfpathlineto{\pgfqpoint{2.698034in}{1.566456in}}%
\pgfpathlineto{\pgfqpoint{2.651003in}{1.510544in}}%
\pgfpathlineto{\pgfqpoint{2.604306in}{1.532746in}}%
\pgfpathlineto{\pgfqpoint{2.555498in}{1.473000in}}%
\pgfpathlineto{\pgfqpoint{2.505534in}{1.469872in}}%
\pgfpathlineto{\pgfqpoint{2.452591in}{1.368398in}}%
\pgfpathlineto{\pgfqpoint{2.397147in}{1.466636in}}%
\pgfpathlineto{\pgfqpoint{2.348431in}{1.472334in}}%
\pgfpathlineto{\pgfqpoint{2.299591in}{1.500452in}}%
\pgfpathlineto{\pgfqpoint{2.249804in}{1.468263in}}%
\pgfpathlineto{\pgfqpoint{2.201242in}{1.530426in}}%
\pgfpathlineto{\pgfqpoint{2.153284in}{1.480410in}}%
\pgfpathlineto{\pgfqpoint{2.104325in}{1.481740in}}%
\pgfpathlineto{\pgfqpoint{2.057441in}{1.556462in}}%
\pgfpathlineto{\pgfqpoint{2.011209in}{1.507382in}}%
\pgfpathlineto{\pgfqpoint{1.963476in}{1.535011in}}%
\pgfpathlineto{\pgfqpoint{1.915908in}{1.489817in}}%
\pgfpathlineto{\pgfqpoint{1.869493in}{1.528099in}}%
\pgfpathlineto{\pgfqpoint{1.822804in}{1.480032in}}%
\pgfpathlineto{\pgfqpoint{1.778099in}{1.614134in}}%
\pgfpathlineto{\pgfqpoint{1.734500in}{1.580423in}}%
\pgfpathlineto{\pgfqpoint{1.689326in}{1.619864in}}%
\pgfpathlineto{\pgfqpoint{1.645274in}{1.512994in}}%
\pgfpathlineto{\pgfqpoint{1.600816in}{1.542199in}}%
\pgfpathlineto{\pgfqpoint{1.555718in}{1.603777in}}%
\pgfpathlineto{\pgfqpoint{1.511740in}{1.618973in}}%
\pgfpathlineto{\pgfqpoint{1.468334in}{1.591694in}}%
\pgfpathlineto{\pgfqpoint{1.422957in}{1.505606in}}%
\pgfpathlineto{\pgfqpoint{1.378287in}{1.607857in}}%
\pgfpathlineto{\pgfqpoint{1.334262in}{1.579287in}}%
\pgfpathlineto{\pgfqpoint{1.289074in}{1.595870in}}%
\pgfpathlineto{\pgfqpoint{1.244609in}{1.539063in}}%
\pgfpathlineto{\pgfqpoint{1.200192in}{1.549785in}}%
\pgfpathlineto{\pgfqpoint{1.155171in}{1.592795in}}%
\pgfpathlineto{\pgfqpoint{1.111790in}{1.576199in}}%
\pgfpathlineto{\pgfqpoint{1.067773in}{1.498449in}}%
\pgfpathlineto{\pgfqpoint{1.021908in}{1.547185in}}%
\pgfpathlineto{\pgfqpoint{0.978015in}{1.562403in}}%
\pgfpathlineto{\pgfqpoint{0.933783in}{1.503511in}}%
\pgfpathlineto{\pgfqpoint{0.887244in}{1.513665in}}%
\pgfpathlineto{\pgfqpoint{0.842612in}{1.574757in}}%
\pgfpathlineto{\pgfqpoint{0.797895in}{1.281056in}}%
\pgfpathclose%
\pgfusepath{fill}%
\end{pgfscope}%
\begin{pgfscope}%
\pgfpathrectangle{\pgfqpoint{0.781402in}{0.773588in}}{\pgfqpoint{1.440244in}{5.415119in}}%
\pgfusepath{clip}%
\pgfsetbuttcap%
\pgfsetroundjoin%
\definecolor{currentfill}{rgb}{0.580392,0.403922,0.741176}%
\pgfsetfillcolor{currentfill}%
\pgfsetlinewidth{0.000000pt}%
\definecolor{currentstroke}{rgb}{0.000000,0.000000,0.000000}%
\pgfsetstrokecolor{currentstroke}%
\pgfsetdash{}{0pt}%
\pgfpathmoveto{\pgfqpoint{0.797895in}{1.738591in}}%
\pgfpathlineto{\pgfqpoint{0.797895in}{1.281056in}}%
\pgfpathlineto{\pgfqpoint{0.842612in}{1.574757in}}%
\pgfpathlineto{\pgfqpoint{0.887244in}{1.513665in}}%
\pgfpathlineto{\pgfqpoint{0.933783in}{1.503511in}}%
\pgfpathlineto{\pgfqpoint{0.978015in}{1.562403in}}%
\pgfpathlineto{\pgfqpoint{1.021908in}{1.547185in}}%
\pgfpathlineto{\pgfqpoint{1.067773in}{1.498449in}}%
\pgfpathlineto{\pgfqpoint{1.111790in}{1.576199in}}%
\pgfpathlineto{\pgfqpoint{1.155171in}{1.592795in}}%
\pgfpathlineto{\pgfqpoint{1.200192in}{1.549785in}}%
\pgfpathlineto{\pgfqpoint{1.244609in}{1.539063in}}%
\pgfpathlineto{\pgfqpoint{1.289074in}{1.595870in}}%
\pgfpathlineto{\pgfqpoint{1.334262in}{1.579287in}}%
\pgfpathlineto{\pgfqpoint{1.378287in}{1.607857in}}%
\pgfpathlineto{\pgfqpoint{1.422957in}{1.505606in}}%
\pgfpathlineto{\pgfqpoint{1.468334in}{1.591694in}}%
\pgfpathlineto{\pgfqpoint{1.511740in}{1.618973in}}%
\pgfpathlineto{\pgfqpoint{1.555718in}{1.603777in}}%
\pgfpathlineto{\pgfqpoint{1.600816in}{1.542199in}}%
\pgfpathlineto{\pgfqpoint{1.645274in}{1.512994in}}%
\pgfpathlineto{\pgfqpoint{1.689326in}{1.619864in}}%
\pgfpathlineto{\pgfqpoint{1.734500in}{1.580423in}}%
\pgfpathlineto{\pgfqpoint{1.778099in}{1.614134in}}%
\pgfpathlineto{\pgfqpoint{1.822804in}{1.480032in}}%
\pgfpathlineto{\pgfqpoint{1.869493in}{1.528099in}}%
\pgfpathlineto{\pgfqpoint{1.915908in}{1.489817in}}%
\pgfpathlineto{\pgfqpoint{1.963476in}{1.535011in}}%
\pgfpathlineto{\pgfqpoint{2.011209in}{1.507382in}}%
\pgfpathlineto{\pgfqpoint{2.057441in}{1.556462in}}%
\pgfpathlineto{\pgfqpoint{2.104325in}{1.481740in}}%
\pgfpathlineto{\pgfqpoint{2.153284in}{1.480410in}}%
\pgfpathlineto{\pgfqpoint{2.201242in}{1.530426in}}%
\pgfpathlineto{\pgfqpoint{2.249804in}{1.468263in}}%
\pgfpathlineto{\pgfqpoint{2.299591in}{1.500452in}}%
\pgfpathlineto{\pgfqpoint{2.348431in}{1.472334in}}%
\pgfpathlineto{\pgfqpoint{2.397147in}{1.466636in}}%
\pgfpathlineto{\pgfqpoint{2.452591in}{1.368398in}}%
\pgfpathlineto{\pgfqpoint{2.505534in}{1.469872in}}%
\pgfpathlineto{\pgfqpoint{2.555498in}{1.473000in}}%
\pgfpathlineto{\pgfqpoint{2.604306in}{1.532746in}}%
\pgfpathlineto{\pgfqpoint{2.651003in}{1.510544in}}%
\pgfpathlineto{\pgfqpoint{2.698034in}{1.566456in}}%
\pgfpathlineto{\pgfqpoint{2.745668in}{1.534324in}}%
\pgfpathlineto{\pgfqpoint{2.790736in}{1.518697in}}%
\pgfpathlineto{\pgfqpoint{2.836381in}{1.494875in}}%
\pgfpathlineto{\pgfqpoint{2.883134in}{1.562943in}}%
\pgfpathlineto{\pgfqpoint{2.927413in}{1.605658in}}%
\pgfpathlineto{\pgfqpoint{2.971917in}{1.562850in}}%
\pgfpathlineto{\pgfqpoint{3.017486in}{1.566672in}}%
\pgfpathlineto{\pgfqpoint{3.062761in}{1.509831in}}%
\pgfpathlineto{\pgfqpoint{3.108180in}{1.547334in}}%
\pgfpathlineto{\pgfqpoint{3.155580in}{1.505975in}}%
\pgfpathlineto{\pgfqpoint{3.200519in}{1.611822in}}%
\pgfpathlineto{\pgfqpoint{3.245458in}{1.541177in}}%
\pgfpathlineto{\pgfqpoint{3.290748in}{1.578655in}}%
\pgfpathlineto{\pgfqpoint{3.336184in}{1.536065in}}%
\pgfpathlineto{\pgfqpoint{3.381274in}{1.580310in}}%
\pgfpathlineto{\pgfqpoint{3.428526in}{1.553507in}}%
\pgfpathlineto{\pgfqpoint{3.473135in}{1.528243in}}%
\pgfpathlineto{\pgfqpoint{3.517815in}{1.611411in}}%
\pgfpathlineto{\pgfqpoint{3.564489in}{1.568396in}}%
\pgfpathlineto{\pgfqpoint{3.609556in}{1.568303in}}%
\pgfpathlineto{\pgfqpoint{3.654909in}{1.568488in}}%
\pgfpathlineto{\pgfqpoint{3.701949in}{1.557015in}}%
\pgfpathlineto{\pgfqpoint{3.747249in}{1.563787in}}%
\pgfpathlineto{\pgfqpoint{3.793040in}{1.516540in}}%
\pgfpathlineto{\pgfqpoint{3.839775in}{1.560708in}}%
\pgfpathlineto{\pgfqpoint{3.884614in}{1.571975in}}%
\pgfpathlineto{\pgfqpoint{3.930307in}{1.533538in}}%
\pgfpathlineto{\pgfqpoint{3.976868in}{1.579625in}}%
\pgfpathlineto{\pgfqpoint{4.022115in}{1.551921in}}%
\pgfpathlineto{\pgfqpoint{4.067507in}{1.522087in}}%
\pgfpathlineto{\pgfqpoint{4.114362in}{1.536141in}}%
\pgfpathlineto{\pgfqpoint{4.159078in}{1.580009in}}%
\pgfpathlineto{\pgfqpoint{4.204543in}{1.546017in}}%
\pgfpathlineto{\pgfqpoint{4.252389in}{1.484992in}}%
\pgfpathlineto{\pgfqpoint{4.297354in}{1.585193in}}%
\pgfpathlineto{\pgfqpoint{4.343118in}{1.489350in}}%
\pgfpathlineto{\pgfqpoint{4.390187in}{1.572779in}}%
\pgfpathlineto{\pgfqpoint{4.435284in}{1.574393in}}%
\pgfpathlineto{\pgfqpoint{4.480684in}{1.471395in}}%
\pgfpathlineto{\pgfqpoint{4.527424in}{1.542123in}}%
\pgfpathlineto{\pgfqpoint{4.572708in}{1.484313in}}%
\pgfpathlineto{\pgfqpoint{4.617918in}{1.553545in}}%
\pgfpathlineto{\pgfqpoint{4.663622in}{1.547884in}}%
\pgfpathlineto{\pgfqpoint{4.708228in}{1.563118in}}%
\pgfpathlineto{\pgfqpoint{4.753449in}{1.570300in}}%
\pgfpathlineto{\pgfqpoint{4.799887in}{1.602354in}}%
\pgfpathlineto{\pgfqpoint{4.845429in}{1.586627in}}%
\pgfpathlineto{\pgfqpoint{4.891298in}{1.536251in}}%
\pgfpathlineto{\pgfqpoint{4.939032in}{1.498860in}}%
\pgfpathlineto{\pgfqpoint{4.984973in}{1.605568in}}%
\pgfpathlineto{\pgfqpoint{5.030678in}{1.549426in}}%
\pgfpathlineto{\pgfqpoint{5.078074in}{1.563684in}}%
\pgfpathlineto{\pgfqpoint{5.124036in}{1.496465in}}%
\pgfpathlineto{\pgfqpoint{5.169192in}{1.503848in}}%
\pgfpathlineto{\pgfqpoint{5.216291in}{1.516594in}}%
\pgfpathlineto{\pgfqpoint{5.262146in}{1.507684in}}%
\pgfpathlineto{\pgfqpoint{5.307244in}{1.578955in}}%
\pgfpathlineto{\pgfqpoint{5.353614in}{1.538996in}}%
\pgfpathlineto{\pgfqpoint{5.399334in}{1.590423in}}%
\pgfpathlineto{\pgfqpoint{5.444526in}{1.561463in}}%
\pgfpathlineto{\pgfqpoint{5.491128in}{1.607472in}}%
\pgfpathlineto{\pgfqpoint{5.536203in}{1.570565in}}%
\pgfpathlineto{\pgfqpoint{5.581803in}{1.528391in}}%
\pgfpathlineto{\pgfqpoint{5.628857in}{1.517186in}}%
\pgfpathlineto{\pgfqpoint{5.674262in}{1.567508in}}%
\pgfpathlineto{\pgfqpoint{5.719569in}{1.549158in}}%
\pgfpathlineto{\pgfqpoint{5.765813in}{1.564435in}}%
\pgfpathlineto{\pgfqpoint{5.810269in}{1.587129in}}%
\pgfpathlineto{\pgfqpoint{5.856181in}{1.516021in}}%
\pgfpathlineto{\pgfqpoint{5.903366in}{1.559782in}}%
\pgfpathlineto{\pgfqpoint{5.948991in}{1.525182in}}%
\pgfpathlineto{\pgfqpoint{5.994686in}{1.484192in}}%
\pgfpathlineto{\pgfqpoint{6.041630in}{1.591384in}}%
\pgfpathlineto{\pgfqpoint{6.087229in}{1.495876in}}%
\pgfpathlineto{\pgfqpoint{6.133235in}{1.552728in}}%
\pgfpathlineto{\pgfqpoint{6.180802in}{1.459558in}}%
\pgfpathlineto{\pgfqpoint{6.227112in}{1.515156in}}%
\pgfpathlineto{\pgfqpoint{6.273584in}{1.547490in}}%
\pgfpathlineto{\pgfqpoint{6.321070in}{1.486201in}}%
\pgfpathlineto{\pgfqpoint{6.367508in}{1.510727in}}%
\pgfpathlineto{\pgfqpoint{6.413399in}{1.533672in}}%
\pgfpathlineto{\pgfqpoint{6.461324in}{1.540551in}}%
\pgfpathlineto{\pgfqpoint{6.507651in}{1.500187in}}%
\pgfpathlineto{\pgfqpoint{6.553117in}{1.519916in}}%
\pgfpathlineto{\pgfqpoint{6.599302in}{1.570688in}}%
\pgfpathlineto{\pgfqpoint{6.643960in}{1.526847in}}%
\pgfpathlineto{\pgfqpoint{6.688504in}{1.551500in}}%
\pgfpathlineto{\pgfqpoint{6.734887in}{1.479787in}}%
\pgfpathlineto{\pgfqpoint{6.779295in}{1.574511in}}%
\pgfpathlineto{\pgfqpoint{6.824012in}{1.572692in}}%
\pgfpathlineto{\pgfqpoint{6.869544in}{1.547619in}}%
\pgfpathlineto{\pgfqpoint{6.914194in}{1.558499in}}%
\pgfpathlineto{\pgfqpoint{6.958763in}{1.576631in}}%
\pgfpathlineto{\pgfqpoint{7.005149in}{1.598260in}}%
\pgfpathlineto{\pgfqpoint{7.050071in}{1.581141in}}%
\pgfpathlineto{\pgfqpoint{7.094205in}{1.593184in}}%
\pgfpathlineto{\pgfqpoint{7.140134in}{1.583456in}}%
\pgfpathlineto{\pgfqpoint{7.184277in}{1.614228in}}%
\pgfpathlineto{\pgfqpoint{7.228120in}{1.557341in}}%
\pgfpathlineto{\pgfqpoint{7.273914in}{1.570827in}}%
\pgfpathlineto{\pgfqpoint{7.318484in}{1.554929in}}%
\pgfpathlineto{\pgfqpoint{7.363749in}{1.582469in}}%
\pgfpathlineto{\pgfqpoint{7.409977in}{1.568201in}}%
\pgfpathlineto{\pgfqpoint{7.455548in}{1.582574in}}%
\pgfpathlineto{\pgfqpoint{7.500683in}{1.584308in}}%
\pgfpathlineto{\pgfqpoint{7.546892in}{1.556177in}}%
\pgfpathlineto{\pgfqpoint{7.591890in}{1.560429in}}%
\pgfpathlineto{\pgfqpoint{7.636592in}{1.558451in}}%
\pgfpathlineto{\pgfqpoint{7.682978in}{1.581928in}}%
\pgfpathlineto{\pgfqpoint{7.728803in}{1.568185in}}%
\pgfpathlineto{\pgfqpoint{7.773226in}{1.615933in}}%
\pgfpathlineto{\pgfqpoint{7.819947in}{1.548381in}}%
\pgfpathlineto{\pgfqpoint{7.865263in}{1.525802in}}%
\pgfpathlineto{\pgfqpoint{7.910161in}{1.562561in}}%
\pgfpathlineto{\pgfqpoint{7.955879in}{1.537689in}}%
\pgfpathlineto{\pgfqpoint{8.000573in}{1.488688in}}%
\pgfpathlineto{\pgfqpoint{8.045278in}{1.601356in}}%
\pgfpathlineto{\pgfqpoint{8.091881in}{1.571345in}}%
\pgfpathlineto{\pgfqpoint{8.136842in}{1.541154in}}%
\pgfpathlineto{\pgfqpoint{8.181791in}{1.500494in}}%
\pgfpathlineto{\pgfqpoint{8.227930in}{1.552209in}}%
\pgfpathlineto{\pgfqpoint{8.273006in}{1.549999in}}%
\pgfpathlineto{\pgfqpoint{8.318789in}{1.548952in}}%
\pgfpathlineto{\pgfqpoint{8.364636in}{1.546337in}}%
\pgfpathlineto{\pgfqpoint{8.409767in}{1.569722in}}%
\pgfpathlineto{\pgfqpoint{8.453780in}{1.587303in}}%
\pgfpathlineto{\pgfqpoint{8.499298in}{1.509364in}}%
\pgfpathlineto{\pgfqpoint{8.543976in}{1.573395in}}%
\pgfpathlineto{\pgfqpoint{8.589124in}{1.511789in}}%
\pgfpathlineto{\pgfqpoint{8.635315in}{1.586244in}}%
\pgfpathlineto{\pgfqpoint{8.681080in}{1.495158in}}%
\pgfpathlineto{\pgfqpoint{8.727404in}{1.517910in}}%
\pgfpathlineto{\pgfqpoint{8.774902in}{1.505111in}}%
\pgfpathlineto{\pgfqpoint{8.821068in}{1.523053in}}%
\pgfpathlineto{\pgfqpoint{8.866242in}{1.528400in}}%
\pgfpathlineto{\pgfqpoint{8.912131in}{1.637496in}}%
\pgfpathlineto{\pgfqpoint{8.957924in}{1.566446in}}%
\pgfpathlineto{\pgfqpoint{9.003947in}{1.496155in}}%
\pgfpathlineto{\pgfqpoint{9.050165in}{1.585890in}}%
\pgfpathlineto{\pgfqpoint{9.095727in}{1.521941in}}%
\pgfpathlineto{\pgfqpoint{9.141270in}{1.506562in}}%
\pgfpathlineto{\pgfqpoint{9.188332in}{1.567783in}}%
\pgfpathlineto{\pgfqpoint{9.233107in}{1.582109in}}%
\pgfpathlineto{\pgfqpoint{9.278439in}{1.561329in}}%
\pgfpathlineto{\pgfqpoint{9.324423in}{1.592727in}}%
\pgfpathlineto{\pgfqpoint{9.369639in}{1.539785in}}%
\pgfpathlineto{\pgfqpoint{9.415178in}{1.559155in}}%
\pgfpathlineto{\pgfqpoint{9.462006in}{1.505764in}}%
\pgfpathlineto{\pgfqpoint{9.507583in}{1.530625in}}%
\pgfpathlineto{\pgfqpoint{9.552197in}{1.615870in}}%
\pgfpathlineto{\pgfqpoint{9.598338in}{1.554603in}}%
\pgfpathlineto{\pgfqpoint{9.644121in}{1.525747in}}%
\pgfpathlineto{\pgfqpoint{9.689497in}{1.563754in}}%
\pgfpathlineto{\pgfqpoint{9.735785in}{1.571327in}}%
\pgfpathlineto{\pgfqpoint{9.781321in}{1.576866in}}%
\pgfpathlineto{\pgfqpoint{9.826599in}{1.550244in}}%
\pgfpathlineto{\pgfqpoint{9.873173in}{1.536439in}}%
\pgfpathlineto{\pgfqpoint{9.918842in}{1.503684in}}%
\pgfpathlineto{\pgfqpoint{9.964562in}{1.544111in}}%
\pgfpathlineto{\pgfqpoint{10.011915in}{1.505468in}}%
\pgfpathlineto{\pgfqpoint{10.057613in}{1.520968in}}%
\pgfpathlineto{\pgfqpoint{10.103098in}{1.503213in}}%
\pgfpathlineto{\pgfqpoint{10.151348in}{1.521257in}}%
\pgfpathlineto{\pgfqpoint{10.197247in}{1.584503in}}%
\pgfpathlineto{\pgfqpoint{10.242601in}{1.498136in}}%
\pgfpathlineto{\pgfqpoint{10.289084in}{1.505217in}}%
\pgfpathlineto{\pgfqpoint{10.335334in}{1.490875in}}%
\pgfpathlineto{\pgfqpoint{10.381570in}{1.532981in}}%
\pgfpathlineto{\pgfqpoint{10.429362in}{1.541046in}}%
\pgfpathlineto{\pgfqpoint{10.475642in}{1.538791in}}%
\pgfpathlineto{\pgfqpoint{10.521105in}{1.508048in}}%
\pgfpathlineto{\pgfqpoint{10.568154in}{1.556578in}}%
\pgfpathlineto{\pgfqpoint{10.613886in}{1.487792in}}%
\pgfpathlineto{\pgfqpoint{10.660112in}{1.477165in}}%
\pgfpathlineto{\pgfqpoint{10.707461in}{1.555544in}}%
\pgfpathlineto{\pgfqpoint{10.753168in}{1.538547in}}%
\pgfpathlineto{\pgfqpoint{10.798768in}{1.525837in}}%
\pgfpathlineto{\pgfqpoint{10.845790in}{1.525429in}}%
\pgfpathlineto{\pgfqpoint{10.891801in}{1.528621in}}%
\pgfpathlineto{\pgfqpoint{10.937381in}{1.533498in}}%
\pgfpathlineto{\pgfqpoint{10.984145in}{1.564078in}}%
\pgfpathlineto{\pgfqpoint{11.029398in}{1.566477in}}%
\pgfpathlineto{\pgfqpoint{11.074633in}{1.559755in}}%
\pgfpathlineto{\pgfqpoint{11.121978in}{1.486227in}}%
\pgfpathlineto{\pgfqpoint{11.167471in}{1.582735in}}%
\pgfpathlineto{\pgfqpoint{11.213355in}{1.501424in}}%
\pgfpathlineto{\pgfqpoint{11.261350in}{1.529538in}}%
\pgfpathlineto{\pgfqpoint{11.307361in}{1.601051in}}%
\pgfpathlineto{\pgfqpoint{11.353756in}{1.545928in}}%
\pgfpathlineto{\pgfqpoint{11.402044in}{1.542273in}}%
\pgfpathlineto{\pgfqpoint{11.448531in}{1.515160in}}%
\pgfpathlineto{\pgfqpoint{11.494874in}{1.564506in}}%
\pgfpathlineto{\pgfqpoint{11.542665in}{1.579485in}}%
\pgfpathlineto{\pgfqpoint{11.588799in}{1.519041in}}%
\pgfpathlineto{\pgfqpoint{11.634589in}{1.509051in}}%
\pgfpathlineto{\pgfqpoint{11.682053in}{1.578824in}}%
\pgfpathlineto{\pgfqpoint{11.727607in}{1.565494in}}%
\pgfpathlineto{\pgfqpoint{11.772894in}{1.515236in}}%
\pgfpathlineto{\pgfqpoint{11.819406in}{1.563906in}}%
\pgfpathlineto{\pgfqpoint{11.864862in}{1.556197in}}%
\pgfpathlineto{\pgfqpoint{11.911501in}{1.512933in}}%
\pgfpathlineto{\pgfqpoint{11.959143in}{1.570733in}}%
\pgfpathlineto{\pgfqpoint{12.004414in}{1.529616in}}%
\pgfpathlineto{\pgfqpoint{12.049869in}{1.537005in}}%
\pgfpathlineto{\pgfqpoint{12.096623in}{1.537181in}}%
\pgfpathlineto{\pgfqpoint{12.141887in}{1.577279in}}%
\pgfpathlineto{\pgfqpoint{12.187925in}{1.534904in}}%
\pgfpathlineto{\pgfqpoint{12.235985in}{1.482695in}}%
\pgfpathlineto{\pgfqpoint{12.282271in}{1.510600in}}%
\pgfpathlineto{\pgfqpoint{12.328136in}{1.497095in}}%
\pgfpathlineto{\pgfqpoint{12.375261in}{1.524465in}}%
\pgfpathlineto{\pgfqpoint{12.421372in}{1.539521in}}%
\pgfpathlineto{\pgfqpoint{12.467453in}{1.508554in}}%
\pgfpathlineto{\pgfqpoint{12.514481in}{1.594296in}}%
\pgfpathlineto{\pgfqpoint{12.560711in}{1.464062in}}%
\pgfpathlineto{\pgfqpoint{12.607615in}{1.536917in}}%
\pgfpathlineto{\pgfqpoint{12.655424in}{1.506971in}}%
\pgfpathlineto{\pgfqpoint{12.701835in}{1.546420in}}%
\pgfpathlineto{\pgfqpoint{12.749025in}{1.508387in}}%
\pgfpathlineto{\pgfqpoint{12.796907in}{1.517135in}}%
\pgfpathlineto{\pgfqpoint{12.843155in}{1.490534in}}%
\pgfpathlineto{\pgfqpoint{12.889988in}{1.540314in}}%
\pgfpathlineto{\pgfqpoint{12.936799in}{1.561665in}}%
\pgfpathlineto{\pgfqpoint{12.983596in}{1.517211in}}%
\pgfpathlineto{\pgfqpoint{13.030499in}{1.505766in}}%
\pgfpathlineto{\pgfqpoint{13.077275in}{1.575979in}}%
\pgfpathlineto{\pgfqpoint{13.123339in}{1.500250in}}%
\pgfpathlineto{\pgfqpoint{13.169075in}{1.552921in}}%
\pgfpathlineto{\pgfqpoint{13.216719in}{1.470586in}}%
\pgfpathlineto{\pgfqpoint{13.262980in}{1.568872in}}%
\pgfpathlineto{\pgfqpoint{13.308974in}{1.527229in}}%
\pgfpathlineto{\pgfqpoint{13.356518in}{1.574697in}}%
\pgfpathlineto{\pgfqpoint{13.401859in}{1.547631in}}%
\pgfpathlineto{\pgfqpoint{13.448069in}{1.521523in}}%
\pgfpathlineto{\pgfqpoint{13.495727in}{1.564865in}}%
\pgfpathlineto{\pgfqpoint{13.541646in}{1.519941in}}%
\pgfpathlineto{\pgfqpoint{13.587249in}{1.524477in}}%
\pgfpathlineto{\pgfqpoint{13.635346in}{1.521893in}}%
\pgfpathlineto{\pgfqpoint{13.682151in}{1.501355in}}%
\pgfpathlineto{\pgfqpoint{13.729143in}{1.534486in}}%
\pgfpathlineto{\pgfqpoint{13.777441in}{1.505462in}}%
\pgfpathlineto{\pgfqpoint{13.825297in}{1.477918in}}%
\pgfpathlineto{\pgfqpoint{13.873610in}{1.498642in}}%
\pgfpathlineto{\pgfqpoint{13.922606in}{1.475342in}}%
\pgfpathlineto{\pgfqpoint{13.969353in}{1.492281in}}%
\pgfpathlineto{\pgfqpoint{14.016344in}{1.538810in}}%
\pgfpathlineto{\pgfqpoint{14.064951in}{1.526672in}}%
\pgfpathlineto{\pgfqpoint{14.112147in}{1.495848in}}%
\pgfpathlineto{\pgfqpoint{14.160147in}{1.488189in}}%
\pgfpathlineto{\pgfqpoint{14.209376in}{1.498058in}}%
\pgfpathlineto{\pgfqpoint{14.255593in}{1.552002in}}%
\pgfpathlineto{\pgfqpoint{14.302418in}{1.547427in}}%
\pgfpathlineto{\pgfqpoint{14.350224in}{1.535720in}}%
\pgfpathlineto{\pgfqpoint{14.397126in}{1.541511in}}%
\pgfpathlineto{\pgfqpoint{14.443729in}{1.563335in}}%
\pgfpathlineto{\pgfqpoint{14.491726in}{1.506448in}}%
\pgfpathlineto{\pgfqpoint{14.538630in}{1.531812in}}%
\pgfpathlineto{\pgfqpoint{14.585477in}{1.511082in}}%
\pgfpathlineto{\pgfqpoint{14.633992in}{1.497269in}}%
\pgfpathlineto{\pgfqpoint{14.680226in}{1.518840in}}%
\pgfpathlineto{\pgfqpoint{14.726757in}{1.507421in}}%
\pgfpathlineto{\pgfqpoint{14.774752in}{1.538923in}}%
\pgfpathlineto{\pgfqpoint{14.821474in}{1.591513in}}%
\pgfpathlineto{\pgfqpoint{14.868242in}{1.548620in}}%
\pgfpathlineto{\pgfqpoint{14.916239in}{1.491863in}}%
\pgfpathlineto{\pgfqpoint{14.963167in}{1.469295in}}%
\pgfpathlineto{\pgfqpoint{15.010214in}{1.546708in}}%
\pgfpathlineto{\pgfqpoint{15.058774in}{1.476105in}}%
\pgfpathlineto{\pgfqpoint{15.105478in}{1.565050in}}%
\pgfpathlineto{\pgfqpoint{15.152658in}{1.543727in}}%
\pgfpathlineto{\pgfqpoint{15.201456in}{1.483789in}}%
\pgfpathlineto{\pgfqpoint{15.248863in}{1.498469in}}%
\pgfpathlineto{\pgfqpoint{15.295866in}{1.546965in}}%
\pgfpathlineto{\pgfqpoint{15.343883in}{1.544535in}}%
\pgfpathlineto{\pgfqpoint{15.389896in}{1.561039in}}%
\pgfpathlineto{\pgfqpoint{15.436052in}{1.517126in}}%
\pgfpathlineto{\pgfqpoint{15.484858in}{1.500648in}}%
\pgfpathlineto{\pgfqpoint{15.532329in}{1.492190in}}%
\pgfpathlineto{\pgfqpoint{15.579815in}{1.510925in}}%
\pgfpathlineto{\pgfqpoint{15.628660in}{1.494626in}}%
\pgfpathlineto{\pgfqpoint{15.676019in}{1.540627in}}%
\pgfpathlineto{\pgfqpoint{15.722814in}{1.540818in}}%
\pgfpathlineto{\pgfqpoint{15.770154in}{1.534549in}}%
\pgfpathlineto{\pgfqpoint{15.817273in}{1.508377in}}%
\pgfpathlineto{\pgfqpoint{15.863592in}{1.514360in}}%
\pgfpathlineto{\pgfqpoint{15.911811in}{1.516610in}}%
\pgfpathlineto{\pgfqpoint{15.957799in}{1.581828in}}%
\pgfpathlineto{\pgfqpoint{16.003926in}{1.547024in}}%
\pgfpathlineto{\pgfqpoint{16.051820in}{1.553606in}}%
\pgfpathlineto{\pgfqpoint{16.098697in}{1.497695in}}%
\pgfpathlineto{\pgfqpoint{16.146338in}{1.480017in}}%
\pgfpathlineto{\pgfqpoint{16.195175in}{1.518044in}}%
\pgfpathlineto{\pgfqpoint{16.242078in}{1.524677in}}%
\pgfpathlineto{\pgfqpoint{16.289817in}{1.480735in}}%
\pgfpathlineto{\pgfqpoint{16.339537in}{1.490275in}}%
\pgfpathlineto{\pgfqpoint{16.387666in}{1.454859in}}%
\pgfpathlineto{\pgfqpoint{16.436622in}{1.435421in}}%
\pgfpathlineto{\pgfqpoint{16.486560in}{1.479180in}}%
\pgfpathlineto{\pgfqpoint{16.533619in}{1.517671in}}%
\pgfpathlineto{\pgfqpoint{16.581135in}{1.493295in}}%
\pgfpathlineto{\pgfqpoint{16.630280in}{1.488775in}}%
\pgfpathlineto{\pgfqpoint{16.677566in}{1.538170in}}%
\pgfpathlineto{\pgfqpoint{16.724568in}{1.517676in}}%
\pgfpathlineto{\pgfqpoint{16.773713in}{1.503750in}}%
\pgfpathlineto{\pgfqpoint{16.821038in}{1.543591in}}%
\pgfpathlineto{\pgfqpoint{16.868266in}{1.525440in}}%
\pgfpathlineto{\pgfqpoint{16.916302in}{1.569257in}}%
\pgfpathlineto{\pgfqpoint{16.963533in}{1.531321in}}%
\pgfpathlineto{\pgfqpoint{17.010585in}{1.515163in}}%
\pgfpathlineto{\pgfqpoint{17.059518in}{1.482082in}}%
\pgfpathlineto{\pgfqpoint{17.107470in}{1.454535in}}%
\pgfpathlineto{\pgfqpoint{17.154507in}{1.527718in}}%
\pgfpathlineto{\pgfqpoint{17.203674in}{1.516527in}}%
\pgfpathlineto{\pgfqpoint{17.250860in}{1.488999in}}%
\pgfpathlineto{\pgfqpoint{17.298941in}{1.495207in}}%
\pgfpathlineto{\pgfqpoint{17.347576in}{1.535034in}}%
\pgfpathlineto{\pgfqpoint{17.394714in}{1.477277in}}%
\pgfpathlineto{\pgfqpoint{17.442362in}{1.481885in}}%
\pgfpathlineto{\pgfqpoint{17.491125in}{1.537184in}}%
\pgfpathlineto{\pgfqpoint{17.538409in}{1.523829in}}%
\pgfpathlineto{\pgfqpoint{17.585742in}{1.600814in}}%
\pgfpathlineto{\pgfqpoint{17.634653in}{1.506229in}}%
\pgfpathlineto{\pgfqpoint{17.681914in}{1.583122in}}%
\pgfpathlineto{\pgfqpoint{17.729727in}{1.487892in}}%
\pgfpathlineto{\pgfqpoint{17.779014in}{1.444273in}}%
\pgfpathlineto{\pgfqpoint{17.826809in}{1.467132in}}%
\pgfpathlineto{\pgfqpoint{17.874600in}{1.523670in}}%
\pgfpathlineto{\pgfqpoint{17.922885in}{1.553495in}}%
\pgfpathlineto{\pgfqpoint{17.970910in}{1.489846in}}%
\pgfpathlineto{\pgfqpoint{18.020026in}{1.421714in}}%
\pgfpathlineto{\pgfqpoint{18.069524in}{1.486525in}}%
\pgfpathlineto{\pgfqpoint{18.117307in}{1.467506in}}%
\pgfpathlineto{\pgfqpoint{18.164405in}{1.513989in}}%
\pgfpathlineto{\pgfqpoint{18.213333in}{1.465073in}}%
\pgfpathlineto{\pgfqpoint{18.260871in}{1.510613in}}%
\pgfpathlineto{\pgfqpoint{18.308441in}{1.541367in}}%
\pgfpathlineto{\pgfqpoint{18.357538in}{1.461378in}}%
\pgfpathlineto{\pgfqpoint{18.405242in}{1.488980in}}%
\pgfpathlineto{\pgfqpoint{18.452284in}{1.468578in}}%
\pgfpathlineto{\pgfqpoint{18.500915in}{1.487375in}}%
\pgfpathlineto{\pgfqpoint{18.548460in}{1.506176in}}%
\pgfpathlineto{\pgfqpoint{18.595724in}{1.507826in}}%
\pgfpathlineto{\pgfqpoint{18.644639in}{1.526806in}}%
\pgfpathlineto{\pgfqpoint{18.693122in}{1.465243in}}%
\pgfpathlineto{\pgfqpoint{18.741665in}{1.505364in}}%
\pgfpathlineto{\pgfqpoint{18.791212in}{1.444044in}}%
\pgfpathlineto{\pgfqpoint{18.839312in}{1.511409in}}%
\pgfpathlineto{\pgfqpoint{18.888188in}{1.487581in}}%
\pgfpathlineto{\pgfqpoint{18.937671in}{1.506432in}}%
\pgfpathlineto{\pgfqpoint{18.985002in}{1.494968in}}%
\pgfpathlineto{\pgfqpoint{19.033794in}{1.491809in}}%
\pgfpathlineto{\pgfqpoint{19.084207in}{1.400470in}}%
\pgfpathlineto{\pgfqpoint{19.132724in}{1.498695in}}%
\pgfpathlineto{\pgfqpoint{19.181302in}{1.471644in}}%
\pgfpathlineto{\pgfqpoint{19.231826in}{1.452129in}}%
\pgfpathlineto{\pgfqpoint{19.279742in}{1.554138in}}%
\pgfpathlineto{\pgfqpoint{19.327829in}{1.473298in}}%
\pgfpathlineto{\pgfqpoint{19.376890in}{1.478216in}}%
\pgfpathlineto{\pgfqpoint{19.425178in}{1.506046in}}%
\pgfpathlineto{\pgfqpoint{19.473154in}{1.482732in}}%
\pgfpathlineto{\pgfqpoint{19.522292in}{1.513186in}}%
\pgfpathlineto{\pgfqpoint{19.569747in}{1.557243in}}%
\pgfpathlineto{\pgfqpoint{19.617707in}{1.483391in}}%
\pgfpathlineto{\pgfqpoint{19.666940in}{1.511122in}}%
\pgfpathlineto{\pgfqpoint{19.714265in}{1.525351in}}%
\pgfpathlineto{\pgfqpoint{19.761678in}{1.504500in}}%
\pgfpathlineto{\pgfqpoint{19.810712in}{1.497489in}}%
\pgfpathlineto{\pgfqpoint{19.858933in}{1.513257in}}%
\pgfpathlineto{\pgfqpoint{19.906615in}{1.532652in}}%
\pgfpathlineto{\pgfqpoint{19.955252in}{1.496075in}}%
\pgfpathlineto{\pgfqpoint{20.003692in}{1.477769in}}%
\pgfpathlineto{\pgfqpoint{20.053312in}{1.505370in}}%
\pgfpathlineto{\pgfqpoint{20.104350in}{1.450999in}}%
\pgfpathlineto{\pgfqpoint{20.153679in}{1.475407in}}%
\pgfpathlineto{\pgfqpoint{20.203615in}{1.498954in}}%
\pgfpathlineto{\pgfqpoint{20.255593in}{1.453003in}}%
\pgfpathlineto{\pgfqpoint{20.305924in}{1.483280in}}%
\pgfpathlineto{\pgfqpoint{20.355462in}{1.455924in}}%
\pgfpathlineto{\pgfqpoint{20.406414in}{1.508474in}}%
\pgfpathlineto{\pgfqpoint{20.455604in}{1.459393in}}%
\pgfpathlineto{\pgfqpoint{20.505208in}{1.523941in}}%
\pgfpathlineto{\pgfqpoint{20.556290in}{1.472967in}}%
\pgfpathlineto{\pgfqpoint{20.605567in}{1.503138in}}%
\pgfpathlineto{\pgfqpoint{20.654668in}{1.483979in}}%
\pgfpathlineto{\pgfqpoint{20.704616in}{1.507557in}}%
\pgfpathlineto{\pgfqpoint{20.753558in}{1.504972in}}%
\pgfpathlineto{\pgfqpoint{20.802965in}{1.588274in}}%
\pgfpathlineto{\pgfqpoint{20.854118in}{1.455757in}}%
\pgfpathlineto{\pgfqpoint{20.904164in}{1.466930in}}%
\pgfpathlineto{\pgfqpoint{20.953585in}{1.516103in}}%
\pgfpathlineto{\pgfqpoint{21.003693in}{1.450949in}}%
\pgfpathlineto{\pgfqpoint{21.053515in}{1.437617in}}%
\pgfpathlineto{\pgfqpoint{21.102370in}{1.514168in}}%
\pgfpathlineto{\pgfqpoint{21.152719in}{1.495756in}}%
\pgfpathlineto{\pgfqpoint{21.201683in}{1.483018in}}%
\pgfpathlineto{\pgfqpoint{21.250655in}{1.492178in}}%
\pgfpathlineto{\pgfqpoint{21.300955in}{1.450986in}}%
\pgfpathlineto{\pgfqpoint{21.350353in}{1.486772in}}%
\pgfpathlineto{\pgfqpoint{21.399733in}{1.454686in}}%
\pgfpathlineto{\pgfqpoint{21.450830in}{1.462158in}}%
\pgfpathlineto{\pgfqpoint{21.501023in}{1.472200in}}%
\pgfpathlineto{\pgfqpoint{21.550957in}{1.489803in}}%
\pgfpathlineto{\pgfqpoint{21.602538in}{1.472393in}}%
\pgfpathlineto{\pgfqpoint{21.652276in}{1.486753in}}%
\pgfpathlineto{\pgfqpoint{21.700864in}{1.492572in}}%
\pgfpathlineto{\pgfqpoint{21.751624in}{1.485587in}}%
\pgfpathlineto{\pgfqpoint{21.800474in}{1.481533in}}%
\pgfpathlineto{\pgfqpoint{21.849580in}{1.518978in}}%
\pgfpathlineto{\pgfqpoint{21.900834in}{1.484586in}}%
\pgfpathlineto{\pgfqpoint{21.950544in}{1.497388in}}%
\pgfpathlineto{\pgfqpoint{21.999803in}{1.514535in}}%
\pgfpathlineto{\pgfqpoint{22.050354in}{1.462006in}}%
\pgfpathlineto{\pgfqpoint{22.099907in}{1.528125in}}%
\pgfpathlineto{\pgfqpoint{22.148817in}{1.527757in}}%
\pgfpathlineto{\pgfqpoint{22.200587in}{1.451044in}}%
\pgfpathlineto{\pgfqpoint{22.250634in}{1.474893in}}%
\pgfpathlineto{\pgfqpoint{22.300933in}{1.455792in}}%
\pgfpathlineto{\pgfqpoint{22.352827in}{1.468034in}}%
\pgfpathlineto{\pgfqpoint{22.402084in}{1.439666in}}%
\pgfpathlineto{\pgfqpoint{22.451976in}{1.474127in}}%
\pgfpathlineto{\pgfqpoint{22.503552in}{1.489534in}}%
\pgfpathlineto{\pgfqpoint{22.554073in}{1.477021in}}%
\pgfpathlineto{\pgfqpoint{22.603657in}{1.492179in}}%
\pgfpathlineto{\pgfqpoint{22.654822in}{1.481284in}}%
\pgfpathlineto{\pgfqpoint{22.703398in}{1.564313in}}%
\pgfpathlineto{\pgfqpoint{22.751700in}{1.515947in}}%
\pgfpathlineto{\pgfqpoint{22.803829in}{1.519235in}}%
\pgfpathlineto{\pgfqpoint{22.853966in}{1.462154in}}%
\pgfpathlineto{\pgfqpoint{22.904236in}{1.432232in}}%
\pgfpathlineto{\pgfqpoint{22.956359in}{1.491616in}}%
\pgfpathlineto{\pgfqpoint{23.006022in}{1.506808in}}%
\pgfpathlineto{\pgfqpoint{23.055187in}{1.520456in}}%
\pgfpathlineto{\pgfqpoint{23.107134in}{1.431065in}}%
\pgfpathlineto{\pgfqpoint{23.157756in}{1.437185in}}%
\pgfpathlineto{\pgfqpoint{23.208694in}{1.458867in}}%
\pgfpathlineto{\pgfqpoint{23.260726in}{1.491507in}}%
\pgfpathlineto{\pgfqpoint{23.310918in}{1.478442in}}%
\pgfpathlineto{\pgfqpoint{23.361310in}{1.464609in}}%
\pgfpathlineto{\pgfqpoint{23.411909in}{1.458922in}}%
\pgfpathlineto{\pgfqpoint{23.462017in}{1.472289in}}%
\pgfpathlineto{\pgfqpoint{23.511867in}{1.483293in}}%
\pgfpathlineto{\pgfqpoint{23.563429in}{1.484534in}}%
\pgfpathlineto{\pgfqpoint{23.611923in}{1.502550in}}%
\pgfpathlineto{\pgfqpoint{23.661575in}{1.457956in}}%
\pgfpathlineto{\pgfqpoint{23.713074in}{1.492267in}}%
\pgfpathlineto{\pgfqpoint{23.762329in}{1.518819in}}%
\pgfpathlineto{\pgfqpoint{23.811512in}{1.487867in}}%
\pgfpathlineto{\pgfqpoint{23.862673in}{1.503511in}}%
\pgfpathlineto{\pgfqpoint{23.913088in}{1.471046in}}%
\pgfpathlineto{\pgfqpoint{23.964514in}{1.429663in}}%
\pgfpathlineto{\pgfqpoint{24.016442in}{1.474912in}}%
\pgfpathlineto{\pgfqpoint{24.066933in}{1.486397in}}%
\pgfpathlineto{\pgfqpoint{24.118637in}{1.404869in}}%
\pgfpathlineto{\pgfqpoint{24.170873in}{1.424458in}}%
\pgfpathlineto{\pgfqpoint{24.221010in}{1.466875in}}%
\pgfpathlineto{\pgfqpoint{24.271277in}{1.436571in}}%
\pgfpathlineto{\pgfqpoint{24.321907in}{1.496870in}}%
\pgfpathlineto{\pgfqpoint{24.371890in}{1.479833in}}%
\pgfpathlineto{\pgfqpoint{24.422475in}{1.471589in}}%
\pgfpathlineto{\pgfqpoint{24.474381in}{1.459339in}}%
\pgfpathlineto{\pgfqpoint{24.524258in}{1.479312in}}%
\pgfpathlineto{\pgfqpoint{24.573998in}{1.488680in}}%
\pgfpathlineto{\pgfqpoint{24.626556in}{1.429175in}}%
\pgfpathlineto{\pgfqpoint{24.676389in}{1.469206in}}%
\pgfpathlineto{\pgfqpoint{24.726499in}{1.497794in}}%
\pgfpathlineto{\pgfqpoint{24.778304in}{1.480893in}}%
\pgfpathlineto{\pgfqpoint{24.828140in}{1.515156in}}%
\pgfpathlineto{\pgfqpoint{24.878143in}{1.454865in}}%
\pgfpathlineto{\pgfqpoint{24.928869in}{1.511209in}}%
\pgfpathlineto{\pgfqpoint{24.978211in}{1.501248in}}%
\pgfpathlineto{\pgfqpoint{25.028630in}{1.453886in}}%
\pgfpathlineto{\pgfqpoint{25.080539in}{1.468689in}}%
\pgfpathlineto{\pgfqpoint{25.131043in}{1.483945in}}%
\pgfpathlineto{\pgfqpoint{25.181554in}{1.480570in}}%
\pgfpathlineto{\pgfqpoint{25.232944in}{1.492010in}}%
\pgfpathlineto{\pgfqpoint{25.282222in}{1.509514in}}%
\pgfpathlineto{\pgfqpoint{25.332608in}{1.455526in}}%
\pgfpathlineto{\pgfqpoint{25.383622in}{1.507858in}}%
\pgfpathlineto{\pgfqpoint{25.433954in}{1.479729in}}%
\pgfpathlineto{\pgfqpoint{25.483381in}{1.476461in}}%
\pgfpathlineto{\pgfqpoint{25.535558in}{1.460130in}}%
\pgfpathlineto{\pgfqpoint{25.586387in}{1.430021in}}%
\pgfpathlineto{\pgfqpoint{25.637513in}{1.447750in}}%
\pgfpathlineto{\pgfqpoint{25.689152in}{1.431142in}}%
\pgfpathlineto{\pgfqpoint{25.740073in}{1.416044in}}%
\pgfpathlineto{\pgfqpoint{25.790287in}{1.509874in}}%
\pgfpathlineto{\pgfqpoint{25.841969in}{1.510289in}}%
\pgfpathlineto{\pgfqpoint{25.891991in}{1.472900in}}%
\pgfpathlineto{\pgfqpoint{25.942005in}{1.461304in}}%
\pgfpathlineto{\pgfqpoint{25.994130in}{1.499292in}}%
\pgfpathlineto{\pgfqpoint{26.044216in}{1.462308in}}%
\pgfpathlineto{\pgfqpoint{26.093895in}{1.464323in}}%
\pgfpathlineto{\pgfqpoint{26.144837in}{1.485889in}}%
\pgfpathlineto{\pgfqpoint{26.195263in}{1.454513in}}%
\pgfpathlineto{\pgfqpoint{26.245717in}{1.478346in}}%
\pgfpathlineto{\pgfqpoint{26.297518in}{1.464632in}}%
\pgfpathlineto{\pgfqpoint{26.346588in}{1.496304in}}%
\pgfpathlineto{\pgfqpoint{26.395829in}{1.491356in}}%
\pgfpathlineto{\pgfqpoint{26.447655in}{1.489283in}}%
\pgfpathlineto{\pgfqpoint{26.499064in}{1.401061in}}%
\pgfpathlineto{\pgfqpoint{26.549446in}{1.488861in}}%
\pgfpathlineto{\pgfqpoint{26.601868in}{1.441924in}}%
\pgfpathlineto{\pgfqpoint{26.653161in}{1.477742in}}%
\pgfpathlineto{\pgfqpoint{26.704176in}{1.474687in}}%
\pgfpathlineto{\pgfqpoint{26.756448in}{1.466846in}}%
\pgfpathlineto{\pgfqpoint{26.807173in}{1.454026in}}%
\pgfpathlineto{\pgfqpoint{26.857706in}{1.441237in}}%
\pgfpathlineto{\pgfqpoint{26.909530in}{1.498993in}}%
\pgfpathlineto{\pgfqpoint{26.960748in}{1.481435in}}%
\pgfpathlineto{\pgfqpoint{27.010539in}{1.518800in}}%
\pgfpathlineto{\pgfqpoint{27.063062in}{1.476219in}}%
\pgfpathlineto{\pgfqpoint{27.113950in}{1.494852in}}%
\pgfpathlineto{\pgfqpoint{27.165274in}{1.437036in}}%
\pgfpathlineto{\pgfqpoint{27.218587in}{1.453594in}}%
\pgfpathlineto{\pgfqpoint{27.269679in}{1.440422in}}%
\pgfpathlineto{\pgfqpoint{27.321160in}{1.441672in}}%
\pgfpathlineto{\pgfqpoint{27.372863in}{1.514552in}}%
\pgfpathlineto{\pgfqpoint{27.423066in}{1.471617in}}%
\pgfpathlineto{\pgfqpoint{27.473173in}{1.466522in}}%
\pgfpathlineto{\pgfqpoint{27.525849in}{1.433163in}}%
\pgfpathlineto{\pgfqpoint{27.576178in}{1.467419in}}%
\pgfpathlineto{\pgfqpoint{27.626352in}{1.461354in}}%
\pgfpathlineto{\pgfqpoint{27.678294in}{1.439046in}}%
\pgfpathlineto{\pgfqpoint{27.728571in}{1.497292in}}%
\pgfpathlineto{\pgfqpoint{27.779331in}{1.453054in}}%
\pgfpathlineto{\pgfqpoint{27.833067in}{1.430942in}}%
\pgfpathlineto{\pgfqpoint{27.883786in}{1.447905in}}%
\pgfpathlineto{\pgfqpoint{27.935127in}{1.426934in}}%
\pgfpathlineto{\pgfqpoint{27.988562in}{1.463974in}}%
\pgfpathlineto{\pgfqpoint{28.039571in}{1.470166in}}%
\pgfpathlineto{\pgfqpoint{28.090650in}{1.466132in}}%
\pgfpathlineto{\pgfqpoint{28.143944in}{1.437680in}}%
\pgfpathlineto{\pgfqpoint{28.195034in}{1.465843in}}%
\pgfpathlineto{\pgfqpoint{28.245320in}{1.463950in}}%
\pgfpathlineto{\pgfqpoint{28.297462in}{1.516530in}}%
\pgfpathlineto{\pgfqpoint{28.347853in}{1.472000in}}%
\pgfpathlineto{\pgfqpoint{28.399138in}{1.435521in}}%
\pgfpathlineto{\pgfqpoint{28.451952in}{1.413812in}}%
\pgfpathlineto{\pgfqpoint{28.502707in}{1.476075in}}%
\pgfpathlineto{\pgfqpoint{28.553749in}{1.468684in}}%
\pgfpathlineto{\pgfqpoint{28.605351in}{1.454000in}}%
\pgfpathlineto{\pgfqpoint{28.656185in}{1.495303in}}%
\pgfpathlineto{\pgfqpoint{28.706300in}{1.488245in}}%
\pgfpathlineto{\pgfqpoint{28.758074in}{1.469018in}}%
\pgfpathlineto{\pgfqpoint{28.808587in}{1.493059in}}%
\pgfpathlineto{\pgfqpoint{28.860347in}{1.440938in}}%
\pgfpathlineto{\pgfqpoint{28.913538in}{1.422930in}}%
\pgfpathlineto{\pgfqpoint{28.965063in}{1.466028in}}%
\pgfpathlineto{\pgfqpoint{29.015651in}{1.498293in}}%
\pgfpathlineto{\pgfqpoint{29.068863in}{1.466339in}}%
\pgfpathlineto{\pgfqpoint{29.121029in}{1.396378in}}%
\pgfpathlineto{\pgfqpoint{29.173700in}{1.453336in}}%
\pgfpathlineto{\pgfqpoint{29.226818in}{1.480796in}}%
\pgfpathlineto{\pgfqpoint{29.279399in}{1.429412in}}%
\pgfpathlineto{\pgfqpoint{29.332261in}{1.459502in}}%
\pgfpathlineto{\pgfqpoint{29.386173in}{1.474166in}}%
\pgfpathlineto{\pgfqpoint{29.438335in}{1.450584in}}%
\pgfpathlineto{\pgfqpoint{29.490075in}{1.448577in}}%
\pgfpathlineto{\pgfqpoint{29.543760in}{1.435204in}}%
\pgfpathlineto{\pgfqpoint{29.596078in}{1.401838in}}%
\pgfpathlineto{\pgfqpoint{29.647988in}{1.476625in}}%
\pgfpathlineto{\pgfqpoint{29.701409in}{1.464372in}}%
\pgfpathlineto{\pgfqpoint{29.753093in}{1.467650in}}%
\pgfpathlineto{\pgfqpoint{29.805208in}{1.447695in}}%
\pgfpathlineto{\pgfqpoint{29.858913in}{1.411402in}}%
\pgfpathlineto{\pgfqpoint{29.910558in}{1.450867in}}%
\pgfpathlineto{\pgfqpoint{29.962066in}{1.466086in}}%
\pgfpathlineto{\pgfqpoint{30.014968in}{1.496976in}}%
\pgfpathlineto{\pgfqpoint{30.066247in}{1.464706in}}%
\pgfpathlineto{\pgfqpoint{30.117082in}{1.447685in}}%
\pgfpathlineto{\pgfqpoint{30.169497in}{1.447263in}}%
\pgfpathlineto{\pgfqpoint{30.221783in}{1.411341in}}%
\pgfpathlineto{\pgfqpoint{30.276704in}{1.374681in}}%
\pgfpathlineto{\pgfqpoint{30.337041in}{1.372834in}}%
\pgfpathlineto{\pgfqpoint{30.397563in}{1.326296in}}%
\pgfpathlineto{\pgfqpoint{30.457369in}{1.372779in}}%
\pgfpathlineto{\pgfqpoint{30.519963in}{1.333525in}}%
\pgfpathlineto{\pgfqpoint{30.582999in}{1.305125in}}%
\pgfpathlineto{\pgfqpoint{30.648377in}{1.321329in}}%
\pgfpathlineto{\pgfqpoint{30.717993in}{1.249212in}}%
\pgfpathlineto{\pgfqpoint{30.786068in}{1.299568in}}%
\pgfpathlineto{\pgfqpoint{30.855684in}{1.273397in}}%
\pgfpathlineto{\pgfqpoint{30.928370in}{1.248520in}}%
\pgfpathlineto{\pgfqpoint{31.000738in}{1.269004in}}%
\pgfpathlineto{\pgfqpoint{31.073353in}{1.239497in}}%
\pgfpathlineto{\pgfqpoint{31.150292in}{1.234987in}}%
\pgfpathlineto{\pgfqpoint{31.225476in}{1.245446in}}%
\pgfpathlineto{\pgfqpoint{31.304546in}{1.197516in}}%
\pgfpathlineto{\pgfqpoint{31.385546in}{1.228932in}}%
\pgfpathlineto{\pgfqpoint{31.463397in}{1.226639in}}%
\pgfpathlineto{\pgfqpoint{31.543273in}{1.177079in}}%
\pgfpathlineto{\pgfqpoint{31.629527in}{1.200709in}}%
\pgfpathlineto{\pgfqpoint{31.715049in}{1.194231in}}%
\pgfpathlineto{\pgfqpoint{31.802724in}{1.145304in}}%
\pgfpathlineto{\pgfqpoint{31.890397in}{1.213882in}}%
\pgfpathlineto{\pgfqpoint{31.975639in}{1.193878in}}%
\pgfpathlineto{\pgfqpoint{32.062211in}{1.165488in}}%
\pgfpathlineto{\pgfqpoint{32.153206in}{1.177862in}}%
\pgfpathlineto{\pgfqpoint{32.244846in}{1.172160in}}%
\pgfpathlineto{\pgfqpoint{32.334387in}{1.189047in}}%
\pgfpathlineto{\pgfqpoint{32.429213in}{1.153535in}}%
\pgfpathlineto{\pgfqpoint{32.518604in}{1.197335in}}%
\pgfpathlineto{\pgfqpoint{32.584565in}{1.438877in}}%
\pgfpathlineto{\pgfqpoint{32.638735in}{1.466197in}}%
\pgfpathlineto{\pgfqpoint{32.691213in}{1.465799in}}%
\pgfpathlineto{\pgfqpoint{32.743153in}{1.453492in}}%
\pgfpathlineto{\pgfqpoint{32.797230in}{1.405600in}}%
\pgfpathlineto{\pgfqpoint{32.850576in}{1.393664in}}%
\pgfpathlineto{\pgfqpoint{32.903667in}{1.442369in}}%
\pgfpathlineto{\pgfqpoint{32.957868in}{1.459985in}}%
\pgfpathlineto{\pgfqpoint{33.010153in}{1.429618in}}%
\pgfpathlineto{\pgfqpoint{33.062733in}{1.414323in}}%
\pgfpathlineto{\pgfqpoint{33.115518in}{1.488284in}}%
\pgfpathlineto{\pgfqpoint{33.167699in}{1.423144in}}%
\pgfpathlineto{\pgfqpoint{33.219830in}{1.230236in}}%
\pgfpathlineto{\pgfqpoint{33.273785in}{0.773588in}}%
\pgfpathlineto{\pgfqpoint{33.326652in}{0.773588in}}%
\pgfpathlineto{\pgfqpoint{33.379482in}{0.773588in}}%
\pgfpathlineto{\pgfqpoint{33.433283in}{0.773588in}}%
\pgfpathlineto{\pgfqpoint{33.484775in}{0.773588in}}%
\pgfpathlineto{\pgfqpoint{33.536879in}{0.773588in}}%
\pgfpathlineto{\pgfqpoint{33.590505in}{0.773588in}}%
\pgfpathlineto{\pgfqpoint{33.642373in}{0.773588in}}%
\pgfpathlineto{\pgfqpoint{33.693680in}{0.773588in}}%
\pgfpathlineto{\pgfqpoint{33.746200in}{0.773588in}}%
\pgfpathlineto{\pgfqpoint{33.784450in}{0.773588in}}%
\pgfpathlineto{\pgfqpoint{33.832264in}{0.773588in}}%
\pgfpathlineto{\pgfqpoint{33.870742in}{1.454760in}}%
\pgfpathlineto{\pgfqpoint{33.912991in}{1.669618in}}%
\pgfpathlineto{\pgfqpoint{33.952831in}{1.906942in}}%
\pgfpathlineto{\pgfqpoint{33.989641in}{2.224563in}}%
\pgfpathlineto{\pgfqpoint{34.021637in}{2.864879in}}%
\pgfpathlineto{\pgfqpoint{34.052161in}{3.729137in}}%
\pgfpathlineto{\pgfqpoint{34.076513in}{5.235025in}}%
\pgfpathlineto{\pgfqpoint{34.101659in}{5.182737in}}%
\pgfpathlineto{\pgfqpoint{34.125436in}{5.364631in}}%
\pgfpathlineto{\pgfqpoint{34.150180in}{5.483715in}}%
\pgfpathlineto{\pgfqpoint{34.173862in}{5.446238in}}%
\pgfpathlineto{\pgfqpoint{34.197440in}{5.442565in}}%
\pgfpathlineto{\pgfqpoint{34.222832in}{5.484292in}}%
\pgfpathlineto{\pgfqpoint{34.246364in}{5.534473in}}%
\pgfpathlineto{\pgfqpoint{34.270997in}{5.460686in}}%
\pgfpathlineto{\pgfqpoint{34.293786in}{5.518952in}}%
\pgfpathlineto{\pgfqpoint{34.318035in}{5.678338in}}%
\pgfpathlineto{\pgfqpoint{34.340984in}{5.584654in}}%
\pgfpathlineto{\pgfqpoint{34.366319in}{5.504363in}}%
\pgfpathlineto{\pgfqpoint{34.388883in}{5.557546in}}%
\pgfpathlineto{\pgfqpoint{34.412956in}{5.442515in}}%
\pgfpathlineto{\pgfqpoint{34.437757in}{5.582839in}}%
\pgfpathlineto{\pgfqpoint{34.460943in}{5.834541in}}%
\pgfpathlineto{\pgfqpoint{34.484122in}{5.740043in}}%
\pgfpathlineto{\pgfqpoint{34.508593in}{5.601241in}}%
\pgfpathlineto{\pgfqpoint{34.531364in}{5.809148in}}%
\pgfpathlineto{\pgfqpoint{34.554213in}{5.745325in}}%
\pgfpathlineto{\pgfqpoint{34.578241in}{5.846668in}}%
\pgfpathlineto{\pgfqpoint{34.601042in}{5.862622in}}%
\pgfpathlineto{\pgfqpoint{34.623987in}{5.791091in}}%
\pgfpathlineto{\pgfqpoint{34.648006in}{5.744292in}}%
\pgfpathlineto{\pgfqpoint{34.671232in}{5.706485in}}%
\pgfpathlineto{\pgfqpoint{34.693969in}{5.908282in}}%
\pgfpathlineto{\pgfqpoint{34.717979in}{5.813609in}}%
\pgfpathlineto{\pgfqpoint{34.741399in}{5.854178in}}%
\pgfpathlineto{\pgfqpoint{34.763884in}{5.918658in}}%
\pgfpathlineto{\pgfqpoint{34.788312in}{5.815038in}}%
\pgfpathlineto{\pgfqpoint{34.810332in}{5.868196in}}%
\pgfpathlineto{\pgfqpoint{34.834146in}{5.857730in}}%
\pgfpathlineto{\pgfqpoint{34.856576in}{5.911984in}}%
\pgfpathlineto{\pgfqpoint{34.880395in}{5.893046in}}%
\pgfpathlineto{\pgfqpoint{34.902821in}{5.880249in}}%
\pgfpathlineto{\pgfqpoint{34.926805in}{5.899825in}}%
\pgfpathlineto{\pgfqpoint{34.948980in}{5.930845in}}%
\pgfpathlineto{\pgfqpoint{34.973284in}{5.726189in}}%
\pgfpathlineto{\pgfqpoint{34.996094in}{5.825752in}}%
\pgfpathlineto{\pgfqpoint{35.020184in}{5.865058in}}%
\pgfpathlineto{\pgfqpoint{35.047391in}{5.775449in}}%
\pgfpathlineto{\pgfqpoint{35.098205in}{5.758501in}}%
\pgfpathlineto{\pgfqpoint{35.149293in}{5.758501in}}%
\pgfpathlineto{\pgfqpoint{35.201170in}{5.758501in}}%
\pgfpathlineto{\pgfqpoint{35.254065in}{5.758501in}}%
\pgfpathlineto{\pgfqpoint{35.305776in}{5.758501in}}%
\pgfpathlineto{\pgfqpoint{35.357710in}{5.758501in}}%
\pgfpathlineto{\pgfqpoint{35.411910in}{5.758501in}}%
\pgfpathlineto{\pgfqpoint{35.464942in}{5.758501in}}%
\pgfpathlineto{\pgfqpoint{35.464942in}{5.758501in}}%
\pgfpathlineto{\pgfqpoint{35.464942in}{5.758501in}}%
\pgfpathlineto{\pgfqpoint{35.411910in}{5.758501in}}%
\pgfpathlineto{\pgfqpoint{35.357710in}{5.758501in}}%
\pgfpathlineto{\pgfqpoint{35.305776in}{5.758501in}}%
\pgfpathlineto{\pgfqpoint{35.254065in}{5.758501in}}%
\pgfpathlineto{\pgfqpoint{35.201170in}{5.758501in}}%
\pgfpathlineto{\pgfqpoint{35.149293in}{5.758501in}}%
\pgfpathlineto{\pgfqpoint{35.098205in}{5.758501in}}%
\pgfpathlineto{\pgfqpoint{35.047391in}{5.775449in}}%
\pgfpathlineto{\pgfqpoint{35.020184in}{5.865058in}}%
\pgfpathlineto{\pgfqpoint{34.996094in}{5.825752in}}%
\pgfpathlineto{\pgfqpoint{34.973284in}{5.726189in}}%
\pgfpathlineto{\pgfqpoint{34.948980in}{5.930845in}}%
\pgfpathlineto{\pgfqpoint{34.926805in}{5.899825in}}%
\pgfpathlineto{\pgfqpoint{34.902821in}{5.880249in}}%
\pgfpathlineto{\pgfqpoint{34.880395in}{5.893046in}}%
\pgfpathlineto{\pgfqpoint{34.856576in}{5.911984in}}%
\pgfpathlineto{\pgfqpoint{34.834146in}{5.857730in}}%
\pgfpathlineto{\pgfqpoint{34.810332in}{5.868196in}}%
\pgfpathlineto{\pgfqpoint{34.788312in}{5.815038in}}%
\pgfpathlineto{\pgfqpoint{34.763884in}{5.918658in}}%
\pgfpathlineto{\pgfqpoint{34.741399in}{5.854178in}}%
\pgfpathlineto{\pgfqpoint{34.717979in}{5.813609in}}%
\pgfpathlineto{\pgfqpoint{34.693969in}{5.908282in}}%
\pgfpathlineto{\pgfqpoint{34.671232in}{5.706485in}}%
\pgfpathlineto{\pgfqpoint{34.648006in}{5.744292in}}%
\pgfpathlineto{\pgfqpoint{34.623987in}{5.791091in}}%
\pgfpathlineto{\pgfqpoint{34.601042in}{5.862622in}}%
\pgfpathlineto{\pgfqpoint{34.578241in}{5.846668in}}%
\pgfpathlineto{\pgfqpoint{34.554213in}{5.745325in}}%
\pgfpathlineto{\pgfqpoint{34.531364in}{5.809148in}}%
\pgfpathlineto{\pgfqpoint{34.508593in}{5.601241in}}%
\pgfpathlineto{\pgfqpoint{34.484122in}{5.740043in}}%
\pgfpathlineto{\pgfqpoint{34.460943in}{5.834541in}}%
\pgfpathlineto{\pgfqpoint{34.437757in}{5.582839in}}%
\pgfpathlineto{\pgfqpoint{34.412956in}{5.442515in}}%
\pgfpathlineto{\pgfqpoint{34.388883in}{5.557546in}}%
\pgfpathlineto{\pgfqpoint{34.366319in}{5.504363in}}%
\pgfpathlineto{\pgfqpoint{34.340984in}{5.584654in}}%
\pgfpathlineto{\pgfqpoint{34.318035in}{5.678338in}}%
\pgfpathlineto{\pgfqpoint{34.293786in}{5.518952in}}%
\pgfpathlineto{\pgfqpoint{34.270997in}{5.460686in}}%
\pgfpathlineto{\pgfqpoint{34.246364in}{5.534473in}}%
\pgfpathlineto{\pgfqpoint{34.222832in}{5.484292in}}%
\pgfpathlineto{\pgfqpoint{34.197440in}{5.442565in}}%
\pgfpathlineto{\pgfqpoint{34.173862in}{5.446238in}}%
\pgfpathlineto{\pgfqpoint{34.150180in}{5.483715in}}%
\pgfpathlineto{\pgfqpoint{34.125436in}{5.364631in}}%
\pgfpathlineto{\pgfqpoint{34.101659in}{5.182737in}}%
\pgfpathlineto{\pgfqpoint{34.076513in}{5.235025in}}%
\pgfpathlineto{\pgfqpoint{34.052161in}{3.729137in}}%
\pgfpathlineto{\pgfqpoint{34.021637in}{2.864879in}}%
\pgfpathlineto{\pgfqpoint{33.989641in}{2.224563in}}%
\pgfpathlineto{\pgfqpoint{33.952831in}{1.906942in}}%
\pgfpathlineto{\pgfqpoint{33.912991in}{1.669618in}}%
\pgfpathlineto{\pgfqpoint{33.870742in}{1.454760in}}%
\pgfpathlineto{\pgfqpoint{33.832264in}{0.773588in}}%
\pgfpathlineto{\pgfqpoint{33.784450in}{0.773588in}}%
\pgfpathlineto{\pgfqpoint{33.746200in}{0.773588in}}%
\pgfpathlineto{\pgfqpoint{33.693680in}{0.773588in}}%
\pgfpathlineto{\pgfqpoint{33.642373in}{0.773588in}}%
\pgfpathlineto{\pgfqpoint{33.590505in}{0.773588in}}%
\pgfpathlineto{\pgfqpoint{33.536879in}{0.773588in}}%
\pgfpathlineto{\pgfqpoint{33.484775in}{0.773588in}}%
\pgfpathlineto{\pgfqpoint{33.433283in}{0.773588in}}%
\pgfpathlineto{\pgfqpoint{33.379482in}{0.773588in}}%
\pgfpathlineto{\pgfqpoint{33.326652in}{0.773588in}}%
\pgfpathlineto{\pgfqpoint{33.273785in}{0.773588in}}%
\pgfpathlineto{\pgfqpoint{33.219830in}{1.230236in}}%
\pgfpathlineto{\pgfqpoint{33.167699in}{1.423144in}}%
\pgfpathlineto{\pgfqpoint{33.115518in}{1.488284in}}%
\pgfpathlineto{\pgfqpoint{33.062733in}{1.414323in}}%
\pgfpathlineto{\pgfqpoint{33.010153in}{1.429618in}}%
\pgfpathlineto{\pgfqpoint{32.957868in}{1.459985in}}%
\pgfpathlineto{\pgfqpoint{32.903667in}{1.442369in}}%
\pgfpathlineto{\pgfqpoint{32.850576in}{1.393664in}}%
\pgfpathlineto{\pgfqpoint{32.797230in}{1.405600in}}%
\pgfpathlineto{\pgfqpoint{32.743153in}{1.453492in}}%
\pgfpathlineto{\pgfqpoint{32.691213in}{1.465799in}}%
\pgfpathlineto{\pgfqpoint{32.638735in}{1.466197in}}%
\pgfpathlineto{\pgfqpoint{32.584565in}{1.527806in}}%
\pgfpathlineto{\pgfqpoint{32.518604in}{1.577644in}}%
\pgfpathlineto{\pgfqpoint{32.429213in}{1.548059in}}%
\pgfpathlineto{\pgfqpoint{32.334387in}{1.559840in}}%
\pgfpathlineto{\pgfqpoint{32.244846in}{1.565071in}}%
\pgfpathlineto{\pgfqpoint{32.153206in}{1.569780in}}%
\pgfpathlineto{\pgfqpoint{32.062211in}{1.548891in}}%
\pgfpathlineto{\pgfqpoint{31.975639in}{1.606211in}}%
\pgfpathlineto{\pgfqpoint{31.890397in}{1.621243in}}%
\pgfpathlineto{\pgfqpoint{31.802724in}{1.559328in}}%
\pgfpathlineto{\pgfqpoint{31.715049in}{1.618107in}}%
\pgfpathlineto{\pgfqpoint{31.629527in}{1.642355in}}%
\pgfpathlineto{\pgfqpoint{31.543273in}{1.623152in}}%
\pgfpathlineto{\pgfqpoint{31.463397in}{1.664462in}}%
\pgfpathlineto{\pgfqpoint{31.385546in}{1.683708in}}%
\pgfpathlineto{\pgfqpoint{31.304546in}{1.621377in}}%
\pgfpathlineto{\pgfqpoint{31.225476in}{1.715736in}}%
\pgfpathlineto{\pgfqpoint{31.150292in}{1.713936in}}%
\pgfpathlineto{\pgfqpoint{31.073353in}{1.696625in}}%
\pgfpathlineto{\pgfqpoint{31.000738in}{1.786581in}}%
\pgfpathlineto{\pgfqpoint{30.928370in}{1.744580in}}%
\pgfpathlineto{\pgfqpoint{30.855684in}{1.770692in}}%
\pgfpathlineto{\pgfqpoint{30.786068in}{1.815928in}}%
\pgfpathlineto{\pgfqpoint{30.717993in}{1.799199in}}%
\pgfpathlineto{\pgfqpoint{30.648377in}{1.818030in}}%
\pgfpathlineto{\pgfqpoint{30.582999in}{1.881027in}}%
\pgfpathlineto{\pgfqpoint{30.519963in}{1.909921in}}%
\pgfpathlineto{\pgfqpoint{30.457369in}{1.933531in}}%
\pgfpathlineto{\pgfqpoint{30.397563in}{1.931511in}}%
\pgfpathlineto{\pgfqpoint{30.337041in}{1.985679in}}%
\pgfpathlineto{\pgfqpoint{30.276704in}{1.990714in}}%
\pgfpathlineto{\pgfqpoint{30.221783in}{2.094139in}}%
\pgfpathlineto{\pgfqpoint{30.169497in}{2.140896in}}%
\pgfpathlineto{\pgfqpoint{30.117082in}{2.164722in}}%
\pgfpathlineto{\pgfqpoint{30.066247in}{2.157383in}}%
\pgfpathlineto{\pgfqpoint{30.014968in}{2.192509in}}%
\pgfpathlineto{\pgfqpoint{29.962066in}{2.109977in}}%
\pgfpathlineto{\pgfqpoint{29.910558in}{2.163910in}}%
\pgfpathlineto{\pgfqpoint{29.858913in}{2.093735in}}%
\pgfpathlineto{\pgfqpoint{29.805208in}{2.142856in}}%
\pgfpathlineto{\pgfqpoint{29.753093in}{2.161564in}}%
\pgfpathlineto{\pgfqpoint{29.701409in}{2.123741in}}%
\pgfpathlineto{\pgfqpoint{29.647988in}{2.152791in}}%
\pgfpathlineto{\pgfqpoint{29.596078in}{2.080609in}}%
\pgfpathlineto{\pgfqpoint{29.543760in}{2.082360in}}%
\pgfpathlineto{\pgfqpoint{29.490075in}{2.134356in}}%
\pgfpathlineto{\pgfqpoint{29.438335in}{2.133563in}}%
\pgfpathlineto{\pgfqpoint{29.386173in}{2.150986in}}%
\pgfpathlineto{\pgfqpoint{29.332261in}{2.082785in}}%
\pgfpathlineto{\pgfqpoint{29.279399in}{2.100503in}}%
\pgfpathlineto{\pgfqpoint{29.226818in}{2.160460in}}%
\pgfpathlineto{\pgfqpoint{29.173700in}{2.077484in}}%
\pgfpathlineto{\pgfqpoint{29.121029in}{2.088575in}}%
\pgfpathlineto{\pgfqpoint{29.068863in}{2.138193in}}%
\pgfpathlineto{\pgfqpoint{29.015651in}{2.178709in}}%
\pgfpathlineto{\pgfqpoint{28.965063in}{2.168914in}}%
\pgfpathlineto{\pgfqpoint{28.913538in}{2.139636in}}%
\pgfpathlineto{\pgfqpoint{28.860347in}{2.116407in}}%
\pgfpathlineto{\pgfqpoint{28.808587in}{2.154364in}}%
\pgfpathlineto{\pgfqpoint{28.758074in}{2.179648in}}%
\pgfpathlineto{\pgfqpoint{28.706300in}{2.197239in}}%
\pgfpathlineto{\pgfqpoint{28.656185in}{2.169046in}}%
\pgfpathlineto{\pgfqpoint{28.605351in}{2.184676in}}%
\pgfpathlineto{\pgfqpoint{28.553749in}{2.135069in}}%
\pgfpathlineto{\pgfqpoint{28.502707in}{2.176910in}}%
\pgfpathlineto{\pgfqpoint{28.451952in}{2.085182in}}%
\pgfpathlineto{\pgfqpoint{28.399138in}{2.108488in}}%
\pgfpathlineto{\pgfqpoint{28.347853in}{2.128908in}}%
\pgfpathlineto{\pgfqpoint{28.297462in}{2.221346in}}%
\pgfpathlineto{\pgfqpoint{28.245320in}{2.152372in}}%
\pgfpathlineto{\pgfqpoint{28.195034in}{2.194715in}}%
\pgfpathlineto{\pgfqpoint{28.143944in}{2.103896in}}%
\pgfpathlineto{\pgfqpoint{28.090650in}{2.154372in}}%
\pgfpathlineto{\pgfqpoint{28.039571in}{2.133339in}}%
\pgfpathlineto{\pgfqpoint{27.988562in}{2.148509in}}%
\pgfpathlineto{\pgfqpoint{27.935127in}{2.142057in}}%
\pgfpathlineto{\pgfqpoint{27.883786in}{2.171648in}}%
\pgfpathlineto{\pgfqpoint{27.833067in}{2.102078in}}%
\pgfpathlineto{\pgfqpoint{27.779331in}{2.127843in}}%
\pgfpathlineto{\pgfqpoint{27.728571in}{2.169913in}}%
\pgfpathlineto{\pgfqpoint{27.678294in}{2.202638in}}%
\pgfpathlineto{\pgfqpoint{27.626352in}{2.188092in}}%
\pgfpathlineto{\pgfqpoint{27.576178in}{2.193467in}}%
\pgfpathlineto{\pgfqpoint{27.525849in}{2.059972in}}%
\pgfpathlineto{\pgfqpoint{27.473173in}{2.149431in}}%
\pgfpathlineto{\pgfqpoint{27.423066in}{2.151542in}}%
\pgfpathlineto{\pgfqpoint{27.372863in}{2.262046in}}%
\pgfpathlineto{\pgfqpoint{27.321160in}{2.084956in}}%
\pgfpathlineto{\pgfqpoint{27.269679in}{2.127160in}}%
\pgfpathlineto{\pgfqpoint{27.218587in}{2.125482in}}%
\pgfpathlineto{\pgfqpoint{27.165274in}{2.106034in}}%
\pgfpathlineto{\pgfqpoint{27.113950in}{2.191289in}}%
\pgfpathlineto{\pgfqpoint{27.063062in}{2.127542in}}%
\pgfpathlineto{\pgfqpoint{27.010539in}{2.206184in}}%
\pgfpathlineto{\pgfqpoint{26.960748in}{2.126317in}}%
\pgfpathlineto{\pgfqpoint{26.909530in}{2.167554in}}%
\pgfpathlineto{\pgfqpoint{26.857706in}{2.142711in}}%
\pgfpathlineto{\pgfqpoint{26.807173in}{2.169425in}}%
\pgfpathlineto{\pgfqpoint{26.756448in}{2.167275in}}%
\pgfpathlineto{\pgfqpoint{26.704176in}{2.127344in}}%
\pgfpathlineto{\pgfqpoint{26.653161in}{2.155346in}}%
\pgfpathlineto{\pgfqpoint{26.601868in}{2.125000in}}%
\pgfpathlineto{\pgfqpoint{26.549446in}{2.196254in}}%
\pgfpathlineto{\pgfqpoint{26.499064in}{2.073278in}}%
\pgfpathlineto{\pgfqpoint{26.447655in}{2.147600in}}%
\pgfpathlineto{\pgfqpoint{26.395829in}{2.216585in}}%
\pgfpathlineto{\pgfqpoint{26.346588in}{2.210846in}}%
\pgfpathlineto{\pgfqpoint{26.297518in}{2.140127in}}%
\pgfpathlineto{\pgfqpoint{26.245717in}{2.188680in}}%
\pgfpathlineto{\pgfqpoint{26.195263in}{2.138438in}}%
\pgfpathlineto{\pgfqpoint{26.144837in}{2.202460in}}%
\pgfpathlineto{\pgfqpoint{26.093895in}{2.172470in}}%
\pgfpathlineto{\pgfqpoint{26.044216in}{2.172740in}}%
\pgfpathlineto{\pgfqpoint{25.994130in}{2.161647in}}%
\pgfpathlineto{\pgfqpoint{25.942005in}{2.189113in}}%
\pgfpathlineto{\pgfqpoint{25.891991in}{2.129676in}}%
\pgfpathlineto{\pgfqpoint{25.841969in}{2.202252in}}%
\pgfpathlineto{\pgfqpoint{25.790287in}{2.177297in}}%
\pgfpathlineto{\pgfqpoint{25.740073in}{2.113429in}}%
\pgfpathlineto{\pgfqpoint{25.689152in}{2.130566in}}%
\pgfpathlineto{\pgfqpoint{25.637513in}{2.153196in}}%
\pgfpathlineto{\pgfqpoint{25.586387in}{2.087238in}}%
\pgfpathlineto{\pgfqpoint{25.535558in}{2.107499in}}%
\pgfpathlineto{\pgfqpoint{25.483381in}{2.187707in}}%
\pgfpathlineto{\pgfqpoint{25.433954in}{2.179510in}}%
\pgfpathlineto{\pgfqpoint{25.383622in}{2.199798in}}%
\pgfpathlineto{\pgfqpoint{25.332608in}{2.124537in}}%
\pgfpathlineto{\pgfqpoint{25.282222in}{2.174410in}}%
\pgfpathlineto{\pgfqpoint{25.232944in}{2.198963in}}%
\pgfpathlineto{\pgfqpoint{25.181554in}{2.148863in}}%
\pgfpathlineto{\pgfqpoint{25.131043in}{2.169034in}}%
\pgfpathlineto{\pgfqpoint{25.080539in}{2.156690in}}%
\pgfpathlineto{\pgfqpoint{25.028630in}{2.166090in}}%
\pgfpathlineto{\pgfqpoint{24.978211in}{2.200234in}}%
\pgfpathlineto{\pgfqpoint{24.928869in}{2.240633in}}%
\pgfpathlineto{\pgfqpoint{24.878143in}{2.139836in}}%
\pgfpathlineto{\pgfqpoint{24.828140in}{2.234647in}}%
\pgfpathlineto{\pgfqpoint{24.778304in}{2.187482in}}%
\pgfpathlineto{\pgfqpoint{24.726499in}{2.192402in}}%
\pgfpathlineto{\pgfqpoint{24.676389in}{2.191465in}}%
\pgfpathlineto{\pgfqpoint{24.626556in}{2.072454in}}%
\pgfpathlineto{\pgfqpoint{24.573998in}{2.204872in}}%
\pgfpathlineto{\pgfqpoint{24.524258in}{2.209164in}}%
\pgfpathlineto{\pgfqpoint{24.474381in}{2.139169in}}%
\pgfpathlineto{\pgfqpoint{24.422475in}{2.160039in}}%
\pgfpathlineto{\pgfqpoint{24.371890in}{2.208071in}}%
\pgfpathlineto{\pgfqpoint{24.321907in}{2.211605in}}%
\pgfpathlineto{\pgfqpoint{24.271277in}{2.149127in}}%
\pgfpathlineto{\pgfqpoint{24.221010in}{2.157432in}}%
\pgfpathlineto{\pgfqpoint{24.170873in}{2.170742in}}%
\pgfpathlineto{\pgfqpoint{24.118637in}{2.144171in}}%
\pgfpathlineto{\pgfqpoint{24.066933in}{2.162367in}}%
\pgfpathlineto{\pgfqpoint{24.016442in}{2.201493in}}%
\pgfpathlineto{\pgfqpoint{23.964514in}{2.116542in}}%
\pgfpathlineto{\pgfqpoint{23.913088in}{2.119330in}}%
\pgfpathlineto{\pgfqpoint{23.862673in}{2.175857in}}%
\pgfpathlineto{\pgfqpoint{23.811512in}{2.259241in}}%
\pgfpathlineto{\pgfqpoint{23.762329in}{2.182126in}}%
\pgfpathlineto{\pgfqpoint{23.713074in}{2.221155in}}%
\pgfpathlineto{\pgfqpoint{23.661575in}{2.159418in}}%
\pgfpathlineto{\pgfqpoint{23.611923in}{2.282936in}}%
\pgfpathlineto{\pgfqpoint{23.563429in}{2.143388in}}%
\pgfpathlineto{\pgfqpoint{23.511867in}{2.137407in}}%
\pgfpathlineto{\pgfqpoint{23.462017in}{2.157637in}}%
\pgfpathlineto{\pgfqpoint{23.411909in}{2.198735in}}%
\pgfpathlineto{\pgfqpoint{23.361310in}{2.170140in}}%
\pgfpathlineto{\pgfqpoint{23.310918in}{2.148300in}}%
\pgfpathlineto{\pgfqpoint{23.260726in}{2.152878in}}%
\pgfpathlineto{\pgfqpoint{23.208694in}{2.124981in}}%
\pgfpathlineto{\pgfqpoint{23.157756in}{2.136082in}}%
\pgfpathlineto{\pgfqpoint{23.107134in}{2.079790in}}%
\pgfpathlineto{\pgfqpoint{23.055187in}{2.235002in}}%
\pgfpathlineto{\pgfqpoint{23.006022in}{2.146954in}}%
\pgfpathlineto{\pgfqpoint{22.956359in}{2.173476in}}%
\pgfpathlineto{\pgfqpoint{22.904236in}{2.104244in}}%
\pgfpathlineto{\pgfqpoint{22.853966in}{2.158422in}}%
\pgfpathlineto{\pgfqpoint{22.803829in}{2.168087in}}%
\pgfpathlineto{\pgfqpoint{22.751700in}{2.248709in}}%
\pgfpathlineto{\pgfqpoint{22.703398in}{2.305817in}}%
\pgfpathlineto{\pgfqpoint{22.654822in}{2.180208in}}%
\pgfpathlineto{\pgfqpoint{22.603657in}{2.179999in}}%
\pgfpathlineto{\pgfqpoint{22.554073in}{2.136690in}}%
\pgfpathlineto{\pgfqpoint{22.503552in}{2.135991in}}%
\pgfpathlineto{\pgfqpoint{22.451976in}{2.206816in}}%
\pgfpathlineto{\pgfqpoint{22.402084in}{2.133288in}}%
\pgfpathlineto{\pgfqpoint{22.352827in}{2.157190in}}%
\pgfpathlineto{\pgfqpoint{22.300933in}{2.094855in}}%
\pgfpathlineto{\pgfqpoint{22.250634in}{2.185196in}}%
\pgfpathlineto{\pgfqpoint{22.200587in}{2.138280in}}%
\pgfpathlineto{\pgfqpoint{22.148817in}{2.251963in}}%
\pgfpathlineto{\pgfqpoint{22.099907in}{2.258881in}}%
\pgfpathlineto{\pgfqpoint{22.050354in}{2.153076in}}%
\pgfpathlineto{\pgfqpoint{21.999803in}{2.276898in}}%
\pgfpathlineto{\pgfqpoint{21.950544in}{2.161953in}}%
\pgfpathlineto{\pgfqpoint{21.900834in}{2.128498in}}%
\pgfpathlineto{\pgfqpoint{21.849580in}{2.230011in}}%
\pgfpathlineto{\pgfqpoint{21.800474in}{2.226448in}}%
\pgfpathlineto{\pgfqpoint{21.751624in}{2.227472in}}%
\pgfpathlineto{\pgfqpoint{21.700864in}{2.213775in}}%
\pgfpathlineto{\pgfqpoint{21.652276in}{2.218841in}}%
\pgfpathlineto{\pgfqpoint{21.602538in}{2.160895in}}%
\pgfpathlineto{\pgfqpoint{21.550957in}{2.167679in}}%
\pgfpathlineto{\pgfqpoint{21.501023in}{2.150548in}}%
\pgfpathlineto{\pgfqpoint{21.450830in}{2.168605in}}%
\pgfpathlineto{\pgfqpoint{21.399733in}{2.175744in}}%
\pgfpathlineto{\pgfqpoint{21.350353in}{2.208270in}}%
\pgfpathlineto{\pgfqpoint{21.300955in}{2.191708in}}%
\pgfpathlineto{\pgfqpoint{21.250655in}{2.245041in}}%
\pgfpathlineto{\pgfqpoint{21.201683in}{2.181293in}}%
\pgfpathlineto{\pgfqpoint{21.152719in}{2.201026in}}%
\pgfpathlineto{\pgfqpoint{21.102370in}{2.241324in}}%
\pgfpathlineto{\pgfqpoint{21.053515in}{2.155531in}}%
\pgfpathlineto{\pgfqpoint{21.003693in}{2.185814in}}%
\pgfpathlineto{\pgfqpoint{20.953585in}{2.212485in}}%
\pgfpathlineto{\pgfqpoint{20.904164in}{2.166381in}}%
\pgfpathlineto{\pgfqpoint{20.854118in}{2.134033in}}%
\pgfpathlineto{\pgfqpoint{20.802965in}{2.283576in}}%
\pgfpathlineto{\pgfqpoint{20.753558in}{2.212760in}}%
\pgfpathlineto{\pgfqpoint{20.704616in}{2.268763in}}%
\pgfpathlineto{\pgfqpoint{20.654668in}{2.192399in}}%
\pgfpathlineto{\pgfqpoint{20.605567in}{2.126089in}}%
\pgfpathlineto{\pgfqpoint{20.556290in}{2.221965in}}%
\pgfpathlineto{\pgfqpoint{20.505208in}{2.258629in}}%
\pgfpathlineto{\pgfqpoint{20.455604in}{2.155524in}}%
\pgfpathlineto{\pgfqpoint{20.406414in}{2.188387in}}%
\pgfpathlineto{\pgfqpoint{20.355462in}{2.199386in}}%
\pgfpathlineto{\pgfqpoint{20.305924in}{2.134136in}}%
\pgfpathlineto{\pgfqpoint{20.255593in}{2.107988in}}%
\pgfpathlineto{\pgfqpoint{20.203615in}{2.192230in}}%
\pgfpathlineto{\pgfqpoint{20.153679in}{2.162851in}}%
\pgfpathlineto{\pgfqpoint{20.104350in}{2.161230in}}%
\pgfpathlineto{\pgfqpoint{20.053312in}{2.243716in}}%
\pgfpathlineto{\pgfqpoint{20.003692in}{2.175230in}}%
\pgfpathlineto{\pgfqpoint{19.955252in}{2.296405in}}%
\pgfpathlineto{\pgfqpoint{19.906615in}{2.281059in}}%
\pgfpathlineto{\pgfqpoint{19.858933in}{2.236420in}}%
\pgfpathlineto{\pgfqpoint{19.810712in}{2.223149in}}%
\pgfpathlineto{\pgfqpoint{19.761678in}{2.193047in}}%
\pgfpathlineto{\pgfqpoint{19.714265in}{2.275859in}}%
\pgfpathlineto{\pgfqpoint{19.666940in}{2.243279in}}%
\pgfpathlineto{\pgfqpoint{19.617707in}{2.182953in}}%
\pgfpathlineto{\pgfqpoint{19.569747in}{2.259987in}}%
\pgfpathlineto{\pgfqpoint{19.522292in}{2.234019in}}%
\pgfpathlineto{\pgfqpoint{19.473154in}{2.190874in}}%
\pgfpathlineto{\pgfqpoint{19.425178in}{2.227549in}}%
\pgfpathlineto{\pgfqpoint{19.376890in}{2.239043in}}%
\pgfpathlineto{\pgfqpoint{19.327829in}{2.241679in}}%
\pgfpathlineto{\pgfqpoint{19.279742in}{2.293533in}}%
\pgfpathlineto{\pgfqpoint{19.231826in}{2.158168in}}%
\pgfpathlineto{\pgfqpoint{19.181302in}{2.175947in}}%
\pgfpathlineto{\pgfqpoint{19.132724in}{2.240721in}}%
\pgfpathlineto{\pgfqpoint{19.084207in}{2.100303in}}%
\pgfpathlineto{\pgfqpoint{19.033794in}{2.192082in}}%
\pgfpathlineto{\pgfqpoint{18.985002in}{2.256048in}}%
\pgfpathlineto{\pgfqpoint{18.937671in}{2.243100in}}%
\pgfpathlineto{\pgfqpoint{18.888188in}{2.195093in}}%
\pgfpathlineto{\pgfqpoint{18.839312in}{2.255904in}}%
\pgfpathlineto{\pgfqpoint{18.791212in}{2.175746in}}%
\pgfpathlineto{\pgfqpoint{18.741665in}{2.223418in}}%
\pgfpathlineto{\pgfqpoint{18.693122in}{2.186383in}}%
\pgfpathlineto{\pgfqpoint{18.644639in}{2.233839in}}%
\pgfpathlineto{\pgfqpoint{18.595724in}{2.207052in}}%
\pgfpathlineto{\pgfqpoint{18.548460in}{2.240166in}}%
\pgfpathlineto{\pgfqpoint{18.500915in}{2.249891in}}%
\pgfpathlineto{\pgfqpoint{18.452284in}{2.172965in}}%
\pgfpathlineto{\pgfqpoint{18.405242in}{2.249085in}}%
\pgfpathlineto{\pgfqpoint{18.357538in}{2.210510in}}%
\pgfpathlineto{\pgfqpoint{18.308441in}{2.227451in}}%
\pgfpathlineto{\pgfqpoint{18.260871in}{2.246190in}}%
\pgfpathlineto{\pgfqpoint{18.213333in}{2.208562in}}%
\pgfpathlineto{\pgfqpoint{18.164405in}{2.303508in}}%
\pgfpathlineto{\pgfqpoint{18.117307in}{2.230183in}}%
\pgfpathlineto{\pgfqpoint{18.069524in}{2.220034in}}%
\pgfpathlineto{\pgfqpoint{18.020026in}{2.099061in}}%
\pgfpathlineto{\pgfqpoint{17.970910in}{2.245278in}}%
\pgfpathlineto{\pgfqpoint{17.922885in}{2.295885in}}%
\pgfpathlineto{\pgfqpoint{17.874600in}{2.265803in}}%
\pgfpathlineto{\pgfqpoint{17.826809in}{2.181075in}}%
\pgfpathlineto{\pgfqpoint{17.779014in}{2.180861in}}%
\pgfpathlineto{\pgfqpoint{17.729727in}{2.181121in}}%
\pgfpathlineto{\pgfqpoint{17.681914in}{2.358516in}}%
\pgfpathlineto{\pgfqpoint{17.634653in}{2.295602in}}%
\pgfpathlineto{\pgfqpoint{17.585742in}{2.379567in}}%
\pgfpathlineto{\pgfqpoint{17.538409in}{2.219811in}}%
\pgfpathlineto{\pgfqpoint{17.491125in}{2.288470in}}%
\pgfpathlineto{\pgfqpoint{17.442362in}{2.234768in}}%
\pgfpathlineto{\pgfqpoint{17.394714in}{2.248043in}}%
\pgfpathlineto{\pgfqpoint{17.347576in}{2.273193in}}%
\pgfpathlineto{\pgfqpoint{17.298941in}{2.177400in}}%
\pgfpathlineto{\pgfqpoint{17.250860in}{2.264188in}}%
\pgfpathlineto{\pgfqpoint{17.203674in}{2.246982in}}%
\pgfpathlineto{\pgfqpoint{17.154507in}{2.297791in}}%
\pgfpathlineto{\pgfqpoint{17.107470in}{2.233795in}}%
\pgfpathlineto{\pgfqpoint{17.059518in}{2.170799in}}%
\pgfpathlineto{\pgfqpoint{17.010585in}{2.251846in}}%
\pgfpathlineto{\pgfqpoint{16.963533in}{2.263073in}}%
\pgfpathlineto{\pgfqpoint{16.916302in}{2.332512in}}%
\pgfpathlineto{\pgfqpoint{16.868266in}{2.244193in}}%
\pgfpathlineto{\pgfqpoint{16.821038in}{2.222802in}}%
\pgfpathlineto{\pgfqpoint{16.773713in}{2.227898in}}%
\pgfpathlineto{\pgfqpoint{16.724568in}{2.247475in}}%
\pgfpathlineto{\pgfqpoint{16.677566in}{2.244778in}}%
\pgfpathlineto{\pgfqpoint{16.630280in}{2.266736in}}%
\pgfpathlineto{\pgfqpoint{16.581135in}{2.238151in}}%
\pgfpathlineto{\pgfqpoint{16.533619in}{2.302678in}}%
\pgfpathlineto{\pgfqpoint{16.486560in}{2.250525in}}%
\pgfpathlineto{\pgfqpoint{16.436622in}{2.137013in}}%
\pgfpathlineto{\pgfqpoint{16.387666in}{2.174071in}}%
\pgfpathlineto{\pgfqpoint{16.339537in}{2.259149in}}%
\pgfpathlineto{\pgfqpoint{16.289817in}{2.219318in}}%
\pgfpathlineto{\pgfqpoint{16.242078in}{2.276775in}}%
\pgfpathlineto{\pgfqpoint{16.195175in}{2.205221in}}%
\pgfpathlineto{\pgfqpoint{16.146338in}{2.216517in}}%
\pgfpathlineto{\pgfqpoint{16.098697in}{2.270580in}}%
\pgfpathlineto{\pgfqpoint{16.051820in}{2.332418in}}%
\pgfpathlineto{\pgfqpoint{16.003926in}{2.336191in}}%
\pgfpathlineto{\pgfqpoint{15.957799in}{2.402601in}}%
\pgfpathlineto{\pgfqpoint{15.911811in}{2.228192in}}%
\pgfpathlineto{\pgfqpoint{15.863592in}{2.332872in}}%
\pgfpathlineto{\pgfqpoint{15.817273in}{2.287207in}}%
\pgfpathlineto{\pgfqpoint{15.770154in}{2.313157in}}%
\pgfpathlineto{\pgfqpoint{15.722814in}{2.300535in}}%
\pgfpathlineto{\pgfqpoint{15.676019in}{2.301145in}}%
\pgfpathlineto{\pgfqpoint{15.628660in}{2.211679in}}%
\pgfpathlineto{\pgfqpoint{15.579815in}{2.258501in}}%
\pgfpathlineto{\pgfqpoint{15.532329in}{2.219452in}}%
\pgfpathlineto{\pgfqpoint{15.484858in}{2.218369in}}%
\pgfpathlineto{\pgfqpoint{15.436052in}{2.260363in}}%
\pgfpathlineto{\pgfqpoint{15.389896in}{2.293893in}}%
\pgfpathlineto{\pgfqpoint{15.343883in}{2.307810in}}%
\pgfpathlineto{\pgfqpoint{15.295866in}{2.280354in}}%
\pgfpathlineto{\pgfqpoint{15.248863in}{2.236209in}}%
\pgfpathlineto{\pgfqpoint{15.201456in}{2.240614in}}%
\pgfpathlineto{\pgfqpoint{15.152658in}{2.295575in}}%
\pgfpathlineto{\pgfqpoint{15.105478in}{2.314598in}}%
\pgfpathlineto{\pgfqpoint{15.058774in}{2.198546in}}%
\pgfpathlineto{\pgfqpoint{15.010214in}{2.267582in}}%
\pgfpathlineto{\pgfqpoint{14.963167in}{2.252252in}}%
\pgfpathlineto{\pgfqpoint{14.916239in}{2.196403in}}%
\pgfpathlineto{\pgfqpoint{14.868242in}{2.282858in}}%
\pgfpathlineto{\pgfqpoint{14.821474in}{2.359201in}}%
\pgfpathlineto{\pgfqpoint{14.774752in}{2.259349in}}%
\pgfpathlineto{\pgfqpoint{14.726757in}{2.231589in}}%
\pgfpathlineto{\pgfqpoint{14.680226in}{2.325095in}}%
\pgfpathlineto{\pgfqpoint{14.633992in}{2.221244in}}%
\pgfpathlineto{\pgfqpoint{14.585477in}{2.246168in}}%
\pgfpathlineto{\pgfqpoint{14.538630in}{2.291021in}}%
\pgfpathlineto{\pgfqpoint{14.491726in}{2.254543in}}%
\pgfpathlineto{\pgfqpoint{14.443729in}{2.346640in}}%
\pgfpathlineto{\pgfqpoint{14.397126in}{2.209372in}}%
\pgfpathlineto{\pgfqpoint{14.350224in}{2.276334in}}%
\pgfpathlineto{\pgfqpoint{14.302418in}{2.274703in}}%
\pgfpathlineto{\pgfqpoint{14.255593in}{2.321238in}}%
\pgfpathlineto{\pgfqpoint{14.209376in}{2.193278in}}%
\pgfpathlineto{\pgfqpoint{14.160147in}{2.182080in}}%
\pgfpathlineto{\pgfqpoint{14.112147in}{2.229533in}}%
\pgfpathlineto{\pgfqpoint{14.064951in}{2.232270in}}%
\pgfpathlineto{\pgfqpoint{14.016344in}{2.286673in}}%
\pgfpathlineto{\pgfqpoint{13.969353in}{2.229832in}}%
\pgfpathlineto{\pgfqpoint{13.922606in}{2.226374in}}%
\pgfpathlineto{\pgfqpoint{13.873610in}{2.212443in}}%
\pgfpathlineto{\pgfqpoint{13.825297in}{2.178671in}}%
\pgfpathlineto{\pgfqpoint{13.777441in}{2.242575in}}%
\pgfpathlineto{\pgfqpoint{13.729143in}{2.300755in}}%
\pgfpathlineto{\pgfqpoint{13.682151in}{2.232128in}}%
\pgfpathlineto{\pgfqpoint{13.635346in}{2.278245in}}%
\pgfpathlineto{\pgfqpoint{13.587249in}{2.335101in}}%
\pgfpathlineto{\pgfqpoint{13.541646in}{2.277760in}}%
\pgfpathlineto{\pgfqpoint{13.495727in}{2.299059in}}%
\pgfpathlineto{\pgfqpoint{13.448069in}{2.240374in}}%
\pgfpathlineto{\pgfqpoint{13.401859in}{2.323804in}}%
\pgfpathlineto{\pgfqpoint{13.356518in}{2.385490in}}%
\pgfpathlineto{\pgfqpoint{13.308974in}{2.312159in}}%
\pgfpathlineto{\pgfqpoint{13.262980in}{2.338157in}}%
\pgfpathlineto{\pgfqpoint{13.216719in}{2.235117in}}%
\pgfpathlineto{\pgfqpoint{13.169075in}{2.384526in}}%
\pgfpathlineto{\pgfqpoint{13.123339in}{2.262921in}}%
\pgfpathlineto{\pgfqpoint{13.077275in}{2.323533in}}%
\pgfpathlineto{\pgfqpoint{13.030499in}{2.298564in}}%
\pgfpathlineto{\pgfqpoint{12.983596in}{2.304244in}}%
\pgfpathlineto{\pgfqpoint{12.936799in}{2.386807in}}%
\pgfpathlineto{\pgfqpoint{12.889988in}{2.289046in}}%
\pgfpathlineto{\pgfqpoint{12.843155in}{2.234339in}}%
\pgfpathlineto{\pgfqpoint{12.796907in}{2.259754in}}%
\pgfpathlineto{\pgfqpoint{12.749025in}{2.253182in}}%
\pgfpathlineto{\pgfqpoint{12.701835in}{2.279136in}}%
\pgfpathlineto{\pgfqpoint{12.655424in}{2.325183in}}%
\pgfpathlineto{\pgfqpoint{12.607615in}{2.239852in}}%
\pgfpathlineto{\pgfqpoint{12.560711in}{2.203625in}}%
\pgfpathlineto{\pgfqpoint{12.514481in}{2.340897in}}%
\pgfpathlineto{\pgfqpoint{12.467453in}{2.296669in}}%
\pgfpathlineto{\pgfqpoint{12.421372in}{2.358343in}}%
\pgfpathlineto{\pgfqpoint{12.375261in}{2.349502in}}%
\pgfpathlineto{\pgfqpoint{12.328136in}{2.257529in}}%
\pgfpathlineto{\pgfqpoint{12.282271in}{2.273976in}}%
\pgfpathlineto{\pgfqpoint{12.235985in}{2.221952in}}%
\pgfpathlineto{\pgfqpoint{12.187925in}{2.280826in}}%
\pgfpathlineto{\pgfqpoint{12.141887in}{2.351371in}}%
\pgfpathlineto{\pgfqpoint{12.096623in}{2.317891in}}%
\pgfpathlineto{\pgfqpoint{12.049869in}{2.301880in}}%
\pgfpathlineto{\pgfqpoint{12.004414in}{2.324057in}}%
\pgfpathlineto{\pgfqpoint{11.959143in}{2.350595in}}%
\pgfpathlineto{\pgfqpoint{11.911501in}{2.203921in}}%
\pgfpathlineto{\pgfqpoint{11.864862in}{2.355287in}}%
\pgfpathlineto{\pgfqpoint{11.819406in}{2.285077in}}%
\pgfpathlineto{\pgfqpoint{11.772894in}{2.306760in}}%
\pgfpathlineto{\pgfqpoint{11.727607in}{2.333441in}}%
\pgfpathlineto{\pgfqpoint{11.682053in}{2.359958in}}%
\pgfpathlineto{\pgfqpoint{11.634589in}{2.229229in}}%
\pgfpathlineto{\pgfqpoint{11.588799in}{2.230089in}}%
\pgfpathlineto{\pgfqpoint{11.542665in}{2.357277in}}%
\pgfpathlineto{\pgfqpoint{11.494874in}{2.291080in}}%
\pgfpathlineto{\pgfqpoint{11.448531in}{2.259828in}}%
\pgfpathlineto{\pgfqpoint{11.402044in}{2.332692in}}%
\pgfpathlineto{\pgfqpoint{11.353756in}{2.235310in}}%
\pgfpathlineto{\pgfqpoint{11.307361in}{2.360674in}}%
\pgfpathlineto{\pgfqpoint{11.261350in}{2.276617in}}%
\pgfpathlineto{\pgfqpoint{11.213355in}{2.260071in}}%
\pgfpathlineto{\pgfqpoint{11.167471in}{2.399865in}}%
\pgfpathlineto{\pgfqpoint{11.121978in}{2.258422in}}%
\pgfpathlineto{\pgfqpoint{11.074633in}{2.335174in}}%
\pgfpathlineto{\pgfqpoint{11.029398in}{2.356543in}}%
\pgfpathlineto{\pgfqpoint{10.984145in}{2.282506in}}%
\pgfpathlineto{\pgfqpoint{10.937381in}{2.301035in}}%
\pgfpathlineto{\pgfqpoint{10.891801in}{2.284564in}}%
\pgfpathlineto{\pgfqpoint{10.845790in}{2.275306in}}%
\pgfpathlineto{\pgfqpoint{10.798768in}{2.302467in}}%
\pgfpathlineto{\pgfqpoint{10.753168in}{2.330547in}}%
\pgfpathlineto{\pgfqpoint{10.707461in}{2.250711in}}%
\pgfpathlineto{\pgfqpoint{10.660112in}{2.222199in}}%
\pgfpathlineto{\pgfqpoint{10.613886in}{2.264712in}}%
\pgfpathlineto{\pgfqpoint{10.568154in}{2.346468in}}%
\pgfpathlineto{\pgfqpoint{10.521105in}{2.287380in}}%
\pgfpathlineto{\pgfqpoint{10.475642in}{2.274485in}}%
\pgfpathlineto{\pgfqpoint{10.429362in}{2.294777in}}%
\pgfpathlineto{\pgfqpoint{10.381570in}{2.293680in}}%
\pgfpathlineto{\pgfqpoint{10.335334in}{2.240312in}}%
\pgfpathlineto{\pgfqpoint{10.289084in}{2.291417in}}%
\pgfpathlineto{\pgfqpoint{10.242601in}{2.276235in}}%
\pgfpathlineto{\pgfqpoint{10.197247in}{2.359779in}}%
\pgfpathlineto{\pgfqpoint{10.151348in}{2.230054in}}%
\pgfpathlineto{\pgfqpoint{10.103098in}{2.261364in}}%
\pgfpathlineto{\pgfqpoint{10.057613in}{2.305435in}}%
\pgfpathlineto{\pgfqpoint{10.011915in}{2.268469in}}%
\pgfpathlineto{\pgfqpoint{9.964562in}{2.297187in}}%
\pgfpathlineto{\pgfqpoint{9.918842in}{2.261142in}}%
\pgfpathlineto{\pgfqpoint{9.873173in}{2.325186in}}%
\pgfpathlineto{\pgfqpoint{9.826599in}{2.328474in}}%
\pgfpathlineto{\pgfqpoint{9.781321in}{2.303450in}}%
\pgfpathlineto{\pgfqpoint{9.735785in}{2.323431in}}%
\pgfpathlineto{\pgfqpoint{9.689497in}{2.353123in}}%
\pgfpathlineto{\pgfqpoint{9.644121in}{2.320379in}}%
\pgfpathlineto{\pgfqpoint{9.598338in}{2.314462in}}%
\pgfpathlineto{\pgfqpoint{9.552197in}{2.394099in}}%
\pgfpathlineto{\pgfqpoint{9.507583in}{2.292822in}}%
\pgfpathlineto{\pgfqpoint{9.462006in}{2.300804in}}%
\pgfpathlineto{\pgfqpoint{9.415178in}{2.314648in}}%
\pgfpathlineto{\pgfqpoint{9.369639in}{2.319684in}}%
\pgfpathlineto{\pgfqpoint{9.324423in}{2.399221in}}%
\pgfpathlineto{\pgfqpoint{9.278439in}{2.335560in}}%
\pgfpathlineto{\pgfqpoint{9.233107in}{2.417828in}}%
\pgfpathlineto{\pgfqpoint{9.188332in}{2.368490in}}%
\pgfpathlineto{\pgfqpoint{9.141270in}{2.295788in}}%
\pgfpathlineto{\pgfqpoint{9.095727in}{2.335942in}}%
\pgfpathlineto{\pgfqpoint{9.050165in}{2.367234in}}%
\pgfpathlineto{\pgfqpoint{9.003947in}{2.257547in}}%
\pgfpathlineto{\pgfqpoint{8.957924in}{2.269362in}}%
\pgfpathlineto{\pgfqpoint{8.912131in}{2.375131in}}%
\pgfpathlineto{\pgfqpoint{8.866242in}{2.269668in}}%
\pgfpathlineto{\pgfqpoint{8.821068in}{2.264133in}}%
\pgfpathlineto{\pgfqpoint{8.774902in}{2.285186in}}%
\pgfpathlineto{\pgfqpoint{8.727404in}{2.294828in}}%
\pgfpathlineto{\pgfqpoint{8.681080in}{2.191512in}}%
\pgfpathlineto{\pgfqpoint{8.635315in}{2.356202in}}%
\pgfpathlineto{\pgfqpoint{8.589124in}{2.336609in}}%
\pgfpathlineto{\pgfqpoint{8.543976in}{2.376578in}}%
\pgfpathlineto{\pgfqpoint{8.499298in}{2.313127in}}%
\pgfpathlineto{\pgfqpoint{8.453780in}{2.398000in}}%
\pgfpathlineto{\pgfqpoint{8.409767in}{2.318752in}}%
\pgfpathlineto{\pgfqpoint{8.364636in}{2.398401in}}%
\pgfpathlineto{\pgfqpoint{8.318789in}{2.273525in}}%
\pgfpathlineto{\pgfqpoint{8.273006in}{2.303852in}}%
\pgfpathlineto{\pgfqpoint{8.227930in}{2.349197in}}%
\pgfpathlineto{\pgfqpoint{8.181791in}{2.272682in}}%
\pgfpathlineto{\pgfqpoint{8.136842in}{2.343950in}}%
\pgfpathlineto{\pgfqpoint{8.091881in}{2.338666in}}%
\pgfpathlineto{\pgfqpoint{8.045278in}{2.365018in}}%
\pgfpathlineto{\pgfqpoint{8.000573in}{2.240229in}}%
\pgfpathlineto{\pgfqpoint{7.955879in}{2.368915in}}%
\pgfpathlineto{\pgfqpoint{7.910161in}{2.351596in}}%
\pgfpathlineto{\pgfqpoint{7.865263in}{2.303547in}}%
\pgfpathlineto{\pgfqpoint{7.819947in}{2.321312in}}%
\pgfpathlineto{\pgfqpoint{7.773226in}{2.358435in}}%
\pgfpathlineto{\pgfqpoint{7.728803in}{2.304006in}}%
\pgfpathlineto{\pgfqpoint{7.682978in}{2.305719in}}%
\pgfpathlineto{\pgfqpoint{7.636592in}{2.356484in}}%
\pgfpathlineto{\pgfqpoint{7.591890in}{2.291838in}}%
\pgfpathlineto{\pgfqpoint{7.546892in}{2.354507in}}%
\pgfpathlineto{\pgfqpoint{7.500683in}{2.330493in}}%
\pgfpathlineto{\pgfqpoint{7.455548in}{2.349659in}}%
\pgfpathlineto{\pgfqpoint{7.409977in}{2.350045in}}%
\pgfpathlineto{\pgfqpoint{7.363749in}{2.355753in}}%
\pgfpathlineto{\pgfqpoint{7.318484in}{2.317625in}}%
\pgfpathlineto{\pgfqpoint{7.273914in}{2.402816in}}%
\pgfpathlineto{\pgfqpoint{7.228120in}{2.333442in}}%
\pgfpathlineto{\pgfqpoint{7.184277in}{2.450307in}}%
\pgfpathlineto{\pgfqpoint{7.140134in}{2.383905in}}%
\pgfpathlineto{\pgfqpoint{7.094205in}{2.405300in}}%
\pgfpathlineto{\pgfqpoint{7.050071in}{2.351361in}}%
\pgfpathlineto{\pgfqpoint{7.005149in}{2.366754in}}%
\pgfpathlineto{\pgfqpoint{6.958763in}{2.393311in}}%
\pgfpathlineto{\pgfqpoint{6.914194in}{2.320511in}}%
\pgfpathlineto{\pgfqpoint{6.869544in}{2.314775in}}%
\pgfpathlineto{\pgfqpoint{6.824012in}{2.284579in}}%
\pgfpathlineto{\pgfqpoint{6.779295in}{2.367685in}}%
\pgfpathlineto{\pgfqpoint{6.734887in}{2.277162in}}%
\pgfpathlineto{\pgfqpoint{6.688504in}{2.375017in}}%
\pgfpathlineto{\pgfqpoint{6.643960in}{2.333142in}}%
\pgfpathlineto{\pgfqpoint{6.599302in}{2.357634in}}%
\pgfpathlineto{\pgfqpoint{6.553117in}{2.281819in}}%
\pgfpathlineto{\pgfqpoint{6.507651in}{2.240583in}}%
\pgfpathlineto{\pgfqpoint{6.461324in}{2.270987in}}%
\pgfpathlineto{\pgfqpoint{6.413399in}{2.233471in}}%
\pgfpathlineto{\pgfqpoint{6.367508in}{2.243112in}}%
\pgfpathlineto{\pgfqpoint{6.321070in}{2.254446in}}%
\pgfpathlineto{\pgfqpoint{6.273584in}{2.254185in}}%
\pgfpathlineto{\pgfqpoint{6.227112in}{2.266056in}}%
\pgfpathlineto{\pgfqpoint{6.180802in}{2.237468in}}%
\pgfpathlineto{\pgfqpoint{6.133235in}{2.311575in}}%
\pgfpathlineto{\pgfqpoint{6.087229in}{2.250652in}}%
\pgfpathlineto{\pgfqpoint{6.041630in}{2.406366in}}%
\pgfpathlineto{\pgfqpoint{5.994686in}{2.234958in}}%
\pgfpathlineto{\pgfqpoint{5.948991in}{2.278529in}}%
\pgfpathlineto{\pgfqpoint{5.903366in}{2.308873in}}%
\pgfpathlineto{\pgfqpoint{5.856181in}{2.261696in}}%
\pgfpathlineto{\pgfqpoint{5.810269in}{2.381697in}}%
\pgfpathlineto{\pgfqpoint{5.765813in}{2.387220in}}%
\pgfpathlineto{\pgfqpoint{5.719569in}{2.319036in}}%
\pgfpathlineto{\pgfqpoint{5.674262in}{2.353354in}}%
\pgfpathlineto{\pgfqpoint{5.628857in}{2.247028in}}%
\pgfpathlineto{\pgfqpoint{5.581803in}{2.273113in}}%
\pgfpathlineto{\pgfqpoint{5.536203in}{2.406439in}}%
\pgfpathlineto{\pgfqpoint{5.491128in}{2.369110in}}%
\pgfpathlineto{\pgfqpoint{5.444526in}{2.326998in}}%
\pgfpathlineto{\pgfqpoint{5.399334in}{2.339578in}}%
\pgfpathlineto{\pgfqpoint{5.353614in}{2.297753in}}%
\pgfpathlineto{\pgfqpoint{5.307244in}{2.374804in}}%
\pgfpathlineto{\pgfqpoint{5.262146in}{2.304795in}}%
\pgfpathlineto{\pgfqpoint{5.216291in}{2.333026in}}%
\pgfpathlineto{\pgfqpoint{5.169192in}{2.320747in}}%
\pgfpathlineto{\pgfqpoint{5.124036in}{2.232536in}}%
\pgfpathlineto{\pgfqpoint{5.078074in}{2.306542in}}%
\pgfpathlineto{\pgfqpoint{5.030678in}{2.295788in}}%
\pgfpathlineto{\pgfqpoint{4.984973in}{2.320591in}}%
\pgfpathlineto{\pgfqpoint{4.939032in}{2.223770in}}%
\pgfpathlineto{\pgfqpoint{4.891298in}{2.270967in}}%
\pgfpathlineto{\pgfqpoint{4.845429in}{2.353522in}}%
\pgfpathlineto{\pgfqpoint{4.799887in}{2.393133in}}%
\pgfpathlineto{\pgfqpoint{4.753449in}{2.340546in}}%
\pgfpathlineto{\pgfqpoint{4.708228in}{2.367505in}}%
\pgfpathlineto{\pgfqpoint{4.663622in}{2.377694in}}%
\pgfpathlineto{\pgfqpoint{4.617918in}{2.337717in}}%
\pgfpathlineto{\pgfqpoint{4.572708in}{2.277965in}}%
\pgfpathlineto{\pgfqpoint{4.527424in}{2.257669in}}%
\pgfpathlineto{\pgfqpoint{4.480684in}{2.264512in}}%
\pgfpathlineto{\pgfqpoint{4.435284in}{2.402740in}}%
\pgfpathlineto{\pgfqpoint{4.390187in}{2.309445in}}%
\pgfpathlineto{\pgfqpoint{4.343118in}{2.288248in}}%
\pgfpathlineto{\pgfqpoint{4.297354in}{2.359011in}}%
\pgfpathlineto{\pgfqpoint{4.252389in}{2.272653in}}%
\pgfpathlineto{\pgfqpoint{4.204543in}{2.291067in}}%
\pgfpathlineto{\pgfqpoint{4.159078in}{2.360627in}}%
\pgfpathlineto{\pgfqpoint{4.114362in}{2.348997in}}%
\pgfpathlineto{\pgfqpoint{4.067507in}{2.326455in}}%
\pgfpathlineto{\pgfqpoint{4.022115in}{2.311396in}}%
\pgfpathlineto{\pgfqpoint{3.976868in}{2.367974in}}%
\pgfpathlineto{\pgfqpoint{3.930307in}{2.272708in}}%
\pgfpathlineto{\pgfqpoint{3.884614in}{2.376055in}}%
\pgfpathlineto{\pgfqpoint{3.839775in}{2.328600in}}%
\pgfpathlineto{\pgfqpoint{3.793040in}{2.272352in}}%
\pgfpathlineto{\pgfqpoint{3.747249in}{2.281275in}}%
\pgfpathlineto{\pgfqpoint{3.701949in}{2.294335in}}%
\pgfpathlineto{\pgfqpoint{3.654909in}{2.311073in}}%
\pgfpathlineto{\pgfqpoint{3.609556in}{2.340637in}}%
\pgfpathlineto{\pgfqpoint{3.564489in}{2.322218in}}%
\pgfpathlineto{\pgfqpoint{3.517815in}{2.390510in}}%
\pgfpathlineto{\pgfqpoint{3.473135in}{2.322586in}}%
\pgfpathlineto{\pgfqpoint{3.428526in}{2.309584in}}%
\pgfpathlineto{\pgfqpoint{3.381274in}{2.354180in}}%
\pgfpathlineto{\pgfqpoint{3.336184in}{2.314984in}}%
\pgfpathlineto{\pgfqpoint{3.290748in}{2.395267in}}%
\pgfpathlineto{\pgfqpoint{3.245458in}{2.358697in}}%
\pgfpathlineto{\pgfqpoint{3.200519in}{2.371658in}}%
\pgfpathlineto{\pgfqpoint{3.155580in}{2.240058in}}%
\pgfpathlineto{\pgfqpoint{3.108180in}{2.359459in}}%
\pgfpathlineto{\pgfqpoint{3.062761in}{2.352393in}}%
\pgfpathlineto{\pgfqpoint{3.017486in}{2.404176in}}%
\pgfpathlineto{\pgfqpoint{2.971917in}{2.365948in}}%
\pgfpathlineto{\pgfqpoint{2.927413in}{2.375345in}}%
\pgfpathlineto{\pgfqpoint{2.883134in}{2.342913in}}%
\pgfpathlineto{\pgfqpoint{2.836381in}{2.274534in}}%
\pgfpathlineto{\pgfqpoint{2.790736in}{2.292757in}}%
\pgfpathlineto{\pgfqpoint{2.745668in}{2.306144in}}%
\pgfpathlineto{\pgfqpoint{2.698034in}{2.323786in}}%
\pgfpathlineto{\pgfqpoint{2.651003in}{2.276180in}}%
\pgfpathlineto{\pgfqpoint{2.604306in}{2.305652in}}%
\pgfpathlineto{\pgfqpoint{2.555498in}{2.171250in}}%
\pgfpathlineto{\pgfqpoint{2.505534in}{2.112084in}}%
\pgfpathlineto{\pgfqpoint{2.452591in}{2.024750in}}%
\pgfpathlineto{\pgfqpoint{2.397147in}{2.195111in}}%
\pgfpathlineto{\pgfqpoint{2.348431in}{2.204576in}}%
\pgfpathlineto{\pgfqpoint{2.299591in}{2.228159in}}%
\pgfpathlineto{\pgfqpoint{2.249804in}{2.194731in}}%
\pgfpathlineto{\pgfqpoint{2.201242in}{2.266072in}}%
\pgfpathlineto{\pgfqpoint{2.153284in}{2.229339in}}%
\pgfpathlineto{\pgfqpoint{2.104325in}{2.219948in}}%
\pgfpathlineto{\pgfqpoint{2.057441in}{2.280610in}}%
\pgfpathlineto{\pgfqpoint{2.011209in}{2.287585in}}%
\pgfpathlineto{\pgfqpoint{1.963476in}{2.242602in}}%
\pgfpathlineto{\pgfqpoint{1.915908in}{2.177911in}}%
\pgfpathlineto{\pgfqpoint{1.869493in}{2.304360in}}%
\pgfpathlineto{\pgfqpoint{1.822804in}{2.275353in}}%
\pgfpathlineto{\pgfqpoint{1.778099in}{2.423736in}}%
\pgfpathlineto{\pgfqpoint{1.734500in}{2.359584in}}%
\pgfpathlineto{\pgfqpoint{1.689326in}{2.354467in}}%
\pgfpathlineto{\pgfqpoint{1.645274in}{2.335672in}}%
\pgfpathlineto{\pgfqpoint{1.600816in}{2.381779in}}%
\pgfpathlineto{\pgfqpoint{1.555718in}{2.376662in}}%
\pgfpathlineto{\pgfqpoint{1.511740in}{2.401650in}}%
\pgfpathlineto{\pgfqpoint{1.468334in}{2.414236in}}%
\pgfpathlineto{\pgfqpoint{1.422957in}{2.293230in}}%
\pgfpathlineto{\pgfqpoint{1.378287in}{2.396877in}}%
\pgfpathlineto{\pgfqpoint{1.334262in}{2.373334in}}%
\pgfpathlineto{\pgfqpoint{1.289074in}{2.354019in}}%
\pgfpathlineto{\pgfqpoint{1.244609in}{2.265636in}}%
\pgfpathlineto{\pgfqpoint{1.200192in}{2.392641in}}%
\pgfpathlineto{\pgfqpoint{1.155171in}{2.420320in}}%
\pgfpathlineto{\pgfqpoint{1.111790in}{2.359420in}}%
\pgfpathlineto{\pgfqpoint{1.067773in}{2.234679in}}%
\pgfpathlineto{\pgfqpoint{1.021908in}{2.375139in}}%
\pgfpathlineto{\pgfqpoint{0.978015in}{2.340910in}}%
\pgfpathlineto{\pgfqpoint{0.933783in}{2.324601in}}%
\pgfpathlineto{\pgfqpoint{0.887244in}{2.236861in}}%
\pgfpathlineto{\pgfqpoint{0.842612in}{2.325647in}}%
\pgfpathlineto{\pgfqpoint{0.797895in}{1.738591in}}%
\pgfpathclose%
\pgfusepath{fill}%
\end{pgfscope}%
\begin{pgfscope}%
\pgfpathrectangle{\pgfqpoint{0.781402in}{0.773588in}}{\pgfqpoint{1.440244in}{5.415119in}}%
\pgfusepath{clip}%
\pgfsetbuttcap%
\pgfsetroundjoin%
\definecolor{currentfill}{rgb}{0.549020,0.337255,0.294118}%
\pgfsetfillcolor{currentfill}%
\pgfsetlinewidth{0.000000pt}%
\definecolor{currentstroke}{rgb}{0.000000,0.000000,0.000000}%
\pgfsetstrokecolor{currentstroke}%
\pgfsetdash{}{0pt}%
\pgfpathmoveto{\pgfqpoint{0.797895in}{2.142129in}}%
\pgfpathlineto{\pgfqpoint{0.797895in}{1.738591in}}%
\pgfpathlineto{\pgfqpoint{0.842612in}{2.325647in}}%
\pgfpathlineto{\pgfqpoint{0.887244in}{2.236861in}}%
\pgfpathlineto{\pgfqpoint{0.933783in}{2.324601in}}%
\pgfpathlineto{\pgfqpoint{0.978015in}{2.340910in}}%
\pgfpathlineto{\pgfqpoint{1.021908in}{2.375139in}}%
\pgfpathlineto{\pgfqpoint{1.067773in}{2.234679in}}%
\pgfpathlineto{\pgfqpoint{1.111790in}{2.359420in}}%
\pgfpathlineto{\pgfqpoint{1.155171in}{2.420320in}}%
\pgfpathlineto{\pgfqpoint{1.200192in}{2.392641in}}%
\pgfpathlineto{\pgfqpoint{1.244609in}{2.265636in}}%
\pgfpathlineto{\pgfqpoint{1.289074in}{2.354019in}}%
\pgfpathlineto{\pgfqpoint{1.334262in}{2.373334in}}%
\pgfpathlineto{\pgfqpoint{1.378287in}{2.396877in}}%
\pgfpathlineto{\pgfqpoint{1.422957in}{2.293230in}}%
\pgfpathlineto{\pgfqpoint{1.468334in}{2.414236in}}%
\pgfpathlineto{\pgfqpoint{1.511740in}{2.401650in}}%
\pgfpathlineto{\pgfqpoint{1.555718in}{2.376662in}}%
\pgfpathlineto{\pgfqpoint{1.600816in}{2.381779in}}%
\pgfpathlineto{\pgfqpoint{1.645274in}{2.335672in}}%
\pgfpathlineto{\pgfqpoint{1.689326in}{2.354467in}}%
\pgfpathlineto{\pgfqpoint{1.734500in}{2.359584in}}%
\pgfpathlineto{\pgfqpoint{1.778099in}{2.423736in}}%
\pgfpathlineto{\pgfqpoint{1.822804in}{2.275353in}}%
\pgfpathlineto{\pgfqpoint{1.869493in}{2.304360in}}%
\pgfpathlineto{\pgfqpoint{1.915908in}{2.177911in}}%
\pgfpathlineto{\pgfqpoint{1.963476in}{2.242602in}}%
\pgfpathlineto{\pgfqpoint{2.011209in}{2.287585in}}%
\pgfpathlineto{\pgfqpoint{2.057441in}{2.280610in}}%
\pgfpathlineto{\pgfqpoint{2.104325in}{2.219948in}}%
\pgfpathlineto{\pgfqpoint{2.153284in}{2.229339in}}%
\pgfpathlineto{\pgfqpoint{2.201242in}{2.266072in}}%
\pgfpathlineto{\pgfqpoint{2.249804in}{2.194731in}}%
\pgfpathlineto{\pgfqpoint{2.299591in}{2.228159in}}%
\pgfpathlineto{\pgfqpoint{2.348431in}{2.204576in}}%
\pgfpathlineto{\pgfqpoint{2.397147in}{2.195111in}}%
\pgfpathlineto{\pgfqpoint{2.452591in}{2.024750in}}%
\pgfpathlineto{\pgfqpoint{2.505534in}{2.112084in}}%
\pgfpathlineto{\pgfqpoint{2.555498in}{2.171250in}}%
\pgfpathlineto{\pgfqpoint{2.604306in}{2.305652in}}%
\pgfpathlineto{\pgfqpoint{2.651003in}{2.276180in}}%
\pgfpathlineto{\pgfqpoint{2.698034in}{2.323786in}}%
\pgfpathlineto{\pgfqpoint{2.745668in}{2.306144in}}%
\pgfpathlineto{\pgfqpoint{2.790736in}{2.292757in}}%
\pgfpathlineto{\pgfqpoint{2.836381in}{2.274534in}}%
\pgfpathlineto{\pgfqpoint{2.883134in}{2.342913in}}%
\pgfpathlineto{\pgfqpoint{2.927413in}{2.375345in}}%
\pgfpathlineto{\pgfqpoint{2.971917in}{2.365948in}}%
\pgfpathlineto{\pgfqpoint{3.017486in}{2.404176in}}%
\pgfpathlineto{\pgfqpoint{3.062761in}{2.352393in}}%
\pgfpathlineto{\pgfqpoint{3.108180in}{2.359459in}}%
\pgfpathlineto{\pgfqpoint{3.155580in}{2.240058in}}%
\pgfpathlineto{\pgfqpoint{3.200519in}{2.371658in}}%
\pgfpathlineto{\pgfqpoint{3.245458in}{2.358697in}}%
\pgfpathlineto{\pgfqpoint{3.290748in}{2.395267in}}%
\pgfpathlineto{\pgfqpoint{3.336184in}{2.314984in}}%
\pgfpathlineto{\pgfqpoint{3.381274in}{2.354180in}}%
\pgfpathlineto{\pgfqpoint{3.428526in}{2.309584in}}%
\pgfpathlineto{\pgfqpoint{3.473135in}{2.322586in}}%
\pgfpathlineto{\pgfqpoint{3.517815in}{2.390510in}}%
\pgfpathlineto{\pgfqpoint{3.564489in}{2.322218in}}%
\pgfpathlineto{\pgfqpoint{3.609556in}{2.340637in}}%
\pgfpathlineto{\pgfqpoint{3.654909in}{2.311073in}}%
\pgfpathlineto{\pgfqpoint{3.701949in}{2.294335in}}%
\pgfpathlineto{\pgfqpoint{3.747249in}{2.281275in}}%
\pgfpathlineto{\pgfqpoint{3.793040in}{2.272352in}}%
\pgfpathlineto{\pgfqpoint{3.839775in}{2.328600in}}%
\pgfpathlineto{\pgfqpoint{3.884614in}{2.376055in}}%
\pgfpathlineto{\pgfqpoint{3.930307in}{2.272708in}}%
\pgfpathlineto{\pgfqpoint{3.976868in}{2.367974in}}%
\pgfpathlineto{\pgfqpoint{4.022115in}{2.311396in}}%
\pgfpathlineto{\pgfqpoint{4.067507in}{2.326455in}}%
\pgfpathlineto{\pgfqpoint{4.114362in}{2.348997in}}%
\pgfpathlineto{\pgfqpoint{4.159078in}{2.360627in}}%
\pgfpathlineto{\pgfqpoint{4.204543in}{2.291067in}}%
\pgfpathlineto{\pgfqpoint{4.252389in}{2.272653in}}%
\pgfpathlineto{\pgfqpoint{4.297354in}{2.359011in}}%
\pgfpathlineto{\pgfqpoint{4.343118in}{2.288248in}}%
\pgfpathlineto{\pgfqpoint{4.390187in}{2.309445in}}%
\pgfpathlineto{\pgfqpoint{4.435284in}{2.402740in}}%
\pgfpathlineto{\pgfqpoint{4.480684in}{2.264512in}}%
\pgfpathlineto{\pgfqpoint{4.527424in}{2.257669in}}%
\pgfpathlineto{\pgfqpoint{4.572708in}{2.277965in}}%
\pgfpathlineto{\pgfqpoint{4.617918in}{2.337717in}}%
\pgfpathlineto{\pgfqpoint{4.663622in}{2.377694in}}%
\pgfpathlineto{\pgfqpoint{4.708228in}{2.367505in}}%
\pgfpathlineto{\pgfqpoint{4.753449in}{2.340546in}}%
\pgfpathlineto{\pgfqpoint{4.799887in}{2.393133in}}%
\pgfpathlineto{\pgfqpoint{4.845429in}{2.353522in}}%
\pgfpathlineto{\pgfqpoint{4.891298in}{2.270967in}}%
\pgfpathlineto{\pgfqpoint{4.939032in}{2.223770in}}%
\pgfpathlineto{\pgfqpoint{4.984973in}{2.320591in}}%
\pgfpathlineto{\pgfqpoint{5.030678in}{2.295788in}}%
\pgfpathlineto{\pgfqpoint{5.078074in}{2.306542in}}%
\pgfpathlineto{\pgfqpoint{5.124036in}{2.232536in}}%
\pgfpathlineto{\pgfqpoint{5.169192in}{2.320747in}}%
\pgfpathlineto{\pgfqpoint{5.216291in}{2.333026in}}%
\pgfpathlineto{\pgfqpoint{5.262146in}{2.304795in}}%
\pgfpathlineto{\pgfqpoint{5.307244in}{2.374804in}}%
\pgfpathlineto{\pgfqpoint{5.353614in}{2.297753in}}%
\pgfpathlineto{\pgfqpoint{5.399334in}{2.339578in}}%
\pgfpathlineto{\pgfqpoint{5.444526in}{2.326998in}}%
\pgfpathlineto{\pgfqpoint{5.491128in}{2.369110in}}%
\pgfpathlineto{\pgfqpoint{5.536203in}{2.406439in}}%
\pgfpathlineto{\pgfqpoint{5.581803in}{2.273113in}}%
\pgfpathlineto{\pgfqpoint{5.628857in}{2.247028in}}%
\pgfpathlineto{\pgfqpoint{5.674262in}{2.353354in}}%
\pgfpathlineto{\pgfqpoint{5.719569in}{2.319036in}}%
\pgfpathlineto{\pgfqpoint{5.765813in}{2.387220in}}%
\pgfpathlineto{\pgfqpoint{5.810269in}{2.381697in}}%
\pgfpathlineto{\pgfqpoint{5.856181in}{2.261696in}}%
\pgfpathlineto{\pgfqpoint{5.903366in}{2.308873in}}%
\pgfpathlineto{\pgfqpoint{5.948991in}{2.278529in}}%
\pgfpathlineto{\pgfqpoint{5.994686in}{2.234958in}}%
\pgfpathlineto{\pgfqpoint{6.041630in}{2.406366in}}%
\pgfpathlineto{\pgfqpoint{6.087229in}{2.250652in}}%
\pgfpathlineto{\pgfqpoint{6.133235in}{2.311575in}}%
\pgfpathlineto{\pgfqpoint{6.180802in}{2.237468in}}%
\pgfpathlineto{\pgfqpoint{6.227112in}{2.266056in}}%
\pgfpathlineto{\pgfqpoint{6.273584in}{2.254185in}}%
\pgfpathlineto{\pgfqpoint{6.321070in}{2.254446in}}%
\pgfpathlineto{\pgfqpoint{6.367508in}{2.243112in}}%
\pgfpathlineto{\pgfqpoint{6.413399in}{2.233471in}}%
\pgfpathlineto{\pgfqpoint{6.461324in}{2.270987in}}%
\pgfpathlineto{\pgfqpoint{6.507651in}{2.240583in}}%
\pgfpathlineto{\pgfqpoint{6.553117in}{2.281819in}}%
\pgfpathlineto{\pgfqpoint{6.599302in}{2.357634in}}%
\pgfpathlineto{\pgfqpoint{6.643960in}{2.333142in}}%
\pgfpathlineto{\pgfqpoint{6.688504in}{2.375017in}}%
\pgfpathlineto{\pgfqpoint{6.734887in}{2.277162in}}%
\pgfpathlineto{\pgfqpoint{6.779295in}{2.367685in}}%
\pgfpathlineto{\pgfqpoint{6.824012in}{2.284579in}}%
\pgfpathlineto{\pgfqpoint{6.869544in}{2.314775in}}%
\pgfpathlineto{\pgfqpoint{6.914194in}{2.320511in}}%
\pgfpathlineto{\pgfqpoint{6.958763in}{2.393311in}}%
\pgfpathlineto{\pgfqpoint{7.005149in}{2.366754in}}%
\pgfpathlineto{\pgfqpoint{7.050071in}{2.351361in}}%
\pgfpathlineto{\pgfqpoint{7.094205in}{2.405300in}}%
\pgfpathlineto{\pgfqpoint{7.140134in}{2.383905in}}%
\pgfpathlineto{\pgfqpoint{7.184277in}{2.450307in}}%
\pgfpathlineto{\pgfqpoint{7.228120in}{2.333442in}}%
\pgfpathlineto{\pgfqpoint{7.273914in}{2.402816in}}%
\pgfpathlineto{\pgfqpoint{7.318484in}{2.317625in}}%
\pgfpathlineto{\pgfqpoint{7.363749in}{2.355753in}}%
\pgfpathlineto{\pgfqpoint{7.409977in}{2.350045in}}%
\pgfpathlineto{\pgfqpoint{7.455548in}{2.349659in}}%
\pgfpathlineto{\pgfqpoint{7.500683in}{2.330493in}}%
\pgfpathlineto{\pgfqpoint{7.546892in}{2.354507in}}%
\pgfpathlineto{\pgfqpoint{7.591890in}{2.291838in}}%
\pgfpathlineto{\pgfqpoint{7.636592in}{2.356484in}}%
\pgfpathlineto{\pgfqpoint{7.682978in}{2.305719in}}%
\pgfpathlineto{\pgfqpoint{7.728803in}{2.304006in}}%
\pgfpathlineto{\pgfqpoint{7.773226in}{2.358435in}}%
\pgfpathlineto{\pgfqpoint{7.819947in}{2.321312in}}%
\pgfpathlineto{\pgfqpoint{7.865263in}{2.303547in}}%
\pgfpathlineto{\pgfqpoint{7.910161in}{2.351596in}}%
\pgfpathlineto{\pgfqpoint{7.955879in}{2.368915in}}%
\pgfpathlineto{\pgfqpoint{8.000573in}{2.240229in}}%
\pgfpathlineto{\pgfqpoint{8.045278in}{2.365018in}}%
\pgfpathlineto{\pgfqpoint{8.091881in}{2.338666in}}%
\pgfpathlineto{\pgfqpoint{8.136842in}{2.343950in}}%
\pgfpathlineto{\pgfqpoint{8.181791in}{2.272682in}}%
\pgfpathlineto{\pgfqpoint{8.227930in}{2.349197in}}%
\pgfpathlineto{\pgfqpoint{8.273006in}{2.303852in}}%
\pgfpathlineto{\pgfqpoint{8.318789in}{2.273525in}}%
\pgfpathlineto{\pgfqpoint{8.364636in}{2.398401in}}%
\pgfpathlineto{\pgfqpoint{8.409767in}{2.318752in}}%
\pgfpathlineto{\pgfqpoint{8.453780in}{2.398000in}}%
\pgfpathlineto{\pgfqpoint{8.499298in}{2.313127in}}%
\pgfpathlineto{\pgfqpoint{8.543976in}{2.376578in}}%
\pgfpathlineto{\pgfqpoint{8.589124in}{2.336609in}}%
\pgfpathlineto{\pgfqpoint{8.635315in}{2.356202in}}%
\pgfpathlineto{\pgfqpoint{8.681080in}{2.191512in}}%
\pgfpathlineto{\pgfqpoint{8.727404in}{2.294828in}}%
\pgfpathlineto{\pgfqpoint{8.774902in}{2.285186in}}%
\pgfpathlineto{\pgfqpoint{8.821068in}{2.264133in}}%
\pgfpathlineto{\pgfqpoint{8.866242in}{2.269668in}}%
\pgfpathlineto{\pgfqpoint{8.912131in}{2.375131in}}%
\pgfpathlineto{\pgfqpoint{8.957924in}{2.269362in}}%
\pgfpathlineto{\pgfqpoint{9.003947in}{2.257547in}}%
\pgfpathlineto{\pgfqpoint{9.050165in}{2.367234in}}%
\pgfpathlineto{\pgfqpoint{9.095727in}{2.335942in}}%
\pgfpathlineto{\pgfqpoint{9.141270in}{2.295788in}}%
\pgfpathlineto{\pgfqpoint{9.188332in}{2.368490in}}%
\pgfpathlineto{\pgfqpoint{9.233107in}{2.417828in}}%
\pgfpathlineto{\pgfqpoint{9.278439in}{2.335560in}}%
\pgfpathlineto{\pgfqpoint{9.324423in}{2.399221in}}%
\pgfpathlineto{\pgfqpoint{9.369639in}{2.319684in}}%
\pgfpathlineto{\pgfqpoint{9.415178in}{2.314648in}}%
\pgfpathlineto{\pgfqpoint{9.462006in}{2.300804in}}%
\pgfpathlineto{\pgfqpoint{9.507583in}{2.292822in}}%
\pgfpathlineto{\pgfqpoint{9.552197in}{2.394099in}}%
\pgfpathlineto{\pgfqpoint{9.598338in}{2.314462in}}%
\pgfpathlineto{\pgfqpoint{9.644121in}{2.320379in}}%
\pgfpathlineto{\pgfqpoint{9.689497in}{2.353123in}}%
\pgfpathlineto{\pgfqpoint{9.735785in}{2.323431in}}%
\pgfpathlineto{\pgfqpoint{9.781321in}{2.303450in}}%
\pgfpathlineto{\pgfqpoint{9.826599in}{2.328474in}}%
\pgfpathlineto{\pgfqpoint{9.873173in}{2.325186in}}%
\pgfpathlineto{\pgfqpoint{9.918842in}{2.261142in}}%
\pgfpathlineto{\pgfqpoint{9.964562in}{2.297187in}}%
\pgfpathlineto{\pgfqpoint{10.011915in}{2.268469in}}%
\pgfpathlineto{\pgfqpoint{10.057613in}{2.305435in}}%
\pgfpathlineto{\pgfqpoint{10.103098in}{2.261364in}}%
\pgfpathlineto{\pgfqpoint{10.151348in}{2.230054in}}%
\pgfpathlineto{\pgfqpoint{10.197247in}{2.359779in}}%
\pgfpathlineto{\pgfqpoint{10.242601in}{2.276235in}}%
\pgfpathlineto{\pgfqpoint{10.289084in}{2.291417in}}%
\pgfpathlineto{\pgfqpoint{10.335334in}{2.240312in}}%
\pgfpathlineto{\pgfqpoint{10.381570in}{2.293680in}}%
\pgfpathlineto{\pgfqpoint{10.429362in}{2.294777in}}%
\pgfpathlineto{\pgfqpoint{10.475642in}{2.274485in}}%
\pgfpathlineto{\pgfqpoint{10.521105in}{2.287380in}}%
\pgfpathlineto{\pgfqpoint{10.568154in}{2.346468in}}%
\pgfpathlineto{\pgfqpoint{10.613886in}{2.264712in}}%
\pgfpathlineto{\pgfqpoint{10.660112in}{2.222199in}}%
\pgfpathlineto{\pgfqpoint{10.707461in}{2.250711in}}%
\pgfpathlineto{\pgfqpoint{10.753168in}{2.330547in}}%
\pgfpathlineto{\pgfqpoint{10.798768in}{2.302467in}}%
\pgfpathlineto{\pgfqpoint{10.845790in}{2.275306in}}%
\pgfpathlineto{\pgfqpoint{10.891801in}{2.284564in}}%
\pgfpathlineto{\pgfqpoint{10.937381in}{2.301035in}}%
\pgfpathlineto{\pgfqpoint{10.984145in}{2.282506in}}%
\pgfpathlineto{\pgfqpoint{11.029398in}{2.356543in}}%
\pgfpathlineto{\pgfqpoint{11.074633in}{2.335174in}}%
\pgfpathlineto{\pgfqpoint{11.121978in}{2.258422in}}%
\pgfpathlineto{\pgfqpoint{11.167471in}{2.399865in}}%
\pgfpathlineto{\pgfqpoint{11.213355in}{2.260071in}}%
\pgfpathlineto{\pgfqpoint{11.261350in}{2.276617in}}%
\pgfpathlineto{\pgfqpoint{11.307361in}{2.360674in}}%
\pgfpathlineto{\pgfqpoint{11.353756in}{2.235310in}}%
\pgfpathlineto{\pgfqpoint{11.402044in}{2.332692in}}%
\pgfpathlineto{\pgfqpoint{11.448531in}{2.259828in}}%
\pgfpathlineto{\pgfqpoint{11.494874in}{2.291080in}}%
\pgfpathlineto{\pgfqpoint{11.542665in}{2.357277in}}%
\pgfpathlineto{\pgfqpoint{11.588799in}{2.230089in}}%
\pgfpathlineto{\pgfqpoint{11.634589in}{2.229229in}}%
\pgfpathlineto{\pgfqpoint{11.682053in}{2.359958in}}%
\pgfpathlineto{\pgfqpoint{11.727607in}{2.333441in}}%
\pgfpathlineto{\pgfqpoint{11.772894in}{2.306760in}}%
\pgfpathlineto{\pgfqpoint{11.819406in}{2.285077in}}%
\pgfpathlineto{\pgfqpoint{11.864862in}{2.355287in}}%
\pgfpathlineto{\pgfqpoint{11.911501in}{2.203921in}}%
\pgfpathlineto{\pgfqpoint{11.959143in}{2.350595in}}%
\pgfpathlineto{\pgfqpoint{12.004414in}{2.324057in}}%
\pgfpathlineto{\pgfqpoint{12.049869in}{2.301880in}}%
\pgfpathlineto{\pgfqpoint{12.096623in}{2.317891in}}%
\pgfpathlineto{\pgfqpoint{12.141887in}{2.351371in}}%
\pgfpathlineto{\pgfqpoint{12.187925in}{2.280826in}}%
\pgfpathlineto{\pgfqpoint{12.235985in}{2.221952in}}%
\pgfpathlineto{\pgfqpoint{12.282271in}{2.273976in}}%
\pgfpathlineto{\pgfqpoint{12.328136in}{2.257529in}}%
\pgfpathlineto{\pgfqpoint{12.375261in}{2.349502in}}%
\pgfpathlineto{\pgfqpoint{12.421372in}{2.358343in}}%
\pgfpathlineto{\pgfqpoint{12.467453in}{2.296669in}}%
\pgfpathlineto{\pgfqpoint{12.514481in}{2.340897in}}%
\pgfpathlineto{\pgfqpoint{12.560711in}{2.203625in}}%
\pgfpathlineto{\pgfqpoint{12.607615in}{2.239852in}}%
\pgfpathlineto{\pgfqpoint{12.655424in}{2.325183in}}%
\pgfpathlineto{\pgfqpoint{12.701835in}{2.279136in}}%
\pgfpathlineto{\pgfqpoint{12.749025in}{2.253182in}}%
\pgfpathlineto{\pgfqpoint{12.796907in}{2.259754in}}%
\pgfpathlineto{\pgfqpoint{12.843155in}{2.234339in}}%
\pgfpathlineto{\pgfqpoint{12.889988in}{2.289046in}}%
\pgfpathlineto{\pgfqpoint{12.936799in}{2.386807in}}%
\pgfpathlineto{\pgfqpoint{12.983596in}{2.304244in}}%
\pgfpathlineto{\pgfqpoint{13.030499in}{2.298564in}}%
\pgfpathlineto{\pgfqpoint{13.077275in}{2.323533in}}%
\pgfpathlineto{\pgfqpoint{13.123339in}{2.262921in}}%
\pgfpathlineto{\pgfqpoint{13.169075in}{2.384526in}}%
\pgfpathlineto{\pgfqpoint{13.216719in}{2.235117in}}%
\pgfpathlineto{\pgfqpoint{13.262980in}{2.338157in}}%
\pgfpathlineto{\pgfqpoint{13.308974in}{2.312159in}}%
\pgfpathlineto{\pgfqpoint{13.356518in}{2.385490in}}%
\pgfpathlineto{\pgfqpoint{13.401859in}{2.323804in}}%
\pgfpathlineto{\pgfqpoint{13.448069in}{2.240374in}}%
\pgfpathlineto{\pgfqpoint{13.495727in}{2.299059in}}%
\pgfpathlineto{\pgfqpoint{13.541646in}{2.277760in}}%
\pgfpathlineto{\pgfqpoint{13.587249in}{2.335101in}}%
\pgfpathlineto{\pgfqpoint{13.635346in}{2.278245in}}%
\pgfpathlineto{\pgfqpoint{13.682151in}{2.232128in}}%
\pgfpathlineto{\pgfqpoint{13.729143in}{2.300755in}}%
\pgfpathlineto{\pgfqpoint{13.777441in}{2.242575in}}%
\pgfpathlineto{\pgfqpoint{13.825297in}{2.178671in}}%
\pgfpathlineto{\pgfqpoint{13.873610in}{2.212443in}}%
\pgfpathlineto{\pgfqpoint{13.922606in}{2.226374in}}%
\pgfpathlineto{\pgfqpoint{13.969353in}{2.229832in}}%
\pgfpathlineto{\pgfqpoint{14.016344in}{2.286673in}}%
\pgfpathlineto{\pgfqpoint{14.064951in}{2.232270in}}%
\pgfpathlineto{\pgfqpoint{14.112147in}{2.229533in}}%
\pgfpathlineto{\pgfqpoint{14.160147in}{2.182080in}}%
\pgfpathlineto{\pgfqpoint{14.209376in}{2.193278in}}%
\pgfpathlineto{\pgfqpoint{14.255593in}{2.321238in}}%
\pgfpathlineto{\pgfqpoint{14.302418in}{2.274703in}}%
\pgfpathlineto{\pgfqpoint{14.350224in}{2.276334in}}%
\pgfpathlineto{\pgfqpoint{14.397126in}{2.209372in}}%
\pgfpathlineto{\pgfqpoint{14.443729in}{2.346640in}}%
\pgfpathlineto{\pgfqpoint{14.491726in}{2.254543in}}%
\pgfpathlineto{\pgfqpoint{14.538630in}{2.291021in}}%
\pgfpathlineto{\pgfqpoint{14.585477in}{2.246168in}}%
\pgfpathlineto{\pgfqpoint{14.633992in}{2.221244in}}%
\pgfpathlineto{\pgfqpoint{14.680226in}{2.325095in}}%
\pgfpathlineto{\pgfqpoint{14.726757in}{2.231589in}}%
\pgfpathlineto{\pgfqpoint{14.774752in}{2.259349in}}%
\pgfpathlineto{\pgfqpoint{14.821474in}{2.359201in}}%
\pgfpathlineto{\pgfqpoint{14.868242in}{2.282858in}}%
\pgfpathlineto{\pgfqpoint{14.916239in}{2.196403in}}%
\pgfpathlineto{\pgfqpoint{14.963167in}{2.252252in}}%
\pgfpathlineto{\pgfqpoint{15.010214in}{2.267582in}}%
\pgfpathlineto{\pgfqpoint{15.058774in}{2.198546in}}%
\pgfpathlineto{\pgfqpoint{15.105478in}{2.314598in}}%
\pgfpathlineto{\pgfqpoint{15.152658in}{2.295575in}}%
\pgfpathlineto{\pgfqpoint{15.201456in}{2.240614in}}%
\pgfpathlineto{\pgfqpoint{15.248863in}{2.236209in}}%
\pgfpathlineto{\pgfqpoint{15.295866in}{2.280354in}}%
\pgfpathlineto{\pgfqpoint{15.343883in}{2.307810in}}%
\pgfpathlineto{\pgfqpoint{15.389896in}{2.293893in}}%
\pgfpathlineto{\pgfqpoint{15.436052in}{2.260363in}}%
\pgfpathlineto{\pgfqpoint{15.484858in}{2.218369in}}%
\pgfpathlineto{\pgfqpoint{15.532329in}{2.219452in}}%
\pgfpathlineto{\pgfqpoint{15.579815in}{2.258501in}}%
\pgfpathlineto{\pgfqpoint{15.628660in}{2.211679in}}%
\pgfpathlineto{\pgfqpoint{15.676019in}{2.301145in}}%
\pgfpathlineto{\pgfqpoint{15.722814in}{2.300535in}}%
\pgfpathlineto{\pgfqpoint{15.770154in}{2.313157in}}%
\pgfpathlineto{\pgfqpoint{15.817273in}{2.287207in}}%
\pgfpathlineto{\pgfqpoint{15.863592in}{2.332872in}}%
\pgfpathlineto{\pgfqpoint{15.911811in}{2.228192in}}%
\pgfpathlineto{\pgfqpoint{15.957799in}{2.402601in}}%
\pgfpathlineto{\pgfqpoint{16.003926in}{2.336191in}}%
\pgfpathlineto{\pgfqpoint{16.051820in}{2.332418in}}%
\pgfpathlineto{\pgfqpoint{16.098697in}{2.270580in}}%
\pgfpathlineto{\pgfqpoint{16.146338in}{2.216517in}}%
\pgfpathlineto{\pgfqpoint{16.195175in}{2.205221in}}%
\pgfpathlineto{\pgfqpoint{16.242078in}{2.276775in}}%
\pgfpathlineto{\pgfqpoint{16.289817in}{2.219318in}}%
\pgfpathlineto{\pgfqpoint{16.339537in}{2.259149in}}%
\pgfpathlineto{\pgfqpoint{16.387666in}{2.174071in}}%
\pgfpathlineto{\pgfqpoint{16.436622in}{2.137013in}}%
\pgfpathlineto{\pgfqpoint{16.486560in}{2.250525in}}%
\pgfpathlineto{\pgfqpoint{16.533619in}{2.302678in}}%
\pgfpathlineto{\pgfqpoint{16.581135in}{2.238151in}}%
\pgfpathlineto{\pgfqpoint{16.630280in}{2.266736in}}%
\pgfpathlineto{\pgfqpoint{16.677566in}{2.244778in}}%
\pgfpathlineto{\pgfqpoint{16.724568in}{2.247475in}}%
\pgfpathlineto{\pgfqpoint{16.773713in}{2.227898in}}%
\pgfpathlineto{\pgfqpoint{16.821038in}{2.222802in}}%
\pgfpathlineto{\pgfqpoint{16.868266in}{2.244193in}}%
\pgfpathlineto{\pgfqpoint{16.916302in}{2.332512in}}%
\pgfpathlineto{\pgfqpoint{16.963533in}{2.263073in}}%
\pgfpathlineto{\pgfqpoint{17.010585in}{2.251846in}}%
\pgfpathlineto{\pgfqpoint{17.059518in}{2.170799in}}%
\pgfpathlineto{\pgfqpoint{17.107470in}{2.233795in}}%
\pgfpathlineto{\pgfqpoint{17.154507in}{2.297791in}}%
\pgfpathlineto{\pgfqpoint{17.203674in}{2.246982in}}%
\pgfpathlineto{\pgfqpoint{17.250860in}{2.264188in}}%
\pgfpathlineto{\pgfqpoint{17.298941in}{2.177400in}}%
\pgfpathlineto{\pgfqpoint{17.347576in}{2.273193in}}%
\pgfpathlineto{\pgfqpoint{17.394714in}{2.248043in}}%
\pgfpathlineto{\pgfqpoint{17.442362in}{2.234768in}}%
\pgfpathlineto{\pgfqpoint{17.491125in}{2.288470in}}%
\pgfpathlineto{\pgfqpoint{17.538409in}{2.219811in}}%
\pgfpathlineto{\pgfqpoint{17.585742in}{2.379567in}}%
\pgfpathlineto{\pgfqpoint{17.634653in}{2.295602in}}%
\pgfpathlineto{\pgfqpoint{17.681914in}{2.358516in}}%
\pgfpathlineto{\pgfqpoint{17.729727in}{2.181121in}}%
\pgfpathlineto{\pgfqpoint{17.779014in}{2.180861in}}%
\pgfpathlineto{\pgfqpoint{17.826809in}{2.181075in}}%
\pgfpathlineto{\pgfqpoint{17.874600in}{2.265803in}}%
\pgfpathlineto{\pgfqpoint{17.922885in}{2.295885in}}%
\pgfpathlineto{\pgfqpoint{17.970910in}{2.245278in}}%
\pgfpathlineto{\pgfqpoint{18.020026in}{2.099061in}}%
\pgfpathlineto{\pgfqpoint{18.069524in}{2.220034in}}%
\pgfpathlineto{\pgfqpoint{18.117307in}{2.230183in}}%
\pgfpathlineto{\pgfqpoint{18.164405in}{2.303508in}}%
\pgfpathlineto{\pgfqpoint{18.213333in}{2.208562in}}%
\pgfpathlineto{\pgfqpoint{18.260871in}{2.246190in}}%
\pgfpathlineto{\pgfqpoint{18.308441in}{2.227451in}}%
\pgfpathlineto{\pgfqpoint{18.357538in}{2.210510in}}%
\pgfpathlineto{\pgfqpoint{18.405242in}{2.249085in}}%
\pgfpathlineto{\pgfqpoint{18.452284in}{2.172965in}}%
\pgfpathlineto{\pgfqpoint{18.500915in}{2.249891in}}%
\pgfpathlineto{\pgfqpoint{18.548460in}{2.240166in}}%
\pgfpathlineto{\pgfqpoint{18.595724in}{2.207052in}}%
\pgfpathlineto{\pgfqpoint{18.644639in}{2.233839in}}%
\pgfpathlineto{\pgfqpoint{18.693122in}{2.186383in}}%
\pgfpathlineto{\pgfqpoint{18.741665in}{2.223418in}}%
\pgfpathlineto{\pgfqpoint{18.791212in}{2.175746in}}%
\pgfpathlineto{\pgfqpoint{18.839312in}{2.255904in}}%
\pgfpathlineto{\pgfqpoint{18.888188in}{2.195093in}}%
\pgfpathlineto{\pgfqpoint{18.937671in}{2.243100in}}%
\pgfpathlineto{\pgfqpoint{18.985002in}{2.256048in}}%
\pgfpathlineto{\pgfqpoint{19.033794in}{2.192082in}}%
\pgfpathlineto{\pgfqpoint{19.084207in}{2.100303in}}%
\pgfpathlineto{\pgfqpoint{19.132724in}{2.240721in}}%
\pgfpathlineto{\pgfqpoint{19.181302in}{2.175947in}}%
\pgfpathlineto{\pgfqpoint{19.231826in}{2.158168in}}%
\pgfpathlineto{\pgfqpoint{19.279742in}{2.293533in}}%
\pgfpathlineto{\pgfqpoint{19.327829in}{2.241679in}}%
\pgfpathlineto{\pgfqpoint{19.376890in}{2.239043in}}%
\pgfpathlineto{\pgfqpoint{19.425178in}{2.227549in}}%
\pgfpathlineto{\pgfqpoint{19.473154in}{2.190874in}}%
\pgfpathlineto{\pgfqpoint{19.522292in}{2.234019in}}%
\pgfpathlineto{\pgfqpoint{19.569747in}{2.259987in}}%
\pgfpathlineto{\pgfqpoint{19.617707in}{2.182953in}}%
\pgfpathlineto{\pgfqpoint{19.666940in}{2.243279in}}%
\pgfpathlineto{\pgfqpoint{19.714265in}{2.275859in}}%
\pgfpathlineto{\pgfqpoint{19.761678in}{2.193047in}}%
\pgfpathlineto{\pgfqpoint{19.810712in}{2.223149in}}%
\pgfpathlineto{\pgfqpoint{19.858933in}{2.236420in}}%
\pgfpathlineto{\pgfqpoint{19.906615in}{2.281059in}}%
\pgfpathlineto{\pgfqpoint{19.955252in}{2.296405in}}%
\pgfpathlineto{\pgfqpoint{20.003692in}{2.175230in}}%
\pgfpathlineto{\pgfqpoint{20.053312in}{2.243716in}}%
\pgfpathlineto{\pgfqpoint{20.104350in}{2.161230in}}%
\pgfpathlineto{\pgfqpoint{20.153679in}{2.162851in}}%
\pgfpathlineto{\pgfqpoint{20.203615in}{2.192230in}}%
\pgfpathlineto{\pgfqpoint{20.255593in}{2.107988in}}%
\pgfpathlineto{\pgfqpoint{20.305924in}{2.134136in}}%
\pgfpathlineto{\pgfqpoint{20.355462in}{2.199386in}}%
\pgfpathlineto{\pgfqpoint{20.406414in}{2.188387in}}%
\pgfpathlineto{\pgfqpoint{20.455604in}{2.155524in}}%
\pgfpathlineto{\pgfqpoint{20.505208in}{2.258629in}}%
\pgfpathlineto{\pgfqpoint{20.556290in}{2.221965in}}%
\pgfpathlineto{\pgfqpoint{20.605567in}{2.126089in}}%
\pgfpathlineto{\pgfqpoint{20.654668in}{2.192399in}}%
\pgfpathlineto{\pgfqpoint{20.704616in}{2.268763in}}%
\pgfpathlineto{\pgfqpoint{20.753558in}{2.212760in}}%
\pgfpathlineto{\pgfqpoint{20.802965in}{2.283576in}}%
\pgfpathlineto{\pgfqpoint{20.854118in}{2.134033in}}%
\pgfpathlineto{\pgfqpoint{20.904164in}{2.166381in}}%
\pgfpathlineto{\pgfqpoint{20.953585in}{2.212485in}}%
\pgfpathlineto{\pgfqpoint{21.003693in}{2.185814in}}%
\pgfpathlineto{\pgfqpoint{21.053515in}{2.155531in}}%
\pgfpathlineto{\pgfqpoint{21.102370in}{2.241324in}}%
\pgfpathlineto{\pgfqpoint{21.152719in}{2.201026in}}%
\pgfpathlineto{\pgfqpoint{21.201683in}{2.181293in}}%
\pgfpathlineto{\pgfqpoint{21.250655in}{2.245041in}}%
\pgfpathlineto{\pgfqpoint{21.300955in}{2.191708in}}%
\pgfpathlineto{\pgfqpoint{21.350353in}{2.208270in}}%
\pgfpathlineto{\pgfqpoint{21.399733in}{2.175744in}}%
\pgfpathlineto{\pgfqpoint{21.450830in}{2.168605in}}%
\pgfpathlineto{\pgfqpoint{21.501023in}{2.150548in}}%
\pgfpathlineto{\pgfqpoint{21.550957in}{2.167679in}}%
\pgfpathlineto{\pgfqpoint{21.602538in}{2.160895in}}%
\pgfpathlineto{\pgfqpoint{21.652276in}{2.218841in}}%
\pgfpathlineto{\pgfqpoint{21.700864in}{2.213775in}}%
\pgfpathlineto{\pgfqpoint{21.751624in}{2.227472in}}%
\pgfpathlineto{\pgfqpoint{21.800474in}{2.226448in}}%
\pgfpathlineto{\pgfqpoint{21.849580in}{2.230011in}}%
\pgfpathlineto{\pgfqpoint{21.900834in}{2.128498in}}%
\pgfpathlineto{\pgfqpoint{21.950544in}{2.161953in}}%
\pgfpathlineto{\pgfqpoint{21.999803in}{2.276898in}}%
\pgfpathlineto{\pgfqpoint{22.050354in}{2.153076in}}%
\pgfpathlineto{\pgfqpoint{22.099907in}{2.258881in}}%
\pgfpathlineto{\pgfqpoint{22.148817in}{2.251963in}}%
\pgfpathlineto{\pgfqpoint{22.200587in}{2.138280in}}%
\pgfpathlineto{\pgfqpoint{22.250634in}{2.185196in}}%
\pgfpathlineto{\pgfqpoint{22.300933in}{2.094855in}}%
\pgfpathlineto{\pgfqpoint{22.352827in}{2.157190in}}%
\pgfpathlineto{\pgfqpoint{22.402084in}{2.133288in}}%
\pgfpathlineto{\pgfqpoint{22.451976in}{2.206816in}}%
\pgfpathlineto{\pgfqpoint{22.503552in}{2.135991in}}%
\pgfpathlineto{\pgfqpoint{22.554073in}{2.136690in}}%
\pgfpathlineto{\pgfqpoint{22.603657in}{2.179999in}}%
\pgfpathlineto{\pgfqpoint{22.654822in}{2.180208in}}%
\pgfpathlineto{\pgfqpoint{22.703398in}{2.305817in}}%
\pgfpathlineto{\pgfqpoint{22.751700in}{2.248709in}}%
\pgfpathlineto{\pgfqpoint{22.803829in}{2.168087in}}%
\pgfpathlineto{\pgfqpoint{22.853966in}{2.158422in}}%
\pgfpathlineto{\pgfqpoint{22.904236in}{2.104244in}}%
\pgfpathlineto{\pgfqpoint{22.956359in}{2.173476in}}%
\pgfpathlineto{\pgfqpoint{23.006022in}{2.146954in}}%
\pgfpathlineto{\pgfqpoint{23.055187in}{2.235002in}}%
\pgfpathlineto{\pgfqpoint{23.107134in}{2.079790in}}%
\pgfpathlineto{\pgfqpoint{23.157756in}{2.136082in}}%
\pgfpathlineto{\pgfqpoint{23.208694in}{2.124981in}}%
\pgfpathlineto{\pgfqpoint{23.260726in}{2.152878in}}%
\pgfpathlineto{\pgfqpoint{23.310918in}{2.148300in}}%
\pgfpathlineto{\pgfqpoint{23.361310in}{2.170140in}}%
\pgfpathlineto{\pgfqpoint{23.411909in}{2.198735in}}%
\pgfpathlineto{\pgfqpoint{23.462017in}{2.157637in}}%
\pgfpathlineto{\pgfqpoint{23.511867in}{2.137407in}}%
\pgfpathlineto{\pgfqpoint{23.563429in}{2.143388in}}%
\pgfpathlineto{\pgfqpoint{23.611923in}{2.282936in}}%
\pgfpathlineto{\pgfqpoint{23.661575in}{2.159418in}}%
\pgfpathlineto{\pgfqpoint{23.713074in}{2.221155in}}%
\pgfpathlineto{\pgfqpoint{23.762329in}{2.182126in}}%
\pgfpathlineto{\pgfqpoint{23.811512in}{2.259241in}}%
\pgfpathlineto{\pgfqpoint{23.862673in}{2.175857in}}%
\pgfpathlineto{\pgfqpoint{23.913088in}{2.119330in}}%
\pgfpathlineto{\pgfqpoint{23.964514in}{2.116542in}}%
\pgfpathlineto{\pgfqpoint{24.016442in}{2.201493in}}%
\pgfpathlineto{\pgfqpoint{24.066933in}{2.162367in}}%
\pgfpathlineto{\pgfqpoint{24.118637in}{2.144171in}}%
\pgfpathlineto{\pgfqpoint{24.170873in}{2.170742in}}%
\pgfpathlineto{\pgfqpoint{24.221010in}{2.157432in}}%
\pgfpathlineto{\pgfqpoint{24.271277in}{2.149127in}}%
\pgfpathlineto{\pgfqpoint{24.321907in}{2.211605in}}%
\pgfpathlineto{\pgfqpoint{24.371890in}{2.208071in}}%
\pgfpathlineto{\pgfqpoint{24.422475in}{2.160039in}}%
\pgfpathlineto{\pgfqpoint{24.474381in}{2.139169in}}%
\pgfpathlineto{\pgfqpoint{24.524258in}{2.209164in}}%
\pgfpathlineto{\pgfqpoint{24.573998in}{2.204872in}}%
\pgfpathlineto{\pgfqpoint{24.626556in}{2.072454in}}%
\pgfpathlineto{\pgfqpoint{24.676389in}{2.191465in}}%
\pgfpathlineto{\pgfqpoint{24.726499in}{2.192402in}}%
\pgfpathlineto{\pgfqpoint{24.778304in}{2.187482in}}%
\pgfpathlineto{\pgfqpoint{24.828140in}{2.234647in}}%
\pgfpathlineto{\pgfqpoint{24.878143in}{2.139836in}}%
\pgfpathlineto{\pgfqpoint{24.928869in}{2.240633in}}%
\pgfpathlineto{\pgfqpoint{24.978211in}{2.200234in}}%
\pgfpathlineto{\pgfqpoint{25.028630in}{2.166090in}}%
\pgfpathlineto{\pgfqpoint{25.080539in}{2.156690in}}%
\pgfpathlineto{\pgfqpoint{25.131043in}{2.169034in}}%
\pgfpathlineto{\pgfqpoint{25.181554in}{2.148863in}}%
\pgfpathlineto{\pgfqpoint{25.232944in}{2.198963in}}%
\pgfpathlineto{\pgfqpoint{25.282222in}{2.174410in}}%
\pgfpathlineto{\pgfqpoint{25.332608in}{2.124537in}}%
\pgfpathlineto{\pgfqpoint{25.383622in}{2.199798in}}%
\pgfpathlineto{\pgfqpoint{25.433954in}{2.179510in}}%
\pgfpathlineto{\pgfqpoint{25.483381in}{2.187707in}}%
\pgfpathlineto{\pgfqpoint{25.535558in}{2.107499in}}%
\pgfpathlineto{\pgfqpoint{25.586387in}{2.087238in}}%
\pgfpathlineto{\pgfqpoint{25.637513in}{2.153196in}}%
\pgfpathlineto{\pgfqpoint{25.689152in}{2.130566in}}%
\pgfpathlineto{\pgfqpoint{25.740073in}{2.113429in}}%
\pgfpathlineto{\pgfqpoint{25.790287in}{2.177297in}}%
\pgfpathlineto{\pgfqpoint{25.841969in}{2.202252in}}%
\pgfpathlineto{\pgfqpoint{25.891991in}{2.129676in}}%
\pgfpathlineto{\pgfqpoint{25.942005in}{2.189113in}}%
\pgfpathlineto{\pgfqpoint{25.994130in}{2.161647in}}%
\pgfpathlineto{\pgfqpoint{26.044216in}{2.172740in}}%
\pgfpathlineto{\pgfqpoint{26.093895in}{2.172470in}}%
\pgfpathlineto{\pgfqpoint{26.144837in}{2.202460in}}%
\pgfpathlineto{\pgfqpoint{26.195263in}{2.138438in}}%
\pgfpathlineto{\pgfqpoint{26.245717in}{2.188680in}}%
\pgfpathlineto{\pgfqpoint{26.297518in}{2.140127in}}%
\pgfpathlineto{\pgfqpoint{26.346588in}{2.210846in}}%
\pgfpathlineto{\pgfqpoint{26.395829in}{2.216585in}}%
\pgfpathlineto{\pgfqpoint{26.447655in}{2.147600in}}%
\pgfpathlineto{\pgfqpoint{26.499064in}{2.073278in}}%
\pgfpathlineto{\pgfqpoint{26.549446in}{2.196254in}}%
\pgfpathlineto{\pgfqpoint{26.601868in}{2.125000in}}%
\pgfpathlineto{\pgfqpoint{26.653161in}{2.155346in}}%
\pgfpathlineto{\pgfqpoint{26.704176in}{2.127344in}}%
\pgfpathlineto{\pgfqpoint{26.756448in}{2.167275in}}%
\pgfpathlineto{\pgfqpoint{26.807173in}{2.169425in}}%
\pgfpathlineto{\pgfqpoint{26.857706in}{2.142711in}}%
\pgfpathlineto{\pgfqpoint{26.909530in}{2.167554in}}%
\pgfpathlineto{\pgfqpoint{26.960748in}{2.126317in}}%
\pgfpathlineto{\pgfqpoint{27.010539in}{2.206184in}}%
\pgfpathlineto{\pgfqpoint{27.063062in}{2.127542in}}%
\pgfpathlineto{\pgfqpoint{27.113950in}{2.191289in}}%
\pgfpathlineto{\pgfqpoint{27.165274in}{2.106034in}}%
\pgfpathlineto{\pgfqpoint{27.218587in}{2.125482in}}%
\pgfpathlineto{\pgfqpoint{27.269679in}{2.127160in}}%
\pgfpathlineto{\pgfqpoint{27.321160in}{2.084956in}}%
\pgfpathlineto{\pgfqpoint{27.372863in}{2.262046in}}%
\pgfpathlineto{\pgfqpoint{27.423066in}{2.151542in}}%
\pgfpathlineto{\pgfqpoint{27.473173in}{2.149431in}}%
\pgfpathlineto{\pgfqpoint{27.525849in}{2.059972in}}%
\pgfpathlineto{\pgfqpoint{27.576178in}{2.193467in}}%
\pgfpathlineto{\pgfqpoint{27.626352in}{2.188092in}}%
\pgfpathlineto{\pgfqpoint{27.678294in}{2.202638in}}%
\pgfpathlineto{\pgfqpoint{27.728571in}{2.169913in}}%
\pgfpathlineto{\pgfqpoint{27.779331in}{2.127843in}}%
\pgfpathlineto{\pgfqpoint{27.833067in}{2.102078in}}%
\pgfpathlineto{\pgfqpoint{27.883786in}{2.171648in}}%
\pgfpathlineto{\pgfqpoint{27.935127in}{2.142057in}}%
\pgfpathlineto{\pgfqpoint{27.988562in}{2.148509in}}%
\pgfpathlineto{\pgfqpoint{28.039571in}{2.133339in}}%
\pgfpathlineto{\pgfqpoint{28.090650in}{2.154372in}}%
\pgfpathlineto{\pgfqpoint{28.143944in}{2.103896in}}%
\pgfpathlineto{\pgfqpoint{28.195034in}{2.194715in}}%
\pgfpathlineto{\pgfqpoint{28.245320in}{2.152372in}}%
\pgfpathlineto{\pgfqpoint{28.297462in}{2.221346in}}%
\pgfpathlineto{\pgfqpoint{28.347853in}{2.128908in}}%
\pgfpathlineto{\pgfqpoint{28.399138in}{2.108488in}}%
\pgfpathlineto{\pgfqpoint{28.451952in}{2.085182in}}%
\pgfpathlineto{\pgfqpoint{28.502707in}{2.176910in}}%
\pgfpathlineto{\pgfqpoint{28.553749in}{2.135069in}}%
\pgfpathlineto{\pgfqpoint{28.605351in}{2.184676in}}%
\pgfpathlineto{\pgfqpoint{28.656185in}{2.169046in}}%
\pgfpathlineto{\pgfqpoint{28.706300in}{2.197239in}}%
\pgfpathlineto{\pgfqpoint{28.758074in}{2.179648in}}%
\pgfpathlineto{\pgfqpoint{28.808587in}{2.154364in}}%
\pgfpathlineto{\pgfqpoint{28.860347in}{2.116407in}}%
\pgfpathlineto{\pgfqpoint{28.913538in}{2.139636in}}%
\pgfpathlineto{\pgfqpoint{28.965063in}{2.168914in}}%
\pgfpathlineto{\pgfqpoint{29.015651in}{2.178709in}}%
\pgfpathlineto{\pgfqpoint{29.068863in}{2.138193in}}%
\pgfpathlineto{\pgfqpoint{29.121029in}{2.088575in}}%
\pgfpathlineto{\pgfqpoint{29.173700in}{2.077484in}}%
\pgfpathlineto{\pgfqpoint{29.226818in}{2.160460in}}%
\pgfpathlineto{\pgfqpoint{29.279399in}{2.100503in}}%
\pgfpathlineto{\pgfqpoint{29.332261in}{2.082785in}}%
\pgfpathlineto{\pgfqpoint{29.386173in}{2.150986in}}%
\pgfpathlineto{\pgfqpoint{29.438335in}{2.133563in}}%
\pgfpathlineto{\pgfqpoint{29.490075in}{2.134356in}}%
\pgfpathlineto{\pgfqpoint{29.543760in}{2.082360in}}%
\pgfpathlineto{\pgfqpoint{29.596078in}{2.080609in}}%
\pgfpathlineto{\pgfqpoint{29.647988in}{2.152791in}}%
\pgfpathlineto{\pgfqpoint{29.701409in}{2.123741in}}%
\pgfpathlineto{\pgfqpoint{29.753093in}{2.161564in}}%
\pgfpathlineto{\pgfqpoint{29.805208in}{2.142856in}}%
\pgfpathlineto{\pgfqpoint{29.858913in}{2.093735in}}%
\pgfpathlineto{\pgfqpoint{29.910558in}{2.163910in}}%
\pgfpathlineto{\pgfqpoint{29.962066in}{2.109977in}}%
\pgfpathlineto{\pgfqpoint{30.014968in}{2.192509in}}%
\pgfpathlineto{\pgfqpoint{30.066247in}{2.157383in}}%
\pgfpathlineto{\pgfqpoint{30.117082in}{2.164722in}}%
\pgfpathlineto{\pgfqpoint{30.169497in}{2.140896in}}%
\pgfpathlineto{\pgfqpoint{30.221783in}{2.094139in}}%
\pgfpathlineto{\pgfqpoint{30.276704in}{1.990714in}}%
\pgfpathlineto{\pgfqpoint{30.337041in}{1.985679in}}%
\pgfpathlineto{\pgfqpoint{30.397563in}{1.931511in}}%
\pgfpathlineto{\pgfqpoint{30.457369in}{1.933531in}}%
\pgfpathlineto{\pgfqpoint{30.519963in}{1.909921in}}%
\pgfpathlineto{\pgfqpoint{30.582999in}{1.881027in}}%
\pgfpathlineto{\pgfqpoint{30.648377in}{1.818030in}}%
\pgfpathlineto{\pgfqpoint{30.717993in}{1.799199in}}%
\pgfpathlineto{\pgfqpoint{30.786068in}{1.815928in}}%
\pgfpathlineto{\pgfqpoint{30.855684in}{1.770692in}}%
\pgfpathlineto{\pgfqpoint{30.928370in}{1.744580in}}%
\pgfpathlineto{\pgfqpoint{31.000738in}{1.786581in}}%
\pgfpathlineto{\pgfqpoint{31.073353in}{1.696625in}}%
\pgfpathlineto{\pgfqpoint{31.150292in}{1.713936in}}%
\pgfpathlineto{\pgfqpoint{31.225476in}{1.715736in}}%
\pgfpathlineto{\pgfqpoint{31.304546in}{1.621377in}}%
\pgfpathlineto{\pgfqpoint{31.385546in}{1.683708in}}%
\pgfpathlineto{\pgfqpoint{31.463397in}{1.664462in}}%
\pgfpathlineto{\pgfqpoint{31.543273in}{1.623152in}}%
\pgfpathlineto{\pgfqpoint{31.629527in}{1.642355in}}%
\pgfpathlineto{\pgfqpoint{31.715049in}{1.618107in}}%
\pgfpathlineto{\pgfqpoint{31.802724in}{1.559328in}}%
\pgfpathlineto{\pgfqpoint{31.890397in}{1.621243in}}%
\pgfpathlineto{\pgfqpoint{31.975639in}{1.606211in}}%
\pgfpathlineto{\pgfqpoint{32.062211in}{1.548891in}}%
\pgfpathlineto{\pgfqpoint{32.153206in}{1.569780in}}%
\pgfpathlineto{\pgfqpoint{32.244846in}{1.565071in}}%
\pgfpathlineto{\pgfqpoint{32.334387in}{1.559840in}}%
\pgfpathlineto{\pgfqpoint{32.429213in}{1.548059in}}%
\pgfpathlineto{\pgfqpoint{32.518604in}{1.577644in}}%
\pgfpathlineto{\pgfqpoint{32.584565in}{1.527806in}}%
\pgfpathlineto{\pgfqpoint{32.638735in}{1.466197in}}%
\pgfpathlineto{\pgfqpoint{32.691213in}{1.465799in}}%
\pgfpathlineto{\pgfqpoint{32.743153in}{1.453492in}}%
\pgfpathlineto{\pgfqpoint{32.797230in}{1.405600in}}%
\pgfpathlineto{\pgfqpoint{32.850576in}{1.393664in}}%
\pgfpathlineto{\pgfqpoint{32.903667in}{1.442369in}}%
\pgfpathlineto{\pgfqpoint{32.957868in}{1.459985in}}%
\pgfpathlineto{\pgfqpoint{33.010153in}{1.429618in}}%
\pgfpathlineto{\pgfqpoint{33.062733in}{1.414323in}}%
\pgfpathlineto{\pgfqpoint{33.115518in}{1.488284in}}%
\pgfpathlineto{\pgfqpoint{33.167699in}{1.423144in}}%
\pgfpathlineto{\pgfqpoint{33.219830in}{1.230236in}}%
\pgfpathlineto{\pgfqpoint{33.273785in}{0.773588in}}%
\pgfpathlineto{\pgfqpoint{33.326652in}{0.773588in}}%
\pgfpathlineto{\pgfqpoint{33.379482in}{0.773588in}}%
\pgfpathlineto{\pgfqpoint{33.433283in}{0.773588in}}%
\pgfpathlineto{\pgfqpoint{33.484775in}{0.773588in}}%
\pgfpathlineto{\pgfqpoint{33.536879in}{0.773588in}}%
\pgfpathlineto{\pgfqpoint{33.590505in}{0.773588in}}%
\pgfpathlineto{\pgfqpoint{33.642373in}{0.773588in}}%
\pgfpathlineto{\pgfqpoint{33.693680in}{0.773588in}}%
\pgfpathlineto{\pgfqpoint{33.746200in}{0.773588in}}%
\pgfpathlineto{\pgfqpoint{33.784450in}{0.773588in}}%
\pgfpathlineto{\pgfqpoint{33.832264in}{0.773588in}}%
\pgfpathlineto{\pgfqpoint{33.870742in}{1.454760in}}%
\pgfpathlineto{\pgfqpoint{33.912991in}{1.669618in}}%
\pgfpathlineto{\pgfqpoint{33.952831in}{1.906942in}}%
\pgfpathlineto{\pgfqpoint{33.989641in}{2.224563in}}%
\pgfpathlineto{\pgfqpoint{34.021637in}{2.864879in}}%
\pgfpathlineto{\pgfqpoint{34.052161in}{3.729137in}}%
\pgfpathlineto{\pgfqpoint{34.076513in}{5.235025in}}%
\pgfpathlineto{\pgfqpoint{34.101659in}{5.182737in}}%
\pgfpathlineto{\pgfqpoint{34.125436in}{5.364631in}}%
\pgfpathlineto{\pgfqpoint{34.150180in}{5.483715in}}%
\pgfpathlineto{\pgfqpoint{34.173862in}{5.446238in}}%
\pgfpathlineto{\pgfqpoint{34.197440in}{5.442565in}}%
\pgfpathlineto{\pgfqpoint{34.222832in}{5.484292in}}%
\pgfpathlineto{\pgfqpoint{34.246364in}{5.534473in}}%
\pgfpathlineto{\pgfqpoint{34.270997in}{5.460686in}}%
\pgfpathlineto{\pgfqpoint{34.293786in}{5.518952in}}%
\pgfpathlineto{\pgfqpoint{34.318035in}{5.678338in}}%
\pgfpathlineto{\pgfqpoint{34.340984in}{5.584654in}}%
\pgfpathlineto{\pgfqpoint{34.366319in}{5.504363in}}%
\pgfpathlineto{\pgfqpoint{34.388883in}{5.557546in}}%
\pgfpathlineto{\pgfqpoint{34.412956in}{5.442515in}}%
\pgfpathlineto{\pgfqpoint{34.437757in}{5.582839in}}%
\pgfpathlineto{\pgfqpoint{34.460943in}{5.834541in}}%
\pgfpathlineto{\pgfqpoint{34.484122in}{5.740043in}}%
\pgfpathlineto{\pgfqpoint{34.508593in}{5.601241in}}%
\pgfpathlineto{\pgfqpoint{34.531364in}{5.809148in}}%
\pgfpathlineto{\pgfqpoint{34.554213in}{5.745325in}}%
\pgfpathlineto{\pgfqpoint{34.578241in}{5.846668in}}%
\pgfpathlineto{\pgfqpoint{34.601042in}{5.862622in}}%
\pgfpathlineto{\pgfqpoint{34.623987in}{5.791091in}}%
\pgfpathlineto{\pgfqpoint{34.648006in}{5.744292in}}%
\pgfpathlineto{\pgfqpoint{34.671232in}{5.706485in}}%
\pgfpathlineto{\pgfqpoint{34.693969in}{5.908282in}}%
\pgfpathlineto{\pgfqpoint{34.717979in}{5.813609in}}%
\pgfpathlineto{\pgfqpoint{34.741399in}{5.854178in}}%
\pgfpathlineto{\pgfqpoint{34.763884in}{5.918658in}}%
\pgfpathlineto{\pgfqpoint{34.788312in}{5.815038in}}%
\pgfpathlineto{\pgfqpoint{34.810332in}{5.868196in}}%
\pgfpathlineto{\pgfqpoint{34.834146in}{5.857730in}}%
\pgfpathlineto{\pgfqpoint{34.856576in}{5.911984in}}%
\pgfpathlineto{\pgfqpoint{34.880395in}{5.893046in}}%
\pgfpathlineto{\pgfqpoint{34.902821in}{5.880249in}}%
\pgfpathlineto{\pgfqpoint{34.926805in}{5.899825in}}%
\pgfpathlineto{\pgfqpoint{34.948980in}{5.930845in}}%
\pgfpathlineto{\pgfqpoint{34.973284in}{5.726189in}}%
\pgfpathlineto{\pgfqpoint{34.996094in}{5.825752in}}%
\pgfpathlineto{\pgfqpoint{35.020184in}{5.865058in}}%
\pgfpathlineto{\pgfqpoint{35.047391in}{5.775449in}}%
\pgfpathlineto{\pgfqpoint{35.098205in}{5.758501in}}%
\pgfpathlineto{\pgfqpoint{35.149293in}{5.758501in}}%
\pgfpathlineto{\pgfqpoint{35.201170in}{5.758501in}}%
\pgfpathlineto{\pgfqpoint{35.254065in}{5.758501in}}%
\pgfpathlineto{\pgfqpoint{35.305776in}{5.758501in}}%
\pgfpathlineto{\pgfqpoint{35.357710in}{5.758501in}}%
\pgfpathlineto{\pgfqpoint{35.411910in}{5.758501in}}%
\pgfpathlineto{\pgfqpoint{35.464942in}{5.758501in}}%
\pgfpathlineto{\pgfqpoint{35.464942in}{5.758501in}}%
\pgfpathlineto{\pgfqpoint{35.464942in}{5.758501in}}%
\pgfpathlineto{\pgfqpoint{35.411910in}{5.758501in}}%
\pgfpathlineto{\pgfqpoint{35.357710in}{5.758501in}}%
\pgfpathlineto{\pgfqpoint{35.305776in}{5.758501in}}%
\pgfpathlineto{\pgfqpoint{35.254065in}{5.758501in}}%
\pgfpathlineto{\pgfqpoint{35.201170in}{5.758501in}}%
\pgfpathlineto{\pgfqpoint{35.149293in}{5.758501in}}%
\pgfpathlineto{\pgfqpoint{35.098205in}{5.758501in}}%
\pgfpathlineto{\pgfqpoint{35.047391in}{5.775449in}}%
\pgfpathlineto{\pgfqpoint{35.020184in}{5.865058in}}%
\pgfpathlineto{\pgfqpoint{34.996094in}{5.825752in}}%
\pgfpathlineto{\pgfqpoint{34.973284in}{5.726189in}}%
\pgfpathlineto{\pgfqpoint{34.948980in}{5.930845in}}%
\pgfpathlineto{\pgfqpoint{34.926805in}{5.899825in}}%
\pgfpathlineto{\pgfqpoint{34.902821in}{5.880249in}}%
\pgfpathlineto{\pgfqpoint{34.880395in}{5.893046in}}%
\pgfpathlineto{\pgfqpoint{34.856576in}{5.911984in}}%
\pgfpathlineto{\pgfqpoint{34.834146in}{5.857730in}}%
\pgfpathlineto{\pgfqpoint{34.810332in}{5.868196in}}%
\pgfpathlineto{\pgfqpoint{34.788312in}{5.815038in}}%
\pgfpathlineto{\pgfqpoint{34.763884in}{5.918658in}}%
\pgfpathlineto{\pgfqpoint{34.741399in}{5.854178in}}%
\pgfpathlineto{\pgfqpoint{34.717979in}{5.813609in}}%
\pgfpathlineto{\pgfqpoint{34.693969in}{5.908282in}}%
\pgfpathlineto{\pgfqpoint{34.671232in}{5.706485in}}%
\pgfpathlineto{\pgfqpoint{34.648006in}{5.744292in}}%
\pgfpathlineto{\pgfqpoint{34.623987in}{5.791091in}}%
\pgfpathlineto{\pgfqpoint{34.601042in}{5.862622in}}%
\pgfpathlineto{\pgfqpoint{34.578241in}{5.846668in}}%
\pgfpathlineto{\pgfqpoint{34.554213in}{5.745325in}}%
\pgfpathlineto{\pgfqpoint{34.531364in}{5.809148in}}%
\pgfpathlineto{\pgfqpoint{34.508593in}{5.601241in}}%
\pgfpathlineto{\pgfqpoint{34.484122in}{5.740043in}}%
\pgfpathlineto{\pgfqpoint{34.460943in}{5.834541in}}%
\pgfpathlineto{\pgfqpoint{34.437757in}{5.582839in}}%
\pgfpathlineto{\pgfqpoint{34.412956in}{5.442515in}}%
\pgfpathlineto{\pgfqpoint{34.388883in}{5.557546in}}%
\pgfpathlineto{\pgfqpoint{34.366319in}{5.504363in}}%
\pgfpathlineto{\pgfqpoint{34.340984in}{5.584654in}}%
\pgfpathlineto{\pgfqpoint{34.318035in}{5.678338in}}%
\pgfpathlineto{\pgfqpoint{34.293786in}{5.518952in}}%
\pgfpathlineto{\pgfqpoint{34.270997in}{5.460686in}}%
\pgfpathlineto{\pgfqpoint{34.246364in}{5.534473in}}%
\pgfpathlineto{\pgfqpoint{34.222832in}{5.484292in}}%
\pgfpathlineto{\pgfqpoint{34.197440in}{5.442565in}}%
\pgfpathlineto{\pgfqpoint{34.173862in}{5.446238in}}%
\pgfpathlineto{\pgfqpoint{34.150180in}{5.483715in}}%
\pgfpathlineto{\pgfqpoint{34.125436in}{5.364631in}}%
\pgfpathlineto{\pgfqpoint{34.101659in}{5.182737in}}%
\pgfpathlineto{\pgfqpoint{34.076513in}{5.235025in}}%
\pgfpathlineto{\pgfqpoint{34.052161in}{3.729137in}}%
\pgfpathlineto{\pgfqpoint{34.021637in}{2.864879in}}%
\pgfpathlineto{\pgfqpoint{33.989641in}{2.224563in}}%
\pgfpathlineto{\pgfqpoint{33.952831in}{1.906942in}}%
\pgfpathlineto{\pgfqpoint{33.912991in}{1.669618in}}%
\pgfpathlineto{\pgfqpoint{33.870742in}{1.454760in}}%
\pgfpathlineto{\pgfqpoint{33.832264in}{0.773588in}}%
\pgfpathlineto{\pgfqpoint{33.784450in}{0.773588in}}%
\pgfpathlineto{\pgfqpoint{33.746200in}{0.773588in}}%
\pgfpathlineto{\pgfqpoint{33.693680in}{0.773588in}}%
\pgfpathlineto{\pgfqpoint{33.642373in}{0.773588in}}%
\pgfpathlineto{\pgfqpoint{33.590505in}{0.773588in}}%
\pgfpathlineto{\pgfqpoint{33.536879in}{0.773588in}}%
\pgfpathlineto{\pgfqpoint{33.484775in}{0.773588in}}%
\pgfpathlineto{\pgfqpoint{33.433283in}{0.773588in}}%
\pgfpathlineto{\pgfqpoint{33.379482in}{0.773588in}}%
\pgfpathlineto{\pgfqpoint{33.326652in}{0.773588in}}%
\pgfpathlineto{\pgfqpoint{33.273785in}{0.773588in}}%
\pgfpathlineto{\pgfqpoint{33.219830in}{1.230236in}}%
\pgfpathlineto{\pgfqpoint{33.167699in}{1.423144in}}%
\pgfpathlineto{\pgfqpoint{33.115518in}{1.488284in}}%
\pgfpathlineto{\pgfqpoint{33.062733in}{1.414323in}}%
\pgfpathlineto{\pgfqpoint{33.010153in}{1.429618in}}%
\pgfpathlineto{\pgfqpoint{32.957868in}{1.459985in}}%
\pgfpathlineto{\pgfqpoint{32.903667in}{1.442369in}}%
\pgfpathlineto{\pgfqpoint{32.850576in}{1.393664in}}%
\pgfpathlineto{\pgfqpoint{32.797230in}{1.405600in}}%
\pgfpathlineto{\pgfqpoint{32.743153in}{1.453492in}}%
\pgfpathlineto{\pgfqpoint{32.691213in}{1.465799in}}%
\pgfpathlineto{\pgfqpoint{32.638735in}{1.466197in}}%
\pgfpathlineto{\pgfqpoint{32.584565in}{1.527806in}}%
\pgfpathlineto{\pgfqpoint{32.518604in}{1.577644in}}%
\pgfpathlineto{\pgfqpoint{32.429213in}{1.548059in}}%
\pgfpathlineto{\pgfqpoint{32.334387in}{1.559840in}}%
\pgfpathlineto{\pgfqpoint{32.244846in}{1.565071in}}%
\pgfpathlineto{\pgfqpoint{32.153206in}{1.569780in}}%
\pgfpathlineto{\pgfqpoint{32.062211in}{1.548891in}}%
\pgfpathlineto{\pgfqpoint{31.975639in}{1.606211in}}%
\pgfpathlineto{\pgfqpoint{31.890397in}{1.621243in}}%
\pgfpathlineto{\pgfqpoint{31.802724in}{1.559328in}}%
\pgfpathlineto{\pgfqpoint{31.715049in}{1.618107in}}%
\pgfpathlineto{\pgfqpoint{31.629527in}{1.642355in}}%
\pgfpathlineto{\pgfqpoint{31.543273in}{1.623152in}}%
\pgfpathlineto{\pgfqpoint{31.463397in}{1.664462in}}%
\pgfpathlineto{\pgfqpoint{31.385546in}{1.683708in}}%
\pgfpathlineto{\pgfqpoint{31.304546in}{1.621377in}}%
\pgfpathlineto{\pgfqpoint{31.225476in}{1.715736in}}%
\pgfpathlineto{\pgfqpoint{31.150292in}{1.713936in}}%
\pgfpathlineto{\pgfqpoint{31.073353in}{1.696625in}}%
\pgfpathlineto{\pgfqpoint{31.000738in}{1.786581in}}%
\pgfpathlineto{\pgfqpoint{30.928370in}{1.744580in}}%
\pgfpathlineto{\pgfqpoint{30.855684in}{1.770692in}}%
\pgfpathlineto{\pgfqpoint{30.786068in}{1.815928in}}%
\pgfpathlineto{\pgfqpoint{30.717993in}{1.799199in}}%
\pgfpathlineto{\pgfqpoint{30.648377in}{1.818030in}}%
\pgfpathlineto{\pgfqpoint{30.582999in}{1.881027in}}%
\pgfpathlineto{\pgfqpoint{30.519963in}{1.909921in}}%
\pgfpathlineto{\pgfqpoint{30.457369in}{1.933531in}}%
\pgfpathlineto{\pgfqpoint{30.397563in}{1.931511in}}%
\pgfpathlineto{\pgfqpoint{30.337041in}{1.985679in}}%
\pgfpathlineto{\pgfqpoint{30.276704in}{1.990714in}}%
\pgfpathlineto{\pgfqpoint{30.221783in}{2.094139in}}%
\pgfpathlineto{\pgfqpoint{30.169497in}{2.140896in}}%
\pgfpathlineto{\pgfqpoint{30.117082in}{2.164722in}}%
\pgfpathlineto{\pgfqpoint{30.066247in}{2.157383in}}%
\pgfpathlineto{\pgfqpoint{30.014968in}{2.192509in}}%
\pgfpathlineto{\pgfqpoint{29.962066in}{2.109977in}}%
\pgfpathlineto{\pgfqpoint{29.910558in}{2.163910in}}%
\pgfpathlineto{\pgfqpoint{29.858913in}{2.093735in}}%
\pgfpathlineto{\pgfqpoint{29.805208in}{2.142856in}}%
\pgfpathlineto{\pgfqpoint{29.753093in}{2.161564in}}%
\pgfpathlineto{\pgfqpoint{29.701409in}{2.123741in}}%
\pgfpathlineto{\pgfqpoint{29.647988in}{2.152791in}}%
\pgfpathlineto{\pgfqpoint{29.596078in}{2.080609in}}%
\pgfpathlineto{\pgfqpoint{29.543760in}{2.082360in}}%
\pgfpathlineto{\pgfqpoint{29.490075in}{2.134356in}}%
\pgfpathlineto{\pgfqpoint{29.438335in}{2.133563in}}%
\pgfpathlineto{\pgfqpoint{29.386173in}{2.150986in}}%
\pgfpathlineto{\pgfqpoint{29.332261in}{2.082785in}}%
\pgfpathlineto{\pgfqpoint{29.279399in}{2.100503in}}%
\pgfpathlineto{\pgfqpoint{29.226818in}{2.160460in}}%
\pgfpathlineto{\pgfqpoint{29.173700in}{2.077484in}}%
\pgfpathlineto{\pgfqpoint{29.121029in}{2.088575in}}%
\pgfpathlineto{\pgfqpoint{29.068863in}{2.138193in}}%
\pgfpathlineto{\pgfqpoint{29.015651in}{2.178709in}}%
\pgfpathlineto{\pgfqpoint{28.965063in}{2.168914in}}%
\pgfpathlineto{\pgfqpoint{28.913538in}{2.139636in}}%
\pgfpathlineto{\pgfqpoint{28.860347in}{2.116407in}}%
\pgfpathlineto{\pgfqpoint{28.808587in}{2.154364in}}%
\pgfpathlineto{\pgfqpoint{28.758074in}{2.179648in}}%
\pgfpathlineto{\pgfqpoint{28.706300in}{2.197239in}}%
\pgfpathlineto{\pgfqpoint{28.656185in}{2.169046in}}%
\pgfpathlineto{\pgfqpoint{28.605351in}{2.184676in}}%
\pgfpathlineto{\pgfqpoint{28.553749in}{2.135069in}}%
\pgfpathlineto{\pgfqpoint{28.502707in}{2.176910in}}%
\pgfpathlineto{\pgfqpoint{28.451952in}{2.085182in}}%
\pgfpathlineto{\pgfqpoint{28.399138in}{2.108488in}}%
\pgfpathlineto{\pgfqpoint{28.347853in}{2.128908in}}%
\pgfpathlineto{\pgfqpoint{28.297462in}{2.221346in}}%
\pgfpathlineto{\pgfqpoint{28.245320in}{2.152372in}}%
\pgfpathlineto{\pgfqpoint{28.195034in}{2.194715in}}%
\pgfpathlineto{\pgfqpoint{28.143944in}{2.103896in}}%
\pgfpathlineto{\pgfqpoint{28.090650in}{2.154372in}}%
\pgfpathlineto{\pgfqpoint{28.039571in}{2.133339in}}%
\pgfpathlineto{\pgfqpoint{27.988562in}{2.148509in}}%
\pgfpathlineto{\pgfqpoint{27.935127in}{2.142057in}}%
\pgfpathlineto{\pgfqpoint{27.883786in}{2.171648in}}%
\pgfpathlineto{\pgfqpoint{27.833067in}{2.102078in}}%
\pgfpathlineto{\pgfqpoint{27.779331in}{2.127843in}}%
\pgfpathlineto{\pgfqpoint{27.728571in}{2.169913in}}%
\pgfpathlineto{\pgfqpoint{27.678294in}{2.202638in}}%
\pgfpathlineto{\pgfqpoint{27.626352in}{2.188092in}}%
\pgfpathlineto{\pgfqpoint{27.576178in}{2.193467in}}%
\pgfpathlineto{\pgfqpoint{27.525849in}{2.059972in}}%
\pgfpathlineto{\pgfqpoint{27.473173in}{2.149431in}}%
\pgfpathlineto{\pgfqpoint{27.423066in}{2.151542in}}%
\pgfpathlineto{\pgfqpoint{27.372863in}{2.262046in}}%
\pgfpathlineto{\pgfqpoint{27.321160in}{2.084956in}}%
\pgfpathlineto{\pgfqpoint{27.269679in}{2.127160in}}%
\pgfpathlineto{\pgfqpoint{27.218587in}{2.125482in}}%
\pgfpathlineto{\pgfqpoint{27.165274in}{2.106034in}}%
\pgfpathlineto{\pgfqpoint{27.113950in}{2.191289in}}%
\pgfpathlineto{\pgfqpoint{27.063062in}{2.127542in}}%
\pgfpathlineto{\pgfqpoint{27.010539in}{2.206184in}}%
\pgfpathlineto{\pgfqpoint{26.960748in}{2.126317in}}%
\pgfpathlineto{\pgfqpoint{26.909530in}{2.167554in}}%
\pgfpathlineto{\pgfqpoint{26.857706in}{2.142711in}}%
\pgfpathlineto{\pgfqpoint{26.807173in}{2.169425in}}%
\pgfpathlineto{\pgfqpoint{26.756448in}{2.167275in}}%
\pgfpathlineto{\pgfqpoint{26.704176in}{2.127344in}}%
\pgfpathlineto{\pgfqpoint{26.653161in}{2.155346in}}%
\pgfpathlineto{\pgfqpoint{26.601868in}{2.125000in}}%
\pgfpathlineto{\pgfqpoint{26.549446in}{2.196254in}}%
\pgfpathlineto{\pgfqpoint{26.499064in}{2.073278in}}%
\pgfpathlineto{\pgfqpoint{26.447655in}{2.147600in}}%
\pgfpathlineto{\pgfqpoint{26.395829in}{2.216585in}}%
\pgfpathlineto{\pgfqpoint{26.346588in}{2.210846in}}%
\pgfpathlineto{\pgfqpoint{26.297518in}{2.140127in}}%
\pgfpathlineto{\pgfqpoint{26.245717in}{2.188680in}}%
\pgfpathlineto{\pgfqpoint{26.195263in}{2.138438in}}%
\pgfpathlineto{\pgfqpoint{26.144837in}{2.202460in}}%
\pgfpathlineto{\pgfqpoint{26.093895in}{2.172470in}}%
\pgfpathlineto{\pgfqpoint{26.044216in}{2.172740in}}%
\pgfpathlineto{\pgfqpoint{25.994130in}{2.161647in}}%
\pgfpathlineto{\pgfqpoint{25.942005in}{2.189113in}}%
\pgfpathlineto{\pgfqpoint{25.891991in}{2.129676in}}%
\pgfpathlineto{\pgfqpoint{25.841969in}{2.202252in}}%
\pgfpathlineto{\pgfqpoint{25.790287in}{2.177297in}}%
\pgfpathlineto{\pgfqpoint{25.740073in}{2.113429in}}%
\pgfpathlineto{\pgfqpoint{25.689152in}{2.130566in}}%
\pgfpathlineto{\pgfqpoint{25.637513in}{2.153196in}}%
\pgfpathlineto{\pgfqpoint{25.586387in}{2.087238in}}%
\pgfpathlineto{\pgfqpoint{25.535558in}{2.107499in}}%
\pgfpathlineto{\pgfqpoint{25.483381in}{2.187707in}}%
\pgfpathlineto{\pgfqpoint{25.433954in}{2.179510in}}%
\pgfpathlineto{\pgfqpoint{25.383622in}{2.199798in}}%
\pgfpathlineto{\pgfqpoint{25.332608in}{2.124537in}}%
\pgfpathlineto{\pgfqpoint{25.282222in}{2.174410in}}%
\pgfpathlineto{\pgfqpoint{25.232944in}{2.198963in}}%
\pgfpathlineto{\pgfqpoint{25.181554in}{2.148863in}}%
\pgfpathlineto{\pgfqpoint{25.131043in}{2.169034in}}%
\pgfpathlineto{\pgfqpoint{25.080539in}{2.156690in}}%
\pgfpathlineto{\pgfqpoint{25.028630in}{2.166090in}}%
\pgfpathlineto{\pgfqpoint{24.978211in}{2.200234in}}%
\pgfpathlineto{\pgfqpoint{24.928869in}{2.240633in}}%
\pgfpathlineto{\pgfqpoint{24.878143in}{2.139836in}}%
\pgfpathlineto{\pgfqpoint{24.828140in}{2.234647in}}%
\pgfpathlineto{\pgfqpoint{24.778304in}{2.187482in}}%
\pgfpathlineto{\pgfqpoint{24.726499in}{2.192402in}}%
\pgfpathlineto{\pgfqpoint{24.676389in}{2.191465in}}%
\pgfpathlineto{\pgfqpoint{24.626556in}{2.072454in}}%
\pgfpathlineto{\pgfqpoint{24.573998in}{2.204872in}}%
\pgfpathlineto{\pgfqpoint{24.524258in}{2.209164in}}%
\pgfpathlineto{\pgfqpoint{24.474381in}{2.139169in}}%
\pgfpathlineto{\pgfqpoint{24.422475in}{2.160039in}}%
\pgfpathlineto{\pgfqpoint{24.371890in}{2.208071in}}%
\pgfpathlineto{\pgfqpoint{24.321907in}{2.211605in}}%
\pgfpathlineto{\pgfqpoint{24.271277in}{2.149127in}}%
\pgfpathlineto{\pgfqpoint{24.221010in}{2.157432in}}%
\pgfpathlineto{\pgfqpoint{24.170873in}{2.170742in}}%
\pgfpathlineto{\pgfqpoint{24.118637in}{2.144171in}}%
\pgfpathlineto{\pgfqpoint{24.066933in}{2.162367in}}%
\pgfpathlineto{\pgfqpoint{24.016442in}{2.201493in}}%
\pgfpathlineto{\pgfqpoint{23.964514in}{2.116542in}}%
\pgfpathlineto{\pgfqpoint{23.913088in}{2.119330in}}%
\pgfpathlineto{\pgfqpoint{23.862673in}{2.175857in}}%
\pgfpathlineto{\pgfqpoint{23.811512in}{2.259241in}}%
\pgfpathlineto{\pgfqpoint{23.762329in}{2.182126in}}%
\pgfpathlineto{\pgfqpoint{23.713074in}{2.221155in}}%
\pgfpathlineto{\pgfqpoint{23.661575in}{2.159418in}}%
\pgfpathlineto{\pgfqpoint{23.611923in}{2.282936in}}%
\pgfpathlineto{\pgfqpoint{23.563429in}{2.143388in}}%
\pgfpathlineto{\pgfqpoint{23.511867in}{2.137407in}}%
\pgfpathlineto{\pgfqpoint{23.462017in}{2.157637in}}%
\pgfpathlineto{\pgfqpoint{23.411909in}{2.198735in}}%
\pgfpathlineto{\pgfqpoint{23.361310in}{2.170140in}}%
\pgfpathlineto{\pgfqpoint{23.310918in}{2.148300in}}%
\pgfpathlineto{\pgfqpoint{23.260726in}{2.152878in}}%
\pgfpathlineto{\pgfqpoint{23.208694in}{2.124981in}}%
\pgfpathlineto{\pgfqpoint{23.157756in}{2.136082in}}%
\pgfpathlineto{\pgfqpoint{23.107134in}{2.079790in}}%
\pgfpathlineto{\pgfqpoint{23.055187in}{2.235002in}}%
\pgfpathlineto{\pgfqpoint{23.006022in}{2.146954in}}%
\pgfpathlineto{\pgfqpoint{22.956359in}{2.173476in}}%
\pgfpathlineto{\pgfqpoint{22.904236in}{2.104244in}}%
\pgfpathlineto{\pgfqpoint{22.853966in}{2.158422in}}%
\pgfpathlineto{\pgfqpoint{22.803829in}{2.168087in}}%
\pgfpathlineto{\pgfqpoint{22.751700in}{2.248709in}}%
\pgfpathlineto{\pgfqpoint{22.703398in}{2.305817in}}%
\pgfpathlineto{\pgfqpoint{22.654822in}{2.180208in}}%
\pgfpathlineto{\pgfqpoint{22.603657in}{2.179999in}}%
\pgfpathlineto{\pgfqpoint{22.554073in}{2.136690in}}%
\pgfpathlineto{\pgfqpoint{22.503552in}{2.135991in}}%
\pgfpathlineto{\pgfqpoint{22.451976in}{2.206816in}}%
\pgfpathlineto{\pgfqpoint{22.402084in}{2.133288in}}%
\pgfpathlineto{\pgfqpoint{22.352827in}{2.157190in}}%
\pgfpathlineto{\pgfqpoint{22.300933in}{2.094855in}}%
\pgfpathlineto{\pgfqpoint{22.250634in}{2.185196in}}%
\pgfpathlineto{\pgfqpoint{22.200587in}{2.138280in}}%
\pgfpathlineto{\pgfqpoint{22.148817in}{2.251963in}}%
\pgfpathlineto{\pgfqpoint{22.099907in}{2.258881in}}%
\pgfpathlineto{\pgfqpoint{22.050354in}{2.153076in}}%
\pgfpathlineto{\pgfqpoint{21.999803in}{2.276898in}}%
\pgfpathlineto{\pgfqpoint{21.950544in}{2.161953in}}%
\pgfpathlineto{\pgfqpoint{21.900834in}{2.128498in}}%
\pgfpathlineto{\pgfqpoint{21.849580in}{2.230011in}}%
\pgfpathlineto{\pgfqpoint{21.800474in}{2.226448in}}%
\pgfpathlineto{\pgfqpoint{21.751624in}{2.227472in}}%
\pgfpathlineto{\pgfqpoint{21.700864in}{2.213775in}}%
\pgfpathlineto{\pgfqpoint{21.652276in}{2.218841in}}%
\pgfpathlineto{\pgfqpoint{21.602538in}{2.160895in}}%
\pgfpathlineto{\pgfqpoint{21.550957in}{2.167679in}}%
\pgfpathlineto{\pgfqpoint{21.501023in}{2.150548in}}%
\pgfpathlineto{\pgfqpoint{21.450830in}{2.168605in}}%
\pgfpathlineto{\pgfqpoint{21.399733in}{2.175744in}}%
\pgfpathlineto{\pgfqpoint{21.350353in}{2.208270in}}%
\pgfpathlineto{\pgfqpoint{21.300955in}{2.191708in}}%
\pgfpathlineto{\pgfqpoint{21.250655in}{2.245041in}}%
\pgfpathlineto{\pgfqpoint{21.201683in}{2.181293in}}%
\pgfpathlineto{\pgfqpoint{21.152719in}{2.201026in}}%
\pgfpathlineto{\pgfqpoint{21.102370in}{2.241324in}}%
\pgfpathlineto{\pgfqpoint{21.053515in}{2.155531in}}%
\pgfpathlineto{\pgfqpoint{21.003693in}{2.185814in}}%
\pgfpathlineto{\pgfqpoint{20.953585in}{2.212485in}}%
\pgfpathlineto{\pgfqpoint{20.904164in}{2.166381in}}%
\pgfpathlineto{\pgfqpoint{20.854118in}{2.134033in}}%
\pgfpathlineto{\pgfqpoint{20.802965in}{2.283576in}}%
\pgfpathlineto{\pgfqpoint{20.753558in}{2.212760in}}%
\pgfpathlineto{\pgfqpoint{20.704616in}{2.268763in}}%
\pgfpathlineto{\pgfqpoint{20.654668in}{2.192399in}}%
\pgfpathlineto{\pgfqpoint{20.605567in}{2.126089in}}%
\pgfpathlineto{\pgfqpoint{20.556290in}{2.221965in}}%
\pgfpathlineto{\pgfqpoint{20.505208in}{2.258629in}}%
\pgfpathlineto{\pgfqpoint{20.455604in}{2.155524in}}%
\pgfpathlineto{\pgfqpoint{20.406414in}{2.188387in}}%
\pgfpathlineto{\pgfqpoint{20.355462in}{2.199386in}}%
\pgfpathlineto{\pgfqpoint{20.305924in}{2.134136in}}%
\pgfpathlineto{\pgfqpoint{20.255593in}{2.107988in}}%
\pgfpathlineto{\pgfqpoint{20.203615in}{2.192230in}}%
\pgfpathlineto{\pgfqpoint{20.153679in}{2.162851in}}%
\pgfpathlineto{\pgfqpoint{20.104350in}{2.161230in}}%
\pgfpathlineto{\pgfqpoint{20.053312in}{2.243716in}}%
\pgfpathlineto{\pgfqpoint{20.003692in}{2.175230in}}%
\pgfpathlineto{\pgfqpoint{19.955252in}{2.296405in}}%
\pgfpathlineto{\pgfqpoint{19.906615in}{2.281059in}}%
\pgfpathlineto{\pgfqpoint{19.858933in}{2.236420in}}%
\pgfpathlineto{\pgfqpoint{19.810712in}{2.223149in}}%
\pgfpathlineto{\pgfqpoint{19.761678in}{2.193047in}}%
\pgfpathlineto{\pgfqpoint{19.714265in}{2.275859in}}%
\pgfpathlineto{\pgfqpoint{19.666940in}{2.243279in}}%
\pgfpathlineto{\pgfqpoint{19.617707in}{2.182953in}}%
\pgfpathlineto{\pgfqpoint{19.569747in}{2.259987in}}%
\pgfpathlineto{\pgfqpoint{19.522292in}{2.234019in}}%
\pgfpathlineto{\pgfqpoint{19.473154in}{2.190874in}}%
\pgfpathlineto{\pgfqpoint{19.425178in}{2.227549in}}%
\pgfpathlineto{\pgfqpoint{19.376890in}{2.239043in}}%
\pgfpathlineto{\pgfqpoint{19.327829in}{2.241679in}}%
\pgfpathlineto{\pgfqpoint{19.279742in}{2.293533in}}%
\pgfpathlineto{\pgfqpoint{19.231826in}{2.158168in}}%
\pgfpathlineto{\pgfqpoint{19.181302in}{2.175947in}}%
\pgfpathlineto{\pgfqpoint{19.132724in}{2.240721in}}%
\pgfpathlineto{\pgfqpoint{19.084207in}{2.100303in}}%
\pgfpathlineto{\pgfqpoint{19.033794in}{2.192082in}}%
\pgfpathlineto{\pgfqpoint{18.985002in}{2.256048in}}%
\pgfpathlineto{\pgfqpoint{18.937671in}{2.243100in}}%
\pgfpathlineto{\pgfqpoint{18.888188in}{2.195093in}}%
\pgfpathlineto{\pgfqpoint{18.839312in}{2.255904in}}%
\pgfpathlineto{\pgfqpoint{18.791212in}{2.175746in}}%
\pgfpathlineto{\pgfqpoint{18.741665in}{2.223418in}}%
\pgfpathlineto{\pgfqpoint{18.693122in}{2.186383in}}%
\pgfpathlineto{\pgfqpoint{18.644639in}{2.233839in}}%
\pgfpathlineto{\pgfqpoint{18.595724in}{2.207052in}}%
\pgfpathlineto{\pgfqpoint{18.548460in}{2.240166in}}%
\pgfpathlineto{\pgfqpoint{18.500915in}{2.249891in}}%
\pgfpathlineto{\pgfqpoint{18.452284in}{2.172965in}}%
\pgfpathlineto{\pgfqpoint{18.405242in}{2.249085in}}%
\pgfpathlineto{\pgfqpoint{18.357538in}{2.210510in}}%
\pgfpathlineto{\pgfqpoint{18.308441in}{2.227451in}}%
\pgfpathlineto{\pgfqpoint{18.260871in}{2.246190in}}%
\pgfpathlineto{\pgfqpoint{18.213333in}{2.208562in}}%
\pgfpathlineto{\pgfqpoint{18.164405in}{2.303508in}}%
\pgfpathlineto{\pgfqpoint{18.117307in}{2.230183in}}%
\pgfpathlineto{\pgfqpoint{18.069524in}{2.220034in}}%
\pgfpathlineto{\pgfqpoint{18.020026in}{2.099061in}}%
\pgfpathlineto{\pgfqpoint{17.970910in}{2.245278in}}%
\pgfpathlineto{\pgfqpoint{17.922885in}{2.295885in}}%
\pgfpathlineto{\pgfqpoint{17.874600in}{2.265803in}}%
\pgfpathlineto{\pgfqpoint{17.826809in}{2.181075in}}%
\pgfpathlineto{\pgfqpoint{17.779014in}{2.180861in}}%
\pgfpathlineto{\pgfqpoint{17.729727in}{2.181121in}}%
\pgfpathlineto{\pgfqpoint{17.681914in}{2.358516in}}%
\pgfpathlineto{\pgfqpoint{17.634653in}{2.295602in}}%
\pgfpathlineto{\pgfqpoint{17.585742in}{2.379567in}}%
\pgfpathlineto{\pgfqpoint{17.538409in}{2.219811in}}%
\pgfpathlineto{\pgfqpoint{17.491125in}{2.288470in}}%
\pgfpathlineto{\pgfqpoint{17.442362in}{2.234768in}}%
\pgfpathlineto{\pgfqpoint{17.394714in}{2.248043in}}%
\pgfpathlineto{\pgfqpoint{17.347576in}{2.273193in}}%
\pgfpathlineto{\pgfqpoint{17.298941in}{2.177400in}}%
\pgfpathlineto{\pgfqpoint{17.250860in}{2.264188in}}%
\pgfpathlineto{\pgfqpoint{17.203674in}{2.246982in}}%
\pgfpathlineto{\pgfqpoint{17.154507in}{2.297791in}}%
\pgfpathlineto{\pgfqpoint{17.107470in}{2.233795in}}%
\pgfpathlineto{\pgfqpoint{17.059518in}{2.170799in}}%
\pgfpathlineto{\pgfqpoint{17.010585in}{2.251846in}}%
\pgfpathlineto{\pgfqpoint{16.963533in}{2.263073in}}%
\pgfpathlineto{\pgfqpoint{16.916302in}{2.332512in}}%
\pgfpathlineto{\pgfqpoint{16.868266in}{2.244193in}}%
\pgfpathlineto{\pgfqpoint{16.821038in}{2.222802in}}%
\pgfpathlineto{\pgfqpoint{16.773713in}{2.227898in}}%
\pgfpathlineto{\pgfqpoint{16.724568in}{2.247475in}}%
\pgfpathlineto{\pgfqpoint{16.677566in}{2.244778in}}%
\pgfpathlineto{\pgfqpoint{16.630280in}{2.266736in}}%
\pgfpathlineto{\pgfqpoint{16.581135in}{2.238151in}}%
\pgfpathlineto{\pgfqpoint{16.533619in}{2.302678in}}%
\pgfpathlineto{\pgfqpoint{16.486560in}{2.250525in}}%
\pgfpathlineto{\pgfqpoint{16.436622in}{2.137013in}}%
\pgfpathlineto{\pgfqpoint{16.387666in}{2.174071in}}%
\pgfpathlineto{\pgfqpoint{16.339537in}{2.259149in}}%
\pgfpathlineto{\pgfqpoint{16.289817in}{2.219318in}}%
\pgfpathlineto{\pgfqpoint{16.242078in}{2.276775in}}%
\pgfpathlineto{\pgfqpoint{16.195175in}{2.205221in}}%
\pgfpathlineto{\pgfqpoint{16.146338in}{2.216517in}}%
\pgfpathlineto{\pgfqpoint{16.098697in}{2.270580in}}%
\pgfpathlineto{\pgfqpoint{16.051820in}{2.332418in}}%
\pgfpathlineto{\pgfqpoint{16.003926in}{2.336191in}}%
\pgfpathlineto{\pgfqpoint{15.957799in}{2.402601in}}%
\pgfpathlineto{\pgfqpoint{15.911811in}{2.228192in}}%
\pgfpathlineto{\pgfqpoint{15.863592in}{2.332872in}}%
\pgfpathlineto{\pgfqpoint{15.817273in}{2.287207in}}%
\pgfpathlineto{\pgfqpoint{15.770154in}{2.313157in}}%
\pgfpathlineto{\pgfqpoint{15.722814in}{2.300535in}}%
\pgfpathlineto{\pgfqpoint{15.676019in}{2.301145in}}%
\pgfpathlineto{\pgfqpoint{15.628660in}{2.211679in}}%
\pgfpathlineto{\pgfqpoint{15.579815in}{2.258501in}}%
\pgfpathlineto{\pgfqpoint{15.532329in}{2.219452in}}%
\pgfpathlineto{\pgfqpoint{15.484858in}{2.218369in}}%
\pgfpathlineto{\pgfqpoint{15.436052in}{2.260363in}}%
\pgfpathlineto{\pgfqpoint{15.389896in}{2.293893in}}%
\pgfpathlineto{\pgfqpoint{15.343883in}{2.307810in}}%
\pgfpathlineto{\pgfqpoint{15.295866in}{2.280354in}}%
\pgfpathlineto{\pgfqpoint{15.248863in}{2.236209in}}%
\pgfpathlineto{\pgfqpoint{15.201456in}{2.240614in}}%
\pgfpathlineto{\pgfqpoint{15.152658in}{2.295575in}}%
\pgfpathlineto{\pgfqpoint{15.105478in}{2.314598in}}%
\pgfpathlineto{\pgfqpoint{15.058774in}{2.198546in}}%
\pgfpathlineto{\pgfqpoint{15.010214in}{2.267582in}}%
\pgfpathlineto{\pgfqpoint{14.963167in}{2.252252in}}%
\pgfpathlineto{\pgfqpoint{14.916239in}{2.196403in}}%
\pgfpathlineto{\pgfqpoint{14.868242in}{2.282858in}}%
\pgfpathlineto{\pgfqpoint{14.821474in}{2.359201in}}%
\pgfpathlineto{\pgfqpoint{14.774752in}{2.259349in}}%
\pgfpathlineto{\pgfqpoint{14.726757in}{2.231589in}}%
\pgfpathlineto{\pgfqpoint{14.680226in}{2.325095in}}%
\pgfpathlineto{\pgfqpoint{14.633992in}{2.221244in}}%
\pgfpathlineto{\pgfqpoint{14.585477in}{2.246168in}}%
\pgfpathlineto{\pgfqpoint{14.538630in}{2.291021in}}%
\pgfpathlineto{\pgfqpoint{14.491726in}{2.254543in}}%
\pgfpathlineto{\pgfqpoint{14.443729in}{2.346640in}}%
\pgfpathlineto{\pgfqpoint{14.397126in}{2.209372in}}%
\pgfpathlineto{\pgfqpoint{14.350224in}{2.276334in}}%
\pgfpathlineto{\pgfqpoint{14.302418in}{2.274703in}}%
\pgfpathlineto{\pgfqpoint{14.255593in}{2.321238in}}%
\pgfpathlineto{\pgfqpoint{14.209376in}{2.193278in}}%
\pgfpathlineto{\pgfqpoint{14.160147in}{2.182080in}}%
\pgfpathlineto{\pgfqpoint{14.112147in}{2.229533in}}%
\pgfpathlineto{\pgfqpoint{14.064951in}{2.232270in}}%
\pgfpathlineto{\pgfqpoint{14.016344in}{2.286673in}}%
\pgfpathlineto{\pgfqpoint{13.969353in}{2.229832in}}%
\pgfpathlineto{\pgfqpoint{13.922606in}{2.226374in}}%
\pgfpathlineto{\pgfqpoint{13.873610in}{2.212443in}}%
\pgfpathlineto{\pgfqpoint{13.825297in}{2.178671in}}%
\pgfpathlineto{\pgfqpoint{13.777441in}{2.242575in}}%
\pgfpathlineto{\pgfqpoint{13.729143in}{2.300755in}}%
\pgfpathlineto{\pgfqpoint{13.682151in}{2.232128in}}%
\pgfpathlineto{\pgfqpoint{13.635346in}{2.278245in}}%
\pgfpathlineto{\pgfqpoint{13.587249in}{2.335101in}}%
\pgfpathlineto{\pgfqpoint{13.541646in}{2.277760in}}%
\pgfpathlineto{\pgfqpoint{13.495727in}{2.299059in}}%
\pgfpathlineto{\pgfqpoint{13.448069in}{2.240374in}}%
\pgfpathlineto{\pgfqpoint{13.401859in}{2.323804in}}%
\pgfpathlineto{\pgfqpoint{13.356518in}{2.385490in}}%
\pgfpathlineto{\pgfqpoint{13.308974in}{2.312159in}}%
\pgfpathlineto{\pgfqpoint{13.262980in}{2.338157in}}%
\pgfpathlineto{\pgfqpoint{13.216719in}{2.235117in}}%
\pgfpathlineto{\pgfqpoint{13.169075in}{2.384526in}}%
\pgfpathlineto{\pgfqpoint{13.123339in}{2.262921in}}%
\pgfpathlineto{\pgfqpoint{13.077275in}{2.323533in}}%
\pgfpathlineto{\pgfqpoint{13.030499in}{2.298564in}}%
\pgfpathlineto{\pgfqpoint{12.983596in}{2.304244in}}%
\pgfpathlineto{\pgfqpoint{12.936799in}{2.386807in}}%
\pgfpathlineto{\pgfqpoint{12.889988in}{2.289046in}}%
\pgfpathlineto{\pgfqpoint{12.843155in}{2.234339in}}%
\pgfpathlineto{\pgfqpoint{12.796907in}{2.259754in}}%
\pgfpathlineto{\pgfqpoint{12.749025in}{2.253182in}}%
\pgfpathlineto{\pgfqpoint{12.701835in}{2.279136in}}%
\pgfpathlineto{\pgfqpoint{12.655424in}{2.325183in}}%
\pgfpathlineto{\pgfqpoint{12.607615in}{2.239852in}}%
\pgfpathlineto{\pgfqpoint{12.560711in}{2.203625in}}%
\pgfpathlineto{\pgfqpoint{12.514481in}{2.340897in}}%
\pgfpathlineto{\pgfqpoint{12.467453in}{2.296669in}}%
\pgfpathlineto{\pgfqpoint{12.421372in}{2.358343in}}%
\pgfpathlineto{\pgfqpoint{12.375261in}{2.349502in}}%
\pgfpathlineto{\pgfqpoint{12.328136in}{2.257529in}}%
\pgfpathlineto{\pgfqpoint{12.282271in}{2.273976in}}%
\pgfpathlineto{\pgfqpoint{12.235985in}{2.221952in}}%
\pgfpathlineto{\pgfqpoint{12.187925in}{2.280826in}}%
\pgfpathlineto{\pgfqpoint{12.141887in}{2.351371in}}%
\pgfpathlineto{\pgfqpoint{12.096623in}{2.317891in}}%
\pgfpathlineto{\pgfqpoint{12.049869in}{2.301880in}}%
\pgfpathlineto{\pgfqpoint{12.004414in}{2.324057in}}%
\pgfpathlineto{\pgfqpoint{11.959143in}{2.350595in}}%
\pgfpathlineto{\pgfqpoint{11.911501in}{2.203921in}}%
\pgfpathlineto{\pgfqpoint{11.864862in}{2.355287in}}%
\pgfpathlineto{\pgfqpoint{11.819406in}{2.285077in}}%
\pgfpathlineto{\pgfqpoint{11.772894in}{2.306760in}}%
\pgfpathlineto{\pgfqpoint{11.727607in}{2.333441in}}%
\pgfpathlineto{\pgfqpoint{11.682053in}{2.359958in}}%
\pgfpathlineto{\pgfqpoint{11.634589in}{2.229229in}}%
\pgfpathlineto{\pgfqpoint{11.588799in}{2.230089in}}%
\pgfpathlineto{\pgfqpoint{11.542665in}{2.357277in}}%
\pgfpathlineto{\pgfqpoint{11.494874in}{2.291080in}}%
\pgfpathlineto{\pgfqpoint{11.448531in}{2.259828in}}%
\pgfpathlineto{\pgfqpoint{11.402044in}{2.332692in}}%
\pgfpathlineto{\pgfqpoint{11.353756in}{2.235310in}}%
\pgfpathlineto{\pgfqpoint{11.307361in}{2.360674in}}%
\pgfpathlineto{\pgfqpoint{11.261350in}{2.276617in}}%
\pgfpathlineto{\pgfqpoint{11.213355in}{2.260071in}}%
\pgfpathlineto{\pgfqpoint{11.167471in}{2.399865in}}%
\pgfpathlineto{\pgfqpoint{11.121978in}{2.258422in}}%
\pgfpathlineto{\pgfqpoint{11.074633in}{2.335174in}}%
\pgfpathlineto{\pgfqpoint{11.029398in}{2.356543in}}%
\pgfpathlineto{\pgfqpoint{10.984145in}{2.282506in}}%
\pgfpathlineto{\pgfqpoint{10.937381in}{2.301035in}}%
\pgfpathlineto{\pgfqpoint{10.891801in}{2.284564in}}%
\pgfpathlineto{\pgfqpoint{10.845790in}{2.275306in}}%
\pgfpathlineto{\pgfqpoint{10.798768in}{2.302467in}}%
\pgfpathlineto{\pgfqpoint{10.753168in}{2.330547in}}%
\pgfpathlineto{\pgfqpoint{10.707461in}{2.250711in}}%
\pgfpathlineto{\pgfqpoint{10.660112in}{2.222199in}}%
\pgfpathlineto{\pgfqpoint{10.613886in}{2.264712in}}%
\pgfpathlineto{\pgfqpoint{10.568154in}{2.346468in}}%
\pgfpathlineto{\pgfqpoint{10.521105in}{2.287380in}}%
\pgfpathlineto{\pgfqpoint{10.475642in}{2.274485in}}%
\pgfpathlineto{\pgfqpoint{10.429362in}{2.294777in}}%
\pgfpathlineto{\pgfqpoint{10.381570in}{2.293680in}}%
\pgfpathlineto{\pgfqpoint{10.335334in}{2.240312in}}%
\pgfpathlineto{\pgfqpoint{10.289084in}{2.291417in}}%
\pgfpathlineto{\pgfqpoint{10.242601in}{2.276235in}}%
\pgfpathlineto{\pgfqpoint{10.197247in}{2.359779in}}%
\pgfpathlineto{\pgfqpoint{10.151348in}{2.230054in}}%
\pgfpathlineto{\pgfqpoint{10.103098in}{2.261364in}}%
\pgfpathlineto{\pgfqpoint{10.057613in}{2.305435in}}%
\pgfpathlineto{\pgfqpoint{10.011915in}{2.268469in}}%
\pgfpathlineto{\pgfqpoint{9.964562in}{2.297187in}}%
\pgfpathlineto{\pgfqpoint{9.918842in}{2.261142in}}%
\pgfpathlineto{\pgfqpoint{9.873173in}{2.325186in}}%
\pgfpathlineto{\pgfqpoint{9.826599in}{2.328474in}}%
\pgfpathlineto{\pgfqpoint{9.781321in}{2.303450in}}%
\pgfpathlineto{\pgfqpoint{9.735785in}{2.323431in}}%
\pgfpathlineto{\pgfqpoint{9.689497in}{2.353123in}}%
\pgfpathlineto{\pgfqpoint{9.644121in}{2.320379in}}%
\pgfpathlineto{\pgfqpoint{9.598338in}{2.314462in}}%
\pgfpathlineto{\pgfqpoint{9.552197in}{2.394099in}}%
\pgfpathlineto{\pgfqpoint{9.507583in}{2.292822in}}%
\pgfpathlineto{\pgfqpoint{9.462006in}{2.300804in}}%
\pgfpathlineto{\pgfqpoint{9.415178in}{2.314648in}}%
\pgfpathlineto{\pgfqpoint{9.369639in}{2.319684in}}%
\pgfpathlineto{\pgfqpoint{9.324423in}{2.399221in}}%
\pgfpathlineto{\pgfqpoint{9.278439in}{2.335560in}}%
\pgfpathlineto{\pgfqpoint{9.233107in}{2.417828in}}%
\pgfpathlineto{\pgfqpoint{9.188332in}{2.368490in}}%
\pgfpathlineto{\pgfqpoint{9.141270in}{2.295788in}}%
\pgfpathlineto{\pgfqpoint{9.095727in}{2.335942in}}%
\pgfpathlineto{\pgfqpoint{9.050165in}{2.367234in}}%
\pgfpathlineto{\pgfqpoint{9.003947in}{2.257547in}}%
\pgfpathlineto{\pgfqpoint{8.957924in}{2.269362in}}%
\pgfpathlineto{\pgfqpoint{8.912131in}{2.375131in}}%
\pgfpathlineto{\pgfqpoint{8.866242in}{2.269668in}}%
\pgfpathlineto{\pgfqpoint{8.821068in}{2.264133in}}%
\pgfpathlineto{\pgfqpoint{8.774902in}{2.285186in}}%
\pgfpathlineto{\pgfqpoint{8.727404in}{2.294828in}}%
\pgfpathlineto{\pgfqpoint{8.681080in}{2.191512in}}%
\pgfpathlineto{\pgfqpoint{8.635315in}{2.356202in}}%
\pgfpathlineto{\pgfqpoint{8.589124in}{2.336609in}}%
\pgfpathlineto{\pgfqpoint{8.543976in}{2.376578in}}%
\pgfpathlineto{\pgfqpoint{8.499298in}{2.313127in}}%
\pgfpathlineto{\pgfqpoint{8.453780in}{2.398000in}}%
\pgfpathlineto{\pgfqpoint{8.409767in}{2.318752in}}%
\pgfpathlineto{\pgfqpoint{8.364636in}{2.398401in}}%
\pgfpathlineto{\pgfqpoint{8.318789in}{2.273525in}}%
\pgfpathlineto{\pgfqpoint{8.273006in}{2.303852in}}%
\pgfpathlineto{\pgfqpoint{8.227930in}{2.349197in}}%
\pgfpathlineto{\pgfqpoint{8.181791in}{2.272682in}}%
\pgfpathlineto{\pgfqpoint{8.136842in}{2.343950in}}%
\pgfpathlineto{\pgfqpoint{8.091881in}{2.338666in}}%
\pgfpathlineto{\pgfqpoint{8.045278in}{2.365018in}}%
\pgfpathlineto{\pgfqpoint{8.000573in}{2.240229in}}%
\pgfpathlineto{\pgfqpoint{7.955879in}{2.368915in}}%
\pgfpathlineto{\pgfqpoint{7.910161in}{2.351596in}}%
\pgfpathlineto{\pgfqpoint{7.865263in}{2.303547in}}%
\pgfpathlineto{\pgfqpoint{7.819947in}{2.321312in}}%
\pgfpathlineto{\pgfqpoint{7.773226in}{2.358435in}}%
\pgfpathlineto{\pgfqpoint{7.728803in}{2.304006in}}%
\pgfpathlineto{\pgfqpoint{7.682978in}{2.305719in}}%
\pgfpathlineto{\pgfqpoint{7.636592in}{2.356484in}}%
\pgfpathlineto{\pgfqpoint{7.591890in}{2.291838in}}%
\pgfpathlineto{\pgfqpoint{7.546892in}{2.354507in}}%
\pgfpathlineto{\pgfqpoint{7.500683in}{2.330493in}}%
\pgfpathlineto{\pgfqpoint{7.455548in}{2.349659in}}%
\pgfpathlineto{\pgfqpoint{7.409977in}{2.350045in}}%
\pgfpathlineto{\pgfqpoint{7.363749in}{2.355753in}}%
\pgfpathlineto{\pgfqpoint{7.318484in}{2.317625in}}%
\pgfpathlineto{\pgfqpoint{7.273914in}{2.402816in}}%
\pgfpathlineto{\pgfqpoint{7.228120in}{2.333442in}}%
\pgfpathlineto{\pgfqpoint{7.184277in}{2.450307in}}%
\pgfpathlineto{\pgfqpoint{7.140134in}{2.383905in}}%
\pgfpathlineto{\pgfqpoint{7.094205in}{2.405300in}}%
\pgfpathlineto{\pgfqpoint{7.050071in}{2.351361in}}%
\pgfpathlineto{\pgfqpoint{7.005149in}{2.366754in}}%
\pgfpathlineto{\pgfqpoint{6.958763in}{2.393311in}}%
\pgfpathlineto{\pgfqpoint{6.914194in}{2.320511in}}%
\pgfpathlineto{\pgfqpoint{6.869544in}{2.314775in}}%
\pgfpathlineto{\pgfqpoint{6.824012in}{2.284579in}}%
\pgfpathlineto{\pgfqpoint{6.779295in}{2.367685in}}%
\pgfpathlineto{\pgfqpoint{6.734887in}{2.277162in}}%
\pgfpathlineto{\pgfqpoint{6.688504in}{2.375017in}}%
\pgfpathlineto{\pgfqpoint{6.643960in}{2.333142in}}%
\pgfpathlineto{\pgfqpoint{6.599302in}{2.357634in}}%
\pgfpathlineto{\pgfqpoint{6.553117in}{2.281819in}}%
\pgfpathlineto{\pgfqpoint{6.507651in}{2.240583in}}%
\pgfpathlineto{\pgfqpoint{6.461324in}{2.270987in}}%
\pgfpathlineto{\pgfqpoint{6.413399in}{2.233471in}}%
\pgfpathlineto{\pgfqpoint{6.367508in}{2.243112in}}%
\pgfpathlineto{\pgfqpoint{6.321070in}{2.254446in}}%
\pgfpathlineto{\pgfqpoint{6.273584in}{2.254185in}}%
\pgfpathlineto{\pgfqpoint{6.227112in}{2.266056in}}%
\pgfpathlineto{\pgfqpoint{6.180802in}{2.237468in}}%
\pgfpathlineto{\pgfqpoint{6.133235in}{2.311575in}}%
\pgfpathlineto{\pgfqpoint{6.087229in}{2.250652in}}%
\pgfpathlineto{\pgfqpoint{6.041630in}{2.406366in}}%
\pgfpathlineto{\pgfqpoint{5.994686in}{2.234958in}}%
\pgfpathlineto{\pgfqpoint{5.948991in}{2.278529in}}%
\pgfpathlineto{\pgfqpoint{5.903366in}{2.308873in}}%
\pgfpathlineto{\pgfqpoint{5.856181in}{2.261696in}}%
\pgfpathlineto{\pgfqpoint{5.810269in}{2.381697in}}%
\pgfpathlineto{\pgfqpoint{5.765813in}{2.387220in}}%
\pgfpathlineto{\pgfqpoint{5.719569in}{2.319036in}}%
\pgfpathlineto{\pgfqpoint{5.674262in}{2.353354in}}%
\pgfpathlineto{\pgfqpoint{5.628857in}{2.247028in}}%
\pgfpathlineto{\pgfqpoint{5.581803in}{2.273113in}}%
\pgfpathlineto{\pgfqpoint{5.536203in}{2.406439in}}%
\pgfpathlineto{\pgfqpoint{5.491128in}{2.369110in}}%
\pgfpathlineto{\pgfqpoint{5.444526in}{2.326998in}}%
\pgfpathlineto{\pgfqpoint{5.399334in}{2.339578in}}%
\pgfpathlineto{\pgfqpoint{5.353614in}{2.297753in}}%
\pgfpathlineto{\pgfqpoint{5.307244in}{2.374804in}}%
\pgfpathlineto{\pgfqpoint{5.262146in}{2.304795in}}%
\pgfpathlineto{\pgfqpoint{5.216291in}{2.333026in}}%
\pgfpathlineto{\pgfqpoint{5.169192in}{2.320747in}}%
\pgfpathlineto{\pgfqpoint{5.124036in}{2.232536in}}%
\pgfpathlineto{\pgfqpoint{5.078074in}{2.306542in}}%
\pgfpathlineto{\pgfqpoint{5.030678in}{2.295788in}}%
\pgfpathlineto{\pgfqpoint{4.984973in}{2.320591in}}%
\pgfpathlineto{\pgfqpoint{4.939032in}{2.223770in}}%
\pgfpathlineto{\pgfqpoint{4.891298in}{2.270967in}}%
\pgfpathlineto{\pgfqpoint{4.845429in}{2.353522in}}%
\pgfpathlineto{\pgfqpoint{4.799887in}{2.393133in}}%
\pgfpathlineto{\pgfqpoint{4.753449in}{2.340546in}}%
\pgfpathlineto{\pgfqpoint{4.708228in}{2.367505in}}%
\pgfpathlineto{\pgfqpoint{4.663622in}{2.377694in}}%
\pgfpathlineto{\pgfqpoint{4.617918in}{2.337717in}}%
\pgfpathlineto{\pgfqpoint{4.572708in}{2.277965in}}%
\pgfpathlineto{\pgfqpoint{4.527424in}{2.257669in}}%
\pgfpathlineto{\pgfqpoint{4.480684in}{2.264512in}}%
\pgfpathlineto{\pgfqpoint{4.435284in}{2.402740in}}%
\pgfpathlineto{\pgfqpoint{4.390187in}{2.309445in}}%
\pgfpathlineto{\pgfqpoint{4.343118in}{2.288248in}}%
\pgfpathlineto{\pgfqpoint{4.297354in}{2.359011in}}%
\pgfpathlineto{\pgfqpoint{4.252389in}{2.272653in}}%
\pgfpathlineto{\pgfqpoint{4.204543in}{2.291067in}}%
\pgfpathlineto{\pgfqpoint{4.159078in}{2.360627in}}%
\pgfpathlineto{\pgfqpoint{4.114362in}{2.348997in}}%
\pgfpathlineto{\pgfqpoint{4.067507in}{2.326455in}}%
\pgfpathlineto{\pgfqpoint{4.022115in}{2.311396in}}%
\pgfpathlineto{\pgfqpoint{3.976868in}{2.367974in}}%
\pgfpathlineto{\pgfqpoint{3.930307in}{2.272708in}}%
\pgfpathlineto{\pgfqpoint{3.884614in}{2.376055in}}%
\pgfpathlineto{\pgfqpoint{3.839775in}{2.328600in}}%
\pgfpathlineto{\pgfqpoint{3.793040in}{2.272352in}}%
\pgfpathlineto{\pgfqpoint{3.747249in}{2.281275in}}%
\pgfpathlineto{\pgfqpoint{3.701949in}{2.294335in}}%
\pgfpathlineto{\pgfqpoint{3.654909in}{2.311073in}}%
\pgfpathlineto{\pgfqpoint{3.609556in}{2.340637in}}%
\pgfpathlineto{\pgfqpoint{3.564489in}{2.322218in}}%
\pgfpathlineto{\pgfqpoint{3.517815in}{2.390510in}}%
\pgfpathlineto{\pgfqpoint{3.473135in}{2.322586in}}%
\pgfpathlineto{\pgfqpoint{3.428526in}{2.309584in}}%
\pgfpathlineto{\pgfqpoint{3.381274in}{2.354180in}}%
\pgfpathlineto{\pgfqpoint{3.336184in}{2.314984in}}%
\pgfpathlineto{\pgfqpoint{3.290748in}{2.395267in}}%
\pgfpathlineto{\pgfqpoint{3.245458in}{2.358697in}}%
\pgfpathlineto{\pgfqpoint{3.200519in}{2.371658in}}%
\pgfpathlineto{\pgfqpoint{3.155580in}{2.240058in}}%
\pgfpathlineto{\pgfqpoint{3.108180in}{2.359459in}}%
\pgfpathlineto{\pgfqpoint{3.062761in}{2.352393in}}%
\pgfpathlineto{\pgfqpoint{3.017486in}{2.404176in}}%
\pgfpathlineto{\pgfqpoint{2.971917in}{2.365948in}}%
\pgfpathlineto{\pgfqpoint{2.927413in}{2.375345in}}%
\pgfpathlineto{\pgfqpoint{2.883134in}{2.342913in}}%
\pgfpathlineto{\pgfqpoint{2.836381in}{2.274534in}}%
\pgfpathlineto{\pgfqpoint{2.790736in}{2.292757in}}%
\pgfpathlineto{\pgfqpoint{2.745668in}{2.306144in}}%
\pgfpathlineto{\pgfqpoint{2.698034in}{2.323786in}}%
\pgfpathlineto{\pgfqpoint{2.651003in}{2.276180in}}%
\pgfpathlineto{\pgfqpoint{2.604306in}{2.305652in}}%
\pgfpathlineto{\pgfqpoint{2.555498in}{2.171250in}}%
\pgfpathlineto{\pgfqpoint{2.505534in}{2.112084in}}%
\pgfpathlineto{\pgfqpoint{2.452591in}{2.024750in}}%
\pgfpathlineto{\pgfqpoint{2.397147in}{2.195111in}}%
\pgfpathlineto{\pgfqpoint{2.348431in}{2.204576in}}%
\pgfpathlineto{\pgfqpoint{2.299591in}{2.228159in}}%
\pgfpathlineto{\pgfqpoint{2.249804in}{2.194731in}}%
\pgfpathlineto{\pgfqpoint{2.201242in}{2.266072in}}%
\pgfpathlineto{\pgfqpoint{2.153284in}{2.229339in}}%
\pgfpathlineto{\pgfqpoint{2.104325in}{2.219948in}}%
\pgfpathlineto{\pgfqpoint{2.057441in}{2.280610in}}%
\pgfpathlineto{\pgfqpoint{2.011209in}{2.287585in}}%
\pgfpathlineto{\pgfqpoint{1.963476in}{2.242602in}}%
\pgfpathlineto{\pgfqpoint{1.915908in}{2.177911in}}%
\pgfpathlineto{\pgfqpoint{1.869493in}{2.304360in}}%
\pgfpathlineto{\pgfqpoint{1.822804in}{2.275353in}}%
\pgfpathlineto{\pgfqpoint{1.778099in}{2.423736in}}%
\pgfpathlineto{\pgfqpoint{1.734500in}{2.982598in}}%
\pgfpathlineto{\pgfqpoint{1.689326in}{3.160374in}}%
\pgfpathlineto{\pgfqpoint{1.645274in}{3.145313in}}%
\pgfpathlineto{\pgfqpoint{1.600816in}{3.168955in}}%
\pgfpathlineto{\pgfqpoint{1.555718in}{3.135647in}}%
\pgfpathlineto{\pgfqpoint{1.511740in}{3.196053in}}%
\pgfpathlineto{\pgfqpoint{1.468334in}{3.240135in}}%
\pgfpathlineto{\pgfqpoint{1.422957in}{3.110191in}}%
\pgfpathlineto{\pgfqpoint{1.378287in}{3.216226in}}%
\pgfpathlineto{\pgfqpoint{1.334262in}{3.167259in}}%
\pgfpathlineto{\pgfqpoint{1.289074in}{3.144981in}}%
\pgfpathlineto{\pgfqpoint{1.244609in}{3.092116in}}%
\pgfpathlineto{\pgfqpoint{1.200192in}{3.142196in}}%
\pgfpathlineto{\pgfqpoint{1.155171in}{3.232716in}}%
\pgfpathlineto{\pgfqpoint{1.111790in}{3.128685in}}%
\pgfpathlineto{\pgfqpoint{1.067773in}{3.002463in}}%
\pgfpathlineto{\pgfqpoint{1.021908in}{3.123971in}}%
\pgfpathlineto{\pgfqpoint{0.978015in}{3.149746in}}%
\pgfpathlineto{\pgfqpoint{0.933783in}{3.104546in}}%
\pgfpathlineto{\pgfqpoint{0.887244in}{3.007158in}}%
\pgfpathlineto{\pgfqpoint{0.842612in}{3.092765in}}%
\pgfpathlineto{\pgfqpoint{0.797895in}{2.142129in}}%
\pgfpathclose%
\pgfusepath{fill}%
\end{pgfscope}%
\begin{pgfscope}%
\pgfsetbuttcap%
\pgfsetroundjoin%
\definecolor{currentfill}{rgb}{0.000000,0.000000,0.000000}%
\pgfsetfillcolor{currentfill}%
\pgfsetlinewidth{0.803000pt}%
\definecolor{currentstroke}{rgb}{0.000000,0.000000,0.000000}%
\pgfsetstrokecolor{currentstroke}%
\pgfsetdash{}{0pt}%
\pgfsys@defobject{currentmarker}{\pgfqpoint{0.000000in}{-0.048611in}}{\pgfqpoint{0.000000in}{0.000000in}}{%
\pgfpathmoveto{\pgfqpoint{0.000000in}{0.000000in}}%
\pgfpathlineto{\pgfqpoint{0.000000in}{-0.048611in}}%
\pgfusepath{stroke,fill}%
}%
\begin{pgfscope}%
\pgfsys@transformshift{0.781402in}{0.773588in}%
\pgfsys@useobject{currentmarker}{}%
\end{pgfscope}%
\end{pgfscope}%
\begin{pgfscope}%
\definecolor{textcolor}{rgb}{0.000000,0.000000,0.000000}%
\pgfsetstrokecolor{textcolor}%
\pgfsetfillcolor{textcolor}%
\pgftext[x=0.781402in,y=0.676366in,,top]{\color{textcolor}\rmfamily\fontsize{10.000000}{12.000000}\selectfont \(\displaystyle {0}\)}%
\end{pgfscope}%
\begin{pgfscope}%
\pgfsetbuttcap%
\pgfsetroundjoin%
\definecolor{currentfill}{rgb}{0.000000,0.000000,0.000000}%
\pgfsetfillcolor{currentfill}%
\pgfsetlinewidth{0.803000pt}%
\definecolor{currentstroke}{rgb}{0.000000,0.000000,0.000000}%
\pgfsetstrokecolor{currentstroke}%
\pgfsetdash{}{0pt}%
\pgfsys@defobject{currentmarker}{\pgfqpoint{0.000000in}{-0.048611in}}{\pgfqpoint{0.000000in}{0.000000in}}{%
\pgfpathmoveto{\pgfqpoint{0.000000in}{0.000000in}}%
\pgfpathlineto{\pgfqpoint{0.000000in}{-0.048611in}}%
\pgfusepath{stroke,fill}%
}%
\begin{pgfscope}%
\pgfsys@transformshift{1.262909in}{0.773588in}%
\pgfsys@useobject{currentmarker}{}%
\end{pgfscope}%
\end{pgfscope}%
\begin{pgfscope}%
\definecolor{textcolor}{rgb}{0.000000,0.000000,0.000000}%
\pgfsetstrokecolor{textcolor}%
\pgfsetfillcolor{textcolor}%
\pgftext[x=1.262909in,y=0.676366in,,top]{\color{textcolor}\rmfamily\fontsize{10.000000}{12.000000}\selectfont \(\displaystyle {50}\)}%
\end{pgfscope}%
\begin{pgfscope}%
\pgfsetbuttcap%
\pgfsetroundjoin%
\definecolor{currentfill}{rgb}{0.000000,0.000000,0.000000}%
\pgfsetfillcolor{currentfill}%
\pgfsetlinewidth{0.803000pt}%
\definecolor{currentstroke}{rgb}{0.000000,0.000000,0.000000}%
\pgfsetstrokecolor{currentstroke}%
\pgfsetdash{}{0pt}%
\pgfsys@defobject{currentmarker}{\pgfqpoint{0.000000in}{-0.048611in}}{\pgfqpoint{0.000000in}{0.000000in}}{%
\pgfpathmoveto{\pgfqpoint{0.000000in}{0.000000in}}%
\pgfpathlineto{\pgfqpoint{0.000000in}{-0.048611in}}%
\pgfusepath{stroke,fill}%
}%
\begin{pgfscope}%
\pgfsys@transformshift{1.744416in}{0.773588in}%
\pgfsys@useobject{currentmarker}{}%
\end{pgfscope}%
\end{pgfscope}%
\begin{pgfscope}%
\definecolor{textcolor}{rgb}{0.000000,0.000000,0.000000}%
\pgfsetstrokecolor{textcolor}%
\pgfsetfillcolor{textcolor}%
\pgftext[x=1.744416in,y=0.676366in,,top]{\color{textcolor}\rmfamily\fontsize{10.000000}{12.000000}\selectfont \(\displaystyle {100}\)}%
\end{pgfscope}%
\begin{pgfscope}%
\pgfsetbuttcap%
\pgfsetroundjoin%
\definecolor{currentfill}{rgb}{0.000000,0.000000,0.000000}%
\pgfsetfillcolor{currentfill}%
\pgfsetlinewidth{0.803000pt}%
\definecolor{currentstroke}{rgb}{0.000000,0.000000,0.000000}%
\pgfsetstrokecolor{currentstroke}%
\pgfsetdash{}{0pt}%
\pgfsys@defobject{currentmarker}{\pgfqpoint{-0.048611in}{0.000000in}}{\pgfqpoint{-0.000000in}{0.000000in}}{%
\pgfpathmoveto{\pgfqpoint{-0.000000in}{0.000000in}}%
\pgfpathlineto{\pgfqpoint{-0.048611in}{0.000000in}}%
\pgfusepath{stroke,fill}%
}%
\begin{pgfscope}%
\pgfsys@transformshift{0.781402in}{0.773588in}%
\pgfsys@useobject{currentmarker}{}%
\end{pgfscope}%
\end{pgfscope}%
\begin{pgfscope}%
\definecolor{textcolor}{rgb}{0.000000,0.000000,0.000000}%
\pgfsetstrokecolor{textcolor}%
\pgfsetfillcolor{textcolor}%
\pgftext[x=0.506711in, y=0.725363in, left, base]{\color{textcolor}\rmfamily\fontsize{10.000000}{12.000000}\selectfont \(\displaystyle {0.0}\)}%
\end{pgfscope}%
\begin{pgfscope}%
\pgfsetbuttcap%
\pgfsetroundjoin%
\definecolor{currentfill}{rgb}{0.000000,0.000000,0.000000}%
\pgfsetfillcolor{currentfill}%
\pgfsetlinewidth{0.803000pt}%
\definecolor{currentstroke}{rgb}{0.000000,0.000000,0.000000}%
\pgfsetstrokecolor{currentstroke}%
\pgfsetdash{}{0pt}%
\pgfsys@defobject{currentmarker}{\pgfqpoint{-0.048611in}{0.000000in}}{\pgfqpoint{-0.000000in}{0.000000in}}{%
\pgfpathmoveto{\pgfqpoint{-0.000000in}{0.000000in}}%
\pgfpathlineto{\pgfqpoint{-0.048611in}{0.000000in}}%
\pgfusepath{stroke,fill}%
}%
\begin{pgfscope}%
\pgfsys@transformshift{0.781402in}{1.434252in}%
\pgfsys@useobject{currentmarker}{}%
\end{pgfscope}%
\end{pgfscope}%
\begin{pgfscope}%
\definecolor{textcolor}{rgb}{0.000000,0.000000,0.000000}%
\pgfsetstrokecolor{textcolor}%
\pgfsetfillcolor{textcolor}%
\pgftext[x=0.506711in, y=1.386027in, left, base]{\color{textcolor}\rmfamily\fontsize{10.000000}{12.000000}\selectfont \(\displaystyle {0.1}\)}%
\end{pgfscope}%
\begin{pgfscope}%
\pgfsetbuttcap%
\pgfsetroundjoin%
\definecolor{currentfill}{rgb}{0.000000,0.000000,0.000000}%
\pgfsetfillcolor{currentfill}%
\pgfsetlinewidth{0.803000pt}%
\definecolor{currentstroke}{rgb}{0.000000,0.000000,0.000000}%
\pgfsetstrokecolor{currentstroke}%
\pgfsetdash{}{0pt}%
\pgfsys@defobject{currentmarker}{\pgfqpoint{-0.048611in}{0.000000in}}{\pgfqpoint{-0.000000in}{0.000000in}}{%
\pgfpathmoveto{\pgfqpoint{-0.000000in}{0.000000in}}%
\pgfpathlineto{\pgfqpoint{-0.048611in}{0.000000in}}%
\pgfusepath{stroke,fill}%
}%
\begin{pgfscope}%
\pgfsys@transformshift{0.781402in}{2.094916in}%
\pgfsys@useobject{currentmarker}{}%
\end{pgfscope}%
\end{pgfscope}%
\begin{pgfscope}%
\definecolor{textcolor}{rgb}{0.000000,0.000000,0.000000}%
\pgfsetstrokecolor{textcolor}%
\pgfsetfillcolor{textcolor}%
\pgftext[x=0.506711in, y=2.046691in, left, base]{\color{textcolor}\rmfamily\fontsize{10.000000}{12.000000}\selectfont \(\displaystyle {0.2}\)}%
\end{pgfscope}%
\begin{pgfscope}%
\pgfsetbuttcap%
\pgfsetroundjoin%
\definecolor{currentfill}{rgb}{0.000000,0.000000,0.000000}%
\pgfsetfillcolor{currentfill}%
\pgfsetlinewidth{0.803000pt}%
\definecolor{currentstroke}{rgb}{0.000000,0.000000,0.000000}%
\pgfsetstrokecolor{currentstroke}%
\pgfsetdash{}{0pt}%
\pgfsys@defobject{currentmarker}{\pgfqpoint{-0.048611in}{0.000000in}}{\pgfqpoint{-0.000000in}{0.000000in}}{%
\pgfpathmoveto{\pgfqpoint{-0.000000in}{0.000000in}}%
\pgfpathlineto{\pgfqpoint{-0.048611in}{0.000000in}}%
\pgfusepath{stroke,fill}%
}%
\begin{pgfscope}%
\pgfsys@transformshift{0.781402in}{2.755580in}%
\pgfsys@useobject{currentmarker}{}%
\end{pgfscope}%
\end{pgfscope}%
\begin{pgfscope}%
\definecolor{textcolor}{rgb}{0.000000,0.000000,0.000000}%
\pgfsetstrokecolor{textcolor}%
\pgfsetfillcolor{textcolor}%
\pgftext[x=0.506711in, y=2.707355in, left, base]{\color{textcolor}\rmfamily\fontsize{10.000000}{12.000000}\selectfont \(\displaystyle {0.3}\)}%
\end{pgfscope}%
\begin{pgfscope}%
\pgfsetbuttcap%
\pgfsetroundjoin%
\definecolor{currentfill}{rgb}{0.000000,0.000000,0.000000}%
\pgfsetfillcolor{currentfill}%
\pgfsetlinewidth{0.803000pt}%
\definecolor{currentstroke}{rgb}{0.000000,0.000000,0.000000}%
\pgfsetstrokecolor{currentstroke}%
\pgfsetdash{}{0pt}%
\pgfsys@defobject{currentmarker}{\pgfqpoint{-0.048611in}{0.000000in}}{\pgfqpoint{-0.000000in}{0.000000in}}{%
\pgfpathmoveto{\pgfqpoint{-0.000000in}{0.000000in}}%
\pgfpathlineto{\pgfqpoint{-0.048611in}{0.000000in}}%
\pgfusepath{stroke,fill}%
}%
\begin{pgfscope}%
\pgfsys@transformshift{0.781402in}{3.416244in}%
\pgfsys@useobject{currentmarker}{}%
\end{pgfscope}%
\end{pgfscope}%
\begin{pgfscope}%
\definecolor{textcolor}{rgb}{0.000000,0.000000,0.000000}%
\pgfsetstrokecolor{textcolor}%
\pgfsetfillcolor{textcolor}%
\pgftext[x=0.506711in, y=3.368019in, left, base]{\color{textcolor}\rmfamily\fontsize{10.000000}{12.000000}\selectfont \(\displaystyle {0.4}\)}%
\end{pgfscope}%
\begin{pgfscope}%
\pgfsetbuttcap%
\pgfsetroundjoin%
\definecolor{currentfill}{rgb}{0.000000,0.000000,0.000000}%
\pgfsetfillcolor{currentfill}%
\pgfsetlinewidth{0.803000pt}%
\definecolor{currentstroke}{rgb}{0.000000,0.000000,0.000000}%
\pgfsetstrokecolor{currentstroke}%
\pgfsetdash{}{0pt}%
\pgfsys@defobject{currentmarker}{\pgfqpoint{-0.048611in}{0.000000in}}{\pgfqpoint{-0.000000in}{0.000000in}}{%
\pgfpathmoveto{\pgfqpoint{-0.000000in}{0.000000in}}%
\pgfpathlineto{\pgfqpoint{-0.048611in}{0.000000in}}%
\pgfusepath{stroke,fill}%
}%
\begin{pgfscope}%
\pgfsys@transformshift{0.781402in}{4.076908in}%
\pgfsys@useobject{currentmarker}{}%
\end{pgfscope}%
\end{pgfscope}%
\begin{pgfscope}%
\definecolor{textcolor}{rgb}{0.000000,0.000000,0.000000}%
\pgfsetstrokecolor{textcolor}%
\pgfsetfillcolor{textcolor}%
\pgftext[x=0.506711in, y=4.028683in, left, base]{\color{textcolor}\rmfamily\fontsize{10.000000}{12.000000}\selectfont \(\displaystyle {0.5}\)}%
\end{pgfscope}%
\begin{pgfscope}%
\pgfsetbuttcap%
\pgfsetroundjoin%
\definecolor{currentfill}{rgb}{0.000000,0.000000,0.000000}%
\pgfsetfillcolor{currentfill}%
\pgfsetlinewidth{0.803000pt}%
\definecolor{currentstroke}{rgb}{0.000000,0.000000,0.000000}%
\pgfsetstrokecolor{currentstroke}%
\pgfsetdash{}{0pt}%
\pgfsys@defobject{currentmarker}{\pgfqpoint{-0.048611in}{0.000000in}}{\pgfqpoint{-0.000000in}{0.000000in}}{%
\pgfpathmoveto{\pgfqpoint{-0.000000in}{0.000000in}}%
\pgfpathlineto{\pgfqpoint{-0.048611in}{0.000000in}}%
\pgfusepath{stroke,fill}%
}%
\begin{pgfscope}%
\pgfsys@transformshift{0.781402in}{4.737572in}%
\pgfsys@useobject{currentmarker}{}%
\end{pgfscope}%
\end{pgfscope}%
\begin{pgfscope}%
\definecolor{textcolor}{rgb}{0.000000,0.000000,0.000000}%
\pgfsetstrokecolor{textcolor}%
\pgfsetfillcolor{textcolor}%
\pgftext[x=0.506711in, y=4.689347in, left, base]{\color{textcolor}\rmfamily\fontsize{10.000000}{12.000000}\selectfont \(\displaystyle {0.6}\)}%
\end{pgfscope}%
\begin{pgfscope}%
\pgfsetbuttcap%
\pgfsetroundjoin%
\definecolor{currentfill}{rgb}{0.000000,0.000000,0.000000}%
\pgfsetfillcolor{currentfill}%
\pgfsetlinewidth{0.803000pt}%
\definecolor{currentstroke}{rgb}{0.000000,0.000000,0.000000}%
\pgfsetstrokecolor{currentstroke}%
\pgfsetdash{}{0pt}%
\pgfsys@defobject{currentmarker}{\pgfqpoint{-0.048611in}{0.000000in}}{\pgfqpoint{-0.000000in}{0.000000in}}{%
\pgfpathmoveto{\pgfqpoint{-0.000000in}{0.000000in}}%
\pgfpathlineto{\pgfqpoint{-0.048611in}{0.000000in}}%
\pgfusepath{stroke,fill}%
}%
\begin{pgfscope}%
\pgfsys@transformshift{0.781402in}{5.398236in}%
\pgfsys@useobject{currentmarker}{}%
\end{pgfscope}%
\end{pgfscope}%
\begin{pgfscope}%
\definecolor{textcolor}{rgb}{0.000000,0.000000,0.000000}%
\pgfsetstrokecolor{textcolor}%
\pgfsetfillcolor{textcolor}%
\pgftext[x=0.506711in, y=5.350010in, left, base]{\color{textcolor}\rmfamily\fontsize{10.000000}{12.000000}\selectfont \(\displaystyle {0.7}\)}%
\end{pgfscope}%
\begin{pgfscope}%
\pgfsetbuttcap%
\pgfsetroundjoin%
\definecolor{currentfill}{rgb}{0.000000,0.000000,0.000000}%
\pgfsetfillcolor{currentfill}%
\pgfsetlinewidth{0.803000pt}%
\definecolor{currentstroke}{rgb}{0.000000,0.000000,0.000000}%
\pgfsetstrokecolor{currentstroke}%
\pgfsetdash{}{0pt}%
\pgfsys@defobject{currentmarker}{\pgfqpoint{-0.048611in}{0.000000in}}{\pgfqpoint{-0.000000in}{0.000000in}}{%
\pgfpathmoveto{\pgfqpoint{-0.000000in}{0.000000in}}%
\pgfpathlineto{\pgfqpoint{-0.048611in}{0.000000in}}%
\pgfusepath{stroke,fill}%
}%
\begin{pgfscope}%
\pgfsys@transformshift{0.781402in}{6.058900in}%
\pgfsys@useobject{currentmarker}{}%
\end{pgfscope}%
\end{pgfscope}%
\begin{pgfscope}%
\definecolor{textcolor}{rgb}{0.000000,0.000000,0.000000}%
\pgfsetstrokecolor{textcolor}%
\pgfsetfillcolor{textcolor}%
\pgftext[x=0.506711in, y=6.010674in, left, base]{\color{textcolor}\rmfamily\fontsize{10.000000}{12.000000}\selectfont \(\displaystyle {0.8}\)}%
\end{pgfscope}%
\begin{pgfscope}%
\pgfsetrectcap%
\pgfsetroundjoin%
\pgfsetlinewidth{0.803000pt}%
\definecolor{currentstroke}{rgb}{0.000000,0.000000,0.000000}%
\pgfsetstrokecolor{currentstroke}%
\pgfsetdash{}{0pt}%
\pgfpathmoveto{\pgfqpoint{2.155342in}{0.707284in}}%
\pgfpathlineto{\pgfqpoint{2.287951in}{0.839893in}}%
\pgfusepath{stroke}%
\end{pgfscope}%
\begin{pgfscope}%
\pgfpathrectangle{\pgfqpoint{0.781402in}{0.773588in}}{\pgfqpoint{1.440244in}{5.415119in}}%
\pgfusepath{clip}%
\pgfsetrectcap%
\pgfsetroundjoin%
\pgfsetlinewidth{1.505625pt}%
\definecolor{currentstroke}{rgb}{0.000000,0.000000,1.000000}%
\pgfsetstrokecolor{currentstroke}%
\pgfsetdash{}{0pt}%
\pgfpathmoveto{\pgfqpoint{1.740140in}{0.773588in}}%
\pgfpathlineto{\pgfqpoint{1.740140in}{6.188708in}}%
\pgfusepath{stroke}%
\end{pgfscope}%
\begin{pgfscope}%
\pgfpathrectangle{\pgfqpoint{0.781402in}{0.773588in}}{\pgfqpoint{1.440244in}{5.415119in}}%
\pgfusepath{clip}%
\pgfsetrectcap%
\pgfsetroundjoin%
\pgfsetlinewidth{1.505625pt}%
\definecolor{currentstroke}{rgb}{0.750000,0.750000,0.000000}%
\pgfsetstrokecolor{currentstroke}%
\pgfsetdash{}{0pt}%
\pgfusepath{stroke}%
\end{pgfscope}%
\begin{pgfscope}%
\pgfpathrectangle{\pgfqpoint{0.781402in}{0.773588in}}{\pgfqpoint{1.440244in}{5.415119in}}%
\pgfusepath{clip}%
\pgfsetrectcap%
\pgfsetroundjoin%
\pgfsetlinewidth{1.505625pt}%
\definecolor{currentstroke}{rgb}{0.750000,0.000000,0.750000}%
\pgfsetstrokecolor{currentstroke}%
\pgfsetdash{}{0pt}%
\pgfusepath{stroke}%
\end{pgfscope}%
\begin{pgfscope}%
\pgfpathrectangle{\pgfqpoint{0.781402in}{0.773588in}}{\pgfqpoint{1.440244in}{5.415119in}}%
\pgfusepath{clip}%
\pgfsetrectcap%
\pgfsetroundjoin%
\pgfsetlinewidth{1.505625pt}%
\definecolor{currentstroke}{rgb}{1.000000,0.000000,0.000000}%
\pgfsetstrokecolor{currentstroke}%
\pgfsetdash{}{0pt}%
\pgfusepath{stroke}%
\end{pgfscope}%
\begin{pgfscope}%
\pgfpathrectangle{\pgfqpoint{0.781402in}{0.773588in}}{\pgfqpoint{1.440244in}{5.415119in}}%
\pgfusepath{clip}%
\pgfsetrectcap%
\pgfsetroundjoin%
\pgfsetlinewidth{1.505625pt}%
\definecolor{currentstroke}{rgb}{0.000000,0.500000,0.000000}%
\pgfsetstrokecolor{currentstroke}%
\pgfsetdash{}{0pt}%
\pgfusepath{stroke}%
\end{pgfscope}%
\begin{pgfscope}%
\pgfpathrectangle{\pgfqpoint{0.781402in}{0.773588in}}{\pgfqpoint{1.440244in}{5.415119in}}%
\pgfusepath{clip}%
\pgfsetrectcap%
\pgfsetroundjoin%
\pgfsetlinewidth{1.505625pt}%
\definecolor{currentstroke}{rgb}{0.000000,0.750000,0.750000}%
\pgfsetstrokecolor{currentstroke}%
\pgfsetdash{}{0pt}%
\pgfusepath{stroke}%
\end{pgfscope}%
\begin{pgfscope}%
\pgfpathrectangle{\pgfqpoint{0.781402in}{0.773588in}}{\pgfqpoint{1.440244in}{5.415119in}}%
\pgfusepath{clip}%
\pgfsetrectcap%
\pgfsetroundjoin%
\pgfsetlinewidth{1.505625pt}%
\definecolor{currentstroke}{rgb}{0.750000,0.000000,0.750000}%
\pgfsetstrokecolor{currentstroke}%
\pgfsetdash{}{0pt}%
\pgfusepath{stroke}%
\end{pgfscope}%
\begin{pgfscope}%
\pgfsetrectcap%
\pgfsetmiterjoin%
\pgfsetlinewidth{0.803000pt}%
\definecolor{currentstroke}{rgb}{0.000000,0.000000,0.000000}%
\pgfsetstrokecolor{currentstroke}%
\pgfsetdash{}{0pt}%
\pgfpathmoveto{\pgfqpoint{0.781402in}{0.773588in}}%
\pgfpathlineto{\pgfqpoint{0.781402in}{6.188708in}}%
\pgfusepath{stroke}%
\end{pgfscope}%
\begin{pgfscope}%
\pgfsetrectcap%
\pgfsetmiterjoin%
\pgfsetlinewidth{0.803000pt}%
\definecolor{currentstroke}{rgb}{0.000000,0.000000,0.000000}%
\pgfsetstrokecolor{currentstroke}%
\pgfsetdash{}{0pt}%
\pgfpathmoveto{\pgfqpoint{0.781402in}{0.773588in}}%
\pgfpathlineto{\pgfqpoint{2.221647in}{0.773588in}}%
\pgfusepath{stroke}%
\end{pgfscope}%
\begin{pgfscope}%
\pgfsetbuttcap%
\pgfsetmiterjoin%
\definecolor{currentfill}{rgb}{1.000000,1.000000,1.000000}%
\pgfsetfillcolor{currentfill}%
\pgfsetlinewidth{0.000000pt}%
\definecolor{currentstroke}{rgb}{0.000000,0.000000,0.000000}%
\pgfsetstrokecolor{currentstroke}%
\pgfsetstrokeopacity{0.000000}%
\pgfsetdash{}{0pt}%
\pgfpathmoveto{\pgfqpoint{2.662073in}{0.773588in}}%
\pgfpathlineto{\pgfqpoint{5.626098in}{0.773588in}}%
\pgfpathlineto{\pgfqpoint{5.626098in}{6.188708in}}%
\pgfpathlineto{\pgfqpoint{2.662073in}{6.188708in}}%
\pgfpathclose%
\pgfusepath{fill}%
\end{pgfscope}%
\begin{pgfscope}%
\pgfpathrectangle{\pgfqpoint{2.662073in}{0.773588in}}{\pgfqpoint{2.964025in}{5.415119in}}%
\pgfusepath{clip}%
\pgfsetbuttcap%
\pgfsetroundjoin%
\definecolor{currentfill}{rgb}{0.121569,0.466667,0.705882}%
\pgfsetfillcolor{currentfill}%
\pgfsetlinewidth{0.000000pt}%
\definecolor{currentstroke}{rgb}{0.000000,0.000000,0.000000}%
\pgfsetstrokecolor{currentstroke}%
\pgfsetdash{}{0pt}%
\pgfpathmoveto{\pgfqpoint{-28.588250in}{0.773588in}}%
\pgfpathlineto{\pgfqpoint{-28.588250in}{0.773588in}}%
\pgfpathlineto{\pgfqpoint{-28.543533in}{0.773588in}}%
\pgfpathlineto{\pgfqpoint{-28.498901in}{0.773588in}}%
\pgfpathlineto{\pgfqpoint{-28.452363in}{0.773588in}}%
\pgfpathlineto{\pgfqpoint{-28.408130in}{0.773588in}}%
\pgfpathlineto{\pgfqpoint{-28.364238in}{0.773588in}}%
\pgfpathlineto{\pgfqpoint{-28.318373in}{0.773588in}}%
\pgfpathlineto{\pgfqpoint{-28.274355in}{0.773588in}}%
\pgfpathlineto{\pgfqpoint{-28.230974in}{0.773588in}}%
\pgfpathlineto{\pgfqpoint{-28.185953in}{0.773588in}}%
\pgfpathlineto{\pgfqpoint{-28.141536in}{0.773588in}}%
\pgfpathlineto{\pgfqpoint{-28.097071in}{0.773588in}}%
\pgfpathlineto{\pgfqpoint{-28.051883in}{0.773588in}}%
\pgfpathlineto{\pgfqpoint{-28.007858in}{0.773588in}}%
\pgfpathlineto{\pgfqpoint{-27.963188in}{0.773588in}}%
\pgfpathlineto{\pgfqpoint{-27.917812in}{0.773588in}}%
\pgfpathlineto{\pgfqpoint{-27.874405in}{0.773588in}}%
\pgfpathlineto{\pgfqpoint{-27.830427in}{0.773588in}}%
\pgfpathlineto{\pgfqpoint{-27.785329in}{0.773588in}}%
\pgfpathlineto{\pgfqpoint{-27.740871in}{0.773588in}}%
\pgfpathlineto{\pgfqpoint{-27.696819in}{0.773588in}}%
\pgfpathlineto{\pgfqpoint{-27.651646in}{0.773588in}}%
\pgfpathlineto{\pgfqpoint{-27.608047in}{0.773588in}}%
\pgfpathlineto{\pgfqpoint{-27.563341in}{0.773588in}}%
\pgfpathlineto{\pgfqpoint{-27.516652in}{0.773588in}}%
\pgfpathlineto{\pgfqpoint{-27.470237in}{0.773588in}}%
\pgfpathlineto{\pgfqpoint{-27.422670in}{0.773588in}}%
\pgfpathlineto{\pgfqpoint{-27.374936in}{0.773588in}}%
\pgfpathlineto{\pgfqpoint{-27.328705in}{0.773588in}}%
\pgfpathlineto{\pgfqpoint{-27.281821in}{0.773588in}}%
\pgfpathlineto{\pgfqpoint{-27.232862in}{0.773588in}}%
\pgfpathlineto{\pgfqpoint{-27.184903in}{0.773588in}}%
\pgfpathlineto{\pgfqpoint{-27.136341in}{0.773588in}}%
\pgfpathlineto{\pgfqpoint{-27.086555in}{0.773588in}}%
\pgfpathlineto{\pgfqpoint{-27.037714in}{0.773588in}}%
\pgfpathlineto{\pgfqpoint{-26.988998in}{0.773588in}}%
\pgfpathlineto{\pgfqpoint{-26.933555in}{0.773588in}}%
\pgfpathlineto{\pgfqpoint{-26.880611in}{0.773588in}}%
\pgfpathlineto{\pgfqpoint{-26.830647in}{0.773588in}}%
\pgfpathlineto{\pgfqpoint{-26.781839in}{0.773588in}}%
\pgfpathlineto{\pgfqpoint{-26.735142in}{0.773588in}}%
\pgfpathlineto{\pgfqpoint{-26.688111in}{0.773588in}}%
\pgfpathlineto{\pgfqpoint{-26.640477in}{0.773588in}}%
\pgfpathlineto{\pgfqpoint{-26.595409in}{0.773588in}}%
\pgfpathlineto{\pgfqpoint{-26.549765in}{0.773588in}}%
\pgfpathlineto{\pgfqpoint{-26.503012in}{0.773588in}}%
\pgfpathlineto{\pgfqpoint{-26.458732in}{0.773588in}}%
\pgfpathlineto{\pgfqpoint{-26.414228in}{0.773588in}}%
\pgfpathlineto{\pgfqpoint{-26.368659in}{0.773588in}}%
\pgfpathlineto{\pgfqpoint{-26.323384in}{0.773588in}}%
\pgfpathlineto{\pgfqpoint{-26.277965in}{0.773588in}}%
\pgfpathlineto{\pgfqpoint{-26.230565in}{0.773588in}}%
\pgfpathlineto{\pgfqpoint{-26.185627in}{0.773588in}}%
\pgfpathlineto{\pgfqpoint{-26.140687in}{0.773588in}}%
\pgfpathlineto{\pgfqpoint{-26.095397in}{0.773588in}}%
\pgfpathlineto{\pgfqpoint{-26.049962in}{0.773588in}}%
\pgfpathlineto{\pgfqpoint{-26.004871in}{0.773588in}}%
\pgfpathlineto{\pgfqpoint{-25.957619in}{0.773588in}}%
\pgfpathlineto{\pgfqpoint{-25.913010in}{0.773588in}}%
\pgfpathlineto{\pgfqpoint{-25.868331in}{0.773588in}}%
\pgfpathlineto{\pgfqpoint{-25.821656in}{0.773588in}}%
\pgfpathlineto{\pgfqpoint{-25.776589in}{0.773588in}}%
\pgfpathlineto{\pgfqpoint{-25.731236in}{0.773588in}}%
\pgfpathlineto{\pgfqpoint{-25.684196in}{0.773588in}}%
\pgfpathlineto{\pgfqpoint{-25.638897in}{0.773588in}}%
\pgfpathlineto{\pgfqpoint{-25.593105in}{0.773588in}}%
\pgfpathlineto{\pgfqpoint{-25.546370in}{0.773588in}}%
\pgfpathlineto{\pgfqpoint{-25.501531in}{0.773588in}}%
\pgfpathlineto{\pgfqpoint{-25.455838in}{0.773588in}}%
\pgfpathlineto{\pgfqpoint{-25.409278in}{0.773588in}}%
\pgfpathlineto{\pgfqpoint{-25.364031in}{0.773588in}}%
\pgfpathlineto{\pgfqpoint{-25.318638in}{0.773588in}}%
\pgfpathlineto{\pgfqpoint{-25.271783in}{0.773588in}}%
\pgfpathlineto{\pgfqpoint{-25.227067in}{0.773588in}}%
\pgfpathlineto{\pgfqpoint{-25.181603in}{0.773588in}}%
\pgfpathlineto{\pgfqpoint{-25.133757in}{0.773588in}}%
\pgfpathlineto{\pgfqpoint{-25.088791in}{0.773588in}}%
\pgfpathlineto{\pgfqpoint{-25.043027in}{0.773588in}}%
\pgfpathlineto{\pgfqpoint{-24.995958in}{0.773588in}}%
\pgfpathlineto{\pgfqpoint{-24.950861in}{0.773588in}}%
\pgfpathlineto{\pgfqpoint{-24.905462in}{0.773588in}}%
\pgfpathlineto{\pgfqpoint{-24.858721in}{0.773588in}}%
\pgfpathlineto{\pgfqpoint{-24.813437in}{0.773588in}}%
\pgfpathlineto{\pgfqpoint{-24.768227in}{0.773588in}}%
\pgfpathlineto{\pgfqpoint{-24.722523in}{0.773588in}}%
\pgfpathlineto{\pgfqpoint{-24.677917in}{0.773588in}}%
\pgfpathlineto{\pgfqpoint{-24.632696in}{0.773588in}}%
\pgfpathlineto{\pgfqpoint{-24.586259in}{0.773588in}}%
\pgfpathlineto{\pgfqpoint{-24.540716in}{0.773588in}}%
\pgfpathlineto{\pgfqpoint{-24.494847in}{0.773588in}}%
\pgfpathlineto{\pgfqpoint{-24.447114in}{0.773588in}}%
\pgfpathlineto{\pgfqpoint{-24.401173in}{0.773588in}}%
\pgfpathlineto{\pgfqpoint{-24.355467in}{0.773588in}}%
\pgfpathlineto{\pgfqpoint{-24.308071in}{0.773588in}}%
\pgfpathlineto{\pgfqpoint{-24.262109in}{0.773588in}}%
\pgfpathlineto{\pgfqpoint{-24.216953in}{0.773588in}}%
\pgfpathlineto{\pgfqpoint{-24.169854in}{0.773588in}}%
\pgfpathlineto{\pgfqpoint{-24.123999in}{0.773588in}}%
\pgfpathlineto{\pgfqpoint{-24.078902in}{0.773588in}}%
\pgfpathlineto{\pgfqpoint{-24.032531in}{0.773588in}}%
\pgfpathlineto{\pgfqpoint{-23.986811in}{0.773588in}}%
\pgfpathlineto{\pgfqpoint{-23.941619in}{0.773588in}}%
\pgfpathlineto{\pgfqpoint{-23.895017in}{0.773588in}}%
\pgfpathlineto{\pgfqpoint{-23.849943in}{0.773588in}}%
\pgfpathlineto{\pgfqpoint{-23.804343in}{0.773588in}}%
\pgfpathlineto{\pgfqpoint{-23.757289in}{0.773588in}}%
\pgfpathlineto{\pgfqpoint{-23.711883in}{0.773588in}}%
\pgfpathlineto{\pgfqpoint{-23.666576in}{0.773588in}}%
\pgfpathlineto{\pgfqpoint{-23.620333in}{0.773588in}}%
\pgfpathlineto{\pgfqpoint{-23.575877in}{0.773588in}}%
\pgfpathlineto{\pgfqpoint{-23.529965in}{0.773588in}}%
\pgfpathlineto{\pgfqpoint{-23.482780in}{0.773588in}}%
\pgfpathlineto{\pgfqpoint{-23.437154in}{0.773588in}}%
\pgfpathlineto{\pgfqpoint{-23.391460in}{0.773588in}}%
\pgfpathlineto{\pgfqpoint{-23.344516in}{0.773588in}}%
\pgfpathlineto{\pgfqpoint{-23.298917in}{0.773588in}}%
\pgfpathlineto{\pgfqpoint{-23.252910in}{0.773588in}}%
\pgfpathlineto{\pgfqpoint{-23.205343in}{0.773588in}}%
\pgfpathlineto{\pgfqpoint{-23.159034in}{0.773588in}}%
\pgfpathlineto{\pgfqpoint{-23.112561in}{0.773588in}}%
\pgfpathlineto{\pgfqpoint{-23.065076in}{0.773588in}}%
\pgfpathlineto{\pgfqpoint{-23.018637in}{0.773588in}}%
\pgfpathlineto{\pgfqpoint{-22.972746in}{0.773588in}}%
\pgfpathlineto{\pgfqpoint{-22.924822in}{0.773588in}}%
\pgfpathlineto{\pgfqpoint{-22.878494in}{0.773588in}}%
\pgfpathlineto{\pgfqpoint{-22.833029in}{0.773588in}}%
\pgfpathlineto{\pgfqpoint{-22.786844in}{0.773588in}}%
\pgfpathlineto{\pgfqpoint{-22.742185in}{0.773588in}}%
\pgfpathlineto{\pgfqpoint{-22.697642in}{0.773588in}}%
\pgfpathlineto{\pgfqpoint{-22.651258in}{0.773588in}}%
\pgfpathlineto{\pgfqpoint{-22.606850in}{0.773588in}}%
\pgfpathlineto{\pgfqpoint{-22.562133in}{0.773588in}}%
\pgfpathlineto{\pgfqpoint{-22.516601in}{0.773588in}}%
\pgfpathlineto{\pgfqpoint{-22.471952in}{0.773588in}}%
\pgfpathlineto{\pgfqpoint{-22.427383in}{0.773588in}}%
\pgfpathlineto{\pgfqpoint{-22.380996in}{0.773588in}}%
\pgfpathlineto{\pgfqpoint{-22.336074in}{0.773588in}}%
\pgfpathlineto{\pgfqpoint{-22.291940in}{0.773588in}}%
\pgfpathlineto{\pgfqpoint{-22.246011in}{0.773588in}}%
\pgfpathlineto{\pgfqpoint{-22.201868in}{0.773588in}}%
\pgfpathlineto{\pgfqpoint{-22.158025in}{0.773588in}}%
\pgfpathlineto{\pgfqpoint{-22.112231in}{0.773588in}}%
\pgfpathlineto{\pgfqpoint{-22.067661in}{0.773588in}}%
\pgfpathlineto{\pgfqpoint{-22.022397in}{0.773588in}}%
\pgfpathlineto{\pgfqpoint{-21.976168in}{0.773588in}}%
\pgfpathlineto{\pgfqpoint{-21.930597in}{0.773588in}}%
\pgfpathlineto{\pgfqpoint{-21.885462in}{0.773588in}}%
\pgfpathlineto{\pgfqpoint{-21.839254in}{0.773588in}}%
\pgfpathlineto{\pgfqpoint{-21.794255in}{0.773588in}}%
\pgfpathlineto{\pgfqpoint{-21.749554in}{0.773588in}}%
\pgfpathlineto{\pgfqpoint{-21.703167in}{0.773588in}}%
\pgfpathlineto{\pgfqpoint{-21.657342in}{0.773588in}}%
\pgfpathlineto{\pgfqpoint{-21.612919in}{0.773588in}}%
\pgfpathlineto{\pgfqpoint{-21.566198in}{0.773588in}}%
\pgfpathlineto{\pgfqpoint{-21.520883in}{0.773588in}}%
\pgfpathlineto{\pgfqpoint{-21.475985in}{0.773588in}}%
\pgfpathlineto{\pgfqpoint{-21.430267in}{0.773588in}}%
\pgfpathlineto{\pgfqpoint{-21.385572in}{0.773588in}}%
\pgfpathlineto{\pgfqpoint{-21.340867in}{0.773588in}}%
\pgfpathlineto{\pgfqpoint{-21.294264in}{0.773588in}}%
\pgfpathlineto{\pgfqpoint{-21.249304in}{0.773588in}}%
\pgfpathlineto{\pgfqpoint{-21.204354in}{0.773588in}}%
\pgfpathlineto{\pgfqpoint{-21.158215in}{0.773588in}}%
\pgfpathlineto{\pgfqpoint{-21.113139in}{0.773588in}}%
\pgfpathlineto{\pgfqpoint{-21.067357in}{0.773588in}}%
\pgfpathlineto{\pgfqpoint{-21.021509in}{0.773588in}}%
\pgfpathlineto{\pgfqpoint{-20.976378in}{0.773588in}}%
\pgfpathlineto{\pgfqpoint{-20.932365in}{0.773588in}}%
\pgfpathlineto{\pgfqpoint{-20.886847in}{0.773588in}}%
\pgfpathlineto{\pgfqpoint{-20.842169in}{0.773588in}}%
\pgfpathlineto{\pgfqpoint{-20.797022in}{0.773588in}}%
\pgfpathlineto{\pgfqpoint{-20.750831in}{0.773588in}}%
\pgfpathlineto{\pgfqpoint{-20.705066in}{0.773588in}}%
\pgfpathlineto{\pgfqpoint{-20.658742in}{0.773588in}}%
\pgfpathlineto{\pgfqpoint{-20.611243in}{0.773588in}}%
\pgfpathlineto{\pgfqpoint{-20.565077in}{0.773588in}}%
\pgfpathlineto{\pgfqpoint{-20.519904in}{0.773588in}}%
\pgfpathlineto{\pgfqpoint{-20.474014in}{0.773588in}}%
\pgfpathlineto{\pgfqpoint{-20.428221in}{0.773588in}}%
\pgfpathlineto{\pgfqpoint{-20.382199in}{0.773588in}}%
\pgfpathlineto{\pgfqpoint{-20.335981in}{0.773588in}}%
\pgfpathlineto{\pgfqpoint{-20.290419in}{0.773588in}}%
\pgfpathlineto{\pgfqpoint{-20.244875in}{0.773588in}}%
\pgfpathlineto{\pgfqpoint{-20.197814in}{0.773588in}}%
\pgfpathlineto{\pgfqpoint{-20.153038in}{0.773588in}}%
\pgfpathlineto{\pgfqpoint{-20.107707in}{0.773588in}}%
\pgfpathlineto{\pgfqpoint{-20.061722in}{0.773588in}}%
\pgfpathlineto{\pgfqpoint{-20.016506in}{0.773588in}}%
\pgfpathlineto{\pgfqpoint{-19.970967in}{0.773588in}}%
\pgfpathlineto{\pgfqpoint{-19.924139in}{0.773588in}}%
\pgfpathlineto{\pgfqpoint{-19.878562in}{0.773588in}}%
\pgfpathlineto{\pgfqpoint{-19.833948in}{0.773588in}}%
\pgfpathlineto{\pgfqpoint{-19.787808in}{0.773588in}}%
\pgfpathlineto{\pgfqpoint{-19.742024in}{0.773588in}}%
\pgfpathlineto{\pgfqpoint{-19.696649in}{0.773588in}}%
\pgfpathlineto{\pgfqpoint{-19.650361in}{0.773588in}}%
\pgfpathlineto{\pgfqpoint{-19.604824in}{0.773588in}}%
\pgfpathlineto{\pgfqpoint{-19.559546in}{0.773588in}}%
\pgfpathlineto{\pgfqpoint{-19.512973in}{0.773588in}}%
\pgfpathlineto{\pgfqpoint{-19.467303in}{0.773588in}}%
\pgfpathlineto{\pgfqpoint{-19.421584in}{0.773588in}}%
\pgfpathlineto{\pgfqpoint{-19.374230in}{0.773588in}}%
\pgfpathlineto{\pgfqpoint{-19.328532in}{0.773588in}}%
\pgfpathlineto{\pgfqpoint{-19.283047in}{0.773588in}}%
\pgfpathlineto{\pgfqpoint{-19.234797in}{0.773588in}}%
\pgfpathlineto{\pgfqpoint{-19.188899in}{0.773588in}}%
\pgfpathlineto{\pgfqpoint{-19.143544in}{0.773588in}}%
\pgfpathlineto{\pgfqpoint{-19.097062in}{0.773588in}}%
\pgfpathlineto{\pgfqpoint{-19.050812in}{0.773588in}}%
\pgfpathlineto{\pgfqpoint{-19.004576in}{0.773588in}}%
\pgfpathlineto{\pgfqpoint{-18.956783in}{0.773588in}}%
\pgfpathlineto{\pgfqpoint{-18.910503in}{0.773588in}}%
\pgfpathlineto{\pgfqpoint{-18.865040in}{0.773588in}}%
\pgfpathlineto{\pgfqpoint{-18.817991in}{0.773588in}}%
\pgfpathlineto{\pgfqpoint{-18.772260in}{0.773588in}}%
\pgfpathlineto{\pgfqpoint{-18.726033in}{0.773588in}}%
\pgfpathlineto{\pgfqpoint{-18.678684in}{0.773588in}}%
\pgfpathlineto{\pgfqpoint{-18.632978in}{0.773588in}}%
\pgfpathlineto{\pgfqpoint{-18.587378in}{0.773588in}}%
\pgfpathlineto{\pgfqpoint{-18.540355in}{0.773588in}}%
\pgfpathlineto{\pgfqpoint{-18.494344in}{0.773588in}}%
\pgfpathlineto{\pgfqpoint{-18.448764in}{0.773588in}}%
\pgfpathlineto{\pgfqpoint{-18.402000in}{0.773588in}}%
\pgfpathlineto{\pgfqpoint{-18.356748in}{0.773588in}}%
\pgfpathlineto{\pgfqpoint{-18.311512in}{0.773588in}}%
\pgfpathlineto{\pgfqpoint{-18.264167in}{0.773588in}}%
\pgfpathlineto{\pgfqpoint{-18.218674in}{0.773588in}}%
\pgfpathlineto{\pgfqpoint{-18.172791in}{0.773588in}}%
\pgfpathlineto{\pgfqpoint{-18.124795in}{0.773588in}}%
\pgfpathlineto{\pgfqpoint{-18.078784in}{0.773588in}}%
\pgfpathlineto{\pgfqpoint{-18.032389in}{0.773588in}}%
\pgfpathlineto{\pgfqpoint{-17.984102in}{0.773588in}}%
\pgfpathlineto{\pgfqpoint{-17.937614in}{0.773588in}}%
\pgfpathlineto{\pgfqpoint{-17.891271in}{0.773588in}}%
\pgfpathlineto{\pgfqpoint{-17.843480in}{0.773588in}}%
\pgfpathlineto{\pgfqpoint{-17.797346in}{0.773588in}}%
\pgfpathlineto{\pgfqpoint{-17.751556in}{0.773588in}}%
\pgfpathlineto{\pgfqpoint{-17.704093in}{0.773588in}}%
\pgfpathlineto{\pgfqpoint{-17.658538in}{0.773588in}}%
\pgfpathlineto{\pgfqpoint{-17.613251in}{0.773588in}}%
\pgfpathlineto{\pgfqpoint{-17.566739in}{0.773588in}}%
\pgfpathlineto{\pgfqpoint{-17.521284in}{0.773588in}}%
\pgfpathlineto{\pgfqpoint{-17.474644in}{0.773588in}}%
\pgfpathlineto{\pgfqpoint{-17.427002in}{0.773588in}}%
\pgfpathlineto{\pgfqpoint{-17.381732in}{0.773588in}}%
\pgfpathlineto{\pgfqpoint{-17.336276in}{0.773588in}}%
\pgfpathlineto{\pgfqpoint{-17.289523in}{0.773588in}}%
\pgfpathlineto{\pgfqpoint{-17.244258in}{0.773588in}}%
\pgfpathlineto{\pgfqpoint{-17.198221in}{0.773588in}}%
\pgfpathlineto{\pgfqpoint{-17.150161in}{0.773588in}}%
\pgfpathlineto{\pgfqpoint{-17.103874in}{0.773588in}}%
\pgfpathlineto{\pgfqpoint{-17.058009in}{0.773588in}}%
\pgfpathlineto{\pgfqpoint{-17.010884in}{0.773588in}}%
\pgfpathlineto{\pgfqpoint{-16.964774in}{0.773588in}}%
\pgfpathlineto{\pgfqpoint{-16.918692in}{0.773588in}}%
\pgfpathlineto{\pgfqpoint{-16.871664in}{0.773588in}}%
\pgfpathlineto{\pgfqpoint{-16.825434in}{0.773588in}}%
\pgfpathlineto{\pgfqpoint{-16.778530in}{0.773588in}}%
\pgfpathlineto{\pgfqpoint{-16.730721in}{0.773588in}}%
\pgfpathlineto{\pgfqpoint{-16.684310in}{0.773588in}}%
\pgfpathlineto{\pgfqpoint{-16.637120in}{0.773588in}}%
\pgfpathlineto{\pgfqpoint{-16.589239in}{0.773588in}}%
\pgfpathlineto{\pgfqpoint{-16.542990in}{0.773588in}}%
\pgfpathlineto{\pgfqpoint{-16.496157in}{0.773588in}}%
\pgfpathlineto{\pgfqpoint{-16.449347in}{0.773588in}}%
\pgfpathlineto{\pgfqpoint{-16.402550in}{0.773588in}}%
\pgfpathlineto{\pgfqpoint{-16.355646in}{0.773588in}}%
\pgfpathlineto{\pgfqpoint{-16.308871in}{0.773588in}}%
\pgfpathlineto{\pgfqpoint{-16.262806in}{0.773588in}}%
\pgfpathlineto{\pgfqpoint{-16.217070in}{0.773588in}}%
\pgfpathlineto{\pgfqpoint{-16.169427in}{0.773588in}}%
\pgfpathlineto{\pgfqpoint{-16.123165in}{0.773588in}}%
\pgfpathlineto{\pgfqpoint{-16.077171in}{0.773588in}}%
\pgfpathlineto{\pgfqpoint{-16.029628in}{0.773588in}}%
\pgfpathlineto{\pgfqpoint{-15.984286in}{0.773588in}}%
\pgfpathlineto{\pgfqpoint{-15.938076in}{0.773588in}}%
\pgfpathlineto{\pgfqpoint{-15.890419in}{0.773588in}}%
\pgfpathlineto{\pgfqpoint{-15.844500in}{0.773588in}}%
\pgfpathlineto{\pgfqpoint{-15.798897in}{0.773588in}}%
\pgfpathlineto{\pgfqpoint{-15.750800in}{0.773588in}}%
\pgfpathlineto{\pgfqpoint{-15.703994in}{0.773588in}}%
\pgfpathlineto{\pgfqpoint{-15.657002in}{0.773588in}}%
\pgfpathlineto{\pgfqpoint{-15.608704in}{0.773588in}}%
\pgfpathlineto{\pgfqpoint{-15.560849in}{0.773588in}}%
\pgfpathlineto{\pgfqpoint{-15.512535in}{0.773588in}}%
\pgfpathlineto{\pgfqpoint{-15.463539in}{0.773588in}}%
\pgfpathlineto{\pgfqpoint{-15.416792in}{0.773588in}}%
\pgfpathlineto{\pgfqpoint{-15.369802in}{0.773588in}}%
\pgfpathlineto{\pgfqpoint{-15.321194in}{0.773588in}}%
\pgfpathlineto{\pgfqpoint{-15.273998in}{0.773588in}}%
\pgfpathlineto{\pgfqpoint{-15.225998in}{0.773588in}}%
\pgfpathlineto{\pgfqpoint{-15.176769in}{0.773588in}}%
\pgfpathlineto{\pgfqpoint{-15.130553in}{0.773588in}}%
\pgfpathlineto{\pgfqpoint{-15.083728in}{0.773588in}}%
\pgfpathlineto{\pgfqpoint{-15.035921in}{0.773588in}}%
\pgfpathlineto{\pgfqpoint{-14.989019in}{0.773588in}}%
\pgfpathlineto{\pgfqpoint{-14.942416in}{0.773588in}}%
\pgfpathlineto{\pgfqpoint{-14.894419in}{0.773588in}}%
\pgfpathlineto{\pgfqpoint{-14.847516in}{0.773588in}}%
\pgfpathlineto{\pgfqpoint{-14.800669in}{0.773588in}}%
\pgfpathlineto{\pgfqpoint{-14.752153in}{0.773588in}}%
\pgfpathlineto{\pgfqpoint{-14.705919in}{0.773588in}}%
\pgfpathlineto{\pgfqpoint{-14.659388in}{0.773588in}}%
\pgfpathlineto{\pgfqpoint{-14.611394in}{0.773588in}}%
\pgfpathlineto{\pgfqpoint{-14.564671in}{0.773588in}}%
\pgfpathlineto{\pgfqpoint{-14.517904in}{0.773588in}}%
\pgfpathlineto{\pgfqpoint{-14.469907in}{0.773588in}}%
\pgfpathlineto{\pgfqpoint{-14.422978in}{0.773588in}}%
\pgfpathlineto{\pgfqpoint{-14.375931in}{0.773588in}}%
\pgfpathlineto{\pgfqpoint{-14.327371in}{0.773588in}}%
\pgfpathlineto{\pgfqpoint{-14.280667in}{0.773588in}}%
\pgfpathlineto{\pgfqpoint{-14.233487in}{0.773588in}}%
\pgfpathlineto{\pgfqpoint{-14.184689in}{0.773588in}}%
\pgfpathlineto{\pgfqpoint{-14.137283in}{0.773588in}}%
\pgfpathlineto{\pgfqpoint{-14.090279in}{0.773588in}}%
\pgfpathlineto{\pgfqpoint{-14.042263in}{0.773588in}}%
\pgfpathlineto{\pgfqpoint{-13.996250in}{0.773588in}}%
\pgfpathlineto{\pgfqpoint{-13.950094in}{0.773588in}}%
\pgfpathlineto{\pgfqpoint{-13.901287in}{0.773588in}}%
\pgfpathlineto{\pgfqpoint{-13.853816in}{0.773588in}}%
\pgfpathlineto{\pgfqpoint{-13.806330in}{0.773588in}}%
\pgfpathlineto{\pgfqpoint{-13.757485in}{0.773588in}}%
\pgfpathlineto{\pgfqpoint{-13.710126in}{0.773588in}}%
\pgfpathlineto{\pgfqpoint{-13.663331in}{0.773588in}}%
\pgfpathlineto{\pgfqpoint{-13.615991in}{0.773588in}}%
\pgfpathlineto{\pgfqpoint{-13.568873in}{0.773588in}}%
\pgfpathlineto{\pgfqpoint{-13.522553in}{0.773588in}}%
\pgfpathlineto{\pgfqpoint{-13.474335in}{0.773588in}}%
\pgfpathlineto{\pgfqpoint{-13.428346in}{0.773588in}}%
\pgfpathlineto{\pgfqpoint{-13.382220in}{0.773588in}}%
\pgfpathlineto{\pgfqpoint{-13.334326in}{0.773588in}}%
\pgfpathlineto{\pgfqpoint{-13.287448in}{0.773588in}}%
\pgfpathlineto{\pgfqpoint{-13.239807in}{0.773588in}}%
\pgfpathlineto{\pgfqpoint{-13.190970in}{0.773588in}}%
\pgfpathlineto{\pgfqpoint{-13.144068in}{0.773588in}}%
\pgfpathlineto{\pgfqpoint{-13.096328in}{0.773588in}}%
\pgfpathlineto{\pgfqpoint{-13.046608in}{0.773588in}}%
\pgfpathlineto{\pgfqpoint{-12.998480in}{0.773588in}}%
\pgfpathlineto{\pgfqpoint{-12.949523in}{0.773588in}}%
\pgfpathlineto{\pgfqpoint{-12.899585in}{0.773588in}}%
\pgfpathlineto{\pgfqpoint{-12.852526in}{0.773588in}}%
\pgfpathlineto{\pgfqpoint{-12.805010in}{0.773588in}}%
\pgfpathlineto{\pgfqpoint{-12.755866in}{0.773588in}}%
\pgfpathlineto{\pgfqpoint{-12.708579in}{0.773588in}}%
\pgfpathlineto{\pgfqpoint{-12.661578in}{0.773588in}}%
\pgfpathlineto{\pgfqpoint{-12.612433in}{0.773588in}}%
\pgfpathlineto{\pgfqpoint{-12.565107in}{0.773588in}}%
\pgfpathlineto{\pgfqpoint{-12.517880in}{0.773588in}}%
\pgfpathlineto{\pgfqpoint{-12.469843in}{0.773588in}}%
\pgfpathlineto{\pgfqpoint{-12.422612in}{0.773588in}}%
\pgfpathlineto{\pgfqpoint{-12.375560in}{0.773588in}}%
\pgfpathlineto{\pgfqpoint{-12.326627in}{0.773588in}}%
\pgfpathlineto{\pgfqpoint{-12.278675in}{0.773588in}}%
\pgfpathlineto{\pgfqpoint{-12.231638in}{0.773588in}}%
\pgfpathlineto{\pgfqpoint{-12.182471in}{0.773588in}}%
\pgfpathlineto{\pgfqpoint{-12.135286in}{0.773588in}}%
\pgfpathlineto{\pgfqpoint{-12.087205in}{0.773588in}}%
\pgfpathlineto{\pgfqpoint{-12.038569in}{0.773588in}}%
\pgfpathlineto{\pgfqpoint{-11.991431in}{0.773588in}}%
\pgfpathlineto{\pgfqpoint{-11.943783in}{0.773588in}}%
\pgfpathlineto{\pgfqpoint{-11.895021in}{0.773588in}}%
\pgfpathlineto{\pgfqpoint{-11.847736in}{0.773588in}}%
\pgfpathlineto{\pgfqpoint{-11.800403in}{0.773588in}}%
\pgfpathlineto{\pgfqpoint{-11.751492in}{0.773588in}}%
\pgfpathlineto{\pgfqpoint{-11.704231in}{0.773588in}}%
\pgfpathlineto{\pgfqpoint{-11.656418in}{0.773588in}}%
\pgfpathlineto{\pgfqpoint{-11.607131in}{0.773588in}}%
\pgfpathlineto{\pgfqpoint{-11.559336in}{0.773588in}}%
\pgfpathlineto{\pgfqpoint{-11.511546in}{0.773588in}}%
\pgfpathlineto{\pgfqpoint{-11.463260in}{0.773588in}}%
\pgfpathlineto{\pgfqpoint{-11.415235in}{0.773588in}}%
\pgfpathlineto{\pgfqpoint{-11.366120in}{0.773588in}}%
\pgfpathlineto{\pgfqpoint{-11.316622in}{0.773588in}}%
\pgfpathlineto{\pgfqpoint{-11.268838in}{0.773588in}}%
\pgfpathlineto{\pgfqpoint{-11.221741in}{0.773588in}}%
\pgfpathlineto{\pgfqpoint{-11.172812in}{0.773588in}}%
\pgfpathlineto{\pgfqpoint{-11.125274in}{0.773588in}}%
\pgfpathlineto{\pgfqpoint{-11.077704in}{0.773588in}}%
\pgfpathlineto{\pgfqpoint{-11.028607in}{0.773588in}}%
\pgfpathlineto{\pgfqpoint{-10.980903in}{0.773588in}}%
\pgfpathlineto{\pgfqpoint{-10.933861in}{0.773588in}}%
\pgfpathlineto{\pgfqpoint{-10.885230in}{0.773588in}}%
\pgfpathlineto{\pgfqpoint{-10.837686in}{0.773588in}}%
\pgfpathlineto{\pgfqpoint{-10.790422in}{0.773588in}}%
\pgfpathlineto{\pgfqpoint{-10.741507in}{0.773588in}}%
\pgfpathlineto{\pgfqpoint{-10.693023in}{0.773588in}}%
\pgfpathlineto{\pgfqpoint{-10.644481in}{0.773588in}}%
\pgfpathlineto{\pgfqpoint{-10.594934in}{0.773588in}}%
\pgfpathlineto{\pgfqpoint{-10.546833in}{0.773588in}}%
\pgfpathlineto{\pgfqpoint{-10.497957in}{0.773588in}}%
\pgfpathlineto{\pgfqpoint{-10.448474in}{0.773588in}}%
\pgfpathlineto{\pgfqpoint{-10.401144in}{0.773588in}}%
\pgfpathlineto{\pgfqpoint{-10.352351in}{0.773588in}}%
\pgfpathlineto{\pgfqpoint{-10.301938in}{0.773588in}}%
\pgfpathlineto{\pgfqpoint{-10.253421in}{0.773588in}}%
\pgfpathlineto{\pgfqpoint{-10.204843in}{0.773588in}}%
\pgfpathlineto{\pgfqpoint{-10.154319in}{0.773588in}}%
\pgfpathlineto{\pgfqpoint{-10.106404in}{0.773588in}}%
\pgfpathlineto{\pgfqpoint{-10.058316in}{0.773588in}}%
\pgfpathlineto{\pgfqpoint{-10.009255in}{0.773588in}}%
\pgfpathlineto{\pgfqpoint{-9.960967in}{0.773588in}}%
\pgfpathlineto{\pgfqpoint{-9.912991in}{0.773588in}}%
\pgfpathlineto{\pgfqpoint{-9.863853in}{0.773588in}}%
\pgfpathlineto{\pgfqpoint{-9.816398in}{0.773588in}}%
\pgfpathlineto{\pgfqpoint{-9.768438in}{0.773588in}}%
\pgfpathlineto{\pgfqpoint{-9.719205in}{0.773588in}}%
\pgfpathlineto{\pgfqpoint{-9.671880in}{0.773588in}}%
\pgfpathlineto{\pgfqpoint{-9.624467in}{0.773588in}}%
\pgfpathlineto{\pgfqpoint{-9.575433in}{0.773588in}}%
\pgfpathlineto{\pgfqpoint{-9.527213in}{0.773588in}}%
\pgfpathlineto{\pgfqpoint{-9.479530in}{0.773588in}}%
\pgfpathlineto{\pgfqpoint{-9.430893in}{0.773588in}}%
\pgfpathlineto{\pgfqpoint{-9.382453in}{0.773588in}}%
\pgfpathlineto{\pgfqpoint{-9.332833in}{0.773588in}}%
\pgfpathlineto{\pgfqpoint{-9.281795in}{0.773588in}}%
\pgfpathlineto{\pgfqpoint{-9.232466in}{0.773588in}}%
\pgfpathlineto{\pgfqpoint{-9.182531in}{0.773588in}}%
\pgfpathlineto{\pgfqpoint{-9.130552in}{0.773588in}}%
\pgfpathlineto{\pgfqpoint{-9.080221in}{0.773588in}}%
\pgfpathlineto{\pgfqpoint{-9.030683in}{0.773588in}}%
\pgfpathlineto{\pgfqpoint{-8.979731in}{0.773588in}}%
\pgfpathlineto{\pgfqpoint{-8.930541in}{0.773588in}}%
\pgfpathlineto{\pgfqpoint{-8.880937in}{0.773588in}}%
\pgfpathlineto{\pgfqpoint{-8.829855in}{0.773588in}}%
\pgfpathlineto{\pgfqpoint{-8.780578in}{0.773588in}}%
\pgfpathlineto{\pgfqpoint{-8.731477in}{0.773588in}}%
\pgfpathlineto{\pgfqpoint{-8.681529in}{0.773588in}}%
\pgfpathlineto{\pgfqpoint{-8.632588in}{0.773588in}}%
\pgfpathlineto{\pgfqpoint{-8.583181in}{0.773588in}}%
\pgfpathlineto{\pgfqpoint{-8.532027in}{0.773588in}}%
\pgfpathlineto{\pgfqpoint{-8.481981in}{0.773588in}}%
\pgfpathlineto{\pgfqpoint{-8.432560in}{0.773588in}}%
\pgfpathlineto{\pgfqpoint{-8.382452in}{0.773588in}}%
\pgfpathlineto{\pgfqpoint{-8.332630in}{0.773588in}}%
\pgfpathlineto{\pgfqpoint{-8.283775in}{0.773588in}}%
\pgfpathlineto{\pgfqpoint{-8.233427in}{0.773588in}}%
\pgfpathlineto{\pgfqpoint{-8.184462in}{0.773588in}}%
\pgfpathlineto{\pgfqpoint{-8.135490in}{0.773588in}}%
\pgfpathlineto{\pgfqpoint{-8.085190in}{0.773588in}}%
\pgfpathlineto{\pgfqpoint{-8.035792in}{0.773588in}}%
\pgfpathlineto{\pgfqpoint{-7.986412in}{0.773588in}}%
\pgfpathlineto{\pgfqpoint{-7.935315in}{0.773588in}}%
\pgfpathlineto{\pgfqpoint{-7.885122in}{0.773588in}}%
\pgfpathlineto{\pgfqpoint{-7.835188in}{0.773588in}}%
\pgfpathlineto{\pgfqpoint{-7.783607in}{0.773588in}}%
\pgfpathlineto{\pgfqpoint{-7.733870in}{0.773588in}}%
\pgfpathlineto{\pgfqpoint{-7.685281in}{0.773588in}}%
\pgfpathlineto{\pgfqpoint{-7.634521in}{0.773588in}}%
\pgfpathlineto{\pgfqpoint{-7.585672in}{0.773588in}}%
\pgfpathlineto{\pgfqpoint{-7.536566in}{0.773588in}}%
\pgfpathlineto{\pgfqpoint{-7.485312in}{0.773588in}}%
\pgfpathlineto{\pgfqpoint{-7.435602in}{0.773588in}}%
\pgfpathlineto{\pgfqpoint{-7.386343in}{0.773588in}}%
\pgfpathlineto{\pgfqpoint{-7.335791in}{0.773588in}}%
\pgfpathlineto{\pgfqpoint{-7.286238in}{0.773588in}}%
\pgfpathlineto{\pgfqpoint{-7.237328in}{0.773588in}}%
\pgfpathlineto{\pgfqpoint{-7.185558in}{0.773588in}}%
\pgfpathlineto{\pgfqpoint{-7.135511in}{0.773588in}}%
\pgfpathlineto{\pgfqpoint{-7.085212in}{0.773588in}}%
\pgfpathlineto{\pgfqpoint{-7.033318in}{0.773588in}}%
\pgfpathlineto{\pgfqpoint{-6.984061in}{0.773588in}}%
\pgfpathlineto{\pgfqpoint{-6.934169in}{0.773588in}}%
\pgfpathlineto{\pgfqpoint{-6.882593in}{0.773588in}}%
\pgfpathlineto{\pgfqpoint{-6.832072in}{0.773588in}}%
\pgfpathlineto{\pgfqpoint{-6.782488in}{0.773588in}}%
\pgfpathlineto{\pgfqpoint{-6.731323in}{0.773588in}}%
\pgfpathlineto{\pgfqpoint{-6.682747in}{0.773588in}}%
\pgfpathlineto{\pgfqpoint{-6.634445in}{0.773588in}}%
\pgfpathlineto{\pgfqpoint{-6.582317in}{0.773588in}}%
\pgfpathlineto{\pgfqpoint{-6.532179in}{0.773588in}}%
\pgfpathlineto{\pgfqpoint{-6.481910in}{0.773588in}}%
\pgfpathlineto{\pgfqpoint{-6.429787in}{0.773588in}}%
\pgfpathlineto{\pgfqpoint{-6.380124in}{0.773588in}}%
\pgfpathlineto{\pgfqpoint{-6.330959in}{0.773588in}}%
\pgfpathlineto{\pgfqpoint{-6.279011in}{0.773588in}}%
\pgfpathlineto{\pgfqpoint{-6.228389in}{0.773588in}}%
\pgfpathlineto{\pgfqpoint{-6.177451in}{0.773588in}}%
\pgfpathlineto{\pgfqpoint{-6.125420in}{0.773588in}}%
\pgfpathlineto{\pgfqpoint{-6.075227in}{0.773588in}}%
\pgfpathlineto{\pgfqpoint{-6.024836in}{0.773588in}}%
\pgfpathlineto{\pgfqpoint{-5.974236in}{0.773588in}}%
\pgfpathlineto{\pgfqpoint{-5.924128in}{0.773588in}}%
\pgfpathlineto{\pgfqpoint{-5.874278in}{0.773588in}}%
\pgfpathlineto{\pgfqpoint{-5.822716in}{0.773588in}}%
\pgfpathlineto{\pgfqpoint{-5.774223in}{0.773588in}}%
\pgfpathlineto{\pgfqpoint{-5.724570in}{0.773588in}}%
\pgfpathlineto{\pgfqpoint{-5.673071in}{0.773588in}}%
\pgfpathlineto{\pgfqpoint{-5.623816in}{0.773588in}}%
\pgfpathlineto{\pgfqpoint{-5.574633in}{0.773588in}}%
\pgfpathlineto{\pgfqpoint{-5.523472in}{0.773588in}}%
\pgfpathlineto{\pgfqpoint{-5.473058in}{0.773588in}}%
\pgfpathlineto{\pgfqpoint{-5.421631in}{0.773588in}}%
\pgfpathlineto{\pgfqpoint{-5.369703in}{0.773588in}}%
\pgfpathlineto{\pgfqpoint{-5.319212in}{0.773588in}}%
\pgfpathlineto{\pgfqpoint{-5.267508in}{0.773588in}}%
\pgfpathlineto{\pgfqpoint{-5.215272in}{0.773588in}}%
\pgfpathlineto{\pgfqpoint{-5.165135in}{0.773588in}}%
\pgfpathlineto{\pgfqpoint{-5.114869in}{0.773588in}}%
\pgfpathlineto{\pgfqpoint{-5.064238in}{0.773588in}}%
\pgfpathlineto{\pgfqpoint{-5.014255in}{0.773588in}}%
\pgfpathlineto{\pgfqpoint{-4.963670in}{0.773588in}}%
\pgfpathlineto{\pgfqpoint{-4.911765in}{0.773588in}}%
\pgfpathlineto{\pgfqpoint{-4.861887in}{0.773588in}}%
\pgfpathlineto{\pgfqpoint{-4.812147in}{0.773588in}}%
\pgfpathlineto{\pgfqpoint{-4.759589in}{0.773588in}}%
\pgfpathlineto{\pgfqpoint{-4.709756in}{0.773588in}}%
\pgfpathlineto{\pgfqpoint{-4.659647in}{0.773588in}}%
\pgfpathlineto{\pgfqpoint{-4.607841in}{0.773588in}}%
\pgfpathlineto{\pgfqpoint{-4.558006in}{0.773588in}}%
\pgfpathlineto{\pgfqpoint{-4.508002in}{0.773588in}}%
\pgfpathlineto{\pgfqpoint{-4.457276in}{0.773588in}}%
\pgfpathlineto{\pgfqpoint{-4.407934in}{0.773588in}}%
\pgfpathlineto{\pgfqpoint{-4.357515in}{0.773588in}}%
\pgfpathlineto{\pgfqpoint{-4.305607in}{0.773588in}}%
\pgfpathlineto{\pgfqpoint{-4.255103in}{0.773588in}}%
\pgfpathlineto{\pgfqpoint{-4.204591in}{0.773588in}}%
\pgfpathlineto{\pgfqpoint{-4.153202in}{0.773588in}}%
\pgfpathlineto{\pgfqpoint{-4.103923in}{0.773588in}}%
\pgfpathlineto{\pgfqpoint{-4.053537in}{0.773588in}}%
\pgfpathlineto{\pgfqpoint{-4.002524in}{0.773588in}}%
\pgfpathlineto{\pgfqpoint{-3.952191in}{0.773588in}}%
\pgfpathlineto{\pgfqpoint{-3.902764in}{0.773588in}}%
\pgfpathlineto{\pgfqpoint{-3.850587in}{0.773588in}}%
\pgfpathlineto{\pgfqpoint{-3.799759in}{0.773588in}}%
\pgfpathlineto{\pgfqpoint{-3.748633in}{0.773588in}}%
\pgfpathlineto{\pgfqpoint{-3.696993in}{0.773588in}}%
\pgfpathlineto{\pgfqpoint{-3.646072in}{0.773588in}}%
\pgfpathlineto{\pgfqpoint{-3.595859in}{0.773588in}}%
\pgfpathlineto{\pgfqpoint{-3.544176in}{0.773588in}}%
\pgfpathlineto{\pgfqpoint{-3.494154in}{0.773588in}}%
\pgfpathlineto{\pgfqpoint{-3.444140in}{0.773588in}}%
\pgfpathlineto{\pgfqpoint{-3.392015in}{0.773588in}}%
\pgfpathlineto{\pgfqpoint{-3.341930in}{0.773588in}}%
\pgfpathlineto{\pgfqpoint{-3.292250in}{0.773588in}}%
\pgfpathlineto{\pgfqpoint{-3.241308in}{0.773588in}}%
\pgfpathlineto{\pgfqpoint{-3.190882in}{0.773588in}}%
\pgfpathlineto{\pgfqpoint{-3.140428in}{0.773588in}}%
\pgfpathlineto{\pgfqpoint{-3.088627in}{0.773588in}}%
\pgfpathlineto{\pgfqpoint{-3.039557in}{0.773588in}}%
\pgfpathlineto{\pgfqpoint{-2.990316in}{0.773588in}}%
\pgfpathlineto{\pgfqpoint{-2.938490in}{0.773588in}}%
\pgfpathlineto{\pgfqpoint{-2.887081in}{0.773588in}}%
\pgfpathlineto{\pgfqpoint{-2.836699in}{0.773588in}}%
\pgfpathlineto{\pgfqpoint{-2.784277in}{0.773588in}}%
\pgfpathlineto{\pgfqpoint{-2.732985in}{0.773588in}}%
\pgfpathlineto{\pgfqpoint{-2.681969in}{0.773588in}}%
\pgfpathlineto{\pgfqpoint{-2.629697in}{0.773588in}}%
\pgfpathlineto{\pgfqpoint{-2.578972in}{0.773588in}}%
\pgfpathlineto{\pgfqpoint{-2.528440in}{0.773588in}}%
\pgfpathlineto{\pgfqpoint{-2.476616in}{0.773588in}}%
\pgfpathlineto{\pgfqpoint{-2.425397in}{0.773588in}}%
\pgfpathlineto{\pgfqpoint{-2.375607in}{0.773588in}}%
\pgfpathlineto{\pgfqpoint{-2.323084in}{0.773588in}}%
\pgfpathlineto{\pgfqpoint{-2.272195in}{0.773588in}}%
\pgfpathlineto{\pgfqpoint{-2.220871in}{0.773588in}}%
\pgfpathlineto{\pgfqpoint{-2.167559in}{0.773588in}}%
\pgfpathlineto{\pgfqpoint{-2.116467in}{0.773588in}}%
\pgfpathlineto{\pgfqpoint{-2.064986in}{0.773588in}}%
\pgfpathlineto{\pgfqpoint{-2.013282in}{0.773588in}}%
\pgfpathlineto{\pgfqpoint{-1.963080in}{0.773588in}}%
\pgfpathlineto{\pgfqpoint{-1.912972in}{0.773588in}}%
\pgfpathlineto{\pgfqpoint{-1.860296in}{0.773588in}}%
\pgfpathlineto{\pgfqpoint{-1.809967in}{0.773588in}}%
\pgfpathlineto{\pgfqpoint{-1.759793in}{0.773588in}}%
\pgfpathlineto{\pgfqpoint{-1.707851in}{0.773588in}}%
\pgfpathlineto{\pgfqpoint{-1.657574in}{0.773588in}}%
\pgfpathlineto{\pgfqpoint{-1.606814in}{0.773588in}}%
\pgfpathlineto{\pgfqpoint{-1.553079in}{0.773588in}}%
\pgfpathlineto{\pgfqpoint{-1.502359in}{0.773588in}}%
\pgfpathlineto{\pgfqpoint{-1.451018in}{0.773588in}}%
\pgfpathlineto{\pgfqpoint{-1.397583in}{0.773588in}}%
\pgfpathlineto{\pgfqpoint{-1.346574in}{0.773588in}}%
\pgfpathlineto{\pgfqpoint{-1.295495in}{0.773588in}}%
\pgfpathlineto{\pgfqpoint{-1.242201in}{0.773588in}}%
\pgfpathlineto{\pgfqpoint{-1.191111in}{0.773588in}}%
\pgfpathlineto{\pgfqpoint{-1.140825in}{0.773588in}}%
\pgfpathlineto{\pgfqpoint{-1.088683in}{0.773588in}}%
\pgfpathlineto{\pgfqpoint{-1.038293in}{0.773588in}}%
\pgfpathlineto{\pgfqpoint{-0.987007in}{0.773588in}}%
\pgfpathlineto{\pgfqpoint{-0.934193in}{0.773588in}}%
\pgfpathlineto{\pgfqpoint{-0.883439in}{0.773588in}}%
\pgfpathlineto{\pgfqpoint{-0.832397in}{0.773588in}}%
\pgfpathlineto{\pgfqpoint{-0.780794in}{0.773588in}}%
\pgfpathlineto{\pgfqpoint{-0.729960in}{0.773588in}}%
\pgfpathlineto{\pgfqpoint{-0.679845in}{0.773588in}}%
\pgfpathlineto{\pgfqpoint{-0.628072in}{0.773588in}}%
\pgfpathlineto{\pgfqpoint{-0.577558in}{0.773588in}}%
\pgfpathlineto{\pgfqpoint{-0.525798in}{0.773588in}}%
\pgfpathlineto{\pgfqpoint{-0.472607in}{0.773588in}}%
\pgfpathlineto{\pgfqpoint{-0.421082in}{0.773588in}}%
\pgfpathlineto{\pgfqpoint{-0.370494in}{0.773588in}}%
\pgfpathlineto{\pgfqpoint{-0.317282in}{0.773588in}}%
\pgfpathlineto{\pgfqpoint{-0.265117in}{0.773588in}}%
\pgfpathlineto{\pgfqpoint{-0.212446in}{0.773588in}}%
\pgfpathlineto{\pgfqpoint{-0.159327in}{0.773588in}}%
\pgfpathlineto{\pgfqpoint{-0.106747in}{0.773588in}}%
\pgfpathlineto{\pgfqpoint{-0.053884in}{0.773588in}}%
\pgfpathlineto{\pgfqpoint{0.000028in}{0.773588in}}%
\pgfpathlineto{\pgfqpoint{0.052190in}{0.773588in}}%
\pgfpathlineto{\pgfqpoint{0.103930in}{0.773588in}}%
\pgfpathlineto{\pgfqpoint{0.157615in}{0.773588in}}%
\pgfpathlineto{\pgfqpoint{0.209933in}{0.773588in}}%
\pgfpathlineto{\pgfqpoint{0.261843in}{0.773588in}}%
\pgfpathlineto{\pgfqpoint{0.315264in}{0.773588in}}%
\pgfpathlineto{\pgfqpoint{0.366948in}{0.773588in}}%
\pgfpathlineto{\pgfqpoint{0.419063in}{0.773588in}}%
\pgfpathlineto{\pgfqpoint{0.472767in}{0.773588in}}%
\pgfpathlineto{\pgfqpoint{0.524412in}{0.773588in}}%
\pgfpathlineto{\pgfqpoint{0.575921in}{0.773588in}}%
\pgfpathlineto{\pgfqpoint{0.628822in}{0.773588in}}%
\pgfpathlineto{\pgfqpoint{0.680101in}{0.773588in}}%
\pgfpathlineto{\pgfqpoint{0.730937in}{0.773588in}}%
\pgfpathlineto{\pgfqpoint{0.783351in}{0.773588in}}%
\pgfpathlineto{\pgfqpoint{0.835638in}{0.773588in}}%
\pgfpathlineto{\pgfqpoint{0.890559in}{0.773588in}}%
\pgfpathlineto{\pgfqpoint{0.950896in}{0.773588in}}%
\pgfpathlineto{\pgfqpoint{1.011418in}{0.773588in}}%
\pgfpathlineto{\pgfqpoint{1.071223in}{0.773588in}}%
\pgfpathlineto{\pgfqpoint{1.133818in}{0.773588in}}%
\pgfpathlineto{\pgfqpoint{1.196854in}{0.773588in}}%
\pgfpathlineto{\pgfqpoint{1.262232in}{0.773588in}}%
\pgfpathlineto{\pgfqpoint{1.331848in}{0.773588in}}%
\pgfpathlineto{\pgfqpoint{1.399923in}{0.773588in}}%
\pgfpathlineto{\pgfqpoint{1.469539in}{0.773588in}}%
\pgfpathlineto{\pgfqpoint{1.542225in}{0.773588in}}%
\pgfpathlineto{\pgfqpoint{1.614592in}{0.773588in}}%
\pgfpathlineto{\pgfqpoint{1.687208in}{0.773588in}}%
\pgfpathlineto{\pgfqpoint{1.764147in}{0.773588in}}%
\pgfpathlineto{\pgfqpoint{1.839330in}{0.773588in}}%
\pgfpathlineto{\pgfqpoint{1.918401in}{0.773588in}}%
\pgfpathlineto{\pgfqpoint{1.999401in}{0.773588in}}%
\pgfpathlineto{\pgfqpoint{2.077252in}{0.773588in}}%
\pgfpathlineto{\pgfqpoint{2.157128in}{0.773588in}}%
\pgfpathlineto{\pgfqpoint{2.243381in}{0.773588in}}%
\pgfpathlineto{\pgfqpoint{2.328903in}{0.773588in}}%
\pgfpathlineto{\pgfqpoint{2.416579in}{0.773588in}}%
\pgfpathlineto{\pgfqpoint{2.504252in}{0.773588in}}%
\pgfpathlineto{\pgfqpoint{2.589494in}{0.773588in}}%
\pgfpathlineto{\pgfqpoint{2.676066in}{0.773588in}}%
\pgfpathlineto{\pgfqpoint{2.767061in}{0.773588in}}%
\pgfpathlineto{\pgfqpoint{2.858701in}{0.773588in}}%
\pgfpathlineto{\pgfqpoint{2.948242in}{0.773588in}}%
\pgfpathlineto{\pgfqpoint{3.043068in}{0.773588in}}%
\pgfpathlineto{\pgfqpoint{3.132459in}{0.773588in}}%
\pgfpathlineto{\pgfqpoint{3.198420in}{0.773588in}}%
\pgfpathlineto{\pgfqpoint{3.252590in}{0.773588in}}%
\pgfpathlineto{\pgfqpoint{3.305067in}{0.773588in}}%
\pgfpathlineto{\pgfqpoint{3.357008in}{0.773588in}}%
\pgfpathlineto{\pgfqpoint{3.411084in}{0.773588in}}%
\pgfpathlineto{\pgfqpoint{3.464430in}{0.773588in}}%
\pgfpathlineto{\pgfqpoint{3.517522in}{0.773588in}}%
\pgfpathlineto{\pgfqpoint{3.571722in}{0.773588in}}%
\pgfpathlineto{\pgfqpoint{3.624008in}{0.773588in}}%
\pgfpathlineto{\pgfqpoint{3.676588in}{0.773588in}}%
\pgfpathlineto{\pgfqpoint{3.729373in}{0.773588in}}%
\pgfpathlineto{\pgfqpoint{3.781554in}{0.773588in}}%
\pgfpathlineto{\pgfqpoint{3.833685in}{0.773588in}}%
\pgfpathlineto{\pgfqpoint{3.887639in}{0.773588in}}%
\pgfpathlineto{\pgfqpoint{3.940507in}{0.773588in}}%
\pgfpathlineto{\pgfqpoint{3.993337in}{0.773588in}}%
\pgfpathlineto{\pgfqpoint{4.047138in}{0.773588in}}%
\pgfpathlineto{\pgfqpoint{4.098630in}{0.773588in}}%
\pgfpathlineto{\pgfqpoint{4.150734in}{0.773588in}}%
\pgfpathlineto{\pgfqpoint{4.204360in}{0.773588in}}%
\pgfpathlineto{\pgfqpoint{4.256228in}{0.773588in}}%
\pgfpathlineto{\pgfqpoint{4.307535in}{0.773588in}}%
\pgfpathlineto{\pgfqpoint{4.360055in}{0.773588in}}%
\pgfpathlineto{\pgfqpoint{4.398305in}{0.773588in}}%
\pgfpathlineto{\pgfqpoint{4.446119in}{0.773588in}}%
\pgfpathlineto{\pgfqpoint{4.484596in}{0.773588in}}%
\pgfpathlineto{\pgfqpoint{4.526846in}{0.773588in}}%
\pgfpathlineto{\pgfqpoint{4.566686in}{0.773588in}}%
\pgfpathlineto{\pgfqpoint{4.603496in}{0.773588in}}%
\pgfpathlineto{\pgfqpoint{4.635492in}{0.773588in}}%
\pgfpathlineto{\pgfqpoint{4.666016in}{0.773588in}}%
\pgfpathlineto{\pgfqpoint{4.690368in}{0.773588in}}%
\pgfpathlineto{\pgfqpoint{4.715513in}{0.773588in}}%
\pgfpathlineto{\pgfqpoint{4.739291in}{0.773588in}}%
\pgfpathlineto{\pgfqpoint{4.764034in}{0.773588in}}%
\pgfpathlineto{\pgfqpoint{4.787716in}{0.773588in}}%
\pgfpathlineto{\pgfqpoint{4.811295in}{0.773588in}}%
\pgfpathlineto{\pgfqpoint{4.836687in}{0.773588in}}%
\pgfpathlineto{\pgfqpoint{4.860218in}{0.773588in}}%
\pgfpathlineto{\pgfqpoint{4.884852in}{0.773588in}}%
\pgfpathlineto{\pgfqpoint{4.907640in}{0.773588in}}%
\pgfpathlineto{\pgfqpoint{4.931890in}{0.773588in}}%
\pgfpathlineto{\pgfqpoint{4.954839in}{0.773588in}}%
\pgfpathlineto{\pgfqpoint{4.980174in}{0.773588in}}%
\pgfpathlineto{\pgfqpoint{5.002737in}{0.773588in}}%
\pgfpathlineto{\pgfqpoint{5.026810in}{0.773588in}}%
\pgfpathlineto{\pgfqpoint{5.051612in}{0.773588in}}%
\pgfpathlineto{\pgfqpoint{5.074798in}{0.773588in}}%
\pgfpathlineto{\pgfqpoint{5.097977in}{0.773588in}}%
\pgfpathlineto{\pgfqpoint{5.122448in}{0.773588in}}%
\pgfpathlineto{\pgfqpoint{5.145219in}{0.773588in}}%
\pgfpathlineto{\pgfqpoint{5.168068in}{0.773588in}}%
\pgfpathlineto{\pgfqpoint{5.192095in}{0.773588in}}%
\pgfpathlineto{\pgfqpoint{5.214897in}{0.773588in}}%
\pgfpathlineto{\pgfqpoint{5.237842in}{0.773588in}}%
\pgfpathlineto{\pgfqpoint{5.261861in}{0.773588in}}%
\pgfpathlineto{\pgfqpoint{5.285086in}{0.773588in}}%
\pgfpathlineto{\pgfqpoint{5.307824in}{0.773588in}}%
\pgfpathlineto{\pgfqpoint{5.331834in}{0.773588in}}%
\pgfpathlineto{\pgfqpoint{5.355254in}{0.773588in}}%
\pgfpathlineto{\pgfqpoint{5.377738in}{0.773588in}}%
\pgfpathlineto{\pgfqpoint{5.402166in}{0.773588in}}%
\pgfpathlineto{\pgfqpoint{5.424187in}{0.773588in}}%
\pgfpathlineto{\pgfqpoint{5.448001in}{0.773588in}}%
\pgfpathlineto{\pgfqpoint{5.470431in}{0.773588in}}%
\pgfpathlineto{\pgfqpoint{5.494250in}{0.773588in}}%
\pgfpathlineto{\pgfqpoint{5.516676in}{0.773588in}}%
\pgfpathlineto{\pgfqpoint{5.540659in}{0.773588in}}%
\pgfpathlineto{\pgfqpoint{5.562835in}{0.773588in}}%
\pgfpathlineto{\pgfqpoint{5.587139in}{0.773588in}}%
\pgfpathlineto{\pgfqpoint{5.609949in}{0.773588in}}%
\pgfpathlineto{\pgfqpoint{5.634039in}{0.773588in}}%
\pgfpathlineto{\pgfqpoint{5.661246in}{0.773588in}}%
\pgfpathlineto{\pgfqpoint{5.712060in}{0.773588in}}%
\pgfpathlineto{\pgfqpoint{5.763148in}{0.773588in}}%
\pgfpathlineto{\pgfqpoint{5.815025in}{0.773588in}}%
\pgfpathlineto{\pgfqpoint{5.867920in}{0.773588in}}%
\pgfpathlineto{\pgfqpoint{5.919631in}{0.773588in}}%
\pgfpathlineto{\pgfqpoint{5.971565in}{0.773588in}}%
\pgfpathlineto{\pgfqpoint{6.025764in}{0.773588in}}%
\pgfpathlineto{\pgfqpoint{6.078797in}{0.773588in}}%
\pgfpathlineto{\pgfqpoint{6.078797in}{2.413621in}}%
\pgfpathlineto{\pgfqpoint{6.078797in}{2.413621in}}%
\pgfpathlineto{\pgfqpoint{6.025764in}{2.413621in}}%
\pgfpathlineto{\pgfqpoint{5.971565in}{2.413621in}}%
\pgfpathlineto{\pgfqpoint{5.919631in}{2.413621in}}%
\pgfpathlineto{\pgfqpoint{5.867920in}{2.413621in}}%
\pgfpathlineto{\pgfqpoint{5.815025in}{2.413621in}}%
\pgfpathlineto{\pgfqpoint{5.763148in}{2.413621in}}%
\pgfpathlineto{\pgfqpoint{5.712060in}{2.413621in}}%
\pgfpathlineto{\pgfqpoint{5.661246in}{2.423078in}}%
\pgfpathlineto{\pgfqpoint{5.634039in}{2.470556in}}%
\pgfpathlineto{\pgfqpoint{5.609949in}{2.424295in}}%
\pgfpathlineto{\pgfqpoint{5.587139in}{2.403966in}}%
\pgfpathlineto{\pgfqpoint{5.562835in}{2.587240in}}%
\pgfpathlineto{\pgfqpoint{5.540659in}{2.468813in}}%
\pgfpathlineto{\pgfqpoint{5.516676in}{2.477526in}}%
\pgfpathlineto{\pgfqpoint{5.494250in}{2.522929in}}%
\pgfpathlineto{\pgfqpoint{5.470431in}{2.570004in}}%
\pgfpathlineto{\pgfqpoint{5.448001in}{2.475980in}}%
\pgfpathlineto{\pgfqpoint{5.424187in}{2.469111in}}%
\pgfpathlineto{\pgfqpoint{5.402166in}{2.457377in}}%
\pgfpathlineto{\pgfqpoint{5.377738in}{2.581556in}}%
\pgfpathlineto{\pgfqpoint{5.355254in}{2.431345in}}%
\pgfpathlineto{\pgfqpoint{5.331834in}{2.431018in}}%
\pgfpathlineto{\pgfqpoint{5.307824in}{2.578334in}}%
\pgfpathlineto{\pgfqpoint{5.285086in}{2.380016in}}%
\pgfpathlineto{\pgfqpoint{5.261861in}{2.409123in}}%
\pgfpathlineto{\pgfqpoint{5.237842in}{2.434375in}}%
\pgfpathlineto{\pgfqpoint{5.214897in}{2.526820in}}%
\pgfpathlineto{\pgfqpoint{5.192095in}{2.553834in}}%
\pgfpathlineto{\pgfqpoint{5.168068in}{2.434395in}}%
\pgfpathlineto{\pgfqpoint{5.145219in}{2.411536in}}%
\pgfpathlineto{\pgfqpoint{5.122448in}{2.282667in}}%
\pgfpathlineto{\pgfqpoint{5.097977in}{2.433623in}}%
\pgfpathlineto{\pgfqpoint{5.074798in}{2.469518in}}%
\pgfpathlineto{\pgfqpoint{5.051612in}{2.338107in}}%
\pgfpathlineto{\pgfqpoint{5.026810in}{2.232666in}}%
\pgfpathlineto{\pgfqpoint{5.002737in}{2.287340in}}%
\pgfpathlineto{\pgfqpoint{4.980174in}{2.271817in}}%
\pgfpathlineto{\pgfqpoint{4.954839in}{2.394574in}}%
\pgfpathlineto{\pgfqpoint{4.931890in}{2.449950in}}%
\pgfpathlineto{\pgfqpoint{4.907640in}{2.361353in}}%
\pgfpathlineto{\pgfqpoint{4.884852in}{2.267974in}}%
\pgfpathlineto{\pgfqpoint{4.860218in}{2.316562in}}%
\pgfpathlineto{\pgfqpoint{4.836687in}{2.257032in}}%
\pgfpathlineto{\pgfqpoint{4.811295in}{2.332444in}}%
\pgfpathlineto{\pgfqpoint{4.787716in}{2.339022in}}%
\pgfpathlineto{\pgfqpoint{4.764034in}{2.446672in}}%
\pgfpathlineto{\pgfqpoint{4.739291in}{2.290107in}}%
\pgfpathlineto{\pgfqpoint{4.715513in}{2.218952in}}%
\pgfpathlineto{\pgfqpoint{4.690368in}{2.293678in}}%
\pgfpathlineto{\pgfqpoint{4.666016in}{1.799827in}}%
\pgfpathlineto{\pgfqpoint{4.635492in}{1.503451in}}%
\pgfpathlineto{\pgfqpoint{4.603496in}{1.243236in}}%
\pgfpathlineto{\pgfqpoint{4.566686in}{1.158641in}}%
\pgfpathlineto{\pgfqpoint{4.526846in}{1.089794in}}%
\pgfpathlineto{\pgfqpoint{4.484596in}{1.025897in}}%
\pgfpathlineto{\pgfqpoint{4.446119in}{0.773588in}}%
\pgfpathlineto{\pgfqpoint{4.398305in}{0.773588in}}%
\pgfpathlineto{\pgfqpoint{4.360055in}{0.773588in}}%
\pgfpathlineto{\pgfqpoint{4.307535in}{0.773588in}}%
\pgfpathlineto{\pgfqpoint{4.256228in}{0.773588in}}%
\pgfpathlineto{\pgfqpoint{4.204360in}{0.773588in}}%
\pgfpathlineto{\pgfqpoint{4.150734in}{0.773588in}}%
\pgfpathlineto{\pgfqpoint{4.098630in}{0.773588in}}%
\pgfpathlineto{\pgfqpoint{4.047138in}{0.773588in}}%
\pgfpathlineto{\pgfqpoint{3.993337in}{0.773588in}}%
\pgfpathlineto{\pgfqpoint{3.940507in}{0.773588in}}%
\pgfpathlineto{\pgfqpoint{3.887639in}{0.773588in}}%
\pgfpathlineto{\pgfqpoint{3.833685in}{0.773588in}}%
\pgfpathlineto{\pgfqpoint{3.781554in}{0.773588in}}%
\pgfpathlineto{\pgfqpoint{3.729373in}{0.773588in}}%
\pgfpathlineto{\pgfqpoint{3.676588in}{0.773588in}}%
\pgfpathlineto{\pgfqpoint{3.624008in}{0.773588in}}%
\pgfpathlineto{\pgfqpoint{3.571722in}{0.773588in}}%
\pgfpathlineto{\pgfqpoint{3.517522in}{0.773588in}}%
\pgfpathlineto{\pgfqpoint{3.464430in}{0.773588in}}%
\pgfpathlineto{\pgfqpoint{3.411084in}{0.773588in}}%
\pgfpathlineto{\pgfqpoint{3.357008in}{0.773588in}}%
\pgfpathlineto{\pgfqpoint{3.305067in}{0.773588in}}%
\pgfpathlineto{\pgfqpoint{3.252590in}{0.773588in}}%
\pgfpathlineto{\pgfqpoint{3.198420in}{0.773588in}}%
\pgfpathlineto{\pgfqpoint{3.132459in}{0.773588in}}%
\pgfpathlineto{\pgfqpoint{3.043068in}{0.773588in}}%
\pgfpathlineto{\pgfqpoint{2.948242in}{0.773588in}}%
\pgfpathlineto{\pgfqpoint{2.858701in}{0.773588in}}%
\pgfpathlineto{\pgfqpoint{2.767061in}{0.773588in}}%
\pgfpathlineto{\pgfqpoint{2.676066in}{0.773588in}}%
\pgfpathlineto{\pgfqpoint{2.589494in}{0.773588in}}%
\pgfpathlineto{\pgfqpoint{2.504252in}{0.773588in}}%
\pgfpathlineto{\pgfqpoint{2.416579in}{0.773588in}}%
\pgfpathlineto{\pgfqpoint{2.328903in}{0.773588in}}%
\pgfpathlineto{\pgfqpoint{2.243381in}{0.773588in}}%
\pgfpathlineto{\pgfqpoint{2.157128in}{0.773588in}}%
\pgfpathlineto{\pgfqpoint{2.077252in}{0.773588in}}%
\pgfpathlineto{\pgfqpoint{1.999401in}{0.773588in}}%
\pgfpathlineto{\pgfqpoint{1.918401in}{0.773588in}}%
\pgfpathlineto{\pgfqpoint{1.839330in}{0.773588in}}%
\pgfpathlineto{\pgfqpoint{1.764147in}{0.773588in}}%
\pgfpathlineto{\pgfqpoint{1.687208in}{0.773588in}}%
\pgfpathlineto{\pgfqpoint{1.614592in}{0.773588in}}%
\pgfpathlineto{\pgfqpoint{1.542225in}{0.773588in}}%
\pgfpathlineto{\pgfqpoint{1.469539in}{0.773588in}}%
\pgfpathlineto{\pgfqpoint{1.399923in}{0.773588in}}%
\pgfpathlineto{\pgfqpoint{1.331848in}{0.773588in}}%
\pgfpathlineto{\pgfqpoint{1.262232in}{0.773588in}}%
\pgfpathlineto{\pgfqpoint{1.196854in}{0.773588in}}%
\pgfpathlineto{\pgfqpoint{1.133818in}{0.773588in}}%
\pgfpathlineto{\pgfqpoint{1.071223in}{0.773588in}}%
\pgfpathlineto{\pgfqpoint{1.011418in}{0.773588in}}%
\pgfpathlineto{\pgfqpoint{0.950896in}{0.773588in}}%
\pgfpathlineto{\pgfqpoint{0.890559in}{0.773588in}}%
\pgfpathlineto{\pgfqpoint{0.835638in}{0.773588in}}%
\pgfpathlineto{\pgfqpoint{0.783351in}{0.773588in}}%
\pgfpathlineto{\pgfqpoint{0.730937in}{0.773588in}}%
\pgfpathlineto{\pgfqpoint{0.680101in}{0.773588in}}%
\pgfpathlineto{\pgfqpoint{0.628822in}{0.773588in}}%
\pgfpathlineto{\pgfqpoint{0.575921in}{0.773588in}}%
\pgfpathlineto{\pgfqpoint{0.524412in}{0.773588in}}%
\pgfpathlineto{\pgfqpoint{0.472767in}{0.773588in}}%
\pgfpathlineto{\pgfqpoint{0.419063in}{0.773588in}}%
\pgfpathlineto{\pgfqpoint{0.366948in}{0.773588in}}%
\pgfpathlineto{\pgfqpoint{0.315264in}{0.773588in}}%
\pgfpathlineto{\pgfqpoint{0.261843in}{0.773588in}}%
\pgfpathlineto{\pgfqpoint{0.209933in}{0.773588in}}%
\pgfpathlineto{\pgfqpoint{0.157615in}{0.773588in}}%
\pgfpathlineto{\pgfqpoint{0.103930in}{0.773588in}}%
\pgfpathlineto{\pgfqpoint{0.052190in}{0.773588in}}%
\pgfpathlineto{\pgfqpoint{0.000028in}{0.773588in}}%
\pgfpathlineto{\pgfqpoint{-0.053884in}{0.773588in}}%
\pgfpathlineto{\pgfqpoint{-0.106747in}{0.773588in}}%
\pgfpathlineto{\pgfqpoint{-0.159327in}{0.773588in}}%
\pgfpathlineto{\pgfqpoint{-0.212446in}{0.773588in}}%
\pgfpathlineto{\pgfqpoint{-0.265117in}{0.773588in}}%
\pgfpathlineto{\pgfqpoint{-0.317282in}{0.773588in}}%
\pgfpathlineto{\pgfqpoint{-0.370494in}{0.773588in}}%
\pgfpathlineto{\pgfqpoint{-0.421082in}{0.773588in}}%
\pgfpathlineto{\pgfqpoint{-0.472607in}{0.773588in}}%
\pgfpathlineto{\pgfqpoint{-0.525798in}{0.773588in}}%
\pgfpathlineto{\pgfqpoint{-0.577558in}{0.773588in}}%
\pgfpathlineto{\pgfqpoint{-0.628072in}{0.773588in}}%
\pgfpathlineto{\pgfqpoint{-0.679845in}{0.773588in}}%
\pgfpathlineto{\pgfqpoint{-0.729960in}{0.773588in}}%
\pgfpathlineto{\pgfqpoint{-0.780794in}{0.773588in}}%
\pgfpathlineto{\pgfqpoint{-0.832397in}{0.773588in}}%
\pgfpathlineto{\pgfqpoint{-0.883439in}{0.773588in}}%
\pgfpathlineto{\pgfqpoint{-0.934193in}{0.773588in}}%
\pgfpathlineto{\pgfqpoint{-0.987007in}{0.773588in}}%
\pgfpathlineto{\pgfqpoint{-1.038293in}{0.773588in}}%
\pgfpathlineto{\pgfqpoint{-1.088683in}{0.773588in}}%
\pgfpathlineto{\pgfqpoint{-1.140825in}{0.773588in}}%
\pgfpathlineto{\pgfqpoint{-1.191111in}{0.773588in}}%
\pgfpathlineto{\pgfqpoint{-1.242201in}{0.773588in}}%
\pgfpathlineto{\pgfqpoint{-1.295495in}{0.773588in}}%
\pgfpathlineto{\pgfqpoint{-1.346574in}{0.773588in}}%
\pgfpathlineto{\pgfqpoint{-1.397583in}{0.773588in}}%
\pgfpathlineto{\pgfqpoint{-1.451018in}{0.773588in}}%
\pgfpathlineto{\pgfqpoint{-1.502359in}{0.773588in}}%
\pgfpathlineto{\pgfqpoint{-1.553079in}{0.773588in}}%
\pgfpathlineto{\pgfqpoint{-1.606814in}{0.773588in}}%
\pgfpathlineto{\pgfqpoint{-1.657574in}{0.773588in}}%
\pgfpathlineto{\pgfqpoint{-1.707851in}{0.773588in}}%
\pgfpathlineto{\pgfqpoint{-1.759793in}{0.773588in}}%
\pgfpathlineto{\pgfqpoint{-1.809967in}{0.773588in}}%
\pgfpathlineto{\pgfqpoint{-1.860296in}{0.773588in}}%
\pgfpathlineto{\pgfqpoint{-1.912972in}{0.773588in}}%
\pgfpathlineto{\pgfqpoint{-1.963080in}{0.773588in}}%
\pgfpathlineto{\pgfqpoint{-2.013282in}{0.773588in}}%
\pgfpathlineto{\pgfqpoint{-2.064986in}{0.773588in}}%
\pgfpathlineto{\pgfqpoint{-2.116467in}{0.773588in}}%
\pgfpathlineto{\pgfqpoint{-2.167559in}{0.773588in}}%
\pgfpathlineto{\pgfqpoint{-2.220871in}{0.773588in}}%
\pgfpathlineto{\pgfqpoint{-2.272195in}{0.773588in}}%
\pgfpathlineto{\pgfqpoint{-2.323084in}{0.773588in}}%
\pgfpathlineto{\pgfqpoint{-2.375607in}{0.773588in}}%
\pgfpathlineto{\pgfqpoint{-2.425397in}{0.773588in}}%
\pgfpathlineto{\pgfqpoint{-2.476616in}{0.773588in}}%
\pgfpathlineto{\pgfqpoint{-2.528440in}{0.773588in}}%
\pgfpathlineto{\pgfqpoint{-2.578972in}{0.773588in}}%
\pgfpathlineto{\pgfqpoint{-2.629697in}{0.773588in}}%
\pgfpathlineto{\pgfqpoint{-2.681969in}{0.773588in}}%
\pgfpathlineto{\pgfqpoint{-2.732985in}{0.773588in}}%
\pgfpathlineto{\pgfqpoint{-2.784277in}{0.773588in}}%
\pgfpathlineto{\pgfqpoint{-2.836699in}{0.773588in}}%
\pgfpathlineto{\pgfqpoint{-2.887081in}{0.773588in}}%
\pgfpathlineto{\pgfqpoint{-2.938490in}{0.773588in}}%
\pgfpathlineto{\pgfqpoint{-2.990316in}{0.773588in}}%
\pgfpathlineto{\pgfqpoint{-3.039557in}{0.773588in}}%
\pgfpathlineto{\pgfqpoint{-3.088627in}{0.773588in}}%
\pgfpathlineto{\pgfqpoint{-3.140428in}{0.773588in}}%
\pgfpathlineto{\pgfqpoint{-3.190882in}{0.773588in}}%
\pgfpathlineto{\pgfqpoint{-3.241308in}{0.773588in}}%
\pgfpathlineto{\pgfqpoint{-3.292250in}{0.773588in}}%
\pgfpathlineto{\pgfqpoint{-3.341930in}{0.773588in}}%
\pgfpathlineto{\pgfqpoint{-3.392015in}{0.773588in}}%
\pgfpathlineto{\pgfqpoint{-3.444140in}{0.773588in}}%
\pgfpathlineto{\pgfqpoint{-3.494154in}{0.773588in}}%
\pgfpathlineto{\pgfqpoint{-3.544176in}{0.773588in}}%
\pgfpathlineto{\pgfqpoint{-3.595859in}{0.773588in}}%
\pgfpathlineto{\pgfqpoint{-3.646072in}{0.773588in}}%
\pgfpathlineto{\pgfqpoint{-3.696993in}{0.773588in}}%
\pgfpathlineto{\pgfqpoint{-3.748633in}{0.773588in}}%
\pgfpathlineto{\pgfqpoint{-3.799759in}{0.773588in}}%
\pgfpathlineto{\pgfqpoint{-3.850587in}{0.773588in}}%
\pgfpathlineto{\pgfqpoint{-3.902764in}{0.773588in}}%
\pgfpathlineto{\pgfqpoint{-3.952191in}{0.773588in}}%
\pgfpathlineto{\pgfqpoint{-4.002524in}{0.773588in}}%
\pgfpathlineto{\pgfqpoint{-4.053537in}{0.773588in}}%
\pgfpathlineto{\pgfqpoint{-4.103923in}{0.773588in}}%
\pgfpathlineto{\pgfqpoint{-4.153202in}{0.773588in}}%
\pgfpathlineto{\pgfqpoint{-4.204591in}{0.773588in}}%
\pgfpathlineto{\pgfqpoint{-4.255103in}{0.773588in}}%
\pgfpathlineto{\pgfqpoint{-4.305607in}{0.773588in}}%
\pgfpathlineto{\pgfqpoint{-4.357515in}{0.773588in}}%
\pgfpathlineto{\pgfqpoint{-4.407934in}{0.773588in}}%
\pgfpathlineto{\pgfqpoint{-4.457276in}{0.773588in}}%
\pgfpathlineto{\pgfqpoint{-4.508002in}{0.773588in}}%
\pgfpathlineto{\pgfqpoint{-4.558006in}{0.773588in}}%
\pgfpathlineto{\pgfqpoint{-4.607841in}{0.773588in}}%
\pgfpathlineto{\pgfqpoint{-4.659647in}{0.773588in}}%
\pgfpathlineto{\pgfqpoint{-4.709756in}{0.773588in}}%
\pgfpathlineto{\pgfqpoint{-4.759589in}{0.773588in}}%
\pgfpathlineto{\pgfqpoint{-4.812147in}{0.773588in}}%
\pgfpathlineto{\pgfqpoint{-4.861887in}{0.773588in}}%
\pgfpathlineto{\pgfqpoint{-4.911765in}{0.773588in}}%
\pgfpathlineto{\pgfqpoint{-4.963670in}{0.773588in}}%
\pgfpathlineto{\pgfqpoint{-5.014255in}{0.773588in}}%
\pgfpathlineto{\pgfqpoint{-5.064238in}{0.773588in}}%
\pgfpathlineto{\pgfqpoint{-5.114869in}{0.773588in}}%
\pgfpathlineto{\pgfqpoint{-5.165135in}{0.773588in}}%
\pgfpathlineto{\pgfqpoint{-5.215272in}{0.773588in}}%
\pgfpathlineto{\pgfqpoint{-5.267508in}{0.773588in}}%
\pgfpathlineto{\pgfqpoint{-5.319212in}{0.773588in}}%
\pgfpathlineto{\pgfqpoint{-5.369703in}{0.773588in}}%
\pgfpathlineto{\pgfqpoint{-5.421631in}{0.773588in}}%
\pgfpathlineto{\pgfqpoint{-5.473058in}{0.773588in}}%
\pgfpathlineto{\pgfqpoint{-5.523472in}{0.773588in}}%
\pgfpathlineto{\pgfqpoint{-5.574633in}{0.773588in}}%
\pgfpathlineto{\pgfqpoint{-5.623816in}{0.773588in}}%
\pgfpathlineto{\pgfqpoint{-5.673071in}{0.773588in}}%
\pgfpathlineto{\pgfqpoint{-5.724570in}{0.773588in}}%
\pgfpathlineto{\pgfqpoint{-5.774223in}{0.773588in}}%
\pgfpathlineto{\pgfqpoint{-5.822716in}{0.773588in}}%
\pgfpathlineto{\pgfqpoint{-5.874278in}{0.773588in}}%
\pgfpathlineto{\pgfqpoint{-5.924128in}{0.773588in}}%
\pgfpathlineto{\pgfqpoint{-5.974236in}{0.773588in}}%
\pgfpathlineto{\pgfqpoint{-6.024836in}{0.773588in}}%
\pgfpathlineto{\pgfqpoint{-6.075227in}{0.773588in}}%
\pgfpathlineto{\pgfqpoint{-6.125420in}{0.773588in}}%
\pgfpathlineto{\pgfqpoint{-6.177451in}{0.773588in}}%
\pgfpathlineto{\pgfqpoint{-6.228389in}{0.773588in}}%
\pgfpathlineto{\pgfqpoint{-6.279011in}{0.773588in}}%
\pgfpathlineto{\pgfqpoint{-6.330959in}{0.773588in}}%
\pgfpathlineto{\pgfqpoint{-6.380124in}{0.773588in}}%
\pgfpathlineto{\pgfqpoint{-6.429787in}{0.773588in}}%
\pgfpathlineto{\pgfqpoint{-6.481910in}{0.773588in}}%
\pgfpathlineto{\pgfqpoint{-6.532179in}{0.773588in}}%
\pgfpathlineto{\pgfqpoint{-6.582317in}{0.773588in}}%
\pgfpathlineto{\pgfqpoint{-6.634445in}{0.773588in}}%
\pgfpathlineto{\pgfqpoint{-6.682747in}{0.773588in}}%
\pgfpathlineto{\pgfqpoint{-6.731323in}{0.773588in}}%
\pgfpathlineto{\pgfqpoint{-6.782488in}{0.773588in}}%
\pgfpathlineto{\pgfqpoint{-6.832072in}{0.773588in}}%
\pgfpathlineto{\pgfqpoint{-6.882593in}{0.773588in}}%
\pgfpathlineto{\pgfqpoint{-6.934169in}{0.773588in}}%
\pgfpathlineto{\pgfqpoint{-6.984061in}{0.773588in}}%
\pgfpathlineto{\pgfqpoint{-7.033318in}{0.773588in}}%
\pgfpathlineto{\pgfqpoint{-7.085212in}{0.773588in}}%
\pgfpathlineto{\pgfqpoint{-7.135511in}{0.773588in}}%
\pgfpathlineto{\pgfqpoint{-7.185558in}{0.773588in}}%
\pgfpathlineto{\pgfqpoint{-7.237328in}{0.773588in}}%
\pgfpathlineto{\pgfqpoint{-7.286238in}{0.773588in}}%
\pgfpathlineto{\pgfqpoint{-7.335791in}{0.773588in}}%
\pgfpathlineto{\pgfqpoint{-7.386343in}{0.773588in}}%
\pgfpathlineto{\pgfqpoint{-7.435602in}{0.773588in}}%
\pgfpathlineto{\pgfqpoint{-7.485312in}{0.773588in}}%
\pgfpathlineto{\pgfqpoint{-7.536566in}{0.773588in}}%
\pgfpathlineto{\pgfqpoint{-7.585672in}{0.773588in}}%
\pgfpathlineto{\pgfqpoint{-7.634521in}{0.773588in}}%
\pgfpathlineto{\pgfqpoint{-7.685281in}{0.773588in}}%
\pgfpathlineto{\pgfqpoint{-7.733870in}{0.773588in}}%
\pgfpathlineto{\pgfqpoint{-7.783607in}{0.773588in}}%
\pgfpathlineto{\pgfqpoint{-7.835188in}{0.773588in}}%
\pgfpathlineto{\pgfqpoint{-7.885122in}{0.773588in}}%
\pgfpathlineto{\pgfqpoint{-7.935315in}{0.773588in}}%
\pgfpathlineto{\pgfqpoint{-7.986412in}{0.773588in}}%
\pgfpathlineto{\pgfqpoint{-8.035792in}{0.773588in}}%
\pgfpathlineto{\pgfqpoint{-8.085190in}{0.773588in}}%
\pgfpathlineto{\pgfqpoint{-8.135490in}{0.773588in}}%
\pgfpathlineto{\pgfqpoint{-8.184462in}{0.773588in}}%
\pgfpathlineto{\pgfqpoint{-8.233427in}{0.773588in}}%
\pgfpathlineto{\pgfqpoint{-8.283775in}{0.773588in}}%
\pgfpathlineto{\pgfqpoint{-8.332630in}{0.773588in}}%
\pgfpathlineto{\pgfqpoint{-8.382452in}{0.773588in}}%
\pgfpathlineto{\pgfqpoint{-8.432560in}{0.773588in}}%
\pgfpathlineto{\pgfqpoint{-8.481981in}{0.773588in}}%
\pgfpathlineto{\pgfqpoint{-8.532027in}{0.773588in}}%
\pgfpathlineto{\pgfqpoint{-8.583181in}{0.773588in}}%
\pgfpathlineto{\pgfqpoint{-8.632588in}{0.773588in}}%
\pgfpathlineto{\pgfqpoint{-8.681529in}{0.773588in}}%
\pgfpathlineto{\pgfqpoint{-8.731477in}{0.773588in}}%
\pgfpathlineto{\pgfqpoint{-8.780578in}{0.773588in}}%
\pgfpathlineto{\pgfqpoint{-8.829855in}{0.773588in}}%
\pgfpathlineto{\pgfqpoint{-8.880937in}{0.773588in}}%
\pgfpathlineto{\pgfqpoint{-8.930541in}{0.773588in}}%
\pgfpathlineto{\pgfqpoint{-8.979731in}{0.773588in}}%
\pgfpathlineto{\pgfqpoint{-9.030683in}{0.773588in}}%
\pgfpathlineto{\pgfqpoint{-9.080221in}{0.773588in}}%
\pgfpathlineto{\pgfqpoint{-9.130552in}{0.773588in}}%
\pgfpathlineto{\pgfqpoint{-9.182531in}{0.773588in}}%
\pgfpathlineto{\pgfqpoint{-9.232466in}{0.773588in}}%
\pgfpathlineto{\pgfqpoint{-9.281795in}{0.773588in}}%
\pgfpathlineto{\pgfqpoint{-9.332833in}{0.773588in}}%
\pgfpathlineto{\pgfqpoint{-9.382453in}{0.773588in}}%
\pgfpathlineto{\pgfqpoint{-9.430893in}{0.773588in}}%
\pgfpathlineto{\pgfqpoint{-9.479530in}{0.773588in}}%
\pgfpathlineto{\pgfqpoint{-9.527213in}{0.773588in}}%
\pgfpathlineto{\pgfqpoint{-9.575433in}{0.773588in}}%
\pgfpathlineto{\pgfqpoint{-9.624467in}{0.773588in}}%
\pgfpathlineto{\pgfqpoint{-9.671880in}{0.773588in}}%
\pgfpathlineto{\pgfqpoint{-9.719205in}{0.773588in}}%
\pgfpathlineto{\pgfqpoint{-9.768438in}{0.773588in}}%
\pgfpathlineto{\pgfqpoint{-9.816398in}{0.773588in}}%
\pgfpathlineto{\pgfqpoint{-9.863853in}{0.773588in}}%
\pgfpathlineto{\pgfqpoint{-9.912991in}{0.773588in}}%
\pgfpathlineto{\pgfqpoint{-9.960967in}{0.773588in}}%
\pgfpathlineto{\pgfqpoint{-10.009255in}{0.773588in}}%
\pgfpathlineto{\pgfqpoint{-10.058316in}{0.773588in}}%
\pgfpathlineto{\pgfqpoint{-10.106404in}{0.773588in}}%
\pgfpathlineto{\pgfqpoint{-10.154319in}{0.773588in}}%
\pgfpathlineto{\pgfqpoint{-10.204843in}{0.773588in}}%
\pgfpathlineto{\pgfqpoint{-10.253421in}{0.773588in}}%
\pgfpathlineto{\pgfqpoint{-10.301938in}{0.773588in}}%
\pgfpathlineto{\pgfqpoint{-10.352351in}{0.773588in}}%
\pgfpathlineto{\pgfqpoint{-10.401144in}{0.773588in}}%
\pgfpathlineto{\pgfqpoint{-10.448474in}{0.773588in}}%
\pgfpathlineto{\pgfqpoint{-10.497957in}{0.773588in}}%
\pgfpathlineto{\pgfqpoint{-10.546833in}{0.773588in}}%
\pgfpathlineto{\pgfqpoint{-10.594934in}{0.773588in}}%
\pgfpathlineto{\pgfqpoint{-10.644481in}{0.773588in}}%
\pgfpathlineto{\pgfqpoint{-10.693023in}{0.773588in}}%
\pgfpathlineto{\pgfqpoint{-10.741507in}{0.773588in}}%
\pgfpathlineto{\pgfqpoint{-10.790422in}{0.773588in}}%
\pgfpathlineto{\pgfqpoint{-10.837686in}{0.773588in}}%
\pgfpathlineto{\pgfqpoint{-10.885230in}{0.773588in}}%
\pgfpathlineto{\pgfqpoint{-10.933861in}{0.773588in}}%
\pgfpathlineto{\pgfqpoint{-10.980903in}{0.773588in}}%
\pgfpathlineto{\pgfqpoint{-11.028607in}{0.773588in}}%
\pgfpathlineto{\pgfqpoint{-11.077704in}{0.773588in}}%
\pgfpathlineto{\pgfqpoint{-11.125274in}{0.773588in}}%
\pgfpathlineto{\pgfqpoint{-11.172812in}{0.773588in}}%
\pgfpathlineto{\pgfqpoint{-11.221741in}{0.773588in}}%
\pgfpathlineto{\pgfqpoint{-11.268838in}{0.773588in}}%
\pgfpathlineto{\pgfqpoint{-11.316622in}{0.773588in}}%
\pgfpathlineto{\pgfqpoint{-11.366120in}{0.773588in}}%
\pgfpathlineto{\pgfqpoint{-11.415235in}{0.773588in}}%
\pgfpathlineto{\pgfqpoint{-11.463260in}{0.773588in}}%
\pgfpathlineto{\pgfqpoint{-11.511546in}{0.773588in}}%
\pgfpathlineto{\pgfqpoint{-11.559336in}{0.773588in}}%
\pgfpathlineto{\pgfqpoint{-11.607131in}{0.773588in}}%
\pgfpathlineto{\pgfqpoint{-11.656418in}{0.773588in}}%
\pgfpathlineto{\pgfqpoint{-11.704231in}{0.773588in}}%
\pgfpathlineto{\pgfqpoint{-11.751492in}{0.773588in}}%
\pgfpathlineto{\pgfqpoint{-11.800403in}{0.773588in}}%
\pgfpathlineto{\pgfqpoint{-11.847736in}{0.773588in}}%
\pgfpathlineto{\pgfqpoint{-11.895021in}{0.773588in}}%
\pgfpathlineto{\pgfqpoint{-11.943783in}{0.773588in}}%
\pgfpathlineto{\pgfqpoint{-11.991431in}{0.773588in}}%
\pgfpathlineto{\pgfqpoint{-12.038569in}{0.773588in}}%
\pgfpathlineto{\pgfqpoint{-12.087205in}{0.773588in}}%
\pgfpathlineto{\pgfqpoint{-12.135286in}{0.773588in}}%
\pgfpathlineto{\pgfqpoint{-12.182471in}{0.773588in}}%
\pgfpathlineto{\pgfqpoint{-12.231638in}{0.773588in}}%
\pgfpathlineto{\pgfqpoint{-12.278675in}{0.773588in}}%
\pgfpathlineto{\pgfqpoint{-12.326627in}{0.773588in}}%
\pgfpathlineto{\pgfqpoint{-12.375560in}{0.773588in}}%
\pgfpathlineto{\pgfqpoint{-12.422612in}{0.773588in}}%
\pgfpathlineto{\pgfqpoint{-12.469843in}{0.773588in}}%
\pgfpathlineto{\pgfqpoint{-12.517880in}{0.773588in}}%
\pgfpathlineto{\pgfqpoint{-12.565107in}{0.773588in}}%
\pgfpathlineto{\pgfqpoint{-12.612433in}{0.773588in}}%
\pgfpathlineto{\pgfqpoint{-12.661578in}{0.773588in}}%
\pgfpathlineto{\pgfqpoint{-12.708579in}{0.773588in}}%
\pgfpathlineto{\pgfqpoint{-12.755866in}{0.773588in}}%
\pgfpathlineto{\pgfqpoint{-12.805010in}{0.773588in}}%
\pgfpathlineto{\pgfqpoint{-12.852526in}{0.773588in}}%
\pgfpathlineto{\pgfqpoint{-12.899585in}{0.773588in}}%
\pgfpathlineto{\pgfqpoint{-12.949523in}{0.773588in}}%
\pgfpathlineto{\pgfqpoint{-12.998480in}{0.773588in}}%
\pgfpathlineto{\pgfqpoint{-13.046608in}{0.773588in}}%
\pgfpathlineto{\pgfqpoint{-13.096328in}{0.773588in}}%
\pgfpathlineto{\pgfqpoint{-13.144068in}{0.773588in}}%
\pgfpathlineto{\pgfqpoint{-13.190970in}{0.773588in}}%
\pgfpathlineto{\pgfqpoint{-13.239807in}{0.773588in}}%
\pgfpathlineto{\pgfqpoint{-13.287448in}{0.773588in}}%
\pgfpathlineto{\pgfqpoint{-13.334326in}{0.773588in}}%
\pgfpathlineto{\pgfqpoint{-13.382220in}{0.773588in}}%
\pgfpathlineto{\pgfqpoint{-13.428346in}{0.773588in}}%
\pgfpathlineto{\pgfqpoint{-13.474335in}{0.773588in}}%
\pgfpathlineto{\pgfqpoint{-13.522553in}{0.773588in}}%
\pgfpathlineto{\pgfqpoint{-13.568873in}{0.773588in}}%
\pgfpathlineto{\pgfqpoint{-13.615991in}{0.773588in}}%
\pgfpathlineto{\pgfqpoint{-13.663331in}{0.773588in}}%
\pgfpathlineto{\pgfqpoint{-13.710126in}{0.773588in}}%
\pgfpathlineto{\pgfqpoint{-13.757485in}{0.773588in}}%
\pgfpathlineto{\pgfqpoint{-13.806330in}{0.773588in}}%
\pgfpathlineto{\pgfqpoint{-13.853816in}{0.773588in}}%
\pgfpathlineto{\pgfqpoint{-13.901287in}{0.773588in}}%
\pgfpathlineto{\pgfqpoint{-13.950094in}{0.773588in}}%
\pgfpathlineto{\pgfqpoint{-13.996250in}{0.773588in}}%
\pgfpathlineto{\pgfqpoint{-14.042263in}{0.773588in}}%
\pgfpathlineto{\pgfqpoint{-14.090279in}{0.773588in}}%
\pgfpathlineto{\pgfqpoint{-14.137283in}{0.773588in}}%
\pgfpathlineto{\pgfqpoint{-14.184689in}{0.773588in}}%
\pgfpathlineto{\pgfqpoint{-14.233487in}{0.773588in}}%
\pgfpathlineto{\pgfqpoint{-14.280667in}{0.773588in}}%
\pgfpathlineto{\pgfqpoint{-14.327371in}{0.773588in}}%
\pgfpathlineto{\pgfqpoint{-14.375931in}{0.773588in}}%
\pgfpathlineto{\pgfqpoint{-14.422978in}{0.773588in}}%
\pgfpathlineto{\pgfqpoint{-14.469907in}{0.773588in}}%
\pgfpathlineto{\pgfqpoint{-14.517904in}{0.773588in}}%
\pgfpathlineto{\pgfqpoint{-14.564671in}{0.773588in}}%
\pgfpathlineto{\pgfqpoint{-14.611394in}{0.773588in}}%
\pgfpathlineto{\pgfqpoint{-14.659388in}{0.773588in}}%
\pgfpathlineto{\pgfqpoint{-14.705919in}{0.773588in}}%
\pgfpathlineto{\pgfqpoint{-14.752153in}{0.773588in}}%
\pgfpathlineto{\pgfqpoint{-14.800669in}{0.773588in}}%
\pgfpathlineto{\pgfqpoint{-14.847516in}{0.773588in}}%
\pgfpathlineto{\pgfqpoint{-14.894419in}{0.773588in}}%
\pgfpathlineto{\pgfqpoint{-14.942416in}{0.773588in}}%
\pgfpathlineto{\pgfqpoint{-14.989019in}{0.773588in}}%
\pgfpathlineto{\pgfqpoint{-15.035921in}{0.773588in}}%
\pgfpathlineto{\pgfqpoint{-15.083728in}{0.773588in}}%
\pgfpathlineto{\pgfqpoint{-15.130553in}{0.773588in}}%
\pgfpathlineto{\pgfqpoint{-15.176769in}{0.773588in}}%
\pgfpathlineto{\pgfqpoint{-15.225998in}{0.773588in}}%
\pgfpathlineto{\pgfqpoint{-15.273998in}{0.773588in}}%
\pgfpathlineto{\pgfqpoint{-15.321194in}{0.773588in}}%
\pgfpathlineto{\pgfqpoint{-15.369802in}{0.773588in}}%
\pgfpathlineto{\pgfqpoint{-15.416792in}{0.773588in}}%
\pgfpathlineto{\pgfqpoint{-15.463539in}{0.773588in}}%
\pgfpathlineto{\pgfqpoint{-15.512535in}{0.773588in}}%
\pgfpathlineto{\pgfqpoint{-15.560849in}{0.773588in}}%
\pgfpathlineto{\pgfqpoint{-15.608704in}{0.773588in}}%
\pgfpathlineto{\pgfqpoint{-15.657002in}{0.773588in}}%
\pgfpathlineto{\pgfqpoint{-15.703994in}{0.773588in}}%
\pgfpathlineto{\pgfqpoint{-15.750800in}{0.773588in}}%
\pgfpathlineto{\pgfqpoint{-15.798897in}{0.773588in}}%
\pgfpathlineto{\pgfqpoint{-15.844500in}{0.773588in}}%
\pgfpathlineto{\pgfqpoint{-15.890419in}{0.773588in}}%
\pgfpathlineto{\pgfqpoint{-15.938076in}{0.773588in}}%
\pgfpathlineto{\pgfqpoint{-15.984286in}{0.773588in}}%
\pgfpathlineto{\pgfqpoint{-16.029628in}{0.773588in}}%
\pgfpathlineto{\pgfqpoint{-16.077171in}{0.773588in}}%
\pgfpathlineto{\pgfqpoint{-16.123165in}{0.773588in}}%
\pgfpathlineto{\pgfqpoint{-16.169427in}{0.773588in}}%
\pgfpathlineto{\pgfqpoint{-16.217070in}{0.773588in}}%
\pgfpathlineto{\pgfqpoint{-16.262806in}{0.773588in}}%
\pgfpathlineto{\pgfqpoint{-16.308871in}{0.773588in}}%
\pgfpathlineto{\pgfqpoint{-16.355646in}{0.773588in}}%
\pgfpathlineto{\pgfqpoint{-16.402550in}{0.773588in}}%
\pgfpathlineto{\pgfqpoint{-16.449347in}{0.773588in}}%
\pgfpathlineto{\pgfqpoint{-16.496157in}{0.773588in}}%
\pgfpathlineto{\pgfqpoint{-16.542990in}{0.773588in}}%
\pgfpathlineto{\pgfqpoint{-16.589239in}{0.773588in}}%
\pgfpathlineto{\pgfqpoint{-16.637120in}{0.773588in}}%
\pgfpathlineto{\pgfqpoint{-16.684310in}{0.773588in}}%
\pgfpathlineto{\pgfqpoint{-16.730721in}{0.773588in}}%
\pgfpathlineto{\pgfqpoint{-16.778530in}{0.773588in}}%
\pgfpathlineto{\pgfqpoint{-16.825434in}{0.773588in}}%
\pgfpathlineto{\pgfqpoint{-16.871664in}{0.773588in}}%
\pgfpathlineto{\pgfqpoint{-16.918692in}{0.773588in}}%
\pgfpathlineto{\pgfqpoint{-16.964774in}{0.773588in}}%
\pgfpathlineto{\pgfqpoint{-17.010884in}{0.773588in}}%
\pgfpathlineto{\pgfqpoint{-17.058009in}{0.773588in}}%
\pgfpathlineto{\pgfqpoint{-17.103874in}{0.773588in}}%
\pgfpathlineto{\pgfqpoint{-17.150161in}{0.773588in}}%
\pgfpathlineto{\pgfqpoint{-17.198221in}{0.773588in}}%
\pgfpathlineto{\pgfqpoint{-17.244258in}{0.773588in}}%
\pgfpathlineto{\pgfqpoint{-17.289523in}{0.773588in}}%
\pgfpathlineto{\pgfqpoint{-17.336276in}{0.773588in}}%
\pgfpathlineto{\pgfqpoint{-17.381732in}{0.773588in}}%
\pgfpathlineto{\pgfqpoint{-17.427002in}{0.773588in}}%
\pgfpathlineto{\pgfqpoint{-17.474644in}{0.773588in}}%
\pgfpathlineto{\pgfqpoint{-17.521284in}{0.773588in}}%
\pgfpathlineto{\pgfqpoint{-17.566739in}{0.773588in}}%
\pgfpathlineto{\pgfqpoint{-17.613251in}{0.773588in}}%
\pgfpathlineto{\pgfqpoint{-17.658538in}{0.773588in}}%
\pgfpathlineto{\pgfqpoint{-17.704093in}{0.773588in}}%
\pgfpathlineto{\pgfqpoint{-17.751556in}{0.773588in}}%
\pgfpathlineto{\pgfqpoint{-17.797346in}{0.773588in}}%
\pgfpathlineto{\pgfqpoint{-17.843480in}{0.773588in}}%
\pgfpathlineto{\pgfqpoint{-17.891271in}{0.773588in}}%
\pgfpathlineto{\pgfqpoint{-17.937614in}{0.773588in}}%
\pgfpathlineto{\pgfqpoint{-17.984102in}{0.773588in}}%
\pgfpathlineto{\pgfqpoint{-18.032389in}{0.773588in}}%
\pgfpathlineto{\pgfqpoint{-18.078784in}{0.773588in}}%
\pgfpathlineto{\pgfqpoint{-18.124795in}{0.773588in}}%
\pgfpathlineto{\pgfqpoint{-18.172791in}{0.773588in}}%
\pgfpathlineto{\pgfqpoint{-18.218674in}{0.773588in}}%
\pgfpathlineto{\pgfqpoint{-18.264167in}{0.773588in}}%
\pgfpathlineto{\pgfqpoint{-18.311512in}{0.773588in}}%
\pgfpathlineto{\pgfqpoint{-18.356748in}{0.773588in}}%
\pgfpathlineto{\pgfqpoint{-18.402000in}{0.773588in}}%
\pgfpathlineto{\pgfqpoint{-18.448764in}{0.773588in}}%
\pgfpathlineto{\pgfqpoint{-18.494344in}{0.773588in}}%
\pgfpathlineto{\pgfqpoint{-18.540355in}{0.773588in}}%
\pgfpathlineto{\pgfqpoint{-18.587378in}{0.773588in}}%
\pgfpathlineto{\pgfqpoint{-18.632978in}{0.773588in}}%
\pgfpathlineto{\pgfqpoint{-18.678684in}{0.773588in}}%
\pgfpathlineto{\pgfqpoint{-18.726033in}{0.773588in}}%
\pgfpathlineto{\pgfqpoint{-18.772260in}{0.773588in}}%
\pgfpathlineto{\pgfqpoint{-18.817991in}{0.773588in}}%
\pgfpathlineto{\pgfqpoint{-18.865040in}{0.773588in}}%
\pgfpathlineto{\pgfqpoint{-18.910503in}{0.773588in}}%
\pgfpathlineto{\pgfqpoint{-18.956783in}{0.773588in}}%
\pgfpathlineto{\pgfqpoint{-19.004576in}{0.773588in}}%
\pgfpathlineto{\pgfqpoint{-19.050812in}{0.773588in}}%
\pgfpathlineto{\pgfqpoint{-19.097062in}{0.773588in}}%
\pgfpathlineto{\pgfqpoint{-19.143544in}{0.773588in}}%
\pgfpathlineto{\pgfqpoint{-19.188899in}{0.773588in}}%
\pgfpathlineto{\pgfqpoint{-19.234797in}{0.773588in}}%
\pgfpathlineto{\pgfqpoint{-19.283047in}{0.773588in}}%
\pgfpathlineto{\pgfqpoint{-19.328532in}{0.773588in}}%
\pgfpathlineto{\pgfqpoint{-19.374230in}{0.773588in}}%
\pgfpathlineto{\pgfqpoint{-19.421584in}{0.773588in}}%
\pgfpathlineto{\pgfqpoint{-19.467303in}{0.773588in}}%
\pgfpathlineto{\pgfqpoint{-19.512973in}{0.773588in}}%
\pgfpathlineto{\pgfqpoint{-19.559546in}{0.773588in}}%
\pgfpathlineto{\pgfqpoint{-19.604824in}{0.773588in}}%
\pgfpathlineto{\pgfqpoint{-19.650361in}{0.773588in}}%
\pgfpathlineto{\pgfqpoint{-19.696649in}{0.773588in}}%
\pgfpathlineto{\pgfqpoint{-19.742024in}{0.773588in}}%
\pgfpathlineto{\pgfqpoint{-19.787808in}{0.773588in}}%
\pgfpathlineto{\pgfqpoint{-19.833948in}{0.773588in}}%
\pgfpathlineto{\pgfqpoint{-19.878562in}{0.773588in}}%
\pgfpathlineto{\pgfqpoint{-19.924139in}{0.773588in}}%
\pgfpathlineto{\pgfqpoint{-19.970967in}{0.773588in}}%
\pgfpathlineto{\pgfqpoint{-20.016506in}{0.773588in}}%
\pgfpathlineto{\pgfqpoint{-20.061722in}{0.773588in}}%
\pgfpathlineto{\pgfqpoint{-20.107707in}{0.773588in}}%
\pgfpathlineto{\pgfqpoint{-20.153038in}{0.773588in}}%
\pgfpathlineto{\pgfqpoint{-20.197814in}{0.773588in}}%
\pgfpathlineto{\pgfqpoint{-20.244875in}{0.773588in}}%
\pgfpathlineto{\pgfqpoint{-20.290419in}{0.773588in}}%
\pgfpathlineto{\pgfqpoint{-20.335981in}{0.773588in}}%
\pgfpathlineto{\pgfqpoint{-20.382199in}{0.773588in}}%
\pgfpathlineto{\pgfqpoint{-20.428221in}{0.773588in}}%
\pgfpathlineto{\pgfqpoint{-20.474014in}{0.773588in}}%
\pgfpathlineto{\pgfqpoint{-20.519904in}{0.773588in}}%
\pgfpathlineto{\pgfqpoint{-20.565077in}{0.773588in}}%
\pgfpathlineto{\pgfqpoint{-20.611243in}{0.773588in}}%
\pgfpathlineto{\pgfqpoint{-20.658742in}{0.773588in}}%
\pgfpathlineto{\pgfqpoint{-20.705066in}{0.773588in}}%
\pgfpathlineto{\pgfqpoint{-20.750831in}{0.773588in}}%
\pgfpathlineto{\pgfqpoint{-20.797022in}{0.773588in}}%
\pgfpathlineto{\pgfqpoint{-20.842169in}{0.773588in}}%
\pgfpathlineto{\pgfqpoint{-20.886847in}{0.773588in}}%
\pgfpathlineto{\pgfqpoint{-20.932365in}{0.773588in}}%
\pgfpathlineto{\pgfqpoint{-20.976378in}{0.773588in}}%
\pgfpathlineto{\pgfqpoint{-21.021509in}{0.773588in}}%
\pgfpathlineto{\pgfqpoint{-21.067357in}{0.773588in}}%
\pgfpathlineto{\pgfqpoint{-21.113139in}{0.773588in}}%
\pgfpathlineto{\pgfqpoint{-21.158215in}{0.773588in}}%
\pgfpathlineto{\pgfqpoint{-21.204354in}{0.773588in}}%
\pgfpathlineto{\pgfqpoint{-21.249304in}{0.773588in}}%
\pgfpathlineto{\pgfqpoint{-21.294264in}{0.773588in}}%
\pgfpathlineto{\pgfqpoint{-21.340867in}{0.773588in}}%
\pgfpathlineto{\pgfqpoint{-21.385572in}{0.773588in}}%
\pgfpathlineto{\pgfqpoint{-21.430267in}{0.773588in}}%
\pgfpathlineto{\pgfqpoint{-21.475985in}{0.773588in}}%
\pgfpathlineto{\pgfqpoint{-21.520883in}{0.773588in}}%
\pgfpathlineto{\pgfqpoint{-21.566198in}{0.773588in}}%
\pgfpathlineto{\pgfqpoint{-21.612919in}{0.773588in}}%
\pgfpathlineto{\pgfqpoint{-21.657342in}{0.773588in}}%
\pgfpathlineto{\pgfqpoint{-21.703167in}{0.773588in}}%
\pgfpathlineto{\pgfqpoint{-21.749554in}{0.773588in}}%
\pgfpathlineto{\pgfqpoint{-21.794255in}{0.773588in}}%
\pgfpathlineto{\pgfqpoint{-21.839254in}{0.773588in}}%
\pgfpathlineto{\pgfqpoint{-21.885462in}{0.773588in}}%
\pgfpathlineto{\pgfqpoint{-21.930597in}{0.773588in}}%
\pgfpathlineto{\pgfqpoint{-21.976168in}{0.773588in}}%
\pgfpathlineto{\pgfqpoint{-22.022397in}{0.773588in}}%
\pgfpathlineto{\pgfqpoint{-22.067661in}{0.773588in}}%
\pgfpathlineto{\pgfqpoint{-22.112231in}{0.773588in}}%
\pgfpathlineto{\pgfqpoint{-22.158025in}{0.773588in}}%
\pgfpathlineto{\pgfqpoint{-22.201868in}{0.773588in}}%
\pgfpathlineto{\pgfqpoint{-22.246011in}{0.773588in}}%
\pgfpathlineto{\pgfqpoint{-22.291940in}{0.773588in}}%
\pgfpathlineto{\pgfqpoint{-22.336074in}{0.773588in}}%
\pgfpathlineto{\pgfqpoint{-22.380996in}{0.773588in}}%
\pgfpathlineto{\pgfqpoint{-22.427383in}{0.773588in}}%
\pgfpathlineto{\pgfqpoint{-22.471952in}{0.773588in}}%
\pgfpathlineto{\pgfqpoint{-22.516601in}{0.773588in}}%
\pgfpathlineto{\pgfqpoint{-22.562133in}{0.773588in}}%
\pgfpathlineto{\pgfqpoint{-22.606850in}{0.773588in}}%
\pgfpathlineto{\pgfqpoint{-22.651258in}{0.773588in}}%
\pgfpathlineto{\pgfqpoint{-22.697642in}{0.773588in}}%
\pgfpathlineto{\pgfqpoint{-22.742185in}{0.773588in}}%
\pgfpathlineto{\pgfqpoint{-22.786844in}{0.773588in}}%
\pgfpathlineto{\pgfqpoint{-22.833029in}{0.773588in}}%
\pgfpathlineto{\pgfqpoint{-22.878494in}{0.773588in}}%
\pgfpathlineto{\pgfqpoint{-22.924822in}{0.773588in}}%
\pgfpathlineto{\pgfqpoint{-22.972746in}{0.773588in}}%
\pgfpathlineto{\pgfqpoint{-23.018637in}{0.773588in}}%
\pgfpathlineto{\pgfqpoint{-23.065076in}{0.773588in}}%
\pgfpathlineto{\pgfqpoint{-23.112561in}{0.773588in}}%
\pgfpathlineto{\pgfqpoint{-23.159034in}{0.773588in}}%
\pgfpathlineto{\pgfqpoint{-23.205343in}{0.773588in}}%
\pgfpathlineto{\pgfqpoint{-23.252910in}{0.773588in}}%
\pgfpathlineto{\pgfqpoint{-23.298917in}{0.773588in}}%
\pgfpathlineto{\pgfqpoint{-23.344516in}{0.773588in}}%
\pgfpathlineto{\pgfqpoint{-23.391460in}{0.773588in}}%
\pgfpathlineto{\pgfqpoint{-23.437154in}{0.773588in}}%
\pgfpathlineto{\pgfqpoint{-23.482780in}{0.773588in}}%
\pgfpathlineto{\pgfqpoint{-23.529965in}{0.773588in}}%
\pgfpathlineto{\pgfqpoint{-23.575877in}{0.773588in}}%
\pgfpathlineto{\pgfqpoint{-23.620333in}{0.773588in}}%
\pgfpathlineto{\pgfqpoint{-23.666576in}{0.773588in}}%
\pgfpathlineto{\pgfqpoint{-23.711883in}{0.773588in}}%
\pgfpathlineto{\pgfqpoint{-23.757289in}{0.773588in}}%
\pgfpathlineto{\pgfqpoint{-23.804343in}{0.773588in}}%
\pgfpathlineto{\pgfqpoint{-23.849943in}{0.773588in}}%
\pgfpathlineto{\pgfqpoint{-23.895017in}{0.773588in}}%
\pgfpathlineto{\pgfqpoint{-23.941619in}{0.773588in}}%
\pgfpathlineto{\pgfqpoint{-23.986811in}{0.773588in}}%
\pgfpathlineto{\pgfqpoint{-24.032531in}{0.773588in}}%
\pgfpathlineto{\pgfqpoint{-24.078902in}{0.773588in}}%
\pgfpathlineto{\pgfqpoint{-24.123999in}{0.773588in}}%
\pgfpathlineto{\pgfqpoint{-24.169854in}{0.773588in}}%
\pgfpathlineto{\pgfqpoint{-24.216953in}{0.773588in}}%
\pgfpathlineto{\pgfqpoint{-24.262109in}{0.773588in}}%
\pgfpathlineto{\pgfqpoint{-24.308071in}{0.773588in}}%
\pgfpathlineto{\pgfqpoint{-24.355467in}{0.773588in}}%
\pgfpathlineto{\pgfqpoint{-24.401173in}{0.773588in}}%
\pgfpathlineto{\pgfqpoint{-24.447114in}{0.773588in}}%
\pgfpathlineto{\pgfqpoint{-24.494847in}{0.773588in}}%
\pgfpathlineto{\pgfqpoint{-24.540716in}{0.773588in}}%
\pgfpathlineto{\pgfqpoint{-24.586259in}{0.773588in}}%
\pgfpathlineto{\pgfqpoint{-24.632696in}{0.773588in}}%
\pgfpathlineto{\pgfqpoint{-24.677917in}{0.773588in}}%
\pgfpathlineto{\pgfqpoint{-24.722523in}{0.773588in}}%
\pgfpathlineto{\pgfqpoint{-24.768227in}{0.773588in}}%
\pgfpathlineto{\pgfqpoint{-24.813437in}{0.773588in}}%
\pgfpathlineto{\pgfqpoint{-24.858721in}{0.773588in}}%
\pgfpathlineto{\pgfqpoint{-24.905462in}{0.773588in}}%
\pgfpathlineto{\pgfqpoint{-24.950861in}{0.773588in}}%
\pgfpathlineto{\pgfqpoint{-24.995958in}{0.773588in}}%
\pgfpathlineto{\pgfqpoint{-25.043027in}{0.773588in}}%
\pgfpathlineto{\pgfqpoint{-25.088791in}{0.773588in}}%
\pgfpathlineto{\pgfqpoint{-25.133757in}{0.773588in}}%
\pgfpathlineto{\pgfqpoint{-25.181603in}{0.773588in}}%
\pgfpathlineto{\pgfqpoint{-25.227067in}{0.773588in}}%
\pgfpathlineto{\pgfqpoint{-25.271783in}{0.773588in}}%
\pgfpathlineto{\pgfqpoint{-25.318638in}{0.773588in}}%
\pgfpathlineto{\pgfqpoint{-25.364031in}{0.773588in}}%
\pgfpathlineto{\pgfqpoint{-25.409278in}{0.773588in}}%
\pgfpathlineto{\pgfqpoint{-25.455838in}{0.773588in}}%
\pgfpathlineto{\pgfqpoint{-25.501531in}{0.773588in}}%
\pgfpathlineto{\pgfqpoint{-25.546370in}{0.773588in}}%
\pgfpathlineto{\pgfqpoint{-25.593105in}{0.773588in}}%
\pgfpathlineto{\pgfqpoint{-25.638897in}{0.773588in}}%
\pgfpathlineto{\pgfqpoint{-25.684196in}{0.773588in}}%
\pgfpathlineto{\pgfqpoint{-25.731236in}{0.773588in}}%
\pgfpathlineto{\pgfqpoint{-25.776589in}{0.773588in}}%
\pgfpathlineto{\pgfqpoint{-25.821656in}{0.773588in}}%
\pgfpathlineto{\pgfqpoint{-25.868331in}{0.773588in}}%
\pgfpathlineto{\pgfqpoint{-25.913010in}{0.773588in}}%
\pgfpathlineto{\pgfqpoint{-25.957619in}{0.773588in}}%
\pgfpathlineto{\pgfqpoint{-26.004871in}{0.773588in}}%
\pgfpathlineto{\pgfqpoint{-26.049962in}{0.773588in}}%
\pgfpathlineto{\pgfqpoint{-26.095397in}{0.773588in}}%
\pgfpathlineto{\pgfqpoint{-26.140687in}{0.773588in}}%
\pgfpathlineto{\pgfqpoint{-26.185627in}{0.773588in}}%
\pgfpathlineto{\pgfqpoint{-26.230565in}{0.773588in}}%
\pgfpathlineto{\pgfqpoint{-26.277965in}{0.773588in}}%
\pgfpathlineto{\pgfqpoint{-26.323384in}{0.773588in}}%
\pgfpathlineto{\pgfqpoint{-26.368659in}{0.773588in}}%
\pgfpathlineto{\pgfqpoint{-26.414228in}{0.773588in}}%
\pgfpathlineto{\pgfqpoint{-26.458732in}{0.773588in}}%
\pgfpathlineto{\pgfqpoint{-26.503012in}{0.773588in}}%
\pgfpathlineto{\pgfqpoint{-26.549765in}{0.773588in}}%
\pgfpathlineto{\pgfqpoint{-26.595409in}{0.773588in}}%
\pgfpathlineto{\pgfqpoint{-26.640477in}{0.773588in}}%
\pgfpathlineto{\pgfqpoint{-26.688111in}{0.773588in}}%
\pgfpathlineto{\pgfqpoint{-26.735142in}{0.773588in}}%
\pgfpathlineto{\pgfqpoint{-26.781839in}{0.773588in}}%
\pgfpathlineto{\pgfqpoint{-26.830647in}{0.773588in}}%
\pgfpathlineto{\pgfqpoint{-26.880611in}{0.773588in}}%
\pgfpathlineto{\pgfqpoint{-26.933555in}{0.773588in}}%
\pgfpathlineto{\pgfqpoint{-26.988998in}{0.773588in}}%
\pgfpathlineto{\pgfqpoint{-27.037714in}{0.773588in}}%
\pgfpathlineto{\pgfqpoint{-27.086555in}{0.773588in}}%
\pgfpathlineto{\pgfqpoint{-27.136341in}{0.773588in}}%
\pgfpathlineto{\pgfqpoint{-27.184903in}{0.773588in}}%
\pgfpathlineto{\pgfqpoint{-27.232862in}{0.773588in}}%
\pgfpathlineto{\pgfqpoint{-27.281821in}{0.773588in}}%
\pgfpathlineto{\pgfqpoint{-27.328705in}{0.773588in}}%
\pgfpathlineto{\pgfqpoint{-27.374936in}{0.773588in}}%
\pgfpathlineto{\pgfqpoint{-27.422670in}{0.773588in}}%
\pgfpathlineto{\pgfqpoint{-27.470237in}{0.773588in}}%
\pgfpathlineto{\pgfqpoint{-27.516652in}{0.773588in}}%
\pgfpathlineto{\pgfqpoint{-27.563341in}{0.773588in}}%
\pgfpathlineto{\pgfqpoint{-27.608047in}{0.773588in}}%
\pgfpathlineto{\pgfqpoint{-27.651646in}{0.773588in}}%
\pgfpathlineto{\pgfqpoint{-27.696819in}{0.773588in}}%
\pgfpathlineto{\pgfqpoint{-27.740871in}{0.773588in}}%
\pgfpathlineto{\pgfqpoint{-27.785329in}{0.773588in}}%
\pgfpathlineto{\pgfqpoint{-27.830427in}{0.773588in}}%
\pgfpathlineto{\pgfqpoint{-27.874405in}{0.773588in}}%
\pgfpathlineto{\pgfqpoint{-27.917812in}{0.773588in}}%
\pgfpathlineto{\pgfqpoint{-27.963188in}{0.773588in}}%
\pgfpathlineto{\pgfqpoint{-28.007858in}{0.773588in}}%
\pgfpathlineto{\pgfqpoint{-28.051883in}{0.773588in}}%
\pgfpathlineto{\pgfqpoint{-28.097071in}{0.773588in}}%
\pgfpathlineto{\pgfqpoint{-28.141536in}{0.773588in}}%
\pgfpathlineto{\pgfqpoint{-28.185953in}{0.773588in}}%
\pgfpathlineto{\pgfqpoint{-28.230974in}{0.773588in}}%
\pgfpathlineto{\pgfqpoint{-28.274355in}{0.773588in}}%
\pgfpathlineto{\pgfqpoint{-28.318373in}{0.773588in}}%
\pgfpathlineto{\pgfqpoint{-28.364238in}{0.773588in}}%
\pgfpathlineto{\pgfqpoint{-28.408130in}{0.773588in}}%
\pgfpathlineto{\pgfqpoint{-28.452363in}{0.773588in}}%
\pgfpathlineto{\pgfqpoint{-28.498901in}{0.773588in}}%
\pgfpathlineto{\pgfqpoint{-28.543533in}{0.773588in}}%
\pgfpathlineto{\pgfqpoint{-28.588250in}{0.773588in}}%
\pgfpathclose%
\pgfusepath{fill}%
\end{pgfscope}%
\begin{pgfscope}%
\pgfpathrectangle{\pgfqpoint{2.662073in}{0.773588in}}{\pgfqpoint{2.964025in}{5.415119in}}%
\pgfusepath{clip}%
\pgfsetbuttcap%
\pgfsetroundjoin%
\definecolor{currentfill}{rgb}{1.000000,0.498039,0.054902}%
\pgfsetfillcolor{currentfill}%
\pgfsetlinewidth{0.000000pt}%
\definecolor{currentstroke}{rgb}{0.000000,0.000000,0.000000}%
\pgfsetstrokecolor{currentstroke}%
\pgfsetdash{}{0pt}%
\pgfpathmoveto{\pgfqpoint{-28.588250in}{0.773588in}}%
\pgfpathlineto{\pgfqpoint{-28.588250in}{0.773588in}}%
\pgfpathlineto{\pgfqpoint{-28.543533in}{0.773588in}}%
\pgfpathlineto{\pgfqpoint{-28.498901in}{0.773588in}}%
\pgfpathlineto{\pgfqpoint{-28.452363in}{0.773588in}}%
\pgfpathlineto{\pgfqpoint{-28.408130in}{0.773588in}}%
\pgfpathlineto{\pgfqpoint{-28.364238in}{0.773588in}}%
\pgfpathlineto{\pgfqpoint{-28.318373in}{0.773588in}}%
\pgfpathlineto{\pgfqpoint{-28.274355in}{0.773588in}}%
\pgfpathlineto{\pgfqpoint{-28.230974in}{0.773588in}}%
\pgfpathlineto{\pgfqpoint{-28.185953in}{0.773588in}}%
\pgfpathlineto{\pgfqpoint{-28.141536in}{0.773588in}}%
\pgfpathlineto{\pgfqpoint{-28.097071in}{0.773588in}}%
\pgfpathlineto{\pgfqpoint{-28.051883in}{0.773588in}}%
\pgfpathlineto{\pgfqpoint{-28.007858in}{0.773588in}}%
\pgfpathlineto{\pgfqpoint{-27.963188in}{0.773588in}}%
\pgfpathlineto{\pgfqpoint{-27.917812in}{0.773588in}}%
\pgfpathlineto{\pgfqpoint{-27.874405in}{0.773588in}}%
\pgfpathlineto{\pgfqpoint{-27.830427in}{0.773588in}}%
\pgfpathlineto{\pgfqpoint{-27.785329in}{0.773588in}}%
\pgfpathlineto{\pgfqpoint{-27.740871in}{0.773588in}}%
\pgfpathlineto{\pgfqpoint{-27.696819in}{0.773588in}}%
\pgfpathlineto{\pgfqpoint{-27.651646in}{0.773588in}}%
\pgfpathlineto{\pgfqpoint{-27.608047in}{0.773588in}}%
\pgfpathlineto{\pgfqpoint{-27.563341in}{0.773588in}}%
\pgfpathlineto{\pgfqpoint{-27.516652in}{0.773588in}}%
\pgfpathlineto{\pgfqpoint{-27.470237in}{0.773588in}}%
\pgfpathlineto{\pgfqpoint{-27.422670in}{0.773588in}}%
\pgfpathlineto{\pgfqpoint{-27.374936in}{0.773588in}}%
\pgfpathlineto{\pgfqpoint{-27.328705in}{0.773588in}}%
\pgfpathlineto{\pgfqpoint{-27.281821in}{0.773588in}}%
\pgfpathlineto{\pgfqpoint{-27.232862in}{0.773588in}}%
\pgfpathlineto{\pgfqpoint{-27.184903in}{0.773588in}}%
\pgfpathlineto{\pgfqpoint{-27.136341in}{0.773588in}}%
\pgfpathlineto{\pgfqpoint{-27.086555in}{0.773588in}}%
\pgfpathlineto{\pgfqpoint{-27.037714in}{0.773588in}}%
\pgfpathlineto{\pgfqpoint{-26.988998in}{0.773588in}}%
\pgfpathlineto{\pgfqpoint{-26.933555in}{0.773588in}}%
\pgfpathlineto{\pgfqpoint{-26.880611in}{0.773588in}}%
\pgfpathlineto{\pgfqpoint{-26.830647in}{0.773588in}}%
\pgfpathlineto{\pgfqpoint{-26.781839in}{0.773588in}}%
\pgfpathlineto{\pgfqpoint{-26.735142in}{0.773588in}}%
\pgfpathlineto{\pgfqpoint{-26.688111in}{0.773588in}}%
\pgfpathlineto{\pgfqpoint{-26.640477in}{0.773588in}}%
\pgfpathlineto{\pgfqpoint{-26.595409in}{0.773588in}}%
\pgfpathlineto{\pgfqpoint{-26.549765in}{0.773588in}}%
\pgfpathlineto{\pgfqpoint{-26.503012in}{0.773588in}}%
\pgfpathlineto{\pgfqpoint{-26.458732in}{0.773588in}}%
\pgfpathlineto{\pgfqpoint{-26.414228in}{0.773588in}}%
\pgfpathlineto{\pgfqpoint{-26.368659in}{0.773588in}}%
\pgfpathlineto{\pgfqpoint{-26.323384in}{0.773588in}}%
\pgfpathlineto{\pgfqpoint{-26.277965in}{0.773588in}}%
\pgfpathlineto{\pgfqpoint{-26.230565in}{0.773588in}}%
\pgfpathlineto{\pgfqpoint{-26.185627in}{0.773588in}}%
\pgfpathlineto{\pgfqpoint{-26.140687in}{0.773588in}}%
\pgfpathlineto{\pgfqpoint{-26.095397in}{0.773588in}}%
\pgfpathlineto{\pgfqpoint{-26.049962in}{0.773588in}}%
\pgfpathlineto{\pgfqpoint{-26.004871in}{0.773588in}}%
\pgfpathlineto{\pgfqpoint{-25.957619in}{0.773588in}}%
\pgfpathlineto{\pgfqpoint{-25.913010in}{0.773588in}}%
\pgfpathlineto{\pgfqpoint{-25.868331in}{0.773588in}}%
\pgfpathlineto{\pgfqpoint{-25.821656in}{0.773588in}}%
\pgfpathlineto{\pgfqpoint{-25.776589in}{0.773588in}}%
\pgfpathlineto{\pgfqpoint{-25.731236in}{0.773588in}}%
\pgfpathlineto{\pgfqpoint{-25.684196in}{0.773588in}}%
\pgfpathlineto{\pgfqpoint{-25.638897in}{0.773588in}}%
\pgfpathlineto{\pgfqpoint{-25.593105in}{0.773588in}}%
\pgfpathlineto{\pgfqpoint{-25.546370in}{0.773588in}}%
\pgfpathlineto{\pgfqpoint{-25.501531in}{0.773588in}}%
\pgfpathlineto{\pgfqpoint{-25.455838in}{0.773588in}}%
\pgfpathlineto{\pgfqpoint{-25.409278in}{0.773588in}}%
\pgfpathlineto{\pgfqpoint{-25.364031in}{0.773588in}}%
\pgfpathlineto{\pgfqpoint{-25.318638in}{0.773588in}}%
\pgfpathlineto{\pgfqpoint{-25.271783in}{0.773588in}}%
\pgfpathlineto{\pgfqpoint{-25.227067in}{0.773588in}}%
\pgfpathlineto{\pgfqpoint{-25.181603in}{0.773588in}}%
\pgfpathlineto{\pgfqpoint{-25.133757in}{0.773588in}}%
\pgfpathlineto{\pgfqpoint{-25.088791in}{0.773588in}}%
\pgfpathlineto{\pgfqpoint{-25.043027in}{0.773588in}}%
\pgfpathlineto{\pgfqpoint{-24.995958in}{0.773588in}}%
\pgfpathlineto{\pgfqpoint{-24.950861in}{0.773588in}}%
\pgfpathlineto{\pgfqpoint{-24.905462in}{0.773588in}}%
\pgfpathlineto{\pgfqpoint{-24.858721in}{0.773588in}}%
\pgfpathlineto{\pgfqpoint{-24.813437in}{0.773588in}}%
\pgfpathlineto{\pgfqpoint{-24.768227in}{0.773588in}}%
\pgfpathlineto{\pgfqpoint{-24.722523in}{0.773588in}}%
\pgfpathlineto{\pgfqpoint{-24.677917in}{0.773588in}}%
\pgfpathlineto{\pgfqpoint{-24.632696in}{0.773588in}}%
\pgfpathlineto{\pgfqpoint{-24.586259in}{0.773588in}}%
\pgfpathlineto{\pgfqpoint{-24.540716in}{0.773588in}}%
\pgfpathlineto{\pgfqpoint{-24.494847in}{0.773588in}}%
\pgfpathlineto{\pgfqpoint{-24.447114in}{0.773588in}}%
\pgfpathlineto{\pgfqpoint{-24.401173in}{0.773588in}}%
\pgfpathlineto{\pgfqpoint{-24.355467in}{0.773588in}}%
\pgfpathlineto{\pgfqpoint{-24.308071in}{0.773588in}}%
\pgfpathlineto{\pgfqpoint{-24.262109in}{0.773588in}}%
\pgfpathlineto{\pgfqpoint{-24.216953in}{0.773588in}}%
\pgfpathlineto{\pgfqpoint{-24.169854in}{0.773588in}}%
\pgfpathlineto{\pgfqpoint{-24.123999in}{0.773588in}}%
\pgfpathlineto{\pgfqpoint{-24.078902in}{0.773588in}}%
\pgfpathlineto{\pgfqpoint{-24.032531in}{0.773588in}}%
\pgfpathlineto{\pgfqpoint{-23.986811in}{0.773588in}}%
\pgfpathlineto{\pgfqpoint{-23.941619in}{0.773588in}}%
\pgfpathlineto{\pgfqpoint{-23.895017in}{0.773588in}}%
\pgfpathlineto{\pgfqpoint{-23.849943in}{0.773588in}}%
\pgfpathlineto{\pgfqpoint{-23.804343in}{0.773588in}}%
\pgfpathlineto{\pgfqpoint{-23.757289in}{0.773588in}}%
\pgfpathlineto{\pgfqpoint{-23.711883in}{0.773588in}}%
\pgfpathlineto{\pgfqpoint{-23.666576in}{0.773588in}}%
\pgfpathlineto{\pgfqpoint{-23.620333in}{0.773588in}}%
\pgfpathlineto{\pgfqpoint{-23.575877in}{0.773588in}}%
\pgfpathlineto{\pgfqpoint{-23.529965in}{0.773588in}}%
\pgfpathlineto{\pgfqpoint{-23.482780in}{0.773588in}}%
\pgfpathlineto{\pgfqpoint{-23.437154in}{0.773588in}}%
\pgfpathlineto{\pgfqpoint{-23.391460in}{0.773588in}}%
\pgfpathlineto{\pgfqpoint{-23.344516in}{0.773588in}}%
\pgfpathlineto{\pgfqpoint{-23.298917in}{0.773588in}}%
\pgfpathlineto{\pgfqpoint{-23.252910in}{0.773588in}}%
\pgfpathlineto{\pgfqpoint{-23.205343in}{0.773588in}}%
\pgfpathlineto{\pgfqpoint{-23.159034in}{0.773588in}}%
\pgfpathlineto{\pgfqpoint{-23.112561in}{0.773588in}}%
\pgfpathlineto{\pgfqpoint{-23.065076in}{0.773588in}}%
\pgfpathlineto{\pgfqpoint{-23.018637in}{0.773588in}}%
\pgfpathlineto{\pgfqpoint{-22.972746in}{0.773588in}}%
\pgfpathlineto{\pgfqpoint{-22.924822in}{0.773588in}}%
\pgfpathlineto{\pgfqpoint{-22.878494in}{0.773588in}}%
\pgfpathlineto{\pgfqpoint{-22.833029in}{0.773588in}}%
\pgfpathlineto{\pgfqpoint{-22.786844in}{0.773588in}}%
\pgfpathlineto{\pgfqpoint{-22.742185in}{0.773588in}}%
\pgfpathlineto{\pgfqpoint{-22.697642in}{0.773588in}}%
\pgfpathlineto{\pgfqpoint{-22.651258in}{0.773588in}}%
\pgfpathlineto{\pgfqpoint{-22.606850in}{0.773588in}}%
\pgfpathlineto{\pgfqpoint{-22.562133in}{0.773588in}}%
\pgfpathlineto{\pgfqpoint{-22.516601in}{0.773588in}}%
\pgfpathlineto{\pgfqpoint{-22.471952in}{0.773588in}}%
\pgfpathlineto{\pgfqpoint{-22.427383in}{0.773588in}}%
\pgfpathlineto{\pgfqpoint{-22.380996in}{0.773588in}}%
\pgfpathlineto{\pgfqpoint{-22.336074in}{0.773588in}}%
\pgfpathlineto{\pgfqpoint{-22.291940in}{0.773588in}}%
\pgfpathlineto{\pgfqpoint{-22.246011in}{0.773588in}}%
\pgfpathlineto{\pgfqpoint{-22.201868in}{0.773588in}}%
\pgfpathlineto{\pgfqpoint{-22.158025in}{0.773588in}}%
\pgfpathlineto{\pgfqpoint{-22.112231in}{0.773588in}}%
\pgfpathlineto{\pgfqpoint{-22.067661in}{0.773588in}}%
\pgfpathlineto{\pgfqpoint{-22.022397in}{0.773588in}}%
\pgfpathlineto{\pgfqpoint{-21.976168in}{0.773588in}}%
\pgfpathlineto{\pgfqpoint{-21.930597in}{0.773588in}}%
\pgfpathlineto{\pgfqpoint{-21.885462in}{0.773588in}}%
\pgfpathlineto{\pgfqpoint{-21.839254in}{0.773588in}}%
\pgfpathlineto{\pgfqpoint{-21.794255in}{0.773588in}}%
\pgfpathlineto{\pgfqpoint{-21.749554in}{0.773588in}}%
\pgfpathlineto{\pgfqpoint{-21.703167in}{0.773588in}}%
\pgfpathlineto{\pgfqpoint{-21.657342in}{0.773588in}}%
\pgfpathlineto{\pgfqpoint{-21.612919in}{0.773588in}}%
\pgfpathlineto{\pgfqpoint{-21.566198in}{0.773588in}}%
\pgfpathlineto{\pgfqpoint{-21.520883in}{0.773588in}}%
\pgfpathlineto{\pgfqpoint{-21.475985in}{0.773588in}}%
\pgfpathlineto{\pgfqpoint{-21.430267in}{0.773588in}}%
\pgfpathlineto{\pgfqpoint{-21.385572in}{0.773588in}}%
\pgfpathlineto{\pgfqpoint{-21.340867in}{0.773588in}}%
\pgfpathlineto{\pgfqpoint{-21.294264in}{0.773588in}}%
\pgfpathlineto{\pgfqpoint{-21.249304in}{0.773588in}}%
\pgfpathlineto{\pgfqpoint{-21.204354in}{0.773588in}}%
\pgfpathlineto{\pgfqpoint{-21.158215in}{0.773588in}}%
\pgfpathlineto{\pgfqpoint{-21.113139in}{0.773588in}}%
\pgfpathlineto{\pgfqpoint{-21.067357in}{0.773588in}}%
\pgfpathlineto{\pgfqpoint{-21.021509in}{0.773588in}}%
\pgfpathlineto{\pgfqpoint{-20.976378in}{0.773588in}}%
\pgfpathlineto{\pgfqpoint{-20.932365in}{0.773588in}}%
\pgfpathlineto{\pgfqpoint{-20.886847in}{0.773588in}}%
\pgfpathlineto{\pgfqpoint{-20.842169in}{0.773588in}}%
\pgfpathlineto{\pgfqpoint{-20.797022in}{0.773588in}}%
\pgfpathlineto{\pgfqpoint{-20.750831in}{0.773588in}}%
\pgfpathlineto{\pgfqpoint{-20.705066in}{0.773588in}}%
\pgfpathlineto{\pgfqpoint{-20.658742in}{0.773588in}}%
\pgfpathlineto{\pgfqpoint{-20.611243in}{0.773588in}}%
\pgfpathlineto{\pgfqpoint{-20.565077in}{0.773588in}}%
\pgfpathlineto{\pgfqpoint{-20.519904in}{0.773588in}}%
\pgfpathlineto{\pgfqpoint{-20.474014in}{0.773588in}}%
\pgfpathlineto{\pgfqpoint{-20.428221in}{0.773588in}}%
\pgfpathlineto{\pgfqpoint{-20.382199in}{0.773588in}}%
\pgfpathlineto{\pgfqpoint{-20.335981in}{0.773588in}}%
\pgfpathlineto{\pgfqpoint{-20.290419in}{0.773588in}}%
\pgfpathlineto{\pgfqpoint{-20.244875in}{0.773588in}}%
\pgfpathlineto{\pgfqpoint{-20.197814in}{0.773588in}}%
\pgfpathlineto{\pgfqpoint{-20.153038in}{0.773588in}}%
\pgfpathlineto{\pgfqpoint{-20.107707in}{0.773588in}}%
\pgfpathlineto{\pgfqpoint{-20.061722in}{0.773588in}}%
\pgfpathlineto{\pgfqpoint{-20.016506in}{0.773588in}}%
\pgfpathlineto{\pgfqpoint{-19.970967in}{0.773588in}}%
\pgfpathlineto{\pgfqpoint{-19.924139in}{0.773588in}}%
\pgfpathlineto{\pgfqpoint{-19.878562in}{0.773588in}}%
\pgfpathlineto{\pgfqpoint{-19.833948in}{0.773588in}}%
\pgfpathlineto{\pgfqpoint{-19.787808in}{0.773588in}}%
\pgfpathlineto{\pgfqpoint{-19.742024in}{0.773588in}}%
\pgfpathlineto{\pgfqpoint{-19.696649in}{0.773588in}}%
\pgfpathlineto{\pgfqpoint{-19.650361in}{0.773588in}}%
\pgfpathlineto{\pgfqpoint{-19.604824in}{0.773588in}}%
\pgfpathlineto{\pgfqpoint{-19.559546in}{0.773588in}}%
\pgfpathlineto{\pgfqpoint{-19.512973in}{0.773588in}}%
\pgfpathlineto{\pgfqpoint{-19.467303in}{0.773588in}}%
\pgfpathlineto{\pgfqpoint{-19.421584in}{0.773588in}}%
\pgfpathlineto{\pgfqpoint{-19.374230in}{0.773588in}}%
\pgfpathlineto{\pgfqpoint{-19.328532in}{0.773588in}}%
\pgfpathlineto{\pgfqpoint{-19.283047in}{0.773588in}}%
\pgfpathlineto{\pgfqpoint{-19.234797in}{0.773588in}}%
\pgfpathlineto{\pgfqpoint{-19.188899in}{0.773588in}}%
\pgfpathlineto{\pgfqpoint{-19.143544in}{0.773588in}}%
\pgfpathlineto{\pgfqpoint{-19.097062in}{0.773588in}}%
\pgfpathlineto{\pgfqpoint{-19.050812in}{0.773588in}}%
\pgfpathlineto{\pgfqpoint{-19.004576in}{0.773588in}}%
\pgfpathlineto{\pgfqpoint{-18.956783in}{0.773588in}}%
\pgfpathlineto{\pgfqpoint{-18.910503in}{0.773588in}}%
\pgfpathlineto{\pgfqpoint{-18.865040in}{0.773588in}}%
\pgfpathlineto{\pgfqpoint{-18.817991in}{0.773588in}}%
\pgfpathlineto{\pgfqpoint{-18.772260in}{0.773588in}}%
\pgfpathlineto{\pgfqpoint{-18.726033in}{0.773588in}}%
\pgfpathlineto{\pgfqpoint{-18.678684in}{0.773588in}}%
\pgfpathlineto{\pgfqpoint{-18.632978in}{0.773588in}}%
\pgfpathlineto{\pgfqpoint{-18.587378in}{0.773588in}}%
\pgfpathlineto{\pgfqpoint{-18.540355in}{0.773588in}}%
\pgfpathlineto{\pgfqpoint{-18.494344in}{0.773588in}}%
\pgfpathlineto{\pgfqpoint{-18.448764in}{0.773588in}}%
\pgfpathlineto{\pgfqpoint{-18.402000in}{0.773588in}}%
\pgfpathlineto{\pgfqpoint{-18.356748in}{0.773588in}}%
\pgfpathlineto{\pgfqpoint{-18.311512in}{0.773588in}}%
\pgfpathlineto{\pgfqpoint{-18.264167in}{0.773588in}}%
\pgfpathlineto{\pgfqpoint{-18.218674in}{0.773588in}}%
\pgfpathlineto{\pgfqpoint{-18.172791in}{0.773588in}}%
\pgfpathlineto{\pgfqpoint{-18.124795in}{0.773588in}}%
\pgfpathlineto{\pgfqpoint{-18.078784in}{0.773588in}}%
\pgfpathlineto{\pgfqpoint{-18.032389in}{0.773588in}}%
\pgfpathlineto{\pgfqpoint{-17.984102in}{0.773588in}}%
\pgfpathlineto{\pgfqpoint{-17.937614in}{0.773588in}}%
\pgfpathlineto{\pgfqpoint{-17.891271in}{0.773588in}}%
\pgfpathlineto{\pgfqpoint{-17.843480in}{0.773588in}}%
\pgfpathlineto{\pgfqpoint{-17.797346in}{0.773588in}}%
\pgfpathlineto{\pgfqpoint{-17.751556in}{0.773588in}}%
\pgfpathlineto{\pgfqpoint{-17.704093in}{0.773588in}}%
\pgfpathlineto{\pgfqpoint{-17.658538in}{0.773588in}}%
\pgfpathlineto{\pgfqpoint{-17.613251in}{0.773588in}}%
\pgfpathlineto{\pgfqpoint{-17.566739in}{0.773588in}}%
\pgfpathlineto{\pgfqpoint{-17.521284in}{0.773588in}}%
\pgfpathlineto{\pgfqpoint{-17.474644in}{0.773588in}}%
\pgfpathlineto{\pgfqpoint{-17.427002in}{0.773588in}}%
\pgfpathlineto{\pgfqpoint{-17.381732in}{0.773588in}}%
\pgfpathlineto{\pgfqpoint{-17.336276in}{0.773588in}}%
\pgfpathlineto{\pgfqpoint{-17.289523in}{0.773588in}}%
\pgfpathlineto{\pgfqpoint{-17.244258in}{0.773588in}}%
\pgfpathlineto{\pgfqpoint{-17.198221in}{0.773588in}}%
\pgfpathlineto{\pgfqpoint{-17.150161in}{0.773588in}}%
\pgfpathlineto{\pgfqpoint{-17.103874in}{0.773588in}}%
\pgfpathlineto{\pgfqpoint{-17.058009in}{0.773588in}}%
\pgfpathlineto{\pgfqpoint{-17.010884in}{0.773588in}}%
\pgfpathlineto{\pgfqpoint{-16.964774in}{0.773588in}}%
\pgfpathlineto{\pgfqpoint{-16.918692in}{0.773588in}}%
\pgfpathlineto{\pgfqpoint{-16.871664in}{0.773588in}}%
\pgfpathlineto{\pgfqpoint{-16.825434in}{0.773588in}}%
\pgfpathlineto{\pgfqpoint{-16.778530in}{0.773588in}}%
\pgfpathlineto{\pgfqpoint{-16.730721in}{0.773588in}}%
\pgfpathlineto{\pgfqpoint{-16.684310in}{0.773588in}}%
\pgfpathlineto{\pgfqpoint{-16.637120in}{0.773588in}}%
\pgfpathlineto{\pgfqpoint{-16.589239in}{0.773588in}}%
\pgfpathlineto{\pgfqpoint{-16.542990in}{0.773588in}}%
\pgfpathlineto{\pgfqpoint{-16.496157in}{0.773588in}}%
\pgfpathlineto{\pgfqpoint{-16.449347in}{0.773588in}}%
\pgfpathlineto{\pgfqpoint{-16.402550in}{0.773588in}}%
\pgfpathlineto{\pgfqpoint{-16.355646in}{0.773588in}}%
\pgfpathlineto{\pgfqpoint{-16.308871in}{0.773588in}}%
\pgfpathlineto{\pgfqpoint{-16.262806in}{0.773588in}}%
\pgfpathlineto{\pgfqpoint{-16.217070in}{0.773588in}}%
\pgfpathlineto{\pgfqpoint{-16.169427in}{0.773588in}}%
\pgfpathlineto{\pgfqpoint{-16.123165in}{0.773588in}}%
\pgfpathlineto{\pgfqpoint{-16.077171in}{0.773588in}}%
\pgfpathlineto{\pgfqpoint{-16.029628in}{0.773588in}}%
\pgfpathlineto{\pgfqpoint{-15.984286in}{0.773588in}}%
\pgfpathlineto{\pgfqpoint{-15.938076in}{0.773588in}}%
\pgfpathlineto{\pgfqpoint{-15.890419in}{0.773588in}}%
\pgfpathlineto{\pgfqpoint{-15.844500in}{0.773588in}}%
\pgfpathlineto{\pgfqpoint{-15.798897in}{0.773588in}}%
\pgfpathlineto{\pgfqpoint{-15.750800in}{0.773588in}}%
\pgfpathlineto{\pgfqpoint{-15.703994in}{0.773588in}}%
\pgfpathlineto{\pgfqpoint{-15.657002in}{0.773588in}}%
\pgfpathlineto{\pgfqpoint{-15.608704in}{0.773588in}}%
\pgfpathlineto{\pgfqpoint{-15.560849in}{0.773588in}}%
\pgfpathlineto{\pgfqpoint{-15.512535in}{0.773588in}}%
\pgfpathlineto{\pgfqpoint{-15.463539in}{0.773588in}}%
\pgfpathlineto{\pgfqpoint{-15.416792in}{0.773588in}}%
\pgfpathlineto{\pgfqpoint{-15.369802in}{0.773588in}}%
\pgfpathlineto{\pgfqpoint{-15.321194in}{0.773588in}}%
\pgfpathlineto{\pgfqpoint{-15.273998in}{0.773588in}}%
\pgfpathlineto{\pgfqpoint{-15.225998in}{0.773588in}}%
\pgfpathlineto{\pgfqpoint{-15.176769in}{0.773588in}}%
\pgfpathlineto{\pgfqpoint{-15.130553in}{0.773588in}}%
\pgfpathlineto{\pgfqpoint{-15.083728in}{0.773588in}}%
\pgfpathlineto{\pgfqpoint{-15.035921in}{0.773588in}}%
\pgfpathlineto{\pgfqpoint{-14.989019in}{0.773588in}}%
\pgfpathlineto{\pgfqpoint{-14.942416in}{0.773588in}}%
\pgfpathlineto{\pgfqpoint{-14.894419in}{0.773588in}}%
\pgfpathlineto{\pgfqpoint{-14.847516in}{0.773588in}}%
\pgfpathlineto{\pgfqpoint{-14.800669in}{0.773588in}}%
\pgfpathlineto{\pgfqpoint{-14.752153in}{0.773588in}}%
\pgfpathlineto{\pgfqpoint{-14.705919in}{0.773588in}}%
\pgfpathlineto{\pgfqpoint{-14.659388in}{0.773588in}}%
\pgfpathlineto{\pgfqpoint{-14.611394in}{0.773588in}}%
\pgfpathlineto{\pgfqpoint{-14.564671in}{0.773588in}}%
\pgfpathlineto{\pgfqpoint{-14.517904in}{0.773588in}}%
\pgfpathlineto{\pgfqpoint{-14.469907in}{0.773588in}}%
\pgfpathlineto{\pgfqpoint{-14.422978in}{0.773588in}}%
\pgfpathlineto{\pgfqpoint{-14.375931in}{0.773588in}}%
\pgfpathlineto{\pgfqpoint{-14.327371in}{0.773588in}}%
\pgfpathlineto{\pgfqpoint{-14.280667in}{0.773588in}}%
\pgfpathlineto{\pgfqpoint{-14.233487in}{0.773588in}}%
\pgfpathlineto{\pgfqpoint{-14.184689in}{0.773588in}}%
\pgfpathlineto{\pgfqpoint{-14.137283in}{0.773588in}}%
\pgfpathlineto{\pgfqpoint{-14.090279in}{0.773588in}}%
\pgfpathlineto{\pgfqpoint{-14.042263in}{0.773588in}}%
\pgfpathlineto{\pgfqpoint{-13.996250in}{0.773588in}}%
\pgfpathlineto{\pgfqpoint{-13.950094in}{0.773588in}}%
\pgfpathlineto{\pgfqpoint{-13.901287in}{0.773588in}}%
\pgfpathlineto{\pgfqpoint{-13.853816in}{0.773588in}}%
\pgfpathlineto{\pgfqpoint{-13.806330in}{0.773588in}}%
\pgfpathlineto{\pgfqpoint{-13.757485in}{0.773588in}}%
\pgfpathlineto{\pgfqpoint{-13.710126in}{0.773588in}}%
\pgfpathlineto{\pgfqpoint{-13.663331in}{0.773588in}}%
\pgfpathlineto{\pgfqpoint{-13.615991in}{0.773588in}}%
\pgfpathlineto{\pgfqpoint{-13.568873in}{0.773588in}}%
\pgfpathlineto{\pgfqpoint{-13.522553in}{0.773588in}}%
\pgfpathlineto{\pgfqpoint{-13.474335in}{0.773588in}}%
\pgfpathlineto{\pgfqpoint{-13.428346in}{0.773588in}}%
\pgfpathlineto{\pgfqpoint{-13.382220in}{0.773588in}}%
\pgfpathlineto{\pgfqpoint{-13.334326in}{0.773588in}}%
\pgfpathlineto{\pgfqpoint{-13.287448in}{0.773588in}}%
\pgfpathlineto{\pgfqpoint{-13.239807in}{0.773588in}}%
\pgfpathlineto{\pgfqpoint{-13.190970in}{0.773588in}}%
\pgfpathlineto{\pgfqpoint{-13.144068in}{0.773588in}}%
\pgfpathlineto{\pgfqpoint{-13.096328in}{0.773588in}}%
\pgfpathlineto{\pgfqpoint{-13.046608in}{0.773588in}}%
\pgfpathlineto{\pgfqpoint{-12.998480in}{0.773588in}}%
\pgfpathlineto{\pgfqpoint{-12.949523in}{0.773588in}}%
\pgfpathlineto{\pgfqpoint{-12.899585in}{0.773588in}}%
\pgfpathlineto{\pgfqpoint{-12.852526in}{0.773588in}}%
\pgfpathlineto{\pgfqpoint{-12.805010in}{0.773588in}}%
\pgfpathlineto{\pgfqpoint{-12.755866in}{0.773588in}}%
\pgfpathlineto{\pgfqpoint{-12.708579in}{0.773588in}}%
\pgfpathlineto{\pgfqpoint{-12.661578in}{0.773588in}}%
\pgfpathlineto{\pgfqpoint{-12.612433in}{0.773588in}}%
\pgfpathlineto{\pgfqpoint{-12.565107in}{0.773588in}}%
\pgfpathlineto{\pgfqpoint{-12.517880in}{0.773588in}}%
\pgfpathlineto{\pgfqpoint{-12.469843in}{0.773588in}}%
\pgfpathlineto{\pgfqpoint{-12.422612in}{0.773588in}}%
\pgfpathlineto{\pgfqpoint{-12.375560in}{0.773588in}}%
\pgfpathlineto{\pgfqpoint{-12.326627in}{0.773588in}}%
\pgfpathlineto{\pgfqpoint{-12.278675in}{0.773588in}}%
\pgfpathlineto{\pgfqpoint{-12.231638in}{0.773588in}}%
\pgfpathlineto{\pgfqpoint{-12.182471in}{0.773588in}}%
\pgfpathlineto{\pgfqpoint{-12.135286in}{0.773588in}}%
\pgfpathlineto{\pgfqpoint{-12.087205in}{0.773588in}}%
\pgfpathlineto{\pgfqpoint{-12.038569in}{0.773588in}}%
\pgfpathlineto{\pgfqpoint{-11.991431in}{0.773588in}}%
\pgfpathlineto{\pgfqpoint{-11.943783in}{0.773588in}}%
\pgfpathlineto{\pgfqpoint{-11.895021in}{0.773588in}}%
\pgfpathlineto{\pgfqpoint{-11.847736in}{0.773588in}}%
\pgfpathlineto{\pgfqpoint{-11.800403in}{0.773588in}}%
\pgfpathlineto{\pgfqpoint{-11.751492in}{0.773588in}}%
\pgfpathlineto{\pgfqpoint{-11.704231in}{0.773588in}}%
\pgfpathlineto{\pgfqpoint{-11.656418in}{0.773588in}}%
\pgfpathlineto{\pgfqpoint{-11.607131in}{0.773588in}}%
\pgfpathlineto{\pgfqpoint{-11.559336in}{0.773588in}}%
\pgfpathlineto{\pgfqpoint{-11.511546in}{0.773588in}}%
\pgfpathlineto{\pgfqpoint{-11.463260in}{0.773588in}}%
\pgfpathlineto{\pgfqpoint{-11.415235in}{0.773588in}}%
\pgfpathlineto{\pgfqpoint{-11.366120in}{0.773588in}}%
\pgfpathlineto{\pgfqpoint{-11.316622in}{0.773588in}}%
\pgfpathlineto{\pgfqpoint{-11.268838in}{0.773588in}}%
\pgfpathlineto{\pgfqpoint{-11.221741in}{0.773588in}}%
\pgfpathlineto{\pgfqpoint{-11.172812in}{0.773588in}}%
\pgfpathlineto{\pgfqpoint{-11.125274in}{0.773588in}}%
\pgfpathlineto{\pgfqpoint{-11.077704in}{0.773588in}}%
\pgfpathlineto{\pgfqpoint{-11.028607in}{0.773588in}}%
\pgfpathlineto{\pgfqpoint{-10.980903in}{0.773588in}}%
\pgfpathlineto{\pgfqpoint{-10.933861in}{0.773588in}}%
\pgfpathlineto{\pgfqpoint{-10.885230in}{0.773588in}}%
\pgfpathlineto{\pgfqpoint{-10.837686in}{0.773588in}}%
\pgfpathlineto{\pgfqpoint{-10.790422in}{0.773588in}}%
\pgfpathlineto{\pgfqpoint{-10.741507in}{0.773588in}}%
\pgfpathlineto{\pgfqpoint{-10.693023in}{0.773588in}}%
\pgfpathlineto{\pgfqpoint{-10.644481in}{0.773588in}}%
\pgfpathlineto{\pgfqpoint{-10.594934in}{0.773588in}}%
\pgfpathlineto{\pgfqpoint{-10.546833in}{0.773588in}}%
\pgfpathlineto{\pgfqpoint{-10.497957in}{0.773588in}}%
\pgfpathlineto{\pgfqpoint{-10.448474in}{0.773588in}}%
\pgfpathlineto{\pgfqpoint{-10.401144in}{0.773588in}}%
\pgfpathlineto{\pgfqpoint{-10.352351in}{0.773588in}}%
\pgfpathlineto{\pgfqpoint{-10.301938in}{0.773588in}}%
\pgfpathlineto{\pgfqpoint{-10.253421in}{0.773588in}}%
\pgfpathlineto{\pgfqpoint{-10.204843in}{0.773588in}}%
\pgfpathlineto{\pgfqpoint{-10.154319in}{0.773588in}}%
\pgfpathlineto{\pgfqpoint{-10.106404in}{0.773588in}}%
\pgfpathlineto{\pgfqpoint{-10.058316in}{0.773588in}}%
\pgfpathlineto{\pgfqpoint{-10.009255in}{0.773588in}}%
\pgfpathlineto{\pgfqpoint{-9.960967in}{0.773588in}}%
\pgfpathlineto{\pgfqpoint{-9.912991in}{0.773588in}}%
\pgfpathlineto{\pgfqpoint{-9.863853in}{0.773588in}}%
\pgfpathlineto{\pgfqpoint{-9.816398in}{0.773588in}}%
\pgfpathlineto{\pgfqpoint{-9.768438in}{0.773588in}}%
\pgfpathlineto{\pgfqpoint{-9.719205in}{0.773588in}}%
\pgfpathlineto{\pgfqpoint{-9.671880in}{0.773588in}}%
\pgfpathlineto{\pgfqpoint{-9.624467in}{0.773588in}}%
\pgfpathlineto{\pgfqpoint{-9.575433in}{0.773588in}}%
\pgfpathlineto{\pgfqpoint{-9.527213in}{0.773588in}}%
\pgfpathlineto{\pgfqpoint{-9.479530in}{0.773588in}}%
\pgfpathlineto{\pgfqpoint{-9.430893in}{0.773588in}}%
\pgfpathlineto{\pgfqpoint{-9.382453in}{0.773588in}}%
\pgfpathlineto{\pgfqpoint{-9.332833in}{0.773588in}}%
\pgfpathlineto{\pgfqpoint{-9.281795in}{0.773588in}}%
\pgfpathlineto{\pgfqpoint{-9.232466in}{0.773588in}}%
\pgfpathlineto{\pgfqpoint{-9.182531in}{0.773588in}}%
\pgfpathlineto{\pgfqpoint{-9.130552in}{0.773588in}}%
\pgfpathlineto{\pgfqpoint{-9.080221in}{0.773588in}}%
\pgfpathlineto{\pgfqpoint{-9.030683in}{0.773588in}}%
\pgfpathlineto{\pgfqpoint{-8.979731in}{0.773588in}}%
\pgfpathlineto{\pgfqpoint{-8.930541in}{0.773588in}}%
\pgfpathlineto{\pgfqpoint{-8.880937in}{0.773588in}}%
\pgfpathlineto{\pgfqpoint{-8.829855in}{0.773588in}}%
\pgfpathlineto{\pgfqpoint{-8.780578in}{0.773588in}}%
\pgfpathlineto{\pgfqpoint{-8.731477in}{0.773588in}}%
\pgfpathlineto{\pgfqpoint{-8.681529in}{0.773588in}}%
\pgfpathlineto{\pgfqpoint{-8.632588in}{0.773588in}}%
\pgfpathlineto{\pgfqpoint{-8.583181in}{0.773588in}}%
\pgfpathlineto{\pgfqpoint{-8.532027in}{0.773588in}}%
\pgfpathlineto{\pgfqpoint{-8.481981in}{0.773588in}}%
\pgfpathlineto{\pgfqpoint{-8.432560in}{0.773588in}}%
\pgfpathlineto{\pgfqpoint{-8.382452in}{0.773588in}}%
\pgfpathlineto{\pgfqpoint{-8.332630in}{0.773588in}}%
\pgfpathlineto{\pgfqpoint{-8.283775in}{0.773588in}}%
\pgfpathlineto{\pgfqpoint{-8.233427in}{0.773588in}}%
\pgfpathlineto{\pgfqpoint{-8.184462in}{0.773588in}}%
\pgfpathlineto{\pgfqpoint{-8.135490in}{0.773588in}}%
\pgfpathlineto{\pgfqpoint{-8.085190in}{0.773588in}}%
\pgfpathlineto{\pgfqpoint{-8.035792in}{0.773588in}}%
\pgfpathlineto{\pgfqpoint{-7.986412in}{0.773588in}}%
\pgfpathlineto{\pgfqpoint{-7.935315in}{0.773588in}}%
\pgfpathlineto{\pgfqpoint{-7.885122in}{0.773588in}}%
\pgfpathlineto{\pgfqpoint{-7.835188in}{0.773588in}}%
\pgfpathlineto{\pgfqpoint{-7.783607in}{0.773588in}}%
\pgfpathlineto{\pgfqpoint{-7.733870in}{0.773588in}}%
\pgfpathlineto{\pgfqpoint{-7.685281in}{0.773588in}}%
\pgfpathlineto{\pgfqpoint{-7.634521in}{0.773588in}}%
\pgfpathlineto{\pgfqpoint{-7.585672in}{0.773588in}}%
\pgfpathlineto{\pgfqpoint{-7.536566in}{0.773588in}}%
\pgfpathlineto{\pgfqpoint{-7.485312in}{0.773588in}}%
\pgfpathlineto{\pgfqpoint{-7.435602in}{0.773588in}}%
\pgfpathlineto{\pgfqpoint{-7.386343in}{0.773588in}}%
\pgfpathlineto{\pgfqpoint{-7.335791in}{0.773588in}}%
\pgfpathlineto{\pgfqpoint{-7.286238in}{0.773588in}}%
\pgfpathlineto{\pgfqpoint{-7.237328in}{0.773588in}}%
\pgfpathlineto{\pgfqpoint{-7.185558in}{0.773588in}}%
\pgfpathlineto{\pgfqpoint{-7.135511in}{0.773588in}}%
\pgfpathlineto{\pgfqpoint{-7.085212in}{0.773588in}}%
\pgfpathlineto{\pgfqpoint{-7.033318in}{0.773588in}}%
\pgfpathlineto{\pgfqpoint{-6.984061in}{0.773588in}}%
\pgfpathlineto{\pgfqpoint{-6.934169in}{0.773588in}}%
\pgfpathlineto{\pgfqpoint{-6.882593in}{0.773588in}}%
\pgfpathlineto{\pgfqpoint{-6.832072in}{0.773588in}}%
\pgfpathlineto{\pgfqpoint{-6.782488in}{0.773588in}}%
\pgfpathlineto{\pgfqpoint{-6.731323in}{0.773588in}}%
\pgfpathlineto{\pgfqpoint{-6.682747in}{0.773588in}}%
\pgfpathlineto{\pgfqpoint{-6.634445in}{0.773588in}}%
\pgfpathlineto{\pgfqpoint{-6.582317in}{0.773588in}}%
\pgfpathlineto{\pgfqpoint{-6.532179in}{0.773588in}}%
\pgfpathlineto{\pgfqpoint{-6.481910in}{0.773588in}}%
\pgfpathlineto{\pgfqpoint{-6.429787in}{0.773588in}}%
\pgfpathlineto{\pgfqpoint{-6.380124in}{0.773588in}}%
\pgfpathlineto{\pgfqpoint{-6.330959in}{0.773588in}}%
\pgfpathlineto{\pgfqpoint{-6.279011in}{0.773588in}}%
\pgfpathlineto{\pgfqpoint{-6.228389in}{0.773588in}}%
\pgfpathlineto{\pgfqpoint{-6.177451in}{0.773588in}}%
\pgfpathlineto{\pgfqpoint{-6.125420in}{0.773588in}}%
\pgfpathlineto{\pgfqpoint{-6.075227in}{0.773588in}}%
\pgfpathlineto{\pgfqpoint{-6.024836in}{0.773588in}}%
\pgfpathlineto{\pgfqpoint{-5.974236in}{0.773588in}}%
\pgfpathlineto{\pgfqpoint{-5.924128in}{0.773588in}}%
\pgfpathlineto{\pgfqpoint{-5.874278in}{0.773588in}}%
\pgfpathlineto{\pgfqpoint{-5.822716in}{0.773588in}}%
\pgfpathlineto{\pgfqpoint{-5.774223in}{0.773588in}}%
\pgfpathlineto{\pgfqpoint{-5.724570in}{0.773588in}}%
\pgfpathlineto{\pgfqpoint{-5.673071in}{0.773588in}}%
\pgfpathlineto{\pgfqpoint{-5.623816in}{0.773588in}}%
\pgfpathlineto{\pgfqpoint{-5.574633in}{0.773588in}}%
\pgfpathlineto{\pgfqpoint{-5.523472in}{0.773588in}}%
\pgfpathlineto{\pgfqpoint{-5.473058in}{0.773588in}}%
\pgfpathlineto{\pgfqpoint{-5.421631in}{0.773588in}}%
\pgfpathlineto{\pgfqpoint{-5.369703in}{0.773588in}}%
\pgfpathlineto{\pgfqpoint{-5.319212in}{0.773588in}}%
\pgfpathlineto{\pgfqpoint{-5.267508in}{0.773588in}}%
\pgfpathlineto{\pgfqpoint{-5.215272in}{0.773588in}}%
\pgfpathlineto{\pgfqpoint{-5.165135in}{0.773588in}}%
\pgfpathlineto{\pgfqpoint{-5.114869in}{0.773588in}}%
\pgfpathlineto{\pgfqpoint{-5.064238in}{0.773588in}}%
\pgfpathlineto{\pgfqpoint{-5.014255in}{0.773588in}}%
\pgfpathlineto{\pgfqpoint{-4.963670in}{0.773588in}}%
\pgfpathlineto{\pgfqpoint{-4.911765in}{0.773588in}}%
\pgfpathlineto{\pgfqpoint{-4.861887in}{0.773588in}}%
\pgfpathlineto{\pgfqpoint{-4.812147in}{0.773588in}}%
\pgfpathlineto{\pgfqpoint{-4.759589in}{0.773588in}}%
\pgfpathlineto{\pgfqpoint{-4.709756in}{0.773588in}}%
\pgfpathlineto{\pgfqpoint{-4.659647in}{0.773588in}}%
\pgfpathlineto{\pgfqpoint{-4.607841in}{0.773588in}}%
\pgfpathlineto{\pgfqpoint{-4.558006in}{0.773588in}}%
\pgfpathlineto{\pgfqpoint{-4.508002in}{0.773588in}}%
\pgfpathlineto{\pgfqpoint{-4.457276in}{0.773588in}}%
\pgfpathlineto{\pgfqpoint{-4.407934in}{0.773588in}}%
\pgfpathlineto{\pgfqpoint{-4.357515in}{0.773588in}}%
\pgfpathlineto{\pgfqpoint{-4.305607in}{0.773588in}}%
\pgfpathlineto{\pgfqpoint{-4.255103in}{0.773588in}}%
\pgfpathlineto{\pgfqpoint{-4.204591in}{0.773588in}}%
\pgfpathlineto{\pgfqpoint{-4.153202in}{0.773588in}}%
\pgfpathlineto{\pgfqpoint{-4.103923in}{0.773588in}}%
\pgfpathlineto{\pgfqpoint{-4.053537in}{0.773588in}}%
\pgfpathlineto{\pgfqpoint{-4.002524in}{0.773588in}}%
\pgfpathlineto{\pgfqpoint{-3.952191in}{0.773588in}}%
\pgfpathlineto{\pgfqpoint{-3.902764in}{0.773588in}}%
\pgfpathlineto{\pgfqpoint{-3.850587in}{0.773588in}}%
\pgfpathlineto{\pgfqpoint{-3.799759in}{0.773588in}}%
\pgfpathlineto{\pgfqpoint{-3.748633in}{0.773588in}}%
\pgfpathlineto{\pgfqpoint{-3.696993in}{0.773588in}}%
\pgfpathlineto{\pgfqpoint{-3.646072in}{0.773588in}}%
\pgfpathlineto{\pgfqpoint{-3.595859in}{0.773588in}}%
\pgfpathlineto{\pgfqpoint{-3.544176in}{0.773588in}}%
\pgfpathlineto{\pgfqpoint{-3.494154in}{0.773588in}}%
\pgfpathlineto{\pgfqpoint{-3.444140in}{0.773588in}}%
\pgfpathlineto{\pgfqpoint{-3.392015in}{0.773588in}}%
\pgfpathlineto{\pgfqpoint{-3.341930in}{0.773588in}}%
\pgfpathlineto{\pgfqpoint{-3.292250in}{0.773588in}}%
\pgfpathlineto{\pgfqpoint{-3.241308in}{0.773588in}}%
\pgfpathlineto{\pgfqpoint{-3.190882in}{0.773588in}}%
\pgfpathlineto{\pgfqpoint{-3.140428in}{0.773588in}}%
\pgfpathlineto{\pgfqpoint{-3.088627in}{0.773588in}}%
\pgfpathlineto{\pgfqpoint{-3.039557in}{0.773588in}}%
\pgfpathlineto{\pgfqpoint{-2.990316in}{0.773588in}}%
\pgfpathlineto{\pgfqpoint{-2.938490in}{0.773588in}}%
\pgfpathlineto{\pgfqpoint{-2.887081in}{0.773588in}}%
\pgfpathlineto{\pgfqpoint{-2.836699in}{0.773588in}}%
\pgfpathlineto{\pgfqpoint{-2.784277in}{0.773588in}}%
\pgfpathlineto{\pgfqpoint{-2.732985in}{0.773588in}}%
\pgfpathlineto{\pgfqpoint{-2.681969in}{0.773588in}}%
\pgfpathlineto{\pgfqpoint{-2.629697in}{0.773588in}}%
\pgfpathlineto{\pgfqpoint{-2.578972in}{0.773588in}}%
\pgfpathlineto{\pgfqpoint{-2.528440in}{0.773588in}}%
\pgfpathlineto{\pgfqpoint{-2.476616in}{0.773588in}}%
\pgfpathlineto{\pgfqpoint{-2.425397in}{0.773588in}}%
\pgfpathlineto{\pgfqpoint{-2.375607in}{0.773588in}}%
\pgfpathlineto{\pgfqpoint{-2.323084in}{0.773588in}}%
\pgfpathlineto{\pgfqpoint{-2.272195in}{0.773588in}}%
\pgfpathlineto{\pgfqpoint{-2.220871in}{0.773588in}}%
\pgfpathlineto{\pgfqpoint{-2.167559in}{0.773588in}}%
\pgfpathlineto{\pgfqpoint{-2.116467in}{0.773588in}}%
\pgfpathlineto{\pgfqpoint{-2.064986in}{0.773588in}}%
\pgfpathlineto{\pgfqpoint{-2.013282in}{0.773588in}}%
\pgfpathlineto{\pgfqpoint{-1.963080in}{0.773588in}}%
\pgfpathlineto{\pgfqpoint{-1.912972in}{0.773588in}}%
\pgfpathlineto{\pgfqpoint{-1.860296in}{0.773588in}}%
\pgfpathlineto{\pgfqpoint{-1.809967in}{0.773588in}}%
\pgfpathlineto{\pgfqpoint{-1.759793in}{0.773588in}}%
\pgfpathlineto{\pgfqpoint{-1.707851in}{0.773588in}}%
\pgfpathlineto{\pgfqpoint{-1.657574in}{0.773588in}}%
\pgfpathlineto{\pgfqpoint{-1.606814in}{0.773588in}}%
\pgfpathlineto{\pgfqpoint{-1.553079in}{0.773588in}}%
\pgfpathlineto{\pgfqpoint{-1.502359in}{0.773588in}}%
\pgfpathlineto{\pgfqpoint{-1.451018in}{0.773588in}}%
\pgfpathlineto{\pgfqpoint{-1.397583in}{0.773588in}}%
\pgfpathlineto{\pgfqpoint{-1.346574in}{0.773588in}}%
\pgfpathlineto{\pgfqpoint{-1.295495in}{0.773588in}}%
\pgfpathlineto{\pgfqpoint{-1.242201in}{0.773588in}}%
\pgfpathlineto{\pgfqpoint{-1.191111in}{0.773588in}}%
\pgfpathlineto{\pgfqpoint{-1.140825in}{0.773588in}}%
\pgfpathlineto{\pgfqpoint{-1.088683in}{0.773588in}}%
\pgfpathlineto{\pgfqpoint{-1.038293in}{0.773588in}}%
\pgfpathlineto{\pgfqpoint{-0.987007in}{0.773588in}}%
\pgfpathlineto{\pgfqpoint{-0.934193in}{0.773588in}}%
\pgfpathlineto{\pgfqpoint{-0.883439in}{0.773588in}}%
\pgfpathlineto{\pgfqpoint{-0.832397in}{0.773588in}}%
\pgfpathlineto{\pgfqpoint{-0.780794in}{0.773588in}}%
\pgfpathlineto{\pgfqpoint{-0.729960in}{0.773588in}}%
\pgfpathlineto{\pgfqpoint{-0.679845in}{0.773588in}}%
\pgfpathlineto{\pgfqpoint{-0.628072in}{0.773588in}}%
\pgfpathlineto{\pgfqpoint{-0.577558in}{0.773588in}}%
\pgfpathlineto{\pgfqpoint{-0.525798in}{0.773588in}}%
\pgfpathlineto{\pgfqpoint{-0.472607in}{0.773588in}}%
\pgfpathlineto{\pgfqpoint{-0.421082in}{0.773588in}}%
\pgfpathlineto{\pgfqpoint{-0.370494in}{0.773588in}}%
\pgfpathlineto{\pgfqpoint{-0.317282in}{0.773588in}}%
\pgfpathlineto{\pgfqpoint{-0.265117in}{0.773588in}}%
\pgfpathlineto{\pgfqpoint{-0.212446in}{0.773588in}}%
\pgfpathlineto{\pgfqpoint{-0.159327in}{0.773588in}}%
\pgfpathlineto{\pgfqpoint{-0.106747in}{0.773588in}}%
\pgfpathlineto{\pgfqpoint{-0.053884in}{0.773588in}}%
\pgfpathlineto{\pgfqpoint{0.000028in}{0.773588in}}%
\pgfpathlineto{\pgfqpoint{0.052190in}{0.773588in}}%
\pgfpathlineto{\pgfqpoint{0.103930in}{0.773588in}}%
\pgfpathlineto{\pgfqpoint{0.157615in}{0.773588in}}%
\pgfpathlineto{\pgfqpoint{0.209933in}{0.773588in}}%
\pgfpathlineto{\pgfqpoint{0.261843in}{0.773588in}}%
\pgfpathlineto{\pgfqpoint{0.315264in}{0.773588in}}%
\pgfpathlineto{\pgfqpoint{0.366948in}{0.773588in}}%
\pgfpathlineto{\pgfqpoint{0.419063in}{0.773588in}}%
\pgfpathlineto{\pgfqpoint{0.472767in}{0.773588in}}%
\pgfpathlineto{\pgfqpoint{0.524412in}{0.773588in}}%
\pgfpathlineto{\pgfqpoint{0.575921in}{0.773588in}}%
\pgfpathlineto{\pgfqpoint{0.628822in}{0.773588in}}%
\pgfpathlineto{\pgfqpoint{0.680101in}{0.773588in}}%
\pgfpathlineto{\pgfqpoint{0.730937in}{0.773588in}}%
\pgfpathlineto{\pgfqpoint{0.783351in}{0.773588in}}%
\pgfpathlineto{\pgfqpoint{0.835638in}{0.773588in}}%
\pgfpathlineto{\pgfqpoint{0.890559in}{0.773588in}}%
\pgfpathlineto{\pgfqpoint{0.950896in}{0.773588in}}%
\pgfpathlineto{\pgfqpoint{1.011418in}{0.773588in}}%
\pgfpathlineto{\pgfqpoint{1.071223in}{0.773588in}}%
\pgfpathlineto{\pgfqpoint{1.133818in}{0.773588in}}%
\pgfpathlineto{\pgfqpoint{1.196854in}{0.773588in}}%
\pgfpathlineto{\pgfqpoint{1.262232in}{0.773588in}}%
\pgfpathlineto{\pgfqpoint{1.331848in}{0.773588in}}%
\pgfpathlineto{\pgfqpoint{1.399923in}{0.773588in}}%
\pgfpathlineto{\pgfqpoint{1.469539in}{0.773588in}}%
\pgfpathlineto{\pgfqpoint{1.542225in}{0.773588in}}%
\pgfpathlineto{\pgfqpoint{1.614592in}{0.773588in}}%
\pgfpathlineto{\pgfqpoint{1.687208in}{0.773588in}}%
\pgfpathlineto{\pgfqpoint{1.764147in}{0.773588in}}%
\pgfpathlineto{\pgfqpoint{1.839330in}{0.773588in}}%
\pgfpathlineto{\pgfqpoint{1.918401in}{0.773588in}}%
\pgfpathlineto{\pgfqpoint{1.999401in}{0.773588in}}%
\pgfpathlineto{\pgfqpoint{2.077252in}{0.773588in}}%
\pgfpathlineto{\pgfqpoint{2.157128in}{0.773588in}}%
\pgfpathlineto{\pgfqpoint{2.243381in}{0.773588in}}%
\pgfpathlineto{\pgfqpoint{2.328903in}{0.773588in}}%
\pgfpathlineto{\pgfqpoint{2.416579in}{0.773588in}}%
\pgfpathlineto{\pgfqpoint{2.504252in}{0.773588in}}%
\pgfpathlineto{\pgfqpoint{2.589494in}{0.773588in}}%
\pgfpathlineto{\pgfqpoint{2.676066in}{0.773588in}}%
\pgfpathlineto{\pgfqpoint{2.767061in}{0.773588in}}%
\pgfpathlineto{\pgfqpoint{2.858701in}{0.773588in}}%
\pgfpathlineto{\pgfqpoint{2.948242in}{0.773588in}}%
\pgfpathlineto{\pgfqpoint{3.043068in}{0.773588in}}%
\pgfpathlineto{\pgfqpoint{3.132459in}{0.773588in}}%
\pgfpathlineto{\pgfqpoint{3.198420in}{0.773588in}}%
\pgfpathlineto{\pgfqpoint{3.252590in}{0.773588in}}%
\pgfpathlineto{\pgfqpoint{3.305067in}{0.773588in}}%
\pgfpathlineto{\pgfqpoint{3.357008in}{0.773588in}}%
\pgfpathlineto{\pgfqpoint{3.411084in}{0.773588in}}%
\pgfpathlineto{\pgfqpoint{3.464430in}{0.773588in}}%
\pgfpathlineto{\pgfqpoint{3.517522in}{0.773588in}}%
\pgfpathlineto{\pgfqpoint{3.571722in}{0.773588in}}%
\pgfpathlineto{\pgfqpoint{3.624008in}{0.773588in}}%
\pgfpathlineto{\pgfqpoint{3.676588in}{0.773588in}}%
\pgfpathlineto{\pgfqpoint{3.729373in}{0.773588in}}%
\pgfpathlineto{\pgfqpoint{3.781554in}{0.773588in}}%
\pgfpathlineto{\pgfqpoint{3.833685in}{0.773588in}}%
\pgfpathlineto{\pgfqpoint{3.887639in}{0.773588in}}%
\pgfpathlineto{\pgfqpoint{3.940507in}{0.773588in}}%
\pgfpathlineto{\pgfqpoint{3.993337in}{0.773588in}}%
\pgfpathlineto{\pgfqpoint{4.047138in}{0.773588in}}%
\pgfpathlineto{\pgfqpoint{4.098630in}{0.773588in}}%
\pgfpathlineto{\pgfqpoint{4.150734in}{0.773588in}}%
\pgfpathlineto{\pgfqpoint{4.204360in}{0.773588in}}%
\pgfpathlineto{\pgfqpoint{4.256228in}{0.773588in}}%
\pgfpathlineto{\pgfqpoint{4.307535in}{0.773588in}}%
\pgfpathlineto{\pgfqpoint{4.360055in}{0.773588in}}%
\pgfpathlineto{\pgfqpoint{4.398305in}{0.773588in}}%
\pgfpathlineto{\pgfqpoint{4.446119in}{0.773588in}}%
\pgfpathlineto{\pgfqpoint{4.484596in}{1.025897in}}%
\pgfpathlineto{\pgfqpoint{4.526846in}{1.089794in}}%
\pgfpathlineto{\pgfqpoint{4.566686in}{1.158641in}}%
\pgfpathlineto{\pgfqpoint{4.603496in}{1.243236in}}%
\pgfpathlineto{\pgfqpoint{4.635492in}{1.503451in}}%
\pgfpathlineto{\pgfqpoint{4.666016in}{1.799827in}}%
\pgfpathlineto{\pgfqpoint{4.690368in}{2.293678in}}%
\pgfpathlineto{\pgfqpoint{4.715513in}{2.218952in}}%
\pgfpathlineto{\pgfqpoint{4.739291in}{2.290107in}}%
\pgfpathlineto{\pgfqpoint{4.764034in}{2.446672in}}%
\pgfpathlineto{\pgfqpoint{4.787716in}{2.339022in}}%
\pgfpathlineto{\pgfqpoint{4.811295in}{2.332444in}}%
\pgfpathlineto{\pgfqpoint{4.836687in}{2.257032in}}%
\pgfpathlineto{\pgfqpoint{4.860218in}{2.316562in}}%
\pgfpathlineto{\pgfqpoint{4.884852in}{2.267974in}}%
\pgfpathlineto{\pgfqpoint{4.907640in}{2.361353in}}%
\pgfpathlineto{\pgfqpoint{4.931890in}{2.449950in}}%
\pgfpathlineto{\pgfqpoint{4.954839in}{2.394574in}}%
\pgfpathlineto{\pgfqpoint{4.980174in}{2.271817in}}%
\pgfpathlineto{\pgfqpoint{5.002737in}{2.287340in}}%
\pgfpathlineto{\pgfqpoint{5.026810in}{2.232666in}}%
\pgfpathlineto{\pgfqpoint{5.051612in}{2.338107in}}%
\pgfpathlineto{\pgfqpoint{5.074798in}{2.469518in}}%
\pgfpathlineto{\pgfqpoint{5.097977in}{2.433623in}}%
\pgfpathlineto{\pgfqpoint{5.122448in}{2.282667in}}%
\pgfpathlineto{\pgfqpoint{5.145219in}{2.411536in}}%
\pgfpathlineto{\pgfqpoint{5.168068in}{2.434395in}}%
\pgfpathlineto{\pgfqpoint{5.192095in}{2.553834in}}%
\pgfpathlineto{\pgfqpoint{5.214897in}{2.526820in}}%
\pgfpathlineto{\pgfqpoint{5.237842in}{2.434375in}}%
\pgfpathlineto{\pgfqpoint{5.261861in}{2.409123in}}%
\pgfpathlineto{\pgfqpoint{5.285086in}{2.380016in}}%
\pgfpathlineto{\pgfqpoint{5.307824in}{2.578334in}}%
\pgfpathlineto{\pgfqpoint{5.331834in}{2.431018in}}%
\pgfpathlineto{\pgfqpoint{5.355254in}{2.431345in}}%
\pgfpathlineto{\pgfqpoint{5.377738in}{2.581556in}}%
\pgfpathlineto{\pgfqpoint{5.402166in}{2.457377in}}%
\pgfpathlineto{\pgfqpoint{5.424187in}{2.469111in}}%
\pgfpathlineto{\pgfqpoint{5.448001in}{2.475980in}}%
\pgfpathlineto{\pgfqpoint{5.470431in}{2.570004in}}%
\pgfpathlineto{\pgfqpoint{5.494250in}{2.522929in}}%
\pgfpathlineto{\pgfqpoint{5.516676in}{2.477526in}}%
\pgfpathlineto{\pgfqpoint{5.540659in}{2.468813in}}%
\pgfpathlineto{\pgfqpoint{5.562835in}{2.587240in}}%
\pgfpathlineto{\pgfqpoint{5.587139in}{2.403966in}}%
\pgfpathlineto{\pgfqpoint{5.609949in}{2.424295in}}%
\pgfpathlineto{\pgfqpoint{5.634039in}{2.470556in}}%
\pgfpathlineto{\pgfqpoint{5.661246in}{2.423078in}}%
\pgfpathlineto{\pgfqpoint{5.712060in}{2.413621in}}%
\pgfpathlineto{\pgfqpoint{5.763148in}{2.413621in}}%
\pgfpathlineto{\pgfqpoint{5.815025in}{2.413621in}}%
\pgfpathlineto{\pgfqpoint{5.867920in}{2.413621in}}%
\pgfpathlineto{\pgfqpoint{5.919631in}{2.413621in}}%
\pgfpathlineto{\pgfqpoint{5.971565in}{2.413621in}}%
\pgfpathlineto{\pgfqpoint{6.025764in}{2.413621in}}%
\pgfpathlineto{\pgfqpoint{6.078797in}{2.413621in}}%
\pgfpathlineto{\pgfqpoint{6.078797in}{3.993842in}}%
\pgfpathlineto{\pgfqpoint{6.078797in}{3.993842in}}%
\pgfpathlineto{\pgfqpoint{6.025764in}{3.993842in}}%
\pgfpathlineto{\pgfqpoint{5.971565in}{3.993842in}}%
\pgfpathlineto{\pgfqpoint{5.919631in}{3.993842in}}%
\pgfpathlineto{\pgfqpoint{5.867920in}{3.993842in}}%
\pgfpathlineto{\pgfqpoint{5.815025in}{3.993842in}}%
\pgfpathlineto{\pgfqpoint{5.763148in}{3.993842in}}%
\pgfpathlineto{\pgfqpoint{5.712060in}{3.993842in}}%
\pgfpathlineto{\pgfqpoint{5.661246in}{4.038936in}}%
\pgfpathlineto{\pgfqpoint{5.634039in}{4.182959in}}%
\pgfpathlineto{\pgfqpoint{5.609949in}{4.101072in}}%
\pgfpathlineto{\pgfqpoint{5.587139in}{3.989480in}}%
\pgfpathlineto{\pgfqpoint{5.562835in}{4.239053in}}%
\pgfpathlineto{\pgfqpoint{5.540659in}{4.231403in}}%
\pgfpathlineto{\pgfqpoint{5.516676in}{4.198446in}}%
\pgfpathlineto{\pgfqpoint{5.494250in}{4.201901in}}%
\pgfpathlineto{\pgfqpoint{5.470431in}{4.232916in}}%
\pgfpathlineto{\pgfqpoint{5.448001in}{4.201329in}}%
\pgfpathlineto{\pgfqpoint{5.424187in}{4.192246in}}%
\pgfpathlineto{\pgfqpoint{5.402166in}{4.142544in}}%
\pgfpathlineto{\pgfqpoint{5.377738in}{4.251407in}}%
\pgfpathlineto{\pgfqpoint{5.355254in}{4.195680in}}%
\pgfpathlineto{\pgfqpoint{5.331834in}{4.168563in}}%
\pgfpathlineto{\pgfqpoint{5.307824in}{4.254313in}}%
\pgfpathlineto{\pgfqpoint{5.285086in}{3.997581in}}%
\pgfpathlineto{\pgfqpoint{5.261861in}{4.071453in}}%
\pgfpathlineto{\pgfqpoint{5.237842in}{4.124102in}}%
\pgfpathlineto{\pgfqpoint{5.214897in}{4.217950in}}%
\pgfpathlineto{\pgfqpoint{5.192095in}{4.194563in}}%
\pgfpathlineto{\pgfqpoint{5.168068in}{4.127126in}}%
\pgfpathlineto{\pgfqpoint{5.145219in}{4.171934in}}%
\pgfpathlineto{\pgfqpoint{5.122448in}{3.944352in}}%
\pgfpathlineto{\pgfqpoint{5.097977in}{4.134872in}}%
\pgfpathlineto{\pgfqpoint{5.074798in}{4.217964in}}%
\pgfpathlineto{\pgfqpoint{5.051612in}{3.898493in}}%
\pgfpathlineto{\pgfqpoint{5.026810in}{3.769623in}}%
\pgfpathlineto{\pgfqpoint{5.002737in}{3.916080in}}%
\pgfpathlineto{\pgfqpoint{4.980174in}{3.888388in}}%
\pgfpathlineto{\pgfqpoint{4.954839in}{3.972416in}}%
\pgfpathlineto{\pgfqpoint{4.931890in}{4.075356in}}%
\pgfpathlineto{\pgfqpoint{4.907640in}{3.930055in}}%
\pgfpathlineto{\pgfqpoint{4.884852in}{3.877862in}}%
\pgfpathlineto{\pgfqpoint{4.860218in}{4.040966in}}%
\pgfpathlineto{\pgfqpoint{4.836687in}{3.988507in}}%
\pgfpathlineto{\pgfqpoint{4.811295in}{3.869449in}}%
\pgfpathlineto{\pgfqpoint{4.787716in}{3.942672in}}%
\pgfpathlineto{\pgfqpoint{4.764034in}{3.941303in}}%
\pgfpathlineto{\pgfqpoint{4.739291in}{3.898067in}}%
\pgfpathlineto{\pgfqpoint{4.715513in}{3.612521in}}%
\pgfpathlineto{\pgfqpoint{4.690368in}{3.753732in}}%
\pgfpathlineto{\pgfqpoint{4.666016in}{2.790032in}}%
\pgfpathlineto{\pgfqpoint{4.635492in}{2.208863in}}%
\pgfpathlineto{\pgfqpoint{4.603496in}{1.768315in}}%
\pgfpathlineto{\pgfqpoint{4.566686in}{1.533299in}}%
\pgfpathlineto{\pgfqpoint{4.526846in}{1.377266in}}%
\pgfpathlineto{\pgfqpoint{4.484596in}{1.246842in}}%
\pgfpathlineto{\pgfqpoint{4.446119in}{0.773588in}}%
\pgfpathlineto{\pgfqpoint{4.398305in}{0.773588in}}%
\pgfpathlineto{\pgfqpoint{4.360055in}{0.773588in}}%
\pgfpathlineto{\pgfqpoint{4.307535in}{0.773588in}}%
\pgfpathlineto{\pgfqpoint{4.256228in}{0.773588in}}%
\pgfpathlineto{\pgfqpoint{4.204360in}{0.773588in}}%
\pgfpathlineto{\pgfqpoint{4.150734in}{0.773588in}}%
\pgfpathlineto{\pgfqpoint{4.098630in}{0.773588in}}%
\pgfpathlineto{\pgfqpoint{4.047138in}{0.773588in}}%
\pgfpathlineto{\pgfqpoint{3.993337in}{0.773588in}}%
\pgfpathlineto{\pgfqpoint{3.940507in}{0.773588in}}%
\pgfpathlineto{\pgfqpoint{3.887639in}{0.773588in}}%
\pgfpathlineto{\pgfqpoint{3.833685in}{0.773588in}}%
\pgfpathlineto{\pgfqpoint{3.781554in}{0.773588in}}%
\pgfpathlineto{\pgfqpoint{3.729373in}{0.773588in}}%
\pgfpathlineto{\pgfqpoint{3.676588in}{0.773588in}}%
\pgfpathlineto{\pgfqpoint{3.624008in}{0.773588in}}%
\pgfpathlineto{\pgfqpoint{3.571722in}{0.773588in}}%
\pgfpathlineto{\pgfqpoint{3.517522in}{0.773588in}}%
\pgfpathlineto{\pgfqpoint{3.464430in}{0.773588in}}%
\pgfpathlineto{\pgfqpoint{3.411084in}{0.773588in}}%
\pgfpathlineto{\pgfqpoint{3.357008in}{0.773588in}}%
\pgfpathlineto{\pgfqpoint{3.305067in}{0.773588in}}%
\pgfpathlineto{\pgfqpoint{3.252590in}{0.773588in}}%
\pgfpathlineto{\pgfqpoint{3.198420in}{0.773588in}}%
\pgfpathlineto{\pgfqpoint{3.132459in}{0.773588in}}%
\pgfpathlineto{\pgfqpoint{3.043068in}{0.773588in}}%
\pgfpathlineto{\pgfqpoint{2.948242in}{0.773588in}}%
\pgfpathlineto{\pgfqpoint{2.858701in}{0.773588in}}%
\pgfpathlineto{\pgfqpoint{2.767061in}{0.773588in}}%
\pgfpathlineto{\pgfqpoint{2.676066in}{0.773588in}}%
\pgfpathlineto{\pgfqpoint{2.589494in}{0.773588in}}%
\pgfpathlineto{\pgfqpoint{2.504252in}{0.773588in}}%
\pgfpathlineto{\pgfqpoint{2.416579in}{0.773588in}}%
\pgfpathlineto{\pgfqpoint{2.328903in}{0.773588in}}%
\pgfpathlineto{\pgfqpoint{2.243381in}{0.773588in}}%
\pgfpathlineto{\pgfqpoint{2.157128in}{0.773588in}}%
\pgfpathlineto{\pgfqpoint{2.077252in}{0.773588in}}%
\pgfpathlineto{\pgfqpoint{1.999401in}{0.773588in}}%
\pgfpathlineto{\pgfqpoint{1.918401in}{0.773588in}}%
\pgfpathlineto{\pgfqpoint{1.839330in}{0.773588in}}%
\pgfpathlineto{\pgfqpoint{1.764147in}{0.773588in}}%
\pgfpathlineto{\pgfqpoint{1.687208in}{0.773588in}}%
\pgfpathlineto{\pgfqpoint{1.614592in}{0.773588in}}%
\pgfpathlineto{\pgfqpoint{1.542225in}{0.773588in}}%
\pgfpathlineto{\pgfqpoint{1.469539in}{0.773588in}}%
\pgfpathlineto{\pgfqpoint{1.399923in}{0.773588in}}%
\pgfpathlineto{\pgfqpoint{1.331848in}{0.773588in}}%
\pgfpathlineto{\pgfqpoint{1.262232in}{0.773588in}}%
\pgfpathlineto{\pgfqpoint{1.196854in}{0.773588in}}%
\pgfpathlineto{\pgfqpoint{1.133818in}{0.773588in}}%
\pgfpathlineto{\pgfqpoint{1.071223in}{0.773588in}}%
\pgfpathlineto{\pgfqpoint{1.011418in}{0.773588in}}%
\pgfpathlineto{\pgfqpoint{0.950896in}{0.773588in}}%
\pgfpathlineto{\pgfqpoint{0.890559in}{0.773588in}}%
\pgfpathlineto{\pgfqpoint{0.835638in}{0.773588in}}%
\pgfpathlineto{\pgfqpoint{0.783351in}{0.773588in}}%
\pgfpathlineto{\pgfqpoint{0.730937in}{0.773588in}}%
\pgfpathlineto{\pgfqpoint{0.680101in}{0.773588in}}%
\pgfpathlineto{\pgfqpoint{0.628822in}{0.773588in}}%
\pgfpathlineto{\pgfqpoint{0.575921in}{0.773588in}}%
\pgfpathlineto{\pgfqpoint{0.524412in}{0.773588in}}%
\pgfpathlineto{\pgfqpoint{0.472767in}{0.773588in}}%
\pgfpathlineto{\pgfqpoint{0.419063in}{0.773588in}}%
\pgfpathlineto{\pgfqpoint{0.366948in}{0.773588in}}%
\pgfpathlineto{\pgfqpoint{0.315264in}{0.773588in}}%
\pgfpathlineto{\pgfqpoint{0.261843in}{0.773588in}}%
\pgfpathlineto{\pgfqpoint{0.209933in}{0.773588in}}%
\pgfpathlineto{\pgfqpoint{0.157615in}{0.773588in}}%
\pgfpathlineto{\pgfqpoint{0.103930in}{0.773588in}}%
\pgfpathlineto{\pgfqpoint{0.052190in}{0.773588in}}%
\pgfpathlineto{\pgfqpoint{0.000028in}{0.773588in}}%
\pgfpathlineto{\pgfqpoint{-0.053884in}{0.773588in}}%
\pgfpathlineto{\pgfqpoint{-0.106747in}{0.773588in}}%
\pgfpathlineto{\pgfqpoint{-0.159327in}{0.773588in}}%
\pgfpathlineto{\pgfqpoint{-0.212446in}{0.773588in}}%
\pgfpathlineto{\pgfqpoint{-0.265117in}{0.773588in}}%
\pgfpathlineto{\pgfqpoint{-0.317282in}{0.773588in}}%
\pgfpathlineto{\pgfqpoint{-0.370494in}{0.773588in}}%
\pgfpathlineto{\pgfqpoint{-0.421082in}{0.773588in}}%
\pgfpathlineto{\pgfqpoint{-0.472607in}{0.773588in}}%
\pgfpathlineto{\pgfqpoint{-0.525798in}{0.773588in}}%
\pgfpathlineto{\pgfqpoint{-0.577558in}{0.773588in}}%
\pgfpathlineto{\pgfqpoint{-0.628072in}{0.773588in}}%
\pgfpathlineto{\pgfqpoint{-0.679845in}{0.773588in}}%
\pgfpathlineto{\pgfqpoint{-0.729960in}{0.773588in}}%
\pgfpathlineto{\pgfqpoint{-0.780794in}{0.773588in}}%
\pgfpathlineto{\pgfqpoint{-0.832397in}{0.773588in}}%
\pgfpathlineto{\pgfqpoint{-0.883439in}{0.773588in}}%
\pgfpathlineto{\pgfqpoint{-0.934193in}{0.773588in}}%
\pgfpathlineto{\pgfqpoint{-0.987007in}{0.773588in}}%
\pgfpathlineto{\pgfqpoint{-1.038293in}{0.773588in}}%
\pgfpathlineto{\pgfqpoint{-1.088683in}{0.773588in}}%
\pgfpathlineto{\pgfqpoint{-1.140825in}{0.773588in}}%
\pgfpathlineto{\pgfqpoint{-1.191111in}{0.773588in}}%
\pgfpathlineto{\pgfqpoint{-1.242201in}{0.773588in}}%
\pgfpathlineto{\pgfqpoint{-1.295495in}{0.773588in}}%
\pgfpathlineto{\pgfqpoint{-1.346574in}{0.773588in}}%
\pgfpathlineto{\pgfqpoint{-1.397583in}{0.773588in}}%
\pgfpathlineto{\pgfqpoint{-1.451018in}{0.773588in}}%
\pgfpathlineto{\pgfqpoint{-1.502359in}{0.773588in}}%
\pgfpathlineto{\pgfqpoint{-1.553079in}{0.773588in}}%
\pgfpathlineto{\pgfqpoint{-1.606814in}{0.773588in}}%
\pgfpathlineto{\pgfqpoint{-1.657574in}{0.773588in}}%
\pgfpathlineto{\pgfqpoint{-1.707851in}{0.773588in}}%
\pgfpathlineto{\pgfqpoint{-1.759793in}{0.773588in}}%
\pgfpathlineto{\pgfqpoint{-1.809967in}{0.773588in}}%
\pgfpathlineto{\pgfqpoint{-1.860296in}{0.773588in}}%
\pgfpathlineto{\pgfqpoint{-1.912972in}{0.773588in}}%
\pgfpathlineto{\pgfqpoint{-1.963080in}{0.773588in}}%
\pgfpathlineto{\pgfqpoint{-2.013282in}{0.773588in}}%
\pgfpathlineto{\pgfqpoint{-2.064986in}{0.773588in}}%
\pgfpathlineto{\pgfqpoint{-2.116467in}{0.773588in}}%
\pgfpathlineto{\pgfqpoint{-2.167559in}{0.773588in}}%
\pgfpathlineto{\pgfqpoint{-2.220871in}{0.773588in}}%
\pgfpathlineto{\pgfqpoint{-2.272195in}{0.773588in}}%
\pgfpathlineto{\pgfqpoint{-2.323084in}{0.773588in}}%
\pgfpathlineto{\pgfqpoint{-2.375607in}{0.773588in}}%
\pgfpathlineto{\pgfqpoint{-2.425397in}{0.773588in}}%
\pgfpathlineto{\pgfqpoint{-2.476616in}{0.773588in}}%
\pgfpathlineto{\pgfqpoint{-2.528440in}{0.773588in}}%
\pgfpathlineto{\pgfqpoint{-2.578972in}{0.773588in}}%
\pgfpathlineto{\pgfqpoint{-2.629697in}{0.773588in}}%
\pgfpathlineto{\pgfqpoint{-2.681969in}{0.773588in}}%
\pgfpathlineto{\pgfqpoint{-2.732985in}{0.773588in}}%
\pgfpathlineto{\pgfqpoint{-2.784277in}{0.773588in}}%
\pgfpathlineto{\pgfqpoint{-2.836699in}{0.773588in}}%
\pgfpathlineto{\pgfqpoint{-2.887081in}{0.773588in}}%
\pgfpathlineto{\pgfqpoint{-2.938490in}{0.773588in}}%
\pgfpathlineto{\pgfqpoint{-2.990316in}{0.773588in}}%
\pgfpathlineto{\pgfqpoint{-3.039557in}{0.773588in}}%
\pgfpathlineto{\pgfqpoint{-3.088627in}{0.773588in}}%
\pgfpathlineto{\pgfqpoint{-3.140428in}{0.773588in}}%
\pgfpathlineto{\pgfqpoint{-3.190882in}{0.773588in}}%
\pgfpathlineto{\pgfqpoint{-3.241308in}{0.773588in}}%
\pgfpathlineto{\pgfqpoint{-3.292250in}{0.773588in}}%
\pgfpathlineto{\pgfqpoint{-3.341930in}{0.773588in}}%
\pgfpathlineto{\pgfqpoint{-3.392015in}{0.773588in}}%
\pgfpathlineto{\pgfqpoint{-3.444140in}{0.773588in}}%
\pgfpathlineto{\pgfqpoint{-3.494154in}{0.773588in}}%
\pgfpathlineto{\pgfqpoint{-3.544176in}{0.773588in}}%
\pgfpathlineto{\pgfqpoint{-3.595859in}{0.773588in}}%
\pgfpathlineto{\pgfqpoint{-3.646072in}{0.773588in}}%
\pgfpathlineto{\pgfqpoint{-3.696993in}{0.773588in}}%
\pgfpathlineto{\pgfqpoint{-3.748633in}{0.773588in}}%
\pgfpathlineto{\pgfqpoint{-3.799759in}{0.773588in}}%
\pgfpathlineto{\pgfqpoint{-3.850587in}{0.773588in}}%
\pgfpathlineto{\pgfqpoint{-3.902764in}{0.773588in}}%
\pgfpathlineto{\pgfqpoint{-3.952191in}{0.773588in}}%
\pgfpathlineto{\pgfqpoint{-4.002524in}{0.773588in}}%
\pgfpathlineto{\pgfqpoint{-4.053537in}{0.773588in}}%
\pgfpathlineto{\pgfqpoint{-4.103923in}{0.773588in}}%
\pgfpathlineto{\pgfqpoint{-4.153202in}{0.773588in}}%
\pgfpathlineto{\pgfqpoint{-4.204591in}{0.773588in}}%
\pgfpathlineto{\pgfqpoint{-4.255103in}{0.773588in}}%
\pgfpathlineto{\pgfqpoint{-4.305607in}{0.773588in}}%
\pgfpathlineto{\pgfqpoint{-4.357515in}{0.773588in}}%
\pgfpathlineto{\pgfqpoint{-4.407934in}{0.773588in}}%
\pgfpathlineto{\pgfqpoint{-4.457276in}{0.773588in}}%
\pgfpathlineto{\pgfqpoint{-4.508002in}{0.773588in}}%
\pgfpathlineto{\pgfqpoint{-4.558006in}{0.773588in}}%
\pgfpathlineto{\pgfqpoint{-4.607841in}{0.773588in}}%
\pgfpathlineto{\pgfqpoint{-4.659647in}{0.773588in}}%
\pgfpathlineto{\pgfqpoint{-4.709756in}{0.773588in}}%
\pgfpathlineto{\pgfqpoint{-4.759589in}{0.773588in}}%
\pgfpathlineto{\pgfqpoint{-4.812147in}{0.773588in}}%
\pgfpathlineto{\pgfqpoint{-4.861887in}{0.773588in}}%
\pgfpathlineto{\pgfqpoint{-4.911765in}{0.773588in}}%
\pgfpathlineto{\pgfqpoint{-4.963670in}{0.773588in}}%
\pgfpathlineto{\pgfqpoint{-5.014255in}{0.773588in}}%
\pgfpathlineto{\pgfqpoint{-5.064238in}{0.773588in}}%
\pgfpathlineto{\pgfqpoint{-5.114869in}{0.773588in}}%
\pgfpathlineto{\pgfqpoint{-5.165135in}{0.773588in}}%
\pgfpathlineto{\pgfqpoint{-5.215272in}{0.773588in}}%
\pgfpathlineto{\pgfqpoint{-5.267508in}{0.773588in}}%
\pgfpathlineto{\pgfqpoint{-5.319212in}{0.773588in}}%
\pgfpathlineto{\pgfqpoint{-5.369703in}{0.773588in}}%
\pgfpathlineto{\pgfqpoint{-5.421631in}{0.773588in}}%
\pgfpathlineto{\pgfqpoint{-5.473058in}{0.773588in}}%
\pgfpathlineto{\pgfqpoint{-5.523472in}{0.773588in}}%
\pgfpathlineto{\pgfqpoint{-5.574633in}{0.773588in}}%
\pgfpathlineto{\pgfqpoint{-5.623816in}{0.773588in}}%
\pgfpathlineto{\pgfqpoint{-5.673071in}{0.773588in}}%
\pgfpathlineto{\pgfqpoint{-5.724570in}{0.773588in}}%
\pgfpathlineto{\pgfqpoint{-5.774223in}{0.773588in}}%
\pgfpathlineto{\pgfqpoint{-5.822716in}{0.773588in}}%
\pgfpathlineto{\pgfqpoint{-5.874278in}{0.773588in}}%
\pgfpathlineto{\pgfqpoint{-5.924128in}{0.773588in}}%
\pgfpathlineto{\pgfqpoint{-5.974236in}{0.773588in}}%
\pgfpathlineto{\pgfqpoint{-6.024836in}{0.773588in}}%
\pgfpathlineto{\pgfqpoint{-6.075227in}{0.773588in}}%
\pgfpathlineto{\pgfqpoint{-6.125420in}{0.773588in}}%
\pgfpathlineto{\pgfqpoint{-6.177451in}{0.773588in}}%
\pgfpathlineto{\pgfqpoint{-6.228389in}{0.773588in}}%
\pgfpathlineto{\pgfqpoint{-6.279011in}{0.773588in}}%
\pgfpathlineto{\pgfqpoint{-6.330959in}{0.773588in}}%
\pgfpathlineto{\pgfqpoint{-6.380124in}{0.773588in}}%
\pgfpathlineto{\pgfqpoint{-6.429787in}{0.773588in}}%
\pgfpathlineto{\pgfqpoint{-6.481910in}{0.773588in}}%
\pgfpathlineto{\pgfqpoint{-6.532179in}{0.773588in}}%
\pgfpathlineto{\pgfqpoint{-6.582317in}{0.773588in}}%
\pgfpathlineto{\pgfqpoint{-6.634445in}{0.773588in}}%
\pgfpathlineto{\pgfqpoint{-6.682747in}{0.773588in}}%
\pgfpathlineto{\pgfqpoint{-6.731323in}{0.773588in}}%
\pgfpathlineto{\pgfqpoint{-6.782488in}{0.773588in}}%
\pgfpathlineto{\pgfqpoint{-6.832072in}{0.773588in}}%
\pgfpathlineto{\pgfqpoint{-6.882593in}{0.773588in}}%
\pgfpathlineto{\pgfqpoint{-6.934169in}{0.773588in}}%
\pgfpathlineto{\pgfqpoint{-6.984061in}{0.773588in}}%
\pgfpathlineto{\pgfqpoint{-7.033318in}{0.773588in}}%
\pgfpathlineto{\pgfqpoint{-7.085212in}{0.773588in}}%
\pgfpathlineto{\pgfqpoint{-7.135511in}{0.773588in}}%
\pgfpathlineto{\pgfqpoint{-7.185558in}{0.773588in}}%
\pgfpathlineto{\pgfqpoint{-7.237328in}{0.773588in}}%
\pgfpathlineto{\pgfqpoint{-7.286238in}{0.773588in}}%
\pgfpathlineto{\pgfqpoint{-7.335791in}{0.773588in}}%
\pgfpathlineto{\pgfqpoint{-7.386343in}{0.773588in}}%
\pgfpathlineto{\pgfqpoint{-7.435602in}{0.773588in}}%
\pgfpathlineto{\pgfqpoint{-7.485312in}{0.773588in}}%
\pgfpathlineto{\pgfqpoint{-7.536566in}{0.773588in}}%
\pgfpathlineto{\pgfqpoint{-7.585672in}{0.773588in}}%
\pgfpathlineto{\pgfqpoint{-7.634521in}{0.773588in}}%
\pgfpathlineto{\pgfqpoint{-7.685281in}{0.773588in}}%
\pgfpathlineto{\pgfqpoint{-7.733870in}{0.773588in}}%
\pgfpathlineto{\pgfqpoint{-7.783607in}{0.773588in}}%
\pgfpathlineto{\pgfqpoint{-7.835188in}{0.773588in}}%
\pgfpathlineto{\pgfqpoint{-7.885122in}{0.773588in}}%
\pgfpathlineto{\pgfqpoint{-7.935315in}{0.773588in}}%
\pgfpathlineto{\pgfqpoint{-7.986412in}{0.773588in}}%
\pgfpathlineto{\pgfqpoint{-8.035792in}{0.773588in}}%
\pgfpathlineto{\pgfqpoint{-8.085190in}{0.773588in}}%
\pgfpathlineto{\pgfqpoint{-8.135490in}{0.773588in}}%
\pgfpathlineto{\pgfqpoint{-8.184462in}{0.773588in}}%
\pgfpathlineto{\pgfqpoint{-8.233427in}{0.773588in}}%
\pgfpathlineto{\pgfqpoint{-8.283775in}{0.773588in}}%
\pgfpathlineto{\pgfqpoint{-8.332630in}{0.773588in}}%
\pgfpathlineto{\pgfqpoint{-8.382452in}{0.773588in}}%
\pgfpathlineto{\pgfqpoint{-8.432560in}{0.773588in}}%
\pgfpathlineto{\pgfqpoint{-8.481981in}{0.773588in}}%
\pgfpathlineto{\pgfqpoint{-8.532027in}{0.773588in}}%
\pgfpathlineto{\pgfqpoint{-8.583181in}{0.773588in}}%
\pgfpathlineto{\pgfqpoint{-8.632588in}{0.773588in}}%
\pgfpathlineto{\pgfqpoint{-8.681529in}{0.773588in}}%
\pgfpathlineto{\pgfqpoint{-8.731477in}{0.773588in}}%
\pgfpathlineto{\pgfqpoint{-8.780578in}{0.773588in}}%
\pgfpathlineto{\pgfqpoint{-8.829855in}{0.773588in}}%
\pgfpathlineto{\pgfqpoint{-8.880937in}{0.773588in}}%
\pgfpathlineto{\pgfqpoint{-8.930541in}{0.773588in}}%
\pgfpathlineto{\pgfqpoint{-8.979731in}{0.773588in}}%
\pgfpathlineto{\pgfqpoint{-9.030683in}{0.773588in}}%
\pgfpathlineto{\pgfqpoint{-9.080221in}{0.773588in}}%
\pgfpathlineto{\pgfqpoint{-9.130552in}{0.773588in}}%
\pgfpathlineto{\pgfqpoint{-9.182531in}{0.773588in}}%
\pgfpathlineto{\pgfqpoint{-9.232466in}{0.773588in}}%
\pgfpathlineto{\pgfqpoint{-9.281795in}{0.773588in}}%
\pgfpathlineto{\pgfqpoint{-9.332833in}{0.773588in}}%
\pgfpathlineto{\pgfqpoint{-9.382453in}{0.773588in}}%
\pgfpathlineto{\pgfqpoint{-9.430893in}{0.773588in}}%
\pgfpathlineto{\pgfqpoint{-9.479530in}{0.773588in}}%
\pgfpathlineto{\pgfqpoint{-9.527213in}{0.773588in}}%
\pgfpathlineto{\pgfqpoint{-9.575433in}{0.773588in}}%
\pgfpathlineto{\pgfqpoint{-9.624467in}{0.773588in}}%
\pgfpathlineto{\pgfqpoint{-9.671880in}{0.773588in}}%
\pgfpathlineto{\pgfqpoint{-9.719205in}{0.773588in}}%
\pgfpathlineto{\pgfqpoint{-9.768438in}{0.773588in}}%
\pgfpathlineto{\pgfqpoint{-9.816398in}{0.773588in}}%
\pgfpathlineto{\pgfqpoint{-9.863853in}{0.773588in}}%
\pgfpathlineto{\pgfqpoint{-9.912991in}{0.773588in}}%
\pgfpathlineto{\pgfqpoint{-9.960967in}{0.773588in}}%
\pgfpathlineto{\pgfqpoint{-10.009255in}{0.773588in}}%
\pgfpathlineto{\pgfqpoint{-10.058316in}{0.773588in}}%
\pgfpathlineto{\pgfqpoint{-10.106404in}{0.773588in}}%
\pgfpathlineto{\pgfqpoint{-10.154319in}{0.773588in}}%
\pgfpathlineto{\pgfqpoint{-10.204843in}{0.773588in}}%
\pgfpathlineto{\pgfqpoint{-10.253421in}{0.773588in}}%
\pgfpathlineto{\pgfqpoint{-10.301938in}{0.773588in}}%
\pgfpathlineto{\pgfqpoint{-10.352351in}{0.773588in}}%
\pgfpathlineto{\pgfqpoint{-10.401144in}{0.773588in}}%
\pgfpathlineto{\pgfqpoint{-10.448474in}{0.773588in}}%
\pgfpathlineto{\pgfqpoint{-10.497957in}{0.773588in}}%
\pgfpathlineto{\pgfqpoint{-10.546833in}{0.773588in}}%
\pgfpathlineto{\pgfqpoint{-10.594934in}{0.773588in}}%
\pgfpathlineto{\pgfqpoint{-10.644481in}{0.773588in}}%
\pgfpathlineto{\pgfqpoint{-10.693023in}{0.773588in}}%
\pgfpathlineto{\pgfqpoint{-10.741507in}{0.773588in}}%
\pgfpathlineto{\pgfqpoint{-10.790422in}{0.773588in}}%
\pgfpathlineto{\pgfqpoint{-10.837686in}{0.773588in}}%
\pgfpathlineto{\pgfqpoint{-10.885230in}{0.773588in}}%
\pgfpathlineto{\pgfqpoint{-10.933861in}{0.773588in}}%
\pgfpathlineto{\pgfqpoint{-10.980903in}{0.773588in}}%
\pgfpathlineto{\pgfqpoint{-11.028607in}{0.773588in}}%
\pgfpathlineto{\pgfqpoint{-11.077704in}{0.773588in}}%
\pgfpathlineto{\pgfqpoint{-11.125274in}{0.773588in}}%
\pgfpathlineto{\pgfqpoint{-11.172812in}{0.773588in}}%
\pgfpathlineto{\pgfqpoint{-11.221741in}{0.773588in}}%
\pgfpathlineto{\pgfqpoint{-11.268838in}{0.773588in}}%
\pgfpathlineto{\pgfqpoint{-11.316622in}{0.773588in}}%
\pgfpathlineto{\pgfqpoint{-11.366120in}{0.773588in}}%
\pgfpathlineto{\pgfqpoint{-11.415235in}{0.773588in}}%
\pgfpathlineto{\pgfqpoint{-11.463260in}{0.773588in}}%
\pgfpathlineto{\pgfqpoint{-11.511546in}{0.773588in}}%
\pgfpathlineto{\pgfqpoint{-11.559336in}{0.773588in}}%
\pgfpathlineto{\pgfqpoint{-11.607131in}{0.773588in}}%
\pgfpathlineto{\pgfqpoint{-11.656418in}{0.773588in}}%
\pgfpathlineto{\pgfqpoint{-11.704231in}{0.773588in}}%
\pgfpathlineto{\pgfqpoint{-11.751492in}{0.773588in}}%
\pgfpathlineto{\pgfqpoint{-11.800403in}{0.773588in}}%
\pgfpathlineto{\pgfqpoint{-11.847736in}{0.773588in}}%
\pgfpathlineto{\pgfqpoint{-11.895021in}{0.773588in}}%
\pgfpathlineto{\pgfqpoint{-11.943783in}{0.773588in}}%
\pgfpathlineto{\pgfqpoint{-11.991431in}{0.773588in}}%
\pgfpathlineto{\pgfqpoint{-12.038569in}{0.773588in}}%
\pgfpathlineto{\pgfqpoint{-12.087205in}{0.773588in}}%
\pgfpathlineto{\pgfqpoint{-12.135286in}{0.773588in}}%
\pgfpathlineto{\pgfqpoint{-12.182471in}{0.773588in}}%
\pgfpathlineto{\pgfqpoint{-12.231638in}{0.773588in}}%
\pgfpathlineto{\pgfqpoint{-12.278675in}{0.773588in}}%
\pgfpathlineto{\pgfqpoint{-12.326627in}{0.773588in}}%
\pgfpathlineto{\pgfqpoint{-12.375560in}{0.773588in}}%
\pgfpathlineto{\pgfqpoint{-12.422612in}{0.773588in}}%
\pgfpathlineto{\pgfqpoint{-12.469843in}{0.773588in}}%
\pgfpathlineto{\pgfqpoint{-12.517880in}{0.773588in}}%
\pgfpathlineto{\pgfqpoint{-12.565107in}{0.773588in}}%
\pgfpathlineto{\pgfqpoint{-12.612433in}{0.773588in}}%
\pgfpathlineto{\pgfqpoint{-12.661578in}{0.773588in}}%
\pgfpathlineto{\pgfqpoint{-12.708579in}{0.773588in}}%
\pgfpathlineto{\pgfqpoint{-12.755866in}{0.773588in}}%
\pgfpathlineto{\pgfqpoint{-12.805010in}{0.773588in}}%
\pgfpathlineto{\pgfqpoint{-12.852526in}{0.773588in}}%
\pgfpathlineto{\pgfqpoint{-12.899585in}{0.773588in}}%
\pgfpathlineto{\pgfqpoint{-12.949523in}{0.773588in}}%
\pgfpathlineto{\pgfqpoint{-12.998480in}{0.773588in}}%
\pgfpathlineto{\pgfqpoint{-13.046608in}{0.773588in}}%
\pgfpathlineto{\pgfqpoint{-13.096328in}{0.773588in}}%
\pgfpathlineto{\pgfqpoint{-13.144068in}{0.773588in}}%
\pgfpathlineto{\pgfqpoint{-13.190970in}{0.773588in}}%
\pgfpathlineto{\pgfqpoint{-13.239807in}{0.773588in}}%
\pgfpathlineto{\pgfqpoint{-13.287448in}{0.773588in}}%
\pgfpathlineto{\pgfqpoint{-13.334326in}{0.773588in}}%
\pgfpathlineto{\pgfqpoint{-13.382220in}{0.773588in}}%
\pgfpathlineto{\pgfqpoint{-13.428346in}{0.773588in}}%
\pgfpathlineto{\pgfqpoint{-13.474335in}{0.773588in}}%
\pgfpathlineto{\pgfqpoint{-13.522553in}{0.773588in}}%
\pgfpathlineto{\pgfqpoint{-13.568873in}{0.773588in}}%
\pgfpathlineto{\pgfqpoint{-13.615991in}{0.773588in}}%
\pgfpathlineto{\pgfqpoint{-13.663331in}{0.773588in}}%
\pgfpathlineto{\pgfqpoint{-13.710126in}{0.773588in}}%
\pgfpathlineto{\pgfqpoint{-13.757485in}{0.773588in}}%
\pgfpathlineto{\pgfqpoint{-13.806330in}{0.773588in}}%
\pgfpathlineto{\pgfqpoint{-13.853816in}{0.773588in}}%
\pgfpathlineto{\pgfqpoint{-13.901287in}{0.773588in}}%
\pgfpathlineto{\pgfqpoint{-13.950094in}{0.773588in}}%
\pgfpathlineto{\pgfqpoint{-13.996250in}{0.773588in}}%
\pgfpathlineto{\pgfqpoint{-14.042263in}{0.773588in}}%
\pgfpathlineto{\pgfqpoint{-14.090279in}{0.773588in}}%
\pgfpathlineto{\pgfqpoint{-14.137283in}{0.773588in}}%
\pgfpathlineto{\pgfqpoint{-14.184689in}{0.773588in}}%
\pgfpathlineto{\pgfqpoint{-14.233487in}{0.773588in}}%
\pgfpathlineto{\pgfqpoint{-14.280667in}{0.773588in}}%
\pgfpathlineto{\pgfqpoint{-14.327371in}{0.773588in}}%
\pgfpathlineto{\pgfqpoint{-14.375931in}{0.773588in}}%
\pgfpathlineto{\pgfqpoint{-14.422978in}{0.773588in}}%
\pgfpathlineto{\pgfqpoint{-14.469907in}{0.773588in}}%
\pgfpathlineto{\pgfqpoint{-14.517904in}{0.773588in}}%
\pgfpathlineto{\pgfqpoint{-14.564671in}{0.773588in}}%
\pgfpathlineto{\pgfqpoint{-14.611394in}{0.773588in}}%
\pgfpathlineto{\pgfqpoint{-14.659388in}{0.773588in}}%
\pgfpathlineto{\pgfqpoint{-14.705919in}{0.773588in}}%
\pgfpathlineto{\pgfqpoint{-14.752153in}{0.773588in}}%
\pgfpathlineto{\pgfqpoint{-14.800669in}{0.773588in}}%
\pgfpathlineto{\pgfqpoint{-14.847516in}{0.773588in}}%
\pgfpathlineto{\pgfqpoint{-14.894419in}{0.773588in}}%
\pgfpathlineto{\pgfqpoint{-14.942416in}{0.773588in}}%
\pgfpathlineto{\pgfqpoint{-14.989019in}{0.773588in}}%
\pgfpathlineto{\pgfqpoint{-15.035921in}{0.773588in}}%
\pgfpathlineto{\pgfqpoint{-15.083728in}{0.773588in}}%
\pgfpathlineto{\pgfqpoint{-15.130553in}{0.773588in}}%
\pgfpathlineto{\pgfqpoint{-15.176769in}{0.773588in}}%
\pgfpathlineto{\pgfqpoint{-15.225998in}{0.773588in}}%
\pgfpathlineto{\pgfqpoint{-15.273998in}{0.773588in}}%
\pgfpathlineto{\pgfqpoint{-15.321194in}{0.773588in}}%
\pgfpathlineto{\pgfqpoint{-15.369802in}{0.773588in}}%
\pgfpathlineto{\pgfqpoint{-15.416792in}{0.773588in}}%
\pgfpathlineto{\pgfqpoint{-15.463539in}{0.773588in}}%
\pgfpathlineto{\pgfqpoint{-15.512535in}{0.773588in}}%
\pgfpathlineto{\pgfqpoint{-15.560849in}{0.773588in}}%
\pgfpathlineto{\pgfqpoint{-15.608704in}{0.773588in}}%
\pgfpathlineto{\pgfqpoint{-15.657002in}{0.773588in}}%
\pgfpathlineto{\pgfqpoint{-15.703994in}{0.773588in}}%
\pgfpathlineto{\pgfqpoint{-15.750800in}{0.773588in}}%
\pgfpathlineto{\pgfqpoint{-15.798897in}{0.773588in}}%
\pgfpathlineto{\pgfqpoint{-15.844500in}{0.773588in}}%
\pgfpathlineto{\pgfqpoint{-15.890419in}{0.773588in}}%
\pgfpathlineto{\pgfqpoint{-15.938076in}{0.773588in}}%
\pgfpathlineto{\pgfqpoint{-15.984286in}{0.773588in}}%
\pgfpathlineto{\pgfqpoint{-16.029628in}{0.773588in}}%
\pgfpathlineto{\pgfqpoint{-16.077171in}{0.773588in}}%
\pgfpathlineto{\pgfqpoint{-16.123165in}{0.773588in}}%
\pgfpathlineto{\pgfqpoint{-16.169427in}{0.773588in}}%
\pgfpathlineto{\pgfqpoint{-16.217070in}{0.773588in}}%
\pgfpathlineto{\pgfqpoint{-16.262806in}{0.773588in}}%
\pgfpathlineto{\pgfqpoint{-16.308871in}{0.773588in}}%
\pgfpathlineto{\pgfqpoint{-16.355646in}{0.773588in}}%
\pgfpathlineto{\pgfqpoint{-16.402550in}{0.773588in}}%
\pgfpathlineto{\pgfqpoint{-16.449347in}{0.773588in}}%
\pgfpathlineto{\pgfqpoint{-16.496157in}{0.773588in}}%
\pgfpathlineto{\pgfqpoint{-16.542990in}{0.773588in}}%
\pgfpathlineto{\pgfqpoint{-16.589239in}{0.773588in}}%
\pgfpathlineto{\pgfqpoint{-16.637120in}{0.773588in}}%
\pgfpathlineto{\pgfqpoint{-16.684310in}{0.773588in}}%
\pgfpathlineto{\pgfqpoint{-16.730721in}{0.773588in}}%
\pgfpathlineto{\pgfqpoint{-16.778530in}{0.773588in}}%
\pgfpathlineto{\pgfqpoint{-16.825434in}{0.773588in}}%
\pgfpathlineto{\pgfqpoint{-16.871664in}{0.773588in}}%
\pgfpathlineto{\pgfqpoint{-16.918692in}{0.773588in}}%
\pgfpathlineto{\pgfqpoint{-16.964774in}{0.773588in}}%
\pgfpathlineto{\pgfqpoint{-17.010884in}{0.773588in}}%
\pgfpathlineto{\pgfqpoint{-17.058009in}{0.773588in}}%
\pgfpathlineto{\pgfqpoint{-17.103874in}{0.773588in}}%
\pgfpathlineto{\pgfqpoint{-17.150161in}{0.773588in}}%
\pgfpathlineto{\pgfqpoint{-17.198221in}{0.773588in}}%
\pgfpathlineto{\pgfqpoint{-17.244258in}{0.773588in}}%
\pgfpathlineto{\pgfqpoint{-17.289523in}{0.773588in}}%
\pgfpathlineto{\pgfqpoint{-17.336276in}{0.773588in}}%
\pgfpathlineto{\pgfqpoint{-17.381732in}{0.773588in}}%
\pgfpathlineto{\pgfqpoint{-17.427002in}{0.773588in}}%
\pgfpathlineto{\pgfqpoint{-17.474644in}{0.773588in}}%
\pgfpathlineto{\pgfqpoint{-17.521284in}{0.773588in}}%
\pgfpathlineto{\pgfqpoint{-17.566739in}{0.773588in}}%
\pgfpathlineto{\pgfqpoint{-17.613251in}{0.773588in}}%
\pgfpathlineto{\pgfqpoint{-17.658538in}{0.773588in}}%
\pgfpathlineto{\pgfqpoint{-17.704093in}{0.773588in}}%
\pgfpathlineto{\pgfqpoint{-17.751556in}{0.773588in}}%
\pgfpathlineto{\pgfqpoint{-17.797346in}{0.773588in}}%
\pgfpathlineto{\pgfqpoint{-17.843480in}{0.773588in}}%
\pgfpathlineto{\pgfqpoint{-17.891271in}{0.773588in}}%
\pgfpathlineto{\pgfqpoint{-17.937614in}{0.773588in}}%
\pgfpathlineto{\pgfqpoint{-17.984102in}{0.773588in}}%
\pgfpathlineto{\pgfqpoint{-18.032389in}{0.773588in}}%
\pgfpathlineto{\pgfqpoint{-18.078784in}{0.773588in}}%
\pgfpathlineto{\pgfqpoint{-18.124795in}{0.773588in}}%
\pgfpathlineto{\pgfqpoint{-18.172791in}{0.773588in}}%
\pgfpathlineto{\pgfqpoint{-18.218674in}{0.773588in}}%
\pgfpathlineto{\pgfqpoint{-18.264167in}{0.773588in}}%
\pgfpathlineto{\pgfqpoint{-18.311512in}{0.773588in}}%
\pgfpathlineto{\pgfqpoint{-18.356748in}{0.773588in}}%
\pgfpathlineto{\pgfqpoint{-18.402000in}{0.773588in}}%
\pgfpathlineto{\pgfqpoint{-18.448764in}{0.773588in}}%
\pgfpathlineto{\pgfqpoint{-18.494344in}{0.773588in}}%
\pgfpathlineto{\pgfqpoint{-18.540355in}{0.773588in}}%
\pgfpathlineto{\pgfqpoint{-18.587378in}{0.773588in}}%
\pgfpathlineto{\pgfqpoint{-18.632978in}{0.773588in}}%
\pgfpathlineto{\pgfqpoint{-18.678684in}{0.773588in}}%
\pgfpathlineto{\pgfqpoint{-18.726033in}{0.773588in}}%
\pgfpathlineto{\pgfqpoint{-18.772260in}{0.773588in}}%
\pgfpathlineto{\pgfqpoint{-18.817991in}{0.773588in}}%
\pgfpathlineto{\pgfqpoint{-18.865040in}{0.773588in}}%
\pgfpathlineto{\pgfqpoint{-18.910503in}{0.773588in}}%
\pgfpathlineto{\pgfqpoint{-18.956783in}{0.773588in}}%
\pgfpathlineto{\pgfqpoint{-19.004576in}{0.773588in}}%
\pgfpathlineto{\pgfqpoint{-19.050812in}{0.773588in}}%
\pgfpathlineto{\pgfqpoint{-19.097062in}{0.773588in}}%
\pgfpathlineto{\pgfqpoint{-19.143544in}{0.773588in}}%
\pgfpathlineto{\pgfqpoint{-19.188899in}{0.773588in}}%
\pgfpathlineto{\pgfqpoint{-19.234797in}{0.773588in}}%
\pgfpathlineto{\pgfqpoint{-19.283047in}{0.773588in}}%
\pgfpathlineto{\pgfqpoint{-19.328532in}{0.773588in}}%
\pgfpathlineto{\pgfqpoint{-19.374230in}{0.773588in}}%
\pgfpathlineto{\pgfqpoint{-19.421584in}{0.773588in}}%
\pgfpathlineto{\pgfqpoint{-19.467303in}{0.773588in}}%
\pgfpathlineto{\pgfqpoint{-19.512973in}{0.773588in}}%
\pgfpathlineto{\pgfqpoint{-19.559546in}{0.773588in}}%
\pgfpathlineto{\pgfqpoint{-19.604824in}{0.773588in}}%
\pgfpathlineto{\pgfqpoint{-19.650361in}{0.773588in}}%
\pgfpathlineto{\pgfqpoint{-19.696649in}{0.773588in}}%
\pgfpathlineto{\pgfqpoint{-19.742024in}{0.773588in}}%
\pgfpathlineto{\pgfqpoint{-19.787808in}{0.773588in}}%
\pgfpathlineto{\pgfqpoint{-19.833948in}{0.773588in}}%
\pgfpathlineto{\pgfqpoint{-19.878562in}{0.773588in}}%
\pgfpathlineto{\pgfqpoint{-19.924139in}{0.773588in}}%
\pgfpathlineto{\pgfqpoint{-19.970967in}{0.773588in}}%
\pgfpathlineto{\pgfqpoint{-20.016506in}{0.773588in}}%
\pgfpathlineto{\pgfqpoint{-20.061722in}{0.773588in}}%
\pgfpathlineto{\pgfqpoint{-20.107707in}{0.773588in}}%
\pgfpathlineto{\pgfqpoint{-20.153038in}{0.773588in}}%
\pgfpathlineto{\pgfqpoint{-20.197814in}{0.773588in}}%
\pgfpathlineto{\pgfqpoint{-20.244875in}{0.773588in}}%
\pgfpathlineto{\pgfqpoint{-20.290419in}{0.773588in}}%
\pgfpathlineto{\pgfqpoint{-20.335981in}{0.773588in}}%
\pgfpathlineto{\pgfqpoint{-20.382199in}{0.773588in}}%
\pgfpathlineto{\pgfqpoint{-20.428221in}{0.773588in}}%
\pgfpathlineto{\pgfqpoint{-20.474014in}{0.773588in}}%
\pgfpathlineto{\pgfqpoint{-20.519904in}{0.773588in}}%
\pgfpathlineto{\pgfqpoint{-20.565077in}{0.773588in}}%
\pgfpathlineto{\pgfqpoint{-20.611243in}{0.773588in}}%
\pgfpathlineto{\pgfqpoint{-20.658742in}{0.773588in}}%
\pgfpathlineto{\pgfqpoint{-20.705066in}{0.773588in}}%
\pgfpathlineto{\pgfqpoint{-20.750831in}{0.773588in}}%
\pgfpathlineto{\pgfqpoint{-20.797022in}{0.773588in}}%
\pgfpathlineto{\pgfqpoint{-20.842169in}{0.773588in}}%
\pgfpathlineto{\pgfqpoint{-20.886847in}{0.773588in}}%
\pgfpathlineto{\pgfqpoint{-20.932365in}{0.773588in}}%
\pgfpathlineto{\pgfqpoint{-20.976378in}{0.773588in}}%
\pgfpathlineto{\pgfqpoint{-21.021509in}{0.773588in}}%
\pgfpathlineto{\pgfqpoint{-21.067357in}{0.773588in}}%
\pgfpathlineto{\pgfqpoint{-21.113139in}{0.773588in}}%
\pgfpathlineto{\pgfqpoint{-21.158215in}{0.773588in}}%
\pgfpathlineto{\pgfqpoint{-21.204354in}{0.773588in}}%
\pgfpathlineto{\pgfqpoint{-21.249304in}{0.773588in}}%
\pgfpathlineto{\pgfqpoint{-21.294264in}{0.773588in}}%
\pgfpathlineto{\pgfqpoint{-21.340867in}{0.773588in}}%
\pgfpathlineto{\pgfqpoint{-21.385572in}{0.773588in}}%
\pgfpathlineto{\pgfqpoint{-21.430267in}{0.773588in}}%
\pgfpathlineto{\pgfqpoint{-21.475985in}{0.773588in}}%
\pgfpathlineto{\pgfqpoint{-21.520883in}{0.773588in}}%
\pgfpathlineto{\pgfqpoint{-21.566198in}{0.773588in}}%
\pgfpathlineto{\pgfqpoint{-21.612919in}{0.773588in}}%
\pgfpathlineto{\pgfqpoint{-21.657342in}{0.773588in}}%
\pgfpathlineto{\pgfqpoint{-21.703167in}{0.773588in}}%
\pgfpathlineto{\pgfqpoint{-21.749554in}{0.773588in}}%
\pgfpathlineto{\pgfqpoint{-21.794255in}{0.773588in}}%
\pgfpathlineto{\pgfqpoint{-21.839254in}{0.773588in}}%
\pgfpathlineto{\pgfqpoint{-21.885462in}{0.773588in}}%
\pgfpathlineto{\pgfqpoint{-21.930597in}{0.773588in}}%
\pgfpathlineto{\pgfqpoint{-21.976168in}{0.773588in}}%
\pgfpathlineto{\pgfqpoint{-22.022397in}{0.773588in}}%
\pgfpathlineto{\pgfqpoint{-22.067661in}{0.773588in}}%
\pgfpathlineto{\pgfqpoint{-22.112231in}{0.773588in}}%
\pgfpathlineto{\pgfqpoint{-22.158025in}{0.773588in}}%
\pgfpathlineto{\pgfqpoint{-22.201868in}{0.773588in}}%
\pgfpathlineto{\pgfqpoint{-22.246011in}{0.773588in}}%
\pgfpathlineto{\pgfqpoint{-22.291940in}{0.773588in}}%
\pgfpathlineto{\pgfqpoint{-22.336074in}{0.773588in}}%
\pgfpathlineto{\pgfqpoint{-22.380996in}{0.773588in}}%
\pgfpathlineto{\pgfqpoint{-22.427383in}{0.773588in}}%
\pgfpathlineto{\pgfqpoint{-22.471952in}{0.773588in}}%
\pgfpathlineto{\pgfqpoint{-22.516601in}{0.773588in}}%
\pgfpathlineto{\pgfqpoint{-22.562133in}{0.773588in}}%
\pgfpathlineto{\pgfqpoint{-22.606850in}{0.773588in}}%
\pgfpathlineto{\pgfqpoint{-22.651258in}{0.773588in}}%
\pgfpathlineto{\pgfqpoint{-22.697642in}{0.773588in}}%
\pgfpathlineto{\pgfqpoint{-22.742185in}{0.773588in}}%
\pgfpathlineto{\pgfqpoint{-22.786844in}{0.773588in}}%
\pgfpathlineto{\pgfqpoint{-22.833029in}{0.773588in}}%
\pgfpathlineto{\pgfqpoint{-22.878494in}{0.773588in}}%
\pgfpathlineto{\pgfqpoint{-22.924822in}{0.773588in}}%
\pgfpathlineto{\pgfqpoint{-22.972746in}{0.773588in}}%
\pgfpathlineto{\pgfqpoint{-23.018637in}{0.773588in}}%
\pgfpathlineto{\pgfqpoint{-23.065076in}{0.773588in}}%
\pgfpathlineto{\pgfqpoint{-23.112561in}{0.773588in}}%
\pgfpathlineto{\pgfqpoint{-23.159034in}{0.773588in}}%
\pgfpathlineto{\pgfqpoint{-23.205343in}{0.773588in}}%
\pgfpathlineto{\pgfqpoint{-23.252910in}{0.773588in}}%
\pgfpathlineto{\pgfqpoint{-23.298917in}{0.773588in}}%
\pgfpathlineto{\pgfqpoint{-23.344516in}{0.773588in}}%
\pgfpathlineto{\pgfqpoint{-23.391460in}{0.773588in}}%
\pgfpathlineto{\pgfqpoint{-23.437154in}{0.773588in}}%
\pgfpathlineto{\pgfqpoint{-23.482780in}{0.773588in}}%
\pgfpathlineto{\pgfqpoint{-23.529965in}{0.773588in}}%
\pgfpathlineto{\pgfqpoint{-23.575877in}{0.773588in}}%
\pgfpathlineto{\pgfqpoint{-23.620333in}{0.773588in}}%
\pgfpathlineto{\pgfqpoint{-23.666576in}{0.773588in}}%
\pgfpathlineto{\pgfqpoint{-23.711883in}{0.773588in}}%
\pgfpathlineto{\pgfqpoint{-23.757289in}{0.773588in}}%
\pgfpathlineto{\pgfqpoint{-23.804343in}{0.773588in}}%
\pgfpathlineto{\pgfqpoint{-23.849943in}{0.773588in}}%
\pgfpathlineto{\pgfqpoint{-23.895017in}{0.773588in}}%
\pgfpathlineto{\pgfqpoint{-23.941619in}{0.773588in}}%
\pgfpathlineto{\pgfqpoint{-23.986811in}{0.773588in}}%
\pgfpathlineto{\pgfqpoint{-24.032531in}{0.773588in}}%
\pgfpathlineto{\pgfqpoint{-24.078902in}{0.773588in}}%
\pgfpathlineto{\pgfqpoint{-24.123999in}{0.773588in}}%
\pgfpathlineto{\pgfqpoint{-24.169854in}{0.773588in}}%
\pgfpathlineto{\pgfqpoint{-24.216953in}{0.773588in}}%
\pgfpathlineto{\pgfqpoint{-24.262109in}{0.773588in}}%
\pgfpathlineto{\pgfqpoint{-24.308071in}{0.773588in}}%
\pgfpathlineto{\pgfqpoint{-24.355467in}{0.773588in}}%
\pgfpathlineto{\pgfqpoint{-24.401173in}{0.773588in}}%
\pgfpathlineto{\pgfqpoint{-24.447114in}{0.773588in}}%
\pgfpathlineto{\pgfqpoint{-24.494847in}{0.773588in}}%
\pgfpathlineto{\pgfqpoint{-24.540716in}{0.773588in}}%
\pgfpathlineto{\pgfqpoint{-24.586259in}{0.773588in}}%
\pgfpathlineto{\pgfqpoint{-24.632696in}{0.773588in}}%
\pgfpathlineto{\pgfqpoint{-24.677917in}{0.773588in}}%
\pgfpathlineto{\pgfqpoint{-24.722523in}{0.773588in}}%
\pgfpathlineto{\pgfqpoint{-24.768227in}{0.773588in}}%
\pgfpathlineto{\pgfqpoint{-24.813437in}{0.773588in}}%
\pgfpathlineto{\pgfqpoint{-24.858721in}{0.773588in}}%
\pgfpathlineto{\pgfqpoint{-24.905462in}{0.773588in}}%
\pgfpathlineto{\pgfqpoint{-24.950861in}{0.773588in}}%
\pgfpathlineto{\pgfqpoint{-24.995958in}{0.773588in}}%
\pgfpathlineto{\pgfqpoint{-25.043027in}{0.773588in}}%
\pgfpathlineto{\pgfqpoint{-25.088791in}{0.773588in}}%
\pgfpathlineto{\pgfqpoint{-25.133757in}{0.773588in}}%
\pgfpathlineto{\pgfqpoint{-25.181603in}{0.773588in}}%
\pgfpathlineto{\pgfqpoint{-25.227067in}{0.773588in}}%
\pgfpathlineto{\pgfqpoint{-25.271783in}{0.773588in}}%
\pgfpathlineto{\pgfqpoint{-25.318638in}{0.773588in}}%
\pgfpathlineto{\pgfqpoint{-25.364031in}{0.773588in}}%
\pgfpathlineto{\pgfqpoint{-25.409278in}{0.773588in}}%
\pgfpathlineto{\pgfqpoint{-25.455838in}{0.773588in}}%
\pgfpathlineto{\pgfqpoint{-25.501531in}{0.773588in}}%
\pgfpathlineto{\pgfqpoint{-25.546370in}{0.773588in}}%
\pgfpathlineto{\pgfqpoint{-25.593105in}{0.773588in}}%
\pgfpathlineto{\pgfqpoint{-25.638897in}{0.773588in}}%
\pgfpathlineto{\pgfqpoint{-25.684196in}{0.773588in}}%
\pgfpathlineto{\pgfqpoint{-25.731236in}{0.773588in}}%
\pgfpathlineto{\pgfqpoint{-25.776589in}{0.773588in}}%
\pgfpathlineto{\pgfqpoint{-25.821656in}{0.773588in}}%
\pgfpathlineto{\pgfqpoint{-25.868331in}{0.773588in}}%
\pgfpathlineto{\pgfqpoint{-25.913010in}{0.773588in}}%
\pgfpathlineto{\pgfqpoint{-25.957619in}{0.773588in}}%
\pgfpathlineto{\pgfqpoint{-26.004871in}{0.773588in}}%
\pgfpathlineto{\pgfqpoint{-26.049962in}{0.773588in}}%
\pgfpathlineto{\pgfqpoint{-26.095397in}{0.773588in}}%
\pgfpathlineto{\pgfqpoint{-26.140687in}{0.773588in}}%
\pgfpathlineto{\pgfqpoint{-26.185627in}{0.773588in}}%
\pgfpathlineto{\pgfqpoint{-26.230565in}{0.773588in}}%
\pgfpathlineto{\pgfqpoint{-26.277965in}{0.773588in}}%
\pgfpathlineto{\pgfqpoint{-26.323384in}{0.773588in}}%
\pgfpathlineto{\pgfqpoint{-26.368659in}{0.773588in}}%
\pgfpathlineto{\pgfqpoint{-26.414228in}{0.773588in}}%
\pgfpathlineto{\pgfqpoint{-26.458732in}{0.773588in}}%
\pgfpathlineto{\pgfqpoint{-26.503012in}{0.773588in}}%
\pgfpathlineto{\pgfqpoint{-26.549765in}{0.773588in}}%
\pgfpathlineto{\pgfqpoint{-26.595409in}{0.773588in}}%
\pgfpathlineto{\pgfqpoint{-26.640477in}{0.773588in}}%
\pgfpathlineto{\pgfqpoint{-26.688111in}{0.773588in}}%
\pgfpathlineto{\pgfqpoint{-26.735142in}{0.773588in}}%
\pgfpathlineto{\pgfqpoint{-26.781839in}{0.773588in}}%
\pgfpathlineto{\pgfqpoint{-26.830647in}{0.773588in}}%
\pgfpathlineto{\pgfqpoint{-26.880611in}{0.773588in}}%
\pgfpathlineto{\pgfqpoint{-26.933555in}{0.773588in}}%
\pgfpathlineto{\pgfqpoint{-26.988998in}{0.773588in}}%
\pgfpathlineto{\pgfqpoint{-27.037714in}{0.773588in}}%
\pgfpathlineto{\pgfqpoint{-27.086555in}{0.773588in}}%
\pgfpathlineto{\pgfqpoint{-27.136341in}{0.773588in}}%
\pgfpathlineto{\pgfqpoint{-27.184903in}{0.773588in}}%
\pgfpathlineto{\pgfqpoint{-27.232862in}{0.773588in}}%
\pgfpathlineto{\pgfqpoint{-27.281821in}{0.773588in}}%
\pgfpathlineto{\pgfqpoint{-27.328705in}{0.773588in}}%
\pgfpathlineto{\pgfqpoint{-27.374936in}{0.773588in}}%
\pgfpathlineto{\pgfqpoint{-27.422670in}{0.773588in}}%
\pgfpathlineto{\pgfqpoint{-27.470237in}{0.773588in}}%
\pgfpathlineto{\pgfqpoint{-27.516652in}{0.773588in}}%
\pgfpathlineto{\pgfqpoint{-27.563341in}{0.773588in}}%
\pgfpathlineto{\pgfqpoint{-27.608047in}{0.773588in}}%
\pgfpathlineto{\pgfqpoint{-27.651646in}{0.773588in}}%
\pgfpathlineto{\pgfqpoint{-27.696819in}{0.773588in}}%
\pgfpathlineto{\pgfqpoint{-27.740871in}{0.773588in}}%
\pgfpathlineto{\pgfqpoint{-27.785329in}{0.773588in}}%
\pgfpathlineto{\pgfqpoint{-27.830427in}{0.773588in}}%
\pgfpathlineto{\pgfqpoint{-27.874405in}{0.773588in}}%
\pgfpathlineto{\pgfqpoint{-27.917812in}{0.773588in}}%
\pgfpathlineto{\pgfqpoint{-27.963188in}{0.773588in}}%
\pgfpathlineto{\pgfqpoint{-28.007858in}{0.773588in}}%
\pgfpathlineto{\pgfqpoint{-28.051883in}{0.773588in}}%
\pgfpathlineto{\pgfqpoint{-28.097071in}{0.773588in}}%
\pgfpathlineto{\pgfqpoint{-28.141536in}{0.773588in}}%
\pgfpathlineto{\pgfqpoint{-28.185953in}{0.773588in}}%
\pgfpathlineto{\pgfqpoint{-28.230974in}{0.773588in}}%
\pgfpathlineto{\pgfqpoint{-28.274355in}{0.773588in}}%
\pgfpathlineto{\pgfqpoint{-28.318373in}{0.773588in}}%
\pgfpathlineto{\pgfqpoint{-28.364238in}{0.773588in}}%
\pgfpathlineto{\pgfqpoint{-28.408130in}{0.773588in}}%
\pgfpathlineto{\pgfqpoint{-28.452363in}{0.773588in}}%
\pgfpathlineto{\pgfqpoint{-28.498901in}{0.773588in}}%
\pgfpathlineto{\pgfqpoint{-28.543533in}{0.773588in}}%
\pgfpathlineto{\pgfqpoint{-28.588250in}{0.773588in}}%
\pgfpathclose%
\pgfusepath{fill}%
\end{pgfscope}%
\begin{pgfscope}%
\pgfpathrectangle{\pgfqpoint{2.662073in}{0.773588in}}{\pgfqpoint{2.964025in}{5.415119in}}%
\pgfusepath{clip}%
\pgfsetbuttcap%
\pgfsetroundjoin%
\definecolor{currentfill}{rgb}{0.172549,0.627451,0.172549}%
\pgfsetfillcolor{currentfill}%
\pgfsetlinewidth{0.000000pt}%
\definecolor{currentstroke}{rgb}{0.000000,0.000000,0.000000}%
\pgfsetstrokecolor{currentstroke}%
\pgfsetdash{}{0pt}%
\pgfpathmoveto{\pgfqpoint{-28.588250in}{0.773588in}}%
\pgfpathlineto{\pgfqpoint{-28.588250in}{0.773588in}}%
\pgfpathlineto{\pgfqpoint{-28.543533in}{0.773588in}}%
\pgfpathlineto{\pgfqpoint{-28.498901in}{0.773588in}}%
\pgfpathlineto{\pgfqpoint{-28.452363in}{0.773588in}}%
\pgfpathlineto{\pgfqpoint{-28.408130in}{0.773588in}}%
\pgfpathlineto{\pgfqpoint{-28.364238in}{0.773588in}}%
\pgfpathlineto{\pgfqpoint{-28.318373in}{0.773588in}}%
\pgfpathlineto{\pgfqpoint{-28.274355in}{0.773588in}}%
\pgfpathlineto{\pgfqpoint{-28.230974in}{0.773588in}}%
\pgfpathlineto{\pgfqpoint{-28.185953in}{0.773588in}}%
\pgfpathlineto{\pgfqpoint{-28.141536in}{0.773588in}}%
\pgfpathlineto{\pgfqpoint{-28.097071in}{0.773588in}}%
\pgfpathlineto{\pgfqpoint{-28.051883in}{0.773588in}}%
\pgfpathlineto{\pgfqpoint{-28.007858in}{0.773588in}}%
\pgfpathlineto{\pgfqpoint{-27.963188in}{0.773588in}}%
\pgfpathlineto{\pgfqpoint{-27.917812in}{0.773588in}}%
\pgfpathlineto{\pgfqpoint{-27.874405in}{0.773588in}}%
\pgfpathlineto{\pgfqpoint{-27.830427in}{0.773588in}}%
\pgfpathlineto{\pgfqpoint{-27.785329in}{0.773588in}}%
\pgfpathlineto{\pgfqpoint{-27.740871in}{0.773588in}}%
\pgfpathlineto{\pgfqpoint{-27.696819in}{0.773588in}}%
\pgfpathlineto{\pgfqpoint{-27.651646in}{0.773588in}}%
\pgfpathlineto{\pgfqpoint{-27.608047in}{0.773588in}}%
\pgfpathlineto{\pgfqpoint{-27.563341in}{0.773588in}}%
\pgfpathlineto{\pgfqpoint{-27.516652in}{0.773588in}}%
\pgfpathlineto{\pgfqpoint{-27.470237in}{0.773588in}}%
\pgfpathlineto{\pgfqpoint{-27.422670in}{0.773588in}}%
\pgfpathlineto{\pgfqpoint{-27.374936in}{0.773588in}}%
\pgfpathlineto{\pgfqpoint{-27.328705in}{0.773588in}}%
\pgfpathlineto{\pgfqpoint{-27.281821in}{0.773588in}}%
\pgfpathlineto{\pgfqpoint{-27.232862in}{0.773588in}}%
\pgfpathlineto{\pgfqpoint{-27.184903in}{0.773588in}}%
\pgfpathlineto{\pgfqpoint{-27.136341in}{0.773588in}}%
\pgfpathlineto{\pgfqpoint{-27.086555in}{0.773588in}}%
\pgfpathlineto{\pgfqpoint{-27.037714in}{0.773588in}}%
\pgfpathlineto{\pgfqpoint{-26.988998in}{0.773588in}}%
\pgfpathlineto{\pgfqpoint{-26.933555in}{0.773588in}}%
\pgfpathlineto{\pgfqpoint{-26.880611in}{0.773588in}}%
\pgfpathlineto{\pgfqpoint{-26.830647in}{0.773588in}}%
\pgfpathlineto{\pgfqpoint{-26.781839in}{0.773588in}}%
\pgfpathlineto{\pgfqpoint{-26.735142in}{0.773588in}}%
\pgfpathlineto{\pgfqpoint{-26.688111in}{0.773588in}}%
\pgfpathlineto{\pgfqpoint{-26.640477in}{0.773588in}}%
\pgfpathlineto{\pgfqpoint{-26.595409in}{0.773588in}}%
\pgfpathlineto{\pgfqpoint{-26.549765in}{0.773588in}}%
\pgfpathlineto{\pgfqpoint{-26.503012in}{0.773588in}}%
\pgfpathlineto{\pgfqpoint{-26.458732in}{0.773588in}}%
\pgfpathlineto{\pgfqpoint{-26.414228in}{0.773588in}}%
\pgfpathlineto{\pgfqpoint{-26.368659in}{0.773588in}}%
\pgfpathlineto{\pgfqpoint{-26.323384in}{0.773588in}}%
\pgfpathlineto{\pgfqpoint{-26.277965in}{0.773588in}}%
\pgfpathlineto{\pgfqpoint{-26.230565in}{0.773588in}}%
\pgfpathlineto{\pgfqpoint{-26.185627in}{0.773588in}}%
\pgfpathlineto{\pgfqpoint{-26.140687in}{0.773588in}}%
\pgfpathlineto{\pgfqpoint{-26.095397in}{0.773588in}}%
\pgfpathlineto{\pgfqpoint{-26.049962in}{0.773588in}}%
\pgfpathlineto{\pgfqpoint{-26.004871in}{0.773588in}}%
\pgfpathlineto{\pgfqpoint{-25.957619in}{0.773588in}}%
\pgfpathlineto{\pgfqpoint{-25.913010in}{0.773588in}}%
\pgfpathlineto{\pgfqpoint{-25.868331in}{0.773588in}}%
\pgfpathlineto{\pgfqpoint{-25.821656in}{0.773588in}}%
\pgfpathlineto{\pgfqpoint{-25.776589in}{0.773588in}}%
\pgfpathlineto{\pgfqpoint{-25.731236in}{0.773588in}}%
\pgfpathlineto{\pgfqpoint{-25.684196in}{0.773588in}}%
\pgfpathlineto{\pgfqpoint{-25.638897in}{0.773588in}}%
\pgfpathlineto{\pgfqpoint{-25.593105in}{0.773588in}}%
\pgfpathlineto{\pgfqpoint{-25.546370in}{0.773588in}}%
\pgfpathlineto{\pgfqpoint{-25.501531in}{0.773588in}}%
\pgfpathlineto{\pgfqpoint{-25.455838in}{0.773588in}}%
\pgfpathlineto{\pgfqpoint{-25.409278in}{0.773588in}}%
\pgfpathlineto{\pgfqpoint{-25.364031in}{0.773588in}}%
\pgfpathlineto{\pgfqpoint{-25.318638in}{0.773588in}}%
\pgfpathlineto{\pgfqpoint{-25.271783in}{0.773588in}}%
\pgfpathlineto{\pgfqpoint{-25.227067in}{0.773588in}}%
\pgfpathlineto{\pgfqpoint{-25.181603in}{0.773588in}}%
\pgfpathlineto{\pgfqpoint{-25.133757in}{0.773588in}}%
\pgfpathlineto{\pgfqpoint{-25.088791in}{0.773588in}}%
\pgfpathlineto{\pgfqpoint{-25.043027in}{0.773588in}}%
\pgfpathlineto{\pgfqpoint{-24.995958in}{0.773588in}}%
\pgfpathlineto{\pgfqpoint{-24.950861in}{0.773588in}}%
\pgfpathlineto{\pgfqpoint{-24.905462in}{0.773588in}}%
\pgfpathlineto{\pgfqpoint{-24.858721in}{0.773588in}}%
\pgfpathlineto{\pgfqpoint{-24.813437in}{0.773588in}}%
\pgfpathlineto{\pgfqpoint{-24.768227in}{0.773588in}}%
\pgfpathlineto{\pgfqpoint{-24.722523in}{0.773588in}}%
\pgfpathlineto{\pgfqpoint{-24.677917in}{0.773588in}}%
\pgfpathlineto{\pgfqpoint{-24.632696in}{0.773588in}}%
\pgfpathlineto{\pgfqpoint{-24.586259in}{0.773588in}}%
\pgfpathlineto{\pgfqpoint{-24.540716in}{0.773588in}}%
\pgfpathlineto{\pgfqpoint{-24.494847in}{0.773588in}}%
\pgfpathlineto{\pgfqpoint{-24.447114in}{0.773588in}}%
\pgfpathlineto{\pgfqpoint{-24.401173in}{0.773588in}}%
\pgfpathlineto{\pgfqpoint{-24.355467in}{0.773588in}}%
\pgfpathlineto{\pgfqpoint{-24.308071in}{0.773588in}}%
\pgfpathlineto{\pgfqpoint{-24.262109in}{0.773588in}}%
\pgfpathlineto{\pgfqpoint{-24.216953in}{0.773588in}}%
\pgfpathlineto{\pgfqpoint{-24.169854in}{0.773588in}}%
\pgfpathlineto{\pgfqpoint{-24.123999in}{0.773588in}}%
\pgfpathlineto{\pgfqpoint{-24.078902in}{0.773588in}}%
\pgfpathlineto{\pgfqpoint{-24.032531in}{0.773588in}}%
\pgfpathlineto{\pgfqpoint{-23.986811in}{0.773588in}}%
\pgfpathlineto{\pgfqpoint{-23.941619in}{0.773588in}}%
\pgfpathlineto{\pgfqpoint{-23.895017in}{0.773588in}}%
\pgfpathlineto{\pgfqpoint{-23.849943in}{0.773588in}}%
\pgfpathlineto{\pgfqpoint{-23.804343in}{0.773588in}}%
\pgfpathlineto{\pgfqpoint{-23.757289in}{0.773588in}}%
\pgfpathlineto{\pgfqpoint{-23.711883in}{0.773588in}}%
\pgfpathlineto{\pgfqpoint{-23.666576in}{0.773588in}}%
\pgfpathlineto{\pgfqpoint{-23.620333in}{0.773588in}}%
\pgfpathlineto{\pgfqpoint{-23.575877in}{0.773588in}}%
\pgfpathlineto{\pgfqpoint{-23.529965in}{0.773588in}}%
\pgfpathlineto{\pgfqpoint{-23.482780in}{0.773588in}}%
\pgfpathlineto{\pgfqpoint{-23.437154in}{0.773588in}}%
\pgfpathlineto{\pgfqpoint{-23.391460in}{0.773588in}}%
\pgfpathlineto{\pgfqpoint{-23.344516in}{0.773588in}}%
\pgfpathlineto{\pgfqpoint{-23.298917in}{0.773588in}}%
\pgfpathlineto{\pgfqpoint{-23.252910in}{0.773588in}}%
\pgfpathlineto{\pgfqpoint{-23.205343in}{0.773588in}}%
\pgfpathlineto{\pgfqpoint{-23.159034in}{0.773588in}}%
\pgfpathlineto{\pgfqpoint{-23.112561in}{0.773588in}}%
\pgfpathlineto{\pgfqpoint{-23.065076in}{0.773588in}}%
\pgfpathlineto{\pgfqpoint{-23.018637in}{0.773588in}}%
\pgfpathlineto{\pgfqpoint{-22.972746in}{0.773588in}}%
\pgfpathlineto{\pgfqpoint{-22.924822in}{0.773588in}}%
\pgfpathlineto{\pgfqpoint{-22.878494in}{0.773588in}}%
\pgfpathlineto{\pgfqpoint{-22.833029in}{0.773588in}}%
\pgfpathlineto{\pgfqpoint{-22.786844in}{0.773588in}}%
\pgfpathlineto{\pgfqpoint{-22.742185in}{0.773588in}}%
\pgfpathlineto{\pgfqpoint{-22.697642in}{0.773588in}}%
\pgfpathlineto{\pgfqpoint{-22.651258in}{0.773588in}}%
\pgfpathlineto{\pgfqpoint{-22.606850in}{0.773588in}}%
\pgfpathlineto{\pgfqpoint{-22.562133in}{0.773588in}}%
\pgfpathlineto{\pgfqpoint{-22.516601in}{0.773588in}}%
\pgfpathlineto{\pgfqpoint{-22.471952in}{0.773588in}}%
\pgfpathlineto{\pgfqpoint{-22.427383in}{0.773588in}}%
\pgfpathlineto{\pgfqpoint{-22.380996in}{0.773588in}}%
\pgfpathlineto{\pgfqpoint{-22.336074in}{0.773588in}}%
\pgfpathlineto{\pgfqpoint{-22.291940in}{0.773588in}}%
\pgfpathlineto{\pgfqpoint{-22.246011in}{0.773588in}}%
\pgfpathlineto{\pgfqpoint{-22.201868in}{0.773588in}}%
\pgfpathlineto{\pgfqpoint{-22.158025in}{0.773588in}}%
\pgfpathlineto{\pgfqpoint{-22.112231in}{0.773588in}}%
\pgfpathlineto{\pgfqpoint{-22.067661in}{0.773588in}}%
\pgfpathlineto{\pgfqpoint{-22.022397in}{0.773588in}}%
\pgfpathlineto{\pgfqpoint{-21.976168in}{0.773588in}}%
\pgfpathlineto{\pgfqpoint{-21.930597in}{0.773588in}}%
\pgfpathlineto{\pgfqpoint{-21.885462in}{0.773588in}}%
\pgfpathlineto{\pgfqpoint{-21.839254in}{0.773588in}}%
\pgfpathlineto{\pgfqpoint{-21.794255in}{0.773588in}}%
\pgfpathlineto{\pgfqpoint{-21.749554in}{0.773588in}}%
\pgfpathlineto{\pgfqpoint{-21.703167in}{0.773588in}}%
\pgfpathlineto{\pgfqpoint{-21.657342in}{0.773588in}}%
\pgfpathlineto{\pgfqpoint{-21.612919in}{0.773588in}}%
\pgfpathlineto{\pgfqpoint{-21.566198in}{0.773588in}}%
\pgfpathlineto{\pgfqpoint{-21.520883in}{0.773588in}}%
\pgfpathlineto{\pgfqpoint{-21.475985in}{0.773588in}}%
\pgfpathlineto{\pgfqpoint{-21.430267in}{0.773588in}}%
\pgfpathlineto{\pgfqpoint{-21.385572in}{0.773588in}}%
\pgfpathlineto{\pgfqpoint{-21.340867in}{0.773588in}}%
\pgfpathlineto{\pgfqpoint{-21.294264in}{0.773588in}}%
\pgfpathlineto{\pgfqpoint{-21.249304in}{0.773588in}}%
\pgfpathlineto{\pgfqpoint{-21.204354in}{0.773588in}}%
\pgfpathlineto{\pgfqpoint{-21.158215in}{0.773588in}}%
\pgfpathlineto{\pgfqpoint{-21.113139in}{0.773588in}}%
\pgfpathlineto{\pgfqpoint{-21.067357in}{0.773588in}}%
\pgfpathlineto{\pgfqpoint{-21.021509in}{0.773588in}}%
\pgfpathlineto{\pgfqpoint{-20.976378in}{0.773588in}}%
\pgfpathlineto{\pgfqpoint{-20.932365in}{0.773588in}}%
\pgfpathlineto{\pgfqpoint{-20.886847in}{0.773588in}}%
\pgfpathlineto{\pgfqpoint{-20.842169in}{0.773588in}}%
\pgfpathlineto{\pgfqpoint{-20.797022in}{0.773588in}}%
\pgfpathlineto{\pgfqpoint{-20.750831in}{0.773588in}}%
\pgfpathlineto{\pgfqpoint{-20.705066in}{0.773588in}}%
\pgfpathlineto{\pgfqpoint{-20.658742in}{0.773588in}}%
\pgfpathlineto{\pgfqpoint{-20.611243in}{0.773588in}}%
\pgfpathlineto{\pgfqpoint{-20.565077in}{0.773588in}}%
\pgfpathlineto{\pgfqpoint{-20.519904in}{0.773588in}}%
\pgfpathlineto{\pgfqpoint{-20.474014in}{0.773588in}}%
\pgfpathlineto{\pgfqpoint{-20.428221in}{0.773588in}}%
\pgfpathlineto{\pgfqpoint{-20.382199in}{0.773588in}}%
\pgfpathlineto{\pgfqpoint{-20.335981in}{0.773588in}}%
\pgfpathlineto{\pgfqpoint{-20.290419in}{0.773588in}}%
\pgfpathlineto{\pgfqpoint{-20.244875in}{0.773588in}}%
\pgfpathlineto{\pgfqpoint{-20.197814in}{0.773588in}}%
\pgfpathlineto{\pgfqpoint{-20.153038in}{0.773588in}}%
\pgfpathlineto{\pgfqpoint{-20.107707in}{0.773588in}}%
\pgfpathlineto{\pgfqpoint{-20.061722in}{0.773588in}}%
\pgfpathlineto{\pgfqpoint{-20.016506in}{0.773588in}}%
\pgfpathlineto{\pgfqpoint{-19.970967in}{0.773588in}}%
\pgfpathlineto{\pgfqpoint{-19.924139in}{0.773588in}}%
\pgfpathlineto{\pgfqpoint{-19.878562in}{0.773588in}}%
\pgfpathlineto{\pgfqpoint{-19.833948in}{0.773588in}}%
\pgfpathlineto{\pgfqpoint{-19.787808in}{0.773588in}}%
\pgfpathlineto{\pgfqpoint{-19.742024in}{0.773588in}}%
\pgfpathlineto{\pgfqpoint{-19.696649in}{0.773588in}}%
\pgfpathlineto{\pgfqpoint{-19.650361in}{0.773588in}}%
\pgfpathlineto{\pgfqpoint{-19.604824in}{0.773588in}}%
\pgfpathlineto{\pgfqpoint{-19.559546in}{0.773588in}}%
\pgfpathlineto{\pgfqpoint{-19.512973in}{0.773588in}}%
\pgfpathlineto{\pgfqpoint{-19.467303in}{0.773588in}}%
\pgfpathlineto{\pgfqpoint{-19.421584in}{0.773588in}}%
\pgfpathlineto{\pgfqpoint{-19.374230in}{0.773588in}}%
\pgfpathlineto{\pgfqpoint{-19.328532in}{0.773588in}}%
\pgfpathlineto{\pgfqpoint{-19.283047in}{0.773588in}}%
\pgfpathlineto{\pgfqpoint{-19.234797in}{0.773588in}}%
\pgfpathlineto{\pgfqpoint{-19.188899in}{0.773588in}}%
\pgfpathlineto{\pgfqpoint{-19.143544in}{0.773588in}}%
\pgfpathlineto{\pgfqpoint{-19.097062in}{0.773588in}}%
\pgfpathlineto{\pgfqpoint{-19.050812in}{0.773588in}}%
\pgfpathlineto{\pgfqpoint{-19.004576in}{0.773588in}}%
\pgfpathlineto{\pgfqpoint{-18.956783in}{0.773588in}}%
\pgfpathlineto{\pgfqpoint{-18.910503in}{0.773588in}}%
\pgfpathlineto{\pgfqpoint{-18.865040in}{0.773588in}}%
\pgfpathlineto{\pgfqpoint{-18.817991in}{0.773588in}}%
\pgfpathlineto{\pgfqpoint{-18.772260in}{0.773588in}}%
\pgfpathlineto{\pgfqpoint{-18.726033in}{0.773588in}}%
\pgfpathlineto{\pgfqpoint{-18.678684in}{0.773588in}}%
\pgfpathlineto{\pgfqpoint{-18.632978in}{0.773588in}}%
\pgfpathlineto{\pgfqpoint{-18.587378in}{0.773588in}}%
\pgfpathlineto{\pgfqpoint{-18.540355in}{0.773588in}}%
\pgfpathlineto{\pgfqpoint{-18.494344in}{0.773588in}}%
\pgfpathlineto{\pgfqpoint{-18.448764in}{0.773588in}}%
\pgfpathlineto{\pgfqpoint{-18.402000in}{0.773588in}}%
\pgfpathlineto{\pgfqpoint{-18.356748in}{0.773588in}}%
\pgfpathlineto{\pgfqpoint{-18.311512in}{0.773588in}}%
\pgfpathlineto{\pgfqpoint{-18.264167in}{0.773588in}}%
\pgfpathlineto{\pgfqpoint{-18.218674in}{0.773588in}}%
\pgfpathlineto{\pgfqpoint{-18.172791in}{0.773588in}}%
\pgfpathlineto{\pgfqpoint{-18.124795in}{0.773588in}}%
\pgfpathlineto{\pgfqpoint{-18.078784in}{0.773588in}}%
\pgfpathlineto{\pgfqpoint{-18.032389in}{0.773588in}}%
\pgfpathlineto{\pgfqpoint{-17.984102in}{0.773588in}}%
\pgfpathlineto{\pgfqpoint{-17.937614in}{0.773588in}}%
\pgfpathlineto{\pgfqpoint{-17.891271in}{0.773588in}}%
\pgfpathlineto{\pgfqpoint{-17.843480in}{0.773588in}}%
\pgfpathlineto{\pgfqpoint{-17.797346in}{0.773588in}}%
\pgfpathlineto{\pgfqpoint{-17.751556in}{0.773588in}}%
\pgfpathlineto{\pgfqpoint{-17.704093in}{0.773588in}}%
\pgfpathlineto{\pgfqpoint{-17.658538in}{0.773588in}}%
\pgfpathlineto{\pgfqpoint{-17.613251in}{0.773588in}}%
\pgfpathlineto{\pgfqpoint{-17.566739in}{0.773588in}}%
\pgfpathlineto{\pgfqpoint{-17.521284in}{0.773588in}}%
\pgfpathlineto{\pgfqpoint{-17.474644in}{0.773588in}}%
\pgfpathlineto{\pgfqpoint{-17.427002in}{0.773588in}}%
\pgfpathlineto{\pgfqpoint{-17.381732in}{0.773588in}}%
\pgfpathlineto{\pgfqpoint{-17.336276in}{0.773588in}}%
\pgfpathlineto{\pgfqpoint{-17.289523in}{0.773588in}}%
\pgfpathlineto{\pgfqpoint{-17.244258in}{0.773588in}}%
\pgfpathlineto{\pgfqpoint{-17.198221in}{0.773588in}}%
\pgfpathlineto{\pgfqpoint{-17.150161in}{0.773588in}}%
\pgfpathlineto{\pgfqpoint{-17.103874in}{0.773588in}}%
\pgfpathlineto{\pgfqpoint{-17.058009in}{0.773588in}}%
\pgfpathlineto{\pgfqpoint{-17.010884in}{0.773588in}}%
\pgfpathlineto{\pgfqpoint{-16.964774in}{0.773588in}}%
\pgfpathlineto{\pgfqpoint{-16.918692in}{0.773588in}}%
\pgfpathlineto{\pgfqpoint{-16.871664in}{0.773588in}}%
\pgfpathlineto{\pgfqpoint{-16.825434in}{0.773588in}}%
\pgfpathlineto{\pgfqpoint{-16.778530in}{0.773588in}}%
\pgfpathlineto{\pgfqpoint{-16.730721in}{0.773588in}}%
\pgfpathlineto{\pgfqpoint{-16.684310in}{0.773588in}}%
\pgfpathlineto{\pgfqpoint{-16.637120in}{0.773588in}}%
\pgfpathlineto{\pgfqpoint{-16.589239in}{0.773588in}}%
\pgfpathlineto{\pgfqpoint{-16.542990in}{0.773588in}}%
\pgfpathlineto{\pgfqpoint{-16.496157in}{0.773588in}}%
\pgfpathlineto{\pgfqpoint{-16.449347in}{0.773588in}}%
\pgfpathlineto{\pgfqpoint{-16.402550in}{0.773588in}}%
\pgfpathlineto{\pgfqpoint{-16.355646in}{0.773588in}}%
\pgfpathlineto{\pgfqpoint{-16.308871in}{0.773588in}}%
\pgfpathlineto{\pgfqpoint{-16.262806in}{0.773588in}}%
\pgfpathlineto{\pgfqpoint{-16.217070in}{0.773588in}}%
\pgfpathlineto{\pgfqpoint{-16.169427in}{0.773588in}}%
\pgfpathlineto{\pgfqpoint{-16.123165in}{0.773588in}}%
\pgfpathlineto{\pgfqpoint{-16.077171in}{0.773588in}}%
\pgfpathlineto{\pgfqpoint{-16.029628in}{0.773588in}}%
\pgfpathlineto{\pgfqpoint{-15.984286in}{0.773588in}}%
\pgfpathlineto{\pgfqpoint{-15.938076in}{0.773588in}}%
\pgfpathlineto{\pgfqpoint{-15.890419in}{0.773588in}}%
\pgfpathlineto{\pgfqpoint{-15.844500in}{0.773588in}}%
\pgfpathlineto{\pgfqpoint{-15.798897in}{0.773588in}}%
\pgfpathlineto{\pgfqpoint{-15.750800in}{0.773588in}}%
\pgfpathlineto{\pgfqpoint{-15.703994in}{0.773588in}}%
\pgfpathlineto{\pgfqpoint{-15.657002in}{0.773588in}}%
\pgfpathlineto{\pgfqpoint{-15.608704in}{0.773588in}}%
\pgfpathlineto{\pgfqpoint{-15.560849in}{0.773588in}}%
\pgfpathlineto{\pgfqpoint{-15.512535in}{0.773588in}}%
\pgfpathlineto{\pgfqpoint{-15.463539in}{0.773588in}}%
\pgfpathlineto{\pgfqpoint{-15.416792in}{0.773588in}}%
\pgfpathlineto{\pgfqpoint{-15.369802in}{0.773588in}}%
\pgfpathlineto{\pgfqpoint{-15.321194in}{0.773588in}}%
\pgfpathlineto{\pgfqpoint{-15.273998in}{0.773588in}}%
\pgfpathlineto{\pgfqpoint{-15.225998in}{0.773588in}}%
\pgfpathlineto{\pgfqpoint{-15.176769in}{0.773588in}}%
\pgfpathlineto{\pgfqpoint{-15.130553in}{0.773588in}}%
\pgfpathlineto{\pgfqpoint{-15.083728in}{0.773588in}}%
\pgfpathlineto{\pgfqpoint{-15.035921in}{0.773588in}}%
\pgfpathlineto{\pgfqpoint{-14.989019in}{0.773588in}}%
\pgfpathlineto{\pgfqpoint{-14.942416in}{0.773588in}}%
\pgfpathlineto{\pgfqpoint{-14.894419in}{0.773588in}}%
\pgfpathlineto{\pgfqpoint{-14.847516in}{0.773588in}}%
\pgfpathlineto{\pgfqpoint{-14.800669in}{0.773588in}}%
\pgfpathlineto{\pgfqpoint{-14.752153in}{0.773588in}}%
\pgfpathlineto{\pgfqpoint{-14.705919in}{0.773588in}}%
\pgfpathlineto{\pgfqpoint{-14.659388in}{0.773588in}}%
\pgfpathlineto{\pgfqpoint{-14.611394in}{0.773588in}}%
\pgfpathlineto{\pgfqpoint{-14.564671in}{0.773588in}}%
\pgfpathlineto{\pgfqpoint{-14.517904in}{0.773588in}}%
\pgfpathlineto{\pgfqpoint{-14.469907in}{0.773588in}}%
\pgfpathlineto{\pgfqpoint{-14.422978in}{0.773588in}}%
\pgfpathlineto{\pgfqpoint{-14.375931in}{0.773588in}}%
\pgfpathlineto{\pgfqpoint{-14.327371in}{0.773588in}}%
\pgfpathlineto{\pgfqpoint{-14.280667in}{0.773588in}}%
\pgfpathlineto{\pgfqpoint{-14.233487in}{0.773588in}}%
\pgfpathlineto{\pgfqpoint{-14.184689in}{0.773588in}}%
\pgfpathlineto{\pgfqpoint{-14.137283in}{0.773588in}}%
\pgfpathlineto{\pgfqpoint{-14.090279in}{0.773588in}}%
\pgfpathlineto{\pgfqpoint{-14.042263in}{0.773588in}}%
\pgfpathlineto{\pgfqpoint{-13.996250in}{0.773588in}}%
\pgfpathlineto{\pgfqpoint{-13.950094in}{0.773588in}}%
\pgfpathlineto{\pgfqpoint{-13.901287in}{0.773588in}}%
\pgfpathlineto{\pgfqpoint{-13.853816in}{0.773588in}}%
\pgfpathlineto{\pgfqpoint{-13.806330in}{0.773588in}}%
\pgfpathlineto{\pgfqpoint{-13.757485in}{0.773588in}}%
\pgfpathlineto{\pgfqpoint{-13.710126in}{0.773588in}}%
\pgfpathlineto{\pgfqpoint{-13.663331in}{0.773588in}}%
\pgfpathlineto{\pgfqpoint{-13.615991in}{0.773588in}}%
\pgfpathlineto{\pgfqpoint{-13.568873in}{0.773588in}}%
\pgfpathlineto{\pgfqpoint{-13.522553in}{0.773588in}}%
\pgfpathlineto{\pgfqpoint{-13.474335in}{0.773588in}}%
\pgfpathlineto{\pgfqpoint{-13.428346in}{0.773588in}}%
\pgfpathlineto{\pgfqpoint{-13.382220in}{0.773588in}}%
\pgfpathlineto{\pgfqpoint{-13.334326in}{0.773588in}}%
\pgfpathlineto{\pgfqpoint{-13.287448in}{0.773588in}}%
\pgfpathlineto{\pgfqpoint{-13.239807in}{0.773588in}}%
\pgfpathlineto{\pgfqpoint{-13.190970in}{0.773588in}}%
\pgfpathlineto{\pgfqpoint{-13.144068in}{0.773588in}}%
\pgfpathlineto{\pgfqpoint{-13.096328in}{0.773588in}}%
\pgfpathlineto{\pgfqpoint{-13.046608in}{0.773588in}}%
\pgfpathlineto{\pgfqpoint{-12.998480in}{0.773588in}}%
\pgfpathlineto{\pgfqpoint{-12.949523in}{0.773588in}}%
\pgfpathlineto{\pgfqpoint{-12.899585in}{0.773588in}}%
\pgfpathlineto{\pgfqpoint{-12.852526in}{0.773588in}}%
\pgfpathlineto{\pgfqpoint{-12.805010in}{0.773588in}}%
\pgfpathlineto{\pgfqpoint{-12.755866in}{0.773588in}}%
\pgfpathlineto{\pgfqpoint{-12.708579in}{0.773588in}}%
\pgfpathlineto{\pgfqpoint{-12.661578in}{0.773588in}}%
\pgfpathlineto{\pgfqpoint{-12.612433in}{0.773588in}}%
\pgfpathlineto{\pgfqpoint{-12.565107in}{0.773588in}}%
\pgfpathlineto{\pgfqpoint{-12.517880in}{0.773588in}}%
\pgfpathlineto{\pgfqpoint{-12.469843in}{0.773588in}}%
\pgfpathlineto{\pgfqpoint{-12.422612in}{0.773588in}}%
\pgfpathlineto{\pgfqpoint{-12.375560in}{0.773588in}}%
\pgfpathlineto{\pgfqpoint{-12.326627in}{0.773588in}}%
\pgfpathlineto{\pgfqpoint{-12.278675in}{0.773588in}}%
\pgfpathlineto{\pgfqpoint{-12.231638in}{0.773588in}}%
\pgfpathlineto{\pgfqpoint{-12.182471in}{0.773588in}}%
\pgfpathlineto{\pgfqpoint{-12.135286in}{0.773588in}}%
\pgfpathlineto{\pgfqpoint{-12.087205in}{0.773588in}}%
\pgfpathlineto{\pgfqpoint{-12.038569in}{0.773588in}}%
\pgfpathlineto{\pgfqpoint{-11.991431in}{0.773588in}}%
\pgfpathlineto{\pgfqpoint{-11.943783in}{0.773588in}}%
\pgfpathlineto{\pgfqpoint{-11.895021in}{0.773588in}}%
\pgfpathlineto{\pgfqpoint{-11.847736in}{0.773588in}}%
\pgfpathlineto{\pgfqpoint{-11.800403in}{0.773588in}}%
\pgfpathlineto{\pgfqpoint{-11.751492in}{0.773588in}}%
\pgfpathlineto{\pgfqpoint{-11.704231in}{0.773588in}}%
\pgfpathlineto{\pgfqpoint{-11.656418in}{0.773588in}}%
\pgfpathlineto{\pgfqpoint{-11.607131in}{0.773588in}}%
\pgfpathlineto{\pgfqpoint{-11.559336in}{0.773588in}}%
\pgfpathlineto{\pgfqpoint{-11.511546in}{0.773588in}}%
\pgfpathlineto{\pgfqpoint{-11.463260in}{0.773588in}}%
\pgfpathlineto{\pgfqpoint{-11.415235in}{0.773588in}}%
\pgfpathlineto{\pgfqpoint{-11.366120in}{0.773588in}}%
\pgfpathlineto{\pgfqpoint{-11.316622in}{0.773588in}}%
\pgfpathlineto{\pgfqpoint{-11.268838in}{0.773588in}}%
\pgfpathlineto{\pgfqpoint{-11.221741in}{0.773588in}}%
\pgfpathlineto{\pgfqpoint{-11.172812in}{0.773588in}}%
\pgfpathlineto{\pgfqpoint{-11.125274in}{0.773588in}}%
\pgfpathlineto{\pgfqpoint{-11.077704in}{0.773588in}}%
\pgfpathlineto{\pgfqpoint{-11.028607in}{0.773588in}}%
\pgfpathlineto{\pgfqpoint{-10.980903in}{0.773588in}}%
\pgfpathlineto{\pgfqpoint{-10.933861in}{0.773588in}}%
\pgfpathlineto{\pgfqpoint{-10.885230in}{0.773588in}}%
\pgfpathlineto{\pgfqpoint{-10.837686in}{0.773588in}}%
\pgfpathlineto{\pgfqpoint{-10.790422in}{0.773588in}}%
\pgfpathlineto{\pgfqpoint{-10.741507in}{0.773588in}}%
\pgfpathlineto{\pgfqpoint{-10.693023in}{0.773588in}}%
\pgfpathlineto{\pgfqpoint{-10.644481in}{0.773588in}}%
\pgfpathlineto{\pgfqpoint{-10.594934in}{0.773588in}}%
\pgfpathlineto{\pgfqpoint{-10.546833in}{0.773588in}}%
\pgfpathlineto{\pgfqpoint{-10.497957in}{0.773588in}}%
\pgfpathlineto{\pgfqpoint{-10.448474in}{0.773588in}}%
\pgfpathlineto{\pgfqpoint{-10.401144in}{0.773588in}}%
\pgfpathlineto{\pgfqpoint{-10.352351in}{0.773588in}}%
\pgfpathlineto{\pgfqpoint{-10.301938in}{0.773588in}}%
\pgfpathlineto{\pgfqpoint{-10.253421in}{0.773588in}}%
\pgfpathlineto{\pgfqpoint{-10.204843in}{0.773588in}}%
\pgfpathlineto{\pgfqpoint{-10.154319in}{0.773588in}}%
\pgfpathlineto{\pgfqpoint{-10.106404in}{0.773588in}}%
\pgfpathlineto{\pgfqpoint{-10.058316in}{0.773588in}}%
\pgfpathlineto{\pgfqpoint{-10.009255in}{0.773588in}}%
\pgfpathlineto{\pgfqpoint{-9.960967in}{0.773588in}}%
\pgfpathlineto{\pgfqpoint{-9.912991in}{0.773588in}}%
\pgfpathlineto{\pgfqpoint{-9.863853in}{0.773588in}}%
\pgfpathlineto{\pgfqpoint{-9.816398in}{0.773588in}}%
\pgfpathlineto{\pgfqpoint{-9.768438in}{0.773588in}}%
\pgfpathlineto{\pgfqpoint{-9.719205in}{0.773588in}}%
\pgfpathlineto{\pgfqpoint{-9.671880in}{0.773588in}}%
\pgfpathlineto{\pgfqpoint{-9.624467in}{0.773588in}}%
\pgfpathlineto{\pgfqpoint{-9.575433in}{0.773588in}}%
\pgfpathlineto{\pgfqpoint{-9.527213in}{0.773588in}}%
\pgfpathlineto{\pgfqpoint{-9.479530in}{0.773588in}}%
\pgfpathlineto{\pgfqpoint{-9.430893in}{0.773588in}}%
\pgfpathlineto{\pgfqpoint{-9.382453in}{0.773588in}}%
\pgfpathlineto{\pgfqpoint{-9.332833in}{0.773588in}}%
\pgfpathlineto{\pgfqpoint{-9.281795in}{0.773588in}}%
\pgfpathlineto{\pgfqpoint{-9.232466in}{0.773588in}}%
\pgfpathlineto{\pgfqpoint{-9.182531in}{0.773588in}}%
\pgfpathlineto{\pgfqpoint{-9.130552in}{0.773588in}}%
\pgfpathlineto{\pgfqpoint{-9.080221in}{0.773588in}}%
\pgfpathlineto{\pgfqpoint{-9.030683in}{0.773588in}}%
\pgfpathlineto{\pgfqpoint{-8.979731in}{0.773588in}}%
\pgfpathlineto{\pgfqpoint{-8.930541in}{0.773588in}}%
\pgfpathlineto{\pgfqpoint{-8.880937in}{0.773588in}}%
\pgfpathlineto{\pgfqpoint{-8.829855in}{0.773588in}}%
\pgfpathlineto{\pgfqpoint{-8.780578in}{0.773588in}}%
\pgfpathlineto{\pgfqpoint{-8.731477in}{0.773588in}}%
\pgfpathlineto{\pgfqpoint{-8.681529in}{0.773588in}}%
\pgfpathlineto{\pgfqpoint{-8.632588in}{0.773588in}}%
\pgfpathlineto{\pgfqpoint{-8.583181in}{0.773588in}}%
\pgfpathlineto{\pgfqpoint{-8.532027in}{0.773588in}}%
\pgfpathlineto{\pgfqpoint{-8.481981in}{0.773588in}}%
\pgfpathlineto{\pgfqpoint{-8.432560in}{0.773588in}}%
\pgfpathlineto{\pgfqpoint{-8.382452in}{0.773588in}}%
\pgfpathlineto{\pgfqpoint{-8.332630in}{0.773588in}}%
\pgfpathlineto{\pgfqpoint{-8.283775in}{0.773588in}}%
\pgfpathlineto{\pgfqpoint{-8.233427in}{0.773588in}}%
\pgfpathlineto{\pgfqpoint{-8.184462in}{0.773588in}}%
\pgfpathlineto{\pgfqpoint{-8.135490in}{0.773588in}}%
\pgfpathlineto{\pgfqpoint{-8.085190in}{0.773588in}}%
\pgfpathlineto{\pgfqpoint{-8.035792in}{0.773588in}}%
\pgfpathlineto{\pgfqpoint{-7.986412in}{0.773588in}}%
\pgfpathlineto{\pgfqpoint{-7.935315in}{0.773588in}}%
\pgfpathlineto{\pgfqpoint{-7.885122in}{0.773588in}}%
\pgfpathlineto{\pgfqpoint{-7.835188in}{0.773588in}}%
\pgfpathlineto{\pgfqpoint{-7.783607in}{0.773588in}}%
\pgfpathlineto{\pgfqpoint{-7.733870in}{0.773588in}}%
\pgfpathlineto{\pgfqpoint{-7.685281in}{0.773588in}}%
\pgfpathlineto{\pgfqpoint{-7.634521in}{0.773588in}}%
\pgfpathlineto{\pgfqpoint{-7.585672in}{0.773588in}}%
\pgfpathlineto{\pgfqpoint{-7.536566in}{0.773588in}}%
\pgfpathlineto{\pgfqpoint{-7.485312in}{0.773588in}}%
\pgfpathlineto{\pgfqpoint{-7.435602in}{0.773588in}}%
\pgfpathlineto{\pgfqpoint{-7.386343in}{0.773588in}}%
\pgfpathlineto{\pgfqpoint{-7.335791in}{0.773588in}}%
\pgfpathlineto{\pgfqpoint{-7.286238in}{0.773588in}}%
\pgfpathlineto{\pgfqpoint{-7.237328in}{0.773588in}}%
\pgfpathlineto{\pgfqpoint{-7.185558in}{0.773588in}}%
\pgfpathlineto{\pgfqpoint{-7.135511in}{0.773588in}}%
\pgfpathlineto{\pgfqpoint{-7.085212in}{0.773588in}}%
\pgfpathlineto{\pgfqpoint{-7.033318in}{0.773588in}}%
\pgfpathlineto{\pgfqpoint{-6.984061in}{0.773588in}}%
\pgfpathlineto{\pgfqpoint{-6.934169in}{0.773588in}}%
\pgfpathlineto{\pgfqpoint{-6.882593in}{0.773588in}}%
\pgfpathlineto{\pgfqpoint{-6.832072in}{0.773588in}}%
\pgfpathlineto{\pgfqpoint{-6.782488in}{0.773588in}}%
\pgfpathlineto{\pgfqpoint{-6.731323in}{0.773588in}}%
\pgfpathlineto{\pgfqpoint{-6.682747in}{0.773588in}}%
\pgfpathlineto{\pgfqpoint{-6.634445in}{0.773588in}}%
\pgfpathlineto{\pgfqpoint{-6.582317in}{0.773588in}}%
\pgfpathlineto{\pgfqpoint{-6.532179in}{0.773588in}}%
\pgfpathlineto{\pgfqpoint{-6.481910in}{0.773588in}}%
\pgfpathlineto{\pgfqpoint{-6.429787in}{0.773588in}}%
\pgfpathlineto{\pgfqpoint{-6.380124in}{0.773588in}}%
\pgfpathlineto{\pgfqpoint{-6.330959in}{0.773588in}}%
\pgfpathlineto{\pgfqpoint{-6.279011in}{0.773588in}}%
\pgfpathlineto{\pgfqpoint{-6.228389in}{0.773588in}}%
\pgfpathlineto{\pgfqpoint{-6.177451in}{0.773588in}}%
\pgfpathlineto{\pgfqpoint{-6.125420in}{0.773588in}}%
\pgfpathlineto{\pgfqpoint{-6.075227in}{0.773588in}}%
\pgfpathlineto{\pgfqpoint{-6.024836in}{0.773588in}}%
\pgfpathlineto{\pgfqpoint{-5.974236in}{0.773588in}}%
\pgfpathlineto{\pgfqpoint{-5.924128in}{0.773588in}}%
\pgfpathlineto{\pgfqpoint{-5.874278in}{0.773588in}}%
\pgfpathlineto{\pgfqpoint{-5.822716in}{0.773588in}}%
\pgfpathlineto{\pgfqpoint{-5.774223in}{0.773588in}}%
\pgfpathlineto{\pgfqpoint{-5.724570in}{0.773588in}}%
\pgfpathlineto{\pgfqpoint{-5.673071in}{0.773588in}}%
\pgfpathlineto{\pgfqpoint{-5.623816in}{0.773588in}}%
\pgfpathlineto{\pgfqpoint{-5.574633in}{0.773588in}}%
\pgfpathlineto{\pgfqpoint{-5.523472in}{0.773588in}}%
\pgfpathlineto{\pgfqpoint{-5.473058in}{0.773588in}}%
\pgfpathlineto{\pgfqpoint{-5.421631in}{0.773588in}}%
\pgfpathlineto{\pgfqpoint{-5.369703in}{0.773588in}}%
\pgfpathlineto{\pgfqpoint{-5.319212in}{0.773588in}}%
\pgfpathlineto{\pgfqpoint{-5.267508in}{0.773588in}}%
\pgfpathlineto{\pgfqpoint{-5.215272in}{0.773588in}}%
\pgfpathlineto{\pgfqpoint{-5.165135in}{0.773588in}}%
\pgfpathlineto{\pgfqpoint{-5.114869in}{0.773588in}}%
\pgfpathlineto{\pgfqpoint{-5.064238in}{0.773588in}}%
\pgfpathlineto{\pgfqpoint{-5.014255in}{0.773588in}}%
\pgfpathlineto{\pgfqpoint{-4.963670in}{0.773588in}}%
\pgfpathlineto{\pgfqpoint{-4.911765in}{0.773588in}}%
\pgfpathlineto{\pgfqpoint{-4.861887in}{0.773588in}}%
\pgfpathlineto{\pgfqpoint{-4.812147in}{0.773588in}}%
\pgfpathlineto{\pgfqpoint{-4.759589in}{0.773588in}}%
\pgfpathlineto{\pgfqpoint{-4.709756in}{0.773588in}}%
\pgfpathlineto{\pgfqpoint{-4.659647in}{0.773588in}}%
\pgfpathlineto{\pgfqpoint{-4.607841in}{0.773588in}}%
\pgfpathlineto{\pgfqpoint{-4.558006in}{0.773588in}}%
\pgfpathlineto{\pgfqpoint{-4.508002in}{0.773588in}}%
\pgfpathlineto{\pgfqpoint{-4.457276in}{0.773588in}}%
\pgfpathlineto{\pgfqpoint{-4.407934in}{0.773588in}}%
\pgfpathlineto{\pgfqpoint{-4.357515in}{0.773588in}}%
\pgfpathlineto{\pgfqpoint{-4.305607in}{0.773588in}}%
\pgfpathlineto{\pgfqpoint{-4.255103in}{0.773588in}}%
\pgfpathlineto{\pgfqpoint{-4.204591in}{0.773588in}}%
\pgfpathlineto{\pgfqpoint{-4.153202in}{0.773588in}}%
\pgfpathlineto{\pgfqpoint{-4.103923in}{0.773588in}}%
\pgfpathlineto{\pgfqpoint{-4.053537in}{0.773588in}}%
\pgfpathlineto{\pgfqpoint{-4.002524in}{0.773588in}}%
\pgfpathlineto{\pgfqpoint{-3.952191in}{0.773588in}}%
\pgfpathlineto{\pgfqpoint{-3.902764in}{0.773588in}}%
\pgfpathlineto{\pgfqpoint{-3.850587in}{0.773588in}}%
\pgfpathlineto{\pgfqpoint{-3.799759in}{0.773588in}}%
\pgfpathlineto{\pgfqpoint{-3.748633in}{0.773588in}}%
\pgfpathlineto{\pgfqpoint{-3.696993in}{0.773588in}}%
\pgfpathlineto{\pgfqpoint{-3.646072in}{0.773588in}}%
\pgfpathlineto{\pgfqpoint{-3.595859in}{0.773588in}}%
\pgfpathlineto{\pgfqpoint{-3.544176in}{0.773588in}}%
\pgfpathlineto{\pgfqpoint{-3.494154in}{0.773588in}}%
\pgfpathlineto{\pgfqpoint{-3.444140in}{0.773588in}}%
\pgfpathlineto{\pgfqpoint{-3.392015in}{0.773588in}}%
\pgfpathlineto{\pgfqpoint{-3.341930in}{0.773588in}}%
\pgfpathlineto{\pgfqpoint{-3.292250in}{0.773588in}}%
\pgfpathlineto{\pgfqpoint{-3.241308in}{0.773588in}}%
\pgfpathlineto{\pgfqpoint{-3.190882in}{0.773588in}}%
\pgfpathlineto{\pgfqpoint{-3.140428in}{0.773588in}}%
\pgfpathlineto{\pgfqpoint{-3.088627in}{0.773588in}}%
\pgfpathlineto{\pgfqpoint{-3.039557in}{0.773588in}}%
\pgfpathlineto{\pgfqpoint{-2.990316in}{0.773588in}}%
\pgfpathlineto{\pgfqpoint{-2.938490in}{0.773588in}}%
\pgfpathlineto{\pgfqpoint{-2.887081in}{0.773588in}}%
\pgfpathlineto{\pgfqpoint{-2.836699in}{0.773588in}}%
\pgfpathlineto{\pgfqpoint{-2.784277in}{0.773588in}}%
\pgfpathlineto{\pgfqpoint{-2.732985in}{0.773588in}}%
\pgfpathlineto{\pgfqpoint{-2.681969in}{0.773588in}}%
\pgfpathlineto{\pgfqpoint{-2.629697in}{0.773588in}}%
\pgfpathlineto{\pgfqpoint{-2.578972in}{0.773588in}}%
\pgfpathlineto{\pgfqpoint{-2.528440in}{0.773588in}}%
\pgfpathlineto{\pgfqpoint{-2.476616in}{0.773588in}}%
\pgfpathlineto{\pgfqpoint{-2.425397in}{0.773588in}}%
\pgfpathlineto{\pgfqpoint{-2.375607in}{0.773588in}}%
\pgfpathlineto{\pgfqpoint{-2.323084in}{0.773588in}}%
\pgfpathlineto{\pgfqpoint{-2.272195in}{0.773588in}}%
\pgfpathlineto{\pgfqpoint{-2.220871in}{0.773588in}}%
\pgfpathlineto{\pgfqpoint{-2.167559in}{0.773588in}}%
\pgfpathlineto{\pgfqpoint{-2.116467in}{0.773588in}}%
\pgfpathlineto{\pgfqpoint{-2.064986in}{0.773588in}}%
\pgfpathlineto{\pgfqpoint{-2.013282in}{0.773588in}}%
\pgfpathlineto{\pgfqpoint{-1.963080in}{0.773588in}}%
\pgfpathlineto{\pgfqpoint{-1.912972in}{0.773588in}}%
\pgfpathlineto{\pgfqpoint{-1.860296in}{0.773588in}}%
\pgfpathlineto{\pgfqpoint{-1.809967in}{0.773588in}}%
\pgfpathlineto{\pgfqpoint{-1.759793in}{0.773588in}}%
\pgfpathlineto{\pgfqpoint{-1.707851in}{0.773588in}}%
\pgfpathlineto{\pgfqpoint{-1.657574in}{0.773588in}}%
\pgfpathlineto{\pgfqpoint{-1.606814in}{0.773588in}}%
\pgfpathlineto{\pgfqpoint{-1.553079in}{0.773588in}}%
\pgfpathlineto{\pgfqpoint{-1.502359in}{0.773588in}}%
\pgfpathlineto{\pgfqpoint{-1.451018in}{0.773588in}}%
\pgfpathlineto{\pgfqpoint{-1.397583in}{0.773588in}}%
\pgfpathlineto{\pgfqpoint{-1.346574in}{0.773588in}}%
\pgfpathlineto{\pgfqpoint{-1.295495in}{0.773588in}}%
\pgfpathlineto{\pgfqpoint{-1.242201in}{0.773588in}}%
\pgfpathlineto{\pgfqpoint{-1.191111in}{0.773588in}}%
\pgfpathlineto{\pgfqpoint{-1.140825in}{0.773588in}}%
\pgfpathlineto{\pgfqpoint{-1.088683in}{0.773588in}}%
\pgfpathlineto{\pgfqpoint{-1.038293in}{0.773588in}}%
\pgfpathlineto{\pgfqpoint{-0.987007in}{0.773588in}}%
\pgfpathlineto{\pgfqpoint{-0.934193in}{0.773588in}}%
\pgfpathlineto{\pgfqpoint{-0.883439in}{0.773588in}}%
\pgfpathlineto{\pgfqpoint{-0.832397in}{0.773588in}}%
\pgfpathlineto{\pgfqpoint{-0.780794in}{0.773588in}}%
\pgfpathlineto{\pgfqpoint{-0.729960in}{0.773588in}}%
\pgfpathlineto{\pgfqpoint{-0.679845in}{0.773588in}}%
\pgfpathlineto{\pgfqpoint{-0.628072in}{0.773588in}}%
\pgfpathlineto{\pgfqpoint{-0.577558in}{0.773588in}}%
\pgfpathlineto{\pgfqpoint{-0.525798in}{0.773588in}}%
\pgfpathlineto{\pgfqpoint{-0.472607in}{0.773588in}}%
\pgfpathlineto{\pgfqpoint{-0.421082in}{0.773588in}}%
\pgfpathlineto{\pgfqpoint{-0.370494in}{0.773588in}}%
\pgfpathlineto{\pgfqpoint{-0.317282in}{0.773588in}}%
\pgfpathlineto{\pgfqpoint{-0.265117in}{0.773588in}}%
\pgfpathlineto{\pgfqpoint{-0.212446in}{0.773588in}}%
\pgfpathlineto{\pgfqpoint{-0.159327in}{0.773588in}}%
\pgfpathlineto{\pgfqpoint{-0.106747in}{0.773588in}}%
\pgfpathlineto{\pgfqpoint{-0.053884in}{0.773588in}}%
\pgfpathlineto{\pgfqpoint{0.000028in}{0.773588in}}%
\pgfpathlineto{\pgfqpoint{0.052190in}{0.773588in}}%
\pgfpathlineto{\pgfqpoint{0.103930in}{0.773588in}}%
\pgfpathlineto{\pgfqpoint{0.157615in}{0.773588in}}%
\pgfpathlineto{\pgfqpoint{0.209933in}{0.773588in}}%
\pgfpathlineto{\pgfqpoint{0.261843in}{0.773588in}}%
\pgfpathlineto{\pgfqpoint{0.315264in}{0.773588in}}%
\pgfpathlineto{\pgfqpoint{0.366948in}{0.773588in}}%
\pgfpathlineto{\pgfqpoint{0.419063in}{0.773588in}}%
\pgfpathlineto{\pgfqpoint{0.472767in}{0.773588in}}%
\pgfpathlineto{\pgfqpoint{0.524412in}{0.773588in}}%
\pgfpathlineto{\pgfqpoint{0.575921in}{0.773588in}}%
\pgfpathlineto{\pgfqpoint{0.628822in}{0.773588in}}%
\pgfpathlineto{\pgfqpoint{0.680101in}{0.773588in}}%
\pgfpathlineto{\pgfqpoint{0.730937in}{0.773588in}}%
\pgfpathlineto{\pgfqpoint{0.783351in}{0.773588in}}%
\pgfpathlineto{\pgfqpoint{0.835638in}{0.773588in}}%
\pgfpathlineto{\pgfqpoint{0.890559in}{0.773588in}}%
\pgfpathlineto{\pgfqpoint{0.950896in}{0.773588in}}%
\pgfpathlineto{\pgfqpoint{1.011418in}{0.773588in}}%
\pgfpathlineto{\pgfqpoint{1.071223in}{0.773588in}}%
\pgfpathlineto{\pgfqpoint{1.133818in}{0.773588in}}%
\pgfpathlineto{\pgfqpoint{1.196854in}{0.773588in}}%
\pgfpathlineto{\pgfqpoint{1.262232in}{0.773588in}}%
\pgfpathlineto{\pgfqpoint{1.331848in}{0.773588in}}%
\pgfpathlineto{\pgfqpoint{1.399923in}{0.773588in}}%
\pgfpathlineto{\pgfqpoint{1.469539in}{0.773588in}}%
\pgfpathlineto{\pgfqpoint{1.542225in}{0.773588in}}%
\pgfpathlineto{\pgfqpoint{1.614592in}{0.773588in}}%
\pgfpathlineto{\pgfqpoint{1.687208in}{0.773588in}}%
\pgfpathlineto{\pgfqpoint{1.764147in}{0.773588in}}%
\pgfpathlineto{\pgfqpoint{1.839330in}{0.773588in}}%
\pgfpathlineto{\pgfqpoint{1.918401in}{0.773588in}}%
\pgfpathlineto{\pgfqpoint{1.999401in}{0.773588in}}%
\pgfpathlineto{\pgfqpoint{2.077252in}{0.773588in}}%
\pgfpathlineto{\pgfqpoint{2.157128in}{0.773588in}}%
\pgfpathlineto{\pgfqpoint{2.243381in}{0.773588in}}%
\pgfpathlineto{\pgfqpoint{2.328903in}{0.773588in}}%
\pgfpathlineto{\pgfqpoint{2.416579in}{0.773588in}}%
\pgfpathlineto{\pgfqpoint{2.504252in}{0.773588in}}%
\pgfpathlineto{\pgfqpoint{2.589494in}{0.773588in}}%
\pgfpathlineto{\pgfqpoint{2.676066in}{0.773588in}}%
\pgfpathlineto{\pgfqpoint{2.767061in}{0.773588in}}%
\pgfpathlineto{\pgfqpoint{2.858701in}{0.773588in}}%
\pgfpathlineto{\pgfqpoint{2.948242in}{0.773588in}}%
\pgfpathlineto{\pgfqpoint{3.043068in}{0.773588in}}%
\pgfpathlineto{\pgfqpoint{3.132459in}{0.773588in}}%
\pgfpathlineto{\pgfqpoint{3.198420in}{0.773588in}}%
\pgfpathlineto{\pgfqpoint{3.252590in}{0.773588in}}%
\pgfpathlineto{\pgfqpoint{3.305067in}{0.773588in}}%
\pgfpathlineto{\pgfqpoint{3.357008in}{0.773588in}}%
\pgfpathlineto{\pgfqpoint{3.411084in}{0.773588in}}%
\pgfpathlineto{\pgfqpoint{3.464430in}{0.773588in}}%
\pgfpathlineto{\pgfqpoint{3.517522in}{0.773588in}}%
\pgfpathlineto{\pgfqpoint{3.571722in}{0.773588in}}%
\pgfpathlineto{\pgfqpoint{3.624008in}{0.773588in}}%
\pgfpathlineto{\pgfqpoint{3.676588in}{0.773588in}}%
\pgfpathlineto{\pgfqpoint{3.729373in}{0.773588in}}%
\pgfpathlineto{\pgfqpoint{3.781554in}{0.773588in}}%
\pgfpathlineto{\pgfqpoint{3.833685in}{0.773588in}}%
\pgfpathlineto{\pgfqpoint{3.887639in}{0.773588in}}%
\pgfpathlineto{\pgfqpoint{3.940507in}{0.773588in}}%
\pgfpathlineto{\pgfqpoint{3.993337in}{0.773588in}}%
\pgfpathlineto{\pgfqpoint{4.047138in}{0.773588in}}%
\pgfpathlineto{\pgfqpoint{4.098630in}{0.773588in}}%
\pgfpathlineto{\pgfqpoint{4.150734in}{0.773588in}}%
\pgfpathlineto{\pgfqpoint{4.204360in}{0.773588in}}%
\pgfpathlineto{\pgfqpoint{4.256228in}{0.773588in}}%
\pgfpathlineto{\pgfqpoint{4.307535in}{0.773588in}}%
\pgfpathlineto{\pgfqpoint{4.360055in}{0.773588in}}%
\pgfpathlineto{\pgfqpoint{4.398305in}{0.773588in}}%
\pgfpathlineto{\pgfqpoint{4.446119in}{0.773588in}}%
\pgfpathlineto{\pgfqpoint{4.484596in}{1.246842in}}%
\pgfpathlineto{\pgfqpoint{4.526846in}{1.377266in}}%
\pgfpathlineto{\pgfqpoint{4.566686in}{1.533299in}}%
\pgfpathlineto{\pgfqpoint{4.603496in}{1.768315in}}%
\pgfpathlineto{\pgfqpoint{4.635492in}{2.208863in}}%
\pgfpathlineto{\pgfqpoint{4.666016in}{2.790032in}}%
\pgfpathlineto{\pgfqpoint{4.690368in}{3.753732in}}%
\pgfpathlineto{\pgfqpoint{4.715513in}{3.612521in}}%
\pgfpathlineto{\pgfqpoint{4.739291in}{3.898067in}}%
\pgfpathlineto{\pgfqpoint{4.764034in}{3.941303in}}%
\pgfpathlineto{\pgfqpoint{4.787716in}{3.942672in}}%
\pgfpathlineto{\pgfqpoint{4.811295in}{3.869449in}}%
\pgfpathlineto{\pgfqpoint{4.836687in}{3.988507in}}%
\pgfpathlineto{\pgfqpoint{4.860218in}{4.040966in}}%
\pgfpathlineto{\pgfqpoint{4.884852in}{3.877862in}}%
\pgfpathlineto{\pgfqpoint{4.907640in}{3.930055in}}%
\pgfpathlineto{\pgfqpoint{4.931890in}{4.075356in}}%
\pgfpathlineto{\pgfqpoint{4.954839in}{3.972416in}}%
\pgfpathlineto{\pgfqpoint{4.980174in}{3.888388in}}%
\pgfpathlineto{\pgfqpoint{5.002737in}{3.916080in}}%
\pgfpathlineto{\pgfqpoint{5.026810in}{3.769623in}}%
\pgfpathlineto{\pgfqpoint{5.051612in}{3.898493in}}%
\pgfpathlineto{\pgfqpoint{5.074798in}{4.217964in}}%
\pgfpathlineto{\pgfqpoint{5.097977in}{4.134872in}}%
\pgfpathlineto{\pgfqpoint{5.122448in}{3.944352in}}%
\pgfpathlineto{\pgfqpoint{5.145219in}{4.171934in}}%
\pgfpathlineto{\pgfqpoint{5.168068in}{4.127126in}}%
\pgfpathlineto{\pgfqpoint{5.192095in}{4.194563in}}%
\pgfpathlineto{\pgfqpoint{5.214897in}{4.217950in}}%
\pgfpathlineto{\pgfqpoint{5.237842in}{4.124102in}}%
\pgfpathlineto{\pgfqpoint{5.261861in}{4.071453in}}%
\pgfpathlineto{\pgfqpoint{5.285086in}{3.997581in}}%
\pgfpathlineto{\pgfqpoint{5.307824in}{4.254313in}}%
\pgfpathlineto{\pgfqpoint{5.331834in}{4.168563in}}%
\pgfpathlineto{\pgfqpoint{5.355254in}{4.195680in}}%
\pgfpathlineto{\pgfqpoint{5.377738in}{4.251407in}}%
\pgfpathlineto{\pgfqpoint{5.402166in}{4.142544in}}%
\pgfpathlineto{\pgfqpoint{5.424187in}{4.192246in}}%
\pgfpathlineto{\pgfqpoint{5.448001in}{4.201329in}}%
\pgfpathlineto{\pgfqpoint{5.470431in}{4.232916in}}%
\pgfpathlineto{\pgfqpoint{5.494250in}{4.201901in}}%
\pgfpathlineto{\pgfqpoint{5.516676in}{4.198446in}}%
\pgfpathlineto{\pgfqpoint{5.540659in}{4.231403in}}%
\pgfpathlineto{\pgfqpoint{5.562835in}{4.239053in}}%
\pgfpathlineto{\pgfqpoint{5.587139in}{3.989480in}}%
\pgfpathlineto{\pgfqpoint{5.609949in}{4.101072in}}%
\pgfpathlineto{\pgfqpoint{5.634039in}{4.182959in}}%
\pgfpathlineto{\pgfqpoint{5.661246in}{4.038936in}}%
\pgfpathlineto{\pgfqpoint{5.712060in}{3.993842in}}%
\pgfpathlineto{\pgfqpoint{5.763148in}{3.993842in}}%
\pgfpathlineto{\pgfqpoint{5.815025in}{3.993842in}}%
\pgfpathlineto{\pgfqpoint{5.867920in}{3.993842in}}%
\pgfpathlineto{\pgfqpoint{5.919631in}{3.993842in}}%
\pgfpathlineto{\pgfqpoint{5.971565in}{3.993842in}}%
\pgfpathlineto{\pgfqpoint{6.025764in}{3.993842in}}%
\pgfpathlineto{\pgfqpoint{6.078797in}{3.993842in}}%
\pgfpathlineto{\pgfqpoint{6.078797in}{5.758501in}}%
\pgfpathlineto{\pgfqpoint{6.078797in}{5.758501in}}%
\pgfpathlineto{\pgfqpoint{6.025764in}{5.758501in}}%
\pgfpathlineto{\pgfqpoint{5.971565in}{5.758501in}}%
\pgfpathlineto{\pgfqpoint{5.919631in}{5.758501in}}%
\pgfpathlineto{\pgfqpoint{5.867920in}{5.758501in}}%
\pgfpathlineto{\pgfqpoint{5.815025in}{5.758501in}}%
\pgfpathlineto{\pgfqpoint{5.763148in}{5.758501in}}%
\pgfpathlineto{\pgfqpoint{5.712060in}{5.758501in}}%
\pgfpathlineto{\pgfqpoint{5.661246in}{5.775449in}}%
\pgfpathlineto{\pgfqpoint{5.634039in}{5.865058in}}%
\pgfpathlineto{\pgfqpoint{5.609949in}{5.825752in}}%
\pgfpathlineto{\pgfqpoint{5.587139in}{5.726189in}}%
\pgfpathlineto{\pgfqpoint{5.562835in}{5.930845in}}%
\pgfpathlineto{\pgfqpoint{5.540659in}{5.899825in}}%
\pgfpathlineto{\pgfqpoint{5.516676in}{5.880249in}}%
\pgfpathlineto{\pgfqpoint{5.494250in}{5.893046in}}%
\pgfpathlineto{\pgfqpoint{5.470431in}{5.911984in}}%
\pgfpathlineto{\pgfqpoint{5.448001in}{5.857730in}}%
\pgfpathlineto{\pgfqpoint{5.424187in}{5.868196in}}%
\pgfpathlineto{\pgfqpoint{5.402166in}{5.815038in}}%
\pgfpathlineto{\pgfqpoint{5.377738in}{5.918658in}}%
\pgfpathlineto{\pgfqpoint{5.355254in}{5.854178in}}%
\pgfpathlineto{\pgfqpoint{5.331834in}{5.813609in}}%
\pgfpathlineto{\pgfqpoint{5.307824in}{5.908282in}}%
\pgfpathlineto{\pgfqpoint{5.285086in}{5.706485in}}%
\pgfpathlineto{\pgfqpoint{5.261861in}{5.744292in}}%
\pgfpathlineto{\pgfqpoint{5.237842in}{5.791091in}}%
\pgfpathlineto{\pgfqpoint{5.214897in}{5.862622in}}%
\pgfpathlineto{\pgfqpoint{5.192095in}{5.846668in}}%
\pgfpathlineto{\pgfqpoint{5.168068in}{5.745325in}}%
\pgfpathlineto{\pgfqpoint{5.145219in}{5.809148in}}%
\pgfpathlineto{\pgfqpoint{5.122448in}{5.601241in}}%
\pgfpathlineto{\pgfqpoint{5.097977in}{5.740043in}}%
\pgfpathlineto{\pgfqpoint{5.074798in}{5.834541in}}%
\pgfpathlineto{\pgfqpoint{5.051612in}{5.582839in}}%
\pgfpathlineto{\pgfqpoint{5.026810in}{5.442515in}}%
\pgfpathlineto{\pgfqpoint{5.002737in}{5.557546in}}%
\pgfpathlineto{\pgfqpoint{4.980174in}{5.504363in}}%
\pgfpathlineto{\pgfqpoint{4.954839in}{5.584654in}}%
\pgfpathlineto{\pgfqpoint{4.931890in}{5.678338in}}%
\pgfpathlineto{\pgfqpoint{4.907640in}{5.518952in}}%
\pgfpathlineto{\pgfqpoint{4.884852in}{5.460686in}}%
\pgfpathlineto{\pgfqpoint{4.860218in}{5.534473in}}%
\pgfpathlineto{\pgfqpoint{4.836687in}{5.484292in}}%
\pgfpathlineto{\pgfqpoint{4.811295in}{5.442565in}}%
\pgfpathlineto{\pgfqpoint{4.787716in}{5.446238in}}%
\pgfpathlineto{\pgfqpoint{4.764034in}{5.483715in}}%
\pgfpathlineto{\pgfqpoint{4.739291in}{5.364631in}}%
\pgfpathlineto{\pgfqpoint{4.715513in}{5.182737in}}%
\pgfpathlineto{\pgfqpoint{4.690368in}{5.235025in}}%
\pgfpathlineto{\pgfqpoint{4.666016in}{3.729137in}}%
\pgfpathlineto{\pgfqpoint{4.635492in}{2.864879in}}%
\pgfpathlineto{\pgfqpoint{4.603496in}{2.224563in}}%
\pgfpathlineto{\pgfqpoint{4.566686in}{1.906942in}}%
\pgfpathlineto{\pgfqpoint{4.526846in}{1.669618in}}%
\pgfpathlineto{\pgfqpoint{4.484596in}{1.454760in}}%
\pgfpathlineto{\pgfqpoint{4.446119in}{0.773588in}}%
\pgfpathlineto{\pgfqpoint{4.398305in}{0.773588in}}%
\pgfpathlineto{\pgfqpoint{4.360055in}{0.773588in}}%
\pgfpathlineto{\pgfqpoint{4.307535in}{0.773588in}}%
\pgfpathlineto{\pgfqpoint{4.256228in}{0.773588in}}%
\pgfpathlineto{\pgfqpoint{4.204360in}{0.773588in}}%
\pgfpathlineto{\pgfqpoint{4.150734in}{0.773588in}}%
\pgfpathlineto{\pgfqpoint{4.098630in}{0.773588in}}%
\pgfpathlineto{\pgfqpoint{4.047138in}{0.773588in}}%
\pgfpathlineto{\pgfqpoint{3.993337in}{0.773588in}}%
\pgfpathlineto{\pgfqpoint{3.940507in}{0.773588in}}%
\pgfpathlineto{\pgfqpoint{3.887639in}{0.773588in}}%
\pgfpathlineto{\pgfqpoint{3.833685in}{0.773588in}}%
\pgfpathlineto{\pgfqpoint{3.781554in}{0.773588in}}%
\pgfpathlineto{\pgfqpoint{3.729373in}{0.773588in}}%
\pgfpathlineto{\pgfqpoint{3.676588in}{0.773588in}}%
\pgfpathlineto{\pgfqpoint{3.624008in}{0.773588in}}%
\pgfpathlineto{\pgfqpoint{3.571722in}{0.773588in}}%
\pgfpathlineto{\pgfqpoint{3.517522in}{0.773588in}}%
\pgfpathlineto{\pgfqpoint{3.464430in}{0.773588in}}%
\pgfpathlineto{\pgfqpoint{3.411084in}{0.773588in}}%
\pgfpathlineto{\pgfqpoint{3.357008in}{0.773588in}}%
\pgfpathlineto{\pgfqpoint{3.305067in}{0.773588in}}%
\pgfpathlineto{\pgfqpoint{3.252590in}{0.773588in}}%
\pgfpathlineto{\pgfqpoint{3.198420in}{0.773588in}}%
\pgfpathlineto{\pgfqpoint{3.132459in}{0.773588in}}%
\pgfpathlineto{\pgfqpoint{3.043068in}{0.773588in}}%
\pgfpathlineto{\pgfqpoint{2.948242in}{0.773588in}}%
\pgfpathlineto{\pgfqpoint{2.858701in}{0.773588in}}%
\pgfpathlineto{\pgfqpoint{2.767061in}{0.773588in}}%
\pgfpathlineto{\pgfqpoint{2.676066in}{0.773588in}}%
\pgfpathlineto{\pgfqpoint{2.589494in}{0.773588in}}%
\pgfpathlineto{\pgfqpoint{2.504252in}{0.773588in}}%
\pgfpathlineto{\pgfqpoint{2.416579in}{0.773588in}}%
\pgfpathlineto{\pgfqpoint{2.328903in}{0.773588in}}%
\pgfpathlineto{\pgfqpoint{2.243381in}{0.773588in}}%
\pgfpathlineto{\pgfqpoint{2.157128in}{0.773588in}}%
\pgfpathlineto{\pgfqpoint{2.077252in}{0.773588in}}%
\pgfpathlineto{\pgfqpoint{1.999401in}{0.773588in}}%
\pgfpathlineto{\pgfqpoint{1.918401in}{0.773588in}}%
\pgfpathlineto{\pgfqpoint{1.839330in}{0.773588in}}%
\pgfpathlineto{\pgfqpoint{1.764147in}{0.773588in}}%
\pgfpathlineto{\pgfqpoint{1.687208in}{0.773588in}}%
\pgfpathlineto{\pgfqpoint{1.614592in}{0.773588in}}%
\pgfpathlineto{\pgfqpoint{1.542225in}{0.773588in}}%
\pgfpathlineto{\pgfqpoint{1.469539in}{0.773588in}}%
\pgfpathlineto{\pgfqpoint{1.399923in}{0.773588in}}%
\pgfpathlineto{\pgfqpoint{1.331848in}{0.773588in}}%
\pgfpathlineto{\pgfqpoint{1.262232in}{0.773588in}}%
\pgfpathlineto{\pgfqpoint{1.196854in}{0.773588in}}%
\pgfpathlineto{\pgfqpoint{1.133818in}{0.773588in}}%
\pgfpathlineto{\pgfqpoint{1.071223in}{0.773588in}}%
\pgfpathlineto{\pgfqpoint{1.011418in}{0.773588in}}%
\pgfpathlineto{\pgfqpoint{0.950896in}{0.773588in}}%
\pgfpathlineto{\pgfqpoint{0.890559in}{0.773588in}}%
\pgfpathlineto{\pgfqpoint{0.835638in}{0.773588in}}%
\pgfpathlineto{\pgfqpoint{0.783351in}{0.773588in}}%
\pgfpathlineto{\pgfqpoint{0.730937in}{0.773588in}}%
\pgfpathlineto{\pgfqpoint{0.680101in}{0.773588in}}%
\pgfpathlineto{\pgfqpoint{0.628822in}{0.773588in}}%
\pgfpathlineto{\pgfqpoint{0.575921in}{0.773588in}}%
\pgfpathlineto{\pgfqpoint{0.524412in}{0.773588in}}%
\pgfpathlineto{\pgfqpoint{0.472767in}{0.773588in}}%
\pgfpathlineto{\pgfqpoint{0.419063in}{0.773588in}}%
\pgfpathlineto{\pgfqpoint{0.366948in}{0.773588in}}%
\pgfpathlineto{\pgfqpoint{0.315264in}{0.773588in}}%
\pgfpathlineto{\pgfqpoint{0.261843in}{0.773588in}}%
\pgfpathlineto{\pgfqpoint{0.209933in}{0.773588in}}%
\pgfpathlineto{\pgfqpoint{0.157615in}{0.773588in}}%
\pgfpathlineto{\pgfqpoint{0.103930in}{0.773588in}}%
\pgfpathlineto{\pgfqpoint{0.052190in}{0.773588in}}%
\pgfpathlineto{\pgfqpoint{0.000028in}{0.773588in}}%
\pgfpathlineto{\pgfqpoint{-0.053884in}{0.773588in}}%
\pgfpathlineto{\pgfqpoint{-0.106747in}{0.773588in}}%
\pgfpathlineto{\pgfqpoint{-0.159327in}{0.773588in}}%
\pgfpathlineto{\pgfqpoint{-0.212446in}{0.773588in}}%
\pgfpathlineto{\pgfqpoint{-0.265117in}{0.773588in}}%
\pgfpathlineto{\pgfqpoint{-0.317282in}{0.773588in}}%
\pgfpathlineto{\pgfqpoint{-0.370494in}{0.773588in}}%
\pgfpathlineto{\pgfqpoint{-0.421082in}{0.773588in}}%
\pgfpathlineto{\pgfqpoint{-0.472607in}{0.773588in}}%
\pgfpathlineto{\pgfqpoint{-0.525798in}{0.773588in}}%
\pgfpathlineto{\pgfqpoint{-0.577558in}{0.773588in}}%
\pgfpathlineto{\pgfqpoint{-0.628072in}{0.773588in}}%
\pgfpathlineto{\pgfqpoint{-0.679845in}{0.773588in}}%
\pgfpathlineto{\pgfqpoint{-0.729960in}{0.773588in}}%
\pgfpathlineto{\pgfqpoint{-0.780794in}{0.773588in}}%
\pgfpathlineto{\pgfqpoint{-0.832397in}{0.773588in}}%
\pgfpathlineto{\pgfqpoint{-0.883439in}{0.773588in}}%
\pgfpathlineto{\pgfqpoint{-0.934193in}{0.773588in}}%
\pgfpathlineto{\pgfqpoint{-0.987007in}{0.773588in}}%
\pgfpathlineto{\pgfqpoint{-1.038293in}{0.773588in}}%
\pgfpathlineto{\pgfqpoint{-1.088683in}{0.773588in}}%
\pgfpathlineto{\pgfqpoint{-1.140825in}{0.773588in}}%
\pgfpathlineto{\pgfqpoint{-1.191111in}{0.773588in}}%
\pgfpathlineto{\pgfqpoint{-1.242201in}{0.773588in}}%
\pgfpathlineto{\pgfqpoint{-1.295495in}{0.773588in}}%
\pgfpathlineto{\pgfqpoint{-1.346574in}{0.773588in}}%
\pgfpathlineto{\pgfqpoint{-1.397583in}{0.773588in}}%
\pgfpathlineto{\pgfqpoint{-1.451018in}{0.773588in}}%
\pgfpathlineto{\pgfqpoint{-1.502359in}{0.773588in}}%
\pgfpathlineto{\pgfqpoint{-1.553079in}{0.773588in}}%
\pgfpathlineto{\pgfqpoint{-1.606814in}{0.773588in}}%
\pgfpathlineto{\pgfqpoint{-1.657574in}{0.773588in}}%
\pgfpathlineto{\pgfqpoint{-1.707851in}{0.773588in}}%
\pgfpathlineto{\pgfqpoint{-1.759793in}{0.773588in}}%
\pgfpathlineto{\pgfqpoint{-1.809967in}{0.773588in}}%
\pgfpathlineto{\pgfqpoint{-1.860296in}{0.773588in}}%
\pgfpathlineto{\pgfqpoint{-1.912972in}{0.773588in}}%
\pgfpathlineto{\pgfqpoint{-1.963080in}{0.773588in}}%
\pgfpathlineto{\pgfqpoint{-2.013282in}{0.773588in}}%
\pgfpathlineto{\pgfqpoint{-2.064986in}{0.773588in}}%
\pgfpathlineto{\pgfqpoint{-2.116467in}{0.773588in}}%
\pgfpathlineto{\pgfqpoint{-2.167559in}{0.773588in}}%
\pgfpathlineto{\pgfqpoint{-2.220871in}{0.773588in}}%
\pgfpathlineto{\pgfqpoint{-2.272195in}{0.773588in}}%
\pgfpathlineto{\pgfqpoint{-2.323084in}{0.773588in}}%
\pgfpathlineto{\pgfqpoint{-2.375607in}{0.773588in}}%
\pgfpathlineto{\pgfqpoint{-2.425397in}{0.773588in}}%
\pgfpathlineto{\pgfqpoint{-2.476616in}{0.773588in}}%
\pgfpathlineto{\pgfqpoint{-2.528440in}{0.773588in}}%
\pgfpathlineto{\pgfqpoint{-2.578972in}{0.773588in}}%
\pgfpathlineto{\pgfqpoint{-2.629697in}{0.773588in}}%
\pgfpathlineto{\pgfqpoint{-2.681969in}{0.773588in}}%
\pgfpathlineto{\pgfqpoint{-2.732985in}{0.773588in}}%
\pgfpathlineto{\pgfqpoint{-2.784277in}{0.773588in}}%
\pgfpathlineto{\pgfqpoint{-2.836699in}{0.773588in}}%
\pgfpathlineto{\pgfqpoint{-2.887081in}{0.773588in}}%
\pgfpathlineto{\pgfqpoint{-2.938490in}{0.773588in}}%
\pgfpathlineto{\pgfqpoint{-2.990316in}{0.773588in}}%
\pgfpathlineto{\pgfqpoint{-3.039557in}{0.773588in}}%
\pgfpathlineto{\pgfqpoint{-3.088627in}{0.773588in}}%
\pgfpathlineto{\pgfqpoint{-3.140428in}{0.773588in}}%
\pgfpathlineto{\pgfqpoint{-3.190882in}{0.773588in}}%
\pgfpathlineto{\pgfqpoint{-3.241308in}{0.773588in}}%
\pgfpathlineto{\pgfqpoint{-3.292250in}{0.773588in}}%
\pgfpathlineto{\pgfqpoint{-3.341930in}{0.773588in}}%
\pgfpathlineto{\pgfqpoint{-3.392015in}{0.773588in}}%
\pgfpathlineto{\pgfqpoint{-3.444140in}{0.773588in}}%
\pgfpathlineto{\pgfqpoint{-3.494154in}{0.773588in}}%
\pgfpathlineto{\pgfqpoint{-3.544176in}{0.773588in}}%
\pgfpathlineto{\pgfqpoint{-3.595859in}{0.773588in}}%
\pgfpathlineto{\pgfqpoint{-3.646072in}{0.773588in}}%
\pgfpathlineto{\pgfqpoint{-3.696993in}{0.773588in}}%
\pgfpathlineto{\pgfqpoint{-3.748633in}{0.773588in}}%
\pgfpathlineto{\pgfqpoint{-3.799759in}{0.773588in}}%
\pgfpathlineto{\pgfqpoint{-3.850587in}{0.773588in}}%
\pgfpathlineto{\pgfqpoint{-3.902764in}{0.773588in}}%
\pgfpathlineto{\pgfqpoint{-3.952191in}{0.773588in}}%
\pgfpathlineto{\pgfqpoint{-4.002524in}{0.773588in}}%
\pgfpathlineto{\pgfqpoint{-4.053537in}{0.773588in}}%
\pgfpathlineto{\pgfqpoint{-4.103923in}{0.773588in}}%
\pgfpathlineto{\pgfqpoint{-4.153202in}{0.773588in}}%
\pgfpathlineto{\pgfqpoint{-4.204591in}{0.773588in}}%
\pgfpathlineto{\pgfqpoint{-4.255103in}{0.773588in}}%
\pgfpathlineto{\pgfqpoint{-4.305607in}{0.773588in}}%
\pgfpathlineto{\pgfqpoint{-4.357515in}{0.773588in}}%
\pgfpathlineto{\pgfqpoint{-4.407934in}{0.773588in}}%
\pgfpathlineto{\pgfqpoint{-4.457276in}{0.773588in}}%
\pgfpathlineto{\pgfqpoint{-4.508002in}{0.773588in}}%
\pgfpathlineto{\pgfqpoint{-4.558006in}{0.773588in}}%
\pgfpathlineto{\pgfqpoint{-4.607841in}{0.773588in}}%
\pgfpathlineto{\pgfqpoint{-4.659647in}{0.773588in}}%
\pgfpathlineto{\pgfqpoint{-4.709756in}{0.773588in}}%
\pgfpathlineto{\pgfqpoint{-4.759589in}{0.773588in}}%
\pgfpathlineto{\pgfqpoint{-4.812147in}{0.773588in}}%
\pgfpathlineto{\pgfqpoint{-4.861887in}{0.773588in}}%
\pgfpathlineto{\pgfqpoint{-4.911765in}{0.773588in}}%
\pgfpathlineto{\pgfqpoint{-4.963670in}{0.773588in}}%
\pgfpathlineto{\pgfqpoint{-5.014255in}{0.773588in}}%
\pgfpathlineto{\pgfqpoint{-5.064238in}{0.773588in}}%
\pgfpathlineto{\pgfqpoint{-5.114869in}{0.773588in}}%
\pgfpathlineto{\pgfqpoint{-5.165135in}{0.773588in}}%
\pgfpathlineto{\pgfqpoint{-5.215272in}{0.773588in}}%
\pgfpathlineto{\pgfqpoint{-5.267508in}{0.773588in}}%
\pgfpathlineto{\pgfqpoint{-5.319212in}{0.773588in}}%
\pgfpathlineto{\pgfqpoint{-5.369703in}{0.773588in}}%
\pgfpathlineto{\pgfqpoint{-5.421631in}{0.773588in}}%
\pgfpathlineto{\pgfqpoint{-5.473058in}{0.773588in}}%
\pgfpathlineto{\pgfqpoint{-5.523472in}{0.773588in}}%
\pgfpathlineto{\pgfqpoint{-5.574633in}{0.773588in}}%
\pgfpathlineto{\pgfqpoint{-5.623816in}{0.773588in}}%
\pgfpathlineto{\pgfqpoint{-5.673071in}{0.773588in}}%
\pgfpathlineto{\pgfqpoint{-5.724570in}{0.773588in}}%
\pgfpathlineto{\pgfqpoint{-5.774223in}{0.773588in}}%
\pgfpathlineto{\pgfqpoint{-5.822716in}{0.773588in}}%
\pgfpathlineto{\pgfqpoint{-5.874278in}{0.773588in}}%
\pgfpathlineto{\pgfqpoint{-5.924128in}{0.773588in}}%
\pgfpathlineto{\pgfqpoint{-5.974236in}{0.773588in}}%
\pgfpathlineto{\pgfqpoint{-6.024836in}{0.773588in}}%
\pgfpathlineto{\pgfqpoint{-6.075227in}{0.773588in}}%
\pgfpathlineto{\pgfqpoint{-6.125420in}{0.773588in}}%
\pgfpathlineto{\pgfqpoint{-6.177451in}{0.773588in}}%
\pgfpathlineto{\pgfqpoint{-6.228389in}{0.773588in}}%
\pgfpathlineto{\pgfqpoint{-6.279011in}{0.773588in}}%
\pgfpathlineto{\pgfqpoint{-6.330959in}{0.773588in}}%
\pgfpathlineto{\pgfqpoint{-6.380124in}{0.773588in}}%
\pgfpathlineto{\pgfqpoint{-6.429787in}{0.773588in}}%
\pgfpathlineto{\pgfqpoint{-6.481910in}{0.773588in}}%
\pgfpathlineto{\pgfqpoint{-6.532179in}{0.773588in}}%
\pgfpathlineto{\pgfqpoint{-6.582317in}{0.773588in}}%
\pgfpathlineto{\pgfqpoint{-6.634445in}{0.773588in}}%
\pgfpathlineto{\pgfqpoint{-6.682747in}{0.773588in}}%
\pgfpathlineto{\pgfqpoint{-6.731323in}{0.773588in}}%
\pgfpathlineto{\pgfqpoint{-6.782488in}{0.773588in}}%
\pgfpathlineto{\pgfqpoint{-6.832072in}{0.773588in}}%
\pgfpathlineto{\pgfqpoint{-6.882593in}{0.773588in}}%
\pgfpathlineto{\pgfqpoint{-6.934169in}{0.773588in}}%
\pgfpathlineto{\pgfqpoint{-6.984061in}{0.773588in}}%
\pgfpathlineto{\pgfqpoint{-7.033318in}{0.773588in}}%
\pgfpathlineto{\pgfqpoint{-7.085212in}{0.773588in}}%
\pgfpathlineto{\pgfqpoint{-7.135511in}{0.773588in}}%
\pgfpathlineto{\pgfqpoint{-7.185558in}{0.773588in}}%
\pgfpathlineto{\pgfqpoint{-7.237328in}{0.773588in}}%
\pgfpathlineto{\pgfqpoint{-7.286238in}{0.773588in}}%
\pgfpathlineto{\pgfqpoint{-7.335791in}{0.773588in}}%
\pgfpathlineto{\pgfqpoint{-7.386343in}{0.773588in}}%
\pgfpathlineto{\pgfqpoint{-7.435602in}{0.773588in}}%
\pgfpathlineto{\pgfqpoint{-7.485312in}{0.773588in}}%
\pgfpathlineto{\pgfqpoint{-7.536566in}{0.773588in}}%
\pgfpathlineto{\pgfqpoint{-7.585672in}{0.773588in}}%
\pgfpathlineto{\pgfqpoint{-7.634521in}{0.773588in}}%
\pgfpathlineto{\pgfqpoint{-7.685281in}{0.773588in}}%
\pgfpathlineto{\pgfqpoint{-7.733870in}{0.773588in}}%
\pgfpathlineto{\pgfqpoint{-7.783607in}{0.773588in}}%
\pgfpathlineto{\pgfqpoint{-7.835188in}{0.773588in}}%
\pgfpathlineto{\pgfqpoint{-7.885122in}{0.773588in}}%
\pgfpathlineto{\pgfqpoint{-7.935315in}{0.773588in}}%
\pgfpathlineto{\pgfqpoint{-7.986412in}{0.773588in}}%
\pgfpathlineto{\pgfqpoint{-8.035792in}{0.773588in}}%
\pgfpathlineto{\pgfqpoint{-8.085190in}{0.773588in}}%
\pgfpathlineto{\pgfqpoint{-8.135490in}{0.773588in}}%
\pgfpathlineto{\pgfqpoint{-8.184462in}{0.773588in}}%
\pgfpathlineto{\pgfqpoint{-8.233427in}{0.773588in}}%
\pgfpathlineto{\pgfqpoint{-8.283775in}{0.773588in}}%
\pgfpathlineto{\pgfqpoint{-8.332630in}{0.773588in}}%
\pgfpathlineto{\pgfqpoint{-8.382452in}{0.773588in}}%
\pgfpathlineto{\pgfqpoint{-8.432560in}{0.773588in}}%
\pgfpathlineto{\pgfqpoint{-8.481981in}{0.773588in}}%
\pgfpathlineto{\pgfqpoint{-8.532027in}{0.773588in}}%
\pgfpathlineto{\pgfqpoint{-8.583181in}{0.773588in}}%
\pgfpathlineto{\pgfqpoint{-8.632588in}{0.773588in}}%
\pgfpathlineto{\pgfqpoint{-8.681529in}{0.773588in}}%
\pgfpathlineto{\pgfqpoint{-8.731477in}{0.773588in}}%
\pgfpathlineto{\pgfqpoint{-8.780578in}{0.773588in}}%
\pgfpathlineto{\pgfqpoint{-8.829855in}{0.773588in}}%
\pgfpathlineto{\pgfqpoint{-8.880937in}{0.773588in}}%
\pgfpathlineto{\pgfqpoint{-8.930541in}{0.773588in}}%
\pgfpathlineto{\pgfqpoint{-8.979731in}{0.773588in}}%
\pgfpathlineto{\pgfqpoint{-9.030683in}{0.773588in}}%
\pgfpathlineto{\pgfqpoint{-9.080221in}{0.773588in}}%
\pgfpathlineto{\pgfqpoint{-9.130552in}{0.773588in}}%
\pgfpathlineto{\pgfqpoint{-9.182531in}{0.773588in}}%
\pgfpathlineto{\pgfqpoint{-9.232466in}{0.773588in}}%
\pgfpathlineto{\pgfqpoint{-9.281795in}{0.773588in}}%
\pgfpathlineto{\pgfqpoint{-9.332833in}{0.773588in}}%
\pgfpathlineto{\pgfqpoint{-9.382453in}{0.773588in}}%
\pgfpathlineto{\pgfqpoint{-9.430893in}{0.773588in}}%
\pgfpathlineto{\pgfqpoint{-9.479530in}{0.773588in}}%
\pgfpathlineto{\pgfqpoint{-9.527213in}{0.773588in}}%
\pgfpathlineto{\pgfqpoint{-9.575433in}{0.773588in}}%
\pgfpathlineto{\pgfqpoint{-9.624467in}{0.773588in}}%
\pgfpathlineto{\pgfqpoint{-9.671880in}{0.773588in}}%
\pgfpathlineto{\pgfqpoint{-9.719205in}{0.773588in}}%
\pgfpathlineto{\pgfqpoint{-9.768438in}{0.773588in}}%
\pgfpathlineto{\pgfqpoint{-9.816398in}{0.773588in}}%
\pgfpathlineto{\pgfqpoint{-9.863853in}{0.773588in}}%
\pgfpathlineto{\pgfqpoint{-9.912991in}{0.773588in}}%
\pgfpathlineto{\pgfqpoint{-9.960967in}{0.773588in}}%
\pgfpathlineto{\pgfqpoint{-10.009255in}{0.773588in}}%
\pgfpathlineto{\pgfqpoint{-10.058316in}{0.773588in}}%
\pgfpathlineto{\pgfqpoint{-10.106404in}{0.773588in}}%
\pgfpathlineto{\pgfqpoint{-10.154319in}{0.773588in}}%
\pgfpathlineto{\pgfqpoint{-10.204843in}{0.773588in}}%
\pgfpathlineto{\pgfqpoint{-10.253421in}{0.773588in}}%
\pgfpathlineto{\pgfqpoint{-10.301938in}{0.773588in}}%
\pgfpathlineto{\pgfqpoint{-10.352351in}{0.773588in}}%
\pgfpathlineto{\pgfqpoint{-10.401144in}{0.773588in}}%
\pgfpathlineto{\pgfqpoint{-10.448474in}{0.773588in}}%
\pgfpathlineto{\pgfqpoint{-10.497957in}{0.773588in}}%
\pgfpathlineto{\pgfqpoint{-10.546833in}{0.773588in}}%
\pgfpathlineto{\pgfqpoint{-10.594934in}{0.773588in}}%
\pgfpathlineto{\pgfqpoint{-10.644481in}{0.773588in}}%
\pgfpathlineto{\pgfqpoint{-10.693023in}{0.773588in}}%
\pgfpathlineto{\pgfqpoint{-10.741507in}{0.773588in}}%
\pgfpathlineto{\pgfqpoint{-10.790422in}{0.773588in}}%
\pgfpathlineto{\pgfqpoint{-10.837686in}{0.773588in}}%
\pgfpathlineto{\pgfqpoint{-10.885230in}{0.773588in}}%
\pgfpathlineto{\pgfqpoint{-10.933861in}{0.773588in}}%
\pgfpathlineto{\pgfqpoint{-10.980903in}{0.773588in}}%
\pgfpathlineto{\pgfqpoint{-11.028607in}{0.773588in}}%
\pgfpathlineto{\pgfqpoint{-11.077704in}{0.773588in}}%
\pgfpathlineto{\pgfqpoint{-11.125274in}{0.773588in}}%
\pgfpathlineto{\pgfqpoint{-11.172812in}{0.773588in}}%
\pgfpathlineto{\pgfqpoint{-11.221741in}{0.773588in}}%
\pgfpathlineto{\pgfqpoint{-11.268838in}{0.773588in}}%
\pgfpathlineto{\pgfqpoint{-11.316622in}{0.773588in}}%
\pgfpathlineto{\pgfqpoint{-11.366120in}{0.773588in}}%
\pgfpathlineto{\pgfqpoint{-11.415235in}{0.773588in}}%
\pgfpathlineto{\pgfqpoint{-11.463260in}{0.773588in}}%
\pgfpathlineto{\pgfqpoint{-11.511546in}{0.773588in}}%
\pgfpathlineto{\pgfqpoint{-11.559336in}{0.773588in}}%
\pgfpathlineto{\pgfqpoint{-11.607131in}{0.773588in}}%
\pgfpathlineto{\pgfqpoint{-11.656418in}{0.773588in}}%
\pgfpathlineto{\pgfqpoint{-11.704231in}{0.773588in}}%
\pgfpathlineto{\pgfqpoint{-11.751492in}{0.773588in}}%
\pgfpathlineto{\pgfqpoint{-11.800403in}{0.773588in}}%
\pgfpathlineto{\pgfqpoint{-11.847736in}{0.773588in}}%
\pgfpathlineto{\pgfqpoint{-11.895021in}{0.773588in}}%
\pgfpathlineto{\pgfqpoint{-11.943783in}{0.773588in}}%
\pgfpathlineto{\pgfqpoint{-11.991431in}{0.773588in}}%
\pgfpathlineto{\pgfqpoint{-12.038569in}{0.773588in}}%
\pgfpathlineto{\pgfqpoint{-12.087205in}{0.773588in}}%
\pgfpathlineto{\pgfqpoint{-12.135286in}{0.773588in}}%
\pgfpathlineto{\pgfqpoint{-12.182471in}{0.773588in}}%
\pgfpathlineto{\pgfqpoint{-12.231638in}{0.773588in}}%
\pgfpathlineto{\pgfqpoint{-12.278675in}{0.773588in}}%
\pgfpathlineto{\pgfqpoint{-12.326627in}{0.773588in}}%
\pgfpathlineto{\pgfqpoint{-12.375560in}{0.773588in}}%
\pgfpathlineto{\pgfqpoint{-12.422612in}{0.773588in}}%
\pgfpathlineto{\pgfqpoint{-12.469843in}{0.773588in}}%
\pgfpathlineto{\pgfqpoint{-12.517880in}{0.773588in}}%
\pgfpathlineto{\pgfqpoint{-12.565107in}{0.773588in}}%
\pgfpathlineto{\pgfqpoint{-12.612433in}{0.773588in}}%
\pgfpathlineto{\pgfqpoint{-12.661578in}{0.773588in}}%
\pgfpathlineto{\pgfqpoint{-12.708579in}{0.773588in}}%
\pgfpathlineto{\pgfqpoint{-12.755866in}{0.773588in}}%
\pgfpathlineto{\pgfqpoint{-12.805010in}{0.773588in}}%
\pgfpathlineto{\pgfqpoint{-12.852526in}{0.773588in}}%
\pgfpathlineto{\pgfqpoint{-12.899585in}{0.773588in}}%
\pgfpathlineto{\pgfqpoint{-12.949523in}{0.773588in}}%
\pgfpathlineto{\pgfqpoint{-12.998480in}{0.773588in}}%
\pgfpathlineto{\pgfqpoint{-13.046608in}{0.773588in}}%
\pgfpathlineto{\pgfqpoint{-13.096328in}{0.773588in}}%
\pgfpathlineto{\pgfqpoint{-13.144068in}{0.773588in}}%
\pgfpathlineto{\pgfqpoint{-13.190970in}{0.773588in}}%
\pgfpathlineto{\pgfqpoint{-13.239807in}{0.773588in}}%
\pgfpathlineto{\pgfqpoint{-13.287448in}{0.773588in}}%
\pgfpathlineto{\pgfqpoint{-13.334326in}{0.773588in}}%
\pgfpathlineto{\pgfqpoint{-13.382220in}{0.773588in}}%
\pgfpathlineto{\pgfqpoint{-13.428346in}{0.773588in}}%
\pgfpathlineto{\pgfqpoint{-13.474335in}{0.773588in}}%
\pgfpathlineto{\pgfqpoint{-13.522553in}{0.773588in}}%
\pgfpathlineto{\pgfqpoint{-13.568873in}{0.773588in}}%
\pgfpathlineto{\pgfqpoint{-13.615991in}{0.773588in}}%
\pgfpathlineto{\pgfqpoint{-13.663331in}{0.773588in}}%
\pgfpathlineto{\pgfqpoint{-13.710126in}{0.773588in}}%
\pgfpathlineto{\pgfqpoint{-13.757485in}{0.773588in}}%
\pgfpathlineto{\pgfqpoint{-13.806330in}{0.773588in}}%
\pgfpathlineto{\pgfqpoint{-13.853816in}{0.773588in}}%
\pgfpathlineto{\pgfqpoint{-13.901287in}{0.773588in}}%
\pgfpathlineto{\pgfqpoint{-13.950094in}{0.773588in}}%
\pgfpathlineto{\pgfqpoint{-13.996250in}{0.773588in}}%
\pgfpathlineto{\pgfqpoint{-14.042263in}{0.773588in}}%
\pgfpathlineto{\pgfqpoint{-14.090279in}{0.773588in}}%
\pgfpathlineto{\pgfqpoint{-14.137283in}{0.773588in}}%
\pgfpathlineto{\pgfqpoint{-14.184689in}{0.773588in}}%
\pgfpathlineto{\pgfqpoint{-14.233487in}{0.773588in}}%
\pgfpathlineto{\pgfqpoint{-14.280667in}{0.773588in}}%
\pgfpathlineto{\pgfqpoint{-14.327371in}{0.773588in}}%
\pgfpathlineto{\pgfqpoint{-14.375931in}{0.773588in}}%
\pgfpathlineto{\pgfqpoint{-14.422978in}{0.773588in}}%
\pgfpathlineto{\pgfqpoint{-14.469907in}{0.773588in}}%
\pgfpathlineto{\pgfqpoint{-14.517904in}{0.773588in}}%
\pgfpathlineto{\pgfqpoint{-14.564671in}{0.773588in}}%
\pgfpathlineto{\pgfqpoint{-14.611394in}{0.773588in}}%
\pgfpathlineto{\pgfqpoint{-14.659388in}{0.773588in}}%
\pgfpathlineto{\pgfqpoint{-14.705919in}{0.773588in}}%
\pgfpathlineto{\pgfqpoint{-14.752153in}{0.773588in}}%
\pgfpathlineto{\pgfqpoint{-14.800669in}{0.773588in}}%
\pgfpathlineto{\pgfqpoint{-14.847516in}{0.773588in}}%
\pgfpathlineto{\pgfqpoint{-14.894419in}{0.773588in}}%
\pgfpathlineto{\pgfqpoint{-14.942416in}{0.773588in}}%
\pgfpathlineto{\pgfqpoint{-14.989019in}{0.773588in}}%
\pgfpathlineto{\pgfqpoint{-15.035921in}{0.773588in}}%
\pgfpathlineto{\pgfqpoint{-15.083728in}{0.773588in}}%
\pgfpathlineto{\pgfqpoint{-15.130553in}{0.773588in}}%
\pgfpathlineto{\pgfqpoint{-15.176769in}{0.773588in}}%
\pgfpathlineto{\pgfqpoint{-15.225998in}{0.773588in}}%
\pgfpathlineto{\pgfqpoint{-15.273998in}{0.773588in}}%
\pgfpathlineto{\pgfqpoint{-15.321194in}{0.773588in}}%
\pgfpathlineto{\pgfqpoint{-15.369802in}{0.773588in}}%
\pgfpathlineto{\pgfqpoint{-15.416792in}{0.773588in}}%
\pgfpathlineto{\pgfqpoint{-15.463539in}{0.773588in}}%
\pgfpathlineto{\pgfqpoint{-15.512535in}{0.773588in}}%
\pgfpathlineto{\pgfqpoint{-15.560849in}{0.773588in}}%
\pgfpathlineto{\pgfqpoint{-15.608704in}{0.773588in}}%
\pgfpathlineto{\pgfqpoint{-15.657002in}{0.773588in}}%
\pgfpathlineto{\pgfqpoint{-15.703994in}{0.773588in}}%
\pgfpathlineto{\pgfqpoint{-15.750800in}{0.773588in}}%
\pgfpathlineto{\pgfqpoint{-15.798897in}{0.773588in}}%
\pgfpathlineto{\pgfqpoint{-15.844500in}{0.773588in}}%
\pgfpathlineto{\pgfqpoint{-15.890419in}{0.773588in}}%
\pgfpathlineto{\pgfqpoint{-15.938076in}{0.773588in}}%
\pgfpathlineto{\pgfqpoint{-15.984286in}{0.773588in}}%
\pgfpathlineto{\pgfqpoint{-16.029628in}{0.773588in}}%
\pgfpathlineto{\pgfqpoint{-16.077171in}{0.773588in}}%
\pgfpathlineto{\pgfqpoint{-16.123165in}{0.773588in}}%
\pgfpathlineto{\pgfqpoint{-16.169427in}{0.773588in}}%
\pgfpathlineto{\pgfqpoint{-16.217070in}{0.773588in}}%
\pgfpathlineto{\pgfqpoint{-16.262806in}{0.773588in}}%
\pgfpathlineto{\pgfqpoint{-16.308871in}{0.773588in}}%
\pgfpathlineto{\pgfqpoint{-16.355646in}{0.773588in}}%
\pgfpathlineto{\pgfqpoint{-16.402550in}{0.773588in}}%
\pgfpathlineto{\pgfqpoint{-16.449347in}{0.773588in}}%
\pgfpathlineto{\pgfqpoint{-16.496157in}{0.773588in}}%
\pgfpathlineto{\pgfqpoint{-16.542990in}{0.773588in}}%
\pgfpathlineto{\pgfqpoint{-16.589239in}{0.773588in}}%
\pgfpathlineto{\pgfqpoint{-16.637120in}{0.773588in}}%
\pgfpathlineto{\pgfqpoint{-16.684310in}{0.773588in}}%
\pgfpathlineto{\pgfqpoint{-16.730721in}{0.773588in}}%
\pgfpathlineto{\pgfqpoint{-16.778530in}{0.773588in}}%
\pgfpathlineto{\pgfqpoint{-16.825434in}{0.773588in}}%
\pgfpathlineto{\pgfqpoint{-16.871664in}{0.773588in}}%
\pgfpathlineto{\pgfqpoint{-16.918692in}{0.773588in}}%
\pgfpathlineto{\pgfqpoint{-16.964774in}{0.773588in}}%
\pgfpathlineto{\pgfqpoint{-17.010884in}{0.773588in}}%
\pgfpathlineto{\pgfqpoint{-17.058009in}{0.773588in}}%
\pgfpathlineto{\pgfqpoint{-17.103874in}{0.773588in}}%
\pgfpathlineto{\pgfqpoint{-17.150161in}{0.773588in}}%
\pgfpathlineto{\pgfqpoint{-17.198221in}{0.773588in}}%
\pgfpathlineto{\pgfqpoint{-17.244258in}{0.773588in}}%
\pgfpathlineto{\pgfqpoint{-17.289523in}{0.773588in}}%
\pgfpathlineto{\pgfqpoint{-17.336276in}{0.773588in}}%
\pgfpathlineto{\pgfqpoint{-17.381732in}{0.773588in}}%
\pgfpathlineto{\pgfqpoint{-17.427002in}{0.773588in}}%
\pgfpathlineto{\pgfqpoint{-17.474644in}{0.773588in}}%
\pgfpathlineto{\pgfqpoint{-17.521284in}{0.773588in}}%
\pgfpathlineto{\pgfqpoint{-17.566739in}{0.773588in}}%
\pgfpathlineto{\pgfqpoint{-17.613251in}{0.773588in}}%
\pgfpathlineto{\pgfqpoint{-17.658538in}{0.773588in}}%
\pgfpathlineto{\pgfqpoint{-17.704093in}{0.773588in}}%
\pgfpathlineto{\pgfqpoint{-17.751556in}{0.773588in}}%
\pgfpathlineto{\pgfqpoint{-17.797346in}{0.773588in}}%
\pgfpathlineto{\pgfqpoint{-17.843480in}{0.773588in}}%
\pgfpathlineto{\pgfqpoint{-17.891271in}{0.773588in}}%
\pgfpathlineto{\pgfqpoint{-17.937614in}{0.773588in}}%
\pgfpathlineto{\pgfqpoint{-17.984102in}{0.773588in}}%
\pgfpathlineto{\pgfqpoint{-18.032389in}{0.773588in}}%
\pgfpathlineto{\pgfqpoint{-18.078784in}{0.773588in}}%
\pgfpathlineto{\pgfqpoint{-18.124795in}{0.773588in}}%
\pgfpathlineto{\pgfqpoint{-18.172791in}{0.773588in}}%
\pgfpathlineto{\pgfqpoint{-18.218674in}{0.773588in}}%
\pgfpathlineto{\pgfqpoint{-18.264167in}{0.773588in}}%
\pgfpathlineto{\pgfqpoint{-18.311512in}{0.773588in}}%
\pgfpathlineto{\pgfqpoint{-18.356748in}{0.773588in}}%
\pgfpathlineto{\pgfqpoint{-18.402000in}{0.773588in}}%
\pgfpathlineto{\pgfqpoint{-18.448764in}{0.773588in}}%
\pgfpathlineto{\pgfqpoint{-18.494344in}{0.773588in}}%
\pgfpathlineto{\pgfqpoint{-18.540355in}{0.773588in}}%
\pgfpathlineto{\pgfqpoint{-18.587378in}{0.773588in}}%
\pgfpathlineto{\pgfqpoint{-18.632978in}{0.773588in}}%
\pgfpathlineto{\pgfqpoint{-18.678684in}{0.773588in}}%
\pgfpathlineto{\pgfqpoint{-18.726033in}{0.773588in}}%
\pgfpathlineto{\pgfqpoint{-18.772260in}{0.773588in}}%
\pgfpathlineto{\pgfqpoint{-18.817991in}{0.773588in}}%
\pgfpathlineto{\pgfqpoint{-18.865040in}{0.773588in}}%
\pgfpathlineto{\pgfqpoint{-18.910503in}{0.773588in}}%
\pgfpathlineto{\pgfqpoint{-18.956783in}{0.773588in}}%
\pgfpathlineto{\pgfqpoint{-19.004576in}{0.773588in}}%
\pgfpathlineto{\pgfqpoint{-19.050812in}{0.773588in}}%
\pgfpathlineto{\pgfqpoint{-19.097062in}{0.773588in}}%
\pgfpathlineto{\pgfqpoint{-19.143544in}{0.773588in}}%
\pgfpathlineto{\pgfqpoint{-19.188899in}{0.773588in}}%
\pgfpathlineto{\pgfqpoint{-19.234797in}{0.773588in}}%
\pgfpathlineto{\pgfqpoint{-19.283047in}{0.773588in}}%
\pgfpathlineto{\pgfqpoint{-19.328532in}{0.773588in}}%
\pgfpathlineto{\pgfqpoint{-19.374230in}{0.773588in}}%
\pgfpathlineto{\pgfqpoint{-19.421584in}{0.773588in}}%
\pgfpathlineto{\pgfqpoint{-19.467303in}{0.773588in}}%
\pgfpathlineto{\pgfqpoint{-19.512973in}{0.773588in}}%
\pgfpathlineto{\pgfqpoint{-19.559546in}{0.773588in}}%
\pgfpathlineto{\pgfqpoint{-19.604824in}{0.773588in}}%
\pgfpathlineto{\pgfqpoint{-19.650361in}{0.773588in}}%
\pgfpathlineto{\pgfqpoint{-19.696649in}{0.773588in}}%
\pgfpathlineto{\pgfqpoint{-19.742024in}{0.773588in}}%
\pgfpathlineto{\pgfqpoint{-19.787808in}{0.773588in}}%
\pgfpathlineto{\pgfqpoint{-19.833948in}{0.773588in}}%
\pgfpathlineto{\pgfqpoint{-19.878562in}{0.773588in}}%
\pgfpathlineto{\pgfqpoint{-19.924139in}{0.773588in}}%
\pgfpathlineto{\pgfqpoint{-19.970967in}{0.773588in}}%
\pgfpathlineto{\pgfqpoint{-20.016506in}{0.773588in}}%
\pgfpathlineto{\pgfqpoint{-20.061722in}{0.773588in}}%
\pgfpathlineto{\pgfqpoint{-20.107707in}{0.773588in}}%
\pgfpathlineto{\pgfqpoint{-20.153038in}{0.773588in}}%
\pgfpathlineto{\pgfqpoint{-20.197814in}{0.773588in}}%
\pgfpathlineto{\pgfqpoint{-20.244875in}{0.773588in}}%
\pgfpathlineto{\pgfqpoint{-20.290419in}{0.773588in}}%
\pgfpathlineto{\pgfqpoint{-20.335981in}{0.773588in}}%
\pgfpathlineto{\pgfqpoint{-20.382199in}{0.773588in}}%
\pgfpathlineto{\pgfqpoint{-20.428221in}{0.773588in}}%
\pgfpathlineto{\pgfqpoint{-20.474014in}{0.773588in}}%
\pgfpathlineto{\pgfqpoint{-20.519904in}{0.773588in}}%
\pgfpathlineto{\pgfqpoint{-20.565077in}{0.773588in}}%
\pgfpathlineto{\pgfqpoint{-20.611243in}{0.773588in}}%
\pgfpathlineto{\pgfqpoint{-20.658742in}{0.773588in}}%
\pgfpathlineto{\pgfqpoint{-20.705066in}{0.773588in}}%
\pgfpathlineto{\pgfqpoint{-20.750831in}{0.773588in}}%
\pgfpathlineto{\pgfqpoint{-20.797022in}{0.773588in}}%
\pgfpathlineto{\pgfqpoint{-20.842169in}{0.773588in}}%
\pgfpathlineto{\pgfqpoint{-20.886847in}{0.773588in}}%
\pgfpathlineto{\pgfqpoint{-20.932365in}{0.773588in}}%
\pgfpathlineto{\pgfqpoint{-20.976378in}{0.773588in}}%
\pgfpathlineto{\pgfqpoint{-21.021509in}{0.773588in}}%
\pgfpathlineto{\pgfqpoint{-21.067357in}{0.773588in}}%
\pgfpathlineto{\pgfqpoint{-21.113139in}{0.773588in}}%
\pgfpathlineto{\pgfqpoint{-21.158215in}{0.773588in}}%
\pgfpathlineto{\pgfqpoint{-21.204354in}{0.773588in}}%
\pgfpathlineto{\pgfqpoint{-21.249304in}{0.773588in}}%
\pgfpathlineto{\pgfqpoint{-21.294264in}{0.773588in}}%
\pgfpathlineto{\pgfqpoint{-21.340867in}{0.773588in}}%
\pgfpathlineto{\pgfqpoint{-21.385572in}{0.773588in}}%
\pgfpathlineto{\pgfqpoint{-21.430267in}{0.773588in}}%
\pgfpathlineto{\pgfqpoint{-21.475985in}{0.773588in}}%
\pgfpathlineto{\pgfqpoint{-21.520883in}{0.773588in}}%
\pgfpathlineto{\pgfqpoint{-21.566198in}{0.773588in}}%
\pgfpathlineto{\pgfqpoint{-21.612919in}{0.773588in}}%
\pgfpathlineto{\pgfqpoint{-21.657342in}{0.773588in}}%
\pgfpathlineto{\pgfqpoint{-21.703167in}{0.773588in}}%
\pgfpathlineto{\pgfqpoint{-21.749554in}{0.773588in}}%
\pgfpathlineto{\pgfqpoint{-21.794255in}{0.773588in}}%
\pgfpathlineto{\pgfqpoint{-21.839254in}{0.773588in}}%
\pgfpathlineto{\pgfqpoint{-21.885462in}{0.773588in}}%
\pgfpathlineto{\pgfqpoint{-21.930597in}{0.773588in}}%
\pgfpathlineto{\pgfqpoint{-21.976168in}{0.773588in}}%
\pgfpathlineto{\pgfqpoint{-22.022397in}{0.773588in}}%
\pgfpathlineto{\pgfqpoint{-22.067661in}{0.773588in}}%
\pgfpathlineto{\pgfqpoint{-22.112231in}{0.773588in}}%
\pgfpathlineto{\pgfqpoint{-22.158025in}{0.773588in}}%
\pgfpathlineto{\pgfqpoint{-22.201868in}{0.773588in}}%
\pgfpathlineto{\pgfqpoint{-22.246011in}{0.773588in}}%
\pgfpathlineto{\pgfqpoint{-22.291940in}{0.773588in}}%
\pgfpathlineto{\pgfqpoint{-22.336074in}{0.773588in}}%
\pgfpathlineto{\pgfqpoint{-22.380996in}{0.773588in}}%
\pgfpathlineto{\pgfqpoint{-22.427383in}{0.773588in}}%
\pgfpathlineto{\pgfqpoint{-22.471952in}{0.773588in}}%
\pgfpathlineto{\pgfqpoint{-22.516601in}{0.773588in}}%
\pgfpathlineto{\pgfqpoint{-22.562133in}{0.773588in}}%
\pgfpathlineto{\pgfqpoint{-22.606850in}{0.773588in}}%
\pgfpathlineto{\pgfqpoint{-22.651258in}{0.773588in}}%
\pgfpathlineto{\pgfqpoint{-22.697642in}{0.773588in}}%
\pgfpathlineto{\pgfqpoint{-22.742185in}{0.773588in}}%
\pgfpathlineto{\pgfqpoint{-22.786844in}{0.773588in}}%
\pgfpathlineto{\pgfqpoint{-22.833029in}{0.773588in}}%
\pgfpathlineto{\pgfqpoint{-22.878494in}{0.773588in}}%
\pgfpathlineto{\pgfqpoint{-22.924822in}{0.773588in}}%
\pgfpathlineto{\pgfqpoint{-22.972746in}{0.773588in}}%
\pgfpathlineto{\pgfqpoint{-23.018637in}{0.773588in}}%
\pgfpathlineto{\pgfqpoint{-23.065076in}{0.773588in}}%
\pgfpathlineto{\pgfqpoint{-23.112561in}{0.773588in}}%
\pgfpathlineto{\pgfqpoint{-23.159034in}{0.773588in}}%
\pgfpathlineto{\pgfqpoint{-23.205343in}{0.773588in}}%
\pgfpathlineto{\pgfqpoint{-23.252910in}{0.773588in}}%
\pgfpathlineto{\pgfqpoint{-23.298917in}{0.773588in}}%
\pgfpathlineto{\pgfqpoint{-23.344516in}{0.773588in}}%
\pgfpathlineto{\pgfqpoint{-23.391460in}{0.773588in}}%
\pgfpathlineto{\pgfqpoint{-23.437154in}{0.773588in}}%
\pgfpathlineto{\pgfqpoint{-23.482780in}{0.773588in}}%
\pgfpathlineto{\pgfqpoint{-23.529965in}{0.773588in}}%
\pgfpathlineto{\pgfqpoint{-23.575877in}{0.773588in}}%
\pgfpathlineto{\pgfqpoint{-23.620333in}{0.773588in}}%
\pgfpathlineto{\pgfqpoint{-23.666576in}{0.773588in}}%
\pgfpathlineto{\pgfqpoint{-23.711883in}{0.773588in}}%
\pgfpathlineto{\pgfqpoint{-23.757289in}{0.773588in}}%
\pgfpathlineto{\pgfqpoint{-23.804343in}{0.773588in}}%
\pgfpathlineto{\pgfqpoint{-23.849943in}{0.773588in}}%
\pgfpathlineto{\pgfqpoint{-23.895017in}{0.773588in}}%
\pgfpathlineto{\pgfqpoint{-23.941619in}{0.773588in}}%
\pgfpathlineto{\pgfqpoint{-23.986811in}{0.773588in}}%
\pgfpathlineto{\pgfqpoint{-24.032531in}{0.773588in}}%
\pgfpathlineto{\pgfqpoint{-24.078902in}{0.773588in}}%
\pgfpathlineto{\pgfqpoint{-24.123999in}{0.773588in}}%
\pgfpathlineto{\pgfqpoint{-24.169854in}{0.773588in}}%
\pgfpathlineto{\pgfqpoint{-24.216953in}{0.773588in}}%
\pgfpathlineto{\pgfqpoint{-24.262109in}{0.773588in}}%
\pgfpathlineto{\pgfqpoint{-24.308071in}{0.773588in}}%
\pgfpathlineto{\pgfqpoint{-24.355467in}{0.773588in}}%
\pgfpathlineto{\pgfqpoint{-24.401173in}{0.773588in}}%
\pgfpathlineto{\pgfqpoint{-24.447114in}{0.773588in}}%
\pgfpathlineto{\pgfqpoint{-24.494847in}{0.773588in}}%
\pgfpathlineto{\pgfqpoint{-24.540716in}{0.773588in}}%
\pgfpathlineto{\pgfqpoint{-24.586259in}{0.773588in}}%
\pgfpathlineto{\pgfqpoint{-24.632696in}{0.773588in}}%
\pgfpathlineto{\pgfqpoint{-24.677917in}{0.773588in}}%
\pgfpathlineto{\pgfqpoint{-24.722523in}{0.773588in}}%
\pgfpathlineto{\pgfqpoint{-24.768227in}{0.773588in}}%
\pgfpathlineto{\pgfqpoint{-24.813437in}{0.773588in}}%
\pgfpathlineto{\pgfqpoint{-24.858721in}{0.773588in}}%
\pgfpathlineto{\pgfqpoint{-24.905462in}{0.773588in}}%
\pgfpathlineto{\pgfqpoint{-24.950861in}{0.773588in}}%
\pgfpathlineto{\pgfqpoint{-24.995958in}{0.773588in}}%
\pgfpathlineto{\pgfqpoint{-25.043027in}{0.773588in}}%
\pgfpathlineto{\pgfqpoint{-25.088791in}{0.773588in}}%
\pgfpathlineto{\pgfqpoint{-25.133757in}{0.773588in}}%
\pgfpathlineto{\pgfqpoint{-25.181603in}{0.773588in}}%
\pgfpathlineto{\pgfqpoint{-25.227067in}{0.773588in}}%
\pgfpathlineto{\pgfqpoint{-25.271783in}{0.773588in}}%
\pgfpathlineto{\pgfqpoint{-25.318638in}{0.773588in}}%
\pgfpathlineto{\pgfqpoint{-25.364031in}{0.773588in}}%
\pgfpathlineto{\pgfqpoint{-25.409278in}{0.773588in}}%
\pgfpathlineto{\pgfqpoint{-25.455838in}{0.773588in}}%
\pgfpathlineto{\pgfqpoint{-25.501531in}{0.773588in}}%
\pgfpathlineto{\pgfqpoint{-25.546370in}{0.773588in}}%
\pgfpathlineto{\pgfqpoint{-25.593105in}{0.773588in}}%
\pgfpathlineto{\pgfqpoint{-25.638897in}{0.773588in}}%
\pgfpathlineto{\pgfqpoint{-25.684196in}{0.773588in}}%
\pgfpathlineto{\pgfqpoint{-25.731236in}{0.773588in}}%
\pgfpathlineto{\pgfqpoint{-25.776589in}{0.773588in}}%
\pgfpathlineto{\pgfqpoint{-25.821656in}{0.773588in}}%
\pgfpathlineto{\pgfqpoint{-25.868331in}{0.773588in}}%
\pgfpathlineto{\pgfqpoint{-25.913010in}{0.773588in}}%
\pgfpathlineto{\pgfqpoint{-25.957619in}{0.773588in}}%
\pgfpathlineto{\pgfqpoint{-26.004871in}{0.773588in}}%
\pgfpathlineto{\pgfqpoint{-26.049962in}{0.773588in}}%
\pgfpathlineto{\pgfqpoint{-26.095397in}{0.773588in}}%
\pgfpathlineto{\pgfqpoint{-26.140687in}{0.773588in}}%
\pgfpathlineto{\pgfqpoint{-26.185627in}{0.773588in}}%
\pgfpathlineto{\pgfqpoint{-26.230565in}{0.773588in}}%
\pgfpathlineto{\pgfqpoint{-26.277965in}{0.773588in}}%
\pgfpathlineto{\pgfqpoint{-26.323384in}{0.773588in}}%
\pgfpathlineto{\pgfqpoint{-26.368659in}{0.773588in}}%
\pgfpathlineto{\pgfqpoint{-26.414228in}{0.773588in}}%
\pgfpathlineto{\pgfqpoint{-26.458732in}{0.773588in}}%
\pgfpathlineto{\pgfqpoint{-26.503012in}{0.773588in}}%
\pgfpathlineto{\pgfqpoint{-26.549765in}{0.773588in}}%
\pgfpathlineto{\pgfqpoint{-26.595409in}{0.773588in}}%
\pgfpathlineto{\pgfqpoint{-26.640477in}{0.773588in}}%
\pgfpathlineto{\pgfqpoint{-26.688111in}{0.773588in}}%
\pgfpathlineto{\pgfqpoint{-26.735142in}{0.773588in}}%
\pgfpathlineto{\pgfqpoint{-26.781839in}{0.773588in}}%
\pgfpathlineto{\pgfqpoint{-26.830647in}{0.773588in}}%
\pgfpathlineto{\pgfqpoint{-26.880611in}{0.773588in}}%
\pgfpathlineto{\pgfqpoint{-26.933555in}{0.773588in}}%
\pgfpathlineto{\pgfqpoint{-26.988998in}{0.773588in}}%
\pgfpathlineto{\pgfqpoint{-27.037714in}{0.773588in}}%
\pgfpathlineto{\pgfqpoint{-27.086555in}{0.773588in}}%
\pgfpathlineto{\pgfqpoint{-27.136341in}{0.773588in}}%
\pgfpathlineto{\pgfqpoint{-27.184903in}{0.773588in}}%
\pgfpathlineto{\pgfqpoint{-27.232862in}{0.773588in}}%
\pgfpathlineto{\pgfqpoint{-27.281821in}{0.773588in}}%
\pgfpathlineto{\pgfqpoint{-27.328705in}{0.773588in}}%
\pgfpathlineto{\pgfqpoint{-27.374936in}{0.773588in}}%
\pgfpathlineto{\pgfqpoint{-27.422670in}{0.773588in}}%
\pgfpathlineto{\pgfqpoint{-27.470237in}{0.773588in}}%
\pgfpathlineto{\pgfqpoint{-27.516652in}{0.773588in}}%
\pgfpathlineto{\pgfqpoint{-27.563341in}{0.773588in}}%
\pgfpathlineto{\pgfqpoint{-27.608047in}{0.773588in}}%
\pgfpathlineto{\pgfqpoint{-27.651646in}{0.773588in}}%
\pgfpathlineto{\pgfqpoint{-27.696819in}{0.773588in}}%
\pgfpathlineto{\pgfqpoint{-27.740871in}{0.773588in}}%
\pgfpathlineto{\pgfqpoint{-27.785329in}{0.773588in}}%
\pgfpathlineto{\pgfqpoint{-27.830427in}{0.773588in}}%
\pgfpathlineto{\pgfqpoint{-27.874405in}{0.773588in}}%
\pgfpathlineto{\pgfqpoint{-27.917812in}{0.773588in}}%
\pgfpathlineto{\pgfqpoint{-27.963188in}{0.773588in}}%
\pgfpathlineto{\pgfqpoint{-28.007858in}{0.773588in}}%
\pgfpathlineto{\pgfqpoint{-28.051883in}{0.773588in}}%
\pgfpathlineto{\pgfqpoint{-28.097071in}{0.773588in}}%
\pgfpathlineto{\pgfqpoint{-28.141536in}{0.773588in}}%
\pgfpathlineto{\pgfqpoint{-28.185953in}{0.773588in}}%
\pgfpathlineto{\pgfqpoint{-28.230974in}{0.773588in}}%
\pgfpathlineto{\pgfqpoint{-28.274355in}{0.773588in}}%
\pgfpathlineto{\pgfqpoint{-28.318373in}{0.773588in}}%
\pgfpathlineto{\pgfqpoint{-28.364238in}{0.773588in}}%
\pgfpathlineto{\pgfqpoint{-28.408130in}{0.773588in}}%
\pgfpathlineto{\pgfqpoint{-28.452363in}{0.773588in}}%
\pgfpathlineto{\pgfqpoint{-28.498901in}{0.773588in}}%
\pgfpathlineto{\pgfqpoint{-28.543533in}{0.773588in}}%
\pgfpathlineto{\pgfqpoint{-28.588250in}{0.773588in}}%
\pgfpathclose%
\pgfusepath{fill}%
\end{pgfscope}%
\begin{pgfscope}%
\pgfpathrectangle{\pgfqpoint{2.662073in}{0.773588in}}{\pgfqpoint{2.964025in}{5.415119in}}%
\pgfusepath{clip}%
\pgfsetbuttcap%
\pgfsetroundjoin%
\definecolor{currentfill}{rgb}{0.839216,0.152941,0.156863}%
\pgfsetfillcolor{currentfill}%
\pgfsetlinewidth{0.000000pt}%
\definecolor{currentstroke}{rgb}{0.000000,0.000000,0.000000}%
\pgfsetstrokecolor{currentstroke}%
\pgfsetdash{}{0pt}%
\pgfpathmoveto{\pgfqpoint{-28.588250in}{1.281056in}}%
\pgfpathlineto{\pgfqpoint{-28.588250in}{0.773588in}}%
\pgfpathlineto{\pgfqpoint{-28.543533in}{0.773588in}}%
\pgfpathlineto{\pgfqpoint{-28.498901in}{0.773588in}}%
\pgfpathlineto{\pgfqpoint{-28.452363in}{0.773588in}}%
\pgfpathlineto{\pgfqpoint{-28.408130in}{0.773588in}}%
\pgfpathlineto{\pgfqpoint{-28.364238in}{0.773588in}}%
\pgfpathlineto{\pgfqpoint{-28.318373in}{0.773588in}}%
\pgfpathlineto{\pgfqpoint{-28.274355in}{0.773588in}}%
\pgfpathlineto{\pgfqpoint{-28.230974in}{0.773588in}}%
\pgfpathlineto{\pgfqpoint{-28.185953in}{0.773588in}}%
\pgfpathlineto{\pgfqpoint{-28.141536in}{0.773588in}}%
\pgfpathlineto{\pgfqpoint{-28.097071in}{0.773588in}}%
\pgfpathlineto{\pgfqpoint{-28.051883in}{0.773588in}}%
\pgfpathlineto{\pgfqpoint{-28.007858in}{0.773588in}}%
\pgfpathlineto{\pgfqpoint{-27.963188in}{0.773588in}}%
\pgfpathlineto{\pgfqpoint{-27.917812in}{0.773588in}}%
\pgfpathlineto{\pgfqpoint{-27.874405in}{0.773588in}}%
\pgfpathlineto{\pgfqpoint{-27.830427in}{0.773588in}}%
\pgfpathlineto{\pgfqpoint{-27.785329in}{0.773588in}}%
\pgfpathlineto{\pgfqpoint{-27.740871in}{0.773588in}}%
\pgfpathlineto{\pgfqpoint{-27.696819in}{0.773588in}}%
\pgfpathlineto{\pgfqpoint{-27.651646in}{0.773588in}}%
\pgfpathlineto{\pgfqpoint{-27.608047in}{0.773588in}}%
\pgfpathlineto{\pgfqpoint{-27.563341in}{0.773588in}}%
\pgfpathlineto{\pgfqpoint{-27.516652in}{0.773588in}}%
\pgfpathlineto{\pgfqpoint{-27.470237in}{0.773588in}}%
\pgfpathlineto{\pgfqpoint{-27.422670in}{0.773588in}}%
\pgfpathlineto{\pgfqpoint{-27.374936in}{0.773588in}}%
\pgfpathlineto{\pgfqpoint{-27.328705in}{0.773588in}}%
\pgfpathlineto{\pgfqpoint{-27.281821in}{0.773588in}}%
\pgfpathlineto{\pgfqpoint{-27.232862in}{0.773588in}}%
\pgfpathlineto{\pgfqpoint{-27.184903in}{0.773588in}}%
\pgfpathlineto{\pgfqpoint{-27.136341in}{0.773588in}}%
\pgfpathlineto{\pgfqpoint{-27.086555in}{0.773588in}}%
\pgfpathlineto{\pgfqpoint{-27.037714in}{0.773588in}}%
\pgfpathlineto{\pgfqpoint{-26.988998in}{0.773588in}}%
\pgfpathlineto{\pgfqpoint{-26.933555in}{0.773588in}}%
\pgfpathlineto{\pgfqpoint{-26.880611in}{0.773588in}}%
\pgfpathlineto{\pgfqpoint{-26.830647in}{0.773588in}}%
\pgfpathlineto{\pgfqpoint{-26.781839in}{0.773588in}}%
\pgfpathlineto{\pgfqpoint{-26.735142in}{0.773588in}}%
\pgfpathlineto{\pgfqpoint{-26.688111in}{0.773588in}}%
\pgfpathlineto{\pgfqpoint{-26.640477in}{0.773588in}}%
\pgfpathlineto{\pgfqpoint{-26.595409in}{0.773588in}}%
\pgfpathlineto{\pgfqpoint{-26.549765in}{0.773588in}}%
\pgfpathlineto{\pgfqpoint{-26.503012in}{0.773588in}}%
\pgfpathlineto{\pgfqpoint{-26.458732in}{0.773588in}}%
\pgfpathlineto{\pgfqpoint{-26.414228in}{0.773588in}}%
\pgfpathlineto{\pgfqpoint{-26.368659in}{0.773588in}}%
\pgfpathlineto{\pgfqpoint{-26.323384in}{0.773588in}}%
\pgfpathlineto{\pgfqpoint{-26.277965in}{0.773588in}}%
\pgfpathlineto{\pgfqpoint{-26.230565in}{0.773588in}}%
\pgfpathlineto{\pgfqpoint{-26.185627in}{0.773588in}}%
\pgfpathlineto{\pgfqpoint{-26.140687in}{0.773588in}}%
\pgfpathlineto{\pgfqpoint{-26.095397in}{0.773588in}}%
\pgfpathlineto{\pgfqpoint{-26.049962in}{0.773588in}}%
\pgfpathlineto{\pgfqpoint{-26.004871in}{0.773588in}}%
\pgfpathlineto{\pgfqpoint{-25.957619in}{0.773588in}}%
\pgfpathlineto{\pgfqpoint{-25.913010in}{0.773588in}}%
\pgfpathlineto{\pgfqpoint{-25.868331in}{0.773588in}}%
\pgfpathlineto{\pgfqpoint{-25.821656in}{0.773588in}}%
\pgfpathlineto{\pgfqpoint{-25.776589in}{0.773588in}}%
\pgfpathlineto{\pgfqpoint{-25.731236in}{0.773588in}}%
\pgfpathlineto{\pgfqpoint{-25.684196in}{0.773588in}}%
\pgfpathlineto{\pgfqpoint{-25.638897in}{0.773588in}}%
\pgfpathlineto{\pgfqpoint{-25.593105in}{0.773588in}}%
\pgfpathlineto{\pgfqpoint{-25.546370in}{0.773588in}}%
\pgfpathlineto{\pgfqpoint{-25.501531in}{0.773588in}}%
\pgfpathlineto{\pgfqpoint{-25.455838in}{0.773588in}}%
\pgfpathlineto{\pgfqpoint{-25.409278in}{0.773588in}}%
\pgfpathlineto{\pgfqpoint{-25.364031in}{0.773588in}}%
\pgfpathlineto{\pgfqpoint{-25.318638in}{0.773588in}}%
\pgfpathlineto{\pgfqpoint{-25.271783in}{0.773588in}}%
\pgfpathlineto{\pgfqpoint{-25.227067in}{0.773588in}}%
\pgfpathlineto{\pgfqpoint{-25.181603in}{0.773588in}}%
\pgfpathlineto{\pgfqpoint{-25.133757in}{0.773588in}}%
\pgfpathlineto{\pgfqpoint{-25.088791in}{0.773588in}}%
\pgfpathlineto{\pgfqpoint{-25.043027in}{0.773588in}}%
\pgfpathlineto{\pgfqpoint{-24.995958in}{0.773588in}}%
\pgfpathlineto{\pgfqpoint{-24.950861in}{0.773588in}}%
\pgfpathlineto{\pgfqpoint{-24.905462in}{0.773588in}}%
\pgfpathlineto{\pgfqpoint{-24.858721in}{0.773588in}}%
\pgfpathlineto{\pgfqpoint{-24.813437in}{0.773588in}}%
\pgfpathlineto{\pgfqpoint{-24.768227in}{0.773588in}}%
\pgfpathlineto{\pgfqpoint{-24.722523in}{0.773588in}}%
\pgfpathlineto{\pgfqpoint{-24.677917in}{0.773588in}}%
\pgfpathlineto{\pgfqpoint{-24.632696in}{0.773588in}}%
\pgfpathlineto{\pgfqpoint{-24.586259in}{0.773588in}}%
\pgfpathlineto{\pgfqpoint{-24.540716in}{0.773588in}}%
\pgfpathlineto{\pgfqpoint{-24.494847in}{0.773588in}}%
\pgfpathlineto{\pgfqpoint{-24.447114in}{0.773588in}}%
\pgfpathlineto{\pgfqpoint{-24.401173in}{0.773588in}}%
\pgfpathlineto{\pgfqpoint{-24.355467in}{0.773588in}}%
\pgfpathlineto{\pgfqpoint{-24.308071in}{0.773588in}}%
\pgfpathlineto{\pgfqpoint{-24.262109in}{0.773588in}}%
\pgfpathlineto{\pgfqpoint{-24.216953in}{0.773588in}}%
\pgfpathlineto{\pgfqpoint{-24.169854in}{0.773588in}}%
\pgfpathlineto{\pgfqpoint{-24.123999in}{0.773588in}}%
\pgfpathlineto{\pgfqpoint{-24.078902in}{0.773588in}}%
\pgfpathlineto{\pgfqpoint{-24.032531in}{0.773588in}}%
\pgfpathlineto{\pgfqpoint{-23.986811in}{0.773588in}}%
\pgfpathlineto{\pgfqpoint{-23.941619in}{0.773588in}}%
\pgfpathlineto{\pgfqpoint{-23.895017in}{0.773588in}}%
\pgfpathlineto{\pgfqpoint{-23.849943in}{0.773588in}}%
\pgfpathlineto{\pgfqpoint{-23.804343in}{0.773588in}}%
\pgfpathlineto{\pgfqpoint{-23.757289in}{0.773588in}}%
\pgfpathlineto{\pgfqpoint{-23.711883in}{0.773588in}}%
\pgfpathlineto{\pgfqpoint{-23.666576in}{0.773588in}}%
\pgfpathlineto{\pgfqpoint{-23.620333in}{0.773588in}}%
\pgfpathlineto{\pgfqpoint{-23.575877in}{0.773588in}}%
\pgfpathlineto{\pgfqpoint{-23.529965in}{0.773588in}}%
\pgfpathlineto{\pgfqpoint{-23.482780in}{0.773588in}}%
\pgfpathlineto{\pgfqpoint{-23.437154in}{0.773588in}}%
\pgfpathlineto{\pgfqpoint{-23.391460in}{0.773588in}}%
\pgfpathlineto{\pgfqpoint{-23.344516in}{0.773588in}}%
\pgfpathlineto{\pgfqpoint{-23.298917in}{0.773588in}}%
\pgfpathlineto{\pgfqpoint{-23.252910in}{0.773588in}}%
\pgfpathlineto{\pgfqpoint{-23.205343in}{0.773588in}}%
\pgfpathlineto{\pgfqpoint{-23.159034in}{0.773588in}}%
\pgfpathlineto{\pgfqpoint{-23.112561in}{0.773588in}}%
\pgfpathlineto{\pgfqpoint{-23.065076in}{0.773588in}}%
\pgfpathlineto{\pgfqpoint{-23.018637in}{0.773588in}}%
\pgfpathlineto{\pgfqpoint{-22.972746in}{0.773588in}}%
\pgfpathlineto{\pgfqpoint{-22.924822in}{0.773588in}}%
\pgfpathlineto{\pgfqpoint{-22.878494in}{0.773588in}}%
\pgfpathlineto{\pgfqpoint{-22.833029in}{0.773588in}}%
\pgfpathlineto{\pgfqpoint{-22.786844in}{0.773588in}}%
\pgfpathlineto{\pgfqpoint{-22.742185in}{0.773588in}}%
\pgfpathlineto{\pgfqpoint{-22.697642in}{0.773588in}}%
\pgfpathlineto{\pgfqpoint{-22.651258in}{0.773588in}}%
\pgfpathlineto{\pgfqpoint{-22.606850in}{0.773588in}}%
\pgfpathlineto{\pgfqpoint{-22.562133in}{0.773588in}}%
\pgfpathlineto{\pgfqpoint{-22.516601in}{0.773588in}}%
\pgfpathlineto{\pgfqpoint{-22.471952in}{0.773588in}}%
\pgfpathlineto{\pgfqpoint{-22.427383in}{0.773588in}}%
\pgfpathlineto{\pgfqpoint{-22.380996in}{0.773588in}}%
\pgfpathlineto{\pgfqpoint{-22.336074in}{0.773588in}}%
\pgfpathlineto{\pgfqpoint{-22.291940in}{0.773588in}}%
\pgfpathlineto{\pgfqpoint{-22.246011in}{0.773588in}}%
\pgfpathlineto{\pgfqpoint{-22.201868in}{0.773588in}}%
\pgfpathlineto{\pgfqpoint{-22.158025in}{0.773588in}}%
\pgfpathlineto{\pgfqpoint{-22.112231in}{0.773588in}}%
\pgfpathlineto{\pgfqpoint{-22.067661in}{0.773588in}}%
\pgfpathlineto{\pgfqpoint{-22.022397in}{0.773588in}}%
\pgfpathlineto{\pgfqpoint{-21.976168in}{0.773588in}}%
\pgfpathlineto{\pgfqpoint{-21.930597in}{0.773588in}}%
\pgfpathlineto{\pgfqpoint{-21.885462in}{0.773588in}}%
\pgfpathlineto{\pgfqpoint{-21.839254in}{0.773588in}}%
\pgfpathlineto{\pgfqpoint{-21.794255in}{0.773588in}}%
\pgfpathlineto{\pgfqpoint{-21.749554in}{0.773588in}}%
\pgfpathlineto{\pgfqpoint{-21.703167in}{0.773588in}}%
\pgfpathlineto{\pgfqpoint{-21.657342in}{0.773588in}}%
\pgfpathlineto{\pgfqpoint{-21.612919in}{0.773588in}}%
\pgfpathlineto{\pgfqpoint{-21.566198in}{0.773588in}}%
\pgfpathlineto{\pgfqpoint{-21.520883in}{0.773588in}}%
\pgfpathlineto{\pgfqpoint{-21.475985in}{0.773588in}}%
\pgfpathlineto{\pgfqpoint{-21.430267in}{0.773588in}}%
\pgfpathlineto{\pgfqpoint{-21.385572in}{0.773588in}}%
\pgfpathlineto{\pgfqpoint{-21.340867in}{0.773588in}}%
\pgfpathlineto{\pgfqpoint{-21.294264in}{0.773588in}}%
\pgfpathlineto{\pgfqpoint{-21.249304in}{0.773588in}}%
\pgfpathlineto{\pgfqpoint{-21.204354in}{0.773588in}}%
\pgfpathlineto{\pgfqpoint{-21.158215in}{0.773588in}}%
\pgfpathlineto{\pgfqpoint{-21.113139in}{0.773588in}}%
\pgfpathlineto{\pgfqpoint{-21.067357in}{0.773588in}}%
\pgfpathlineto{\pgfqpoint{-21.021509in}{0.773588in}}%
\pgfpathlineto{\pgfqpoint{-20.976378in}{0.773588in}}%
\pgfpathlineto{\pgfqpoint{-20.932365in}{0.773588in}}%
\pgfpathlineto{\pgfqpoint{-20.886847in}{0.773588in}}%
\pgfpathlineto{\pgfqpoint{-20.842169in}{0.773588in}}%
\pgfpathlineto{\pgfqpoint{-20.797022in}{0.773588in}}%
\pgfpathlineto{\pgfqpoint{-20.750831in}{0.773588in}}%
\pgfpathlineto{\pgfqpoint{-20.705066in}{0.773588in}}%
\pgfpathlineto{\pgfqpoint{-20.658742in}{0.773588in}}%
\pgfpathlineto{\pgfqpoint{-20.611243in}{0.773588in}}%
\pgfpathlineto{\pgfqpoint{-20.565077in}{0.773588in}}%
\pgfpathlineto{\pgfqpoint{-20.519904in}{0.773588in}}%
\pgfpathlineto{\pgfqpoint{-20.474014in}{0.773588in}}%
\pgfpathlineto{\pgfqpoint{-20.428221in}{0.773588in}}%
\pgfpathlineto{\pgfqpoint{-20.382199in}{0.773588in}}%
\pgfpathlineto{\pgfqpoint{-20.335981in}{0.773588in}}%
\pgfpathlineto{\pgfqpoint{-20.290419in}{0.773588in}}%
\pgfpathlineto{\pgfqpoint{-20.244875in}{0.773588in}}%
\pgfpathlineto{\pgfqpoint{-20.197814in}{0.773588in}}%
\pgfpathlineto{\pgfqpoint{-20.153038in}{0.773588in}}%
\pgfpathlineto{\pgfqpoint{-20.107707in}{0.773588in}}%
\pgfpathlineto{\pgfqpoint{-20.061722in}{0.773588in}}%
\pgfpathlineto{\pgfqpoint{-20.016506in}{0.773588in}}%
\pgfpathlineto{\pgfqpoint{-19.970967in}{0.773588in}}%
\pgfpathlineto{\pgfqpoint{-19.924139in}{0.773588in}}%
\pgfpathlineto{\pgfqpoint{-19.878562in}{0.773588in}}%
\pgfpathlineto{\pgfqpoint{-19.833948in}{0.773588in}}%
\pgfpathlineto{\pgfqpoint{-19.787808in}{0.773588in}}%
\pgfpathlineto{\pgfqpoint{-19.742024in}{0.773588in}}%
\pgfpathlineto{\pgfqpoint{-19.696649in}{0.773588in}}%
\pgfpathlineto{\pgfqpoint{-19.650361in}{0.773588in}}%
\pgfpathlineto{\pgfqpoint{-19.604824in}{0.773588in}}%
\pgfpathlineto{\pgfqpoint{-19.559546in}{0.773588in}}%
\pgfpathlineto{\pgfqpoint{-19.512973in}{0.773588in}}%
\pgfpathlineto{\pgfqpoint{-19.467303in}{0.773588in}}%
\pgfpathlineto{\pgfqpoint{-19.421584in}{0.773588in}}%
\pgfpathlineto{\pgfqpoint{-19.374230in}{0.773588in}}%
\pgfpathlineto{\pgfqpoint{-19.328532in}{0.773588in}}%
\pgfpathlineto{\pgfqpoint{-19.283047in}{0.773588in}}%
\pgfpathlineto{\pgfqpoint{-19.234797in}{0.773588in}}%
\pgfpathlineto{\pgfqpoint{-19.188899in}{0.773588in}}%
\pgfpathlineto{\pgfqpoint{-19.143544in}{0.773588in}}%
\pgfpathlineto{\pgfqpoint{-19.097062in}{0.773588in}}%
\pgfpathlineto{\pgfqpoint{-19.050812in}{0.773588in}}%
\pgfpathlineto{\pgfqpoint{-19.004576in}{0.773588in}}%
\pgfpathlineto{\pgfqpoint{-18.956783in}{0.773588in}}%
\pgfpathlineto{\pgfqpoint{-18.910503in}{0.773588in}}%
\pgfpathlineto{\pgfqpoint{-18.865040in}{0.773588in}}%
\pgfpathlineto{\pgfqpoint{-18.817991in}{0.773588in}}%
\pgfpathlineto{\pgfqpoint{-18.772260in}{0.773588in}}%
\pgfpathlineto{\pgfqpoint{-18.726033in}{0.773588in}}%
\pgfpathlineto{\pgfqpoint{-18.678684in}{0.773588in}}%
\pgfpathlineto{\pgfqpoint{-18.632978in}{0.773588in}}%
\pgfpathlineto{\pgfqpoint{-18.587378in}{0.773588in}}%
\pgfpathlineto{\pgfqpoint{-18.540355in}{0.773588in}}%
\pgfpathlineto{\pgfqpoint{-18.494344in}{0.773588in}}%
\pgfpathlineto{\pgfqpoint{-18.448764in}{0.773588in}}%
\pgfpathlineto{\pgfqpoint{-18.402000in}{0.773588in}}%
\pgfpathlineto{\pgfqpoint{-18.356748in}{0.773588in}}%
\pgfpathlineto{\pgfqpoint{-18.311512in}{0.773588in}}%
\pgfpathlineto{\pgfqpoint{-18.264167in}{0.773588in}}%
\pgfpathlineto{\pgfqpoint{-18.218674in}{0.773588in}}%
\pgfpathlineto{\pgfqpoint{-18.172791in}{0.773588in}}%
\pgfpathlineto{\pgfqpoint{-18.124795in}{0.773588in}}%
\pgfpathlineto{\pgfqpoint{-18.078784in}{0.773588in}}%
\pgfpathlineto{\pgfqpoint{-18.032389in}{0.773588in}}%
\pgfpathlineto{\pgfqpoint{-17.984102in}{0.773588in}}%
\pgfpathlineto{\pgfqpoint{-17.937614in}{0.773588in}}%
\pgfpathlineto{\pgfqpoint{-17.891271in}{0.773588in}}%
\pgfpathlineto{\pgfqpoint{-17.843480in}{0.773588in}}%
\pgfpathlineto{\pgfqpoint{-17.797346in}{0.773588in}}%
\pgfpathlineto{\pgfqpoint{-17.751556in}{0.773588in}}%
\pgfpathlineto{\pgfqpoint{-17.704093in}{0.773588in}}%
\pgfpathlineto{\pgfqpoint{-17.658538in}{0.773588in}}%
\pgfpathlineto{\pgfqpoint{-17.613251in}{0.773588in}}%
\pgfpathlineto{\pgfqpoint{-17.566739in}{0.773588in}}%
\pgfpathlineto{\pgfqpoint{-17.521284in}{0.773588in}}%
\pgfpathlineto{\pgfqpoint{-17.474644in}{0.773588in}}%
\pgfpathlineto{\pgfqpoint{-17.427002in}{0.773588in}}%
\pgfpathlineto{\pgfqpoint{-17.381732in}{0.773588in}}%
\pgfpathlineto{\pgfqpoint{-17.336276in}{0.773588in}}%
\pgfpathlineto{\pgfqpoint{-17.289523in}{0.773588in}}%
\pgfpathlineto{\pgfqpoint{-17.244258in}{0.773588in}}%
\pgfpathlineto{\pgfqpoint{-17.198221in}{0.773588in}}%
\pgfpathlineto{\pgfqpoint{-17.150161in}{0.773588in}}%
\pgfpathlineto{\pgfqpoint{-17.103874in}{0.773588in}}%
\pgfpathlineto{\pgfqpoint{-17.058009in}{0.773588in}}%
\pgfpathlineto{\pgfqpoint{-17.010884in}{0.773588in}}%
\pgfpathlineto{\pgfqpoint{-16.964774in}{0.773588in}}%
\pgfpathlineto{\pgfqpoint{-16.918692in}{0.773588in}}%
\pgfpathlineto{\pgfqpoint{-16.871664in}{0.773588in}}%
\pgfpathlineto{\pgfqpoint{-16.825434in}{0.773588in}}%
\pgfpathlineto{\pgfqpoint{-16.778530in}{0.773588in}}%
\pgfpathlineto{\pgfqpoint{-16.730721in}{0.773588in}}%
\pgfpathlineto{\pgfqpoint{-16.684310in}{0.773588in}}%
\pgfpathlineto{\pgfqpoint{-16.637120in}{0.773588in}}%
\pgfpathlineto{\pgfqpoint{-16.589239in}{0.773588in}}%
\pgfpathlineto{\pgfqpoint{-16.542990in}{0.773588in}}%
\pgfpathlineto{\pgfqpoint{-16.496157in}{0.773588in}}%
\pgfpathlineto{\pgfqpoint{-16.449347in}{0.773588in}}%
\pgfpathlineto{\pgfqpoint{-16.402550in}{0.773588in}}%
\pgfpathlineto{\pgfqpoint{-16.355646in}{0.773588in}}%
\pgfpathlineto{\pgfqpoint{-16.308871in}{0.773588in}}%
\pgfpathlineto{\pgfqpoint{-16.262806in}{0.773588in}}%
\pgfpathlineto{\pgfqpoint{-16.217070in}{0.773588in}}%
\pgfpathlineto{\pgfqpoint{-16.169427in}{0.773588in}}%
\pgfpathlineto{\pgfqpoint{-16.123165in}{0.773588in}}%
\pgfpathlineto{\pgfqpoint{-16.077171in}{0.773588in}}%
\pgfpathlineto{\pgfqpoint{-16.029628in}{0.773588in}}%
\pgfpathlineto{\pgfqpoint{-15.984286in}{0.773588in}}%
\pgfpathlineto{\pgfqpoint{-15.938076in}{0.773588in}}%
\pgfpathlineto{\pgfqpoint{-15.890419in}{0.773588in}}%
\pgfpathlineto{\pgfqpoint{-15.844500in}{0.773588in}}%
\pgfpathlineto{\pgfqpoint{-15.798897in}{0.773588in}}%
\pgfpathlineto{\pgfqpoint{-15.750800in}{0.773588in}}%
\pgfpathlineto{\pgfqpoint{-15.703994in}{0.773588in}}%
\pgfpathlineto{\pgfqpoint{-15.657002in}{0.773588in}}%
\pgfpathlineto{\pgfqpoint{-15.608704in}{0.773588in}}%
\pgfpathlineto{\pgfqpoint{-15.560849in}{0.773588in}}%
\pgfpathlineto{\pgfqpoint{-15.512535in}{0.773588in}}%
\pgfpathlineto{\pgfqpoint{-15.463539in}{0.773588in}}%
\pgfpathlineto{\pgfqpoint{-15.416792in}{0.773588in}}%
\pgfpathlineto{\pgfqpoint{-15.369802in}{0.773588in}}%
\pgfpathlineto{\pgfqpoint{-15.321194in}{0.773588in}}%
\pgfpathlineto{\pgfqpoint{-15.273998in}{0.773588in}}%
\pgfpathlineto{\pgfqpoint{-15.225998in}{0.773588in}}%
\pgfpathlineto{\pgfqpoint{-15.176769in}{0.773588in}}%
\pgfpathlineto{\pgfqpoint{-15.130553in}{0.773588in}}%
\pgfpathlineto{\pgfqpoint{-15.083728in}{0.773588in}}%
\pgfpathlineto{\pgfqpoint{-15.035921in}{0.773588in}}%
\pgfpathlineto{\pgfqpoint{-14.989019in}{0.773588in}}%
\pgfpathlineto{\pgfqpoint{-14.942416in}{0.773588in}}%
\pgfpathlineto{\pgfqpoint{-14.894419in}{0.773588in}}%
\pgfpathlineto{\pgfqpoint{-14.847516in}{0.773588in}}%
\pgfpathlineto{\pgfqpoint{-14.800669in}{0.773588in}}%
\pgfpathlineto{\pgfqpoint{-14.752153in}{0.773588in}}%
\pgfpathlineto{\pgfqpoint{-14.705919in}{0.773588in}}%
\pgfpathlineto{\pgfqpoint{-14.659388in}{0.773588in}}%
\pgfpathlineto{\pgfqpoint{-14.611394in}{0.773588in}}%
\pgfpathlineto{\pgfqpoint{-14.564671in}{0.773588in}}%
\pgfpathlineto{\pgfqpoint{-14.517904in}{0.773588in}}%
\pgfpathlineto{\pgfqpoint{-14.469907in}{0.773588in}}%
\pgfpathlineto{\pgfqpoint{-14.422978in}{0.773588in}}%
\pgfpathlineto{\pgfqpoint{-14.375931in}{0.773588in}}%
\pgfpathlineto{\pgfqpoint{-14.327371in}{0.773588in}}%
\pgfpathlineto{\pgfqpoint{-14.280667in}{0.773588in}}%
\pgfpathlineto{\pgfqpoint{-14.233487in}{0.773588in}}%
\pgfpathlineto{\pgfqpoint{-14.184689in}{0.773588in}}%
\pgfpathlineto{\pgfqpoint{-14.137283in}{0.773588in}}%
\pgfpathlineto{\pgfqpoint{-14.090279in}{0.773588in}}%
\pgfpathlineto{\pgfqpoint{-14.042263in}{0.773588in}}%
\pgfpathlineto{\pgfqpoint{-13.996250in}{0.773588in}}%
\pgfpathlineto{\pgfqpoint{-13.950094in}{0.773588in}}%
\pgfpathlineto{\pgfqpoint{-13.901287in}{0.773588in}}%
\pgfpathlineto{\pgfqpoint{-13.853816in}{0.773588in}}%
\pgfpathlineto{\pgfqpoint{-13.806330in}{0.773588in}}%
\pgfpathlineto{\pgfqpoint{-13.757485in}{0.773588in}}%
\pgfpathlineto{\pgfqpoint{-13.710126in}{0.773588in}}%
\pgfpathlineto{\pgfqpoint{-13.663331in}{0.773588in}}%
\pgfpathlineto{\pgfqpoint{-13.615991in}{0.773588in}}%
\pgfpathlineto{\pgfqpoint{-13.568873in}{0.773588in}}%
\pgfpathlineto{\pgfqpoint{-13.522553in}{0.773588in}}%
\pgfpathlineto{\pgfqpoint{-13.474335in}{0.773588in}}%
\pgfpathlineto{\pgfqpoint{-13.428346in}{0.773588in}}%
\pgfpathlineto{\pgfqpoint{-13.382220in}{0.773588in}}%
\pgfpathlineto{\pgfqpoint{-13.334326in}{0.773588in}}%
\pgfpathlineto{\pgfqpoint{-13.287448in}{0.773588in}}%
\pgfpathlineto{\pgfqpoint{-13.239807in}{0.773588in}}%
\pgfpathlineto{\pgfqpoint{-13.190970in}{0.773588in}}%
\pgfpathlineto{\pgfqpoint{-13.144068in}{0.773588in}}%
\pgfpathlineto{\pgfqpoint{-13.096328in}{0.773588in}}%
\pgfpathlineto{\pgfqpoint{-13.046608in}{0.773588in}}%
\pgfpathlineto{\pgfqpoint{-12.998480in}{0.773588in}}%
\pgfpathlineto{\pgfqpoint{-12.949523in}{0.773588in}}%
\pgfpathlineto{\pgfqpoint{-12.899585in}{0.773588in}}%
\pgfpathlineto{\pgfqpoint{-12.852526in}{0.773588in}}%
\pgfpathlineto{\pgfqpoint{-12.805010in}{0.773588in}}%
\pgfpathlineto{\pgfqpoint{-12.755866in}{0.773588in}}%
\pgfpathlineto{\pgfqpoint{-12.708579in}{0.773588in}}%
\pgfpathlineto{\pgfqpoint{-12.661578in}{0.773588in}}%
\pgfpathlineto{\pgfqpoint{-12.612433in}{0.773588in}}%
\pgfpathlineto{\pgfqpoint{-12.565107in}{0.773588in}}%
\pgfpathlineto{\pgfqpoint{-12.517880in}{0.773588in}}%
\pgfpathlineto{\pgfqpoint{-12.469843in}{0.773588in}}%
\pgfpathlineto{\pgfqpoint{-12.422612in}{0.773588in}}%
\pgfpathlineto{\pgfqpoint{-12.375560in}{0.773588in}}%
\pgfpathlineto{\pgfqpoint{-12.326627in}{0.773588in}}%
\pgfpathlineto{\pgfqpoint{-12.278675in}{0.773588in}}%
\pgfpathlineto{\pgfqpoint{-12.231638in}{0.773588in}}%
\pgfpathlineto{\pgfqpoint{-12.182471in}{0.773588in}}%
\pgfpathlineto{\pgfqpoint{-12.135286in}{0.773588in}}%
\pgfpathlineto{\pgfqpoint{-12.087205in}{0.773588in}}%
\pgfpathlineto{\pgfqpoint{-12.038569in}{0.773588in}}%
\pgfpathlineto{\pgfqpoint{-11.991431in}{0.773588in}}%
\pgfpathlineto{\pgfqpoint{-11.943783in}{0.773588in}}%
\pgfpathlineto{\pgfqpoint{-11.895021in}{0.773588in}}%
\pgfpathlineto{\pgfqpoint{-11.847736in}{0.773588in}}%
\pgfpathlineto{\pgfqpoint{-11.800403in}{0.773588in}}%
\pgfpathlineto{\pgfqpoint{-11.751492in}{0.773588in}}%
\pgfpathlineto{\pgfqpoint{-11.704231in}{0.773588in}}%
\pgfpathlineto{\pgfqpoint{-11.656418in}{0.773588in}}%
\pgfpathlineto{\pgfqpoint{-11.607131in}{0.773588in}}%
\pgfpathlineto{\pgfqpoint{-11.559336in}{0.773588in}}%
\pgfpathlineto{\pgfqpoint{-11.511546in}{0.773588in}}%
\pgfpathlineto{\pgfqpoint{-11.463260in}{0.773588in}}%
\pgfpathlineto{\pgfqpoint{-11.415235in}{0.773588in}}%
\pgfpathlineto{\pgfqpoint{-11.366120in}{0.773588in}}%
\pgfpathlineto{\pgfqpoint{-11.316622in}{0.773588in}}%
\pgfpathlineto{\pgfqpoint{-11.268838in}{0.773588in}}%
\pgfpathlineto{\pgfqpoint{-11.221741in}{0.773588in}}%
\pgfpathlineto{\pgfqpoint{-11.172812in}{0.773588in}}%
\pgfpathlineto{\pgfqpoint{-11.125274in}{0.773588in}}%
\pgfpathlineto{\pgfqpoint{-11.077704in}{0.773588in}}%
\pgfpathlineto{\pgfqpoint{-11.028607in}{0.773588in}}%
\pgfpathlineto{\pgfqpoint{-10.980903in}{0.773588in}}%
\pgfpathlineto{\pgfqpoint{-10.933861in}{0.773588in}}%
\pgfpathlineto{\pgfqpoint{-10.885230in}{0.773588in}}%
\pgfpathlineto{\pgfqpoint{-10.837686in}{0.773588in}}%
\pgfpathlineto{\pgfqpoint{-10.790422in}{0.773588in}}%
\pgfpathlineto{\pgfqpoint{-10.741507in}{0.773588in}}%
\pgfpathlineto{\pgfqpoint{-10.693023in}{0.773588in}}%
\pgfpathlineto{\pgfqpoint{-10.644481in}{0.773588in}}%
\pgfpathlineto{\pgfqpoint{-10.594934in}{0.773588in}}%
\pgfpathlineto{\pgfqpoint{-10.546833in}{0.773588in}}%
\pgfpathlineto{\pgfqpoint{-10.497957in}{0.773588in}}%
\pgfpathlineto{\pgfqpoint{-10.448474in}{0.773588in}}%
\pgfpathlineto{\pgfqpoint{-10.401144in}{0.773588in}}%
\pgfpathlineto{\pgfqpoint{-10.352351in}{0.773588in}}%
\pgfpathlineto{\pgfqpoint{-10.301938in}{0.773588in}}%
\pgfpathlineto{\pgfqpoint{-10.253421in}{0.773588in}}%
\pgfpathlineto{\pgfqpoint{-10.204843in}{0.773588in}}%
\pgfpathlineto{\pgfqpoint{-10.154319in}{0.773588in}}%
\pgfpathlineto{\pgfqpoint{-10.106404in}{0.773588in}}%
\pgfpathlineto{\pgfqpoint{-10.058316in}{0.773588in}}%
\pgfpathlineto{\pgfqpoint{-10.009255in}{0.773588in}}%
\pgfpathlineto{\pgfqpoint{-9.960967in}{0.773588in}}%
\pgfpathlineto{\pgfqpoint{-9.912991in}{0.773588in}}%
\pgfpathlineto{\pgfqpoint{-9.863853in}{0.773588in}}%
\pgfpathlineto{\pgfqpoint{-9.816398in}{0.773588in}}%
\pgfpathlineto{\pgfqpoint{-9.768438in}{0.773588in}}%
\pgfpathlineto{\pgfqpoint{-9.719205in}{0.773588in}}%
\pgfpathlineto{\pgfqpoint{-9.671880in}{0.773588in}}%
\pgfpathlineto{\pgfqpoint{-9.624467in}{0.773588in}}%
\pgfpathlineto{\pgfqpoint{-9.575433in}{0.773588in}}%
\pgfpathlineto{\pgfqpoint{-9.527213in}{0.773588in}}%
\pgfpathlineto{\pgfqpoint{-9.479530in}{0.773588in}}%
\pgfpathlineto{\pgfqpoint{-9.430893in}{0.773588in}}%
\pgfpathlineto{\pgfqpoint{-9.382453in}{0.773588in}}%
\pgfpathlineto{\pgfqpoint{-9.332833in}{0.773588in}}%
\pgfpathlineto{\pgfqpoint{-9.281795in}{0.773588in}}%
\pgfpathlineto{\pgfqpoint{-9.232466in}{0.773588in}}%
\pgfpathlineto{\pgfqpoint{-9.182531in}{0.773588in}}%
\pgfpathlineto{\pgfqpoint{-9.130552in}{0.773588in}}%
\pgfpathlineto{\pgfqpoint{-9.080221in}{0.773588in}}%
\pgfpathlineto{\pgfqpoint{-9.030683in}{0.773588in}}%
\pgfpathlineto{\pgfqpoint{-8.979731in}{0.773588in}}%
\pgfpathlineto{\pgfqpoint{-8.930541in}{0.773588in}}%
\pgfpathlineto{\pgfqpoint{-8.880937in}{0.773588in}}%
\pgfpathlineto{\pgfqpoint{-8.829855in}{0.773588in}}%
\pgfpathlineto{\pgfqpoint{-8.780578in}{0.773588in}}%
\pgfpathlineto{\pgfqpoint{-8.731477in}{0.773588in}}%
\pgfpathlineto{\pgfqpoint{-8.681529in}{0.773588in}}%
\pgfpathlineto{\pgfqpoint{-8.632588in}{0.773588in}}%
\pgfpathlineto{\pgfqpoint{-8.583181in}{0.773588in}}%
\pgfpathlineto{\pgfqpoint{-8.532027in}{0.773588in}}%
\pgfpathlineto{\pgfqpoint{-8.481981in}{0.773588in}}%
\pgfpathlineto{\pgfqpoint{-8.432560in}{0.773588in}}%
\pgfpathlineto{\pgfqpoint{-8.382452in}{0.773588in}}%
\pgfpathlineto{\pgfqpoint{-8.332630in}{0.773588in}}%
\pgfpathlineto{\pgfqpoint{-8.283775in}{0.773588in}}%
\pgfpathlineto{\pgfqpoint{-8.233427in}{0.773588in}}%
\pgfpathlineto{\pgfqpoint{-8.184462in}{0.773588in}}%
\pgfpathlineto{\pgfqpoint{-8.135490in}{0.773588in}}%
\pgfpathlineto{\pgfqpoint{-8.085190in}{0.773588in}}%
\pgfpathlineto{\pgfqpoint{-8.035792in}{0.773588in}}%
\pgfpathlineto{\pgfqpoint{-7.986412in}{0.773588in}}%
\pgfpathlineto{\pgfqpoint{-7.935315in}{0.773588in}}%
\pgfpathlineto{\pgfqpoint{-7.885122in}{0.773588in}}%
\pgfpathlineto{\pgfqpoint{-7.835188in}{0.773588in}}%
\pgfpathlineto{\pgfqpoint{-7.783607in}{0.773588in}}%
\pgfpathlineto{\pgfqpoint{-7.733870in}{0.773588in}}%
\pgfpathlineto{\pgfqpoint{-7.685281in}{0.773588in}}%
\pgfpathlineto{\pgfqpoint{-7.634521in}{0.773588in}}%
\pgfpathlineto{\pgfqpoint{-7.585672in}{0.773588in}}%
\pgfpathlineto{\pgfqpoint{-7.536566in}{0.773588in}}%
\pgfpathlineto{\pgfqpoint{-7.485312in}{0.773588in}}%
\pgfpathlineto{\pgfqpoint{-7.435602in}{0.773588in}}%
\pgfpathlineto{\pgfqpoint{-7.386343in}{0.773588in}}%
\pgfpathlineto{\pgfqpoint{-7.335791in}{0.773588in}}%
\pgfpathlineto{\pgfqpoint{-7.286238in}{0.773588in}}%
\pgfpathlineto{\pgfqpoint{-7.237328in}{0.773588in}}%
\pgfpathlineto{\pgfqpoint{-7.185558in}{0.773588in}}%
\pgfpathlineto{\pgfqpoint{-7.135511in}{0.773588in}}%
\pgfpathlineto{\pgfqpoint{-7.085212in}{0.773588in}}%
\pgfpathlineto{\pgfqpoint{-7.033318in}{0.773588in}}%
\pgfpathlineto{\pgfqpoint{-6.984061in}{0.773588in}}%
\pgfpathlineto{\pgfqpoint{-6.934169in}{0.773588in}}%
\pgfpathlineto{\pgfqpoint{-6.882593in}{0.773588in}}%
\pgfpathlineto{\pgfqpoint{-6.832072in}{0.773588in}}%
\pgfpathlineto{\pgfqpoint{-6.782488in}{0.773588in}}%
\pgfpathlineto{\pgfqpoint{-6.731323in}{0.773588in}}%
\pgfpathlineto{\pgfqpoint{-6.682747in}{0.773588in}}%
\pgfpathlineto{\pgfqpoint{-6.634445in}{0.773588in}}%
\pgfpathlineto{\pgfqpoint{-6.582317in}{0.773588in}}%
\pgfpathlineto{\pgfqpoint{-6.532179in}{0.773588in}}%
\pgfpathlineto{\pgfqpoint{-6.481910in}{0.773588in}}%
\pgfpathlineto{\pgfqpoint{-6.429787in}{0.773588in}}%
\pgfpathlineto{\pgfqpoint{-6.380124in}{0.773588in}}%
\pgfpathlineto{\pgfqpoint{-6.330959in}{0.773588in}}%
\pgfpathlineto{\pgfqpoint{-6.279011in}{0.773588in}}%
\pgfpathlineto{\pgfqpoint{-6.228389in}{0.773588in}}%
\pgfpathlineto{\pgfqpoint{-6.177451in}{0.773588in}}%
\pgfpathlineto{\pgfqpoint{-6.125420in}{0.773588in}}%
\pgfpathlineto{\pgfqpoint{-6.075227in}{0.773588in}}%
\pgfpathlineto{\pgfqpoint{-6.024836in}{0.773588in}}%
\pgfpathlineto{\pgfqpoint{-5.974236in}{0.773588in}}%
\pgfpathlineto{\pgfqpoint{-5.924128in}{0.773588in}}%
\pgfpathlineto{\pgfqpoint{-5.874278in}{0.773588in}}%
\pgfpathlineto{\pgfqpoint{-5.822716in}{0.773588in}}%
\pgfpathlineto{\pgfqpoint{-5.774223in}{0.773588in}}%
\pgfpathlineto{\pgfqpoint{-5.724570in}{0.773588in}}%
\pgfpathlineto{\pgfqpoint{-5.673071in}{0.773588in}}%
\pgfpathlineto{\pgfqpoint{-5.623816in}{0.773588in}}%
\pgfpathlineto{\pgfqpoint{-5.574633in}{0.773588in}}%
\pgfpathlineto{\pgfqpoint{-5.523472in}{0.773588in}}%
\pgfpathlineto{\pgfqpoint{-5.473058in}{0.773588in}}%
\pgfpathlineto{\pgfqpoint{-5.421631in}{0.773588in}}%
\pgfpathlineto{\pgfqpoint{-5.369703in}{0.773588in}}%
\pgfpathlineto{\pgfqpoint{-5.319212in}{0.773588in}}%
\pgfpathlineto{\pgfqpoint{-5.267508in}{0.773588in}}%
\pgfpathlineto{\pgfqpoint{-5.215272in}{0.773588in}}%
\pgfpathlineto{\pgfqpoint{-5.165135in}{0.773588in}}%
\pgfpathlineto{\pgfqpoint{-5.114869in}{0.773588in}}%
\pgfpathlineto{\pgfqpoint{-5.064238in}{0.773588in}}%
\pgfpathlineto{\pgfqpoint{-5.014255in}{0.773588in}}%
\pgfpathlineto{\pgfqpoint{-4.963670in}{0.773588in}}%
\pgfpathlineto{\pgfqpoint{-4.911765in}{0.773588in}}%
\pgfpathlineto{\pgfqpoint{-4.861887in}{0.773588in}}%
\pgfpathlineto{\pgfqpoint{-4.812147in}{0.773588in}}%
\pgfpathlineto{\pgfqpoint{-4.759589in}{0.773588in}}%
\pgfpathlineto{\pgfqpoint{-4.709756in}{0.773588in}}%
\pgfpathlineto{\pgfqpoint{-4.659647in}{0.773588in}}%
\pgfpathlineto{\pgfqpoint{-4.607841in}{0.773588in}}%
\pgfpathlineto{\pgfqpoint{-4.558006in}{0.773588in}}%
\pgfpathlineto{\pgfqpoint{-4.508002in}{0.773588in}}%
\pgfpathlineto{\pgfqpoint{-4.457276in}{0.773588in}}%
\pgfpathlineto{\pgfqpoint{-4.407934in}{0.773588in}}%
\pgfpathlineto{\pgfqpoint{-4.357515in}{0.773588in}}%
\pgfpathlineto{\pgfqpoint{-4.305607in}{0.773588in}}%
\pgfpathlineto{\pgfqpoint{-4.255103in}{0.773588in}}%
\pgfpathlineto{\pgfqpoint{-4.204591in}{0.773588in}}%
\pgfpathlineto{\pgfqpoint{-4.153202in}{0.773588in}}%
\pgfpathlineto{\pgfqpoint{-4.103923in}{0.773588in}}%
\pgfpathlineto{\pgfqpoint{-4.053537in}{0.773588in}}%
\pgfpathlineto{\pgfqpoint{-4.002524in}{0.773588in}}%
\pgfpathlineto{\pgfqpoint{-3.952191in}{0.773588in}}%
\pgfpathlineto{\pgfqpoint{-3.902764in}{0.773588in}}%
\pgfpathlineto{\pgfqpoint{-3.850587in}{0.773588in}}%
\pgfpathlineto{\pgfqpoint{-3.799759in}{0.773588in}}%
\pgfpathlineto{\pgfqpoint{-3.748633in}{0.773588in}}%
\pgfpathlineto{\pgfqpoint{-3.696993in}{0.773588in}}%
\pgfpathlineto{\pgfqpoint{-3.646072in}{0.773588in}}%
\pgfpathlineto{\pgfqpoint{-3.595859in}{0.773588in}}%
\pgfpathlineto{\pgfqpoint{-3.544176in}{0.773588in}}%
\pgfpathlineto{\pgfqpoint{-3.494154in}{0.773588in}}%
\pgfpathlineto{\pgfqpoint{-3.444140in}{0.773588in}}%
\pgfpathlineto{\pgfqpoint{-3.392015in}{0.773588in}}%
\pgfpathlineto{\pgfqpoint{-3.341930in}{0.773588in}}%
\pgfpathlineto{\pgfqpoint{-3.292250in}{0.773588in}}%
\pgfpathlineto{\pgfqpoint{-3.241308in}{0.773588in}}%
\pgfpathlineto{\pgfqpoint{-3.190882in}{0.773588in}}%
\pgfpathlineto{\pgfqpoint{-3.140428in}{0.773588in}}%
\pgfpathlineto{\pgfqpoint{-3.088627in}{0.773588in}}%
\pgfpathlineto{\pgfqpoint{-3.039557in}{0.773588in}}%
\pgfpathlineto{\pgfqpoint{-2.990316in}{0.773588in}}%
\pgfpathlineto{\pgfqpoint{-2.938490in}{0.773588in}}%
\pgfpathlineto{\pgfqpoint{-2.887081in}{0.773588in}}%
\pgfpathlineto{\pgfqpoint{-2.836699in}{0.773588in}}%
\pgfpathlineto{\pgfqpoint{-2.784277in}{0.773588in}}%
\pgfpathlineto{\pgfqpoint{-2.732985in}{0.773588in}}%
\pgfpathlineto{\pgfqpoint{-2.681969in}{0.773588in}}%
\pgfpathlineto{\pgfqpoint{-2.629697in}{0.773588in}}%
\pgfpathlineto{\pgfqpoint{-2.578972in}{0.773588in}}%
\pgfpathlineto{\pgfqpoint{-2.528440in}{0.773588in}}%
\pgfpathlineto{\pgfqpoint{-2.476616in}{0.773588in}}%
\pgfpathlineto{\pgfqpoint{-2.425397in}{0.773588in}}%
\pgfpathlineto{\pgfqpoint{-2.375607in}{0.773588in}}%
\pgfpathlineto{\pgfqpoint{-2.323084in}{0.773588in}}%
\pgfpathlineto{\pgfqpoint{-2.272195in}{0.773588in}}%
\pgfpathlineto{\pgfqpoint{-2.220871in}{0.773588in}}%
\pgfpathlineto{\pgfqpoint{-2.167559in}{0.773588in}}%
\pgfpathlineto{\pgfqpoint{-2.116467in}{0.773588in}}%
\pgfpathlineto{\pgfqpoint{-2.064986in}{0.773588in}}%
\pgfpathlineto{\pgfqpoint{-2.013282in}{0.773588in}}%
\pgfpathlineto{\pgfqpoint{-1.963080in}{0.773588in}}%
\pgfpathlineto{\pgfqpoint{-1.912972in}{0.773588in}}%
\pgfpathlineto{\pgfqpoint{-1.860296in}{0.773588in}}%
\pgfpathlineto{\pgfqpoint{-1.809967in}{0.773588in}}%
\pgfpathlineto{\pgfqpoint{-1.759793in}{0.773588in}}%
\pgfpathlineto{\pgfqpoint{-1.707851in}{0.773588in}}%
\pgfpathlineto{\pgfqpoint{-1.657574in}{0.773588in}}%
\pgfpathlineto{\pgfqpoint{-1.606814in}{0.773588in}}%
\pgfpathlineto{\pgfqpoint{-1.553079in}{0.773588in}}%
\pgfpathlineto{\pgfqpoint{-1.502359in}{0.773588in}}%
\pgfpathlineto{\pgfqpoint{-1.451018in}{0.773588in}}%
\pgfpathlineto{\pgfqpoint{-1.397583in}{0.773588in}}%
\pgfpathlineto{\pgfqpoint{-1.346574in}{0.773588in}}%
\pgfpathlineto{\pgfqpoint{-1.295495in}{0.773588in}}%
\pgfpathlineto{\pgfqpoint{-1.242201in}{0.773588in}}%
\pgfpathlineto{\pgfqpoint{-1.191111in}{0.773588in}}%
\pgfpathlineto{\pgfqpoint{-1.140825in}{0.773588in}}%
\pgfpathlineto{\pgfqpoint{-1.088683in}{0.773588in}}%
\pgfpathlineto{\pgfqpoint{-1.038293in}{0.773588in}}%
\pgfpathlineto{\pgfqpoint{-0.987007in}{0.773588in}}%
\pgfpathlineto{\pgfqpoint{-0.934193in}{0.773588in}}%
\pgfpathlineto{\pgfqpoint{-0.883439in}{0.773588in}}%
\pgfpathlineto{\pgfqpoint{-0.832397in}{0.773588in}}%
\pgfpathlineto{\pgfqpoint{-0.780794in}{0.773588in}}%
\pgfpathlineto{\pgfqpoint{-0.729960in}{0.773588in}}%
\pgfpathlineto{\pgfqpoint{-0.679845in}{0.773588in}}%
\pgfpathlineto{\pgfqpoint{-0.628072in}{0.773588in}}%
\pgfpathlineto{\pgfqpoint{-0.577558in}{0.773588in}}%
\pgfpathlineto{\pgfqpoint{-0.525798in}{0.773588in}}%
\pgfpathlineto{\pgfqpoint{-0.472607in}{0.773588in}}%
\pgfpathlineto{\pgfqpoint{-0.421082in}{0.773588in}}%
\pgfpathlineto{\pgfqpoint{-0.370494in}{0.773588in}}%
\pgfpathlineto{\pgfqpoint{-0.317282in}{0.773588in}}%
\pgfpathlineto{\pgfqpoint{-0.265117in}{0.773588in}}%
\pgfpathlineto{\pgfqpoint{-0.212446in}{0.773588in}}%
\pgfpathlineto{\pgfqpoint{-0.159327in}{0.773588in}}%
\pgfpathlineto{\pgfqpoint{-0.106747in}{0.773588in}}%
\pgfpathlineto{\pgfqpoint{-0.053884in}{0.773588in}}%
\pgfpathlineto{\pgfqpoint{0.000028in}{0.773588in}}%
\pgfpathlineto{\pgfqpoint{0.052190in}{0.773588in}}%
\pgfpathlineto{\pgfqpoint{0.103930in}{0.773588in}}%
\pgfpathlineto{\pgfqpoint{0.157615in}{0.773588in}}%
\pgfpathlineto{\pgfqpoint{0.209933in}{0.773588in}}%
\pgfpathlineto{\pgfqpoint{0.261843in}{0.773588in}}%
\pgfpathlineto{\pgfqpoint{0.315264in}{0.773588in}}%
\pgfpathlineto{\pgfqpoint{0.366948in}{0.773588in}}%
\pgfpathlineto{\pgfqpoint{0.419063in}{0.773588in}}%
\pgfpathlineto{\pgfqpoint{0.472767in}{0.773588in}}%
\pgfpathlineto{\pgfqpoint{0.524412in}{0.773588in}}%
\pgfpathlineto{\pgfqpoint{0.575921in}{0.773588in}}%
\pgfpathlineto{\pgfqpoint{0.628822in}{0.773588in}}%
\pgfpathlineto{\pgfqpoint{0.680101in}{0.773588in}}%
\pgfpathlineto{\pgfqpoint{0.730937in}{0.773588in}}%
\pgfpathlineto{\pgfqpoint{0.783351in}{0.773588in}}%
\pgfpathlineto{\pgfqpoint{0.835638in}{0.773588in}}%
\pgfpathlineto{\pgfqpoint{0.890559in}{0.773588in}}%
\pgfpathlineto{\pgfqpoint{0.950896in}{0.773588in}}%
\pgfpathlineto{\pgfqpoint{1.011418in}{0.773588in}}%
\pgfpathlineto{\pgfqpoint{1.071223in}{0.773588in}}%
\pgfpathlineto{\pgfqpoint{1.133818in}{0.773588in}}%
\pgfpathlineto{\pgfqpoint{1.196854in}{0.773588in}}%
\pgfpathlineto{\pgfqpoint{1.262232in}{0.773588in}}%
\pgfpathlineto{\pgfqpoint{1.331848in}{0.773588in}}%
\pgfpathlineto{\pgfqpoint{1.399923in}{0.773588in}}%
\pgfpathlineto{\pgfqpoint{1.469539in}{0.773588in}}%
\pgfpathlineto{\pgfqpoint{1.542225in}{0.773588in}}%
\pgfpathlineto{\pgfqpoint{1.614592in}{0.773588in}}%
\pgfpathlineto{\pgfqpoint{1.687208in}{0.773588in}}%
\pgfpathlineto{\pgfqpoint{1.764147in}{0.773588in}}%
\pgfpathlineto{\pgfqpoint{1.839330in}{0.773588in}}%
\pgfpathlineto{\pgfqpoint{1.918401in}{0.773588in}}%
\pgfpathlineto{\pgfqpoint{1.999401in}{0.773588in}}%
\pgfpathlineto{\pgfqpoint{2.077252in}{0.773588in}}%
\pgfpathlineto{\pgfqpoint{2.157128in}{0.773588in}}%
\pgfpathlineto{\pgfqpoint{2.243381in}{0.773588in}}%
\pgfpathlineto{\pgfqpoint{2.328903in}{0.773588in}}%
\pgfpathlineto{\pgfqpoint{2.416579in}{0.773588in}}%
\pgfpathlineto{\pgfqpoint{2.504252in}{0.773588in}}%
\pgfpathlineto{\pgfqpoint{2.589494in}{0.773588in}}%
\pgfpathlineto{\pgfqpoint{2.676066in}{0.773588in}}%
\pgfpathlineto{\pgfqpoint{2.767061in}{0.773588in}}%
\pgfpathlineto{\pgfqpoint{2.858701in}{0.773588in}}%
\pgfpathlineto{\pgfqpoint{2.948242in}{0.773588in}}%
\pgfpathlineto{\pgfqpoint{3.043068in}{0.773588in}}%
\pgfpathlineto{\pgfqpoint{3.132459in}{0.773588in}}%
\pgfpathlineto{\pgfqpoint{3.198420in}{0.773588in}}%
\pgfpathlineto{\pgfqpoint{3.252590in}{0.773588in}}%
\pgfpathlineto{\pgfqpoint{3.305067in}{0.773588in}}%
\pgfpathlineto{\pgfqpoint{3.357008in}{0.773588in}}%
\pgfpathlineto{\pgfqpoint{3.411084in}{0.773588in}}%
\pgfpathlineto{\pgfqpoint{3.464430in}{0.773588in}}%
\pgfpathlineto{\pgfqpoint{3.517522in}{0.773588in}}%
\pgfpathlineto{\pgfqpoint{3.571722in}{0.773588in}}%
\pgfpathlineto{\pgfqpoint{3.624008in}{0.773588in}}%
\pgfpathlineto{\pgfqpoint{3.676588in}{0.773588in}}%
\pgfpathlineto{\pgfqpoint{3.729373in}{0.773588in}}%
\pgfpathlineto{\pgfqpoint{3.781554in}{0.773588in}}%
\pgfpathlineto{\pgfqpoint{3.833685in}{0.773588in}}%
\pgfpathlineto{\pgfqpoint{3.887639in}{0.773588in}}%
\pgfpathlineto{\pgfqpoint{3.940507in}{0.773588in}}%
\pgfpathlineto{\pgfqpoint{3.993337in}{0.773588in}}%
\pgfpathlineto{\pgfqpoint{4.047138in}{0.773588in}}%
\pgfpathlineto{\pgfqpoint{4.098630in}{0.773588in}}%
\pgfpathlineto{\pgfqpoint{4.150734in}{0.773588in}}%
\pgfpathlineto{\pgfqpoint{4.204360in}{0.773588in}}%
\pgfpathlineto{\pgfqpoint{4.256228in}{0.773588in}}%
\pgfpathlineto{\pgfqpoint{4.307535in}{0.773588in}}%
\pgfpathlineto{\pgfqpoint{4.360055in}{0.773588in}}%
\pgfpathlineto{\pgfqpoint{4.398305in}{0.773588in}}%
\pgfpathlineto{\pgfqpoint{4.446119in}{0.773588in}}%
\pgfpathlineto{\pgfqpoint{4.484596in}{1.454760in}}%
\pgfpathlineto{\pgfqpoint{4.526846in}{1.669618in}}%
\pgfpathlineto{\pgfqpoint{4.566686in}{1.906942in}}%
\pgfpathlineto{\pgfqpoint{4.603496in}{2.224563in}}%
\pgfpathlineto{\pgfqpoint{4.635492in}{2.864879in}}%
\pgfpathlineto{\pgfqpoint{4.666016in}{3.729137in}}%
\pgfpathlineto{\pgfqpoint{4.690368in}{5.235025in}}%
\pgfpathlineto{\pgfqpoint{4.715513in}{5.182737in}}%
\pgfpathlineto{\pgfqpoint{4.739291in}{5.364631in}}%
\pgfpathlineto{\pgfqpoint{4.764034in}{5.483715in}}%
\pgfpathlineto{\pgfqpoint{4.787716in}{5.446238in}}%
\pgfpathlineto{\pgfqpoint{4.811295in}{5.442565in}}%
\pgfpathlineto{\pgfqpoint{4.836687in}{5.484292in}}%
\pgfpathlineto{\pgfqpoint{4.860218in}{5.534473in}}%
\pgfpathlineto{\pgfqpoint{4.884852in}{5.460686in}}%
\pgfpathlineto{\pgfqpoint{4.907640in}{5.518952in}}%
\pgfpathlineto{\pgfqpoint{4.931890in}{5.678338in}}%
\pgfpathlineto{\pgfqpoint{4.954839in}{5.584654in}}%
\pgfpathlineto{\pgfqpoint{4.980174in}{5.504363in}}%
\pgfpathlineto{\pgfqpoint{5.002737in}{5.557546in}}%
\pgfpathlineto{\pgfqpoint{5.026810in}{5.442515in}}%
\pgfpathlineto{\pgfqpoint{5.051612in}{5.582839in}}%
\pgfpathlineto{\pgfqpoint{5.074798in}{5.834541in}}%
\pgfpathlineto{\pgfqpoint{5.097977in}{5.740043in}}%
\pgfpathlineto{\pgfqpoint{5.122448in}{5.601241in}}%
\pgfpathlineto{\pgfqpoint{5.145219in}{5.809148in}}%
\pgfpathlineto{\pgfqpoint{5.168068in}{5.745325in}}%
\pgfpathlineto{\pgfqpoint{5.192095in}{5.846668in}}%
\pgfpathlineto{\pgfqpoint{5.214897in}{5.862622in}}%
\pgfpathlineto{\pgfqpoint{5.237842in}{5.791091in}}%
\pgfpathlineto{\pgfqpoint{5.261861in}{5.744292in}}%
\pgfpathlineto{\pgfqpoint{5.285086in}{5.706485in}}%
\pgfpathlineto{\pgfqpoint{5.307824in}{5.908282in}}%
\pgfpathlineto{\pgfqpoint{5.331834in}{5.813609in}}%
\pgfpathlineto{\pgfqpoint{5.355254in}{5.854178in}}%
\pgfpathlineto{\pgfqpoint{5.377738in}{5.918658in}}%
\pgfpathlineto{\pgfqpoint{5.402166in}{5.815038in}}%
\pgfpathlineto{\pgfqpoint{5.424187in}{5.868196in}}%
\pgfpathlineto{\pgfqpoint{5.448001in}{5.857730in}}%
\pgfpathlineto{\pgfqpoint{5.470431in}{5.911984in}}%
\pgfpathlineto{\pgfqpoint{5.494250in}{5.893046in}}%
\pgfpathlineto{\pgfqpoint{5.516676in}{5.880249in}}%
\pgfpathlineto{\pgfqpoint{5.540659in}{5.899825in}}%
\pgfpathlineto{\pgfqpoint{5.562835in}{5.930845in}}%
\pgfpathlineto{\pgfqpoint{5.587139in}{5.726189in}}%
\pgfpathlineto{\pgfqpoint{5.609949in}{5.825752in}}%
\pgfpathlineto{\pgfqpoint{5.634039in}{5.865058in}}%
\pgfpathlineto{\pgfqpoint{5.661246in}{5.775449in}}%
\pgfpathlineto{\pgfqpoint{5.712060in}{5.758501in}}%
\pgfpathlineto{\pgfqpoint{5.763148in}{5.758501in}}%
\pgfpathlineto{\pgfqpoint{5.815025in}{5.758501in}}%
\pgfpathlineto{\pgfqpoint{5.867920in}{5.758501in}}%
\pgfpathlineto{\pgfqpoint{5.919631in}{5.758501in}}%
\pgfpathlineto{\pgfqpoint{5.971565in}{5.758501in}}%
\pgfpathlineto{\pgfqpoint{6.025764in}{5.758501in}}%
\pgfpathlineto{\pgfqpoint{6.078797in}{5.758501in}}%
\pgfpathlineto{\pgfqpoint{6.078797in}{5.758501in}}%
\pgfpathlineto{\pgfqpoint{6.078797in}{5.758501in}}%
\pgfpathlineto{\pgfqpoint{6.025764in}{5.758501in}}%
\pgfpathlineto{\pgfqpoint{5.971565in}{5.758501in}}%
\pgfpathlineto{\pgfqpoint{5.919631in}{5.758501in}}%
\pgfpathlineto{\pgfqpoint{5.867920in}{5.758501in}}%
\pgfpathlineto{\pgfqpoint{5.815025in}{5.758501in}}%
\pgfpathlineto{\pgfqpoint{5.763148in}{5.758501in}}%
\pgfpathlineto{\pgfqpoint{5.712060in}{5.758501in}}%
\pgfpathlineto{\pgfqpoint{5.661246in}{5.775449in}}%
\pgfpathlineto{\pgfqpoint{5.634039in}{5.865058in}}%
\pgfpathlineto{\pgfqpoint{5.609949in}{5.825752in}}%
\pgfpathlineto{\pgfqpoint{5.587139in}{5.726189in}}%
\pgfpathlineto{\pgfqpoint{5.562835in}{5.930845in}}%
\pgfpathlineto{\pgfqpoint{5.540659in}{5.899825in}}%
\pgfpathlineto{\pgfqpoint{5.516676in}{5.880249in}}%
\pgfpathlineto{\pgfqpoint{5.494250in}{5.893046in}}%
\pgfpathlineto{\pgfqpoint{5.470431in}{5.911984in}}%
\pgfpathlineto{\pgfqpoint{5.448001in}{5.857730in}}%
\pgfpathlineto{\pgfqpoint{5.424187in}{5.868196in}}%
\pgfpathlineto{\pgfqpoint{5.402166in}{5.815038in}}%
\pgfpathlineto{\pgfqpoint{5.377738in}{5.918658in}}%
\pgfpathlineto{\pgfqpoint{5.355254in}{5.854178in}}%
\pgfpathlineto{\pgfqpoint{5.331834in}{5.813609in}}%
\pgfpathlineto{\pgfqpoint{5.307824in}{5.908282in}}%
\pgfpathlineto{\pgfqpoint{5.285086in}{5.706485in}}%
\pgfpathlineto{\pgfqpoint{5.261861in}{5.744292in}}%
\pgfpathlineto{\pgfqpoint{5.237842in}{5.791091in}}%
\pgfpathlineto{\pgfqpoint{5.214897in}{5.862622in}}%
\pgfpathlineto{\pgfqpoint{5.192095in}{5.846668in}}%
\pgfpathlineto{\pgfqpoint{5.168068in}{5.745325in}}%
\pgfpathlineto{\pgfqpoint{5.145219in}{5.809148in}}%
\pgfpathlineto{\pgfqpoint{5.122448in}{5.601241in}}%
\pgfpathlineto{\pgfqpoint{5.097977in}{5.740043in}}%
\pgfpathlineto{\pgfqpoint{5.074798in}{5.834541in}}%
\pgfpathlineto{\pgfqpoint{5.051612in}{5.582839in}}%
\pgfpathlineto{\pgfqpoint{5.026810in}{5.442515in}}%
\pgfpathlineto{\pgfqpoint{5.002737in}{5.557546in}}%
\pgfpathlineto{\pgfqpoint{4.980174in}{5.504363in}}%
\pgfpathlineto{\pgfqpoint{4.954839in}{5.584654in}}%
\pgfpathlineto{\pgfqpoint{4.931890in}{5.678338in}}%
\pgfpathlineto{\pgfqpoint{4.907640in}{5.518952in}}%
\pgfpathlineto{\pgfqpoint{4.884852in}{5.460686in}}%
\pgfpathlineto{\pgfqpoint{4.860218in}{5.534473in}}%
\pgfpathlineto{\pgfqpoint{4.836687in}{5.484292in}}%
\pgfpathlineto{\pgfqpoint{4.811295in}{5.442565in}}%
\pgfpathlineto{\pgfqpoint{4.787716in}{5.446238in}}%
\pgfpathlineto{\pgfqpoint{4.764034in}{5.483715in}}%
\pgfpathlineto{\pgfqpoint{4.739291in}{5.364631in}}%
\pgfpathlineto{\pgfqpoint{4.715513in}{5.182737in}}%
\pgfpathlineto{\pgfqpoint{4.690368in}{5.235025in}}%
\pgfpathlineto{\pgfqpoint{4.666016in}{3.729137in}}%
\pgfpathlineto{\pgfqpoint{4.635492in}{2.864879in}}%
\pgfpathlineto{\pgfqpoint{4.603496in}{2.224563in}}%
\pgfpathlineto{\pgfqpoint{4.566686in}{1.906942in}}%
\pgfpathlineto{\pgfqpoint{4.526846in}{1.669618in}}%
\pgfpathlineto{\pgfqpoint{4.484596in}{1.454760in}}%
\pgfpathlineto{\pgfqpoint{4.446119in}{0.773588in}}%
\pgfpathlineto{\pgfqpoint{4.398305in}{0.773588in}}%
\pgfpathlineto{\pgfqpoint{4.360055in}{0.773588in}}%
\pgfpathlineto{\pgfqpoint{4.307535in}{0.773588in}}%
\pgfpathlineto{\pgfqpoint{4.256228in}{0.773588in}}%
\pgfpathlineto{\pgfqpoint{4.204360in}{0.773588in}}%
\pgfpathlineto{\pgfqpoint{4.150734in}{0.773588in}}%
\pgfpathlineto{\pgfqpoint{4.098630in}{0.773588in}}%
\pgfpathlineto{\pgfqpoint{4.047138in}{0.773588in}}%
\pgfpathlineto{\pgfqpoint{3.993337in}{0.773588in}}%
\pgfpathlineto{\pgfqpoint{3.940507in}{0.773588in}}%
\pgfpathlineto{\pgfqpoint{3.887639in}{0.773588in}}%
\pgfpathlineto{\pgfqpoint{3.833685in}{1.230236in}}%
\pgfpathlineto{\pgfqpoint{3.781554in}{1.423144in}}%
\pgfpathlineto{\pgfqpoint{3.729373in}{1.488284in}}%
\pgfpathlineto{\pgfqpoint{3.676588in}{1.414323in}}%
\pgfpathlineto{\pgfqpoint{3.624008in}{1.429618in}}%
\pgfpathlineto{\pgfqpoint{3.571722in}{1.459985in}}%
\pgfpathlineto{\pgfqpoint{3.517522in}{1.442369in}}%
\pgfpathlineto{\pgfqpoint{3.464430in}{1.393664in}}%
\pgfpathlineto{\pgfqpoint{3.411084in}{1.405600in}}%
\pgfpathlineto{\pgfqpoint{3.357008in}{1.453492in}}%
\pgfpathlineto{\pgfqpoint{3.305067in}{1.465799in}}%
\pgfpathlineto{\pgfqpoint{3.252590in}{1.466197in}}%
\pgfpathlineto{\pgfqpoint{3.198420in}{1.438877in}}%
\pgfpathlineto{\pgfqpoint{3.132459in}{1.197335in}}%
\pgfpathlineto{\pgfqpoint{3.043068in}{1.153535in}}%
\pgfpathlineto{\pgfqpoint{2.948242in}{1.189047in}}%
\pgfpathlineto{\pgfqpoint{2.858701in}{1.172160in}}%
\pgfpathlineto{\pgfqpoint{2.767061in}{1.177862in}}%
\pgfpathlineto{\pgfqpoint{2.676066in}{1.165488in}}%
\pgfpathlineto{\pgfqpoint{2.589494in}{1.193878in}}%
\pgfpathlineto{\pgfqpoint{2.504252in}{1.213882in}}%
\pgfpathlineto{\pgfqpoint{2.416579in}{1.145304in}}%
\pgfpathlineto{\pgfqpoint{2.328903in}{1.194231in}}%
\pgfpathlineto{\pgfqpoint{2.243381in}{1.200709in}}%
\pgfpathlineto{\pgfqpoint{2.157128in}{1.177079in}}%
\pgfpathlineto{\pgfqpoint{2.077252in}{1.226639in}}%
\pgfpathlineto{\pgfqpoint{1.999401in}{1.228932in}}%
\pgfpathlineto{\pgfqpoint{1.918401in}{1.197516in}}%
\pgfpathlineto{\pgfqpoint{1.839330in}{1.245446in}}%
\pgfpathlineto{\pgfqpoint{1.764147in}{1.234987in}}%
\pgfpathlineto{\pgfqpoint{1.687208in}{1.239497in}}%
\pgfpathlineto{\pgfqpoint{1.614592in}{1.269004in}}%
\pgfpathlineto{\pgfqpoint{1.542225in}{1.248520in}}%
\pgfpathlineto{\pgfqpoint{1.469539in}{1.273397in}}%
\pgfpathlineto{\pgfqpoint{1.399923in}{1.299568in}}%
\pgfpathlineto{\pgfqpoint{1.331848in}{1.249212in}}%
\pgfpathlineto{\pgfqpoint{1.262232in}{1.321329in}}%
\pgfpathlineto{\pgfqpoint{1.196854in}{1.305125in}}%
\pgfpathlineto{\pgfqpoint{1.133818in}{1.333525in}}%
\pgfpathlineto{\pgfqpoint{1.071223in}{1.372779in}}%
\pgfpathlineto{\pgfqpoint{1.011418in}{1.326296in}}%
\pgfpathlineto{\pgfqpoint{0.950896in}{1.372834in}}%
\pgfpathlineto{\pgfqpoint{0.890559in}{1.374681in}}%
\pgfpathlineto{\pgfqpoint{0.835638in}{1.411341in}}%
\pgfpathlineto{\pgfqpoint{0.783351in}{1.447263in}}%
\pgfpathlineto{\pgfqpoint{0.730937in}{1.447685in}}%
\pgfpathlineto{\pgfqpoint{0.680101in}{1.464706in}}%
\pgfpathlineto{\pgfqpoint{0.628822in}{1.496976in}}%
\pgfpathlineto{\pgfqpoint{0.575921in}{1.466086in}}%
\pgfpathlineto{\pgfqpoint{0.524412in}{1.450867in}}%
\pgfpathlineto{\pgfqpoint{0.472767in}{1.411402in}}%
\pgfpathlineto{\pgfqpoint{0.419063in}{1.447695in}}%
\pgfpathlineto{\pgfqpoint{0.366948in}{1.467650in}}%
\pgfpathlineto{\pgfqpoint{0.315264in}{1.464372in}}%
\pgfpathlineto{\pgfqpoint{0.261843in}{1.476625in}}%
\pgfpathlineto{\pgfqpoint{0.209933in}{1.401838in}}%
\pgfpathlineto{\pgfqpoint{0.157615in}{1.435204in}}%
\pgfpathlineto{\pgfqpoint{0.103930in}{1.448577in}}%
\pgfpathlineto{\pgfqpoint{0.052190in}{1.450584in}}%
\pgfpathlineto{\pgfqpoint{0.000028in}{1.474166in}}%
\pgfpathlineto{\pgfqpoint{-0.053884in}{1.459502in}}%
\pgfpathlineto{\pgfqpoint{-0.106747in}{1.429412in}}%
\pgfpathlineto{\pgfqpoint{-0.159327in}{1.480796in}}%
\pgfpathlineto{\pgfqpoint{-0.212446in}{1.453336in}}%
\pgfpathlineto{\pgfqpoint{-0.265117in}{1.396378in}}%
\pgfpathlineto{\pgfqpoint{-0.317282in}{1.466339in}}%
\pgfpathlineto{\pgfqpoint{-0.370494in}{1.498293in}}%
\pgfpathlineto{\pgfqpoint{-0.421082in}{1.466028in}}%
\pgfpathlineto{\pgfqpoint{-0.472607in}{1.422930in}}%
\pgfpathlineto{\pgfqpoint{-0.525798in}{1.440938in}}%
\pgfpathlineto{\pgfqpoint{-0.577558in}{1.493059in}}%
\pgfpathlineto{\pgfqpoint{-0.628072in}{1.469018in}}%
\pgfpathlineto{\pgfqpoint{-0.679845in}{1.488245in}}%
\pgfpathlineto{\pgfqpoint{-0.729960in}{1.495303in}}%
\pgfpathlineto{\pgfqpoint{-0.780794in}{1.454000in}}%
\pgfpathlineto{\pgfqpoint{-0.832397in}{1.468684in}}%
\pgfpathlineto{\pgfqpoint{-0.883439in}{1.476075in}}%
\pgfpathlineto{\pgfqpoint{-0.934193in}{1.413812in}}%
\pgfpathlineto{\pgfqpoint{-0.987007in}{1.435521in}}%
\pgfpathlineto{\pgfqpoint{-1.038293in}{1.472000in}}%
\pgfpathlineto{\pgfqpoint{-1.088683in}{1.516530in}}%
\pgfpathlineto{\pgfqpoint{-1.140825in}{1.463950in}}%
\pgfpathlineto{\pgfqpoint{-1.191111in}{1.465843in}}%
\pgfpathlineto{\pgfqpoint{-1.242201in}{1.437680in}}%
\pgfpathlineto{\pgfqpoint{-1.295495in}{1.466132in}}%
\pgfpathlineto{\pgfqpoint{-1.346574in}{1.470166in}}%
\pgfpathlineto{\pgfqpoint{-1.397583in}{1.463974in}}%
\pgfpathlineto{\pgfqpoint{-1.451018in}{1.426934in}}%
\pgfpathlineto{\pgfqpoint{-1.502359in}{1.447905in}}%
\pgfpathlineto{\pgfqpoint{-1.553079in}{1.430942in}}%
\pgfpathlineto{\pgfqpoint{-1.606814in}{1.453054in}}%
\pgfpathlineto{\pgfqpoint{-1.657574in}{1.497292in}}%
\pgfpathlineto{\pgfqpoint{-1.707851in}{1.439046in}}%
\pgfpathlineto{\pgfqpoint{-1.759793in}{1.461354in}}%
\pgfpathlineto{\pgfqpoint{-1.809967in}{1.467419in}}%
\pgfpathlineto{\pgfqpoint{-1.860296in}{1.433163in}}%
\pgfpathlineto{\pgfqpoint{-1.912972in}{1.466522in}}%
\pgfpathlineto{\pgfqpoint{-1.963080in}{1.471617in}}%
\pgfpathlineto{\pgfqpoint{-2.013282in}{1.514552in}}%
\pgfpathlineto{\pgfqpoint{-2.064986in}{1.441672in}}%
\pgfpathlineto{\pgfqpoint{-2.116467in}{1.440422in}}%
\pgfpathlineto{\pgfqpoint{-2.167559in}{1.453594in}}%
\pgfpathlineto{\pgfqpoint{-2.220871in}{1.437036in}}%
\pgfpathlineto{\pgfqpoint{-2.272195in}{1.494852in}}%
\pgfpathlineto{\pgfqpoint{-2.323084in}{1.476219in}}%
\pgfpathlineto{\pgfqpoint{-2.375607in}{1.518800in}}%
\pgfpathlineto{\pgfqpoint{-2.425397in}{1.481435in}}%
\pgfpathlineto{\pgfqpoint{-2.476616in}{1.498993in}}%
\pgfpathlineto{\pgfqpoint{-2.528440in}{1.441237in}}%
\pgfpathlineto{\pgfqpoint{-2.578972in}{1.454026in}}%
\pgfpathlineto{\pgfqpoint{-2.629697in}{1.466846in}}%
\pgfpathlineto{\pgfqpoint{-2.681969in}{1.474687in}}%
\pgfpathlineto{\pgfqpoint{-2.732985in}{1.477742in}}%
\pgfpathlineto{\pgfqpoint{-2.784277in}{1.441924in}}%
\pgfpathlineto{\pgfqpoint{-2.836699in}{1.488861in}}%
\pgfpathlineto{\pgfqpoint{-2.887081in}{1.401061in}}%
\pgfpathlineto{\pgfqpoint{-2.938490in}{1.489283in}}%
\pgfpathlineto{\pgfqpoint{-2.990316in}{1.491356in}}%
\pgfpathlineto{\pgfqpoint{-3.039557in}{1.496304in}}%
\pgfpathlineto{\pgfqpoint{-3.088627in}{1.464632in}}%
\pgfpathlineto{\pgfqpoint{-3.140428in}{1.478346in}}%
\pgfpathlineto{\pgfqpoint{-3.190882in}{1.454513in}}%
\pgfpathlineto{\pgfqpoint{-3.241308in}{1.485889in}}%
\pgfpathlineto{\pgfqpoint{-3.292250in}{1.464323in}}%
\pgfpathlineto{\pgfqpoint{-3.341930in}{1.462308in}}%
\pgfpathlineto{\pgfqpoint{-3.392015in}{1.499292in}}%
\pgfpathlineto{\pgfqpoint{-3.444140in}{1.461304in}}%
\pgfpathlineto{\pgfqpoint{-3.494154in}{1.472900in}}%
\pgfpathlineto{\pgfqpoint{-3.544176in}{1.510289in}}%
\pgfpathlineto{\pgfqpoint{-3.595859in}{1.509874in}}%
\pgfpathlineto{\pgfqpoint{-3.646072in}{1.416044in}}%
\pgfpathlineto{\pgfqpoint{-3.696993in}{1.431142in}}%
\pgfpathlineto{\pgfqpoint{-3.748633in}{1.447750in}}%
\pgfpathlineto{\pgfqpoint{-3.799759in}{1.430021in}}%
\pgfpathlineto{\pgfqpoint{-3.850587in}{1.460130in}}%
\pgfpathlineto{\pgfqpoint{-3.902764in}{1.476461in}}%
\pgfpathlineto{\pgfqpoint{-3.952191in}{1.479729in}}%
\pgfpathlineto{\pgfqpoint{-4.002524in}{1.507858in}}%
\pgfpathlineto{\pgfqpoint{-4.053537in}{1.455526in}}%
\pgfpathlineto{\pgfqpoint{-4.103923in}{1.509514in}}%
\pgfpathlineto{\pgfqpoint{-4.153202in}{1.492010in}}%
\pgfpathlineto{\pgfqpoint{-4.204591in}{1.480570in}}%
\pgfpathlineto{\pgfqpoint{-4.255103in}{1.483945in}}%
\pgfpathlineto{\pgfqpoint{-4.305607in}{1.468689in}}%
\pgfpathlineto{\pgfqpoint{-4.357515in}{1.453886in}}%
\pgfpathlineto{\pgfqpoint{-4.407934in}{1.501248in}}%
\pgfpathlineto{\pgfqpoint{-4.457276in}{1.511209in}}%
\pgfpathlineto{\pgfqpoint{-4.508002in}{1.454865in}}%
\pgfpathlineto{\pgfqpoint{-4.558006in}{1.515156in}}%
\pgfpathlineto{\pgfqpoint{-4.607841in}{1.480893in}}%
\pgfpathlineto{\pgfqpoint{-4.659647in}{1.497794in}}%
\pgfpathlineto{\pgfqpoint{-4.709756in}{1.469206in}}%
\pgfpathlineto{\pgfqpoint{-4.759589in}{1.429175in}}%
\pgfpathlineto{\pgfqpoint{-4.812147in}{1.488680in}}%
\pgfpathlineto{\pgfqpoint{-4.861887in}{1.479312in}}%
\pgfpathlineto{\pgfqpoint{-4.911765in}{1.459339in}}%
\pgfpathlineto{\pgfqpoint{-4.963670in}{1.471589in}}%
\pgfpathlineto{\pgfqpoint{-5.014255in}{1.479833in}}%
\pgfpathlineto{\pgfqpoint{-5.064238in}{1.496870in}}%
\pgfpathlineto{\pgfqpoint{-5.114869in}{1.436571in}}%
\pgfpathlineto{\pgfqpoint{-5.165135in}{1.466875in}}%
\pgfpathlineto{\pgfqpoint{-5.215272in}{1.424458in}}%
\pgfpathlineto{\pgfqpoint{-5.267508in}{1.404869in}}%
\pgfpathlineto{\pgfqpoint{-5.319212in}{1.486397in}}%
\pgfpathlineto{\pgfqpoint{-5.369703in}{1.474912in}}%
\pgfpathlineto{\pgfqpoint{-5.421631in}{1.429663in}}%
\pgfpathlineto{\pgfqpoint{-5.473058in}{1.471046in}}%
\pgfpathlineto{\pgfqpoint{-5.523472in}{1.503511in}}%
\pgfpathlineto{\pgfqpoint{-5.574633in}{1.487867in}}%
\pgfpathlineto{\pgfqpoint{-5.623816in}{1.518819in}}%
\pgfpathlineto{\pgfqpoint{-5.673071in}{1.492267in}}%
\pgfpathlineto{\pgfqpoint{-5.724570in}{1.457956in}}%
\pgfpathlineto{\pgfqpoint{-5.774223in}{1.502550in}}%
\pgfpathlineto{\pgfqpoint{-5.822716in}{1.484534in}}%
\pgfpathlineto{\pgfqpoint{-5.874278in}{1.483293in}}%
\pgfpathlineto{\pgfqpoint{-5.924128in}{1.472289in}}%
\pgfpathlineto{\pgfqpoint{-5.974236in}{1.458922in}}%
\pgfpathlineto{\pgfqpoint{-6.024836in}{1.464609in}}%
\pgfpathlineto{\pgfqpoint{-6.075227in}{1.478442in}}%
\pgfpathlineto{\pgfqpoint{-6.125420in}{1.491507in}}%
\pgfpathlineto{\pgfqpoint{-6.177451in}{1.458867in}}%
\pgfpathlineto{\pgfqpoint{-6.228389in}{1.437185in}}%
\pgfpathlineto{\pgfqpoint{-6.279011in}{1.431065in}}%
\pgfpathlineto{\pgfqpoint{-6.330959in}{1.520456in}}%
\pgfpathlineto{\pgfqpoint{-6.380124in}{1.506808in}}%
\pgfpathlineto{\pgfqpoint{-6.429787in}{1.491616in}}%
\pgfpathlineto{\pgfqpoint{-6.481910in}{1.432232in}}%
\pgfpathlineto{\pgfqpoint{-6.532179in}{1.462154in}}%
\pgfpathlineto{\pgfqpoint{-6.582317in}{1.519235in}}%
\pgfpathlineto{\pgfqpoint{-6.634445in}{1.515947in}}%
\pgfpathlineto{\pgfqpoint{-6.682747in}{1.564313in}}%
\pgfpathlineto{\pgfqpoint{-6.731323in}{1.481284in}}%
\pgfpathlineto{\pgfqpoint{-6.782488in}{1.492179in}}%
\pgfpathlineto{\pgfqpoint{-6.832072in}{1.477021in}}%
\pgfpathlineto{\pgfqpoint{-6.882593in}{1.489534in}}%
\pgfpathlineto{\pgfqpoint{-6.934169in}{1.474127in}}%
\pgfpathlineto{\pgfqpoint{-6.984061in}{1.439666in}}%
\pgfpathlineto{\pgfqpoint{-7.033318in}{1.468034in}}%
\pgfpathlineto{\pgfqpoint{-7.085212in}{1.455792in}}%
\pgfpathlineto{\pgfqpoint{-7.135511in}{1.474893in}}%
\pgfpathlineto{\pgfqpoint{-7.185558in}{1.451044in}}%
\pgfpathlineto{\pgfqpoint{-7.237328in}{1.527757in}}%
\pgfpathlineto{\pgfqpoint{-7.286238in}{1.528125in}}%
\pgfpathlineto{\pgfqpoint{-7.335791in}{1.462006in}}%
\pgfpathlineto{\pgfqpoint{-7.386343in}{1.514535in}}%
\pgfpathlineto{\pgfqpoint{-7.435602in}{1.497388in}}%
\pgfpathlineto{\pgfqpoint{-7.485312in}{1.484586in}}%
\pgfpathlineto{\pgfqpoint{-7.536566in}{1.518978in}}%
\pgfpathlineto{\pgfqpoint{-7.585672in}{1.481533in}}%
\pgfpathlineto{\pgfqpoint{-7.634521in}{1.485587in}}%
\pgfpathlineto{\pgfqpoint{-7.685281in}{1.492572in}}%
\pgfpathlineto{\pgfqpoint{-7.733870in}{1.486753in}}%
\pgfpathlineto{\pgfqpoint{-7.783607in}{1.472393in}}%
\pgfpathlineto{\pgfqpoint{-7.835188in}{1.489803in}}%
\pgfpathlineto{\pgfqpoint{-7.885122in}{1.472200in}}%
\pgfpathlineto{\pgfqpoint{-7.935315in}{1.462158in}}%
\pgfpathlineto{\pgfqpoint{-7.986412in}{1.454686in}}%
\pgfpathlineto{\pgfqpoint{-8.035792in}{1.486772in}}%
\pgfpathlineto{\pgfqpoint{-8.085190in}{1.450986in}}%
\pgfpathlineto{\pgfqpoint{-8.135490in}{1.492178in}}%
\pgfpathlineto{\pgfqpoint{-8.184462in}{1.483018in}}%
\pgfpathlineto{\pgfqpoint{-8.233427in}{1.495756in}}%
\pgfpathlineto{\pgfqpoint{-8.283775in}{1.514168in}}%
\pgfpathlineto{\pgfqpoint{-8.332630in}{1.437617in}}%
\pgfpathlineto{\pgfqpoint{-8.382452in}{1.450949in}}%
\pgfpathlineto{\pgfqpoint{-8.432560in}{1.516103in}}%
\pgfpathlineto{\pgfqpoint{-8.481981in}{1.466930in}}%
\pgfpathlineto{\pgfqpoint{-8.532027in}{1.455757in}}%
\pgfpathlineto{\pgfqpoint{-8.583181in}{1.588274in}}%
\pgfpathlineto{\pgfqpoint{-8.632588in}{1.504972in}}%
\pgfpathlineto{\pgfqpoint{-8.681529in}{1.507557in}}%
\pgfpathlineto{\pgfqpoint{-8.731477in}{1.483979in}}%
\pgfpathlineto{\pgfqpoint{-8.780578in}{1.503138in}}%
\pgfpathlineto{\pgfqpoint{-8.829855in}{1.472967in}}%
\pgfpathlineto{\pgfqpoint{-8.880937in}{1.523941in}}%
\pgfpathlineto{\pgfqpoint{-8.930541in}{1.459393in}}%
\pgfpathlineto{\pgfqpoint{-8.979731in}{1.508474in}}%
\pgfpathlineto{\pgfqpoint{-9.030683in}{1.455924in}}%
\pgfpathlineto{\pgfqpoint{-9.080221in}{1.483280in}}%
\pgfpathlineto{\pgfqpoint{-9.130552in}{1.453003in}}%
\pgfpathlineto{\pgfqpoint{-9.182531in}{1.498954in}}%
\pgfpathlineto{\pgfqpoint{-9.232466in}{1.475407in}}%
\pgfpathlineto{\pgfqpoint{-9.281795in}{1.450999in}}%
\pgfpathlineto{\pgfqpoint{-9.332833in}{1.505370in}}%
\pgfpathlineto{\pgfqpoint{-9.382453in}{1.477769in}}%
\pgfpathlineto{\pgfqpoint{-9.430893in}{1.496075in}}%
\pgfpathlineto{\pgfqpoint{-9.479530in}{1.532652in}}%
\pgfpathlineto{\pgfqpoint{-9.527213in}{1.513257in}}%
\pgfpathlineto{\pgfqpoint{-9.575433in}{1.497489in}}%
\pgfpathlineto{\pgfqpoint{-9.624467in}{1.504500in}}%
\pgfpathlineto{\pgfqpoint{-9.671880in}{1.525351in}}%
\pgfpathlineto{\pgfqpoint{-9.719205in}{1.511122in}}%
\pgfpathlineto{\pgfqpoint{-9.768438in}{1.483391in}}%
\pgfpathlineto{\pgfqpoint{-9.816398in}{1.557243in}}%
\pgfpathlineto{\pgfqpoint{-9.863853in}{1.513186in}}%
\pgfpathlineto{\pgfqpoint{-9.912991in}{1.482732in}}%
\pgfpathlineto{\pgfqpoint{-9.960967in}{1.506046in}}%
\pgfpathlineto{\pgfqpoint{-10.009255in}{1.478216in}}%
\pgfpathlineto{\pgfqpoint{-10.058316in}{1.473298in}}%
\pgfpathlineto{\pgfqpoint{-10.106404in}{1.554138in}}%
\pgfpathlineto{\pgfqpoint{-10.154319in}{1.452129in}}%
\pgfpathlineto{\pgfqpoint{-10.204843in}{1.471644in}}%
\pgfpathlineto{\pgfqpoint{-10.253421in}{1.498695in}}%
\pgfpathlineto{\pgfqpoint{-10.301938in}{1.400470in}}%
\pgfpathlineto{\pgfqpoint{-10.352351in}{1.491809in}}%
\pgfpathlineto{\pgfqpoint{-10.401144in}{1.494968in}}%
\pgfpathlineto{\pgfqpoint{-10.448474in}{1.506432in}}%
\pgfpathlineto{\pgfqpoint{-10.497957in}{1.487581in}}%
\pgfpathlineto{\pgfqpoint{-10.546833in}{1.511409in}}%
\pgfpathlineto{\pgfqpoint{-10.594934in}{1.444044in}}%
\pgfpathlineto{\pgfqpoint{-10.644481in}{1.505364in}}%
\pgfpathlineto{\pgfqpoint{-10.693023in}{1.465243in}}%
\pgfpathlineto{\pgfqpoint{-10.741507in}{1.526806in}}%
\pgfpathlineto{\pgfqpoint{-10.790422in}{1.507826in}}%
\pgfpathlineto{\pgfqpoint{-10.837686in}{1.506176in}}%
\pgfpathlineto{\pgfqpoint{-10.885230in}{1.487375in}}%
\pgfpathlineto{\pgfqpoint{-10.933861in}{1.468578in}}%
\pgfpathlineto{\pgfqpoint{-10.980903in}{1.488980in}}%
\pgfpathlineto{\pgfqpoint{-11.028607in}{1.461378in}}%
\pgfpathlineto{\pgfqpoint{-11.077704in}{1.541367in}}%
\pgfpathlineto{\pgfqpoint{-11.125274in}{1.510613in}}%
\pgfpathlineto{\pgfqpoint{-11.172812in}{1.465073in}}%
\pgfpathlineto{\pgfqpoint{-11.221741in}{1.513989in}}%
\pgfpathlineto{\pgfqpoint{-11.268838in}{1.467506in}}%
\pgfpathlineto{\pgfqpoint{-11.316622in}{1.486525in}}%
\pgfpathlineto{\pgfqpoint{-11.366120in}{1.421714in}}%
\pgfpathlineto{\pgfqpoint{-11.415235in}{1.489846in}}%
\pgfpathlineto{\pgfqpoint{-11.463260in}{1.553495in}}%
\pgfpathlineto{\pgfqpoint{-11.511546in}{1.523670in}}%
\pgfpathlineto{\pgfqpoint{-11.559336in}{1.467132in}}%
\pgfpathlineto{\pgfqpoint{-11.607131in}{1.444273in}}%
\pgfpathlineto{\pgfqpoint{-11.656418in}{1.487892in}}%
\pgfpathlineto{\pgfqpoint{-11.704231in}{1.583122in}}%
\pgfpathlineto{\pgfqpoint{-11.751492in}{1.506229in}}%
\pgfpathlineto{\pgfqpoint{-11.800403in}{1.600814in}}%
\pgfpathlineto{\pgfqpoint{-11.847736in}{1.523829in}}%
\pgfpathlineto{\pgfqpoint{-11.895021in}{1.537184in}}%
\pgfpathlineto{\pgfqpoint{-11.943783in}{1.481885in}}%
\pgfpathlineto{\pgfqpoint{-11.991431in}{1.477277in}}%
\pgfpathlineto{\pgfqpoint{-12.038569in}{1.535034in}}%
\pgfpathlineto{\pgfqpoint{-12.087205in}{1.495207in}}%
\pgfpathlineto{\pgfqpoint{-12.135286in}{1.488999in}}%
\pgfpathlineto{\pgfqpoint{-12.182471in}{1.516527in}}%
\pgfpathlineto{\pgfqpoint{-12.231638in}{1.527718in}}%
\pgfpathlineto{\pgfqpoint{-12.278675in}{1.454535in}}%
\pgfpathlineto{\pgfqpoint{-12.326627in}{1.482082in}}%
\pgfpathlineto{\pgfqpoint{-12.375560in}{1.515163in}}%
\pgfpathlineto{\pgfqpoint{-12.422612in}{1.531321in}}%
\pgfpathlineto{\pgfqpoint{-12.469843in}{1.569257in}}%
\pgfpathlineto{\pgfqpoint{-12.517880in}{1.525440in}}%
\pgfpathlineto{\pgfqpoint{-12.565107in}{1.543591in}}%
\pgfpathlineto{\pgfqpoint{-12.612433in}{1.503750in}}%
\pgfpathlineto{\pgfqpoint{-12.661578in}{1.517676in}}%
\pgfpathlineto{\pgfqpoint{-12.708579in}{1.538170in}}%
\pgfpathlineto{\pgfqpoint{-12.755866in}{1.488775in}}%
\pgfpathlineto{\pgfqpoint{-12.805010in}{1.493295in}}%
\pgfpathlineto{\pgfqpoint{-12.852526in}{1.517671in}}%
\pgfpathlineto{\pgfqpoint{-12.899585in}{1.479180in}}%
\pgfpathlineto{\pgfqpoint{-12.949523in}{1.435421in}}%
\pgfpathlineto{\pgfqpoint{-12.998480in}{1.454859in}}%
\pgfpathlineto{\pgfqpoint{-13.046608in}{1.490275in}}%
\pgfpathlineto{\pgfqpoint{-13.096328in}{1.480735in}}%
\pgfpathlineto{\pgfqpoint{-13.144068in}{1.524677in}}%
\pgfpathlineto{\pgfqpoint{-13.190970in}{1.518044in}}%
\pgfpathlineto{\pgfqpoint{-13.239807in}{1.480017in}}%
\pgfpathlineto{\pgfqpoint{-13.287448in}{1.497695in}}%
\pgfpathlineto{\pgfqpoint{-13.334326in}{1.553606in}}%
\pgfpathlineto{\pgfqpoint{-13.382220in}{1.547024in}}%
\pgfpathlineto{\pgfqpoint{-13.428346in}{1.581828in}}%
\pgfpathlineto{\pgfqpoint{-13.474335in}{1.516610in}}%
\pgfpathlineto{\pgfqpoint{-13.522553in}{1.514360in}}%
\pgfpathlineto{\pgfqpoint{-13.568873in}{1.508377in}}%
\pgfpathlineto{\pgfqpoint{-13.615991in}{1.534549in}}%
\pgfpathlineto{\pgfqpoint{-13.663331in}{1.540818in}}%
\pgfpathlineto{\pgfqpoint{-13.710126in}{1.540627in}}%
\pgfpathlineto{\pgfqpoint{-13.757485in}{1.494626in}}%
\pgfpathlineto{\pgfqpoint{-13.806330in}{1.510925in}}%
\pgfpathlineto{\pgfqpoint{-13.853816in}{1.492190in}}%
\pgfpathlineto{\pgfqpoint{-13.901287in}{1.500648in}}%
\pgfpathlineto{\pgfqpoint{-13.950094in}{1.517126in}}%
\pgfpathlineto{\pgfqpoint{-13.996250in}{1.561039in}}%
\pgfpathlineto{\pgfqpoint{-14.042263in}{1.544535in}}%
\pgfpathlineto{\pgfqpoint{-14.090279in}{1.546965in}}%
\pgfpathlineto{\pgfqpoint{-14.137283in}{1.498469in}}%
\pgfpathlineto{\pgfqpoint{-14.184689in}{1.483789in}}%
\pgfpathlineto{\pgfqpoint{-14.233487in}{1.543727in}}%
\pgfpathlineto{\pgfqpoint{-14.280667in}{1.565050in}}%
\pgfpathlineto{\pgfqpoint{-14.327371in}{1.476105in}}%
\pgfpathlineto{\pgfqpoint{-14.375931in}{1.546708in}}%
\pgfpathlineto{\pgfqpoint{-14.422978in}{1.469295in}}%
\pgfpathlineto{\pgfqpoint{-14.469907in}{1.491863in}}%
\pgfpathlineto{\pgfqpoint{-14.517904in}{1.548620in}}%
\pgfpathlineto{\pgfqpoint{-14.564671in}{1.591513in}}%
\pgfpathlineto{\pgfqpoint{-14.611394in}{1.538923in}}%
\pgfpathlineto{\pgfqpoint{-14.659388in}{1.507421in}}%
\pgfpathlineto{\pgfqpoint{-14.705919in}{1.518840in}}%
\pgfpathlineto{\pgfqpoint{-14.752153in}{1.497269in}}%
\pgfpathlineto{\pgfqpoint{-14.800669in}{1.511082in}}%
\pgfpathlineto{\pgfqpoint{-14.847516in}{1.531812in}}%
\pgfpathlineto{\pgfqpoint{-14.894419in}{1.506448in}}%
\pgfpathlineto{\pgfqpoint{-14.942416in}{1.563335in}}%
\pgfpathlineto{\pgfqpoint{-14.989019in}{1.541511in}}%
\pgfpathlineto{\pgfqpoint{-15.035921in}{1.535720in}}%
\pgfpathlineto{\pgfqpoint{-15.083728in}{1.547427in}}%
\pgfpathlineto{\pgfqpoint{-15.130553in}{1.552002in}}%
\pgfpathlineto{\pgfqpoint{-15.176769in}{1.498058in}}%
\pgfpathlineto{\pgfqpoint{-15.225998in}{1.488189in}}%
\pgfpathlineto{\pgfqpoint{-15.273998in}{1.495848in}}%
\pgfpathlineto{\pgfqpoint{-15.321194in}{1.526672in}}%
\pgfpathlineto{\pgfqpoint{-15.369802in}{1.538810in}}%
\pgfpathlineto{\pgfqpoint{-15.416792in}{1.492281in}}%
\pgfpathlineto{\pgfqpoint{-15.463539in}{1.475342in}}%
\pgfpathlineto{\pgfqpoint{-15.512535in}{1.498642in}}%
\pgfpathlineto{\pgfqpoint{-15.560849in}{1.477918in}}%
\pgfpathlineto{\pgfqpoint{-15.608704in}{1.505462in}}%
\pgfpathlineto{\pgfqpoint{-15.657002in}{1.534486in}}%
\pgfpathlineto{\pgfqpoint{-15.703994in}{1.501355in}}%
\pgfpathlineto{\pgfqpoint{-15.750800in}{1.521893in}}%
\pgfpathlineto{\pgfqpoint{-15.798897in}{1.524477in}}%
\pgfpathlineto{\pgfqpoint{-15.844500in}{1.519941in}}%
\pgfpathlineto{\pgfqpoint{-15.890419in}{1.564865in}}%
\pgfpathlineto{\pgfqpoint{-15.938076in}{1.521523in}}%
\pgfpathlineto{\pgfqpoint{-15.984286in}{1.547631in}}%
\pgfpathlineto{\pgfqpoint{-16.029628in}{1.574697in}}%
\pgfpathlineto{\pgfqpoint{-16.077171in}{1.527229in}}%
\pgfpathlineto{\pgfqpoint{-16.123165in}{1.568872in}}%
\pgfpathlineto{\pgfqpoint{-16.169427in}{1.470586in}}%
\pgfpathlineto{\pgfqpoint{-16.217070in}{1.552921in}}%
\pgfpathlineto{\pgfqpoint{-16.262806in}{1.500250in}}%
\pgfpathlineto{\pgfqpoint{-16.308871in}{1.575979in}}%
\pgfpathlineto{\pgfqpoint{-16.355646in}{1.505766in}}%
\pgfpathlineto{\pgfqpoint{-16.402550in}{1.517211in}}%
\pgfpathlineto{\pgfqpoint{-16.449347in}{1.561665in}}%
\pgfpathlineto{\pgfqpoint{-16.496157in}{1.540314in}}%
\pgfpathlineto{\pgfqpoint{-16.542990in}{1.490534in}}%
\pgfpathlineto{\pgfqpoint{-16.589239in}{1.517135in}}%
\pgfpathlineto{\pgfqpoint{-16.637120in}{1.508387in}}%
\pgfpathlineto{\pgfqpoint{-16.684310in}{1.546420in}}%
\pgfpathlineto{\pgfqpoint{-16.730721in}{1.506971in}}%
\pgfpathlineto{\pgfqpoint{-16.778530in}{1.536917in}}%
\pgfpathlineto{\pgfqpoint{-16.825434in}{1.464062in}}%
\pgfpathlineto{\pgfqpoint{-16.871664in}{1.594296in}}%
\pgfpathlineto{\pgfqpoint{-16.918692in}{1.508554in}}%
\pgfpathlineto{\pgfqpoint{-16.964774in}{1.539521in}}%
\pgfpathlineto{\pgfqpoint{-17.010884in}{1.524465in}}%
\pgfpathlineto{\pgfqpoint{-17.058009in}{1.497095in}}%
\pgfpathlineto{\pgfqpoint{-17.103874in}{1.510600in}}%
\pgfpathlineto{\pgfqpoint{-17.150161in}{1.482695in}}%
\pgfpathlineto{\pgfqpoint{-17.198221in}{1.534904in}}%
\pgfpathlineto{\pgfqpoint{-17.244258in}{1.577279in}}%
\pgfpathlineto{\pgfqpoint{-17.289523in}{1.537181in}}%
\pgfpathlineto{\pgfqpoint{-17.336276in}{1.537005in}}%
\pgfpathlineto{\pgfqpoint{-17.381732in}{1.529616in}}%
\pgfpathlineto{\pgfqpoint{-17.427002in}{1.570733in}}%
\pgfpathlineto{\pgfqpoint{-17.474644in}{1.512933in}}%
\pgfpathlineto{\pgfqpoint{-17.521284in}{1.556197in}}%
\pgfpathlineto{\pgfqpoint{-17.566739in}{1.563906in}}%
\pgfpathlineto{\pgfqpoint{-17.613251in}{1.515236in}}%
\pgfpathlineto{\pgfqpoint{-17.658538in}{1.565494in}}%
\pgfpathlineto{\pgfqpoint{-17.704093in}{1.578824in}}%
\pgfpathlineto{\pgfqpoint{-17.751556in}{1.509051in}}%
\pgfpathlineto{\pgfqpoint{-17.797346in}{1.519041in}}%
\pgfpathlineto{\pgfqpoint{-17.843480in}{1.579485in}}%
\pgfpathlineto{\pgfqpoint{-17.891271in}{1.564506in}}%
\pgfpathlineto{\pgfqpoint{-17.937614in}{1.515160in}}%
\pgfpathlineto{\pgfqpoint{-17.984102in}{1.542273in}}%
\pgfpathlineto{\pgfqpoint{-18.032389in}{1.545928in}}%
\pgfpathlineto{\pgfqpoint{-18.078784in}{1.601051in}}%
\pgfpathlineto{\pgfqpoint{-18.124795in}{1.529538in}}%
\pgfpathlineto{\pgfqpoint{-18.172791in}{1.501424in}}%
\pgfpathlineto{\pgfqpoint{-18.218674in}{1.582735in}}%
\pgfpathlineto{\pgfqpoint{-18.264167in}{1.486227in}}%
\pgfpathlineto{\pgfqpoint{-18.311512in}{1.559755in}}%
\pgfpathlineto{\pgfqpoint{-18.356748in}{1.566477in}}%
\pgfpathlineto{\pgfqpoint{-18.402000in}{1.564078in}}%
\pgfpathlineto{\pgfqpoint{-18.448764in}{1.533498in}}%
\pgfpathlineto{\pgfqpoint{-18.494344in}{1.528621in}}%
\pgfpathlineto{\pgfqpoint{-18.540355in}{1.525429in}}%
\pgfpathlineto{\pgfqpoint{-18.587378in}{1.525837in}}%
\pgfpathlineto{\pgfqpoint{-18.632978in}{1.538547in}}%
\pgfpathlineto{\pgfqpoint{-18.678684in}{1.555544in}}%
\pgfpathlineto{\pgfqpoint{-18.726033in}{1.477165in}}%
\pgfpathlineto{\pgfqpoint{-18.772260in}{1.487792in}}%
\pgfpathlineto{\pgfqpoint{-18.817991in}{1.556578in}}%
\pgfpathlineto{\pgfqpoint{-18.865040in}{1.508048in}}%
\pgfpathlineto{\pgfqpoint{-18.910503in}{1.538791in}}%
\pgfpathlineto{\pgfqpoint{-18.956783in}{1.541046in}}%
\pgfpathlineto{\pgfqpoint{-19.004576in}{1.532981in}}%
\pgfpathlineto{\pgfqpoint{-19.050812in}{1.490875in}}%
\pgfpathlineto{\pgfqpoint{-19.097062in}{1.505217in}}%
\pgfpathlineto{\pgfqpoint{-19.143544in}{1.498136in}}%
\pgfpathlineto{\pgfqpoint{-19.188899in}{1.584503in}}%
\pgfpathlineto{\pgfqpoint{-19.234797in}{1.521257in}}%
\pgfpathlineto{\pgfqpoint{-19.283047in}{1.503213in}}%
\pgfpathlineto{\pgfqpoint{-19.328532in}{1.520968in}}%
\pgfpathlineto{\pgfqpoint{-19.374230in}{1.505468in}}%
\pgfpathlineto{\pgfqpoint{-19.421584in}{1.544111in}}%
\pgfpathlineto{\pgfqpoint{-19.467303in}{1.503684in}}%
\pgfpathlineto{\pgfqpoint{-19.512973in}{1.536439in}}%
\pgfpathlineto{\pgfqpoint{-19.559546in}{1.550244in}}%
\pgfpathlineto{\pgfqpoint{-19.604824in}{1.576866in}}%
\pgfpathlineto{\pgfqpoint{-19.650361in}{1.571327in}}%
\pgfpathlineto{\pgfqpoint{-19.696649in}{1.563754in}}%
\pgfpathlineto{\pgfqpoint{-19.742024in}{1.525747in}}%
\pgfpathlineto{\pgfqpoint{-19.787808in}{1.554603in}}%
\pgfpathlineto{\pgfqpoint{-19.833948in}{1.615870in}}%
\pgfpathlineto{\pgfqpoint{-19.878562in}{1.530625in}}%
\pgfpathlineto{\pgfqpoint{-19.924139in}{1.505764in}}%
\pgfpathlineto{\pgfqpoint{-19.970967in}{1.559155in}}%
\pgfpathlineto{\pgfqpoint{-20.016506in}{1.539785in}}%
\pgfpathlineto{\pgfqpoint{-20.061722in}{1.592727in}}%
\pgfpathlineto{\pgfqpoint{-20.107707in}{1.561329in}}%
\pgfpathlineto{\pgfqpoint{-20.153038in}{1.582109in}}%
\pgfpathlineto{\pgfqpoint{-20.197814in}{1.567783in}}%
\pgfpathlineto{\pgfqpoint{-20.244875in}{1.506562in}}%
\pgfpathlineto{\pgfqpoint{-20.290419in}{1.521941in}}%
\pgfpathlineto{\pgfqpoint{-20.335981in}{1.585890in}}%
\pgfpathlineto{\pgfqpoint{-20.382199in}{1.496155in}}%
\pgfpathlineto{\pgfqpoint{-20.428221in}{1.566446in}}%
\pgfpathlineto{\pgfqpoint{-20.474014in}{1.637496in}}%
\pgfpathlineto{\pgfqpoint{-20.519904in}{1.528400in}}%
\pgfpathlineto{\pgfqpoint{-20.565077in}{1.523053in}}%
\pgfpathlineto{\pgfqpoint{-20.611243in}{1.505111in}}%
\pgfpathlineto{\pgfqpoint{-20.658742in}{1.517910in}}%
\pgfpathlineto{\pgfqpoint{-20.705066in}{1.495158in}}%
\pgfpathlineto{\pgfqpoint{-20.750831in}{1.586244in}}%
\pgfpathlineto{\pgfqpoint{-20.797022in}{1.511789in}}%
\pgfpathlineto{\pgfqpoint{-20.842169in}{1.573395in}}%
\pgfpathlineto{\pgfqpoint{-20.886847in}{1.509364in}}%
\pgfpathlineto{\pgfqpoint{-20.932365in}{1.587303in}}%
\pgfpathlineto{\pgfqpoint{-20.976378in}{1.569722in}}%
\pgfpathlineto{\pgfqpoint{-21.021509in}{1.546337in}}%
\pgfpathlineto{\pgfqpoint{-21.067357in}{1.548952in}}%
\pgfpathlineto{\pgfqpoint{-21.113139in}{1.549999in}}%
\pgfpathlineto{\pgfqpoint{-21.158215in}{1.552209in}}%
\pgfpathlineto{\pgfqpoint{-21.204354in}{1.500494in}}%
\pgfpathlineto{\pgfqpoint{-21.249304in}{1.541154in}}%
\pgfpathlineto{\pgfqpoint{-21.294264in}{1.571345in}}%
\pgfpathlineto{\pgfqpoint{-21.340867in}{1.601356in}}%
\pgfpathlineto{\pgfqpoint{-21.385572in}{1.488688in}}%
\pgfpathlineto{\pgfqpoint{-21.430267in}{1.537689in}}%
\pgfpathlineto{\pgfqpoint{-21.475985in}{1.562561in}}%
\pgfpathlineto{\pgfqpoint{-21.520883in}{1.525802in}}%
\pgfpathlineto{\pgfqpoint{-21.566198in}{1.548381in}}%
\pgfpathlineto{\pgfqpoint{-21.612919in}{1.615933in}}%
\pgfpathlineto{\pgfqpoint{-21.657342in}{1.568185in}}%
\pgfpathlineto{\pgfqpoint{-21.703167in}{1.581928in}}%
\pgfpathlineto{\pgfqpoint{-21.749554in}{1.558451in}}%
\pgfpathlineto{\pgfqpoint{-21.794255in}{1.560429in}}%
\pgfpathlineto{\pgfqpoint{-21.839254in}{1.556177in}}%
\pgfpathlineto{\pgfqpoint{-21.885462in}{1.584308in}}%
\pgfpathlineto{\pgfqpoint{-21.930597in}{1.582574in}}%
\pgfpathlineto{\pgfqpoint{-21.976168in}{1.568201in}}%
\pgfpathlineto{\pgfqpoint{-22.022397in}{1.582469in}}%
\pgfpathlineto{\pgfqpoint{-22.067661in}{1.554929in}}%
\pgfpathlineto{\pgfqpoint{-22.112231in}{1.570827in}}%
\pgfpathlineto{\pgfqpoint{-22.158025in}{1.557341in}}%
\pgfpathlineto{\pgfqpoint{-22.201868in}{1.614228in}}%
\pgfpathlineto{\pgfqpoint{-22.246011in}{1.583456in}}%
\pgfpathlineto{\pgfqpoint{-22.291940in}{1.593184in}}%
\pgfpathlineto{\pgfqpoint{-22.336074in}{1.581141in}}%
\pgfpathlineto{\pgfqpoint{-22.380996in}{1.598260in}}%
\pgfpathlineto{\pgfqpoint{-22.427383in}{1.576631in}}%
\pgfpathlineto{\pgfqpoint{-22.471952in}{1.558499in}}%
\pgfpathlineto{\pgfqpoint{-22.516601in}{1.547619in}}%
\pgfpathlineto{\pgfqpoint{-22.562133in}{1.572692in}}%
\pgfpathlineto{\pgfqpoint{-22.606850in}{1.574511in}}%
\pgfpathlineto{\pgfqpoint{-22.651258in}{1.479787in}}%
\pgfpathlineto{\pgfqpoint{-22.697642in}{1.551500in}}%
\pgfpathlineto{\pgfqpoint{-22.742185in}{1.526847in}}%
\pgfpathlineto{\pgfqpoint{-22.786844in}{1.570688in}}%
\pgfpathlineto{\pgfqpoint{-22.833029in}{1.519916in}}%
\pgfpathlineto{\pgfqpoint{-22.878494in}{1.500187in}}%
\pgfpathlineto{\pgfqpoint{-22.924822in}{1.540551in}}%
\pgfpathlineto{\pgfqpoint{-22.972746in}{1.533672in}}%
\pgfpathlineto{\pgfqpoint{-23.018637in}{1.510727in}}%
\pgfpathlineto{\pgfqpoint{-23.065076in}{1.486201in}}%
\pgfpathlineto{\pgfqpoint{-23.112561in}{1.547490in}}%
\pgfpathlineto{\pgfqpoint{-23.159034in}{1.515156in}}%
\pgfpathlineto{\pgfqpoint{-23.205343in}{1.459558in}}%
\pgfpathlineto{\pgfqpoint{-23.252910in}{1.552728in}}%
\pgfpathlineto{\pgfqpoint{-23.298917in}{1.495876in}}%
\pgfpathlineto{\pgfqpoint{-23.344516in}{1.591384in}}%
\pgfpathlineto{\pgfqpoint{-23.391460in}{1.484192in}}%
\pgfpathlineto{\pgfqpoint{-23.437154in}{1.525182in}}%
\pgfpathlineto{\pgfqpoint{-23.482780in}{1.559782in}}%
\pgfpathlineto{\pgfqpoint{-23.529965in}{1.516021in}}%
\pgfpathlineto{\pgfqpoint{-23.575877in}{1.587129in}}%
\pgfpathlineto{\pgfqpoint{-23.620333in}{1.564435in}}%
\pgfpathlineto{\pgfqpoint{-23.666576in}{1.549158in}}%
\pgfpathlineto{\pgfqpoint{-23.711883in}{1.567508in}}%
\pgfpathlineto{\pgfqpoint{-23.757289in}{1.517186in}}%
\pgfpathlineto{\pgfqpoint{-23.804343in}{1.528391in}}%
\pgfpathlineto{\pgfqpoint{-23.849943in}{1.570565in}}%
\pgfpathlineto{\pgfqpoint{-23.895017in}{1.607472in}}%
\pgfpathlineto{\pgfqpoint{-23.941619in}{1.561463in}}%
\pgfpathlineto{\pgfqpoint{-23.986811in}{1.590423in}}%
\pgfpathlineto{\pgfqpoint{-24.032531in}{1.538996in}}%
\pgfpathlineto{\pgfqpoint{-24.078902in}{1.578955in}}%
\pgfpathlineto{\pgfqpoint{-24.123999in}{1.507684in}}%
\pgfpathlineto{\pgfqpoint{-24.169854in}{1.516594in}}%
\pgfpathlineto{\pgfqpoint{-24.216953in}{1.503848in}}%
\pgfpathlineto{\pgfqpoint{-24.262109in}{1.496465in}}%
\pgfpathlineto{\pgfqpoint{-24.308071in}{1.563684in}}%
\pgfpathlineto{\pgfqpoint{-24.355467in}{1.549426in}}%
\pgfpathlineto{\pgfqpoint{-24.401173in}{1.605568in}}%
\pgfpathlineto{\pgfqpoint{-24.447114in}{1.498860in}}%
\pgfpathlineto{\pgfqpoint{-24.494847in}{1.536251in}}%
\pgfpathlineto{\pgfqpoint{-24.540716in}{1.586627in}}%
\pgfpathlineto{\pgfqpoint{-24.586259in}{1.602354in}}%
\pgfpathlineto{\pgfqpoint{-24.632696in}{1.570300in}}%
\pgfpathlineto{\pgfqpoint{-24.677917in}{1.563118in}}%
\pgfpathlineto{\pgfqpoint{-24.722523in}{1.547884in}}%
\pgfpathlineto{\pgfqpoint{-24.768227in}{1.553545in}}%
\pgfpathlineto{\pgfqpoint{-24.813437in}{1.484313in}}%
\pgfpathlineto{\pgfqpoint{-24.858721in}{1.542123in}}%
\pgfpathlineto{\pgfqpoint{-24.905462in}{1.471395in}}%
\pgfpathlineto{\pgfqpoint{-24.950861in}{1.574393in}}%
\pgfpathlineto{\pgfqpoint{-24.995958in}{1.572779in}}%
\pgfpathlineto{\pgfqpoint{-25.043027in}{1.489350in}}%
\pgfpathlineto{\pgfqpoint{-25.088791in}{1.585193in}}%
\pgfpathlineto{\pgfqpoint{-25.133757in}{1.484992in}}%
\pgfpathlineto{\pgfqpoint{-25.181603in}{1.546017in}}%
\pgfpathlineto{\pgfqpoint{-25.227067in}{1.580009in}}%
\pgfpathlineto{\pgfqpoint{-25.271783in}{1.536141in}}%
\pgfpathlineto{\pgfqpoint{-25.318638in}{1.522087in}}%
\pgfpathlineto{\pgfqpoint{-25.364031in}{1.551921in}}%
\pgfpathlineto{\pgfqpoint{-25.409278in}{1.579625in}}%
\pgfpathlineto{\pgfqpoint{-25.455838in}{1.533538in}}%
\pgfpathlineto{\pgfqpoint{-25.501531in}{1.571975in}}%
\pgfpathlineto{\pgfqpoint{-25.546370in}{1.560708in}}%
\pgfpathlineto{\pgfqpoint{-25.593105in}{1.516540in}}%
\pgfpathlineto{\pgfqpoint{-25.638897in}{1.563787in}}%
\pgfpathlineto{\pgfqpoint{-25.684196in}{1.557015in}}%
\pgfpathlineto{\pgfqpoint{-25.731236in}{1.568488in}}%
\pgfpathlineto{\pgfqpoint{-25.776589in}{1.568303in}}%
\pgfpathlineto{\pgfqpoint{-25.821656in}{1.568396in}}%
\pgfpathlineto{\pgfqpoint{-25.868331in}{1.611411in}}%
\pgfpathlineto{\pgfqpoint{-25.913010in}{1.528243in}}%
\pgfpathlineto{\pgfqpoint{-25.957619in}{1.553507in}}%
\pgfpathlineto{\pgfqpoint{-26.004871in}{1.580310in}}%
\pgfpathlineto{\pgfqpoint{-26.049962in}{1.536065in}}%
\pgfpathlineto{\pgfqpoint{-26.095397in}{1.578655in}}%
\pgfpathlineto{\pgfqpoint{-26.140687in}{1.541177in}}%
\pgfpathlineto{\pgfqpoint{-26.185627in}{1.611822in}}%
\pgfpathlineto{\pgfqpoint{-26.230565in}{1.505975in}}%
\pgfpathlineto{\pgfqpoint{-26.277965in}{1.547334in}}%
\pgfpathlineto{\pgfqpoint{-26.323384in}{1.509831in}}%
\pgfpathlineto{\pgfqpoint{-26.368659in}{1.566672in}}%
\pgfpathlineto{\pgfqpoint{-26.414228in}{1.562850in}}%
\pgfpathlineto{\pgfqpoint{-26.458732in}{1.605658in}}%
\pgfpathlineto{\pgfqpoint{-26.503012in}{1.562943in}}%
\pgfpathlineto{\pgfqpoint{-26.549765in}{1.494875in}}%
\pgfpathlineto{\pgfqpoint{-26.595409in}{1.518697in}}%
\pgfpathlineto{\pgfqpoint{-26.640477in}{1.534324in}}%
\pgfpathlineto{\pgfqpoint{-26.688111in}{1.566456in}}%
\pgfpathlineto{\pgfqpoint{-26.735142in}{1.510544in}}%
\pgfpathlineto{\pgfqpoint{-26.781839in}{1.532746in}}%
\pgfpathlineto{\pgfqpoint{-26.830647in}{1.473000in}}%
\pgfpathlineto{\pgfqpoint{-26.880611in}{1.469872in}}%
\pgfpathlineto{\pgfqpoint{-26.933555in}{1.368398in}}%
\pgfpathlineto{\pgfqpoint{-26.988998in}{1.466636in}}%
\pgfpathlineto{\pgfqpoint{-27.037714in}{1.472334in}}%
\pgfpathlineto{\pgfqpoint{-27.086555in}{1.500452in}}%
\pgfpathlineto{\pgfqpoint{-27.136341in}{1.468263in}}%
\pgfpathlineto{\pgfqpoint{-27.184903in}{1.530426in}}%
\pgfpathlineto{\pgfqpoint{-27.232862in}{1.480410in}}%
\pgfpathlineto{\pgfqpoint{-27.281821in}{1.481740in}}%
\pgfpathlineto{\pgfqpoint{-27.328705in}{1.556462in}}%
\pgfpathlineto{\pgfqpoint{-27.374936in}{1.507382in}}%
\pgfpathlineto{\pgfqpoint{-27.422670in}{1.535011in}}%
\pgfpathlineto{\pgfqpoint{-27.470237in}{1.489817in}}%
\pgfpathlineto{\pgfqpoint{-27.516652in}{1.528099in}}%
\pgfpathlineto{\pgfqpoint{-27.563341in}{1.480032in}}%
\pgfpathlineto{\pgfqpoint{-27.608047in}{1.614134in}}%
\pgfpathlineto{\pgfqpoint{-27.651646in}{1.580423in}}%
\pgfpathlineto{\pgfqpoint{-27.696819in}{1.619864in}}%
\pgfpathlineto{\pgfqpoint{-27.740871in}{1.512994in}}%
\pgfpathlineto{\pgfqpoint{-27.785329in}{1.542199in}}%
\pgfpathlineto{\pgfqpoint{-27.830427in}{1.603777in}}%
\pgfpathlineto{\pgfqpoint{-27.874405in}{1.618973in}}%
\pgfpathlineto{\pgfqpoint{-27.917812in}{1.591694in}}%
\pgfpathlineto{\pgfqpoint{-27.963188in}{1.505606in}}%
\pgfpathlineto{\pgfqpoint{-28.007858in}{1.607857in}}%
\pgfpathlineto{\pgfqpoint{-28.051883in}{1.579287in}}%
\pgfpathlineto{\pgfqpoint{-28.097071in}{1.595870in}}%
\pgfpathlineto{\pgfqpoint{-28.141536in}{1.539063in}}%
\pgfpathlineto{\pgfqpoint{-28.185953in}{1.549785in}}%
\pgfpathlineto{\pgfqpoint{-28.230974in}{1.592795in}}%
\pgfpathlineto{\pgfqpoint{-28.274355in}{1.576199in}}%
\pgfpathlineto{\pgfqpoint{-28.318373in}{1.498449in}}%
\pgfpathlineto{\pgfqpoint{-28.364238in}{1.547185in}}%
\pgfpathlineto{\pgfqpoint{-28.408130in}{1.562403in}}%
\pgfpathlineto{\pgfqpoint{-28.452363in}{1.503511in}}%
\pgfpathlineto{\pgfqpoint{-28.498901in}{1.513665in}}%
\pgfpathlineto{\pgfqpoint{-28.543533in}{1.574757in}}%
\pgfpathlineto{\pgfqpoint{-28.588250in}{1.281056in}}%
\pgfpathclose%
\pgfusepath{fill}%
\end{pgfscope}%
\begin{pgfscope}%
\pgfpathrectangle{\pgfqpoint{2.662073in}{0.773588in}}{\pgfqpoint{2.964025in}{5.415119in}}%
\pgfusepath{clip}%
\pgfsetbuttcap%
\pgfsetroundjoin%
\definecolor{currentfill}{rgb}{0.580392,0.403922,0.741176}%
\pgfsetfillcolor{currentfill}%
\pgfsetlinewidth{0.000000pt}%
\definecolor{currentstroke}{rgb}{0.000000,0.000000,0.000000}%
\pgfsetstrokecolor{currentstroke}%
\pgfsetdash{}{0pt}%
\pgfpathmoveto{\pgfqpoint{-28.588250in}{1.738591in}}%
\pgfpathlineto{\pgfqpoint{-28.588250in}{1.281056in}}%
\pgfpathlineto{\pgfqpoint{-28.543533in}{1.574757in}}%
\pgfpathlineto{\pgfqpoint{-28.498901in}{1.513665in}}%
\pgfpathlineto{\pgfqpoint{-28.452363in}{1.503511in}}%
\pgfpathlineto{\pgfqpoint{-28.408130in}{1.562403in}}%
\pgfpathlineto{\pgfqpoint{-28.364238in}{1.547185in}}%
\pgfpathlineto{\pgfqpoint{-28.318373in}{1.498449in}}%
\pgfpathlineto{\pgfqpoint{-28.274355in}{1.576199in}}%
\pgfpathlineto{\pgfqpoint{-28.230974in}{1.592795in}}%
\pgfpathlineto{\pgfqpoint{-28.185953in}{1.549785in}}%
\pgfpathlineto{\pgfqpoint{-28.141536in}{1.539063in}}%
\pgfpathlineto{\pgfqpoint{-28.097071in}{1.595870in}}%
\pgfpathlineto{\pgfqpoint{-28.051883in}{1.579287in}}%
\pgfpathlineto{\pgfqpoint{-28.007858in}{1.607857in}}%
\pgfpathlineto{\pgfqpoint{-27.963188in}{1.505606in}}%
\pgfpathlineto{\pgfqpoint{-27.917812in}{1.591694in}}%
\pgfpathlineto{\pgfqpoint{-27.874405in}{1.618973in}}%
\pgfpathlineto{\pgfqpoint{-27.830427in}{1.603777in}}%
\pgfpathlineto{\pgfqpoint{-27.785329in}{1.542199in}}%
\pgfpathlineto{\pgfqpoint{-27.740871in}{1.512994in}}%
\pgfpathlineto{\pgfqpoint{-27.696819in}{1.619864in}}%
\pgfpathlineto{\pgfqpoint{-27.651646in}{1.580423in}}%
\pgfpathlineto{\pgfqpoint{-27.608047in}{1.614134in}}%
\pgfpathlineto{\pgfqpoint{-27.563341in}{1.480032in}}%
\pgfpathlineto{\pgfqpoint{-27.516652in}{1.528099in}}%
\pgfpathlineto{\pgfqpoint{-27.470237in}{1.489817in}}%
\pgfpathlineto{\pgfqpoint{-27.422670in}{1.535011in}}%
\pgfpathlineto{\pgfqpoint{-27.374936in}{1.507382in}}%
\pgfpathlineto{\pgfqpoint{-27.328705in}{1.556462in}}%
\pgfpathlineto{\pgfqpoint{-27.281821in}{1.481740in}}%
\pgfpathlineto{\pgfqpoint{-27.232862in}{1.480410in}}%
\pgfpathlineto{\pgfqpoint{-27.184903in}{1.530426in}}%
\pgfpathlineto{\pgfqpoint{-27.136341in}{1.468263in}}%
\pgfpathlineto{\pgfqpoint{-27.086555in}{1.500452in}}%
\pgfpathlineto{\pgfqpoint{-27.037714in}{1.472334in}}%
\pgfpathlineto{\pgfqpoint{-26.988998in}{1.466636in}}%
\pgfpathlineto{\pgfqpoint{-26.933555in}{1.368398in}}%
\pgfpathlineto{\pgfqpoint{-26.880611in}{1.469872in}}%
\pgfpathlineto{\pgfqpoint{-26.830647in}{1.473000in}}%
\pgfpathlineto{\pgfqpoint{-26.781839in}{1.532746in}}%
\pgfpathlineto{\pgfqpoint{-26.735142in}{1.510544in}}%
\pgfpathlineto{\pgfqpoint{-26.688111in}{1.566456in}}%
\pgfpathlineto{\pgfqpoint{-26.640477in}{1.534324in}}%
\pgfpathlineto{\pgfqpoint{-26.595409in}{1.518697in}}%
\pgfpathlineto{\pgfqpoint{-26.549765in}{1.494875in}}%
\pgfpathlineto{\pgfqpoint{-26.503012in}{1.562943in}}%
\pgfpathlineto{\pgfqpoint{-26.458732in}{1.605658in}}%
\pgfpathlineto{\pgfqpoint{-26.414228in}{1.562850in}}%
\pgfpathlineto{\pgfqpoint{-26.368659in}{1.566672in}}%
\pgfpathlineto{\pgfqpoint{-26.323384in}{1.509831in}}%
\pgfpathlineto{\pgfqpoint{-26.277965in}{1.547334in}}%
\pgfpathlineto{\pgfqpoint{-26.230565in}{1.505975in}}%
\pgfpathlineto{\pgfqpoint{-26.185627in}{1.611822in}}%
\pgfpathlineto{\pgfqpoint{-26.140687in}{1.541177in}}%
\pgfpathlineto{\pgfqpoint{-26.095397in}{1.578655in}}%
\pgfpathlineto{\pgfqpoint{-26.049962in}{1.536065in}}%
\pgfpathlineto{\pgfqpoint{-26.004871in}{1.580310in}}%
\pgfpathlineto{\pgfqpoint{-25.957619in}{1.553507in}}%
\pgfpathlineto{\pgfqpoint{-25.913010in}{1.528243in}}%
\pgfpathlineto{\pgfqpoint{-25.868331in}{1.611411in}}%
\pgfpathlineto{\pgfqpoint{-25.821656in}{1.568396in}}%
\pgfpathlineto{\pgfqpoint{-25.776589in}{1.568303in}}%
\pgfpathlineto{\pgfqpoint{-25.731236in}{1.568488in}}%
\pgfpathlineto{\pgfqpoint{-25.684196in}{1.557015in}}%
\pgfpathlineto{\pgfqpoint{-25.638897in}{1.563787in}}%
\pgfpathlineto{\pgfqpoint{-25.593105in}{1.516540in}}%
\pgfpathlineto{\pgfqpoint{-25.546370in}{1.560708in}}%
\pgfpathlineto{\pgfqpoint{-25.501531in}{1.571975in}}%
\pgfpathlineto{\pgfqpoint{-25.455838in}{1.533538in}}%
\pgfpathlineto{\pgfqpoint{-25.409278in}{1.579625in}}%
\pgfpathlineto{\pgfqpoint{-25.364031in}{1.551921in}}%
\pgfpathlineto{\pgfqpoint{-25.318638in}{1.522087in}}%
\pgfpathlineto{\pgfqpoint{-25.271783in}{1.536141in}}%
\pgfpathlineto{\pgfqpoint{-25.227067in}{1.580009in}}%
\pgfpathlineto{\pgfqpoint{-25.181603in}{1.546017in}}%
\pgfpathlineto{\pgfqpoint{-25.133757in}{1.484992in}}%
\pgfpathlineto{\pgfqpoint{-25.088791in}{1.585193in}}%
\pgfpathlineto{\pgfqpoint{-25.043027in}{1.489350in}}%
\pgfpathlineto{\pgfqpoint{-24.995958in}{1.572779in}}%
\pgfpathlineto{\pgfqpoint{-24.950861in}{1.574393in}}%
\pgfpathlineto{\pgfqpoint{-24.905462in}{1.471395in}}%
\pgfpathlineto{\pgfqpoint{-24.858721in}{1.542123in}}%
\pgfpathlineto{\pgfqpoint{-24.813437in}{1.484313in}}%
\pgfpathlineto{\pgfqpoint{-24.768227in}{1.553545in}}%
\pgfpathlineto{\pgfqpoint{-24.722523in}{1.547884in}}%
\pgfpathlineto{\pgfqpoint{-24.677917in}{1.563118in}}%
\pgfpathlineto{\pgfqpoint{-24.632696in}{1.570300in}}%
\pgfpathlineto{\pgfqpoint{-24.586259in}{1.602354in}}%
\pgfpathlineto{\pgfqpoint{-24.540716in}{1.586627in}}%
\pgfpathlineto{\pgfqpoint{-24.494847in}{1.536251in}}%
\pgfpathlineto{\pgfqpoint{-24.447114in}{1.498860in}}%
\pgfpathlineto{\pgfqpoint{-24.401173in}{1.605568in}}%
\pgfpathlineto{\pgfqpoint{-24.355467in}{1.549426in}}%
\pgfpathlineto{\pgfqpoint{-24.308071in}{1.563684in}}%
\pgfpathlineto{\pgfqpoint{-24.262109in}{1.496465in}}%
\pgfpathlineto{\pgfqpoint{-24.216953in}{1.503848in}}%
\pgfpathlineto{\pgfqpoint{-24.169854in}{1.516594in}}%
\pgfpathlineto{\pgfqpoint{-24.123999in}{1.507684in}}%
\pgfpathlineto{\pgfqpoint{-24.078902in}{1.578955in}}%
\pgfpathlineto{\pgfqpoint{-24.032531in}{1.538996in}}%
\pgfpathlineto{\pgfqpoint{-23.986811in}{1.590423in}}%
\pgfpathlineto{\pgfqpoint{-23.941619in}{1.561463in}}%
\pgfpathlineto{\pgfqpoint{-23.895017in}{1.607472in}}%
\pgfpathlineto{\pgfqpoint{-23.849943in}{1.570565in}}%
\pgfpathlineto{\pgfqpoint{-23.804343in}{1.528391in}}%
\pgfpathlineto{\pgfqpoint{-23.757289in}{1.517186in}}%
\pgfpathlineto{\pgfqpoint{-23.711883in}{1.567508in}}%
\pgfpathlineto{\pgfqpoint{-23.666576in}{1.549158in}}%
\pgfpathlineto{\pgfqpoint{-23.620333in}{1.564435in}}%
\pgfpathlineto{\pgfqpoint{-23.575877in}{1.587129in}}%
\pgfpathlineto{\pgfqpoint{-23.529965in}{1.516021in}}%
\pgfpathlineto{\pgfqpoint{-23.482780in}{1.559782in}}%
\pgfpathlineto{\pgfqpoint{-23.437154in}{1.525182in}}%
\pgfpathlineto{\pgfqpoint{-23.391460in}{1.484192in}}%
\pgfpathlineto{\pgfqpoint{-23.344516in}{1.591384in}}%
\pgfpathlineto{\pgfqpoint{-23.298917in}{1.495876in}}%
\pgfpathlineto{\pgfqpoint{-23.252910in}{1.552728in}}%
\pgfpathlineto{\pgfqpoint{-23.205343in}{1.459558in}}%
\pgfpathlineto{\pgfqpoint{-23.159034in}{1.515156in}}%
\pgfpathlineto{\pgfqpoint{-23.112561in}{1.547490in}}%
\pgfpathlineto{\pgfqpoint{-23.065076in}{1.486201in}}%
\pgfpathlineto{\pgfqpoint{-23.018637in}{1.510727in}}%
\pgfpathlineto{\pgfqpoint{-22.972746in}{1.533672in}}%
\pgfpathlineto{\pgfqpoint{-22.924822in}{1.540551in}}%
\pgfpathlineto{\pgfqpoint{-22.878494in}{1.500187in}}%
\pgfpathlineto{\pgfqpoint{-22.833029in}{1.519916in}}%
\pgfpathlineto{\pgfqpoint{-22.786844in}{1.570688in}}%
\pgfpathlineto{\pgfqpoint{-22.742185in}{1.526847in}}%
\pgfpathlineto{\pgfqpoint{-22.697642in}{1.551500in}}%
\pgfpathlineto{\pgfqpoint{-22.651258in}{1.479787in}}%
\pgfpathlineto{\pgfqpoint{-22.606850in}{1.574511in}}%
\pgfpathlineto{\pgfqpoint{-22.562133in}{1.572692in}}%
\pgfpathlineto{\pgfqpoint{-22.516601in}{1.547619in}}%
\pgfpathlineto{\pgfqpoint{-22.471952in}{1.558499in}}%
\pgfpathlineto{\pgfqpoint{-22.427383in}{1.576631in}}%
\pgfpathlineto{\pgfqpoint{-22.380996in}{1.598260in}}%
\pgfpathlineto{\pgfqpoint{-22.336074in}{1.581141in}}%
\pgfpathlineto{\pgfqpoint{-22.291940in}{1.593184in}}%
\pgfpathlineto{\pgfqpoint{-22.246011in}{1.583456in}}%
\pgfpathlineto{\pgfqpoint{-22.201868in}{1.614228in}}%
\pgfpathlineto{\pgfqpoint{-22.158025in}{1.557341in}}%
\pgfpathlineto{\pgfqpoint{-22.112231in}{1.570827in}}%
\pgfpathlineto{\pgfqpoint{-22.067661in}{1.554929in}}%
\pgfpathlineto{\pgfqpoint{-22.022397in}{1.582469in}}%
\pgfpathlineto{\pgfqpoint{-21.976168in}{1.568201in}}%
\pgfpathlineto{\pgfqpoint{-21.930597in}{1.582574in}}%
\pgfpathlineto{\pgfqpoint{-21.885462in}{1.584308in}}%
\pgfpathlineto{\pgfqpoint{-21.839254in}{1.556177in}}%
\pgfpathlineto{\pgfqpoint{-21.794255in}{1.560429in}}%
\pgfpathlineto{\pgfqpoint{-21.749554in}{1.558451in}}%
\pgfpathlineto{\pgfqpoint{-21.703167in}{1.581928in}}%
\pgfpathlineto{\pgfqpoint{-21.657342in}{1.568185in}}%
\pgfpathlineto{\pgfqpoint{-21.612919in}{1.615933in}}%
\pgfpathlineto{\pgfqpoint{-21.566198in}{1.548381in}}%
\pgfpathlineto{\pgfqpoint{-21.520883in}{1.525802in}}%
\pgfpathlineto{\pgfqpoint{-21.475985in}{1.562561in}}%
\pgfpathlineto{\pgfqpoint{-21.430267in}{1.537689in}}%
\pgfpathlineto{\pgfqpoint{-21.385572in}{1.488688in}}%
\pgfpathlineto{\pgfqpoint{-21.340867in}{1.601356in}}%
\pgfpathlineto{\pgfqpoint{-21.294264in}{1.571345in}}%
\pgfpathlineto{\pgfqpoint{-21.249304in}{1.541154in}}%
\pgfpathlineto{\pgfqpoint{-21.204354in}{1.500494in}}%
\pgfpathlineto{\pgfqpoint{-21.158215in}{1.552209in}}%
\pgfpathlineto{\pgfqpoint{-21.113139in}{1.549999in}}%
\pgfpathlineto{\pgfqpoint{-21.067357in}{1.548952in}}%
\pgfpathlineto{\pgfqpoint{-21.021509in}{1.546337in}}%
\pgfpathlineto{\pgfqpoint{-20.976378in}{1.569722in}}%
\pgfpathlineto{\pgfqpoint{-20.932365in}{1.587303in}}%
\pgfpathlineto{\pgfqpoint{-20.886847in}{1.509364in}}%
\pgfpathlineto{\pgfqpoint{-20.842169in}{1.573395in}}%
\pgfpathlineto{\pgfqpoint{-20.797022in}{1.511789in}}%
\pgfpathlineto{\pgfqpoint{-20.750831in}{1.586244in}}%
\pgfpathlineto{\pgfqpoint{-20.705066in}{1.495158in}}%
\pgfpathlineto{\pgfqpoint{-20.658742in}{1.517910in}}%
\pgfpathlineto{\pgfqpoint{-20.611243in}{1.505111in}}%
\pgfpathlineto{\pgfqpoint{-20.565077in}{1.523053in}}%
\pgfpathlineto{\pgfqpoint{-20.519904in}{1.528400in}}%
\pgfpathlineto{\pgfqpoint{-20.474014in}{1.637496in}}%
\pgfpathlineto{\pgfqpoint{-20.428221in}{1.566446in}}%
\pgfpathlineto{\pgfqpoint{-20.382199in}{1.496155in}}%
\pgfpathlineto{\pgfqpoint{-20.335981in}{1.585890in}}%
\pgfpathlineto{\pgfqpoint{-20.290419in}{1.521941in}}%
\pgfpathlineto{\pgfqpoint{-20.244875in}{1.506562in}}%
\pgfpathlineto{\pgfqpoint{-20.197814in}{1.567783in}}%
\pgfpathlineto{\pgfqpoint{-20.153038in}{1.582109in}}%
\pgfpathlineto{\pgfqpoint{-20.107707in}{1.561329in}}%
\pgfpathlineto{\pgfqpoint{-20.061722in}{1.592727in}}%
\pgfpathlineto{\pgfqpoint{-20.016506in}{1.539785in}}%
\pgfpathlineto{\pgfqpoint{-19.970967in}{1.559155in}}%
\pgfpathlineto{\pgfqpoint{-19.924139in}{1.505764in}}%
\pgfpathlineto{\pgfqpoint{-19.878562in}{1.530625in}}%
\pgfpathlineto{\pgfqpoint{-19.833948in}{1.615870in}}%
\pgfpathlineto{\pgfqpoint{-19.787808in}{1.554603in}}%
\pgfpathlineto{\pgfqpoint{-19.742024in}{1.525747in}}%
\pgfpathlineto{\pgfqpoint{-19.696649in}{1.563754in}}%
\pgfpathlineto{\pgfqpoint{-19.650361in}{1.571327in}}%
\pgfpathlineto{\pgfqpoint{-19.604824in}{1.576866in}}%
\pgfpathlineto{\pgfqpoint{-19.559546in}{1.550244in}}%
\pgfpathlineto{\pgfqpoint{-19.512973in}{1.536439in}}%
\pgfpathlineto{\pgfqpoint{-19.467303in}{1.503684in}}%
\pgfpathlineto{\pgfqpoint{-19.421584in}{1.544111in}}%
\pgfpathlineto{\pgfqpoint{-19.374230in}{1.505468in}}%
\pgfpathlineto{\pgfqpoint{-19.328532in}{1.520968in}}%
\pgfpathlineto{\pgfqpoint{-19.283047in}{1.503213in}}%
\pgfpathlineto{\pgfqpoint{-19.234797in}{1.521257in}}%
\pgfpathlineto{\pgfqpoint{-19.188899in}{1.584503in}}%
\pgfpathlineto{\pgfqpoint{-19.143544in}{1.498136in}}%
\pgfpathlineto{\pgfqpoint{-19.097062in}{1.505217in}}%
\pgfpathlineto{\pgfqpoint{-19.050812in}{1.490875in}}%
\pgfpathlineto{\pgfqpoint{-19.004576in}{1.532981in}}%
\pgfpathlineto{\pgfqpoint{-18.956783in}{1.541046in}}%
\pgfpathlineto{\pgfqpoint{-18.910503in}{1.538791in}}%
\pgfpathlineto{\pgfqpoint{-18.865040in}{1.508048in}}%
\pgfpathlineto{\pgfqpoint{-18.817991in}{1.556578in}}%
\pgfpathlineto{\pgfqpoint{-18.772260in}{1.487792in}}%
\pgfpathlineto{\pgfqpoint{-18.726033in}{1.477165in}}%
\pgfpathlineto{\pgfqpoint{-18.678684in}{1.555544in}}%
\pgfpathlineto{\pgfqpoint{-18.632978in}{1.538547in}}%
\pgfpathlineto{\pgfqpoint{-18.587378in}{1.525837in}}%
\pgfpathlineto{\pgfqpoint{-18.540355in}{1.525429in}}%
\pgfpathlineto{\pgfqpoint{-18.494344in}{1.528621in}}%
\pgfpathlineto{\pgfqpoint{-18.448764in}{1.533498in}}%
\pgfpathlineto{\pgfqpoint{-18.402000in}{1.564078in}}%
\pgfpathlineto{\pgfqpoint{-18.356748in}{1.566477in}}%
\pgfpathlineto{\pgfqpoint{-18.311512in}{1.559755in}}%
\pgfpathlineto{\pgfqpoint{-18.264167in}{1.486227in}}%
\pgfpathlineto{\pgfqpoint{-18.218674in}{1.582735in}}%
\pgfpathlineto{\pgfqpoint{-18.172791in}{1.501424in}}%
\pgfpathlineto{\pgfqpoint{-18.124795in}{1.529538in}}%
\pgfpathlineto{\pgfqpoint{-18.078784in}{1.601051in}}%
\pgfpathlineto{\pgfqpoint{-18.032389in}{1.545928in}}%
\pgfpathlineto{\pgfqpoint{-17.984102in}{1.542273in}}%
\pgfpathlineto{\pgfqpoint{-17.937614in}{1.515160in}}%
\pgfpathlineto{\pgfqpoint{-17.891271in}{1.564506in}}%
\pgfpathlineto{\pgfqpoint{-17.843480in}{1.579485in}}%
\pgfpathlineto{\pgfqpoint{-17.797346in}{1.519041in}}%
\pgfpathlineto{\pgfqpoint{-17.751556in}{1.509051in}}%
\pgfpathlineto{\pgfqpoint{-17.704093in}{1.578824in}}%
\pgfpathlineto{\pgfqpoint{-17.658538in}{1.565494in}}%
\pgfpathlineto{\pgfqpoint{-17.613251in}{1.515236in}}%
\pgfpathlineto{\pgfqpoint{-17.566739in}{1.563906in}}%
\pgfpathlineto{\pgfqpoint{-17.521284in}{1.556197in}}%
\pgfpathlineto{\pgfqpoint{-17.474644in}{1.512933in}}%
\pgfpathlineto{\pgfqpoint{-17.427002in}{1.570733in}}%
\pgfpathlineto{\pgfqpoint{-17.381732in}{1.529616in}}%
\pgfpathlineto{\pgfqpoint{-17.336276in}{1.537005in}}%
\pgfpathlineto{\pgfqpoint{-17.289523in}{1.537181in}}%
\pgfpathlineto{\pgfqpoint{-17.244258in}{1.577279in}}%
\pgfpathlineto{\pgfqpoint{-17.198221in}{1.534904in}}%
\pgfpathlineto{\pgfqpoint{-17.150161in}{1.482695in}}%
\pgfpathlineto{\pgfqpoint{-17.103874in}{1.510600in}}%
\pgfpathlineto{\pgfqpoint{-17.058009in}{1.497095in}}%
\pgfpathlineto{\pgfqpoint{-17.010884in}{1.524465in}}%
\pgfpathlineto{\pgfqpoint{-16.964774in}{1.539521in}}%
\pgfpathlineto{\pgfqpoint{-16.918692in}{1.508554in}}%
\pgfpathlineto{\pgfqpoint{-16.871664in}{1.594296in}}%
\pgfpathlineto{\pgfqpoint{-16.825434in}{1.464062in}}%
\pgfpathlineto{\pgfqpoint{-16.778530in}{1.536917in}}%
\pgfpathlineto{\pgfqpoint{-16.730721in}{1.506971in}}%
\pgfpathlineto{\pgfqpoint{-16.684310in}{1.546420in}}%
\pgfpathlineto{\pgfqpoint{-16.637120in}{1.508387in}}%
\pgfpathlineto{\pgfqpoint{-16.589239in}{1.517135in}}%
\pgfpathlineto{\pgfqpoint{-16.542990in}{1.490534in}}%
\pgfpathlineto{\pgfqpoint{-16.496157in}{1.540314in}}%
\pgfpathlineto{\pgfqpoint{-16.449347in}{1.561665in}}%
\pgfpathlineto{\pgfqpoint{-16.402550in}{1.517211in}}%
\pgfpathlineto{\pgfqpoint{-16.355646in}{1.505766in}}%
\pgfpathlineto{\pgfqpoint{-16.308871in}{1.575979in}}%
\pgfpathlineto{\pgfqpoint{-16.262806in}{1.500250in}}%
\pgfpathlineto{\pgfqpoint{-16.217070in}{1.552921in}}%
\pgfpathlineto{\pgfqpoint{-16.169427in}{1.470586in}}%
\pgfpathlineto{\pgfqpoint{-16.123165in}{1.568872in}}%
\pgfpathlineto{\pgfqpoint{-16.077171in}{1.527229in}}%
\pgfpathlineto{\pgfqpoint{-16.029628in}{1.574697in}}%
\pgfpathlineto{\pgfqpoint{-15.984286in}{1.547631in}}%
\pgfpathlineto{\pgfqpoint{-15.938076in}{1.521523in}}%
\pgfpathlineto{\pgfqpoint{-15.890419in}{1.564865in}}%
\pgfpathlineto{\pgfqpoint{-15.844500in}{1.519941in}}%
\pgfpathlineto{\pgfqpoint{-15.798897in}{1.524477in}}%
\pgfpathlineto{\pgfqpoint{-15.750800in}{1.521893in}}%
\pgfpathlineto{\pgfqpoint{-15.703994in}{1.501355in}}%
\pgfpathlineto{\pgfqpoint{-15.657002in}{1.534486in}}%
\pgfpathlineto{\pgfqpoint{-15.608704in}{1.505462in}}%
\pgfpathlineto{\pgfqpoint{-15.560849in}{1.477918in}}%
\pgfpathlineto{\pgfqpoint{-15.512535in}{1.498642in}}%
\pgfpathlineto{\pgfqpoint{-15.463539in}{1.475342in}}%
\pgfpathlineto{\pgfqpoint{-15.416792in}{1.492281in}}%
\pgfpathlineto{\pgfqpoint{-15.369802in}{1.538810in}}%
\pgfpathlineto{\pgfqpoint{-15.321194in}{1.526672in}}%
\pgfpathlineto{\pgfqpoint{-15.273998in}{1.495848in}}%
\pgfpathlineto{\pgfqpoint{-15.225998in}{1.488189in}}%
\pgfpathlineto{\pgfqpoint{-15.176769in}{1.498058in}}%
\pgfpathlineto{\pgfqpoint{-15.130553in}{1.552002in}}%
\pgfpathlineto{\pgfqpoint{-15.083728in}{1.547427in}}%
\pgfpathlineto{\pgfqpoint{-15.035921in}{1.535720in}}%
\pgfpathlineto{\pgfqpoint{-14.989019in}{1.541511in}}%
\pgfpathlineto{\pgfqpoint{-14.942416in}{1.563335in}}%
\pgfpathlineto{\pgfqpoint{-14.894419in}{1.506448in}}%
\pgfpathlineto{\pgfqpoint{-14.847516in}{1.531812in}}%
\pgfpathlineto{\pgfqpoint{-14.800669in}{1.511082in}}%
\pgfpathlineto{\pgfqpoint{-14.752153in}{1.497269in}}%
\pgfpathlineto{\pgfqpoint{-14.705919in}{1.518840in}}%
\pgfpathlineto{\pgfqpoint{-14.659388in}{1.507421in}}%
\pgfpathlineto{\pgfqpoint{-14.611394in}{1.538923in}}%
\pgfpathlineto{\pgfqpoint{-14.564671in}{1.591513in}}%
\pgfpathlineto{\pgfqpoint{-14.517904in}{1.548620in}}%
\pgfpathlineto{\pgfqpoint{-14.469907in}{1.491863in}}%
\pgfpathlineto{\pgfqpoint{-14.422978in}{1.469295in}}%
\pgfpathlineto{\pgfqpoint{-14.375931in}{1.546708in}}%
\pgfpathlineto{\pgfqpoint{-14.327371in}{1.476105in}}%
\pgfpathlineto{\pgfqpoint{-14.280667in}{1.565050in}}%
\pgfpathlineto{\pgfqpoint{-14.233487in}{1.543727in}}%
\pgfpathlineto{\pgfqpoint{-14.184689in}{1.483789in}}%
\pgfpathlineto{\pgfqpoint{-14.137283in}{1.498469in}}%
\pgfpathlineto{\pgfqpoint{-14.090279in}{1.546965in}}%
\pgfpathlineto{\pgfqpoint{-14.042263in}{1.544535in}}%
\pgfpathlineto{\pgfqpoint{-13.996250in}{1.561039in}}%
\pgfpathlineto{\pgfqpoint{-13.950094in}{1.517126in}}%
\pgfpathlineto{\pgfqpoint{-13.901287in}{1.500648in}}%
\pgfpathlineto{\pgfqpoint{-13.853816in}{1.492190in}}%
\pgfpathlineto{\pgfqpoint{-13.806330in}{1.510925in}}%
\pgfpathlineto{\pgfqpoint{-13.757485in}{1.494626in}}%
\pgfpathlineto{\pgfqpoint{-13.710126in}{1.540627in}}%
\pgfpathlineto{\pgfqpoint{-13.663331in}{1.540818in}}%
\pgfpathlineto{\pgfqpoint{-13.615991in}{1.534549in}}%
\pgfpathlineto{\pgfqpoint{-13.568873in}{1.508377in}}%
\pgfpathlineto{\pgfqpoint{-13.522553in}{1.514360in}}%
\pgfpathlineto{\pgfqpoint{-13.474335in}{1.516610in}}%
\pgfpathlineto{\pgfqpoint{-13.428346in}{1.581828in}}%
\pgfpathlineto{\pgfqpoint{-13.382220in}{1.547024in}}%
\pgfpathlineto{\pgfqpoint{-13.334326in}{1.553606in}}%
\pgfpathlineto{\pgfqpoint{-13.287448in}{1.497695in}}%
\pgfpathlineto{\pgfqpoint{-13.239807in}{1.480017in}}%
\pgfpathlineto{\pgfqpoint{-13.190970in}{1.518044in}}%
\pgfpathlineto{\pgfqpoint{-13.144068in}{1.524677in}}%
\pgfpathlineto{\pgfqpoint{-13.096328in}{1.480735in}}%
\pgfpathlineto{\pgfqpoint{-13.046608in}{1.490275in}}%
\pgfpathlineto{\pgfqpoint{-12.998480in}{1.454859in}}%
\pgfpathlineto{\pgfqpoint{-12.949523in}{1.435421in}}%
\pgfpathlineto{\pgfqpoint{-12.899585in}{1.479180in}}%
\pgfpathlineto{\pgfqpoint{-12.852526in}{1.517671in}}%
\pgfpathlineto{\pgfqpoint{-12.805010in}{1.493295in}}%
\pgfpathlineto{\pgfqpoint{-12.755866in}{1.488775in}}%
\pgfpathlineto{\pgfqpoint{-12.708579in}{1.538170in}}%
\pgfpathlineto{\pgfqpoint{-12.661578in}{1.517676in}}%
\pgfpathlineto{\pgfqpoint{-12.612433in}{1.503750in}}%
\pgfpathlineto{\pgfqpoint{-12.565107in}{1.543591in}}%
\pgfpathlineto{\pgfqpoint{-12.517880in}{1.525440in}}%
\pgfpathlineto{\pgfqpoint{-12.469843in}{1.569257in}}%
\pgfpathlineto{\pgfqpoint{-12.422612in}{1.531321in}}%
\pgfpathlineto{\pgfqpoint{-12.375560in}{1.515163in}}%
\pgfpathlineto{\pgfqpoint{-12.326627in}{1.482082in}}%
\pgfpathlineto{\pgfqpoint{-12.278675in}{1.454535in}}%
\pgfpathlineto{\pgfqpoint{-12.231638in}{1.527718in}}%
\pgfpathlineto{\pgfqpoint{-12.182471in}{1.516527in}}%
\pgfpathlineto{\pgfqpoint{-12.135286in}{1.488999in}}%
\pgfpathlineto{\pgfqpoint{-12.087205in}{1.495207in}}%
\pgfpathlineto{\pgfqpoint{-12.038569in}{1.535034in}}%
\pgfpathlineto{\pgfqpoint{-11.991431in}{1.477277in}}%
\pgfpathlineto{\pgfqpoint{-11.943783in}{1.481885in}}%
\pgfpathlineto{\pgfqpoint{-11.895021in}{1.537184in}}%
\pgfpathlineto{\pgfqpoint{-11.847736in}{1.523829in}}%
\pgfpathlineto{\pgfqpoint{-11.800403in}{1.600814in}}%
\pgfpathlineto{\pgfqpoint{-11.751492in}{1.506229in}}%
\pgfpathlineto{\pgfqpoint{-11.704231in}{1.583122in}}%
\pgfpathlineto{\pgfqpoint{-11.656418in}{1.487892in}}%
\pgfpathlineto{\pgfqpoint{-11.607131in}{1.444273in}}%
\pgfpathlineto{\pgfqpoint{-11.559336in}{1.467132in}}%
\pgfpathlineto{\pgfqpoint{-11.511546in}{1.523670in}}%
\pgfpathlineto{\pgfqpoint{-11.463260in}{1.553495in}}%
\pgfpathlineto{\pgfqpoint{-11.415235in}{1.489846in}}%
\pgfpathlineto{\pgfqpoint{-11.366120in}{1.421714in}}%
\pgfpathlineto{\pgfqpoint{-11.316622in}{1.486525in}}%
\pgfpathlineto{\pgfqpoint{-11.268838in}{1.467506in}}%
\pgfpathlineto{\pgfqpoint{-11.221741in}{1.513989in}}%
\pgfpathlineto{\pgfqpoint{-11.172812in}{1.465073in}}%
\pgfpathlineto{\pgfqpoint{-11.125274in}{1.510613in}}%
\pgfpathlineto{\pgfqpoint{-11.077704in}{1.541367in}}%
\pgfpathlineto{\pgfqpoint{-11.028607in}{1.461378in}}%
\pgfpathlineto{\pgfqpoint{-10.980903in}{1.488980in}}%
\pgfpathlineto{\pgfqpoint{-10.933861in}{1.468578in}}%
\pgfpathlineto{\pgfqpoint{-10.885230in}{1.487375in}}%
\pgfpathlineto{\pgfqpoint{-10.837686in}{1.506176in}}%
\pgfpathlineto{\pgfqpoint{-10.790422in}{1.507826in}}%
\pgfpathlineto{\pgfqpoint{-10.741507in}{1.526806in}}%
\pgfpathlineto{\pgfqpoint{-10.693023in}{1.465243in}}%
\pgfpathlineto{\pgfqpoint{-10.644481in}{1.505364in}}%
\pgfpathlineto{\pgfqpoint{-10.594934in}{1.444044in}}%
\pgfpathlineto{\pgfqpoint{-10.546833in}{1.511409in}}%
\pgfpathlineto{\pgfqpoint{-10.497957in}{1.487581in}}%
\pgfpathlineto{\pgfqpoint{-10.448474in}{1.506432in}}%
\pgfpathlineto{\pgfqpoint{-10.401144in}{1.494968in}}%
\pgfpathlineto{\pgfqpoint{-10.352351in}{1.491809in}}%
\pgfpathlineto{\pgfqpoint{-10.301938in}{1.400470in}}%
\pgfpathlineto{\pgfqpoint{-10.253421in}{1.498695in}}%
\pgfpathlineto{\pgfqpoint{-10.204843in}{1.471644in}}%
\pgfpathlineto{\pgfqpoint{-10.154319in}{1.452129in}}%
\pgfpathlineto{\pgfqpoint{-10.106404in}{1.554138in}}%
\pgfpathlineto{\pgfqpoint{-10.058316in}{1.473298in}}%
\pgfpathlineto{\pgfqpoint{-10.009255in}{1.478216in}}%
\pgfpathlineto{\pgfqpoint{-9.960967in}{1.506046in}}%
\pgfpathlineto{\pgfqpoint{-9.912991in}{1.482732in}}%
\pgfpathlineto{\pgfqpoint{-9.863853in}{1.513186in}}%
\pgfpathlineto{\pgfqpoint{-9.816398in}{1.557243in}}%
\pgfpathlineto{\pgfqpoint{-9.768438in}{1.483391in}}%
\pgfpathlineto{\pgfqpoint{-9.719205in}{1.511122in}}%
\pgfpathlineto{\pgfqpoint{-9.671880in}{1.525351in}}%
\pgfpathlineto{\pgfqpoint{-9.624467in}{1.504500in}}%
\pgfpathlineto{\pgfqpoint{-9.575433in}{1.497489in}}%
\pgfpathlineto{\pgfqpoint{-9.527213in}{1.513257in}}%
\pgfpathlineto{\pgfqpoint{-9.479530in}{1.532652in}}%
\pgfpathlineto{\pgfqpoint{-9.430893in}{1.496075in}}%
\pgfpathlineto{\pgfqpoint{-9.382453in}{1.477769in}}%
\pgfpathlineto{\pgfqpoint{-9.332833in}{1.505370in}}%
\pgfpathlineto{\pgfqpoint{-9.281795in}{1.450999in}}%
\pgfpathlineto{\pgfqpoint{-9.232466in}{1.475407in}}%
\pgfpathlineto{\pgfqpoint{-9.182531in}{1.498954in}}%
\pgfpathlineto{\pgfqpoint{-9.130552in}{1.453003in}}%
\pgfpathlineto{\pgfqpoint{-9.080221in}{1.483280in}}%
\pgfpathlineto{\pgfqpoint{-9.030683in}{1.455924in}}%
\pgfpathlineto{\pgfqpoint{-8.979731in}{1.508474in}}%
\pgfpathlineto{\pgfqpoint{-8.930541in}{1.459393in}}%
\pgfpathlineto{\pgfqpoint{-8.880937in}{1.523941in}}%
\pgfpathlineto{\pgfqpoint{-8.829855in}{1.472967in}}%
\pgfpathlineto{\pgfqpoint{-8.780578in}{1.503138in}}%
\pgfpathlineto{\pgfqpoint{-8.731477in}{1.483979in}}%
\pgfpathlineto{\pgfqpoint{-8.681529in}{1.507557in}}%
\pgfpathlineto{\pgfqpoint{-8.632588in}{1.504972in}}%
\pgfpathlineto{\pgfqpoint{-8.583181in}{1.588274in}}%
\pgfpathlineto{\pgfqpoint{-8.532027in}{1.455757in}}%
\pgfpathlineto{\pgfqpoint{-8.481981in}{1.466930in}}%
\pgfpathlineto{\pgfqpoint{-8.432560in}{1.516103in}}%
\pgfpathlineto{\pgfqpoint{-8.382452in}{1.450949in}}%
\pgfpathlineto{\pgfqpoint{-8.332630in}{1.437617in}}%
\pgfpathlineto{\pgfqpoint{-8.283775in}{1.514168in}}%
\pgfpathlineto{\pgfqpoint{-8.233427in}{1.495756in}}%
\pgfpathlineto{\pgfqpoint{-8.184462in}{1.483018in}}%
\pgfpathlineto{\pgfqpoint{-8.135490in}{1.492178in}}%
\pgfpathlineto{\pgfqpoint{-8.085190in}{1.450986in}}%
\pgfpathlineto{\pgfqpoint{-8.035792in}{1.486772in}}%
\pgfpathlineto{\pgfqpoint{-7.986412in}{1.454686in}}%
\pgfpathlineto{\pgfqpoint{-7.935315in}{1.462158in}}%
\pgfpathlineto{\pgfqpoint{-7.885122in}{1.472200in}}%
\pgfpathlineto{\pgfqpoint{-7.835188in}{1.489803in}}%
\pgfpathlineto{\pgfqpoint{-7.783607in}{1.472393in}}%
\pgfpathlineto{\pgfqpoint{-7.733870in}{1.486753in}}%
\pgfpathlineto{\pgfqpoint{-7.685281in}{1.492572in}}%
\pgfpathlineto{\pgfqpoint{-7.634521in}{1.485587in}}%
\pgfpathlineto{\pgfqpoint{-7.585672in}{1.481533in}}%
\pgfpathlineto{\pgfqpoint{-7.536566in}{1.518978in}}%
\pgfpathlineto{\pgfqpoint{-7.485312in}{1.484586in}}%
\pgfpathlineto{\pgfqpoint{-7.435602in}{1.497388in}}%
\pgfpathlineto{\pgfqpoint{-7.386343in}{1.514535in}}%
\pgfpathlineto{\pgfqpoint{-7.335791in}{1.462006in}}%
\pgfpathlineto{\pgfqpoint{-7.286238in}{1.528125in}}%
\pgfpathlineto{\pgfqpoint{-7.237328in}{1.527757in}}%
\pgfpathlineto{\pgfqpoint{-7.185558in}{1.451044in}}%
\pgfpathlineto{\pgfqpoint{-7.135511in}{1.474893in}}%
\pgfpathlineto{\pgfqpoint{-7.085212in}{1.455792in}}%
\pgfpathlineto{\pgfqpoint{-7.033318in}{1.468034in}}%
\pgfpathlineto{\pgfqpoint{-6.984061in}{1.439666in}}%
\pgfpathlineto{\pgfqpoint{-6.934169in}{1.474127in}}%
\pgfpathlineto{\pgfqpoint{-6.882593in}{1.489534in}}%
\pgfpathlineto{\pgfqpoint{-6.832072in}{1.477021in}}%
\pgfpathlineto{\pgfqpoint{-6.782488in}{1.492179in}}%
\pgfpathlineto{\pgfqpoint{-6.731323in}{1.481284in}}%
\pgfpathlineto{\pgfqpoint{-6.682747in}{1.564313in}}%
\pgfpathlineto{\pgfqpoint{-6.634445in}{1.515947in}}%
\pgfpathlineto{\pgfqpoint{-6.582317in}{1.519235in}}%
\pgfpathlineto{\pgfqpoint{-6.532179in}{1.462154in}}%
\pgfpathlineto{\pgfqpoint{-6.481910in}{1.432232in}}%
\pgfpathlineto{\pgfqpoint{-6.429787in}{1.491616in}}%
\pgfpathlineto{\pgfqpoint{-6.380124in}{1.506808in}}%
\pgfpathlineto{\pgfqpoint{-6.330959in}{1.520456in}}%
\pgfpathlineto{\pgfqpoint{-6.279011in}{1.431065in}}%
\pgfpathlineto{\pgfqpoint{-6.228389in}{1.437185in}}%
\pgfpathlineto{\pgfqpoint{-6.177451in}{1.458867in}}%
\pgfpathlineto{\pgfqpoint{-6.125420in}{1.491507in}}%
\pgfpathlineto{\pgfqpoint{-6.075227in}{1.478442in}}%
\pgfpathlineto{\pgfqpoint{-6.024836in}{1.464609in}}%
\pgfpathlineto{\pgfqpoint{-5.974236in}{1.458922in}}%
\pgfpathlineto{\pgfqpoint{-5.924128in}{1.472289in}}%
\pgfpathlineto{\pgfqpoint{-5.874278in}{1.483293in}}%
\pgfpathlineto{\pgfqpoint{-5.822716in}{1.484534in}}%
\pgfpathlineto{\pgfqpoint{-5.774223in}{1.502550in}}%
\pgfpathlineto{\pgfqpoint{-5.724570in}{1.457956in}}%
\pgfpathlineto{\pgfqpoint{-5.673071in}{1.492267in}}%
\pgfpathlineto{\pgfqpoint{-5.623816in}{1.518819in}}%
\pgfpathlineto{\pgfqpoint{-5.574633in}{1.487867in}}%
\pgfpathlineto{\pgfqpoint{-5.523472in}{1.503511in}}%
\pgfpathlineto{\pgfqpoint{-5.473058in}{1.471046in}}%
\pgfpathlineto{\pgfqpoint{-5.421631in}{1.429663in}}%
\pgfpathlineto{\pgfqpoint{-5.369703in}{1.474912in}}%
\pgfpathlineto{\pgfqpoint{-5.319212in}{1.486397in}}%
\pgfpathlineto{\pgfqpoint{-5.267508in}{1.404869in}}%
\pgfpathlineto{\pgfqpoint{-5.215272in}{1.424458in}}%
\pgfpathlineto{\pgfqpoint{-5.165135in}{1.466875in}}%
\pgfpathlineto{\pgfqpoint{-5.114869in}{1.436571in}}%
\pgfpathlineto{\pgfqpoint{-5.064238in}{1.496870in}}%
\pgfpathlineto{\pgfqpoint{-5.014255in}{1.479833in}}%
\pgfpathlineto{\pgfqpoint{-4.963670in}{1.471589in}}%
\pgfpathlineto{\pgfqpoint{-4.911765in}{1.459339in}}%
\pgfpathlineto{\pgfqpoint{-4.861887in}{1.479312in}}%
\pgfpathlineto{\pgfqpoint{-4.812147in}{1.488680in}}%
\pgfpathlineto{\pgfqpoint{-4.759589in}{1.429175in}}%
\pgfpathlineto{\pgfqpoint{-4.709756in}{1.469206in}}%
\pgfpathlineto{\pgfqpoint{-4.659647in}{1.497794in}}%
\pgfpathlineto{\pgfqpoint{-4.607841in}{1.480893in}}%
\pgfpathlineto{\pgfqpoint{-4.558006in}{1.515156in}}%
\pgfpathlineto{\pgfqpoint{-4.508002in}{1.454865in}}%
\pgfpathlineto{\pgfqpoint{-4.457276in}{1.511209in}}%
\pgfpathlineto{\pgfqpoint{-4.407934in}{1.501248in}}%
\pgfpathlineto{\pgfqpoint{-4.357515in}{1.453886in}}%
\pgfpathlineto{\pgfqpoint{-4.305607in}{1.468689in}}%
\pgfpathlineto{\pgfqpoint{-4.255103in}{1.483945in}}%
\pgfpathlineto{\pgfqpoint{-4.204591in}{1.480570in}}%
\pgfpathlineto{\pgfqpoint{-4.153202in}{1.492010in}}%
\pgfpathlineto{\pgfqpoint{-4.103923in}{1.509514in}}%
\pgfpathlineto{\pgfqpoint{-4.053537in}{1.455526in}}%
\pgfpathlineto{\pgfqpoint{-4.002524in}{1.507858in}}%
\pgfpathlineto{\pgfqpoint{-3.952191in}{1.479729in}}%
\pgfpathlineto{\pgfqpoint{-3.902764in}{1.476461in}}%
\pgfpathlineto{\pgfqpoint{-3.850587in}{1.460130in}}%
\pgfpathlineto{\pgfqpoint{-3.799759in}{1.430021in}}%
\pgfpathlineto{\pgfqpoint{-3.748633in}{1.447750in}}%
\pgfpathlineto{\pgfqpoint{-3.696993in}{1.431142in}}%
\pgfpathlineto{\pgfqpoint{-3.646072in}{1.416044in}}%
\pgfpathlineto{\pgfqpoint{-3.595859in}{1.509874in}}%
\pgfpathlineto{\pgfqpoint{-3.544176in}{1.510289in}}%
\pgfpathlineto{\pgfqpoint{-3.494154in}{1.472900in}}%
\pgfpathlineto{\pgfqpoint{-3.444140in}{1.461304in}}%
\pgfpathlineto{\pgfqpoint{-3.392015in}{1.499292in}}%
\pgfpathlineto{\pgfqpoint{-3.341930in}{1.462308in}}%
\pgfpathlineto{\pgfqpoint{-3.292250in}{1.464323in}}%
\pgfpathlineto{\pgfqpoint{-3.241308in}{1.485889in}}%
\pgfpathlineto{\pgfqpoint{-3.190882in}{1.454513in}}%
\pgfpathlineto{\pgfqpoint{-3.140428in}{1.478346in}}%
\pgfpathlineto{\pgfqpoint{-3.088627in}{1.464632in}}%
\pgfpathlineto{\pgfqpoint{-3.039557in}{1.496304in}}%
\pgfpathlineto{\pgfqpoint{-2.990316in}{1.491356in}}%
\pgfpathlineto{\pgfqpoint{-2.938490in}{1.489283in}}%
\pgfpathlineto{\pgfqpoint{-2.887081in}{1.401061in}}%
\pgfpathlineto{\pgfqpoint{-2.836699in}{1.488861in}}%
\pgfpathlineto{\pgfqpoint{-2.784277in}{1.441924in}}%
\pgfpathlineto{\pgfqpoint{-2.732985in}{1.477742in}}%
\pgfpathlineto{\pgfqpoint{-2.681969in}{1.474687in}}%
\pgfpathlineto{\pgfqpoint{-2.629697in}{1.466846in}}%
\pgfpathlineto{\pgfqpoint{-2.578972in}{1.454026in}}%
\pgfpathlineto{\pgfqpoint{-2.528440in}{1.441237in}}%
\pgfpathlineto{\pgfqpoint{-2.476616in}{1.498993in}}%
\pgfpathlineto{\pgfqpoint{-2.425397in}{1.481435in}}%
\pgfpathlineto{\pgfqpoint{-2.375607in}{1.518800in}}%
\pgfpathlineto{\pgfqpoint{-2.323084in}{1.476219in}}%
\pgfpathlineto{\pgfqpoint{-2.272195in}{1.494852in}}%
\pgfpathlineto{\pgfqpoint{-2.220871in}{1.437036in}}%
\pgfpathlineto{\pgfqpoint{-2.167559in}{1.453594in}}%
\pgfpathlineto{\pgfqpoint{-2.116467in}{1.440422in}}%
\pgfpathlineto{\pgfqpoint{-2.064986in}{1.441672in}}%
\pgfpathlineto{\pgfqpoint{-2.013282in}{1.514552in}}%
\pgfpathlineto{\pgfqpoint{-1.963080in}{1.471617in}}%
\pgfpathlineto{\pgfqpoint{-1.912972in}{1.466522in}}%
\pgfpathlineto{\pgfqpoint{-1.860296in}{1.433163in}}%
\pgfpathlineto{\pgfqpoint{-1.809967in}{1.467419in}}%
\pgfpathlineto{\pgfqpoint{-1.759793in}{1.461354in}}%
\pgfpathlineto{\pgfqpoint{-1.707851in}{1.439046in}}%
\pgfpathlineto{\pgfqpoint{-1.657574in}{1.497292in}}%
\pgfpathlineto{\pgfqpoint{-1.606814in}{1.453054in}}%
\pgfpathlineto{\pgfqpoint{-1.553079in}{1.430942in}}%
\pgfpathlineto{\pgfqpoint{-1.502359in}{1.447905in}}%
\pgfpathlineto{\pgfqpoint{-1.451018in}{1.426934in}}%
\pgfpathlineto{\pgfqpoint{-1.397583in}{1.463974in}}%
\pgfpathlineto{\pgfqpoint{-1.346574in}{1.470166in}}%
\pgfpathlineto{\pgfqpoint{-1.295495in}{1.466132in}}%
\pgfpathlineto{\pgfqpoint{-1.242201in}{1.437680in}}%
\pgfpathlineto{\pgfqpoint{-1.191111in}{1.465843in}}%
\pgfpathlineto{\pgfqpoint{-1.140825in}{1.463950in}}%
\pgfpathlineto{\pgfqpoint{-1.088683in}{1.516530in}}%
\pgfpathlineto{\pgfqpoint{-1.038293in}{1.472000in}}%
\pgfpathlineto{\pgfqpoint{-0.987007in}{1.435521in}}%
\pgfpathlineto{\pgfqpoint{-0.934193in}{1.413812in}}%
\pgfpathlineto{\pgfqpoint{-0.883439in}{1.476075in}}%
\pgfpathlineto{\pgfqpoint{-0.832397in}{1.468684in}}%
\pgfpathlineto{\pgfqpoint{-0.780794in}{1.454000in}}%
\pgfpathlineto{\pgfqpoint{-0.729960in}{1.495303in}}%
\pgfpathlineto{\pgfqpoint{-0.679845in}{1.488245in}}%
\pgfpathlineto{\pgfqpoint{-0.628072in}{1.469018in}}%
\pgfpathlineto{\pgfqpoint{-0.577558in}{1.493059in}}%
\pgfpathlineto{\pgfqpoint{-0.525798in}{1.440938in}}%
\pgfpathlineto{\pgfqpoint{-0.472607in}{1.422930in}}%
\pgfpathlineto{\pgfqpoint{-0.421082in}{1.466028in}}%
\pgfpathlineto{\pgfqpoint{-0.370494in}{1.498293in}}%
\pgfpathlineto{\pgfqpoint{-0.317282in}{1.466339in}}%
\pgfpathlineto{\pgfqpoint{-0.265117in}{1.396378in}}%
\pgfpathlineto{\pgfqpoint{-0.212446in}{1.453336in}}%
\pgfpathlineto{\pgfqpoint{-0.159327in}{1.480796in}}%
\pgfpathlineto{\pgfqpoint{-0.106747in}{1.429412in}}%
\pgfpathlineto{\pgfqpoint{-0.053884in}{1.459502in}}%
\pgfpathlineto{\pgfqpoint{0.000028in}{1.474166in}}%
\pgfpathlineto{\pgfqpoint{0.052190in}{1.450584in}}%
\pgfpathlineto{\pgfqpoint{0.103930in}{1.448577in}}%
\pgfpathlineto{\pgfqpoint{0.157615in}{1.435204in}}%
\pgfpathlineto{\pgfqpoint{0.209933in}{1.401838in}}%
\pgfpathlineto{\pgfqpoint{0.261843in}{1.476625in}}%
\pgfpathlineto{\pgfqpoint{0.315264in}{1.464372in}}%
\pgfpathlineto{\pgfqpoint{0.366948in}{1.467650in}}%
\pgfpathlineto{\pgfqpoint{0.419063in}{1.447695in}}%
\pgfpathlineto{\pgfqpoint{0.472767in}{1.411402in}}%
\pgfpathlineto{\pgfqpoint{0.524412in}{1.450867in}}%
\pgfpathlineto{\pgfqpoint{0.575921in}{1.466086in}}%
\pgfpathlineto{\pgfqpoint{0.628822in}{1.496976in}}%
\pgfpathlineto{\pgfqpoint{0.680101in}{1.464706in}}%
\pgfpathlineto{\pgfqpoint{0.730937in}{1.447685in}}%
\pgfpathlineto{\pgfqpoint{0.783351in}{1.447263in}}%
\pgfpathlineto{\pgfqpoint{0.835638in}{1.411341in}}%
\pgfpathlineto{\pgfqpoint{0.890559in}{1.374681in}}%
\pgfpathlineto{\pgfqpoint{0.950896in}{1.372834in}}%
\pgfpathlineto{\pgfqpoint{1.011418in}{1.326296in}}%
\pgfpathlineto{\pgfqpoint{1.071223in}{1.372779in}}%
\pgfpathlineto{\pgfqpoint{1.133818in}{1.333525in}}%
\pgfpathlineto{\pgfqpoint{1.196854in}{1.305125in}}%
\pgfpathlineto{\pgfqpoint{1.262232in}{1.321329in}}%
\pgfpathlineto{\pgfqpoint{1.331848in}{1.249212in}}%
\pgfpathlineto{\pgfqpoint{1.399923in}{1.299568in}}%
\pgfpathlineto{\pgfqpoint{1.469539in}{1.273397in}}%
\pgfpathlineto{\pgfqpoint{1.542225in}{1.248520in}}%
\pgfpathlineto{\pgfqpoint{1.614592in}{1.269004in}}%
\pgfpathlineto{\pgfqpoint{1.687208in}{1.239497in}}%
\pgfpathlineto{\pgfqpoint{1.764147in}{1.234987in}}%
\pgfpathlineto{\pgfqpoint{1.839330in}{1.245446in}}%
\pgfpathlineto{\pgfqpoint{1.918401in}{1.197516in}}%
\pgfpathlineto{\pgfqpoint{1.999401in}{1.228932in}}%
\pgfpathlineto{\pgfqpoint{2.077252in}{1.226639in}}%
\pgfpathlineto{\pgfqpoint{2.157128in}{1.177079in}}%
\pgfpathlineto{\pgfqpoint{2.243381in}{1.200709in}}%
\pgfpathlineto{\pgfqpoint{2.328903in}{1.194231in}}%
\pgfpathlineto{\pgfqpoint{2.416579in}{1.145304in}}%
\pgfpathlineto{\pgfqpoint{2.504252in}{1.213882in}}%
\pgfpathlineto{\pgfqpoint{2.589494in}{1.193878in}}%
\pgfpathlineto{\pgfqpoint{2.676066in}{1.165488in}}%
\pgfpathlineto{\pgfqpoint{2.767061in}{1.177862in}}%
\pgfpathlineto{\pgfqpoint{2.858701in}{1.172160in}}%
\pgfpathlineto{\pgfqpoint{2.948242in}{1.189047in}}%
\pgfpathlineto{\pgfqpoint{3.043068in}{1.153535in}}%
\pgfpathlineto{\pgfqpoint{3.132459in}{1.197335in}}%
\pgfpathlineto{\pgfqpoint{3.198420in}{1.438877in}}%
\pgfpathlineto{\pgfqpoint{3.252590in}{1.466197in}}%
\pgfpathlineto{\pgfqpoint{3.305067in}{1.465799in}}%
\pgfpathlineto{\pgfqpoint{3.357008in}{1.453492in}}%
\pgfpathlineto{\pgfqpoint{3.411084in}{1.405600in}}%
\pgfpathlineto{\pgfqpoint{3.464430in}{1.393664in}}%
\pgfpathlineto{\pgfqpoint{3.517522in}{1.442369in}}%
\pgfpathlineto{\pgfqpoint{3.571722in}{1.459985in}}%
\pgfpathlineto{\pgfqpoint{3.624008in}{1.429618in}}%
\pgfpathlineto{\pgfqpoint{3.676588in}{1.414323in}}%
\pgfpathlineto{\pgfqpoint{3.729373in}{1.488284in}}%
\pgfpathlineto{\pgfqpoint{3.781554in}{1.423144in}}%
\pgfpathlineto{\pgfqpoint{3.833685in}{1.230236in}}%
\pgfpathlineto{\pgfqpoint{3.887639in}{0.773588in}}%
\pgfpathlineto{\pgfqpoint{3.940507in}{0.773588in}}%
\pgfpathlineto{\pgfqpoint{3.993337in}{0.773588in}}%
\pgfpathlineto{\pgfqpoint{4.047138in}{0.773588in}}%
\pgfpathlineto{\pgfqpoint{4.098630in}{0.773588in}}%
\pgfpathlineto{\pgfqpoint{4.150734in}{0.773588in}}%
\pgfpathlineto{\pgfqpoint{4.204360in}{0.773588in}}%
\pgfpathlineto{\pgfqpoint{4.256228in}{0.773588in}}%
\pgfpathlineto{\pgfqpoint{4.307535in}{0.773588in}}%
\pgfpathlineto{\pgfqpoint{4.360055in}{0.773588in}}%
\pgfpathlineto{\pgfqpoint{4.398305in}{0.773588in}}%
\pgfpathlineto{\pgfqpoint{4.446119in}{0.773588in}}%
\pgfpathlineto{\pgfqpoint{4.484596in}{1.454760in}}%
\pgfpathlineto{\pgfqpoint{4.526846in}{1.669618in}}%
\pgfpathlineto{\pgfqpoint{4.566686in}{1.906942in}}%
\pgfpathlineto{\pgfqpoint{4.603496in}{2.224563in}}%
\pgfpathlineto{\pgfqpoint{4.635492in}{2.864879in}}%
\pgfpathlineto{\pgfqpoint{4.666016in}{3.729137in}}%
\pgfpathlineto{\pgfqpoint{4.690368in}{5.235025in}}%
\pgfpathlineto{\pgfqpoint{4.715513in}{5.182737in}}%
\pgfpathlineto{\pgfqpoint{4.739291in}{5.364631in}}%
\pgfpathlineto{\pgfqpoint{4.764034in}{5.483715in}}%
\pgfpathlineto{\pgfqpoint{4.787716in}{5.446238in}}%
\pgfpathlineto{\pgfqpoint{4.811295in}{5.442565in}}%
\pgfpathlineto{\pgfqpoint{4.836687in}{5.484292in}}%
\pgfpathlineto{\pgfqpoint{4.860218in}{5.534473in}}%
\pgfpathlineto{\pgfqpoint{4.884852in}{5.460686in}}%
\pgfpathlineto{\pgfqpoint{4.907640in}{5.518952in}}%
\pgfpathlineto{\pgfqpoint{4.931890in}{5.678338in}}%
\pgfpathlineto{\pgfqpoint{4.954839in}{5.584654in}}%
\pgfpathlineto{\pgfqpoint{4.980174in}{5.504363in}}%
\pgfpathlineto{\pgfqpoint{5.002737in}{5.557546in}}%
\pgfpathlineto{\pgfqpoint{5.026810in}{5.442515in}}%
\pgfpathlineto{\pgfqpoint{5.051612in}{5.582839in}}%
\pgfpathlineto{\pgfqpoint{5.074798in}{5.834541in}}%
\pgfpathlineto{\pgfqpoint{5.097977in}{5.740043in}}%
\pgfpathlineto{\pgfqpoint{5.122448in}{5.601241in}}%
\pgfpathlineto{\pgfqpoint{5.145219in}{5.809148in}}%
\pgfpathlineto{\pgfqpoint{5.168068in}{5.745325in}}%
\pgfpathlineto{\pgfqpoint{5.192095in}{5.846668in}}%
\pgfpathlineto{\pgfqpoint{5.214897in}{5.862622in}}%
\pgfpathlineto{\pgfqpoint{5.237842in}{5.791091in}}%
\pgfpathlineto{\pgfqpoint{5.261861in}{5.744292in}}%
\pgfpathlineto{\pgfqpoint{5.285086in}{5.706485in}}%
\pgfpathlineto{\pgfqpoint{5.307824in}{5.908282in}}%
\pgfpathlineto{\pgfqpoint{5.331834in}{5.813609in}}%
\pgfpathlineto{\pgfqpoint{5.355254in}{5.854178in}}%
\pgfpathlineto{\pgfqpoint{5.377738in}{5.918658in}}%
\pgfpathlineto{\pgfqpoint{5.402166in}{5.815038in}}%
\pgfpathlineto{\pgfqpoint{5.424187in}{5.868196in}}%
\pgfpathlineto{\pgfqpoint{5.448001in}{5.857730in}}%
\pgfpathlineto{\pgfqpoint{5.470431in}{5.911984in}}%
\pgfpathlineto{\pgfqpoint{5.494250in}{5.893046in}}%
\pgfpathlineto{\pgfqpoint{5.516676in}{5.880249in}}%
\pgfpathlineto{\pgfqpoint{5.540659in}{5.899825in}}%
\pgfpathlineto{\pgfqpoint{5.562835in}{5.930845in}}%
\pgfpathlineto{\pgfqpoint{5.587139in}{5.726189in}}%
\pgfpathlineto{\pgfqpoint{5.609949in}{5.825752in}}%
\pgfpathlineto{\pgfqpoint{5.634039in}{5.865058in}}%
\pgfpathlineto{\pgfqpoint{5.661246in}{5.775449in}}%
\pgfpathlineto{\pgfqpoint{5.712060in}{5.758501in}}%
\pgfpathlineto{\pgfqpoint{5.763148in}{5.758501in}}%
\pgfpathlineto{\pgfqpoint{5.815025in}{5.758501in}}%
\pgfpathlineto{\pgfqpoint{5.867920in}{5.758501in}}%
\pgfpathlineto{\pgfqpoint{5.919631in}{5.758501in}}%
\pgfpathlineto{\pgfqpoint{5.971565in}{5.758501in}}%
\pgfpathlineto{\pgfqpoint{6.025764in}{5.758501in}}%
\pgfpathlineto{\pgfqpoint{6.078797in}{5.758501in}}%
\pgfpathlineto{\pgfqpoint{6.078797in}{5.758501in}}%
\pgfpathlineto{\pgfqpoint{6.078797in}{5.758501in}}%
\pgfpathlineto{\pgfqpoint{6.025764in}{5.758501in}}%
\pgfpathlineto{\pgfqpoint{5.971565in}{5.758501in}}%
\pgfpathlineto{\pgfqpoint{5.919631in}{5.758501in}}%
\pgfpathlineto{\pgfqpoint{5.867920in}{5.758501in}}%
\pgfpathlineto{\pgfqpoint{5.815025in}{5.758501in}}%
\pgfpathlineto{\pgfqpoint{5.763148in}{5.758501in}}%
\pgfpathlineto{\pgfqpoint{5.712060in}{5.758501in}}%
\pgfpathlineto{\pgfqpoint{5.661246in}{5.775449in}}%
\pgfpathlineto{\pgfqpoint{5.634039in}{5.865058in}}%
\pgfpathlineto{\pgfqpoint{5.609949in}{5.825752in}}%
\pgfpathlineto{\pgfqpoint{5.587139in}{5.726189in}}%
\pgfpathlineto{\pgfqpoint{5.562835in}{5.930845in}}%
\pgfpathlineto{\pgfqpoint{5.540659in}{5.899825in}}%
\pgfpathlineto{\pgfqpoint{5.516676in}{5.880249in}}%
\pgfpathlineto{\pgfqpoint{5.494250in}{5.893046in}}%
\pgfpathlineto{\pgfqpoint{5.470431in}{5.911984in}}%
\pgfpathlineto{\pgfqpoint{5.448001in}{5.857730in}}%
\pgfpathlineto{\pgfqpoint{5.424187in}{5.868196in}}%
\pgfpathlineto{\pgfqpoint{5.402166in}{5.815038in}}%
\pgfpathlineto{\pgfqpoint{5.377738in}{5.918658in}}%
\pgfpathlineto{\pgfqpoint{5.355254in}{5.854178in}}%
\pgfpathlineto{\pgfqpoint{5.331834in}{5.813609in}}%
\pgfpathlineto{\pgfqpoint{5.307824in}{5.908282in}}%
\pgfpathlineto{\pgfqpoint{5.285086in}{5.706485in}}%
\pgfpathlineto{\pgfqpoint{5.261861in}{5.744292in}}%
\pgfpathlineto{\pgfqpoint{5.237842in}{5.791091in}}%
\pgfpathlineto{\pgfqpoint{5.214897in}{5.862622in}}%
\pgfpathlineto{\pgfqpoint{5.192095in}{5.846668in}}%
\pgfpathlineto{\pgfqpoint{5.168068in}{5.745325in}}%
\pgfpathlineto{\pgfqpoint{5.145219in}{5.809148in}}%
\pgfpathlineto{\pgfqpoint{5.122448in}{5.601241in}}%
\pgfpathlineto{\pgfqpoint{5.097977in}{5.740043in}}%
\pgfpathlineto{\pgfqpoint{5.074798in}{5.834541in}}%
\pgfpathlineto{\pgfqpoint{5.051612in}{5.582839in}}%
\pgfpathlineto{\pgfqpoint{5.026810in}{5.442515in}}%
\pgfpathlineto{\pgfqpoint{5.002737in}{5.557546in}}%
\pgfpathlineto{\pgfqpoint{4.980174in}{5.504363in}}%
\pgfpathlineto{\pgfqpoint{4.954839in}{5.584654in}}%
\pgfpathlineto{\pgfqpoint{4.931890in}{5.678338in}}%
\pgfpathlineto{\pgfqpoint{4.907640in}{5.518952in}}%
\pgfpathlineto{\pgfqpoint{4.884852in}{5.460686in}}%
\pgfpathlineto{\pgfqpoint{4.860218in}{5.534473in}}%
\pgfpathlineto{\pgfqpoint{4.836687in}{5.484292in}}%
\pgfpathlineto{\pgfqpoint{4.811295in}{5.442565in}}%
\pgfpathlineto{\pgfqpoint{4.787716in}{5.446238in}}%
\pgfpathlineto{\pgfqpoint{4.764034in}{5.483715in}}%
\pgfpathlineto{\pgfqpoint{4.739291in}{5.364631in}}%
\pgfpathlineto{\pgfqpoint{4.715513in}{5.182737in}}%
\pgfpathlineto{\pgfqpoint{4.690368in}{5.235025in}}%
\pgfpathlineto{\pgfqpoint{4.666016in}{3.729137in}}%
\pgfpathlineto{\pgfqpoint{4.635492in}{2.864879in}}%
\pgfpathlineto{\pgfqpoint{4.603496in}{2.224563in}}%
\pgfpathlineto{\pgfqpoint{4.566686in}{1.906942in}}%
\pgfpathlineto{\pgfqpoint{4.526846in}{1.669618in}}%
\pgfpathlineto{\pgfqpoint{4.484596in}{1.454760in}}%
\pgfpathlineto{\pgfqpoint{4.446119in}{0.773588in}}%
\pgfpathlineto{\pgfqpoint{4.398305in}{0.773588in}}%
\pgfpathlineto{\pgfqpoint{4.360055in}{0.773588in}}%
\pgfpathlineto{\pgfqpoint{4.307535in}{0.773588in}}%
\pgfpathlineto{\pgfqpoint{4.256228in}{0.773588in}}%
\pgfpathlineto{\pgfqpoint{4.204360in}{0.773588in}}%
\pgfpathlineto{\pgfqpoint{4.150734in}{0.773588in}}%
\pgfpathlineto{\pgfqpoint{4.098630in}{0.773588in}}%
\pgfpathlineto{\pgfqpoint{4.047138in}{0.773588in}}%
\pgfpathlineto{\pgfqpoint{3.993337in}{0.773588in}}%
\pgfpathlineto{\pgfqpoint{3.940507in}{0.773588in}}%
\pgfpathlineto{\pgfqpoint{3.887639in}{0.773588in}}%
\pgfpathlineto{\pgfqpoint{3.833685in}{1.230236in}}%
\pgfpathlineto{\pgfqpoint{3.781554in}{1.423144in}}%
\pgfpathlineto{\pgfqpoint{3.729373in}{1.488284in}}%
\pgfpathlineto{\pgfqpoint{3.676588in}{1.414323in}}%
\pgfpathlineto{\pgfqpoint{3.624008in}{1.429618in}}%
\pgfpathlineto{\pgfqpoint{3.571722in}{1.459985in}}%
\pgfpathlineto{\pgfqpoint{3.517522in}{1.442369in}}%
\pgfpathlineto{\pgfqpoint{3.464430in}{1.393664in}}%
\pgfpathlineto{\pgfqpoint{3.411084in}{1.405600in}}%
\pgfpathlineto{\pgfqpoint{3.357008in}{1.453492in}}%
\pgfpathlineto{\pgfqpoint{3.305067in}{1.465799in}}%
\pgfpathlineto{\pgfqpoint{3.252590in}{1.466197in}}%
\pgfpathlineto{\pgfqpoint{3.198420in}{1.527806in}}%
\pgfpathlineto{\pgfqpoint{3.132459in}{1.577644in}}%
\pgfpathlineto{\pgfqpoint{3.043068in}{1.548059in}}%
\pgfpathlineto{\pgfqpoint{2.948242in}{1.559840in}}%
\pgfpathlineto{\pgfqpoint{2.858701in}{1.565071in}}%
\pgfpathlineto{\pgfqpoint{2.767061in}{1.569780in}}%
\pgfpathlineto{\pgfqpoint{2.676066in}{1.548891in}}%
\pgfpathlineto{\pgfqpoint{2.589494in}{1.606211in}}%
\pgfpathlineto{\pgfqpoint{2.504252in}{1.621243in}}%
\pgfpathlineto{\pgfqpoint{2.416579in}{1.559328in}}%
\pgfpathlineto{\pgfqpoint{2.328903in}{1.618107in}}%
\pgfpathlineto{\pgfqpoint{2.243381in}{1.642355in}}%
\pgfpathlineto{\pgfqpoint{2.157128in}{1.623152in}}%
\pgfpathlineto{\pgfqpoint{2.077252in}{1.664462in}}%
\pgfpathlineto{\pgfqpoint{1.999401in}{1.683708in}}%
\pgfpathlineto{\pgfqpoint{1.918401in}{1.621377in}}%
\pgfpathlineto{\pgfqpoint{1.839330in}{1.715736in}}%
\pgfpathlineto{\pgfqpoint{1.764147in}{1.713936in}}%
\pgfpathlineto{\pgfqpoint{1.687208in}{1.696625in}}%
\pgfpathlineto{\pgfqpoint{1.614592in}{1.786581in}}%
\pgfpathlineto{\pgfqpoint{1.542225in}{1.744580in}}%
\pgfpathlineto{\pgfqpoint{1.469539in}{1.770692in}}%
\pgfpathlineto{\pgfqpoint{1.399923in}{1.815928in}}%
\pgfpathlineto{\pgfqpoint{1.331848in}{1.799199in}}%
\pgfpathlineto{\pgfqpoint{1.262232in}{1.818030in}}%
\pgfpathlineto{\pgfqpoint{1.196854in}{1.881027in}}%
\pgfpathlineto{\pgfqpoint{1.133818in}{1.909921in}}%
\pgfpathlineto{\pgfqpoint{1.071223in}{1.933531in}}%
\pgfpathlineto{\pgfqpoint{1.011418in}{1.931511in}}%
\pgfpathlineto{\pgfqpoint{0.950896in}{1.985679in}}%
\pgfpathlineto{\pgfqpoint{0.890559in}{1.990714in}}%
\pgfpathlineto{\pgfqpoint{0.835638in}{2.094139in}}%
\pgfpathlineto{\pgfqpoint{0.783351in}{2.140896in}}%
\pgfpathlineto{\pgfqpoint{0.730937in}{2.164722in}}%
\pgfpathlineto{\pgfqpoint{0.680101in}{2.157383in}}%
\pgfpathlineto{\pgfqpoint{0.628822in}{2.192509in}}%
\pgfpathlineto{\pgfqpoint{0.575921in}{2.109977in}}%
\pgfpathlineto{\pgfqpoint{0.524412in}{2.163910in}}%
\pgfpathlineto{\pgfqpoint{0.472767in}{2.093735in}}%
\pgfpathlineto{\pgfqpoint{0.419063in}{2.142856in}}%
\pgfpathlineto{\pgfqpoint{0.366948in}{2.161564in}}%
\pgfpathlineto{\pgfqpoint{0.315264in}{2.123741in}}%
\pgfpathlineto{\pgfqpoint{0.261843in}{2.152791in}}%
\pgfpathlineto{\pgfqpoint{0.209933in}{2.080609in}}%
\pgfpathlineto{\pgfqpoint{0.157615in}{2.082360in}}%
\pgfpathlineto{\pgfqpoint{0.103930in}{2.134356in}}%
\pgfpathlineto{\pgfqpoint{0.052190in}{2.133563in}}%
\pgfpathlineto{\pgfqpoint{0.000028in}{2.150986in}}%
\pgfpathlineto{\pgfqpoint{-0.053884in}{2.082785in}}%
\pgfpathlineto{\pgfqpoint{-0.106747in}{2.100503in}}%
\pgfpathlineto{\pgfqpoint{-0.159327in}{2.160460in}}%
\pgfpathlineto{\pgfqpoint{-0.212446in}{2.077484in}}%
\pgfpathlineto{\pgfqpoint{-0.265117in}{2.088575in}}%
\pgfpathlineto{\pgfqpoint{-0.317282in}{2.138193in}}%
\pgfpathlineto{\pgfqpoint{-0.370494in}{2.178709in}}%
\pgfpathlineto{\pgfqpoint{-0.421082in}{2.168914in}}%
\pgfpathlineto{\pgfqpoint{-0.472607in}{2.139636in}}%
\pgfpathlineto{\pgfqpoint{-0.525798in}{2.116407in}}%
\pgfpathlineto{\pgfqpoint{-0.577558in}{2.154364in}}%
\pgfpathlineto{\pgfqpoint{-0.628072in}{2.179648in}}%
\pgfpathlineto{\pgfqpoint{-0.679845in}{2.197239in}}%
\pgfpathlineto{\pgfqpoint{-0.729960in}{2.169046in}}%
\pgfpathlineto{\pgfqpoint{-0.780794in}{2.184676in}}%
\pgfpathlineto{\pgfqpoint{-0.832397in}{2.135069in}}%
\pgfpathlineto{\pgfqpoint{-0.883439in}{2.176910in}}%
\pgfpathlineto{\pgfqpoint{-0.934193in}{2.085182in}}%
\pgfpathlineto{\pgfqpoint{-0.987007in}{2.108488in}}%
\pgfpathlineto{\pgfqpoint{-1.038293in}{2.128908in}}%
\pgfpathlineto{\pgfqpoint{-1.088683in}{2.221346in}}%
\pgfpathlineto{\pgfqpoint{-1.140825in}{2.152372in}}%
\pgfpathlineto{\pgfqpoint{-1.191111in}{2.194715in}}%
\pgfpathlineto{\pgfqpoint{-1.242201in}{2.103896in}}%
\pgfpathlineto{\pgfqpoint{-1.295495in}{2.154372in}}%
\pgfpathlineto{\pgfqpoint{-1.346574in}{2.133339in}}%
\pgfpathlineto{\pgfqpoint{-1.397583in}{2.148509in}}%
\pgfpathlineto{\pgfqpoint{-1.451018in}{2.142057in}}%
\pgfpathlineto{\pgfqpoint{-1.502359in}{2.171648in}}%
\pgfpathlineto{\pgfqpoint{-1.553079in}{2.102078in}}%
\pgfpathlineto{\pgfqpoint{-1.606814in}{2.127843in}}%
\pgfpathlineto{\pgfqpoint{-1.657574in}{2.169913in}}%
\pgfpathlineto{\pgfqpoint{-1.707851in}{2.202638in}}%
\pgfpathlineto{\pgfqpoint{-1.759793in}{2.188092in}}%
\pgfpathlineto{\pgfqpoint{-1.809967in}{2.193467in}}%
\pgfpathlineto{\pgfqpoint{-1.860296in}{2.059972in}}%
\pgfpathlineto{\pgfqpoint{-1.912972in}{2.149431in}}%
\pgfpathlineto{\pgfqpoint{-1.963080in}{2.151542in}}%
\pgfpathlineto{\pgfqpoint{-2.013282in}{2.262046in}}%
\pgfpathlineto{\pgfqpoint{-2.064986in}{2.084956in}}%
\pgfpathlineto{\pgfqpoint{-2.116467in}{2.127160in}}%
\pgfpathlineto{\pgfqpoint{-2.167559in}{2.125482in}}%
\pgfpathlineto{\pgfqpoint{-2.220871in}{2.106034in}}%
\pgfpathlineto{\pgfqpoint{-2.272195in}{2.191289in}}%
\pgfpathlineto{\pgfqpoint{-2.323084in}{2.127542in}}%
\pgfpathlineto{\pgfqpoint{-2.375607in}{2.206184in}}%
\pgfpathlineto{\pgfqpoint{-2.425397in}{2.126317in}}%
\pgfpathlineto{\pgfqpoint{-2.476616in}{2.167554in}}%
\pgfpathlineto{\pgfqpoint{-2.528440in}{2.142711in}}%
\pgfpathlineto{\pgfqpoint{-2.578972in}{2.169425in}}%
\pgfpathlineto{\pgfqpoint{-2.629697in}{2.167275in}}%
\pgfpathlineto{\pgfqpoint{-2.681969in}{2.127344in}}%
\pgfpathlineto{\pgfqpoint{-2.732985in}{2.155346in}}%
\pgfpathlineto{\pgfqpoint{-2.784277in}{2.125000in}}%
\pgfpathlineto{\pgfqpoint{-2.836699in}{2.196254in}}%
\pgfpathlineto{\pgfqpoint{-2.887081in}{2.073278in}}%
\pgfpathlineto{\pgfqpoint{-2.938490in}{2.147600in}}%
\pgfpathlineto{\pgfqpoint{-2.990316in}{2.216585in}}%
\pgfpathlineto{\pgfqpoint{-3.039557in}{2.210846in}}%
\pgfpathlineto{\pgfqpoint{-3.088627in}{2.140127in}}%
\pgfpathlineto{\pgfqpoint{-3.140428in}{2.188680in}}%
\pgfpathlineto{\pgfqpoint{-3.190882in}{2.138438in}}%
\pgfpathlineto{\pgfqpoint{-3.241308in}{2.202460in}}%
\pgfpathlineto{\pgfqpoint{-3.292250in}{2.172470in}}%
\pgfpathlineto{\pgfqpoint{-3.341930in}{2.172740in}}%
\pgfpathlineto{\pgfqpoint{-3.392015in}{2.161647in}}%
\pgfpathlineto{\pgfqpoint{-3.444140in}{2.189113in}}%
\pgfpathlineto{\pgfqpoint{-3.494154in}{2.129676in}}%
\pgfpathlineto{\pgfqpoint{-3.544176in}{2.202252in}}%
\pgfpathlineto{\pgfqpoint{-3.595859in}{2.177297in}}%
\pgfpathlineto{\pgfqpoint{-3.646072in}{2.113429in}}%
\pgfpathlineto{\pgfqpoint{-3.696993in}{2.130566in}}%
\pgfpathlineto{\pgfqpoint{-3.748633in}{2.153196in}}%
\pgfpathlineto{\pgfqpoint{-3.799759in}{2.087238in}}%
\pgfpathlineto{\pgfqpoint{-3.850587in}{2.107499in}}%
\pgfpathlineto{\pgfqpoint{-3.902764in}{2.187707in}}%
\pgfpathlineto{\pgfqpoint{-3.952191in}{2.179510in}}%
\pgfpathlineto{\pgfqpoint{-4.002524in}{2.199798in}}%
\pgfpathlineto{\pgfqpoint{-4.053537in}{2.124537in}}%
\pgfpathlineto{\pgfqpoint{-4.103923in}{2.174410in}}%
\pgfpathlineto{\pgfqpoint{-4.153202in}{2.198963in}}%
\pgfpathlineto{\pgfqpoint{-4.204591in}{2.148863in}}%
\pgfpathlineto{\pgfqpoint{-4.255103in}{2.169034in}}%
\pgfpathlineto{\pgfqpoint{-4.305607in}{2.156690in}}%
\pgfpathlineto{\pgfqpoint{-4.357515in}{2.166090in}}%
\pgfpathlineto{\pgfqpoint{-4.407934in}{2.200234in}}%
\pgfpathlineto{\pgfqpoint{-4.457276in}{2.240633in}}%
\pgfpathlineto{\pgfqpoint{-4.508002in}{2.139836in}}%
\pgfpathlineto{\pgfqpoint{-4.558006in}{2.234647in}}%
\pgfpathlineto{\pgfqpoint{-4.607841in}{2.187482in}}%
\pgfpathlineto{\pgfqpoint{-4.659647in}{2.192402in}}%
\pgfpathlineto{\pgfqpoint{-4.709756in}{2.191465in}}%
\pgfpathlineto{\pgfqpoint{-4.759589in}{2.072454in}}%
\pgfpathlineto{\pgfqpoint{-4.812147in}{2.204872in}}%
\pgfpathlineto{\pgfqpoint{-4.861887in}{2.209164in}}%
\pgfpathlineto{\pgfqpoint{-4.911765in}{2.139169in}}%
\pgfpathlineto{\pgfqpoint{-4.963670in}{2.160039in}}%
\pgfpathlineto{\pgfqpoint{-5.014255in}{2.208071in}}%
\pgfpathlineto{\pgfqpoint{-5.064238in}{2.211605in}}%
\pgfpathlineto{\pgfqpoint{-5.114869in}{2.149127in}}%
\pgfpathlineto{\pgfqpoint{-5.165135in}{2.157432in}}%
\pgfpathlineto{\pgfqpoint{-5.215272in}{2.170742in}}%
\pgfpathlineto{\pgfqpoint{-5.267508in}{2.144171in}}%
\pgfpathlineto{\pgfqpoint{-5.319212in}{2.162367in}}%
\pgfpathlineto{\pgfqpoint{-5.369703in}{2.201493in}}%
\pgfpathlineto{\pgfqpoint{-5.421631in}{2.116542in}}%
\pgfpathlineto{\pgfqpoint{-5.473058in}{2.119330in}}%
\pgfpathlineto{\pgfqpoint{-5.523472in}{2.175857in}}%
\pgfpathlineto{\pgfqpoint{-5.574633in}{2.259241in}}%
\pgfpathlineto{\pgfqpoint{-5.623816in}{2.182126in}}%
\pgfpathlineto{\pgfqpoint{-5.673071in}{2.221155in}}%
\pgfpathlineto{\pgfqpoint{-5.724570in}{2.159418in}}%
\pgfpathlineto{\pgfqpoint{-5.774223in}{2.282936in}}%
\pgfpathlineto{\pgfqpoint{-5.822716in}{2.143388in}}%
\pgfpathlineto{\pgfqpoint{-5.874278in}{2.137407in}}%
\pgfpathlineto{\pgfqpoint{-5.924128in}{2.157637in}}%
\pgfpathlineto{\pgfqpoint{-5.974236in}{2.198735in}}%
\pgfpathlineto{\pgfqpoint{-6.024836in}{2.170140in}}%
\pgfpathlineto{\pgfqpoint{-6.075227in}{2.148300in}}%
\pgfpathlineto{\pgfqpoint{-6.125420in}{2.152878in}}%
\pgfpathlineto{\pgfqpoint{-6.177451in}{2.124981in}}%
\pgfpathlineto{\pgfqpoint{-6.228389in}{2.136082in}}%
\pgfpathlineto{\pgfqpoint{-6.279011in}{2.079790in}}%
\pgfpathlineto{\pgfqpoint{-6.330959in}{2.235002in}}%
\pgfpathlineto{\pgfqpoint{-6.380124in}{2.146954in}}%
\pgfpathlineto{\pgfqpoint{-6.429787in}{2.173476in}}%
\pgfpathlineto{\pgfqpoint{-6.481910in}{2.104244in}}%
\pgfpathlineto{\pgfqpoint{-6.532179in}{2.158422in}}%
\pgfpathlineto{\pgfqpoint{-6.582317in}{2.168087in}}%
\pgfpathlineto{\pgfqpoint{-6.634445in}{2.248709in}}%
\pgfpathlineto{\pgfqpoint{-6.682747in}{2.305817in}}%
\pgfpathlineto{\pgfqpoint{-6.731323in}{2.180208in}}%
\pgfpathlineto{\pgfqpoint{-6.782488in}{2.179999in}}%
\pgfpathlineto{\pgfqpoint{-6.832072in}{2.136690in}}%
\pgfpathlineto{\pgfqpoint{-6.882593in}{2.135991in}}%
\pgfpathlineto{\pgfqpoint{-6.934169in}{2.206816in}}%
\pgfpathlineto{\pgfqpoint{-6.984061in}{2.133288in}}%
\pgfpathlineto{\pgfqpoint{-7.033318in}{2.157190in}}%
\pgfpathlineto{\pgfqpoint{-7.085212in}{2.094855in}}%
\pgfpathlineto{\pgfqpoint{-7.135511in}{2.185196in}}%
\pgfpathlineto{\pgfqpoint{-7.185558in}{2.138280in}}%
\pgfpathlineto{\pgfqpoint{-7.237328in}{2.251963in}}%
\pgfpathlineto{\pgfqpoint{-7.286238in}{2.258881in}}%
\pgfpathlineto{\pgfqpoint{-7.335791in}{2.153076in}}%
\pgfpathlineto{\pgfqpoint{-7.386343in}{2.276898in}}%
\pgfpathlineto{\pgfqpoint{-7.435602in}{2.161953in}}%
\pgfpathlineto{\pgfqpoint{-7.485312in}{2.128498in}}%
\pgfpathlineto{\pgfqpoint{-7.536566in}{2.230011in}}%
\pgfpathlineto{\pgfqpoint{-7.585672in}{2.226448in}}%
\pgfpathlineto{\pgfqpoint{-7.634521in}{2.227472in}}%
\pgfpathlineto{\pgfqpoint{-7.685281in}{2.213775in}}%
\pgfpathlineto{\pgfqpoint{-7.733870in}{2.218841in}}%
\pgfpathlineto{\pgfqpoint{-7.783607in}{2.160895in}}%
\pgfpathlineto{\pgfqpoint{-7.835188in}{2.167679in}}%
\pgfpathlineto{\pgfqpoint{-7.885122in}{2.150548in}}%
\pgfpathlineto{\pgfqpoint{-7.935315in}{2.168605in}}%
\pgfpathlineto{\pgfqpoint{-7.986412in}{2.175744in}}%
\pgfpathlineto{\pgfqpoint{-8.035792in}{2.208270in}}%
\pgfpathlineto{\pgfqpoint{-8.085190in}{2.191708in}}%
\pgfpathlineto{\pgfqpoint{-8.135490in}{2.245041in}}%
\pgfpathlineto{\pgfqpoint{-8.184462in}{2.181293in}}%
\pgfpathlineto{\pgfqpoint{-8.233427in}{2.201026in}}%
\pgfpathlineto{\pgfqpoint{-8.283775in}{2.241324in}}%
\pgfpathlineto{\pgfqpoint{-8.332630in}{2.155531in}}%
\pgfpathlineto{\pgfqpoint{-8.382452in}{2.185814in}}%
\pgfpathlineto{\pgfqpoint{-8.432560in}{2.212485in}}%
\pgfpathlineto{\pgfqpoint{-8.481981in}{2.166381in}}%
\pgfpathlineto{\pgfqpoint{-8.532027in}{2.134033in}}%
\pgfpathlineto{\pgfqpoint{-8.583181in}{2.283576in}}%
\pgfpathlineto{\pgfqpoint{-8.632588in}{2.212760in}}%
\pgfpathlineto{\pgfqpoint{-8.681529in}{2.268763in}}%
\pgfpathlineto{\pgfqpoint{-8.731477in}{2.192399in}}%
\pgfpathlineto{\pgfqpoint{-8.780578in}{2.126089in}}%
\pgfpathlineto{\pgfqpoint{-8.829855in}{2.221965in}}%
\pgfpathlineto{\pgfqpoint{-8.880937in}{2.258629in}}%
\pgfpathlineto{\pgfqpoint{-8.930541in}{2.155524in}}%
\pgfpathlineto{\pgfqpoint{-8.979731in}{2.188387in}}%
\pgfpathlineto{\pgfqpoint{-9.030683in}{2.199386in}}%
\pgfpathlineto{\pgfqpoint{-9.080221in}{2.134136in}}%
\pgfpathlineto{\pgfqpoint{-9.130552in}{2.107988in}}%
\pgfpathlineto{\pgfqpoint{-9.182531in}{2.192230in}}%
\pgfpathlineto{\pgfqpoint{-9.232466in}{2.162851in}}%
\pgfpathlineto{\pgfqpoint{-9.281795in}{2.161230in}}%
\pgfpathlineto{\pgfqpoint{-9.332833in}{2.243716in}}%
\pgfpathlineto{\pgfqpoint{-9.382453in}{2.175230in}}%
\pgfpathlineto{\pgfqpoint{-9.430893in}{2.296405in}}%
\pgfpathlineto{\pgfqpoint{-9.479530in}{2.281059in}}%
\pgfpathlineto{\pgfqpoint{-9.527213in}{2.236420in}}%
\pgfpathlineto{\pgfqpoint{-9.575433in}{2.223149in}}%
\pgfpathlineto{\pgfqpoint{-9.624467in}{2.193047in}}%
\pgfpathlineto{\pgfqpoint{-9.671880in}{2.275859in}}%
\pgfpathlineto{\pgfqpoint{-9.719205in}{2.243279in}}%
\pgfpathlineto{\pgfqpoint{-9.768438in}{2.182953in}}%
\pgfpathlineto{\pgfqpoint{-9.816398in}{2.259987in}}%
\pgfpathlineto{\pgfqpoint{-9.863853in}{2.234019in}}%
\pgfpathlineto{\pgfqpoint{-9.912991in}{2.190874in}}%
\pgfpathlineto{\pgfqpoint{-9.960967in}{2.227549in}}%
\pgfpathlineto{\pgfqpoint{-10.009255in}{2.239043in}}%
\pgfpathlineto{\pgfqpoint{-10.058316in}{2.241679in}}%
\pgfpathlineto{\pgfqpoint{-10.106404in}{2.293533in}}%
\pgfpathlineto{\pgfqpoint{-10.154319in}{2.158168in}}%
\pgfpathlineto{\pgfqpoint{-10.204843in}{2.175947in}}%
\pgfpathlineto{\pgfqpoint{-10.253421in}{2.240721in}}%
\pgfpathlineto{\pgfqpoint{-10.301938in}{2.100303in}}%
\pgfpathlineto{\pgfqpoint{-10.352351in}{2.192082in}}%
\pgfpathlineto{\pgfqpoint{-10.401144in}{2.256048in}}%
\pgfpathlineto{\pgfqpoint{-10.448474in}{2.243100in}}%
\pgfpathlineto{\pgfqpoint{-10.497957in}{2.195093in}}%
\pgfpathlineto{\pgfqpoint{-10.546833in}{2.255904in}}%
\pgfpathlineto{\pgfqpoint{-10.594934in}{2.175746in}}%
\pgfpathlineto{\pgfqpoint{-10.644481in}{2.223418in}}%
\pgfpathlineto{\pgfqpoint{-10.693023in}{2.186383in}}%
\pgfpathlineto{\pgfqpoint{-10.741507in}{2.233839in}}%
\pgfpathlineto{\pgfqpoint{-10.790422in}{2.207052in}}%
\pgfpathlineto{\pgfqpoint{-10.837686in}{2.240166in}}%
\pgfpathlineto{\pgfqpoint{-10.885230in}{2.249891in}}%
\pgfpathlineto{\pgfqpoint{-10.933861in}{2.172965in}}%
\pgfpathlineto{\pgfqpoint{-10.980903in}{2.249085in}}%
\pgfpathlineto{\pgfqpoint{-11.028607in}{2.210510in}}%
\pgfpathlineto{\pgfqpoint{-11.077704in}{2.227451in}}%
\pgfpathlineto{\pgfqpoint{-11.125274in}{2.246190in}}%
\pgfpathlineto{\pgfqpoint{-11.172812in}{2.208562in}}%
\pgfpathlineto{\pgfqpoint{-11.221741in}{2.303508in}}%
\pgfpathlineto{\pgfqpoint{-11.268838in}{2.230183in}}%
\pgfpathlineto{\pgfqpoint{-11.316622in}{2.220034in}}%
\pgfpathlineto{\pgfqpoint{-11.366120in}{2.099061in}}%
\pgfpathlineto{\pgfqpoint{-11.415235in}{2.245278in}}%
\pgfpathlineto{\pgfqpoint{-11.463260in}{2.295885in}}%
\pgfpathlineto{\pgfqpoint{-11.511546in}{2.265803in}}%
\pgfpathlineto{\pgfqpoint{-11.559336in}{2.181075in}}%
\pgfpathlineto{\pgfqpoint{-11.607131in}{2.180861in}}%
\pgfpathlineto{\pgfqpoint{-11.656418in}{2.181121in}}%
\pgfpathlineto{\pgfqpoint{-11.704231in}{2.358516in}}%
\pgfpathlineto{\pgfqpoint{-11.751492in}{2.295602in}}%
\pgfpathlineto{\pgfqpoint{-11.800403in}{2.379567in}}%
\pgfpathlineto{\pgfqpoint{-11.847736in}{2.219811in}}%
\pgfpathlineto{\pgfqpoint{-11.895021in}{2.288470in}}%
\pgfpathlineto{\pgfqpoint{-11.943783in}{2.234768in}}%
\pgfpathlineto{\pgfqpoint{-11.991431in}{2.248043in}}%
\pgfpathlineto{\pgfqpoint{-12.038569in}{2.273193in}}%
\pgfpathlineto{\pgfqpoint{-12.087205in}{2.177400in}}%
\pgfpathlineto{\pgfqpoint{-12.135286in}{2.264188in}}%
\pgfpathlineto{\pgfqpoint{-12.182471in}{2.246982in}}%
\pgfpathlineto{\pgfqpoint{-12.231638in}{2.297791in}}%
\pgfpathlineto{\pgfqpoint{-12.278675in}{2.233795in}}%
\pgfpathlineto{\pgfqpoint{-12.326627in}{2.170799in}}%
\pgfpathlineto{\pgfqpoint{-12.375560in}{2.251846in}}%
\pgfpathlineto{\pgfqpoint{-12.422612in}{2.263073in}}%
\pgfpathlineto{\pgfqpoint{-12.469843in}{2.332512in}}%
\pgfpathlineto{\pgfqpoint{-12.517880in}{2.244193in}}%
\pgfpathlineto{\pgfqpoint{-12.565107in}{2.222802in}}%
\pgfpathlineto{\pgfqpoint{-12.612433in}{2.227898in}}%
\pgfpathlineto{\pgfqpoint{-12.661578in}{2.247475in}}%
\pgfpathlineto{\pgfqpoint{-12.708579in}{2.244778in}}%
\pgfpathlineto{\pgfqpoint{-12.755866in}{2.266736in}}%
\pgfpathlineto{\pgfqpoint{-12.805010in}{2.238151in}}%
\pgfpathlineto{\pgfqpoint{-12.852526in}{2.302678in}}%
\pgfpathlineto{\pgfqpoint{-12.899585in}{2.250525in}}%
\pgfpathlineto{\pgfqpoint{-12.949523in}{2.137013in}}%
\pgfpathlineto{\pgfqpoint{-12.998480in}{2.174071in}}%
\pgfpathlineto{\pgfqpoint{-13.046608in}{2.259149in}}%
\pgfpathlineto{\pgfqpoint{-13.096328in}{2.219318in}}%
\pgfpathlineto{\pgfqpoint{-13.144068in}{2.276775in}}%
\pgfpathlineto{\pgfqpoint{-13.190970in}{2.205221in}}%
\pgfpathlineto{\pgfqpoint{-13.239807in}{2.216517in}}%
\pgfpathlineto{\pgfqpoint{-13.287448in}{2.270580in}}%
\pgfpathlineto{\pgfqpoint{-13.334326in}{2.332418in}}%
\pgfpathlineto{\pgfqpoint{-13.382220in}{2.336191in}}%
\pgfpathlineto{\pgfqpoint{-13.428346in}{2.402601in}}%
\pgfpathlineto{\pgfqpoint{-13.474335in}{2.228192in}}%
\pgfpathlineto{\pgfqpoint{-13.522553in}{2.332872in}}%
\pgfpathlineto{\pgfqpoint{-13.568873in}{2.287207in}}%
\pgfpathlineto{\pgfqpoint{-13.615991in}{2.313157in}}%
\pgfpathlineto{\pgfqpoint{-13.663331in}{2.300535in}}%
\pgfpathlineto{\pgfqpoint{-13.710126in}{2.301145in}}%
\pgfpathlineto{\pgfqpoint{-13.757485in}{2.211679in}}%
\pgfpathlineto{\pgfqpoint{-13.806330in}{2.258501in}}%
\pgfpathlineto{\pgfqpoint{-13.853816in}{2.219452in}}%
\pgfpathlineto{\pgfqpoint{-13.901287in}{2.218369in}}%
\pgfpathlineto{\pgfqpoint{-13.950094in}{2.260363in}}%
\pgfpathlineto{\pgfqpoint{-13.996250in}{2.293893in}}%
\pgfpathlineto{\pgfqpoint{-14.042263in}{2.307810in}}%
\pgfpathlineto{\pgfqpoint{-14.090279in}{2.280354in}}%
\pgfpathlineto{\pgfqpoint{-14.137283in}{2.236209in}}%
\pgfpathlineto{\pgfqpoint{-14.184689in}{2.240614in}}%
\pgfpathlineto{\pgfqpoint{-14.233487in}{2.295575in}}%
\pgfpathlineto{\pgfqpoint{-14.280667in}{2.314598in}}%
\pgfpathlineto{\pgfqpoint{-14.327371in}{2.198546in}}%
\pgfpathlineto{\pgfqpoint{-14.375931in}{2.267582in}}%
\pgfpathlineto{\pgfqpoint{-14.422978in}{2.252252in}}%
\pgfpathlineto{\pgfqpoint{-14.469907in}{2.196403in}}%
\pgfpathlineto{\pgfqpoint{-14.517904in}{2.282858in}}%
\pgfpathlineto{\pgfqpoint{-14.564671in}{2.359201in}}%
\pgfpathlineto{\pgfqpoint{-14.611394in}{2.259349in}}%
\pgfpathlineto{\pgfqpoint{-14.659388in}{2.231589in}}%
\pgfpathlineto{\pgfqpoint{-14.705919in}{2.325095in}}%
\pgfpathlineto{\pgfqpoint{-14.752153in}{2.221244in}}%
\pgfpathlineto{\pgfqpoint{-14.800669in}{2.246168in}}%
\pgfpathlineto{\pgfqpoint{-14.847516in}{2.291021in}}%
\pgfpathlineto{\pgfqpoint{-14.894419in}{2.254543in}}%
\pgfpathlineto{\pgfqpoint{-14.942416in}{2.346640in}}%
\pgfpathlineto{\pgfqpoint{-14.989019in}{2.209372in}}%
\pgfpathlineto{\pgfqpoint{-15.035921in}{2.276334in}}%
\pgfpathlineto{\pgfqpoint{-15.083728in}{2.274703in}}%
\pgfpathlineto{\pgfqpoint{-15.130553in}{2.321238in}}%
\pgfpathlineto{\pgfqpoint{-15.176769in}{2.193278in}}%
\pgfpathlineto{\pgfqpoint{-15.225998in}{2.182080in}}%
\pgfpathlineto{\pgfqpoint{-15.273998in}{2.229533in}}%
\pgfpathlineto{\pgfqpoint{-15.321194in}{2.232270in}}%
\pgfpathlineto{\pgfqpoint{-15.369802in}{2.286673in}}%
\pgfpathlineto{\pgfqpoint{-15.416792in}{2.229832in}}%
\pgfpathlineto{\pgfqpoint{-15.463539in}{2.226374in}}%
\pgfpathlineto{\pgfqpoint{-15.512535in}{2.212443in}}%
\pgfpathlineto{\pgfqpoint{-15.560849in}{2.178671in}}%
\pgfpathlineto{\pgfqpoint{-15.608704in}{2.242575in}}%
\pgfpathlineto{\pgfqpoint{-15.657002in}{2.300755in}}%
\pgfpathlineto{\pgfqpoint{-15.703994in}{2.232128in}}%
\pgfpathlineto{\pgfqpoint{-15.750800in}{2.278245in}}%
\pgfpathlineto{\pgfqpoint{-15.798897in}{2.335101in}}%
\pgfpathlineto{\pgfqpoint{-15.844500in}{2.277760in}}%
\pgfpathlineto{\pgfqpoint{-15.890419in}{2.299059in}}%
\pgfpathlineto{\pgfqpoint{-15.938076in}{2.240374in}}%
\pgfpathlineto{\pgfqpoint{-15.984286in}{2.323804in}}%
\pgfpathlineto{\pgfqpoint{-16.029628in}{2.385490in}}%
\pgfpathlineto{\pgfqpoint{-16.077171in}{2.312159in}}%
\pgfpathlineto{\pgfqpoint{-16.123165in}{2.338157in}}%
\pgfpathlineto{\pgfqpoint{-16.169427in}{2.235117in}}%
\pgfpathlineto{\pgfqpoint{-16.217070in}{2.384526in}}%
\pgfpathlineto{\pgfqpoint{-16.262806in}{2.262921in}}%
\pgfpathlineto{\pgfqpoint{-16.308871in}{2.323533in}}%
\pgfpathlineto{\pgfqpoint{-16.355646in}{2.298564in}}%
\pgfpathlineto{\pgfqpoint{-16.402550in}{2.304244in}}%
\pgfpathlineto{\pgfqpoint{-16.449347in}{2.386807in}}%
\pgfpathlineto{\pgfqpoint{-16.496157in}{2.289046in}}%
\pgfpathlineto{\pgfqpoint{-16.542990in}{2.234339in}}%
\pgfpathlineto{\pgfqpoint{-16.589239in}{2.259754in}}%
\pgfpathlineto{\pgfqpoint{-16.637120in}{2.253182in}}%
\pgfpathlineto{\pgfqpoint{-16.684310in}{2.279136in}}%
\pgfpathlineto{\pgfqpoint{-16.730721in}{2.325183in}}%
\pgfpathlineto{\pgfqpoint{-16.778530in}{2.239852in}}%
\pgfpathlineto{\pgfqpoint{-16.825434in}{2.203625in}}%
\pgfpathlineto{\pgfqpoint{-16.871664in}{2.340897in}}%
\pgfpathlineto{\pgfqpoint{-16.918692in}{2.296669in}}%
\pgfpathlineto{\pgfqpoint{-16.964774in}{2.358343in}}%
\pgfpathlineto{\pgfqpoint{-17.010884in}{2.349502in}}%
\pgfpathlineto{\pgfqpoint{-17.058009in}{2.257529in}}%
\pgfpathlineto{\pgfqpoint{-17.103874in}{2.273976in}}%
\pgfpathlineto{\pgfqpoint{-17.150161in}{2.221952in}}%
\pgfpathlineto{\pgfqpoint{-17.198221in}{2.280826in}}%
\pgfpathlineto{\pgfqpoint{-17.244258in}{2.351371in}}%
\pgfpathlineto{\pgfqpoint{-17.289523in}{2.317891in}}%
\pgfpathlineto{\pgfqpoint{-17.336276in}{2.301880in}}%
\pgfpathlineto{\pgfqpoint{-17.381732in}{2.324057in}}%
\pgfpathlineto{\pgfqpoint{-17.427002in}{2.350595in}}%
\pgfpathlineto{\pgfqpoint{-17.474644in}{2.203921in}}%
\pgfpathlineto{\pgfqpoint{-17.521284in}{2.355287in}}%
\pgfpathlineto{\pgfqpoint{-17.566739in}{2.285077in}}%
\pgfpathlineto{\pgfqpoint{-17.613251in}{2.306760in}}%
\pgfpathlineto{\pgfqpoint{-17.658538in}{2.333441in}}%
\pgfpathlineto{\pgfqpoint{-17.704093in}{2.359958in}}%
\pgfpathlineto{\pgfqpoint{-17.751556in}{2.229229in}}%
\pgfpathlineto{\pgfqpoint{-17.797346in}{2.230089in}}%
\pgfpathlineto{\pgfqpoint{-17.843480in}{2.357277in}}%
\pgfpathlineto{\pgfqpoint{-17.891271in}{2.291080in}}%
\pgfpathlineto{\pgfqpoint{-17.937614in}{2.259828in}}%
\pgfpathlineto{\pgfqpoint{-17.984102in}{2.332692in}}%
\pgfpathlineto{\pgfqpoint{-18.032389in}{2.235310in}}%
\pgfpathlineto{\pgfqpoint{-18.078784in}{2.360674in}}%
\pgfpathlineto{\pgfqpoint{-18.124795in}{2.276617in}}%
\pgfpathlineto{\pgfqpoint{-18.172791in}{2.260071in}}%
\pgfpathlineto{\pgfqpoint{-18.218674in}{2.399865in}}%
\pgfpathlineto{\pgfqpoint{-18.264167in}{2.258422in}}%
\pgfpathlineto{\pgfqpoint{-18.311512in}{2.335174in}}%
\pgfpathlineto{\pgfqpoint{-18.356748in}{2.356543in}}%
\pgfpathlineto{\pgfqpoint{-18.402000in}{2.282506in}}%
\pgfpathlineto{\pgfqpoint{-18.448764in}{2.301035in}}%
\pgfpathlineto{\pgfqpoint{-18.494344in}{2.284564in}}%
\pgfpathlineto{\pgfqpoint{-18.540355in}{2.275306in}}%
\pgfpathlineto{\pgfqpoint{-18.587378in}{2.302467in}}%
\pgfpathlineto{\pgfqpoint{-18.632978in}{2.330547in}}%
\pgfpathlineto{\pgfqpoint{-18.678684in}{2.250711in}}%
\pgfpathlineto{\pgfqpoint{-18.726033in}{2.222199in}}%
\pgfpathlineto{\pgfqpoint{-18.772260in}{2.264712in}}%
\pgfpathlineto{\pgfqpoint{-18.817991in}{2.346468in}}%
\pgfpathlineto{\pgfqpoint{-18.865040in}{2.287380in}}%
\pgfpathlineto{\pgfqpoint{-18.910503in}{2.274485in}}%
\pgfpathlineto{\pgfqpoint{-18.956783in}{2.294777in}}%
\pgfpathlineto{\pgfqpoint{-19.004576in}{2.293680in}}%
\pgfpathlineto{\pgfqpoint{-19.050812in}{2.240312in}}%
\pgfpathlineto{\pgfqpoint{-19.097062in}{2.291417in}}%
\pgfpathlineto{\pgfqpoint{-19.143544in}{2.276235in}}%
\pgfpathlineto{\pgfqpoint{-19.188899in}{2.359779in}}%
\pgfpathlineto{\pgfqpoint{-19.234797in}{2.230054in}}%
\pgfpathlineto{\pgfqpoint{-19.283047in}{2.261364in}}%
\pgfpathlineto{\pgfqpoint{-19.328532in}{2.305435in}}%
\pgfpathlineto{\pgfqpoint{-19.374230in}{2.268469in}}%
\pgfpathlineto{\pgfqpoint{-19.421584in}{2.297187in}}%
\pgfpathlineto{\pgfqpoint{-19.467303in}{2.261142in}}%
\pgfpathlineto{\pgfqpoint{-19.512973in}{2.325186in}}%
\pgfpathlineto{\pgfqpoint{-19.559546in}{2.328474in}}%
\pgfpathlineto{\pgfqpoint{-19.604824in}{2.303450in}}%
\pgfpathlineto{\pgfqpoint{-19.650361in}{2.323431in}}%
\pgfpathlineto{\pgfqpoint{-19.696649in}{2.353123in}}%
\pgfpathlineto{\pgfqpoint{-19.742024in}{2.320379in}}%
\pgfpathlineto{\pgfqpoint{-19.787808in}{2.314462in}}%
\pgfpathlineto{\pgfqpoint{-19.833948in}{2.394099in}}%
\pgfpathlineto{\pgfqpoint{-19.878562in}{2.292822in}}%
\pgfpathlineto{\pgfqpoint{-19.924139in}{2.300804in}}%
\pgfpathlineto{\pgfqpoint{-19.970967in}{2.314648in}}%
\pgfpathlineto{\pgfqpoint{-20.016506in}{2.319684in}}%
\pgfpathlineto{\pgfqpoint{-20.061722in}{2.399221in}}%
\pgfpathlineto{\pgfqpoint{-20.107707in}{2.335560in}}%
\pgfpathlineto{\pgfqpoint{-20.153038in}{2.417828in}}%
\pgfpathlineto{\pgfqpoint{-20.197814in}{2.368490in}}%
\pgfpathlineto{\pgfqpoint{-20.244875in}{2.295788in}}%
\pgfpathlineto{\pgfqpoint{-20.290419in}{2.335942in}}%
\pgfpathlineto{\pgfqpoint{-20.335981in}{2.367234in}}%
\pgfpathlineto{\pgfqpoint{-20.382199in}{2.257547in}}%
\pgfpathlineto{\pgfqpoint{-20.428221in}{2.269362in}}%
\pgfpathlineto{\pgfqpoint{-20.474014in}{2.375131in}}%
\pgfpathlineto{\pgfqpoint{-20.519904in}{2.269668in}}%
\pgfpathlineto{\pgfqpoint{-20.565077in}{2.264133in}}%
\pgfpathlineto{\pgfqpoint{-20.611243in}{2.285186in}}%
\pgfpathlineto{\pgfqpoint{-20.658742in}{2.294828in}}%
\pgfpathlineto{\pgfqpoint{-20.705066in}{2.191512in}}%
\pgfpathlineto{\pgfqpoint{-20.750831in}{2.356202in}}%
\pgfpathlineto{\pgfqpoint{-20.797022in}{2.336609in}}%
\pgfpathlineto{\pgfqpoint{-20.842169in}{2.376578in}}%
\pgfpathlineto{\pgfqpoint{-20.886847in}{2.313127in}}%
\pgfpathlineto{\pgfqpoint{-20.932365in}{2.398000in}}%
\pgfpathlineto{\pgfqpoint{-20.976378in}{2.318752in}}%
\pgfpathlineto{\pgfqpoint{-21.021509in}{2.398401in}}%
\pgfpathlineto{\pgfqpoint{-21.067357in}{2.273525in}}%
\pgfpathlineto{\pgfqpoint{-21.113139in}{2.303852in}}%
\pgfpathlineto{\pgfqpoint{-21.158215in}{2.349197in}}%
\pgfpathlineto{\pgfqpoint{-21.204354in}{2.272682in}}%
\pgfpathlineto{\pgfqpoint{-21.249304in}{2.343950in}}%
\pgfpathlineto{\pgfqpoint{-21.294264in}{2.338666in}}%
\pgfpathlineto{\pgfqpoint{-21.340867in}{2.365018in}}%
\pgfpathlineto{\pgfqpoint{-21.385572in}{2.240229in}}%
\pgfpathlineto{\pgfqpoint{-21.430267in}{2.368915in}}%
\pgfpathlineto{\pgfqpoint{-21.475985in}{2.351596in}}%
\pgfpathlineto{\pgfqpoint{-21.520883in}{2.303547in}}%
\pgfpathlineto{\pgfqpoint{-21.566198in}{2.321312in}}%
\pgfpathlineto{\pgfqpoint{-21.612919in}{2.358435in}}%
\pgfpathlineto{\pgfqpoint{-21.657342in}{2.304006in}}%
\pgfpathlineto{\pgfqpoint{-21.703167in}{2.305719in}}%
\pgfpathlineto{\pgfqpoint{-21.749554in}{2.356484in}}%
\pgfpathlineto{\pgfqpoint{-21.794255in}{2.291838in}}%
\pgfpathlineto{\pgfqpoint{-21.839254in}{2.354507in}}%
\pgfpathlineto{\pgfqpoint{-21.885462in}{2.330493in}}%
\pgfpathlineto{\pgfqpoint{-21.930597in}{2.349659in}}%
\pgfpathlineto{\pgfqpoint{-21.976168in}{2.350045in}}%
\pgfpathlineto{\pgfqpoint{-22.022397in}{2.355753in}}%
\pgfpathlineto{\pgfqpoint{-22.067661in}{2.317625in}}%
\pgfpathlineto{\pgfqpoint{-22.112231in}{2.402816in}}%
\pgfpathlineto{\pgfqpoint{-22.158025in}{2.333442in}}%
\pgfpathlineto{\pgfqpoint{-22.201868in}{2.450307in}}%
\pgfpathlineto{\pgfqpoint{-22.246011in}{2.383905in}}%
\pgfpathlineto{\pgfqpoint{-22.291940in}{2.405300in}}%
\pgfpathlineto{\pgfqpoint{-22.336074in}{2.351361in}}%
\pgfpathlineto{\pgfqpoint{-22.380996in}{2.366754in}}%
\pgfpathlineto{\pgfqpoint{-22.427383in}{2.393311in}}%
\pgfpathlineto{\pgfqpoint{-22.471952in}{2.320511in}}%
\pgfpathlineto{\pgfqpoint{-22.516601in}{2.314775in}}%
\pgfpathlineto{\pgfqpoint{-22.562133in}{2.284579in}}%
\pgfpathlineto{\pgfqpoint{-22.606850in}{2.367685in}}%
\pgfpathlineto{\pgfqpoint{-22.651258in}{2.277162in}}%
\pgfpathlineto{\pgfqpoint{-22.697642in}{2.375017in}}%
\pgfpathlineto{\pgfqpoint{-22.742185in}{2.333142in}}%
\pgfpathlineto{\pgfqpoint{-22.786844in}{2.357634in}}%
\pgfpathlineto{\pgfqpoint{-22.833029in}{2.281819in}}%
\pgfpathlineto{\pgfqpoint{-22.878494in}{2.240583in}}%
\pgfpathlineto{\pgfqpoint{-22.924822in}{2.270987in}}%
\pgfpathlineto{\pgfqpoint{-22.972746in}{2.233471in}}%
\pgfpathlineto{\pgfqpoint{-23.018637in}{2.243112in}}%
\pgfpathlineto{\pgfqpoint{-23.065076in}{2.254446in}}%
\pgfpathlineto{\pgfqpoint{-23.112561in}{2.254185in}}%
\pgfpathlineto{\pgfqpoint{-23.159034in}{2.266056in}}%
\pgfpathlineto{\pgfqpoint{-23.205343in}{2.237468in}}%
\pgfpathlineto{\pgfqpoint{-23.252910in}{2.311575in}}%
\pgfpathlineto{\pgfqpoint{-23.298917in}{2.250652in}}%
\pgfpathlineto{\pgfqpoint{-23.344516in}{2.406366in}}%
\pgfpathlineto{\pgfqpoint{-23.391460in}{2.234958in}}%
\pgfpathlineto{\pgfqpoint{-23.437154in}{2.278529in}}%
\pgfpathlineto{\pgfqpoint{-23.482780in}{2.308873in}}%
\pgfpathlineto{\pgfqpoint{-23.529965in}{2.261696in}}%
\pgfpathlineto{\pgfqpoint{-23.575877in}{2.381697in}}%
\pgfpathlineto{\pgfqpoint{-23.620333in}{2.387220in}}%
\pgfpathlineto{\pgfqpoint{-23.666576in}{2.319036in}}%
\pgfpathlineto{\pgfqpoint{-23.711883in}{2.353354in}}%
\pgfpathlineto{\pgfqpoint{-23.757289in}{2.247028in}}%
\pgfpathlineto{\pgfqpoint{-23.804343in}{2.273113in}}%
\pgfpathlineto{\pgfqpoint{-23.849943in}{2.406439in}}%
\pgfpathlineto{\pgfqpoint{-23.895017in}{2.369110in}}%
\pgfpathlineto{\pgfqpoint{-23.941619in}{2.326998in}}%
\pgfpathlineto{\pgfqpoint{-23.986811in}{2.339578in}}%
\pgfpathlineto{\pgfqpoint{-24.032531in}{2.297753in}}%
\pgfpathlineto{\pgfqpoint{-24.078902in}{2.374804in}}%
\pgfpathlineto{\pgfqpoint{-24.123999in}{2.304795in}}%
\pgfpathlineto{\pgfqpoint{-24.169854in}{2.333026in}}%
\pgfpathlineto{\pgfqpoint{-24.216953in}{2.320747in}}%
\pgfpathlineto{\pgfqpoint{-24.262109in}{2.232536in}}%
\pgfpathlineto{\pgfqpoint{-24.308071in}{2.306542in}}%
\pgfpathlineto{\pgfqpoint{-24.355467in}{2.295788in}}%
\pgfpathlineto{\pgfqpoint{-24.401173in}{2.320591in}}%
\pgfpathlineto{\pgfqpoint{-24.447114in}{2.223770in}}%
\pgfpathlineto{\pgfqpoint{-24.494847in}{2.270967in}}%
\pgfpathlineto{\pgfqpoint{-24.540716in}{2.353522in}}%
\pgfpathlineto{\pgfqpoint{-24.586259in}{2.393133in}}%
\pgfpathlineto{\pgfqpoint{-24.632696in}{2.340546in}}%
\pgfpathlineto{\pgfqpoint{-24.677917in}{2.367505in}}%
\pgfpathlineto{\pgfqpoint{-24.722523in}{2.377694in}}%
\pgfpathlineto{\pgfqpoint{-24.768227in}{2.337717in}}%
\pgfpathlineto{\pgfqpoint{-24.813437in}{2.277965in}}%
\pgfpathlineto{\pgfqpoint{-24.858721in}{2.257669in}}%
\pgfpathlineto{\pgfqpoint{-24.905462in}{2.264512in}}%
\pgfpathlineto{\pgfqpoint{-24.950861in}{2.402740in}}%
\pgfpathlineto{\pgfqpoint{-24.995958in}{2.309445in}}%
\pgfpathlineto{\pgfqpoint{-25.043027in}{2.288248in}}%
\pgfpathlineto{\pgfqpoint{-25.088791in}{2.359011in}}%
\pgfpathlineto{\pgfqpoint{-25.133757in}{2.272653in}}%
\pgfpathlineto{\pgfqpoint{-25.181603in}{2.291067in}}%
\pgfpathlineto{\pgfqpoint{-25.227067in}{2.360627in}}%
\pgfpathlineto{\pgfqpoint{-25.271783in}{2.348997in}}%
\pgfpathlineto{\pgfqpoint{-25.318638in}{2.326455in}}%
\pgfpathlineto{\pgfqpoint{-25.364031in}{2.311396in}}%
\pgfpathlineto{\pgfqpoint{-25.409278in}{2.367974in}}%
\pgfpathlineto{\pgfqpoint{-25.455838in}{2.272708in}}%
\pgfpathlineto{\pgfqpoint{-25.501531in}{2.376055in}}%
\pgfpathlineto{\pgfqpoint{-25.546370in}{2.328600in}}%
\pgfpathlineto{\pgfqpoint{-25.593105in}{2.272352in}}%
\pgfpathlineto{\pgfqpoint{-25.638897in}{2.281275in}}%
\pgfpathlineto{\pgfqpoint{-25.684196in}{2.294335in}}%
\pgfpathlineto{\pgfqpoint{-25.731236in}{2.311073in}}%
\pgfpathlineto{\pgfqpoint{-25.776589in}{2.340637in}}%
\pgfpathlineto{\pgfqpoint{-25.821656in}{2.322218in}}%
\pgfpathlineto{\pgfqpoint{-25.868331in}{2.390510in}}%
\pgfpathlineto{\pgfqpoint{-25.913010in}{2.322586in}}%
\pgfpathlineto{\pgfqpoint{-25.957619in}{2.309584in}}%
\pgfpathlineto{\pgfqpoint{-26.004871in}{2.354180in}}%
\pgfpathlineto{\pgfqpoint{-26.049962in}{2.314984in}}%
\pgfpathlineto{\pgfqpoint{-26.095397in}{2.395267in}}%
\pgfpathlineto{\pgfqpoint{-26.140687in}{2.358697in}}%
\pgfpathlineto{\pgfqpoint{-26.185627in}{2.371658in}}%
\pgfpathlineto{\pgfqpoint{-26.230565in}{2.240058in}}%
\pgfpathlineto{\pgfqpoint{-26.277965in}{2.359459in}}%
\pgfpathlineto{\pgfqpoint{-26.323384in}{2.352393in}}%
\pgfpathlineto{\pgfqpoint{-26.368659in}{2.404176in}}%
\pgfpathlineto{\pgfqpoint{-26.414228in}{2.365948in}}%
\pgfpathlineto{\pgfqpoint{-26.458732in}{2.375345in}}%
\pgfpathlineto{\pgfqpoint{-26.503012in}{2.342913in}}%
\pgfpathlineto{\pgfqpoint{-26.549765in}{2.274534in}}%
\pgfpathlineto{\pgfqpoint{-26.595409in}{2.292757in}}%
\pgfpathlineto{\pgfqpoint{-26.640477in}{2.306144in}}%
\pgfpathlineto{\pgfqpoint{-26.688111in}{2.323786in}}%
\pgfpathlineto{\pgfqpoint{-26.735142in}{2.276180in}}%
\pgfpathlineto{\pgfqpoint{-26.781839in}{2.305652in}}%
\pgfpathlineto{\pgfqpoint{-26.830647in}{2.171250in}}%
\pgfpathlineto{\pgfqpoint{-26.880611in}{2.112084in}}%
\pgfpathlineto{\pgfqpoint{-26.933555in}{2.024750in}}%
\pgfpathlineto{\pgfqpoint{-26.988998in}{2.195111in}}%
\pgfpathlineto{\pgfqpoint{-27.037714in}{2.204576in}}%
\pgfpathlineto{\pgfqpoint{-27.086555in}{2.228159in}}%
\pgfpathlineto{\pgfqpoint{-27.136341in}{2.194731in}}%
\pgfpathlineto{\pgfqpoint{-27.184903in}{2.266072in}}%
\pgfpathlineto{\pgfqpoint{-27.232862in}{2.229339in}}%
\pgfpathlineto{\pgfqpoint{-27.281821in}{2.219948in}}%
\pgfpathlineto{\pgfqpoint{-27.328705in}{2.280610in}}%
\pgfpathlineto{\pgfqpoint{-27.374936in}{2.287585in}}%
\pgfpathlineto{\pgfqpoint{-27.422670in}{2.242602in}}%
\pgfpathlineto{\pgfqpoint{-27.470237in}{2.177911in}}%
\pgfpathlineto{\pgfqpoint{-27.516652in}{2.304360in}}%
\pgfpathlineto{\pgfqpoint{-27.563341in}{2.275353in}}%
\pgfpathlineto{\pgfqpoint{-27.608047in}{2.423736in}}%
\pgfpathlineto{\pgfqpoint{-27.651646in}{2.359584in}}%
\pgfpathlineto{\pgfqpoint{-27.696819in}{2.354467in}}%
\pgfpathlineto{\pgfqpoint{-27.740871in}{2.335672in}}%
\pgfpathlineto{\pgfqpoint{-27.785329in}{2.381779in}}%
\pgfpathlineto{\pgfqpoint{-27.830427in}{2.376662in}}%
\pgfpathlineto{\pgfqpoint{-27.874405in}{2.401650in}}%
\pgfpathlineto{\pgfqpoint{-27.917812in}{2.414236in}}%
\pgfpathlineto{\pgfqpoint{-27.963188in}{2.293230in}}%
\pgfpathlineto{\pgfqpoint{-28.007858in}{2.396877in}}%
\pgfpathlineto{\pgfqpoint{-28.051883in}{2.373334in}}%
\pgfpathlineto{\pgfqpoint{-28.097071in}{2.354019in}}%
\pgfpathlineto{\pgfqpoint{-28.141536in}{2.265636in}}%
\pgfpathlineto{\pgfqpoint{-28.185953in}{2.392641in}}%
\pgfpathlineto{\pgfqpoint{-28.230974in}{2.420320in}}%
\pgfpathlineto{\pgfqpoint{-28.274355in}{2.359420in}}%
\pgfpathlineto{\pgfqpoint{-28.318373in}{2.234679in}}%
\pgfpathlineto{\pgfqpoint{-28.364238in}{2.375139in}}%
\pgfpathlineto{\pgfqpoint{-28.408130in}{2.340910in}}%
\pgfpathlineto{\pgfqpoint{-28.452363in}{2.324601in}}%
\pgfpathlineto{\pgfqpoint{-28.498901in}{2.236861in}}%
\pgfpathlineto{\pgfqpoint{-28.543533in}{2.325647in}}%
\pgfpathlineto{\pgfqpoint{-28.588250in}{1.738591in}}%
\pgfpathclose%
\pgfusepath{fill}%
\end{pgfscope}%
\begin{pgfscope}%
\pgfpathrectangle{\pgfqpoint{2.662073in}{0.773588in}}{\pgfqpoint{2.964025in}{5.415119in}}%
\pgfusepath{clip}%
\pgfsetbuttcap%
\pgfsetroundjoin%
\definecolor{currentfill}{rgb}{0.549020,0.337255,0.294118}%
\pgfsetfillcolor{currentfill}%
\pgfsetlinewidth{0.000000pt}%
\definecolor{currentstroke}{rgb}{0.000000,0.000000,0.000000}%
\pgfsetstrokecolor{currentstroke}%
\pgfsetdash{}{0pt}%
\pgfpathmoveto{\pgfqpoint{-28.588250in}{2.142129in}}%
\pgfpathlineto{\pgfqpoint{-28.588250in}{1.738591in}}%
\pgfpathlineto{\pgfqpoint{-28.543533in}{2.325647in}}%
\pgfpathlineto{\pgfqpoint{-28.498901in}{2.236861in}}%
\pgfpathlineto{\pgfqpoint{-28.452363in}{2.324601in}}%
\pgfpathlineto{\pgfqpoint{-28.408130in}{2.340910in}}%
\pgfpathlineto{\pgfqpoint{-28.364238in}{2.375139in}}%
\pgfpathlineto{\pgfqpoint{-28.318373in}{2.234679in}}%
\pgfpathlineto{\pgfqpoint{-28.274355in}{2.359420in}}%
\pgfpathlineto{\pgfqpoint{-28.230974in}{2.420320in}}%
\pgfpathlineto{\pgfqpoint{-28.185953in}{2.392641in}}%
\pgfpathlineto{\pgfqpoint{-28.141536in}{2.265636in}}%
\pgfpathlineto{\pgfqpoint{-28.097071in}{2.354019in}}%
\pgfpathlineto{\pgfqpoint{-28.051883in}{2.373334in}}%
\pgfpathlineto{\pgfqpoint{-28.007858in}{2.396877in}}%
\pgfpathlineto{\pgfqpoint{-27.963188in}{2.293230in}}%
\pgfpathlineto{\pgfqpoint{-27.917812in}{2.414236in}}%
\pgfpathlineto{\pgfqpoint{-27.874405in}{2.401650in}}%
\pgfpathlineto{\pgfqpoint{-27.830427in}{2.376662in}}%
\pgfpathlineto{\pgfqpoint{-27.785329in}{2.381779in}}%
\pgfpathlineto{\pgfqpoint{-27.740871in}{2.335672in}}%
\pgfpathlineto{\pgfqpoint{-27.696819in}{2.354467in}}%
\pgfpathlineto{\pgfqpoint{-27.651646in}{2.359584in}}%
\pgfpathlineto{\pgfqpoint{-27.608047in}{2.423736in}}%
\pgfpathlineto{\pgfqpoint{-27.563341in}{2.275353in}}%
\pgfpathlineto{\pgfqpoint{-27.516652in}{2.304360in}}%
\pgfpathlineto{\pgfqpoint{-27.470237in}{2.177911in}}%
\pgfpathlineto{\pgfqpoint{-27.422670in}{2.242602in}}%
\pgfpathlineto{\pgfqpoint{-27.374936in}{2.287585in}}%
\pgfpathlineto{\pgfqpoint{-27.328705in}{2.280610in}}%
\pgfpathlineto{\pgfqpoint{-27.281821in}{2.219948in}}%
\pgfpathlineto{\pgfqpoint{-27.232862in}{2.229339in}}%
\pgfpathlineto{\pgfqpoint{-27.184903in}{2.266072in}}%
\pgfpathlineto{\pgfqpoint{-27.136341in}{2.194731in}}%
\pgfpathlineto{\pgfqpoint{-27.086555in}{2.228159in}}%
\pgfpathlineto{\pgfqpoint{-27.037714in}{2.204576in}}%
\pgfpathlineto{\pgfqpoint{-26.988998in}{2.195111in}}%
\pgfpathlineto{\pgfqpoint{-26.933555in}{2.024750in}}%
\pgfpathlineto{\pgfqpoint{-26.880611in}{2.112084in}}%
\pgfpathlineto{\pgfqpoint{-26.830647in}{2.171250in}}%
\pgfpathlineto{\pgfqpoint{-26.781839in}{2.305652in}}%
\pgfpathlineto{\pgfqpoint{-26.735142in}{2.276180in}}%
\pgfpathlineto{\pgfqpoint{-26.688111in}{2.323786in}}%
\pgfpathlineto{\pgfqpoint{-26.640477in}{2.306144in}}%
\pgfpathlineto{\pgfqpoint{-26.595409in}{2.292757in}}%
\pgfpathlineto{\pgfqpoint{-26.549765in}{2.274534in}}%
\pgfpathlineto{\pgfqpoint{-26.503012in}{2.342913in}}%
\pgfpathlineto{\pgfqpoint{-26.458732in}{2.375345in}}%
\pgfpathlineto{\pgfqpoint{-26.414228in}{2.365948in}}%
\pgfpathlineto{\pgfqpoint{-26.368659in}{2.404176in}}%
\pgfpathlineto{\pgfqpoint{-26.323384in}{2.352393in}}%
\pgfpathlineto{\pgfqpoint{-26.277965in}{2.359459in}}%
\pgfpathlineto{\pgfqpoint{-26.230565in}{2.240058in}}%
\pgfpathlineto{\pgfqpoint{-26.185627in}{2.371658in}}%
\pgfpathlineto{\pgfqpoint{-26.140687in}{2.358697in}}%
\pgfpathlineto{\pgfqpoint{-26.095397in}{2.395267in}}%
\pgfpathlineto{\pgfqpoint{-26.049962in}{2.314984in}}%
\pgfpathlineto{\pgfqpoint{-26.004871in}{2.354180in}}%
\pgfpathlineto{\pgfqpoint{-25.957619in}{2.309584in}}%
\pgfpathlineto{\pgfqpoint{-25.913010in}{2.322586in}}%
\pgfpathlineto{\pgfqpoint{-25.868331in}{2.390510in}}%
\pgfpathlineto{\pgfqpoint{-25.821656in}{2.322218in}}%
\pgfpathlineto{\pgfqpoint{-25.776589in}{2.340637in}}%
\pgfpathlineto{\pgfqpoint{-25.731236in}{2.311073in}}%
\pgfpathlineto{\pgfqpoint{-25.684196in}{2.294335in}}%
\pgfpathlineto{\pgfqpoint{-25.638897in}{2.281275in}}%
\pgfpathlineto{\pgfqpoint{-25.593105in}{2.272352in}}%
\pgfpathlineto{\pgfqpoint{-25.546370in}{2.328600in}}%
\pgfpathlineto{\pgfqpoint{-25.501531in}{2.376055in}}%
\pgfpathlineto{\pgfqpoint{-25.455838in}{2.272708in}}%
\pgfpathlineto{\pgfqpoint{-25.409278in}{2.367974in}}%
\pgfpathlineto{\pgfqpoint{-25.364031in}{2.311396in}}%
\pgfpathlineto{\pgfqpoint{-25.318638in}{2.326455in}}%
\pgfpathlineto{\pgfqpoint{-25.271783in}{2.348997in}}%
\pgfpathlineto{\pgfqpoint{-25.227067in}{2.360627in}}%
\pgfpathlineto{\pgfqpoint{-25.181603in}{2.291067in}}%
\pgfpathlineto{\pgfqpoint{-25.133757in}{2.272653in}}%
\pgfpathlineto{\pgfqpoint{-25.088791in}{2.359011in}}%
\pgfpathlineto{\pgfqpoint{-25.043027in}{2.288248in}}%
\pgfpathlineto{\pgfqpoint{-24.995958in}{2.309445in}}%
\pgfpathlineto{\pgfqpoint{-24.950861in}{2.402740in}}%
\pgfpathlineto{\pgfqpoint{-24.905462in}{2.264512in}}%
\pgfpathlineto{\pgfqpoint{-24.858721in}{2.257669in}}%
\pgfpathlineto{\pgfqpoint{-24.813437in}{2.277965in}}%
\pgfpathlineto{\pgfqpoint{-24.768227in}{2.337717in}}%
\pgfpathlineto{\pgfqpoint{-24.722523in}{2.377694in}}%
\pgfpathlineto{\pgfqpoint{-24.677917in}{2.367505in}}%
\pgfpathlineto{\pgfqpoint{-24.632696in}{2.340546in}}%
\pgfpathlineto{\pgfqpoint{-24.586259in}{2.393133in}}%
\pgfpathlineto{\pgfqpoint{-24.540716in}{2.353522in}}%
\pgfpathlineto{\pgfqpoint{-24.494847in}{2.270967in}}%
\pgfpathlineto{\pgfqpoint{-24.447114in}{2.223770in}}%
\pgfpathlineto{\pgfqpoint{-24.401173in}{2.320591in}}%
\pgfpathlineto{\pgfqpoint{-24.355467in}{2.295788in}}%
\pgfpathlineto{\pgfqpoint{-24.308071in}{2.306542in}}%
\pgfpathlineto{\pgfqpoint{-24.262109in}{2.232536in}}%
\pgfpathlineto{\pgfqpoint{-24.216953in}{2.320747in}}%
\pgfpathlineto{\pgfqpoint{-24.169854in}{2.333026in}}%
\pgfpathlineto{\pgfqpoint{-24.123999in}{2.304795in}}%
\pgfpathlineto{\pgfqpoint{-24.078902in}{2.374804in}}%
\pgfpathlineto{\pgfqpoint{-24.032531in}{2.297753in}}%
\pgfpathlineto{\pgfqpoint{-23.986811in}{2.339578in}}%
\pgfpathlineto{\pgfqpoint{-23.941619in}{2.326998in}}%
\pgfpathlineto{\pgfqpoint{-23.895017in}{2.369110in}}%
\pgfpathlineto{\pgfqpoint{-23.849943in}{2.406439in}}%
\pgfpathlineto{\pgfqpoint{-23.804343in}{2.273113in}}%
\pgfpathlineto{\pgfqpoint{-23.757289in}{2.247028in}}%
\pgfpathlineto{\pgfqpoint{-23.711883in}{2.353354in}}%
\pgfpathlineto{\pgfqpoint{-23.666576in}{2.319036in}}%
\pgfpathlineto{\pgfqpoint{-23.620333in}{2.387220in}}%
\pgfpathlineto{\pgfqpoint{-23.575877in}{2.381697in}}%
\pgfpathlineto{\pgfqpoint{-23.529965in}{2.261696in}}%
\pgfpathlineto{\pgfqpoint{-23.482780in}{2.308873in}}%
\pgfpathlineto{\pgfqpoint{-23.437154in}{2.278529in}}%
\pgfpathlineto{\pgfqpoint{-23.391460in}{2.234958in}}%
\pgfpathlineto{\pgfqpoint{-23.344516in}{2.406366in}}%
\pgfpathlineto{\pgfqpoint{-23.298917in}{2.250652in}}%
\pgfpathlineto{\pgfqpoint{-23.252910in}{2.311575in}}%
\pgfpathlineto{\pgfqpoint{-23.205343in}{2.237468in}}%
\pgfpathlineto{\pgfqpoint{-23.159034in}{2.266056in}}%
\pgfpathlineto{\pgfqpoint{-23.112561in}{2.254185in}}%
\pgfpathlineto{\pgfqpoint{-23.065076in}{2.254446in}}%
\pgfpathlineto{\pgfqpoint{-23.018637in}{2.243112in}}%
\pgfpathlineto{\pgfqpoint{-22.972746in}{2.233471in}}%
\pgfpathlineto{\pgfqpoint{-22.924822in}{2.270987in}}%
\pgfpathlineto{\pgfqpoint{-22.878494in}{2.240583in}}%
\pgfpathlineto{\pgfqpoint{-22.833029in}{2.281819in}}%
\pgfpathlineto{\pgfqpoint{-22.786844in}{2.357634in}}%
\pgfpathlineto{\pgfqpoint{-22.742185in}{2.333142in}}%
\pgfpathlineto{\pgfqpoint{-22.697642in}{2.375017in}}%
\pgfpathlineto{\pgfqpoint{-22.651258in}{2.277162in}}%
\pgfpathlineto{\pgfqpoint{-22.606850in}{2.367685in}}%
\pgfpathlineto{\pgfqpoint{-22.562133in}{2.284579in}}%
\pgfpathlineto{\pgfqpoint{-22.516601in}{2.314775in}}%
\pgfpathlineto{\pgfqpoint{-22.471952in}{2.320511in}}%
\pgfpathlineto{\pgfqpoint{-22.427383in}{2.393311in}}%
\pgfpathlineto{\pgfqpoint{-22.380996in}{2.366754in}}%
\pgfpathlineto{\pgfqpoint{-22.336074in}{2.351361in}}%
\pgfpathlineto{\pgfqpoint{-22.291940in}{2.405300in}}%
\pgfpathlineto{\pgfqpoint{-22.246011in}{2.383905in}}%
\pgfpathlineto{\pgfqpoint{-22.201868in}{2.450307in}}%
\pgfpathlineto{\pgfqpoint{-22.158025in}{2.333442in}}%
\pgfpathlineto{\pgfqpoint{-22.112231in}{2.402816in}}%
\pgfpathlineto{\pgfqpoint{-22.067661in}{2.317625in}}%
\pgfpathlineto{\pgfqpoint{-22.022397in}{2.355753in}}%
\pgfpathlineto{\pgfqpoint{-21.976168in}{2.350045in}}%
\pgfpathlineto{\pgfqpoint{-21.930597in}{2.349659in}}%
\pgfpathlineto{\pgfqpoint{-21.885462in}{2.330493in}}%
\pgfpathlineto{\pgfqpoint{-21.839254in}{2.354507in}}%
\pgfpathlineto{\pgfqpoint{-21.794255in}{2.291838in}}%
\pgfpathlineto{\pgfqpoint{-21.749554in}{2.356484in}}%
\pgfpathlineto{\pgfqpoint{-21.703167in}{2.305719in}}%
\pgfpathlineto{\pgfqpoint{-21.657342in}{2.304006in}}%
\pgfpathlineto{\pgfqpoint{-21.612919in}{2.358435in}}%
\pgfpathlineto{\pgfqpoint{-21.566198in}{2.321312in}}%
\pgfpathlineto{\pgfqpoint{-21.520883in}{2.303547in}}%
\pgfpathlineto{\pgfqpoint{-21.475985in}{2.351596in}}%
\pgfpathlineto{\pgfqpoint{-21.430267in}{2.368915in}}%
\pgfpathlineto{\pgfqpoint{-21.385572in}{2.240229in}}%
\pgfpathlineto{\pgfqpoint{-21.340867in}{2.365018in}}%
\pgfpathlineto{\pgfqpoint{-21.294264in}{2.338666in}}%
\pgfpathlineto{\pgfqpoint{-21.249304in}{2.343950in}}%
\pgfpathlineto{\pgfqpoint{-21.204354in}{2.272682in}}%
\pgfpathlineto{\pgfqpoint{-21.158215in}{2.349197in}}%
\pgfpathlineto{\pgfqpoint{-21.113139in}{2.303852in}}%
\pgfpathlineto{\pgfqpoint{-21.067357in}{2.273525in}}%
\pgfpathlineto{\pgfqpoint{-21.021509in}{2.398401in}}%
\pgfpathlineto{\pgfqpoint{-20.976378in}{2.318752in}}%
\pgfpathlineto{\pgfqpoint{-20.932365in}{2.398000in}}%
\pgfpathlineto{\pgfqpoint{-20.886847in}{2.313127in}}%
\pgfpathlineto{\pgfqpoint{-20.842169in}{2.376578in}}%
\pgfpathlineto{\pgfqpoint{-20.797022in}{2.336609in}}%
\pgfpathlineto{\pgfqpoint{-20.750831in}{2.356202in}}%
\pgfpathlineto{\pgfqpoint{-20.705066in}{2.191512in}}%
\pgfpathlineto{\pgfqpoint{-20.658742in}{2.294828in}}%
\pgfpathlineto{\pgfqpoint{-20.611243in}{2.285186in}}%
\pgfpathlineto{\pgfqpoint{-20.565077in}{2.264133in}}%
\pgfpathlineto{\pgfqpoint{-20.519904in}{2.269668in}}%
\pgfpathlineto{\pgfqpoint{-20.474014in}{2.375131in}}%
\pgfpathlineto{\pgfqpoint{-20.428221in}{2.269362in}}%
\pgfpathlineto{\pgfqpoint{-20.382199in}{2.257547in}}%
\pgfpathlineto{\pgfqpoint{-20.335981in}{2.367234in}}%
\pgfpathlineto{\pgfqpoint{-20.290419in}{2.335942in}}%
\pgfpathlineto{\pgfqpoint{-20.244875in}{2.295788in}}%
\pgfpathlineto{\pgfqpoint{-20.197814in}{2.368490in}}%
\pgfpathlineto{\pgfqpoint{-20.153038in}{2.417828in}}%
\pgfpathlineto{\pgfqpoint{-20.107707in}{2.335560in}}%
\pgfpathlineto{\pgfqpoint{-20.061722in}{2.399221in}}%
\pgfpathlineto{\pgfqpoint{-20.016506in}{2.319684in}}%
\pgfpathlineto{\pgfqpoint{-19.970967in}{2.314648in}}%
\pgfpathlineto{\pgfqpoint{-19.924139in}{2.300804in}}%
\pgfpathlineto{\pgfqpoint{-19.878562in}{2.292822in}}%
\pgfpathlineto{\pgfqpoint{-19.833948in}{2.394099in}}%
\pgfpathlineto{\pgfqpoint{-19.787808in}{2.314462in}}%
\pgfpathlineto{\pgfqpoint{-19.742024in}{2.320379in}}%
\pgfpathlineto{\pgfqpoint{-19.696649in}{2.353123in}}%
\pgfpathlineto{\pgfqpoint{-19.650361in}{2.323431in}}%
\pgfpathlineto{\pgfqpoint{-19.604824in}{2.303450in}}%
\pgfpathlineto{\pgfqpoint{-19.559546in}{2.328474in}}%
\pgfpathlineto{\pgfqpoint{-19.512973in}{2.325186in}}%
\pgfpathlineto{\pgfqpoint{-19.467303in}{2.261142in}}%
\pgfpathlineto{\pgfqpoint{-19.421584in}{2.297187in}}%
\pgfpathlineto{\pgfqpoint{-19.374230in}{2.268469in}}%
\pgfpathlineto{\pgfqpoint{-19.328532in}{2.305435in}}%
\pgfpathlineto{\pgfqpoint{-19.283047in}{2.261364in}}%
\pgfpathlineto{\pgfqpoint{-19.234797in}{2.230054in}}%
\pgfpathlineto{\pgfqpoint{-19.188899in}{2.359779in}}%
\pgfpathlineto{\pgfqpoint{-19.143544in}{2.276235in}}%
\pgfpathlineto{\pgfqpoint{-19.097062in}{2.291417in}}%
\pgfpathlineto{\pgfqpoint{-19.050812in}{2.240312in}}%
\pgfpathlineto{\pgfqpoint{-19.004576in}{2.293680in}}%
\pgfpathlineto{\pgfqpoint{-18.956783in}{2.294777in}}%
\pgfpathlineto{\pgfqpoint{-18.910503in}{2.274485in}}%
\pgfpathlineto{\pgfqpoint{-18.865040in}{2.287380in}}%
\pgfpathlineto{\pgfqpoint{-18.817991in}{2.346468in}}%
\pgfpathlineto{\pgfqpoint{-18.772260in}{2.264712in}}%
\pgfpathlineto{\pgfqpoint{-18.726033in}{2.222199in}}%
\pgfpathlineto{\pgfqpoint{-18.678684in}{2.250711in}}%
\pgfpathlineto{\pgfqpoint{-18.632978in}{2.330547in}}%
\pgfpathlineto{\pgfqpoint{-18.587378in}{2.302467in}}%
\pgfpathlineto{\pgfqpoint{-18.540355in}{2.275306in}}%
\pgfpathlineto{\pgfqpoint{-18.494344in}{2.284564in}}%
\pgfpathlineto{\pgfqpoint{-18.448764in}{2.301035in}}%
\pgfpathlineto{\pgfqpoint{-18.402000in}{2.282506in}}%
\pgfpathlineto{\pgfqpoint{-18.356748in}{2.356543in}}%
\pgfpathlineto{\pgfqpoint{-18.311512in}{2.335174in}}%
\pgfpathlineto{\pgfqpoint{-18.264167in}{2.258422in}}%
\pgfpathlineto{\pgfqpoint{-18.218674in}{2.399865in}}%
\pgfpathlineto{\pgfqpoint{-18.172791in}{2.260071in}}%
\pgfpathlineto{\pgfqpoint{-18.124795in}{2.276617in}}%
\pgfpathlineto{\pgfqpoint{-18.078784in}{2.360674in}}%
\pgfpathlineto{\pgfqpoint{-18.032389in}{2.235310in}}%
\pgfpathlineto{\pgfqpoint{-17.984102in}{2.332692in}}%
\pgfpathlineto{\pgfqpoint{-17.937614in}{2.259828in}}%
\pgfpathlineto{\pgfqpoint{-17.891271in}{2.291080in}}%
\pgfpathlineto{\pgfqpoint{-17.843480in}{2.357277in}}%
\pgfpathlineto{\pgfqpoint{-17.797346in}{2.230089in}}%
\pgfpathlineto{\pgfqpoint{-17.751556in}{2.229229in}}%
\pgfpathlineto{\pgfqpoint{-17.704093in}{2.359958in}}%
\pgfpathlineto{\pgfqpoint{-17.658538in}{2.333441in}}%
\pgfpathlineto{\pgfqpoint{-17.613251in}{2.306760in}}%
\pgfpathlineto{\pgfqpoint{-17.566739in}{2.285077in}}%
\pgfpathlineto{\pgfqpoint{-17.521284in}{2.355287in}}%
\pgfpathlineto{\pgfqpoint{-17.474644in}{2.203921in}}%
\pgfpathlineto{\pgfqpoint{-17.427002in}{2.350595in}}%
\pgfpathlineto{\pgfqpoint{-17.381732in}{2.324057in}}%
\pgfpathlineto{\pgfqpoint{-17.336276in}{2.301880in}}%
\pgfpathlineto{\pgfqpoint{-17.289523in}{2.317891in}}%
\pgfpathlineto{\pgfqpoint{-17.244258in}{2.351371in}}%
\pgfpathlineto{\pgfqpoint{-17.198221in}{2.280826in}}%
\pgfpathlineto{\pgfqpoint{-17.150161in}{2.221952in}}%
\pgfpathlineto{\pgfqpoint{-17.103874in}{2.273976in}}%
\pgfpathlineto{\pgfqpoint{-17.058009in}{2.257529in}}%
\pgfpathlineto{\pgfqpoint{-17.010884in}{2.349502in}}%
\pgfpathlineto{\pgfqpoint{-16.964774in}{2.358343in}}%
\pgfpathlineto{\pgfqpoint{-16.918692in}{2.296669in}}%
\pgfpathlineto{\pgfqpoint{-16.871664in}{2.340897in}}%
\pgfpathlineto{\pgfqpoint{-16.825434in}{2.203625in}}%
\pgfpathlineto{\pgfqpoint{-16.778530in}{2.239852in}}%
\pgfpathlineto{\pgfqpoint{-16.730721in}{2.325183in}}%
\pgfpathlineto{\pgfqpoint{-16.684310in}{2.279136in}}%
\pgfpathlineto{\pgfqpoint{-16.637120in}{2.253182in}}%
\pgfpathlineto{\pgfqpoint{-16.589239in}{2.259754in}}%
\pgfpathlineto{\pgfqpoint{-16.542990in}{2.234339in}}%
\pgfpathlineto{\pgfqpoint{-16.496157in}{2.289046in}}%
\pgfpathlineto{\pgfqpoint{-16.449347in}{2.386807in}}%
\pgfpathlineto{\pgfqpoint{-16.402550in}{2.304244in}}%
\pgfpathlineto{\pgfqpoint{-16.355646in}{2.298564in}}%
\pgfpathlineto{\pgfqpoint{-16.308871in}{2.323533in}}%
\pgfpathlineto{\pgfqpoint{-16.262806in}{2.262921in}}%
\pgfpathlineto{\pgfqpoint{-16.217070in}{2.384526in}}%
\pgfpathlineto{\pgfqpoint{-16.169427in}{2.235117in}}%
\pgfpathlineto{\pgfqpoint{-16.123165in}{2.338157in}}%
\pgfpathlineto{\pgfqpoint{-16.077171in}{2.312159in}}%
\pgfpathlineto{\pgfqpoint{-16.029628in}{2.385490in}}%
\pgfpathlineto{\pgfqpoint{-15.984286in}{2.323804in}}%
\pgfpathlineto{\pgfqpoint{-15.938076in}{2.240374in}}%
\pgfpathlineto{\pgfqpoint{-15.890419in}{2.299059in}}%
\pgfpathlineto{\pgfqpoint{-15.844500in}{2.277760in}}%
\pgfpathlineto{\pgfqpoint{-15.798897in}{2.335101in}}%
\pgfpathlineto{\pgfqpoint{-15.750800in}{2.278245in}}%
\pgfpathlineto{\pgfqpoint{-15.703994in}{2.232128in}}%
\pgfpathlineto{\pgfqpoint{-15.657002in}{2.300755in}}%
\pgfpathlineto{\pgfqpoint{-15.608704in}{2.242575in}}%
\pgfpathlineto{\pgfqpoint{-15.560849in}{2.178671in}}%
\pgfpathlineto{\pgfqpoint{-15.512535in}{2.212443in}}%
\pgfpathlineto{\pgfqpoint{-15.463539in}{2.226374in}}%
\pgfpathlineto{\pgfqpoint{-15.416792in}{2.229832in}}%
\pgfpathlineto{\pgfqpoint{-15.369802in}{2.286673in}}%
\pgfpathlineto{\pgfqpoint{-15.321194in}{2.232270in}}%
\pgfpathlineto{\pgfqpoint{-15.273998in}{2.229533in}}%
\pgfpathlineto{\pgfqpoint{-15.225998in}{2.182080in}}%
\pgfpathlineto{\pgfqpoint{-15.176769in}{2.193278in}}%
\pgfpathlineto{\pgfqpoint{-15.130553in}{2.321238in}}%
\pgfpathlineto{\pgfqpoint{-15.083728in}{2.274703in}}%
\pgfpathlineto{\pgfqpoint{-15.035921in}{2.276334in}}%
\pgfpathlineto{\pgfqpoint{-14.989019in}{2.209372in}}%
\pgfpathlineto{\pgfqpoint{-14.942416in}{2.346640in}}%
\pgfpathlineto{\pgfqpoint{-14.894419in}{2.254543in}}%
\pgfpathlineto{\pgfqpoint{-14.847516in}{2.291021in}}%
\pgfpathlineto{\pgfqpoint{-14.800669in}{2.246168in}}%
\pgfpathlineto{\pgfqpoint{-14.752153in}{2.221244in}}%
\pgfpathlineto{\pgfqpoint{-14.705919in}{2.325095in}}%
\pgfpathlineto{\pgfqpoint{-14.659388in}{2.231589in}}%
\pgfpathlineto{\pgfqpoint{-14.611394in}{2.259349in}}%
\pgfpathlineto{\pgfqpoint{-14.564671in}{2.359201in}}%
\pgfpathlineto{\pgfqpoint{-14.517904in}{2.282858in}}%
\pgfpathlineto{\pgfqpoint{-14.469907in}{2.196403in}}%
\pgfpathlineto{\pgfqpoint{-14.422978in}{2.252252in}}%
\pgfpathlineto{\pgfqpoint{-14.375931in}{2.267582in}}%
\pgfpathlineto{\pgfqpoint{-14.327371in}{2.198546in}}%
\pgfpathlineto{\pgfqpoint{-14.280667in}{2.314598in}}%
\pgfpathlineto{\pgfqpoint{-14.233487in}{2.295575in}}%
\pgfpathlineto{\pgfqpoint{-14.184689in}{2.240614in}}%
\pgfpathlineto{\pgfqpoint{-14.137283in}{2.236209in}}%
\pgfpathlineto{\pgfqpoint{-14.090279in}{2.280354in}}%
\pgfpathlineto{\pgfqpoint{-14.042263in}{2.307810in}}%
\pgfpathlineto{\pgfqpoint{-13.996250in}{2.293893in}}%
\pgfpathlineto{\pgfqpoint{-13.950094in}{2.260363in}}%
\pgfpathlineto{\pgfqpoint{-13.901287in}{2.218369in}}%
\pgfpathlineto{\pgfqpoint{-13.853816in}{2.219452in}}%
\pgfpathlineto{\pgfqpoint{-13.806330in}{2.258501in}}%
\pgfpathlineto{\pgfqpoint{-13.757485in}{2.211679in}}%
\pgfpathlineto{\pgfqpoint{-13.710126in}{2.301145in}}%
\pgfpathlineto{\pgfqpoint{-13.663331in}{2.300535in}}%
\pgfpathlineto{\pgfqpoint{-13.615991in}{2.313157in}}%
\pgfpathlineto{\pgfqpoint{-13.568873in}{2.287207in}}%
\pgfpathlineto{\pgfqpoint{-13.522553in}{2.332872in}}%
\pgfpathlineto{\pgfqpoint{-13.474335in}{2.228192in}}%
\pgfpathlineto{\pgfqpoint{-13.428346in}{2.402601in}}%
\pgfpathlineto{\pgfqpoint{-13.382220in}{2.336191in}}%
\pgfpathlineto{\pgfqpoint{-13.334326in}{2.332418in}}%
\pgfpathlineto{\pgfqpoint{-13.287448in}{2.270580in}}%
\pgfpathlineto{\pgfqpoint{-13.239807in}{2.216517in}}%
\pgfpathlineto{\pgfqpoint{-13.190970in}{2.205221in}}%
\pgfpathlineto{\pgfqpoint{-13.144068in}{2.276775in}}%
\pgfpathlineto{\pgfqpoint{-13.096328in}{2.219318in}}%
\pgfpathlineto{\pgfqpoint{-13.046608in}{2.259149in}}%
\pgfpathlineto{\pgfqpoint{-12.998480in}{2.174071in}}%
\pgfpathlineto{\pgfqpoint{-12.949523in}{2.137013in}}%
\pgfpathlineto{\pgfqpoint{-12.899585in}{2.250525in}}%
\pgfpathlineto{\pgfqpoint{-12.852526in}{2.302678in}}%
\pgfpathlineto{\pgfqpoint{-12.805010in}{2.238151in}}%
\pgfpathlineto{\pgfqpoint{-12.755866in}{2.266736in}}%
\pgfpathlineto{\pgfqpoint{-12.708579in}{2.244778in}}%
\pgfpathlineto{\pgfqpoint{-12.661578in}{2.247475in}}%
\pgfpathlineto{\pgfqpoint{-12.612433in}{2.227898in}}%
\pgfpathlineto{\pgfqpoint{-12.565107in}{2.222802in}}%
\pgfpathlineto{\pgfqpoint{-12.517880in}{2.244193in}}%
\pgfpathlineto{\pgfqpoint{-12.469843in}{2.332512in}}%
\pgfpathlineto{\pgfqpoint{-12.422612in}{2.263073in}}%
\pgfpathlineto{\pgfqpoint{-12.375560in}{2.251846in}}%
\pgfpathlineto{\pgfqpoint{-12.326627in}{2.170799in}}%
\pgfpathlineto{\pgfqpoint{-12.278675in}{2.233795in}}%
\pgfpathlineto{\pgfqpoint{-12.231638in}{2.297791in}}%
\pgfpathlineto{\pgfqpoint{-12.182471in}{2.246982in}}%
\pgfpathlineto{\pgfqpoint{-12.135286in}{2.264188in}}%
\pgfpathlineto{\pgfqpoint{-12.087205in}{2.177400in}}%
\pgfpathlineto{\pgfqpoint{-12.038569in}{2.273193in}}%
\pgfpathlineto{\pgfqpoint{-11.991431in}{2.248043in}}%
\pgfpathlineto{\pgfqpoint{-11.943783in}{2.234768in}}%
\pgfpathlineto{\pgfqpoint{-11.895021in}{2.288470in}}%
\pgfpathlineto{\pgfqpoint{-11.847736in}{2.219811in}}%
\pgfpathlineto{\pgfqpoint{-11.800403in}{2.379567in}}%
\pgfpathlineto{\pgfqpoint{-11.751492in}{2.295602in}}%
\pgfpathlineto{\pgfqpoint{-11.704231in}{2.358516in}}%
\pgfpathlineto{\pgfqpoint{-11.656418in}{2.181121in}}%
\pgfpathlineto{\pgfqpoint{-11.607131in}{2.180861in}}%
\pgfpathlineto{\pgfqpoint{-11.559336in}{2.181075in}}%
\pgfpathlineto{\pgfqpoint{-11.511546in}{2.265803in}}%
\pgfpathlineto{\pgfqpoint{-11.463260in}{2.295885in}}%
\pgfpathlineto{\pgfqpoint{-11.415235in}{2.245278in}}%
\pgfpathlineto{\pgfqpoint{-11.366120in}{2.099061in}}%
\pgfpathlineto{\pgfqpoint{-11.316622in}{2.220034in}}%
\pgfpathlineto{\pgfqpoint{-11.268838in}{2.230183in}}%
\pgfpathlineto{\pgfqpoint{-11.221741in}{2.303508in}}%
\pgfpathlineto{\pgfqpoint{-11.172812in}{2.208562in}}%
\pgfpathlineto{\pgfqpoint{-11.125274in}{2.246190in}}%
\pgfpathlineto{\pgfqpoint{-11.077704in}{2.227451in}}%
\pgfpathlineto{\pgfqpoint{-11.028607in}{2.210510in}}%
\pgfpathlineto{\pgfqpoint{-10.980903in}{2.249085in}}%
\pgfpathlineto{\pgfqpoint{-10.933861in}{2.172965in}}%
\pgfpathlineto{\pgfqpoint{-10.885230in}{2.249891in}}%
\pgfpathlineto{\pgfqpoint{-10.837686in}{2.240166in}}%
\pgfpathlineto{\pgfqpoint{-10.790422in}{2.207052in}}%
\pgfpathlineto{\pgfqpoint{-10.741507in}{2.233839in}}%
\pgfpathlineto{\pgfqpoint{-10.693023in}{2.186383in}}%
\pgfpathlineto{\pgfqpoint{-10.644481in}{2.223418in}}%
\pgfpathlineto{\pgfqpoint{-10.594934in}{2.175746in}}%
\pgfpathlineto{\pgfqpoint{-10.546833in}{2.255904in}}%
\pgfpathlineto{\pgfqpoint{-10.497957in}{2.195093in}}%
\pgfpathlineto{\pgfqpoint{-10.448474in}{2.243100in}}%
\pgfpathlineto{\pgfqpoint{-10.401144in}{2.256048in}}%
\pgfpathlineto{\pgfqpoint{-10.352351in}{2.192082in}}%
\pgfpathlineto{\pgfqpoint{-10.301938in}{2.100303in}}%
\pgfpathlineto{\pgfqpoint{-10.253421in}{2.240721in}}%
\pgfpathlineto{\pgfqpoint{-10.204843in}{2.175947in}}%
\pgfpathlineto{\pgfqpoint{-10.154319in}{2.158168in}}%
\pgfpathlineto{\pgfqpoint{-10.106404in}{2.293533in}}%
\pgfpathlineto{\pgfqpoint{-10.058316in}{2.241679in}}%
\pgfpathlineto{\pgfqpoint{-10.009255in}{2.239043in}}%
\pgfpathlineto{\pgfqpoint{-9.960967in}{2.227549in}}%
\pgfpathlineto{\pgfqpoint{-9.912991in}{2.190874in}}%
\pgfpathlineto{\pgfqpoint{-9.863853in}{2.234019in}}%
\pgfpathlineto{\pgfqpoint{-9.816398in}{2.259987in}}%
\pgfpathlineto{\pgfqpoint{-9.768438in}{2.182953in}}%
\pgfpathlineto{\pgfqpoint{-9.719205in}{2.243279in}}%
\pgfpathlineto{\pgfqpoint{-9.671880in}{2.275859in}}%
\pgfpathlineto{\pgfqpoint{-9.624467in}{2.193047in}}%
\pgfpathlineto{\pgfqpoint{-9.575433in}{2.223149in}}%
\pgfpathlineto{\pgfqpoint{-9.527213in}{2.236420in}}%
\pgfpathlineto{\pgfqpoint{-9.479530in}{2.281059in}}%
\pgfpathlineto{\pgfqpoint{-9.430893in}{2.296405in}}%
\pgfpathlineto{\pgfqpoint{-9.382453in}{2.175230in}}%
\pgfpathlineto{\pgfqpoint{-9.332833in}{2.243716in}}%
\pgfpathlineto{\pgfqpoint{-9.281795in}{2.161230in}}%
\pgfpathlineto{\pgfqpoint{-9.232466in}{2.162851in}}%
\pgfpathlineto{\pgfqpoint{-9.182531in}{2.192230in}}%
\pgfpathlineto{\pgfqpoint{-9.130552in}{2.107988in}}%
\pgfpathlineto{\pgfqpoint{-9.080221in}{2.134136in}}%
\pgfpathlineto{\pgfqpoint{-9.030683in}{2.199386in}}%
\pgfpathlineto{\pgfqpoint{-8.979731in}{2.188387in}}%
\pgfpathlineto{\pgfqpoint{-8.930541in}{2.155524in}}%
\pgfpathlineto{\pgfqpoint{-8.880937in}{2.258629in}}%
\pgfpathlineto{\pgfqpoint{-8.829855in}{2.221965in}}%
\pgfpathlineto{\pgfqpoint{-8.780578in}{2.126089in}}%
\pgfpathlineto{\pgfqpoint{-8.731477in}{2.192399in}}%
\pgfpathlineto{\pgfqpoint{-8.681529in}{2.268763in}}%
\pgfpathlineto{\pgfqpoint{-8.632588in}{2.212760in}}%
\pgfpathlineto{\pgfqpoint{-8.583181in}{2.283576in}}%
\pgfpathlineto{\pgfqpoint{-8.532027in}{2.134033in}}%
\pgfpathlineto{\pgfqpoint{-8.481981in}{2.166381in}}%
\pgfpathlineto{\pgfqpoint{-8.432560in}{2.212485in}}%
\pgfpathlineto{\pgfqpoint{-8.382452in}{2.185814in}}%
\pgfpathlineto{\pgfqpoint{-8.332630in}{2.155531in}}%
\pgfpathlineto{\pgfqpoint{-8.283775in}{2.241324in}}%
\pgfpathlineto{\pgfqpoint{-8.233427in}{2.201026in}}%
\pgfpathlineto{\pgfqpoint{-8.184462in}{2.181293in}}%
\pgfpathlineto{\pgfqpoint{-8.135490in}{2.245041in}}%
\pgfpathlineto{\pgfqpoint{-8.085190in}{2.191708in}}%
\pgfpathlineto{\pgfqpoint{-8.035792in}{2.208270in}}%
\pgfpathlineto{\pgfqpoint{-7.986412in}{2.175744in}}%
\pgfpathlineto{\pgfqpoint{-7.935315in}{2.168605in}}%
\pgfpathlineto{\pgfqpoint{-7.885122in}{2.150548in}}%
\pgfpathlineto{\pgfqpoint{-7.835188in}{2.167679in}}%
\pgfpathlineto{\pgfqpoint{-7.783607in}{2.160895in}}%
\pgfpathlineto{\pgfqpoint{-7.733870in}{2.218841in}}%
\pgfpathlineto{\pgfqpoint{-7.685281in}{2.213775in}}%
\pgfpathlineto{\pgfqpoint{-7.634521in}{2.227472in}}%
\pgfpathlineto{\pgfqpoint{-7.585672in}{2.226448in}}%
\pgfpathlineto{\pgfqpoint{-7.536566in}{2.230011in}}%
\pgfpathlineto{\pgfqpoint{-7.485312in}{2.128498in}}%
\pgfpathlineto{\pgfqpoint{-7.435602in}{2.161953in}}%
\pgfpathlineto{\pgfqpoint{-7.386343in}{2.276898in}}%
\pgfpathlineto{\pgfqpoint{-7.335791in}{2.153076in}}%
\pgfpathlineto{\pgfqpoint{-7.286238in}{2.258881in}}%
\pgfpathlineto{\pgfqpoint{-7.237328in}{2.251963in}}%
\pgfpathlineto{\pgfqpoint{-7.185558in}{2.138280in}}%
\pgfpathlineto{\pgfqpoint{-7.135511in}{2.185196in}}%
\pgfpathlineto{\pgfqpoint{-7.085212in}{2.094855in}}%
\pgfpathlineto{\pgfqpoint{-7.033318in}{2.157190in}}%
\pgfpathlineto{\pgfqpoint{-6.984061in}{2.133288in}}%
\pgfpathlineto{\pgfqpoint{-6.934169in}{2.206816in}}%
\pgfpathlineto{\pgfqpoint{-6.882593in}{2.135991in}}%
\pgfpathlineto{\pgfqpoint{-6.832072in}{2.136690in}}%
\pgfpathlineto{\pgfqpoint{-6.782488in}{2.179999in}}%
\pgfpathlineto{\pgfqpoint{-6.731323in}{2.180208in}}%
\pgfpathlineto{\pgfqpoint{-6.682747in}{2.305817in}}%
\pgfpathlineto{\pgfqpoint{-6.634445in}{2.248709in}}%
\pgfpathlineto{\pgfqpoint{-6.582317in}{2.168087in}}%
\pgfpathlineto{\pgfqpoint{-6.532179in}{2.158422in}}%
\pgfpathlineto{\pgfqpoint{-6.481910in}{2.104244in}}%
\pgfpathlineto{\pgfqpoint{-6.429787in}{2.173476in}}%
\pgfpathlineto{\pgfqpoint{-6.380124in}{2.146954in}}%
\pgfpathlineto{\pgfqpoint{-6.330959in}{2.235002in}}%
\pgfpathlineto{\pgfqpoint{-6.279011in}{2.079790in}}%
\pgfpathlineto{\pgfqpoint{-6.228389in}{2.136082in}}%
\pgfpathlineto{\pgfqpoint{-6.177451in}{2.124981in}}%
\pgfpathlineto{\pgfqpoint{-6.125420in}{2.152878in}}%
\pgfpathlineto{\pgfqpoint{-6.075227in}{2.148300in}}%
\pgfpathlineto{\pgfqpoint{-6.024836in}{2.170140in}}%
\pgfpathlineto{\pgfqpoint{-5.974236in}{2.198735in}}%
\pgfpathlineto{\pgfqpoint{-5.924128in}{2.157637in}}%
\pgfpathlineto{\pgfqpoint{-5.874278in}{2.137407in}}%
\pgfpathlineto{\pgfqpoint{-5.822716in}{2.143388in}}%
\pgfpathlineto{\pgfqpoint{-5.774223in}{2.282936in}}%
\pgfpathlineto{\pgfqpoint{-5.724570in}{2.159418in}}%
\pgfpathlineto{\pgfqpoint{-5.673071in}{2.221155in}}%
\pgfpathlineto{\pgfqpoint{-5.623816in}{2.182126in}}%
\pgfpathlineto{\pgfqpoint{-5.574633in}{2.259241in}}%
\pgfpathlineto{\pgfqpoint{-5.523472in}{2.175857in}}%
\pgfpathlineto{\pgfqpoint{-5.473058in}{2.119330in}}%
\pgfpathlineto{\pgfqpoint{-5.421631in}{2.116542in}}%
\pgfpathlineto{\pgfqpoint{-5.369703in}{2.201493in}}%
\pgfpathlineto{\pgfqpoint{-5.319212in}{2.162367in}}%
\pgfpathlineto{\pgfqpoint{-5.267508in}{2.144171in}}%
\pgfpathlineto{\pgfqpoint{-5.215272in}{2.170742in}}%
\pgfpathlineto{\pgfqpoint{-5.165135in}{2.157432in}}%
\pgfpathlineto{\pgfqpoint{-5.114869in}{2.149127in}}%
\pgfpathlineto{\pgfqpoint{-5.064238in}{2.211605in}}%
\pgfpathlineto{\pgfqpoint{-5.014255in}{2.208071in}}%
\pgfpathlineto{\pgfqpoint{-4.963670in}{2.160039in}}%
\pgfpathlineto{\pgfqpoint{-4.911765in}{2.139169in}}%
\pgfpathlineto{\pgfqpoint{-4.861887in}{2.209164in}}%
\pgfpathlineto{\pgfqpoint{-4.812147in}{2.204872in}}%
\pgfpathlineto{\pgfqpoint{-4.759589in}{2.072454in}}%
\pgfpathlineto{\pgfqpoint{-4.709756in}{2.191465in}}%
\pgfpathlineto{\pgfqpoint{-4.659647in}{2.192402in}}%
\pgfpathlineto{\pgfqpoint{-4.607841in}{2.187482in}}%
\pgfpathlineto{\pgfqpoint{-4.558006in}{2.234647in}}%
\pgfpathlineto{\pgfqpoint{-4.508002in}{2.139836in}}%
\pgfpathlineto{\pgfqpoint{-4.457276in}{2.240633in}}%
\pgfpathlineto{\pgfqpoint{-4.407934in}{2.200234in}}%
\pgfpathlineto{\pgfqpoint{-4.357515in}{2.166090in}}%
\pgfpathlineto{\pgfqpoint{-4.305607in}{2.156690in}}%
\pgfpathlineto{\pgfqpoint{-4.255103in}{2.169034in}}%
\pgfpathlineto{\pgfqpoint{-4.204591in}{2.148863in}}%
\pgfpathlineto{\pgfqpoint{-4.153202in}{2.198963in}}%
\pgfpathlineto{\pgfqpoint{-4.103923in}{2.174410in}}%
\pgfpathlineto{\pgfqpoint{-4.053537in}{2.124537in}}%
\pgfpathlineto{\pgfqpoint{-4.002524in}{2.199798in}}%
\pgfpathlineto{\pgfqpoint{-3.952191in}{2.179510in}}%
\pgfpathlineto{\pgfqpoint{-3.902764in}{2.187707in}}%
\pgfpathlineto{\pgfqpoint{-3.850587in}{2.107499in}}%
\pgfpathlineto{\pgfqpoint{-3.799759in}{2.087238in}}%
\pgfpathlineto{\pgfqpoint{-3.748633in}{2.153196in}}%
\pgfpathlineto{\pgfqpoint{-3.696993in}{2.130566in}}%
\pgfpathlineto{\pgfqpoint{-3.646072in}{2.113429in}}%
\pgfpathlineto{\pgfqpoint{-3.595859in}{2.177297in}}%
\pgfpathlineto{\pgfqpoint{-3.544176in}{2.202252in}}%
\pgfpathlineto{\pgfqpoint{-3.494154in}{2.129676in}}%
\pgfpathlineto{\pgfqpoint{-3.444140in}{2.189113in}}%
\pgfpathlineto{\pgfqpoint{-3.392015in}{2.161647in}}%
\pgfpathlineto{\pgfqpoint{-3.341930in}{2.172740in}}%
\pgfpathlineto{\pgfqpoint{-3.292250in}{2.172470in}}%
\pgfpathlineto{\pgfqpoint{-3.241308in}{2.202460in}}%
\pgfpathlineto{\pgfqpoint{-3.190882in}{2.138438in}}%
\pgfpathlineto{\pgfqpoint{-3.140428in}{2.188680in}}%
\pgfpathlineto{\pgfqpoint{-3.088627in}{2.140127in}}%
\pgfpathlineto{\pgfqpoint{-3.039557in}{2.210846in}}%
\pgfpathlineto{\pgfqpoint{-2.990316in}{2.216585in}}%
\pgfpathlineto{\pgfqpoint{-2.938490in}{2.147600in}}%
\pgfpathlineto{\pgfqpoint{-2.887081in}{2.073278in}}%
\pgfpathlineto{\pgfqpoint{-2.836699in}{2.196254in}}%
\pgfpathlineto{\pgfqpoint{-2.784277in}{2.125000in}}%
\pgfpathlineto{\pgfqpoint{-2.732985in}{2.155346in}}%
\pgfpathlineto{\pgfqpoint{-2.681969in}{2.127344in}}%
\pgfpathlineto{\pgfqpoint{-2.629697in}{2.167275in}}%
\pgfpathlineto{\pgfqpoint{-2.578972in}{2.169425in}}%
\pgfpathlineto{\pgfqpoint{-2.528440in}{2.142711in}}%
\pgfpathlineto{\pgfqpoint{-2.476616in}{2.167554in}}%
\pgfpathlineto{\pgfqpoint{-2.425397in}{2.126317in}}%
\pgfpathlineto{\pgfqpoint{-2.375607in}{2.206184in}}%
\pgfpathlineto{\pgfqpoint{-2.323084in}{2.127542in}}%
\pgfpathlineto{\pgfqpoint{-2.272195in}{2.191289in}}%
\pgfpathlineto{\pgfqpoint{-2.220871in}{2.106034in}}%
\pgfpathlineto{\pgfqpoint{-2.167559in}{2.125482in}}%
\pgfpathlineto{\pgfqpoint{-2.116467in}{2.127160in}}%
\pgfpathlineto{\pgfqpoint{-2.064986in}{2.084956in}}%
\pgfpathlineto{\pgfqpoint{-2.013282in}{2.262046in}}%
\pgfpathlineto{\pgfqpoint{-1.963080in}{2.151542in}}%
\pgfpathlineto{\pgfqpoint{-1.912972in}{2.149431in}}%
\pgfpathlineto{\pgfqpoint{-1.860296in}{2.059972in}}%
\pgfpathlineto{\pgfqpoint{-1.809967in}{2.193467in}}%
\pgfpathlineto{\pgfqpoint{-1.759793in}{2.188092in}}%
\pgfpathlineto{\pgfqpoint{-1.707851in}{2.202638in}}%
\pgfpathlineto{\pgfqpoint{-1.657574in}{2.169913in}}%
\pgfpathlineto{\pgfqpoint{-1.606814in}{2.127843in}}%
\pgfpathlineto{\pgfqpoint{-1.553079in}{2.102078in}}%
\pgfpathlineto{\pgfqpoint{-1.502359in}{2.171648in}}%
\pgfpathlineto{\pgfqpoint{-1.451018in}{2.142057in}}%
\pgfpathlineto{\pgfqpoint{-1.397583in}{2.148509in}}%
\pgfpathlineto{\pgfqpoint{-1.346574in}{2.133339in}}%
\pgfpathlineto{\pgfqpoint{-1.295495in}{2.154372in}}%
\pgfpathlineto{\pgfqpoint{-1.242201in}{2.103896in}}%
\pgfpathlineto{\pgfqpoint{-1.191111in}{2.194715in}}%
\pgfpathlineto{\pgfqpoint{-1.140825in}{2.152372in}}%
\pgfpathlineto{\pgfqpoint{-1.088683in}{2.221346in}}%
\pgfpathlineto{\pgfqpoint{-1.038293in}{2.128908in}}%
\pgfpathlineto{\pgfqpoint{-0.987007in}{2.108488in}}%
\pgfpathlineto{\pgfqpoint{-0.934193in}{2.085182in}}%
\pgfpathlineto{\pgfqpoint{-0.883439in}{2.176910in}}%
\pgfpathlineto{\pgfqpoint{-0.832397in}{2.135069in}}%
\pgfpathlineto{\pgfqpoint{-0.780794in}{2.184676in}}%
\pgfpathlineto{\pgfqpoint{-0.729960in}{2.169046in}}%
\pgfpathlineto{\pgfqpoint{-0.679845in}{2.197239in}}%
\pgfpathlineto{\pgfqpoint{-0.628072in}{2.179648in}}%
\pgfpathlineto{\pgfqpoint{-0.577558in}{2.154364in}}%
\pgfpathlineto{\pgfqpoint{-0.525798in}{2.116407in}}%
\pgfpathlineto{\pgfqpoint{-0.472607in}{2.139636in}}%
\pgfpathlineto{\pgfqpoint{-0.421082in}{2.168914in}}%
\pgfpathlineto{\pgfqpoint{-0.370494in}{2.178709in}}%
\pgfpathlineto{\pgfqpoint{-0.317282in}{2.138193in}}%
\pgfpathlineto{\pgfqpoint{-0.265117in}{2.088575in}}%
\pgfpathlineto{\pgfqpoint{-0.212446in}{2.077484in}}%
\pgfpathlineto{\pgfqpoint{-0.159327in}{2.160460in}}%
\pgfpathlineto{\pgfqpoint{-0.106747in}{2.100503in}}%
\pgfpathlineto{\pgfqpoint{-0.053884in}{2.082785in}}%
\pgfpathlineto{\pgfqpoint{0.000028in}{2.150986in}}%
\pgfpathlineto{\pgfqpoint{0.052190in}{2.133563in}}%
\pgfpathlineto{\pgfqpoint{0.103930in}{2.134356in}}%
\pgfpathlineto{\pgfqpoint{0.157615in}{2.082360in}}%
\pgfpathlineto{\pgfqpoint{0.209933in}{2.080609in}}%
\pgfpathlineto{\pgfqpoint{0.261843in}{2.152791in}}%
\pgfpathlineto{\pgfqpoint{0.315264in}{2.123741in}}%
\pgfpathlineto{\pgfqpoint{0.366948in}{2.161564in}}%
\pgfpathlineto{\pgfqpoint{0.419063in}{2.142856in}}%
\pgfpathlineto{\pgfqpoint{0.472767in}{2.093735in}}%
\pgfpathlineto{\pgfqpoint{0.524412in}{2.163910in}}%
\pgfpathlineto{\pgfqpoint{0.575921in}{2.109977in}}%
\pgfpathlineto{\pgfqpoint{0.628822in}{2.192509in}}%
\pgfpathlineto{\pgfqpoint{0.680101in}{2.157383in}}%
\pgfpathlineto{\pgfqpoint{0.730937in}{2.164722in}}%
\pgfpathlineto{\pgfqpoint{0.783351in}{2.140896in}}%
\pgfpathlineto{\pgfqpoint{0.835638in}{2.094139in}}%
\pgfpathlineto{\pgfqpoint{0.890559in}{1.990714in}}%
\pgfpathlineto{\pgfqpoint{0.950896in}{1.985679in}}%
\pgfpathlineto{\pgfqpoint{1.011418in}{1.931511in}}%
\pgfpathlineto{\pgfqpoint{1.071223in}{1.933531in}}%
\pgfpathlineto{\pgfqpoint{1.133818in}{1.909921in}}%
\pgfpathlineto{\pgfqpoint{1.196854in}{1.881027in}}%
\pgfpathlineto{\pgfqpoint{1.262232in}{1.818030in}}%
\pgfpathlineto{\pgfqpoint{1.331848in}{1.799199in}}%
\pgfpathlineto{\pgfqpoint{1.399923in}{1.815928in}}%
\pgfpathlineto{\pgfqpoint{1.469539in}{1.770692in}}%
\pgfpathlineto{\pgfqpoint{1.542225in}{1.744580in}}%
\pgfpathlineto{\pgfqpoint{1.614592in}{1.786581in}}%
\pgfpathlineto{\pgfqpoint{1.687208in}{1.696625in}}%
\pgfpathlineto{\pgfqpoint{1.764147in}{1.713936in}}%
\pgfpathlineto{\pgfqpoint{1.839330in}{1.715736in}}%
\pgfpathlineto{\pgfqpoint{1.918401in}{1.621377in}}%
\pgfpathlineto{\pgfqpoint{1.999401in}{1.683708in}}%
\pgfpathlineto{\pgfqpoint{2.077252in}{1.664462in}}%
\pgfpathlineto{\pgfqpoint{2.157128in}{1.623152in}}%
\pgfpathlineto{\pgfqpoint{2.243381in}{1.642355in}}%
\pgfpathlineto{\pgfqpoint{2.328903in}{1.618107in}}%
\pgfpathlineto{\pgfqpoint{2.416579in}{1.559328in}}%
\pgfpathlineto{\pgfqpoint{2.504252in}{1.621243in}}%
\pgfpathlineto{\pgfqpoint{2.589494in}{1.606211in}}%
\pgfpathlineto{\pgfqpoint{2.676066in}{1.548891in}}%
\pgfpathlineto{\pgfqpoint{2.767061in}{1.569780in}}%
\pgfpathlineto{\pgfqpoint{2.858701in}{1.565071in}}%
\pgfpathlineto{\pgfqpoint{2.948242in}{1.559840in}}%
\pgfpathlineto{\pgfqpoint{3.043068in}{1.548059in}}%
\pgfpathlineto{\pgfqpoint{3.132459in}{1.577644in}}%
\pgfpathlineto{\pgfqpoint{3.198420in}{1.527806in}}%
\pgfpathlineto{\pgfqpoint{3.252590in}{1.466197in}}%
\pgfpathlineto{\pgfqpoint{3.305067in}{1.465799in}}%
\pgfpathlineto{\pgfqpoint{3.357008in}{1.453492in}}%
\pgfpathlineto{\pgfqpoint{3.411084in}{1.405600in}}%
\pgfpathlineto{\pgfqpoint{3.464430in}{1.393664in}}%
\pgfpathlineto{\pgfqpoint{3.517522in}{1.442369in}}%
\pgfpathlineto{\pgfqpoint{3.571722in}{1.459985in}}%
\pgfpathlineto{\pgfqpoint{3.624008in}{1.429618in}}%
\pgfpathlineto{\pgfqpoint{3.676588in}{1.414323in}}%
\pgfpathlineto{\pgfqpoint{3.729373in}{1.488284in}}%
\pgfpathlineto{\pgfqpoint{3.781554in}{1.423144in}}%
\pgfpathlineto{\pgfqpoint{3.833685in}{1.230236in}}%
\pgfpathlineto{\pgfqpoint{3.887639in}{0.773588in}}%
\pgfpathlineto{\pgfqpoint{3.940507in}{0.773588in}}%
\pgfpathlineto{\pgfqpoint{3.993337in}{0.773588in}}%
\pgfpathlineto{\pgfqpoint{4.047138in}{0.773588in}}%
\pgfpathlineto{\pgfqpoint{4.098630in}{0.773588in}}%
\pgfpathlineto{\pgfqpoint{4.150734in}{0.773588in}}%
\pgfpathlineto{\pgfqpoint{4.204360in}{0.773588in}}%
\pgfpathlineto{\pgfqpoint{4.256228in}{0.773588in}}%
\pgfpathlineto{\pgfqpoint{4.307535in}{0.773588in}}%
\pgfpathlineto{\pgfqpoint{4.360055in}{0.773588in}}%
\pgfpathlineto{\pgfqpoint{4.398305in}{0.773588in}}%
\pgfpathlineto{\pgfqpoint{4.446119in}{0.773588in}}%
\pgfpathlineto{\pgfqpoint{4.484596in}{1.454760in}}%
\pgfpathlineto{\pgfqpoint{4.526846in}{1.669618in}}%
\pgfpathlineto{\pgfqpoint{4.566686in}{1.906942in}}%
\pgfpathlineto{\pgfqpoint{4.603496in}{2.224563in}}%
\pgfpathlineto{\pgfqpoint{4.635492in}{2.864879in}}%
\pgfpathlineto{\pgfqpoint{4.666016in}{3.729137in}}%
\pgfpathlineto{\pgfqpoint{4.690368in}{5.235025in}}%
\pgfpathlineto{\pgfqpoint{4.715513in}{5.182737in}}%
\pgfpathlineto{\pgfqpoint{4.739291in}{5.364631in}}%
\pgfpathlineto{\pgfqpoint{4.764034in}{5.483715in}}%
\pgfpathlineto{\pgfqpoint{4.787716in}{5.446238in}}%
\pgfpathlineto{\pgfqpoint{4.811295in}{5.442565in}}%
\pgfpathlineto{\pgfqpoint{4.836687in}{5.484292in}}%
\pgfpathlineto{\pgfqpoint{4.860218in}{5.534473in}}%
\pgfpathlineto{\pgfqpoint{4.884852in}{5.460686in}}%
\pgfpathlineto{\pgfqpoint{4.907640in}{5.518952in}}%
\pgfpathlineto{\pgfqpoint{4.931890in}{5.678338in}}%
\pgfpathlineto{\pgfqpoint{4.954839in}{5.584654in}}%
\pgfpathlineto{\pgfqpoint{4.980174in}{5.504363in}}%
\pgfpathlineto{\pgfqpoint{5.002737in}{5.557546in}}%
\pgfpathlineto{\pgfqpoint{5.026810in}{5.442515in}}%
\pgfpathlineto{\pgfqpoint{5.051612in}{5.582839in}}%
\pgfpathlineto{\pgfqpoint{5.074798in}{5.834541in}}%
\pgfpathlineto{\pgfqpoint{5.097977in}{5.740043in}}%
\pgfpathlineto{\pgfqpoint{5.122448in}{5.601241in}}%
\pgfpathlineto{\pgfqpoint{5.145219in}{5.809148in}}%
\pgfpathlineto{\pgfqpoint{5.168068in}{5.745325in}}%
\pgfpathlineto{\pgfqpoint{5.192095in}{5.846668in}}%
\pgfpathlineto{\pgfqpoint{5.214897in}{5.862622in}}%
\pgfpathlineto{\pgfqpoint{5.237842in}{5.791091in}}%
\pgfpathlineto{\pgfqpoint{5.261861in}{5.744292in}}%
\pgfpathlineto{\pgfqpoint{5.285086in}{5.706485in}}%
\pgfpathlineto{\pgfqpoint{5.307824in}{5.908282in}}%
\pgfpathlineto{\pgfqpoint{5.331834in}{5.813609in}}%
\pgfpathlineto{\pgfqpoint{5.355254in}{5.854178in}}%
\pgfpathlineto{\pgfqpoint{5.377738in}{5.918658in}}%
\pgfpathlineto{\pgfqpoint{5.402166in}{5.815038in}}%
\pgfpathlineto{\pgfqpoint{5.424187in}{5.868196in}}%
\pgfpathlineto{\pgfqpoint{5.448001in}{5.857730in}}%
\pgfpathlineto{\pgfqpoint{5.470431in}{5.911984in}}%
\pgfpathlineto{\pgfqpoint{5.494250in}{5.893046in}}%
\pgfpathlineto{\pgfqpoint{5.516676in}{5.880249in}}%
\pgfpathlineto{\pgfqpoint{5.540659in}{5.899825in}}%
\pgfpathlineto{\pgfqpoint{5.562835in}{5.930845in}}%
\pgfpathlineto{\pgfqpoint{5.587139in}{5.726189in}}%
\pgfpathlineto{\pgfqpoint{5.609949in}{5.825752in}}%
\pgfpathlineto{\pgfqpoint{5.634039in}{5.865058in}}%
\pgfpathlineto{\pgfqpoint{5.661246in}{5.775449in}}%
\pgfpathlineto{\pgfqpoint{5.712060in}{5.758501in}}%
\pgfpathlineto{\pgfqpoint{5.763148in}{5.758501in}}%
\pgfpathlineto{\pgfqpoint{5.815025in}{5.758501in}}%
\pgfpathlineto{\pgfqpoint{5.867920in}{5.758501in}}%
\pgfpathlineto{\pgfqpoint{5.919631in}{5.758501in}}%
\pgfpathlineto{\pgfqpoint{5.971565in}{5.758501in}}%
\pgfpathlineto{\pgfqpoint{6.025764in}{5.758501in}}%
\pgfpathlineto{\pgfqpoint{6.078797in}{5.758501in}}%
\pgfpathlineto{\pgfqpoint{6.078797in}{5.758501in}}%
\pgfpathlineto{\pgfqpoint{6.078797in}{5.758501in}}%
\pgfpathlineto{\pgfqpoint{6.025764in}{5.758501in}}%
\pgfpathlineto{\pgfqpoint{5.971565in}{5.758501in}}%
\pgfpathlineto{\pgfqpoint{5.919631in}{5.758501in}}%
\pgfpathlineto{\pgfqpoint{5.867920in}{5.758501in}}%
\pgfpathlineto{\pgfqpoint{5.815025in}{5.758501in}}%
\pgfpathlineto{\pgfqpoint{5.763148in}{5.758501in}}%
\pgfpathlineto{\pgfqpoint{5.712060in}{5.758501in}}%
\pgfpathlineto{\pgfqpoint{5.661246in}{5.775449in}}%
\pgfpathlineto{\pgfqpoint{5.634039in}{5.865058in}}%
\pgfpathlineto{\pgfqpoint{5.609949in}{5.825752in}}%
\pgfpathlineto{\pgfqpoint{5.587139in}{5.726189in}}%
\pgfpathlineto{\pgfqpoint{5.562835in}{5.930845in}}%
\pgfpathlineto{\pgfqpoint{5.540659in}{5.899825in}}%
\pgfpathlineto{\pgfqpoint{5.516676in}{5.880249in}}%
\pgfpathlineto{\pgfqpoint{5.494250in}{5.893046in}}%
\pgfpathlineto{\pgfqpoint{5.470431in}{5.911984in}}%
\pgfpathlineto{\pgfqpoint{5.448001in}{5.857730in}}%
\pgfpathlineto{\pgfqpoint{5.424187in}{5.868196in}}%
\pgfpathlineto{\pgfqpoint{5.402166in}{5.815038in}}%
\pgfpathlineto{\pgfqpoint{5.377738in}{5.918658in}}%
\pgfpathlineto{\pgfqpoint{5.355254in}{5.854178in}}%
\pgfpathlineto{\pgfqpoint{5.331834in}{5.813609in}}%
\pgfpathlineto{\pgfqpoint{5.307824in}{5.908282in}}%
\pgfpathlineto{\pgfqpoint{5.285086in}{5.706485in}}%
\pgfpathlineto{\pgfqpoint{5.261861in}{5.744292in}}%
\pgfpathlineto{\pgfqpoint{5.237842in}{5.791091in}}%
\pgfpathlineto{\pgfqpoint{5.214897in}{5.862622in}}%
\pgfpathlineto{\pgfqpoint{5.192095in}{5.846668in}}%
\pgfpathlineto{\pgfqpoint{5.168068in}{5.745325in}}%
\pgfpathlineto{\pgfqpoint{5.145219in}{5.809148in}}%
\pgfpathlineto{\pgfqpoint{5.122448in}{5.601241in}}%
\pgfpathlineto{\pgfqpoint{5.097977in}{5.740043in}}%
\pgfpathlineto{\pgfqpoint{5.074798in}{5.834541in}}%
\pgfpathlineto{\pgfqpoint{5.051612in}{5.582839in}}%
\pgfpathlineto{\pgfqpoint{5.026810in}{5.442515in}}%
\pgfpathlineto{\pgfqpoint{5.002737in}{5.557546in}}%
\pgfpathlineto{\pgfqpoint{4.980174in}{5.504363in}}%
\pgfpathlineto{\pgfqpoint{4.954839in}{5.584654in}}%
\pgfpathlineto{\pgfqpoint{4.931890in}{5.678338in}}%
\pgfpathlineto{\pgfqpoint{4.907640in}{5.518952in}}%
\pgfpathlineto{\pgfqpoint{4.884852in}{5.460686in}}%
\pgfpathlineto{\pgfqpoint{4.860218in}{5.534473in}}%
\pgfpathlineto{\pgfqpoint{4.836687in}{5.484292in}}%
\pgfpathlineto{\pgfqpoint{4.811295in}{5.442565in}}%
\pgfpathlineto{\pgfqpoint{4.787716in}{5.446238in}}%
\pgfpathlineto{\pgfqpoint{4.764034in}{5.483715in}}%
\pgfpathlineto{\pgfqpoint{4.739291in}{5.364631in}}%
\pgfpathlineto{\pgfqpoint{4.715513in}{5.182737in}}%
\pgfpathlineto{\pgfqpoint{4.690368in}{5.235025in}}%
\pgfpathlineto{\pgfqpoint{4.666016in}{3.729137in}}%
\pgfpathlineto{\pgfqpoint{4.635492in}{2.864879in}}%
\pgfpathlineto{\pgfqpoint{4.603496in}{2.224563in}}%
\pgfpathlineto{\pgfqpoint{4.566686in}{1.906942in}}%
\pgfpathlineto{\pgfqpoint{4.526846in}{1.669618in}}%
\pgfpathlineto{\pgfqpoint{4.484596in}{1.454760in}}%
\pgfpathlineto{\pgfqpoint{4.446119in}{0.773588in}}%
\pgfpathlineto{\pgfqpoint{4.398305in}{0.773588in}}%
\pgfpathlineto{\pgfqpoint{4.360055in}{0.773588in}}%
\pgfpathlineto{\pgfqpoint{4.307535in}{0.773588in}}%
\pgfpathlineto{\pgfqpoint{4.256228in}{0.773588in}}%
\pgfpathlineto{\pgfqpoint{4.204360in}{0.773588in}}%
\pgfpathlineto{\pgfqpoint{4.150734in}{0.773588in}}%
\pgfpathlineto{\pgfqpoint{4.098630in}{0.773588in}}%
\pgfpathlineto{\pgfqpoint{4.047138in}{0.773588in}}%
\pgfpathlineto{\pgfqpoint{3.993337in}{0.773588in}}%
\pgfpathlineto{\pgfqpoint{3.940507in}{0.773588in}}%
\pgfpathlineto{\pgfqpoint{3.887639in}{0.773588in}}%
\pgfpathlineto{\pgfqpoint{3.833685in}{1.230236in}}%
\pgfpathlineto{\pgfqpoint{3.781554in}{1.423144in}}%
\pgfpathlineto{\pgfqpoint{3.729373in}{1.488284in}}%
\pgfpathlineto{\pgfqpoint{3.676588in}{1.414323in}}%
\pgfpathlineto{\pgfqpoint{3.624008in}{1.429618in}}%
\pgfpathlineto{\pgfqpoint{3.571722in}{1.459985in}}%
\pgfpathlineto{\pgfqpoint{3.517522in}{1.442369in}}%
\pgfpathlineto{\pgfqpoint{3.464430in}{1.393664in}}%
\pgfpathlineto{\pgfqpoint{3.411084in}{1.405600in}}%
\pgfpathlineto{\pgfqpoint{3.357008in}{1.453492in}}%
\pgfpathlineto{\pgfqpoint{3.305067in}{1.465799in}}%
\pgfpathlineto{\pgfqpoint{3.252590in}{1.466197in}}%
\pgfpathlineto{\pgfqpoint{3.198420in}{1.527806in}}%
\pgfpathlineto{\pgfqpoint{3.132459in}{1.577644in}}%
\pgfpathlineto{\pgfqpoint{3.043068in}{1.548059in}}%
\pgfpathlineto{\pgfqpoint{2.948242in}{1.559840in}}%
\pgfpathlineto{\pgfqpoint{2.858701in}{1.565071in}}%
\pgfpathlineto{\pgfqpoint{2.767061in}{1.569780in}}%
\pgfpathlineto{\pgfqpoint{2.676066in}{1.548891in}}%
\pgfpathlineto{\pgfqpoint{2.589494in}{1.606211in}}%
\pgfpathlineto{\pgfqpoint{2.504252in}{1.621243in}}%
\pgfpathlineto{\pgfqpoint{2.416579in}{1.559328in}}%
\pgfpathlineto{\pgfqpoint{2.328903in}{1.618107in}}%
\pgfpathlineto{\pgfqpoint{2.243381in}{1.642355in}}%
\pgfpathlineto{\pgfqpoint{2.157128in}{1.623152in}}%
\pgfpathlineto{\pgfqpoint{2.077252in}{1.664462in}}%
\pgfpathlineto{\pgfqpoint{1.999401in}{1.683708in}}%
\pgfpathlineto{\pgfqpoint{1.918401in}{1.621377in}}%
\pgfpathlineto{\pgfqpoint{1.839330in}{1.715736in}}%
\pgfpathlineto{\pgfqpoint{1.764147in}{1.713936in}}%
\pgfpathlineto{\pgfqpoint{1.687208in}{1.696625in}}%
\pgfpathlineto{\pgfqpoint{1.614592in}{1.786581in}}%
\pgfpathlineto{\pgfqpoint{1.542225in}{1.744580in}}%
\pgfpathlineto{\pgfqpoint{1.469539in}{1.770692in}}%
\pgfpathlineto{\pgfqpoint{1.399923in}{1.815928in}}%
\pgfpathlineto{\pgfqpoint{1.331848in}{1.799199in}}%
\pgfpathlineto{\pgfqpoint{1.262232in}{1.818030in}}%
\pgfpathlineto{\pgfqpoint{1.196854in}{1.881027in}}%
\pgfpathlineto{\pgfqpoint{1.133818in}{1.909921in}}%
\pgfpathlineto{\pgfqpoint{1.071223in}{1.933531in}}%
\pgfpathlineto{\pgfqpoint{1.011418in}{1.931511in}}%
\pgfpathlineto{\pgfqpoint{0.950896in}{1.985679in}}%
\pgfpathlineto{\pgfqpoint{0.890559in}{1.990714in}}%
\pgfpathlineto{\pgfqpoint{0.835638in}{2.094139in}}%
\pgfpathlineto{\pgfqpoint{0.783351in}{2.140896in}}%
\pgfpathlineto{\pgfqpoint{0.730937in}{2.164722in}}%
\pgfpathlineto{\pgfqpoint{0.680101in}{2.157383in}}%
\pgfpathlineto{\pgfqpoint{0.628822in}{2.192509in}}%
\pgfpathlineto{\pgfqpoint{0.575921in}{2.109977in}}%
\pgfpathlineto{\pgfqpoint{0.524412in}{2.163910in}}%
\pgfpathlineto{\pgfqpoint{0.472767in}{2.093735in}}%
\pgfpathlineto{\pgfqpoint{0.419063in}{2.142856in}}%
\pgfpathlineto{\pgfqpoint{0.366948in}{2.161564in}}%
\pgfpathlineto{\pgfqpoint{0.315264in}{2.123741in}}%
\pgfpathlineto{\pgfqpoint{0.261843in}{2.152791in}}%
\pgfpathlineto{\pgfqpoint{0.209933in}{2.080609in}}%
\pgfpathlineto{\pgfqpoint{0.157615in}{2.082360in}}%
\pgfpathlineto{\pgfqpoint{0.103930in}{2.134356in}}%
\pgfpathlineto{\pgfqpoint{0.052190in}{2.133563in}}%
\pgfpathlineto{\pgfqpoint{0.000028in}{2.150986in}}%
\pgfpathlineto{\pgfqpoint{-0.053884in}{2.082785in}}%
\pgfpathlineto{\pgfqpoint{-0.106747in}{2.100503in}}%
\pgfpathlineto{\pgfqpoint{-0.159327in}{2.160460in}}%
\pgfpathlineto{\pgfqpoint{-0.212446in}{2.077484in}}%
\pgfpathlineto{\pgfqpoint{-0.265117in}{2.088575in}}%
\pgfpathlineto{\pgfqpoint{-0.317282in}{2.138193in}}%
\pgfpathlineto{\pgfqpoint{-0.370494in}{2.178709in}}%
\pgfpathlineto{\pgfqpoint{-0.421082in}{2.168914in}}%
\pgfpathlineto{\pgfqpoint{-0.472607in}{2.139636in}}%
\pgfpathlineto{\pgfqpoint{-0.525798in}{2.116407in}}%
\pgfpathlineto{\pgfqpoint{-0.577558in}{2.154364in}}%
\pgfpathlineto{\pgfqpoint{-0.628072in}{2.179648in}}%
\pgfpathlineto{\pgfqpoint{-0.679845in}{2.197239in}}%
\pgfpathlineto{\pgfqpoint{-0.729960in}{2.169046in}}%
\pgfpathlineto{\pgfqpoint{-0.780794in}{2.184676in}}%
\pgfpathlineto{\pgfqpoint{-0.832397in}{2.135069in}}%
\pgfpathlineto{\pgfqpoint{-0.883439in}{2.176910in}}%
\pgfpathlineto{\pgfqpoint{-0.934193in}{2.085182in}}%
\pgfpathlineto{\pgfqpoint{-0.987007in}{2.108488in}}%
\pgfpathlineto{\pgfqpoint{-1.038293in}{2.128908in}}%
\pgfpathlineto{\pgfqpoint{-1.088683in}{2.221346in}}%
\pgfpathlineto{\pgfqpoint{-1.140825in}{2.152372in}}%
\pgfpathlineto{\pgfqpoint{-1.191111in}{2.194715in}}%
\pgfpathlineto{\pgfqpoint{-1.242201in}{2.103896in}}%
\pgfpathlineto{\pgfqpoint{-1.295495in}{2.154372in}}%
\pgfpathlineto{\pgfqpoint{-1.346574in}{2.133339in}}%
\pgfpathlineto{\pgfqpoint{-1.397583in}{2.148509in}}%
\pgfpathlineto{\pgfqpoint{-1.451018in}{2.142057in}}%
\pgfpathlineto{\pgfqpoint{-1.502359in}{2.171648in}}%
\pgfpathlineto{\pgfqpoint{-1.553079in}{2.102078in}}%
\pgfpathlineto{\pgfqpoint{-1.606814in}{2.127843in}}%
\pgfpathlineto{\pgfqpoint{-1.657574in}{2.169913in}}%
\pgfpathlineto{\pgfqpoint{-1.707851in}{2.202638in}}%
\pgfpathlineto{\pgfqpoint{-1.759793in}{2.188092in}}%
\pgfpathlineto{\pgfqpoint{-1.809967in}{2.193467in}}%
\pgfpathlineto{\pgfqpoint{-1.860296in}{2.059972in}}%
\pgfpathlineto{\pgfqpoint{-1.912972in}{2.149431in}}%
\pgfpathlineto{\pgfqpoint{-1.963080in}{2.151542in}}%
\pgfpathlineto{\pgfqpoint{-2.013282in}{2.262046in}}%
\pgfpathlineto{\pgfqpoint{-2.064986in}{2.084956in}}%
\pgfpathlineto{\pgfqpoint{-2.116467in}{2.127160in}}%
\pgfpathlineto{\pgfqpoint{-2.167559in}{2.125482in}}%
\pgfpathlineto{\pgfqpoint{-2.220871in}{2.106034in}}%
\pgfpathlineto{\pgfqpoint{-2.272195in}{2.191289in}}%
\pgfpathlineto{\pgfqpoint{-2.323084in}{2.127542in}}%
\pgfpathlineto{\pgfqpoint{-2.375607in}{2.206184in}}%
\pgfpathlineto{\pgfqpoint{-2.425397in}{2.126317in}}%
\pgfpathlineto{\pgfqpoint{-2.476616in}{2.167554in}}%
\pgfpathlineto{\pgfqpoint{-2.528440in}{2.142711in}}%
\pgfpathlineto{\pgfqpoint{-2.578972in}{2.169425in}}%
\pgfpathlineto{\pgfqpoint{-2.629697in}{2.167275in}}%
\pgfpathlineto{\pgfqpoint{-2.681969in}{2.127344in}}%
\pgfpathlineto{\pgfqpoint{-2.732985in}{2.155346in}}%
\pgfpathlineto{\pgfqpoint{-2.784277in}{2.125000in}}%
\pgfpathlineto{\pgfqpoint{-2.836699in}{2.196254in}}%
\pgfpathlineto{\pgfqpoint{-2.887081in}{2.073278in}}%
\pgfpathlineto{\pgfqpoint{-2.938490in}{2.147600in}}%
\pgfpathlineto{\pgfqpoint{-2.990316in}{2.216585in}}%
\pgfpathlineto{\pgfqpoint{-3.039557in}{2.210846in}}%
\pgfpathlineto{\pgfqpoint{-3.088627in}{2.140127in}}%
\pgfpathlineto{\pgfqpoint{-3.140428in}{2.188680in}}%
\pgfpathlineto{\pgfqpoint{-3.190882in}{2.138438in}}%
\pgfpathlineto{\pgfqpoint{-3.241308in}{2.202460in}}%
\pgfpathlineto{\pgfqpoint{-3.292250in}{2.172470in}}%
\pgfpathlineto{\pgfqpoint{-3.341930in}{2.172740in}}%
\pgfpathlineto{\pgfqpoint{-3.392015in}{2.161647in}}%
\pgfpathlineto{\pgfqpoint{-3.444140in}{2.189113in}}%
\pgfpathlineto{\pgfqpoint{-3.494154in}{2.129676in}}%
\pgfpathlineto{\pgfqpoint{-3.544176in}{2.202252in}}%
\pgfpathlineto{\pgfqpoint{-3.595859in}{2.177297in}}%
\pgfpathlineto{\pgfqpoint{-3.646072in}{2.113429in}}%
\pgfpathlineto{\pgfqpoint{-3.696993in}{2.130566in}}%
\pgfpathlineto{\pgfqpoint{-3.748633in}{2.153196in}}%
\pgfpathlineto{\pgfqpoint{-3.799759in}{2.087238in}}%
\pgfpathlineto{\pgfqpoint{-3.850587in}{2.107499in}}%
\pgfpathlineto{\pgfqpoint{-3.902764in}{2.187707in}}%
\pgfpathlineto{\pgfqpoint{-3.952191in}{2.179510in}}%
\pgfpathlineto{\pgfqpoint{-4.002524in}{2.199798in}}%
\pgfpathlineto{\pgfqpoint{-4.053537in}{2.124537in}}%
\pgfpathlineto{\pgfqpoint{-4.103923in}{2.174410in}}%
\pgfpathlineto{\pgfqpoint{-4.153202in}{2.198963in}}%
\pgfpathlineto{\pgfqpoint{-4.204591in}{2.148863in}}%
\pgfpathlineto{\pgfqpoint{-4.255103in}{2.169034in}}%
\pgfpathlineto{\pgfqpoint{-4.305607in}{2.156690in}}%
\pgfpathlineto{\pgfqpoint{-4.357515in}{2.166090in}}%
\pgfpathlineto{\pgfqpoint{-4.407934in}{2.200234in}}%
\pgfpathlineto{\pgfqpoint{-4.457276in}{2.240633in}}%
\pgfpathlineto{\pgfqpoint{-4.508002in}{2.139836in}}%
\pgfpathlineto{\pgfqpoint{-4.558006in}{2.234647in}}%
\pgfpathlineto{\pgfqpoint{-4.607841in}{2.187482in}}%
\pgfpathlineto{\pgfqpoint{-4.659647in}{2.192402in}}%
\pgfpathlineto{\pgfqpoint{-4.709756in}{2.191465in}}%
\pgfpathlineto{\pgfqpoint{-4.759589in}{2.072454in}}%
\pgfpathlineto{\pgfqpoint{-4.812147in}{2.204872in}}%
\pgfpathlineto{\pgfqpoint{-4.861887in}{2.209164in}}%
\pgfpathlineto{\pgfqpoint{-4.911765in}{2.139169in}}%
\pgfpathlineto{\pgfqpoint{-4.963670in}{2.160039in}}%
\pgfpathlineto{\pgfqpoint{-5.014255in}{2.208071in}}%
\pgfpathlineto{\pgfqpoint{-5.064238in}{2.211605in}}%
\pgfpathlineto{\pgfqpoint{-5.114869in}{2.149127in}}%
\pgfpathlineto{\pgfqpoint{-5.165135in}{2.157432in}}%
\pgfpathlineto{\pgfqpoint{-5.215272in}{2.170742in}}%
\pgfpathlineto{\pgfqpoint{-5.267508in}{2.144171in}}%
\pgfpathlineto{\pgfqpoint{-5.319212in}{2.162367in}}%
\pgfpathlineto{\pgfqpoint{-5.369703in}{2.201493in}}%
\pgfpathlineto{\pgfqpoint{-5.421631in}{2.116542in}}%
\pgfpathlineto{\pgfqpoint{-5.473058in}{2.119330in}}%
\pgfpathlineto{\pgfqpoint{-5.523472in}{2.175857in}}%
\pgfpathlineto{\pgfqpoint{-5.574633in}{2.259241in}}%
\pgfpathlineto{\pgfqpoint{-5.623816in}{2.182126in}}%
\pgfpathlineto{\pgfqpoint{-5.673071in}{2.221155in}}%
\pgfpathlineto{\pgfqpoint{-5.724570in}{2.159418in}}%
\pgfpathlineto{\pgfqpoint{-5.774223in}{2.282936in}}%
\pgfpathlineto{\pgfqpoint{-5.822716in}{2.143388in}}%
\pgfpathlineto{\pgfqpoint{-5.874278in}{2.137407in}}%
\pgfpathlineto{\pgfqpoint{-5.924128in}{2.157637in}}%
\pgfpathlineto{\pgfqpoint{-5.974236in}{2.198735in}}%
\pgfpathlineto{\pgfqpoint{-6.024836in}{2.170140in}}%
\pgfpathlineto{\pgfqpoint{-6.075227in}{2.148300in}}%
\pgfpathlineto{\pgfqpoint{-6.125420in}{2.152878in}}%
\pgfpathlineto{\pgfqpoint{-6.177451in}{2.124981in}}%
\pgfpathlineto{\pgfqpoint{-6.228389in}{2.136082in}}%
\pgfpathlineto{\pgfqpoint{-6.279011in}{2.079790in}}%
\pgfpathlineto{\pgfqpoint{-6.330959in}{2.235002in}}%
\pgfpathlineto{\pgfqpoint{-6.380124in}{2.146954in}}%
\pgfpathlineto{\pgfqpoint{-6.429787in}{2.173476in}}%
\pgfpathlineto{\pgfqpoint{-6.481910in}{2.104244in}}%
\pgfpathlineto{\pgfqpoint{-6.532179in}{2.158422in}}%
\pgfpathlineto{\pgfqpoint{-6.582317in}{2.168087in}}%
\pgfpathlineto{\pgfqpoint{-6.634445in}{2.248709in}}%
\pgfpathlineto{\pgfqpoint{-6.682747in}{2.305817in}}%
\pgfpathlineto{\pgfqpoint{-6.731323in}{2.180208in}}%
\pgfpathlineto{\pgfqpoint{-6.782488in}{2.179999in}}%
\pgfpathlineto{\pgfqpoint{-6.832072in}{2.136690in}}%
\pgfpathlineto{\pgfqpoint{-6.882593in}{2.135991in}}%
\pgfpathlineto{\pgfqpoint{-6.934169in}{2.206816in}}%
\pgfpathlineto{\pgfqpoint{-6.984061in}{2.133288in}}%
\pgfpathlineto{\pgfqpoint{-7.033318in}{2.157190in}}%
\pgfpathlineto{\pgfqpoint{-7.085212in}{2.094855in}}%
\pgfpathlineto{\pgfqpoint{-7.135511in}{2.185196in}}%
\pgfpathlineto{\pgfqpoint{-7.185558in}{2.138280in}}%
\pgfpathlineto{\pgfqpoint{-7.237328in}{2.251963in}}%
\pgfpathlineto{\pgfqpoint{-7.286238in}{2.258881in}}%
\pgfpathlineto{\pgfqpoint{-7.335791in}{2.153076in}}%
\pgfpathlineto{\pgfqpoint{-7.386343in}{2.276898in}}%
\pgfpathlineto{\pgfqpoint{-7.435602in}{2.161953in}}%
\pgfpathlineto{\pgfqpoint{-7.485312in}{2.128498in}}%
\pgfpathlineto{\pgfqpoint{-7.536566in}{2.230011in}}%
\pgfpathlineto{\pgfqpoint{-7.585672in}{2.226448in}}%
\pgfpathlineto{\pgfqpoint{-7.634521in}{2.227472in}}%
\pgfpathlineto{\pgfqpoint{-7.685281in}{2.213775in}}%
\pgfpathlineto{\pgfqpoint{-7.733870in}{2.218841in}}%
\pgfpathlineto{\pgfqpoint{-7.783607in}{2.160895in}}%
\pgfpathlineto{\pgfqpoint{-7.835188in}{2.167679in}}%
\pgfpathlineto{\pgfqpoint{-7.885122in}{2.150548in}}%
\pgfpathlineto{\pgfqpoint{-7.935315in}{2.168605in}}%
\pgfpathlineto{\pgfqpoint{-7.986412in}{2.175744in}}%
\pgfpathlineto{\pgfqpoint{-8.035792in}{2.208270in}}%
\pgfpathlineto{\pgfqpoint{-8.085190in}{2.191708in}}%
\pgfpathlineto{\pgfqpoint{-8.135490in}{2.245041in}}%
\pgfpathlineto{\pgfqpoint{-8.184462in}{2.181293in}}%
\pgfpathlineto{\pgfqpoint{-8.233427in}{2.201026in}}%
\pgfpathlineto{\pgfqpoint{-8.283775in}{2.241324in}}%
\pgfpathlineto{\pgfqpoint{-8.332630in}{2.155531in}}%
\pgfpathlineto{\pgfqpoint{-8.382452in}{2.185814in}}%
\pgfpathlineto{\pgfqpoint{-8.432560in}{2.212485in}}%
\pgfpathlineto{\pgfqpoint{-8.481981in}{2.166381in}}%
\pgfpathlineto{\pgfqpoint{-8.532027in}{2.134033in}}%
\pgfpathlineto{\pgfqpoint{-8.583181in}{2.283576in}}%
\pgfpathlineto{\pgfqpoint{-8.632588in}{2.212760in}}%
\pgfpathlineto{\pgfqpoint{-8.681529in}{2.268763in}}%
\pgfpathlineto{\pgfqpoint{-8.731477in}{2.192399in}}%
\pgfpathlineto{\pgfqpoint{-8.780578in}{2.126089in}}%
\pgfpathlineto{\pgfqpoint{-8.829855in}{2.221965in}}%
\pgfpathlineto{\pgfqpoint{-8.880937in}{2.258629in}}%
\pgfpathlineto{\pgfqpoint{-8.930541in}{2.155524in}}%
\pgfpathlineto{\pgfqpoint{-8.979731in}{2.188387in}}%
\pgfpathlineto{\pgfqpoint{-9.030683in}{2.199386in}}%
\pgfpathlineto{\pgfqpoint{-9.080221in}{2.134136in}}%
\pgfpathlineto{\pgfqpoint{-9.130552in}{2.107988in}}%
\pgfpathlineto{\pgfqpoint{-9.182531in}{2.192230in}}%
\pgfpathlineto{\pgfqpoint{-9.232466in}{2.162851in}}%
\pgfpathlineto{\pgfqpoint{-9.281795in}{2.161230in}}%
\pgfpathlineto{\pgfqpoint{-9.332833in}{2.243716in}}%
\pgfpathlineto{\pgfqpoint{-9.382453in}{2.175230in}}%
\pgfpathlineto{\pgfqpoint{-9.430893in}{2.296405in}}%
\pgfpathlineto{\pgfqpoint{-9.479530in}{2.281059in}}%
\pgfpathlineto{\pgfqpoint{-9.527213in}{2.236420in}}%
\pgfpathlineto{\pgfqpoint{-9.575433in}{2.223149in}}%
\pgfpathlineto{\pgfqpoint{-9.624467in}{2.193047in}}%
\pgfpathlineto{\pgfqpoint{-9.671880in}{2.275859in}}%
\pgfpathlineto{\pgfqpoint{-9.719205in}{2.243279in}}%
\pgfpathlineto{\pgfqpoint{-9.768438in}{2.182953in}}%
\pgfpathlineto{\pgfqpoint{-9.816398in}{2.259987in}}%
\pgfpathlineto{\pgfqpoint{-9.863853in}{2.234019in}}%
\pgfpathlineto{\pgfqpoint{-9.912991in}{2.190874in}}%
\pgfpathlineto{\pgfqpoint{-9.960967in}{2.227549in}}%
\pgfpathlineto{\pgfqpoint{-10.009255in}{2.239043in}}%
\pgfpathlineto{\pgfqpoint{-10.058316in}{2.241679in}}%
\pgfpathlineto{\pgfqpoint{-10.106404in}{2.293533in}}%
\pgfpathlineto{\pgfqpoint{-10.154319in}{2.158168in}}%
\pgfpathlineto{\pgfqpoint{-10.204843in}{2.175947in}}%
\pgfpathlineto{\pgfqpoint{-10.253421in}{2.240721in}}%
\pgfpathlineto{\pgfqpoint{-10.301938in}{2.100303in}}%
\pgfpathlineto{\pgfqpoint{-10.352351in}{2.192082in}}%
\pgfpathlineto{\pgfqpoint{-10.401144in}{2.256048in}}%
\pgfpathlineto{\pgfqpoint{-10.448474in}{2.243100in}}%
\pgfpathlineto{\pgfqpoint{-10.497957in}{2.195093in}}%
\pgfpathlineto{\pgfqpoint{-10.546833in}{2.255904in}}%
\pgfpathlineto{\pgfqpoint{-10.594934in}{2.175746in}}%
\pgfpathlineto{\pgfqpoint{-10.644481in}{2.223418in}}%
\pgfpathlineto{\pgfqpoint{-10.693023in}{2.186383in}}%
\pgfpathlineto{\pgfqpoint{-10.741507in}{2.233839in}}%
\pgfpathlineto{\pgfqpoint{-10.790422in}{2.207052in}}%
\pgfpathlineto{\pgfqpoint{-10.837686in}{2.240166in}}%
\pgfpathlineto{\pgfqpoint{-10.885230in}{2.249891in}}%
\pgfpathlineto{\pgfqpoint{-10.933861in}{2.172965in}}%
\pgfpathlineto{\pgfqpoint{-10.980903in}{2.249085in}}%
\pgfpathlineto{\pgfqpoint{-11.028607in}{2.210510in}}%
\pgfpathlineto{\pgfqpoint{-11.077704in}{2.227451in}}%
\pgfpathlineto{\pgfqpoint{-11.125274in}{2.246190in}}%
\pgfpathlineto{\pgfqpoint{-11.172812in}{2.208562in}}%
\pgfpathlineto{\pgfqpoint{-11.221741in}{2.303508in}}%
\pgfpathlineto{\pgfqpoint{-11.268838in}{2.230183in}}%
\pgfpathlineto{\pgfqpoint{-11.316622in}{2.220034in}}%
\pgfpathlineto{\pgfqpoint{-11.366120in}{2.099061in}}%
\pgfpathlineto{\pgfqpoint{-11.415235in}{2.245278in}}%
\pgfpathlineto{\pgfqpoint{-11.463260in}{2.295885in}}%
\pgfpathlineto{\pgfqpoint{-11.511546in}{2.265803in}}%
\pgfpathlineto{\pgfqpoint{-11.559336in}{2.181075in}}%
\pgfpathlineto{\pgfqpoint{-11.607131in}{2.180861in}}%
\pgfpathlineto{\pgfqpoint{-11.656418in}{2.181121in}}%
\pgfpathlineto{\pgfqpoint{-11.704231in}{2.358516in}}%
\pgfpathlineto{\pgfqpoint{-11.751492in}{2.295602in}}%
\pgfpathlineto{\pgfqpoint{-11.800403in}{2.379567in}}%
\pgfpathlineto{\pgfqpoint{-11.847736in}{2.219811in}}%
\pgfpathlineto{\pgfqpoint{-11.895021in}{2.288470in}}%
\pgfpathlineto{\pgfqpoint{-11.943783in}{2.234768in}}%
\pgfpathlineto{\pgfqpoint{-11.991431in}{2.248043in}}%
\pgfpathlineto{\pgfqpoint{-12.038569in}{2.273193in}}%
\pgfpathlineto{\pgfqpoint{-12.087205in}{2.177400in}}%
\pgfpathlineto{\pgfqpoint{-12.135286in}{2.264188in}}%
\pgfpathlineto{\pgfqpoint{-12.182471in}{2.246982in}}%
\pgfpathlineto{\pgfqpoint{-12.231638in}{2.297791in}}%
\pgfpathlineto{\pgfqpoint{-12.278675in}{2.233795in}}%
\pgfpathlineto{\pgfqpoint{-12.326627in}{2.170799in}}%
\pgfpathlineto{\pgfqpoint{-12.375560in}{2.251846in}}%
\pgfpathlineto{\pgfqpoint{-12.422612in}{2.263073in}}%
\pgfpathlineto{\pgfqpoint{-12.469843in}{2.332512in}}%
\pgfpathlineto{\pgfqpoint{-12.517880in}{2.244193in}}%
\pgfpathlineto{\pgfqpoint{-12.565107in}{2.222802in}}%
\pgfpathlineto{\pgfqpoint{-12.612433in}{2.227898in}}%
\pgfpathlineto{\pgfqpoint{-12.661578in}{2.247475in}}%
\pgfpathlineto{\pgfqpoint{-12.708579in}{2.244778in}}%
\pgfpathlineto{\pgfqpoint{-12.755866in}{2.266736in}}%
\pgfpathlineto{\pgfqpoint{-12.805010in}{2.238151in}}%
\pgfpathlineto{\pgfqpoint{-12.852526in}{2.302678in}}%
\pgfpathlineto{\pgfqpoint{-12.899585in}{2.250525in}}%
\pgfpathlineto{\pgfqpoint{-12.949523in}{2.137013in}}%
\pgfpathlineto{\pgfqpoint{-12.998480in}{2.174071in}}%
\pgfpathlineto{\pgfqpoint{-13.046608in}{2.259149in}}%
\pgfpathlineto{\pgfqpoint{-13.096328in}{2.219318in}}%
\pgfpathlineto{\pgfqpoint{-13.144068in}{2.276775in}}%
\pgfpathlineto{\pgfqpoint{-13.190970in}{2.205221in}}%
\pgfpathlineto{\pgfqpoint{-13.239807in}{2.216517in}}%
\pgfpathlineto{\pgfqpoint{-13.287448in}{2.270580in}}%
\pgfpathlineto{\pgfqpoint{-13.334326in}{2.332418in}}%
\pgfpathlineto{\pgfqpoint{-13.382220in}{2.336191in}}%
\pgfpathlineto{\pgfqpoint{-13.428346in}{2.402601in}}%
\pgfpathlineto{\pgfqpoint{-13.474335in}{2.228192in}}%
\pgfpathlineto{\pgfqpoint{-13.522553in}{2.332872in}}%
\pgfpathlineto{\pgfqpoint{-13.568873in}{2.287207in}}%
\pgfpathlineto{\pgfqpoint{-13.615991in}{2.313157in}}%
\pgfpathlineto{\pgfqpoint{-13.663331in}{2.300535in}}%
\pgfpathlineto{\pgfqpoint{-13.710126in}{2.301145in}}%
\pgfpathlineto{\pgfqpoint{-13.757485in}{2.211679in}}%
\pgfpathlineto{\pgfqpoint{-13.806330in}{2.258501in}}%
\pgfpathlineto{\pgfqpoint{-13.853816in}{2.219452in}}%
\pgfpathlineto{\pgfqpoint{-13.901287in}{2.218369in}}%
\pgfpathlineto{\pgfqpoint{-13.950094in}{2.260363in}}%
\pgfpathlineto{\pgfqpoint{-13.996250in}{2.293893in}}%
\pgfpathlineto{\pgfqpoint{-14.042263in}{2.307810in}}%
\pgfpathlineto{\pgfqpoint{-14.090279in}{2.280354in}}%
\pgfpathlineto{\pgfqpoint{-14.137283in}{2.236209in}}%
\pgfpathlineto{\pgfqpoint{-14.184689in}{2.240614in}}%
\pgfpathlineto{\pgfqpoint{-14.233487in}{2.295575in}}%
\pgfpathlineto{\pgfqpoint{-14.280667in}{2.314598in}}%
\pgfpathlineto{\pgfqpoint{-14.327371in}{2.198546in}}%
\pgfpathlineto{\pgfqpoint{-14.375931in}{2.267582in}}%
\pgfpathlineto{\pgfqpoint{-14.422978in}{2.252252in}}%
\pgfpathlineto{\pgfqpoint{-14.469907in}{2.196403in}}%
\pgfpathlineto{\pgfqpoint{-14.517904in}{2.282858in}}%
\pgfpathlineto{\pgfqpoint{-14.564671in}{2.359201in}}%
\pgfpathlineto{\pgfqpoint{-14.611394in}{2.259349in}}%
\pgfpathlineto{\pgfqpoint{-14.659388in}{2.231589in}}%
\pgfpathlineto{\pgfqpoint{-14.705919in}{2.325095in}}%
\pgfpathlineto{\pgfqpoint{-14.752153in}{2.221244in}}%
\pgfpathlineto{\pgfqpoint{-14.800669in}{2.246168in}}%
\pgfpathlineto{\pgfqpoint{-14.847516in}{2.291021in}}%
\pgfpathlineto{\pgfqpoint{-14.894419in}{2.254543in}}%
\pgfpathlineto{\pgfqpoint{-14.942416in}{2.346640in}}%
\pgfpathlineto{\pgfqpoint{-14.989019in}{2.209372in}}%
\pgfpathlineto{\pgfqpoint{-15.035921in}{2.276334in}}%
\pgfpathlineto{\pgfqpoint{-15.083728in}{2.274703in}}%
\pgfpathlineto{\pgfqpoint{-15.130553in}{2.321238in}}%
\pgfpathlineto{\pgfqpoint{-15.176769in}{2.193278in}}%
\pgfpathlineto{\pgfqpoint{-15.225998in}{2.182080in}}%
\pgfpathlineto{\pgfqpoint{-15.273998in}{2.229533in}}%
\pgfpathlineto{\pgfqpoint{-15.321194in}{2.232270in}}%
\pgfpathlineto{\pgfqpoint{-15.369802in}{2.286673in}}%
\pgfpathlineto{\pgfqpoint{-15.416792in}{2.229832in}}%
\pgfpathlineto{\pgfqpoint{-15.463539in}{2.226374in}}%
\pgfpathlineto{\pgfqpoint{-15.512535in}{2.212443in}}%
\pgfpathlineto{\pgfqpoint{-15.560849in}{2.178671in}}%
\pgfpathlineto{\pgfqpoint{-15.608704in}{2.242575in}}%
\pgfpathlineto{\pgfqpoint{-15.657002in}{2.300755in}}%
\pgfpathlineto{\pgfqpoint{-15.703994in}{2.232128in}}%
\pgfpathlineto{\pgfqpoint{-15.750800in}{2.278245in}}%
\pgfpathlineto{\pgfqpoint{-15.798897in}{2.335101in}}%
\pgfpathlineto{\pgfqpoint{-15.844500in}{2.277760in}}%
\pgfpathlineto{\pgfqpoint{-15.890419in}{2.299059in}}%
\pgfpathlineto{\pgfqpoint{-15.938076in}{2.240374in}}%
\pgfpathlineto{\pgfqpoint{-15.984286in}{2.323804in}}%
\pgfpathlineto{\pgfqpoint{-16.029628in}{2.385490in}}%
\pgfpathlineto{\pgfqpoint{-16.077171in}{2.312159in}}%
\pgfpathlineto{\pgfqpoint{-16.123165in}{2.338157in}}%
\pgfpathlineto{\pgfqpoint{-16.169427in}{2.235117in}}%
\pgfpathlineto{\pgfqpoint{-16.217070in}{2.384526in}}%
\pgfpathlineto{\pgfqpoint{-16.262806in}{2.262921in}}%
\pgfpathlineto{\pgfqpoint{-16.308871in}{2.323533in}}%
\pgfpathlineto{\pgfqpoint{-16.355646in}{2.298564in}}%
\pgfpathlineto{\pgfqpoint{-16.402550in}{2.304244in}}%
\pgfpathlineto{\pgfqpoint{-16.449347in}{2.386807in}}%
\pgfpathlineto{\pgfqpoint{-16.496157in}{2.289046in}}%
\pgfpathlineto{\pgfqpoint{-16.542990in}{2.234339in}}%
\pgfpathlineto{\pgfqpoint{-16.589239in}{2.259754in}}%
\pgfpathlineto{\pgfqpoint{-16.637120in}{2.253182in}}%
\pgfpathlineto{\pgfqpoint{-16.684310in}{2.279136in}}%
\pgfpathlineto{\pgfqpoint{-16.730721in}{2.325183in}}%
\pgfpathlineto{\pgfqpoint{-16.778530in}{2.239852in}}%
\pgfpathlineto{\pgfqpoint{-16.825434in}{2.203625in}}%
\pgfpathlineto{\pgfqpoint{-16.871664in}{2.340897in}}%
\pgfpathlineto{\pgfqpoint{-16.918692in}{2.296669in}}%
\pgfpathlineto{\pgfqpoint{-16.964774in}{2.358343in}}%
\pgfpathlineto{\pgfqpoint{-17.010884in}{2.349502in}}%
\pgfpathlineto{\pgfqpoint{-17.058009in}{2.257529in}}%
\pgfpathlineto{\pgfqpoint{-17.103874in}{2.273976in}}%
\pgfpathlineto{\pgfqpoint{-17.150161in}{2.221952in}}%
\pgfpathlineto{\pgfqpoint{-17.198221in}{2.280826in}}%
\pgfpathlineto{\pgfqpoint{-17.244258in}{2.351371in}}%
\pgfpathlineto{\pgfqpoint{-17.289523in}{2.317891in}}%
\pgfpathlineto{\pgfqpoint{-17.336276in}{2.301880in}}%
\pgfpathlineto{\pgfqpoint{-17.381732in}{2.324057in}}%
\pgfpathlineto{\pgfqpoint{-17.427002in}{2.350595in}}%
\pgfpathlineto{\pgfqpoint{-17.474644in}{2.203921in}}%
\pgfpathlineto{\pgfqpoint{-17.521284in}{2.355287in}}%
\pgfpathlineto{\pgfqpoint{-17.566739in}{2.285077in}}%
\pgfpathlineto{\pgfqpoint{-17.613251in}{2.306760in}}%
\pgfpathlineto{\pgfqpoint{-17.658538in}{2.333441in}}%
\pgfpathlineto{\pgfqpoint{-17.704093in}{2.359958in}}%
\pgfpathlineto{\pgfqpoint{-17.751556in}{2.229229in}}%
\pgfpathlineto{\pgfqpoint{-17.797346in}{2.230089in}}%
\pgfpathlineto{\pgfqpoint{-17.843480in}{2.357277in}}%
\pgfpathlineto{\pgfqpoint{-17.891271in}{2.291080in}}%
\pgfpathlineto{\pgfqpoint{-17.937614in}{2.259828in}}%
\pgfpathlineto{\pgfqpoint{-17.984102in}{2.332692in}}%
\pgfpathlineto{\pgfqpoint{-18.032389in}{2.235310in}}%
\pgfpathlineto{\pgfqpoint{-18.078784in}{2.360674in}}%
\pgfpathlineto{\pgfqpoint{-18.124795in}{2.276617in}}%
\pgfpathlineto{\pgfqpoint{-18.172791in}{2.260071in}}%
\pgfpathlineto{\pgfqpoint{-18.218674in}{2.399865in}}%
\pgfpathlineto{\pgfqpoint{-18.264167in}{2.258422in}}%
\pgfpathlineto{\pgfqpoint{-18.311512in}{2.335174in}}%
\pgfpathlineto{\pgfqpoint{-18.356748in}{2.356543in}}%
\pgfpathlineto{\pgfqpoint{-18.402000in}{2.282506in}}%
\pgfpathlineto{\pgfqpoint{-18.448764in}{2.301035in}}%
\pgfpathlineto{\pgfqpoint{-18.494344in}{2.284564in}}%
\pgfpathlineto{\pgfqpoint{-18.540355in}{2.275306in}}%
\pgfpathlineto{\pgfqpoint{-18.587378in}{2.302467in}}%
\pgfpathlineto{\pgfqpoint{-18.632978in}{2.330547in}}%
\pgfpathlineto{\pgfqpoint{-18.678684in}{2.250711in}}%
\pgfpathlineto{\pgfqpoint{-18.726033in}{2.222199in}}%
\pgfpathlineto{\pgfqpoint{-18.772260in}{2.264712in}}%
\pgfpathlineto{\pgfqpoint{-18.817991in}{2.346468in}}%
\pgfpathlineto{\pgfqpoint{-18.865040in}{2.287380in}}%
\pgfpathlineto{\pgfqpoint{-18.910503in}{2.274485in}}%
\pgfpathlineto{\pgfqpoint{-18.956783in}{2.294777in}}%
\pgfpathlineto{\pgfqpoint{-19.004576in}{2.293680in}}%
\pgfpathlineto{\pgfqpoint{-19.050812in}{2.240312in}}%
\pgfpathlineto{\pgfqpoint{-19.097062in}{2.291417in}}%
\pgfpathlineto{\pgfqpoint{-19.143544in}{2.276235in}}%
\pgfpathlineto{\pgfqpoint{-19.188899in}{2.359779in}}%
\pgfpathlineto{\pgfqpoint{-19.234797in}{2.230054in}}%
\pgfpathlineto{\pgfqpoint{-19.283047in}{2.261364in}}%
\pgfpathlineto{\pgfqpoint{-19.328532in}{2.305435in}}%
\pgfpathlineto{\pgfqpoint{-19.374230in}{2.268469in}}%
\pgfpathlineto{\pgfqpoint{-19.421584in}{2.297187in}}%
\pgfpathlineto{\pgfqpoint{-19.467303in}{2.261142in}}%
\pgfpathlineto{\pgfqpoint{-19.512973in}{2.325186in}}%
\pgfpathlineto{\pgfqpoint{-19.559546in}{2.328474in}}%
\pgfpathlineto{\pgfqpoint{-19.604824in}{2.303450in}}%
\pgfpathlineto{\pgfqpoint{-19.650361in}{2.323431in}}%
\pgfpathlineto{\pgfqpoint{-19.696649in}{2.353123in}}%
\pgfpathlineto{\pgfqpoint{-19.742024in}{2.320379in}}%
\pgfpathlineto{\pgfqpoint{-19.787808in}{2.314462in}}%
\pgfpathlineto{\pgfqpoint{-19.833948in}{2.394099in}}%
\pgfpathlineto{\pgfqpoint{-19.878562in}{2.292822in}}%
\pgfpathlineto{\pgfqpoint{-19.924139in}{2.300804in}}%
\pgfpathlineto{\pgfqpoint{-19.970967in}{2.314648in}}%
\pgfpathlineto{\pgfqpoint{-20.016506in}{2.319684in}}%
\pgfpathlineto{\pgfqpoint{-20.061722in}{2.399221in}}%
\pgfpathlineto{\pgfqpoint{-20.107707in}{2.335560in}}%
\pgfpathlineto{\pgfqpoint{-20.153038in}{2.417828in}}%
\pgfpathlineto{\pgfqpoint{-20.197814in}{2.368490in}}%
\pgfpathlineto{\pgfqpoint{-20.244875in}{2.295788in}}%
\pgfpathlineto{\pgfqpoint{-20.290419in}{2.335942in}}%
\pgfpathlineto{\pgfqpoint{-20.335981in}{2.367234in}}%
\pgfpathlineto{\pgfqpoint{-20.382199in}{2.257547in}}%
\pgfpathlineto{\pgfqpoint{-20.428221in}{2.269362in}}%
\pgfpathlineto{\pgfqpoint{-20.474014in}{2.375131in}}%
\pgfpathlineto{\pgfqpoint{-20.519904in}{2.269668in}}%
\pgfpathlineto{\pgfqpoint{-20.565077in}{2.264133in}}%
\pgfpathlineto{\pgfqpoint{-20.611243in}{2.285186in}}%
\pgfpathlineto{\pgfqpoint{-20.658742in}{2.294828in}}%
\pgfpathlineto{\pgfqpoint{-20.705066in}{2.191512in}}%
\pgfpathlineto{\pgfqpoint{-20.750831in}{2.356202in}}%
\pgfpathlineto{\pgfqpoint{-20.797022in}{2.336609in}}%
\pgfpathlineto{\pgfqpoint{-20.842169in}{2.376578in}}%
\pgfpathlineto{\pgfqpoint{-20.886847in}{2.313127in}}%
\pgfpathlineto{\pgfqpoint{-20.932365in}{2.398000in}}%
\pgfpathlineto{\pgfqpoint{-20.976378in}{2.318752in}}%
\pgfpathlineto{\pgfqpoint{-21.021509in}{2.398401in}}%
\pgfpathlineto{\pgfqpoint{-21.067357in}{2.273525in}}%
\pgfpathlineto{\pgfqpoint{-21.113139in}{2.303852in}}%
\pgfpathlineto{\pgfqpoint{-21.158215in}{2.349197in}}%
\pgfpathlineto{\pgfqpoint{-21.204354in}{2.272682in}}%
\pgfpathlineto{\pgfqpoint{-21.249304in}{2.343950in}}%
\pgfpathlineto{\pgfqpoint{-21.294264in}{2.338666in}}%
\pgfpathlineto{\pgfqpoint{-21.340867in}{2.365018in}}%
\pgfpathlineto{\pgfqpoint{-21.385572in}{2.240229in}}%
\pgfpathlineto{\pgfqpoint{-21.430267in}{2.368915in}}%
\pgfpathlineto{\pgfqpoint{-21.475985in}{2.351596in}}%
\pgfpathlineto{\pgfqpoint{-21.520883in}{2.303547in}}%
\pgfpathlineto{\pgfqpoint{-21.566198in}{2.321312in}}%
\pgfpathlineto{\pgfqpoint{-21.612919in}{2.358435in}}%
\pgfpathlineto{\pgfqpoint{-21.657342in}{2.304006in}}%
\pgfpathlineto{\pgfqpoint{-21.703167in}{2.305719in}}%
\pgfpathlineto{\pgfqpoint{-21.749554in}{2.356484in}}%
\pgfpathlineto{\pgfqpoint{-21.794255in}{2.291838in}}%
\pgfpathlineto{\pgfqpoint{-21.839254in}{2.354507in}}%
\pgfpathlineto{\pgfqpoint{-21.885462in}{2.330493in}}%
\pgfpathlineto{\pgfqpoint{-21.930597in}{2.349659in}}%
\pgfpathlineto{\pgfqpoint{-21.976168in}{2.350045in}}%
\pgfpathlineto{\pgfqpoint{-22.022397in}{2.355753in}}%
\pgfpathlineto{\pgfqpoint{-22.067661in}{2.317625in}}%
\pgfpathlineto{\pgfqpoint{-22.112231in}{2.402816in}}%
\pgfpathlineto{\pgfqpoint{-22.158025in}{2.333442in}}%
\pgfpathlineto{\pgfqpoint{-22.201868in}{2.450307in}}%
\pgfpathlineto{\pgfqpoint{-22.246011in}{2.383905in}}%
\pgfpathlineto{\pgfqpoint{-22.291940in}{2.405300in}}%
\pgfpathlineto{\pgfqpoint{-22.336074in}{2.351361in}}%
\pgfpathlineto{\pgfqpoint{-22.380996in}{2.366754in}}%
\pgfpathlineto{\pgfqpoint{-22.427383in}{2.393311in}}%
\pgfpathlineto{\pgfqpoint{-22.471952in}{2.320511in}}%
\pgfpathlineto{\pgfqpoint{-22.516601in}{2.314775in}}%
\pgfpathlineto{\pgfqpoint{-22.562133in}{2.284579in}}%
\pgfpathlineto{\pgfqpoint{-22.606850in}{2.367685in}}%
\pgfpathlineto{\pgfqpoint{-22.651258in}{2.277162in}}%
\pgfpathlineto{\pgfqpoint{-22.697642in}{2.375017in}}%
\pgfpathlineto{\pgfqpoint{-22.742185in}{2.333142in}}%
\pgfpathlineto{\pgfqpoint{-22.786844in}{2.357634in}}%
\pgfpathlineto{\pgfqpoint{-22.833029in}{2.281819in}}%
\pgfpathlineto{\pgfqpoint{-22.878494in}{2.240583in}}%
\pgfpathlineto{\pgfqpoint{-22.924822in}{2.270987in}}%
\pgfpathlineto{\pgfqpoint{-22.972746in}{2.233471in}}%
\pgfpathlineto{\pgfqpoint{-23.018637in}{2.243112in}}%
\pgfpathlineto{\pgfqpoint{-23.065076in}{2.254446in}}%
\pgfpathlineto{\pgfqpoint{-23.112561in}{2.254185in}}%
\pgfpathlineto{\pgfqpoint{-23.159034in}{2.266056in}}%
\pgfpathlineto{\pgfqpoint{-23.205343in}{2.237468in}}%
\pgfpathlineto{\pgfqpoint{-23.252910in}{2.311575in}}%
\pgfpathlineto{\pgfqpoint{-23.298917in}{2.250652in}}%
\pgfpathlineto{\pgfqpoint{-23.344516in}{2.406366in}}%
\pgfpathlineto{\pgfqpoint{-23.391460in}{2.234958in}}%
\pgfpathlineto{\pgfqpoint{-23.437154in}{2.278529in}}%
\pgfpathlineto{\pgfqpoint{-23.482780in}{2.308873in}}%
\pgfpathlineto{\pgfqpoint{-23.529965in}{2.261696in}}%
\pgfpathlineto{\pgfqpoint{-23.575877in}{2.381697in}}%
\pgfpathlineto{\pgfqpoint{-23.620333in}{2.387220in}}%
\pgfpathlineto{\pgfqpoint{-23.666576in}{2.319036in}}%
\pgfpathlineto{\pgfqpoint{-23.711883in}{2.353354in}}%
\pgfpathlineto{\pgfqpoint{-23.757289in}{2.247028in}}%
\pgfpathlineto{\pgfqpoint{-23.804343in}{2.273113in}}%
\pgfpathlineto{\pgfqpoint{-23.849943in}{2.406439in}}%
\pgfpathlineto{\pgfqpoint{-23.895017in}{2.369110in}}%
\pgfpathlineto{\pgfqpoint{-23.941619in}{2.326998in}}%
\pgfpathlineto{\pgfqpoint{-23.986811in}{2.339578in}}%
\pgfpathlineto{\pgfqpoint{-24.032531in}{2.297753in}}%
\pgfpathlineto{\pgfqpoint{-24.078902in}{2.374804in}}%
\pgfpathlineto{\pgfqpoint{-24.123999in}{2.304795in}}%
\pgfpathlineto{\pgfqpoint{-24.169854in}{2.333026in}}%
\pgfpathlineto{\pgfqpoint{-24.216953in}{2.320747in}}%
\pgfpathlineto{\pgfqpoint{-24.262109in}{2.232536in}}%
\pgfpathlineto{\pgfqpoint{-24.308071in}{2.306542in}}%
\pgfpathlineto{\pgfqpoint{-24.355467in}{2.295788in}}%
\pgfpathlineto{\pgfqpoint{-24.401173in}{2.320591in}}%
\pgfpathlineto{\pgfqpoint{-24.447114in}{2.223770in}}%
\pgfpathlineto{\pgfqpoint{-24.494847in}{2.270967in}}%
\pgfpathlineto{\pgfqpoint{-24.540716in}{2.353522in}}%
\pgfpathlineto{\pgfqpoint{-24.586259in}{2.393133in}}%
\pgfpathlineto{\pgfqpoint{-24.632696in}{2.340546in}}%
\pgfpathlineto{\pgfqpoint{-24.677917in}{2.367505in}}%
\pgfpathlineto{\pgfqpoint{-24.722523in}{2.377694in}}%
\pgfpathlineto{\pgfqpoint{-24.768227in}{2.337717in}}%
\pgfpathlineto{\pgfqpoint{-24.813437in}{2.277965in}}%
\pgfpathlineto{\pgfqpoint{-24.858721in}{2.257669in}}%
\pgfpathlineto{\pgfqpoint{-24.905462in}{2.264512in}}%
\pgfpathlineto{\pgfqpoint{-24.950861in}{2.402740in}}%
\pgfpathlineto{\pgfqpoint{-24.995958in}{2.309445in}}%
\pgfpathlineto{\pgfqpoint{-25.043027in}{2.288248in}}%
\pgfpathlineto{\pgfqpoint{-25.088791in}{2.359011in}}%
\pgfpathlineto{\pgfqpoint{-25.133757in}{2.272653in}}%
\pgfpathlineto{\pgfqpoint{-25.181603in}{2.291067in}}%
\pgfpathlineto{\pgfqpoint{-25.227067in}{2.360627in}}%
\pgfpathlineto{\pgfqpoint{-25.271783in}{2.348997in}}%
\pgfpathlineto{\pgfqpoint{-25.318638in}{2.326455in}}%
\pgfpathlineto{\pgfqpoint{-25.364031in}{2.311396in}}%
\pgfpathlineto{\pgfqpoint{-25.409278in}{2.367974in}}%
\pgfpathlineto{\pgfqpoint{-25.455838in}{2.272708in}}%
\pgfpathlineto{\pgfqpoint{-25.501531in}{2.376055in}}%
\pgfpathlineto{\pgfqpoint{-25.546370in}{2.328600in}}%
\pgfpathlineto{\pgfqpoint{-25.593105in}{2.272352in}}%
\pgfpathlineto{\pgfqpoint{-25.638897in}{2.281275in}}%
\pgfpathlineto{\pgfqpoint{-25.684196in}{2.294335in}}%
\pgfpathlineto{\pgfqpoint{-25.731236in}{2.311073in}}%
\pgfpathlineto{\pgfqpoint{-25.776589in}{2.340637in}}%
\pgfpathlineto{\pgfqpoint{-25.821656in}{2.322218in}}%
\pgfpathlineto{\pgfqpoint{-25.868331in}{2.390510in}}%
\pgfpathlineto{\pgfqpoint{-25.913010in}{2.322586in}}%
\pgfpathlineto{\pgfqpoint{-25.957619in}{2.309584in}}%
\pgfpathlineto{\pgfqpoint{-26.004871in}{2.354180in}}%
\pgfpathlineto{\pgfqpoint{-26.049962in}{2.314984in}}%
\pgfpathlineto{\pgfqpoint{-26.095397in}{2.395267in}}%
\pgfpathlineto{\pgfqpoint{-26.140687in}{2.358697in}}%
\pgfpathlineto{\pgfqpoint{-26.185627in}{2.371658in}}%
\pgfpathlineto{\pgfqpoint{-26.230565in}{2.240058in}}%
\pgfpathlineto{\pgfqpoint{-26.277965in}{2.359459in}}%
\pgfpathlineto{\pgfqpoint{-26.323384in}{2.352393in}}%
\pgfpathlineto{\pgfqpoint{-26.368659in}{2.404176in}}%
\pgfpathlineto{\pgfqpoint{-26.414228in}{2.365948in}}%
\pgfpathlineto{\pgfqpoint{-26.458732in}{2.375345in}}%
\pgfpathlineto{\pgfqpoint{-26.503012in}{2.342913in}}%
\pgfpathlineto{\pgfqpoint{-26.549765in}{2.274534in}}%
\pgfpathlineto{\pgfqpoint{-26.595409in}{2.292757in}}%
\pgfpathlineto{\pgfqpoint{-26.640477in}{2.306144in}}%
\pgfpathlineto{\pgfqpoint{-26.688111in}{2.323786in}}%
\pgfpathlineto{\pgfqpoint{-26.735142in}{2.276180in}}%
\pgfpathlineto{\pgfqpoint{-26.781839in}{2.305652in}}%
\pgfpathlineto{\pgfqpoint{-26.830647in}{2.171250in}}%
\pgfpathlineto{\pgfqpoint{-26.880611in}{2.112084in}}%
\pgfpathlineto{\pgfqpoint{-26.933555in}{2.024750in}}%
\pgfpathlineto{\pgfqpoint{-26.988998in}{2.195111in}}%
\pgfpathlineto{\pgfqpoint{-27.037714in}{2.204576in}}%
\pgfpathlineto{\pgfqpoint{-27.086555in}{2.228159in}}%
\pgfpathlineto{\pgfqpoint{-27.136341in}{2.194731in}}%
\pgfpathlineto{\pgfqpoint{-27.184903in}{2.266072in}}%
\pgfpathlineto{\pgfqpoint{-27.232862in}{2.229339in}}%
\pgfpathlineto{\pgfqpoint{-27.281821in}{2.219948in}}%
\pgfpathlineto{\pgfqpoint{-27.328705in}{2.280610in}}%
\pgfpathlineto{\pgfqpoint{-27.374936in}{2.287585in}}%
\pgfpathlineto{\pgfqpoint{-27.422670in}{2.242602in}}%
\pgfpathlineto{\pgfqpoint{-27.470237in}{2.177911in}}%
\pgfpathlineto{\pgfqpoint{-27.516652in}{2.304360in}}%
\pgfpathlineto{\pgfqpoint{-27.563341in}{2.275353in}}%
\pgfpathlineto{\pgfqpoint{-27.608047in}{2.423736in}}%
\pgfpathlineto{\pgfqpoint{-27.651646in}{2.982598in}}%
\pgfpathlineto{\pgfqpoint{-27.696819in}{3.160374in}}%
\pgfpathlineto{\pgfqpoint{-27.740871in}{3.145313in}}%
\pgfpathlineto{\pgfqpoint{-27.785329in}{3.168955in}}%
\pgfpathlineto{\pgfqpoint{-27.830427in}{3.135647in}}%
\pgfpathlineto{\pgfqpoint{-27.874405in}{3.196053in}}%
\pgfpathlineto{\pgfqpoint{-27.917812in}{3.240135in}}%
\pgfpathlineto{\pgfqpoint{-27.963188in}{3.110191in}}%
\pgfpathlineto{\pgfqpoint{-28.007858in}{3.216226in}}%
\pgfpathlineto{\pgfqpoint{-28.051883in}{3.167259in}}%
\pgfpathlineto{\pgfqpoint{-28.097071in}{3.144981in}}%
\pgfpathlineto{\pgfqpoint{-28.141536in}{3.092116in}}%
\pgfpathlineto{\pgfqpoint{-28.185953in}{3.142196in}}%
\pgfpathlineto{\pgfqpoint{-28.230974in}{3.232716in}}%
\pgfpathlineto{\pgfqpoint{-28.274355in}{3.128685in}}%
\pgfpathlineto{\pgfqpoint{-28.318373in}{3.002463in}}%
\pgfpathlineto{\pgfqpoint{-28.364238in}{3.123971in}}%
\pgfpathlineto{\pgfqpoint{-28.408130in}{3.149746in}}%
\pgfpathlineto{\pgfqpoint{-28.452363in}{3.104546in}}%
\pgfpathlineto{\pgfqpoint{-28.498901in}{3.007158in}}%
\pgfpathlineto{\pgfqpoint{-28.543533in}{3.092765in}}%
\pgfpathlineto{\pgfqpoint{-28.588250in}{2.142129in}}%
\pgfpathclose%
\pgfusepath{fill}%
\end{pgfscope}%
\begin{pgfscope}%
\pgfsetbuttcap%
\pgfsetroundjoin%
\definecolor{currentfill}{rgb}{0.000000,0.000000,0.000000}%
\pgfsetfillcolor{currentfill}%
\pgfsetlinewidth{0.803000pt}%
\definecolor{currentstroke}{rgb}{0.000000,0.000000,0.000000}%
\pgfsetstrokecolor{currentstroke}%
\pgfsetdash{}{0pt}%
\pgfsys@defobject{currentmarker}{\pgfqpoint{0.000000in}{-0.048611in}}{\pgfqpoint{0.000000in}{0.000000in}}{%
\pgfpathmoveto{\pgfqpoint{0.000000in}{0.000000in}}%
\pgfpathlineto{\pgfqpoint{0.000000in}{-0.048611in}}%
\pgfusepath{stroke,fill}%
}%
\begin{pgfscope}%
\pgfsys@transformshift{2.693180in}{0.773588in}%
\pgfsys@useobject{currentmarker}{}%
\end{pgfscope}%
\end{pgfscope}%
\begin{pgfscope}%
\definecolor{textcolor}{rgb}{0.000000,0.000000,0.000000}%
\pgfsetstrokecolor{textcolor}%
\pgfsetfillcolor{textcolor}%
\pgftext[x=2.693180in,y=0.676366in,,top]{\color{textcolor}\rmfamily\fontsize{10.000000}{12.000000}\selectfont \(\displaystyle {3250}\)}%
\end{pgfscope}%
\begin{pgfscope}%
\pgfsetbuttcap%
\pgfsetroundjoin%
\definecolor{currentfill}{rgb}{0.000000,0.000000,0.000000}%
\pgfsetfillcolor{currentfill}%
\pgfsetlinewidth{0.803000pt}%
\definecolor{currentstroke}{rgb}{0.000000,0.000000,0.000000}%
\pgfsetstrokecolor{currentstroke}%
\pgfsetdash{}{0pt}%
\pgfsys@defobject{currentmarker}{\pgfqpoint{0.000000in}{-0.048611in}}{\pgfqpoint{0.000000in}{0.000000in}}{%
\pgfpathmoveto{\pgfqpoint{0.000000in}{0.000000in}}%
\pgfpathlineto{\pgfqpoint{0.000000in}{-0.048611in}}%
\pgfusepath{stroke,fill}%
}%
\begin{pgfscope}%
\pgfsys@transformshift{3.174687in}{0.773588in}%
\pgfsys@useobject{currentmarker}{}%
\end{pgfscope}%
\end{pgfscope}%
\begin{pgfscope}%
\definecolor{textcolor}{rgb}{0.000000,0.000000,0.000000}%
\pgfsetstrokecolor{textcolor}%
\pgfsetfillcolor{textcolor}%
\pgftext[x=3.174687in,y=0.676366in,,top]{\color{textcolor}\rmfamily\fontsize{10.000000}{12.000000}\selectfont \(\displaystyle {3300}\)}%
\end{pgfscope}%
\begin{pgfscope}%
\pgfsetbuttcap%
\pgfsetroundjoin%
\definecolor{currentfill}{rgb}{0.000000,0.000000,0.000000}%
\pgfsetfillcolor{currentfill}%
\pgfsetlinewidth{0.803000pt}%
\definecolor{currentstroke}{rgb}{0.000000,0.000000,0.000000}%
\pgfsetstrokecolor{currentstroke}%
\pgfsetdash{}{0pt}%
\pgfsys@defobject{currentmarker}{\pgfqpoint{0.000000in}{-0.048611in}}{\pgfqpoint{0.000000in}{0.000000in}}{%
\pgfpathmoveto{\pgfqpoint{0.000000in}{0.000000in}}%
\pgfpathlineto{\pgfqpoint{0.000000in}{-0.048611in}}%
\pgfusepath{stroke,fill}%
}%
\begin{pgfscope}%
\pgfsys@transformshift{3.656193in}{0.773588in}%
\pgfsys@useobject{currentmarker}{}%
\end{pgfscope}%
\end{pgfscope}%
\begin{pgfscope}%
\definecolor{textcolor}{rgb}{0.000000,0.000000,0.000000}%
\pgfsetstrokecolor{textcolor}%
\pgfsetfillcolor{textcolor}%
\pgftext[x=3.656193in,y=0.676366in,,top]{\color{textcolor}\rmfamily\fontsize{10.000000}{12.000000}\selectfont \(\displaystyle {3350}\)}%
\end{pgfscope}%
\begin{pgfscope}%
\pgfsetbuttcap%
\pgfsetroundjoin%
\definecolor{currentfill}{rgb}{0.000000,0.000000,0.000000}%
\pgfsetfillcolor{currentfill}%
\pgfsetlinewidth{0.803000pt}%
\definecolor{currentstroke}{rgb}{0.000000,0.000000,0.000000}%
\pgfsetstrokecolor{currentstroke}%
\pgfsetdash{}{0pt}%
\pgfsys@defobject{currentmarker}{\pgfqpoint{0.000000in}{-0.048611in}}{\pgfqpoint{0.000000in}{0.000000in}}{%
\pgfpathmoveto{\pgfqpoint{0.000000in}{0.000000in}}%
\pgfpathlineto{\pgfqpoint{0.000000in}{-0.048611in}}%
\pgfusepath{stroke,fill}%
}%
\begin{pgfscope}%
\pgfsys@transformshift{4.137700in}{0.773588in}%
\pgfsys@useobject{currentmarker}{}%
\end{pgfscope}%
\end{pgfscope}%
\begin{pgfscope}%
\definecolor{textcolor}{rgb}{0.000000,0.000000,0.000000}%
\pgfsetstrokecolor{textcolor}%
\pgfsetfillcolor{textcolor}%
\pgftext[x=4.137700in,y=0.676366in,,top]{\color{textcolor}\rmfamily\fontsize{10.000000}{12.000000}\selectfont \(\displaystyle {3400}\)}%
\end{pgfscope}%
\begin{pgfscope}%
\pgfsetbuttcap%
\pgfsetroundjoin%
\definecolor{currentfill}{rgb}{0.000000,0.000000,0.000000}%
\pgfsetfillcolor{currentfill}%
\pgfsetlinewidth{0.803000pt}%
\definecolor{currentstroke}{rgb}{0.000000,0.000000,0.000000}%
\pgfsetstrokecolor{currentstroke}%
\pgfsetdash{}{0pt}%
\pgfsys@defobject{currentmarker}{\pgfqpoint{0.000000in}{-0.048611in}}{\pgfqpoint{0.000000in}{0.000000in}}{%
\pgfpathmoveto{\pgfqpoint{0.000000in}{0.000000in}}%
\pgfpathlineto{\pgfqpoint{0.000000in}{-0.048611in}}%
\pgfusepath{stroke,fill}%
}%
\begin{pgfscope}%
\pgfsys@transformshift{4.619206in}{0.773588in}%
\pgfsys@useobject{currentmarker}{}%
\end{pgfscope}%
\end{pgfscope}%
\begin{pgfscope}%
\definecolor{textcolor}{rgb}{0.000000,0.000000,0.000000}%
\pgfsetstrokecolor{textcolor}%
\pgfsetfillcolor{textcolor}%
\pgftext[x=4.619206in,y=0.676366in,,top]{\color{textcolor}\rmfamily\fontsize{10.000000}{12.000000}\selectfont \(\displaystyle {3450}\)}%
\end{pgfscope}%
\begin{pgfscope}%
\pgfsetbuttcap%
\pgfsetroundjoin%
\definecolor{currentfill}{rgb}{0.000000,0.000000,0.000000}%
\pgfsetfillcolor{currentfill}%
\pgfsetlinewidth{0.803000pt}%
\definecolor{currentstroke}{rgb}{0.000000,0.000000,0.000000}%
\pgfsetstrokecolor{currentstroke}%
\pgfsetdash{}{0pt}%
\pgfsys@defobject{currentmarker}{\pgfqpoint{0.000000in}{-0.048611in}}{\pgfqpoint{0.000000in}{0.000000in}}{%
\pgfpathmoveto{\pgfqpoint{0.000000in}{0.000000in}}%
\pgfpathlineto{\pgfqpoint{0.000000in}{-0.048611in}}%
\pgfusepath{stroke,fill}%
}%
\begin{pgfscope}%
\pgfsys@transformshift{5.100713in}{0.773588in}%
\pgfsys@useobject{currentmarker}{}%
\end{pgfscope}%
\end{pgfscope}%
\begin{pgfscope}%
\definecolor{textcolor}{rgb}{0.000000,0.000000,0.000000}%
\pgfsetstrokecolor{textcolor}%
\pgfsetfillcolor{textcolor}%
\pgftext[x=5.100713in,y=0.676366in,,top]{\color{textcolor}\rmfamily\fontsize{10.000000}{12.000000}\selectfont \(\displaystyle {3500}\)}%
\end{pgfscope}%
\begin{pgfscope}%
\pgfsetbuttcap%
\pgfsetroundjoin%
\definecolor{currentfill}{rgb}{0.000000,0.000000,0.000000}%
\pgfsetfillcolor{currentfill}%
\pgfsetlinewidth{0.803000pt}%
\definecolor{currentstroke}{rgb}{0.000000,0.000000,0.000000}%
\pgfsetstrokecolor{currentstroke}%
\pgfsetdash{}{0pt}%
\pgfsys@defobject{currentmarker}{\pgfqpoint{0.000000in}{-0.048611in}}{\pgfqpoint{0.000000in}{0.000000in}}{%
\pgfpathmoveto{\pgfqpoint{0.000000in}{0.000000in}}%
\pgfpathlineto{\pgfqpoint{0.000000in}{-0.048611in}}%
\pgfusepath{stroke,fill}%
}%
\begin{pgfscope}%
\pgfsys@transformshift{5.582219in}{0.773588in}%
\pgfsys@useobject{currentmarker}{}%
\end{pgfscope}%
\end{pgfscope}%
\begin{pgfscope}%
\definecolor{textcolor}{rgb}{0.000000,0.000000,0.000000}%
\pgfsetstrokecolor{textcolor}%
\pgfsetfillcolor{textcolor}%
\pgftext[x=5.582219in,y=0.676366in,,top]{\color{textcolor}\rmfamily\fontsize{10.000000}{12.000000}\selectfont \(\displaystyle {3550}\)}%
\end{pgfscope}%
\begin{pgfscope}%
\pgfsetrectcap%
\pgfsetroundjoin%
\pgfsetlinewidth{0.803000pt}%
\definecolor{currentstroke}{rgb}{0.000000,0.000000,0.000000}%
\pgfsetstrokecolor{currentstroke}%
\pgfsetdash{}{0pt}%
\pgfpathmoveto{\pgfqpoint{2.595769in}{0.707284in}}%
\pgfpathlineto{\pgfqpoint{2.728378in}{0.839893in}}%
\pgfusepath{stroke}%
\end{pgfscope}%
\begin{pgfscope}%
\pgfpathrectangle{\pgfqpoint{2.662073in}{0.773588in}}{\pgfqpoint{2.964025in}{5.415119in}}%
\pgfusepath{clip}%
\pgfsetrectcap%
\pgfsetroundjoin%
\pgfsetlinewidth{1.505625pt}%
\definecolor{currentstroke}{rgb}{0.000000,0.000000,1.000000}%
\pgfsetstrokecolor{currentstroke}%
\pgfsetdash{}{0pt}%
\pgfusepath{stroke}%
\end{pgfscope}%
\begin{pgfscope}%
\pgfpathrectangle{\pgfqpoint{2.662073in}{0.773588in}}{\pgfqpoint{2.964025in}{5.415119in}}%
\pgfusepath{clip}%
\pgfsetrectcap%
\pgfsetroundjoin%
\pgfsetlinewidth{1.505625pt}%
\definecolor{currentstroke}{rgb}{0.750000,0.750000,0.000000}%
\pgfsetstrokecolor{currentstroke}%
\pgfsetdash{}{0pt}%
\pgfpathmoveto{\pgfqpoint{3.143580in}{0.773588in}}%
\pgfpathlineto{\pgfqpoint{3.143580in}{6.188708in}}%
\pgfusepath{stroke}%
\end{pgfscope}%
\begin{pgfscope}%
\pgfpathrectangle{\pgfqpoint{2.662073in}{0.773588in}}{\pgfqpoint{2.964025in}{5.415119in}}%
\pgfusepath{clip}%
\pgfsetrectcap%
\pgfsetroundjoin%
\pgfsetlinewidth{1.505625pt}%
\definecolor{currentstroke}{rgb}{0.750000,0.000000,0.750000}%
\pgfsetstrokecolor{currentstroke}%
\pgfsetdash{}{0pt}%
\pgfpathmoveto{\pgfqpoint{3.173660in}{0.773588in}}%
\pgfpathlineto{\pgfqpoint{3.173660in}{6.188708in}}%
\pgfusepath{stroke}%
\end{pgfscope}%
\begin{pgfscope}%
\pgfpathrectangle{\pgfqpoint{2.662073in}{0.773588in}}{\pgfqpoint{2.964025in}{5.415119in}}%
\pgfusepath{clip}%
\pgfsetrectcap%
\pgfsetroundjoin%
\pgfsetlinewidth{1.505625pt}%
\definecolor{currentstroke}{rgb}{1.000000,0.000000,0.000000}%
\pgfsetstrokecolor{currentstroke}%
\pgfsetdash{}{0pt}%
\pgfpathmoveto{\pgfqpoint{3.842751in}{0.773588in}}%
\pgfpathlineto{\pgfqpoint{3.842751in}{6.188708in}}%
\pgfusepath{stroke}%
\end{pgfscope}%
\begin{pgfscope}%
\pgfpathrectangle{\pgfqpoint{2.662073in}{0.773588in}}{\pgfqpoint{2.964025in}{5.415119in}}%
\pgfusepath{clip}%
\pgfsetrectcap%
\pgfsetroundjoin%
\pgfsetlinewidth{1.505625pt}%
\definecolor{currentstroke}{rgb}{0.000000,0.500000,0.000000}%
\pgfsetstrokecolor{currentstroke}%
\pgfsetdash{}{0pt}%
\pgfpathmoveto{\pgfqpoint{3.842811in}{0.773588in}}%
\pgfpathlineto{\pgfqpoint{3.842811in}{6.188708in}}%
\pgfusepath{stroke}%
\end{pgfscope}%
\begin{pgfscope}%
\pgfpathrectangle{\pgfqpoint{2.662073in}{0.773588in}}{\pgfqpoint{2.964025in}{5.415119in}}%
\pgfusepath{clip}%
\pgfsetrectcap%
\pgfsetroundjoin%
\pgfsetlinewidth{1.505625pt}%
\definecolor{currentstroke}{rgb}{0.000000,0.750000,0.750000}%
\pgfsetstrokecolor{currentstroke}%
\pgfsetdash{}{0pt}%
\pgfpathmoveto{\pgfqpoint{4.248620in}{0.773588in}}%
\pgfpathlineto{\pgfqpoint{4.248620in}{6.188708in}}%
\pgfusepath{stroke}%
\end{pgfscope}%
\begin{pgfscope}%
\pgfpathrectangle{\pgfqpoint{2.662073in}{0.773588in}}{\pgfqpoint{2.964025in}{5.415119in}}%
\pgfusepath{clip}%
\pgfsetrectcap%
\pgfsetroundjoin%
\pgfsetlinewidth{1.505625pt}%
\definecolor{currentstroke}{rgb}{0.750000,0.000000,0.750000}%
\pgfsetstrokecolor{currentstroke}%
\pgfsetdash{}{0pt}%
\pgfpathmoveto{\pgfqpoint{4.663085in}{0.773588in}}%
\pgfpathlineto{\pgfqpoint{4.663085in}{6.188708in}}%
\pgfusepath{stroke}%
\end{pgfscope}%
\begin{pgfscope}%
\pgfsetrectcap%
\pgfsetmiterjoin%
\pgfsetlinewidth{0.803000pt}%
\definecolor{currentstroke}{rgb}{0.000000,0.000000,0.000000}%
\pgfsetstrokecolor{currentstroke}%
\pgfsetdash{}{0pt}%
\pgfpathmoveto{\pgfqpoint{2.662073in}{0.773588in}}%
\pgfpathlineto{\pgfqpoint{5.626098in}{0.773588in}}%
\pgfusepath{stroke}%
\end{pgfscope}%
\begin{pgfscope}%
\pgfsetbuttcap%
\pgfsetmiterjoin%
\pgfsetlinewidth{0.000000pt}%
\definecolor{currentstroke}{rgb}{0.000000,0.000000,0.000000}%
\pgfsetstrokecolor{currentstroke}%
\pgfsetstrokeopacity{0.000000}%
\pgfsetdash{}{0pt}%
\pgfpathmoveto{\pgfqpoint{0.781402in}{0.773588in}}%
\pgfpathlineto{\pgfqpoint{5.626098in}{0.773588in}}%
\pgfpathlineto{\pgfqpoint{5.626098in}{6.188708in}}%
\pgfpathlineto{\pgfqpoint{0.781402in}{6.188708in}}%
\pgfpathclose%
\pgfusepath{}%
\end{pgfscope}%
\begin{pgfscope}%
\definecolor{textcolor}{rgb}{0.000000,0.000000,0.000000}%
\pgfsetstrokecolor{textcolor}%
\pgfsetfillcolor{textcolor}%
\pgftext[x=3.203750in,y=0.356922in,,top]{\color{textcolor}\rmfamily\fontsize{10.000000}{12.000000}\selectfont Time (milliseconds)}%
\end{pgfscope}%
\begin{pgfscope}%
\definecolor{textcolor}{rgb}{0.000000,0.000000,0.000000}%
\pgfsetstrokecolor{textcolor}%
\pgfsetfillcolor{textcolor}%
\pgftext[x=0.364736in,y=3.481148in,,bottom,rotate=90.000000]{\color{textcolor}\rmfamily\fontsize{10.000000}{12.000000}\selectfont Throughput (million operations/second)}%
\end{pgfscope}%
\begin{pgfscope}%
\pgfsetbuttcap%
\pgfsetmiterjoin%
\definecolor{currentfill}{rgb}{1.000000,1.000000,1.000000}%
\pgfsetfillcolor{currentfill}%
\pgfsetfillopacity{0.800000}%
\pgfsetlinewidth{1.003750pt}%
\definecolor{currentstroke}{rgb}{0.800000,0.800000,0.800000}%
\pgfsetstrokecolor{currentstroke}%
\pgfsetstrokeopacity{0.800000}%
\pgfsetdash{}{0pt}%
\pgfpathmoveto{\pgfqpoint{0.878625in}{3.559851in}}%
\pgfpathlineto{\pgfqpoint{3.799925in}{3.559851in}}%
\pgfpathquadraticcurveto{\pgfqpoint{3.827703in}{3.559851in}}{\pgfqpoint{3.827703in}{3.587628in}}%
\pgfpathlineto{\pgfqpoint{3.827703in}{6.091486in}}%
\pgfpathquadraticcurveto{\pgfqpoint{3.827703in}{6.119263in}}{\pgfqpoint{3.799925in}{6.119263in}}%
\pgfpathlineto{\pgfqpoint{0.878625in}{6.119263in}}%
\pgfpathquadraticcurveto{\pgfqpoint{0.850847in}{6.119263in}}{\pgfqpoint{0.850847in}{6.091486in}}%
\pgfpathlineto{\pgfqpoint{0.850847in}{3.587628in}}%
\pgfpathquadraticcurveto{\pgfqpoint{0.850847in}{3.559851in}}{\pgfqpoint{0.878625in}{3.559851in}}%
\pgfpathclose%
\pgfusepath{stroke,fill}%
\end{pgfscope}%
\begin{pgfscope}%
\pgfsetrectcap%
\pgfsetroundjoin%
\pgfsetlinewidth{1.505625pt}%
\definecolor{currentstroke}{rgb}{0.000000,0.000000,1.000000}%
\pgfsetstrokecolor{currentstroke}%
\pgfsetdash{}{0pt}%
\pgfpathmoveto{\pgfqpoint{0.906402in}{6.015097in}}%
\pgfpathlineto{\pgfqpoint{1.184180in}{6.015097in}}%
\pgfusepath{stroke}%
\end{pgfscope}%
\begin{pgfscope}%
\definecolor{textcolor}{rgb}{0.000000,0.000000,0.000000}%
\pgfsetstrokecolor{textcolor}%
\pgfsetfillcolor{textcolor}%
\pgftext[x=1.295291in,y=5.966486in,left,base]{\color{textcolor}\rmfamily\fontsize{10.000000}{12.000000}\selectfont Started migration}%
\end{pgfscope}%
\begin{pgfscope}%
\pgfsetrectcap%
\pgfsetroundjoin%
\pgfsetlinewidth{1.505625pt}%
\definecolor{currentstroke}{rgb}{0.750000,0.750000,0.000000}%
\pgfsetstrokecolor{currentstroke}%
\pgfsetdash{}{0pt}%
\pgfpathmoveto{\pgfqpoint{0.906402in}{5.821424in}}%
\pgfpathlineto{\pgfqpoint{1.184180in}{5.821424in}}%
\pgfusepath{stroke}%
\end{pgfscope}%
\begin{pgfscope}%
\definecolor{textcolor}{rgb}{0.000000,0.000000,0.000000}%
\pgfsetstrokecolor{textcolor}%
\pgfsetfillcolor{textcolor}%
\pgftext[x=1.295291in,y=5.772813in,left,base]{\color{textcolor}\rmfamily\fontsize{10.000000}{12.000000}\selectfont Finished prefill writes}%
\end{pgfscope}%
\begin{pgfscope}%
\pgfsetrectcap%
\pgfsetroundjoin%
\pgfsetlinewidth{1.505625pt}%
\definecolor{currentstroke}{rgb}{0.750000,0.000000,0.750000}%
\pgfsetstrokecolor{currentstroke}%
\pgfsetdash{}{0pt}%
\pgfpathmoveto{\pgfqpoint{0.906402in}{5.627751in}}%
\pgfpathlineto{\pgfqpoint{1.184180in}{5.627751in}}%
\pgfusepath{stroke}%
\end{pgfscope}%
\begin{pgfscope}%
\definecolor{textcolor}{rgb}{0.000000,0.000000,0.000000}%
\pgfsetstrokecolor{textcolor}%
\pgfsetfillcolor{textcolor}%
\pgftext[x=1.295291in,y=5.579140in,left,base]{\color{textcolor}\rmfamily\fontsize{10.000000}{12.000000}\selectfont finished writes}%
\end{pgfscope}%
\begin{pgfscope}%
\pgfsetrectcap%
\pgfsetroundjoin%
\pgfsetlinewidth{1.505625pt}%
\definecolor{currentstroke}{rgb}{1.000000,0.000000,0.000000}%
\pgfsetstrokecolor{currentstroke}%
\pgfsetdash{}{0pt}%
\pgfpathmoveto{\pgfqpoint{0.906402in}{5.434078in}}%
\pgfpathlineto{\pgfqpoint{1.184180in}{5.434078in}}%
\pgfusepath{stroke}%
\end{pgfscope}%
\begin{pgfscope}%
\definecolor{textcolor}{rgb}{0.000000,0.000000,0.000000}%
\pgfsetstrokecolor{textcolor}%
\pgfsetfillcolor{textcolor}%
\pgftext[x=1.295291in,y=5.385467in,left,base]{\color{textcolor}\rmfamily\fontsize{10.000000}{12.000000}\selectfont finished reads}%
\end{pgfscope}%
\begin{pgfscope}%
\pgfsetrectcap%
\pgfsetroundjoin%
\pgfsetlinewidth{1.505625pt}%
\definecolor{currentstroke}{rgb}{0.000000,0.500000,0.000000}%
\pgfsetstrokecolor{currentstroke}%
\pgfsetdash{}{0pt}%
\pgfpathmoveto{\pgfqpoint{0.906402in}{5.240406in}}%
\pgfpathlineto{\pgfqpoint{1.184180in}{5.240406in}}%
\pgfusepath{stroke}%
\end{pgfscope}%
\begin{pgfscope}%
\definecolor{textcolor}{rgb}{0.000000,0.000000,0.000000}%
\pgfsetstrokecolor{textcolor}%
\pgfsetfillcolor{textcolor}%
\pgftext[x=1.295291in,y=5.191794in,left,base]{\color{textcolor}\rmfamily\fontsize{10.000000}{12.000000}\selectfont Transferred ownership to the destination}%
\end{pgfscope}%
\begin{pgfscope}%
\pgfsetrectcap%
\pgfsetroundjoin%
\pgfsetlinewidth{1.505625pt}%
\definecolor{currentstroke}{rgb}{0.000000,0.750000,0.750000}%
\pgfsetstrokecolor{currentstroke}%
\pgfsetdash{}{0pt}%
\pgfpathmoveto{\pgfqpoint{0.906402in}{5.046733in}}%
\pgfpathlineto{\pgfqpoint{1.184180in}{5.046733in}}%
\pgfusepath{stroke}%
\end{pgfscope}%
\begin{pgfscope}%
\definecolor{textcolor}{rgb}{0.000000,0.000000,0.000000}%
\pgfsetstrokecolor{textcolor}%
\pgfsetfillcolor{textcolor}%
\pgftext[x=1.295291in,y=4.998122in,left,base]{\color{textcolor}\rmfamily\fontsize{10.000000}{12.000000}\selectfont Started reading dirty pages}%
\end{pgfscope}%
\begin{pgfscope}%
\pgfsetrectcap%
\pgfsetroundjoin%
\pgfsetlinewidth{1.505625pt}%
\definecolor{currentstroke}{rgb}{0.750000,0.000000,0.750000}%
\pgfsetstrokecolor{currentstroke}%
\pgfsetdash{}{0pt}%
\pgfpathmoveto{\pgfqpoint{0.906402in}{4.853060in}}%
\pgfpathlineto{\pgfqpoint{1.184180in}{4.853060in}}%
\pgfusepath{stroke}%
\end{pgfscope}%
\begin{pgfscope}%
\definecolor{textcolor}{rgb}{0.000000,0.000000,0.000000}%
\pgfsetstrokecolor{textcolor}%
\pgfsetfillcolor{textcolor}%
\pgftext[x=1.295291in,y=4.804449in,left,base]{\color{textcolor}\rmfamily\fontsize{10.000000}{12.000000}\selectfont Finished reading dirty pages}%
\end{pgfscope}%
\begin{pgfscope}%
\pgfsetbuttcap%
\pgfsetmiterjoin%
\definecolor{currentfill}{rgb}{0.121569,0.466667,0.705882}%
\pgfsetfillcolor{currentfill}%
\pgfsetlinewidth{0.000000pt}%
\definecolor{currentstroke}{rgb}{0.000000,0.000000,0.000000}%
\pgfsetstrokecolor{currentstroke}%
\pgfsetstrokeopacity{0.000000}%
\pgfsetdash{}{0pt}%
\pgfpathmoveto{\pgfqpoint{0.906402in}{4.610776in}}%
\pgfpathlineto{\pgfqpoint{1.184180in}{4.610776in}}%
\pgfpathlineto{\pgfqpoint{1.184180in}{4.707998in}}%
\pgfpathlineto{\pgfqpoint{0.906402in}{4.707998in}}%
\pgfpathclose%
\pgfusepath{fill}%
\end{pgfscope}%
\begin{pgfscope}%
\definecolor{textcolor}{rgb}{0.000000,0.000000,0.000000}%
\pgfsetstrokecolor{textcolor}%
\pgfsetfillcolor{textcolor}%
\pgftext[x=1.295291in,y=4.610776in,left,base]{\color{textcolor}\rmfamily\fontsize{10.000000}{12.000000}\selectfont MP read at destination}%
\end{pgfscope}%
\begin{pgfscope}%
\pgfsetbuttcap%
\pgfsetmiterjoin%
\definecolor{currentfill}{rgb}{1.000000,0.498039,0.054902}%
\pgfsetfillcolor{currentfill}%
\pgfsetlinewidth{0.000000pt}%
\definecolor{currentstroke}{rgb}{0.000000,0.000000,0.000000}%
\pgfsetstrokecolor{currentstroke}%
\pgfsetstrokeopacity{0.000000}%
\pgfsetdash{}{0pt}%
\pgfpathmoveto{\pgfqpoint{0.906402in}{4.417103in}}%
\pgfpathlineto{\pgfqpoint{1.184180in}{4.417103in}}%
\pgfpathlineto{\pgfqpoint{1.184180in}{4.514326in}}%
\pgfpathlineto{\pgfqpoint{0.906402in}{4.514326in}}%
\pgfpathclose%
\pgfusepath{fill}%
\end{pgfscope}%
\begin{pgfscope}%
\definecolor{textcolor}{rgb}{0.000000,0.000000,0.000000}%
\pgfsetstrokecolor{textcolor}%
\pgfsetfillcolor{textcolor}%
\pgftext[x=1.295291in,y=4.417103in,left,base]{\color{textcolor}\rmfamily\fontsize{10.000000}{12.000000}\selectfont MP update at destination}%
\end{pgfscope}%
\begin{pgfscope}%
\pgfsetbuttcap%
\pgfsetmiterjoin%
\definecolor{currentfill}{rgb}{0.172549,0.627451,0.172549}%
\pgfsetfillcolor{currentfill}%
\pgfsetlinewidth{0.000000pt}%
\definecolor{currentstroke}{rgb}{0.000000,0.000000,0.000000}%
\pgfsetstrokecolor{currentstroke}%
\pgfsetstrokeopacity{0.000000}%
\pgfsetdash{}{0pt}%
\pgfpathmoveto{\pgfqpoint{0.906402in}{4.223431in}}%
\pgfpathlineto{\pgfqpoint{1.184180in}{4.223431in}}%
\pgfpathlineto{\pgfqpoint{1.184180in}{4.320653in}}%
\pgfpathlineto{\pgfqpoint{0.906402in}{4.320653in}}%
\pgfpathclose%
\pgfusepath{fill}%
\end{pgfscope}%
\begin{pgfscope}%
\definecolor{textcolor}{rgb}{0.000000,0.000000,0.000000}%
\pgfsetstrokecolor{textcolor}%
\pgfsetfillcolor{textcolor}%
\pgftext[x=1.295291in,y=4.223431in,left,base]{\color{textcolor}\rmfamily\fontsize{10.000000}{12.000000}\selectfont MP insert at destination}%
\end{pgfscope}%
\begin{pgfscope}%
\pgfsetbuttcap%
\pgfsetmiterjoin%
\definecolor{currentfill}{rgb}{0.839216,0.152941,0.156863}%
\pgfsetfillcolor{currentfill}%
\pgfsetlinewidth{0.000000pt}%
\definecolor{currentstroke}{rgb}{0.000000,0.000000,0.000000}%
\pgfsetstrokecolor{currentstroke}%
\pgfsetstrokeopacity{0.000000}%
\pgfsetdash{}{0pt}%
\pgfpathmoveto{\pgfqpoint{0.906402in}{4.029758in}}%
\pgfpathlineto{\pgfqpoint{1.184180in}{4.029758in}}%
\pgfpathlineto{\pgfqpoint{1.184180in}{4.126980in}}%
\pgfpathlineto{\pgfqpoint{0.906402in}{4.126980in}}%
\pgfpathclose%
\pgfusepath{fill}%
\end{pgfscope}%
\begin{pgfscope}%
\definecolor{textcolor}{rgb}{0.000000,0.000000,0.000000}%
\pgfsetstrokecolor{textcolor}%
\pgfsetfillcolor{textcolor}%
\pgftext[x=1.295291in,y=4.029758in,left,base]{\color{textcolor}\rmfamily\fontsize{10.000000}{12.000000}\selectfont MP read at source}%
\end{pgfscope}%
\begin{pgfscope}%
\pgfsetbuttcap%
\pgfsetmiterjoin%
\definecolor{currentfill}{rgb}{0.580392,0.403922,0.741176}%
\pgfsetfillcolor{currentfill}%
\pgfsetlinewidth{0.000000pt}%
\definecolor{currentstroke}{rgb}{0.000000,0.000000,0.000000}%
\pgfsetstrokecolor{currentstroke}%
\pgfsetstrokeopacity{0.000000}%
\pgfsetdash{}{0pt}%
\pgfpathmoveto{\pgfqpoint{0.906402in}{3.836085in}}%
\pgfpathlineto{\pgfqpoint{1.184180in}{3.836085in}}%
\pgfpathlineto{\pgfqpoint{1.184180in}{3.933307in}}%
\pgfpathlineto{\pgfqpoint{0.906402in}{3.933307in}}%
\pgfpathclose%
\pgfusepath{fill}%
\end{pgfscope}%
\begin{pgfscope}%
\definecolor{textcolor}{rgb}{0.000000,0.000000,0.000000}%
\pgfsetstrokecolor{textcolor}%
\pgfsetfillcolor{textcolor}%
\pgftext[x=1.295291in,y=3.836085in,left,base]{\color{textcolor}\rmfamily\fontsize{10.000000}{12.000000}\selectfont MP update at source}%
\end{pgfscope}%
\begin{pgfscope}%
\pgfsetbuttcap%
\pgfsetmiterjoin%
\definecolor{currentfill}{rgb}{0.549020,0.337255,0.294118}%
\pgfsetfillcolor{currentfill}%
\pgfsetlinewidth{0.000000pt}%
\definecolor{currentstroke}{rgb}{0.000000,0.000000,0.000000}%
\pgfsetstrokecolor{currentstroke}%
\pgfsetstrokeopacity{0.000000}%
\pgfsetdash{}{0pt}%
\pgfpathmoveto{\pgfqpoint{0.906402in}{3.642412in}}%
\pgfpathlineto{\pgfqpoint{1.184180in}{3.642412in}}%
\pgfpathlineto{\pgfqpoint{1.184180in}{3.739634in}}%
\pgfpathlineto{\pgfqpoint{0.906402in}{3.739634in}}%
\pgfpathclose%
\pgfusepath{fill}%
\end{pgfscope}%
\begin{pgfscope}%
\definecolor{textcolor}{rgb}{0.000000,0.000000,0.000000}%
\pgfsetstrokecolor{textcolor}%
\pgfsetfillcolor{textcolor}%
\pgftext[x=1.295291in,y=3.642412in,left,base]{\color{textcolor}\rmfamily\fontsize{10.000000}{12.000000}\selectfont MP insert at source}%
\end{pgfscope}%
\end{pgfpicture}%
\makeatother%
\endgroup%

    \end{center}
    \caption{Migration timeline of a map (4KB pages)}
    \label{fig:map}
\end{figure}

\begin{figure}[tp]
    \begin{center}
        %% Creator: Matplotlib, PGF backend
%%
%% To include the figure in your LaTeX document, write
%%   \input{<filename>.pgf}
%%
%% Make sure the required packages are loaded in your preamble
%%   \usepackage{pgf}
%%
%% and, on pdftex
%%   \usepackage[utf8]{inputenc}\DeclareUnicodeCharacter{2212}{-}
%%
%% or, on luatex and xetex
%%   \usepackage{unicode-math}
%%
%% Figures using additional raster images can only be included by \input if
%% they are in the same directory as the main LaTeX file. For loading figures
%% from other directories you can use the `import` package
%%   \usepackage{import}
%%
%% and then include the figures with
%%   \import{<path to file>}{<filename>.pgf}
%%
%% Matplotlib used the following preamble
%%
\begingroup%
\makeatletter%
\begin{pgfpicture}%
\pgfpathrectangle{\pgfpointorigin}{\pgfqpoint{6.251220in}{7.032623in}}%
\pgfusepath{use as bounding box, clip}%
\begin{pgfscope}%
\pgfsetbuttcap%
\pgfsetmiterjoin%
\definecolor{currentfill}{rgb}{1.000000,1.000000,1.000000}%
\pgfsetfillcolor{currentfill}%
\pgfsetlinewidth{0.000000pt}%
\definecolor{currentstroke}{rgb}{1.000000,1.000000,1.000000}%
\pgfsetstrokecolor{currentstroke}%
\pgfsetdash{}{0pt}%
\pgfpathmoveto{\pgfqpoint{0.000000in}{0.000000in}}%
\pgfpathlineto{\pgfqpoint{6.251220in}{0.000000in}}%
\pgfpathlineto{\pgfqpoint{6.251220in}{7.032623in}}%
\pgfpathlineto{\pgfqpoint{0.000000in}{7.032623in}}%
\pgfpathclose%
\pgfusepath{fill}%
\end{pgfscope}%
\begin{pgfscope}%
\pgfsetbuttcap%
\pgfsetmiterjoin%
\definecolor{currentfill}{rgb}{1.000000,1.000000,1.000000}%
\pgfsetfillcolor{currentfill}%
\pgfsetlinewidth{0.000000pt}%
\definecolor{currentstroke}{rgb}{0.000000,0.000000,0.000000}%
\pgfsetstrokecolor{currentstroke}%
\pgfsetstrokeopacity{0.000000}%
\pgfsetdash{}{0pt}%
\pgfpathmoveto{\pgfqpoint{0.781402in}{0.773588in}}%
\pgfpathlineto{\pgfqpoint{2.891753in}{0.773588in}}%
\pgfpathlineto{\pgfqpoint{2.891753in}{6.188708in}}%
\pgfpathlineto{\pgfqpoint{0.781402in}{6.188708in}}%
\pgfpathclose%
\pgfusepath{fill}%
\end{pgfscope}%
\begin{pgfscope}%
\pgfpathrectangle{\pgfqpoint{0.781402in}{0.773588in}}{\pgfqpoint{2.110351in}{5.415119in}}%
\pgfusepath{clip}%
\pgfsetbuttcap%
\pgfsetroundjoin%
\definecolor{currentfill}{rgb}{0.121569,0.466667,0.705882}%
\pgfsetfillcolor{currentfill}%
\pgfsetlinewidth{0.000000pt}%
\definecolor{currentstroke}{rgb}{0.000000,0.000000,0.000000}%
\pgfsetstrokecolor{currentstroke}%
\pgfsetdash{}{0pt}%
\pgfpathmoveto{\pgfqpoint{0.807094in}{0.773588in}}%
\pgfpathlineto{\pgfqpoint{0.807094in}{0.773588in}}%
\pgfpathlineto{\pgfqpoint{0.875335in}{0.773588in}}%
\pgfpathlineto{\pgfqpoint{0.942110in}{0.773588in}}%
\pgfpathlineto{\pgfqpoint{1.012853in}{0.773588in}}%
\pgfpathlineto{\pgfqpoint{1.079942in}{0.773588in}}%
\pgfpathlineto{\pgfqpoint{1.147369in}{0.773588in}}%
\pgfpathlineto{\pgfqpoint{1.216322in}{0.773588in}}%
\pgfpathlineto{\pgfqpoint{1.283036in}{0.773588in}}%
\pgfpathlineto{\pgfqpoint{1.349373in}{0.773588in}}%
\pgfpathlineto{\pgfqpoint{1.421095in}{0.773588in}}%
\pgfpathlineto{\pgfqpoint{1.489306in}{0.773588in}}%
\pgfpathlineto{\pgfqpoint{1.557461in}{0.773588in}}%
\pgfpathlineto{\pgfqpoint{1.628862in}{0.773588in}}%
\pgfpathlineto{\pgfqpoint{1.698662in}{0.773588in}}%
\pgfpathlineto{\pgfqpoint{1.766402in}{0.773588in}}%
\pgfpathlineto{\pgfqpoint{1.835890in}{0.773588in}}%
\pgfpathlineto{\pgfqpoint{1.904458in}{0.773588in}}%
\pgfpathlineto{\pgfqpoint{1.970951in}{0.773588in}}%
\pgfpathlineto{\pgfqpoint{2.041179in}{0.773588in}}%
\pgfpathlineto{\pgfqpoint{2.108883in}{0.773588in}}%
\pgfpathlineto{\pgfqpoint{2.177337in}{0.773588in}}%
\pgfpathlineto{\pgfqpoint{2.248172in}{0.773588in}}%
\pgfpathlineto{\pgfqpoint{2.317152in}{0.773588in}}%
\pgfpathlineto{\pgfqpoint{2.387948in}{0.773588in}}%
\pgfpathlineto{\pgfqpoint{2.462769in}{0.773588in}}%
\pgfpathlineto{\pgfqpoint{2.534117in}{0.773588in}}%
\pgfpathlineto{\pgfqpoint{2.606023in}{0.773588in}}%
\pgfpathlineto{\pgfqpoint{2.681331in}{0.773588in}}%
\pgfpathlineto{\pgfqpoint{2.753491in}{0.773588in}}%
\pgfpathlineto{\pgfqpoint{2.826501in}{0.773588in}}%
\pgfpathlineto{\pgfqpoint{2.902674in}{0.773588in}}%
\pgfpathlineto{\pgfqpoint{2.975015in}{0.773588in}}%
\pgfpathlineto{\pgfqpoint{3.047640in}{0.773588in}}%
\pgfpathlineto{\pgfqpoint{3.123114in}{0.773588in}}%
\pgfpathlineto{\pgfqpoint{3.195433in}{0.773588in}}%
\pgfpathlineto{\pgfqpoint{3.277599in}{0.773588in}}%
\pgfpathlineto{\pgfqpoint{3.356465in}{0.773588in}}%
\pgfpathlineto{\pgfqpoint{3.430573in}{0.773588in}}%
\pgfpathlineto{\pgfqpoint{3.503039in}{0.773588in}}%
\pgfpathlineto{\pgfqpoint{3.575543in}{0.773588in}}%
\pgfpathlineto{\pgfqpoint{3.643958in}{0.773588in}}%
\pgfpathlineto{\pgfqpoint{3.712535in}{0.773588in}}%
\pgfpathlineto{\pgfqpoint{3.781205in}{0.773588in}}%
\pgfpathlineto{\pgfqpoint{3.848285in}{0.773588in}}%
\pgfpathlineto{\pgfqpoint{3.914312in}{0.773588in}}%
\pgfpathlineto{\pgfqpoint{3.981945in}{0.773588in}}%
\pgfpathlineto{\pgfqpoint{4.048987in}{0.773588in}}%
\pgfpathlineto{\pgfqpoint{4.116566in}{0.773588in}}%
\pgfpathlineto{\pgfqpoint{4.186150in}{0.773588in}}%
\pgfpathlineto{\pgfqpoint{4.253047in}{0.773588in}}%
\pgfpathlineto{\pgfqpoint{4.319299in}{0.773588in}}%
\pgfpathlineto{\pgfqpoint{4.387175in}{0.773588in}}%
\pgfpathlineto{\pgfqpoint{4.454265in}{0.773588in}}%
\pgfpathlineto{\pgfqpoint{4.520559in}{0.773588in}}%
\pgfpathlineto{\pgfqpoint{4.590503in}{0.773588in}}%
\pgfpathlineto{\pgfqpoint{4.658575in}{0.773588in}}%
\pgfpathlineto{\pgfqpoint{4.726503in}{0.773588in}}%
\pgfpathlineto{\pgfqpoint{4.796564in}{0.773588in}}%
\pgfpathlineto{\pgfqpoint{4.866311in}{0.773588in}}%
\pgfpathlineto{\pgfqpoint{4.935251in}{0.773588in}}%
\pgfpathlineto{\pgfqpoint{5.005143in}{0.773588in}}%
\pgfpathlineto{\pgfqpoint{5.073090in}{0.773588in}}%
\pgfpathlineto{\pgfqpoint{5.142206in}{0.773588in}}%
\pgfpathlineto{\pgfqpoint{5.213575in}{0.773588in}}%
\pgfpathlineto{\pgfqpoint{5.283251in}{0.773588in}}%
\pgfpathlineto{\pgfqpoint{5.353942in}{0.773588in}}%
\pgfpathlineto{\pgfqpoint{5.424203in}{0.773588in}}%
\pgfpathlineto{\pgfqpoint{5.492693in}{0.773588in}}%
\pgfpathlineto{\pgfqpoint{5.561464in}{0.773588in}}%
\pgfpathlineto{\pgfqpoint{5.633136in}{0.773588in}}%
\pgfpathlineto{\pgfqpoint{5.700467in}{0.773588in}}%
\pgfpathlineto{\pgfqpoint{5.766709in}{0.773588in}}%
\pgfpathlineto{\pgfqpoint{5.834669in}{0.773588in}}%
\pgfpathlineto{\pgfqpoint{5.901623in}{0.773588in}}%
\pgfpathlineto{\pgfqpoint{5.968494in}{0.773588in}}%
\pgfpathlineto{\pgfqpoint{6.036866in}{0.773588in}}%
\pgfpathlineto{\pgfqpoint{6.103260in}{0.773588in}}%
\pgfpathlineto{\pgfqpoint{6.169451in}{0.773588in}}%
\pgfpathlineto{\pgfqpoint{6.236861in}{0.773588in}}%
\pgfpathlineto{\pgfqpoint{6.303479in}{0.773588in}}%
\pgfpathlineto{\pgfqpoint{6.369842in}{0.773588in}}%
\pgfpathlineto{\pgfqpoint{6.437563in}{0.773588in}}%
\pgfpathlineto{\pgfqpoint{6.502885in}{0.773588in}}%
\pgfpathlineto{\pgfqpoint{6.568004in}{0.773588in}}%
\pgfpathlineto{\pgfqpoint{6.636218in}{0.773588in}}%
\pgfpathlineto{\pgfqpoint{6.704271in}{0.773588in}}%
\pgfpathlineto{\pgfqpoint{6.770581in}{0.773588in}}%
\pgfpathlineto{\pgfqpoint{6.839348in}{0.773588in}}%
\pgfpathlineto{\pgfqpoint{6.906544in}{0.773588in}}%
\pgfpathlineto{\pgfqpoint{6.973333in}{0.773588in}}%
\pgfpathlineto{\pgfqpoint{7.040805in}{0.773588in}}%
\pgfpathlineto{\pgfqpoint{7.106301in}{0.773588in}}%
\pgfpathlineto{\pgfqpoint{7.171034in}{0.773588in}}%
\pgfpathlineto{\pgfqpoint{7.238831in}{0.773588in}}%
\pgfpathlineto{\pgfqpoint{7.308295in}{0.773588in}}%
\pgfpathlineto{\pgfqpoint{7.379810in}{0.773588in}}%
\pgfpathlineto{\pgfqpoint{7.455306in}{0.773588in}}%
\pgfpathlineto{\pgfqpoint{7.527107in}{0.773588in}}%
\pgfpathlineto{\pgfqpoint{7.598690in}{0.773588in}}%
\pgfpathlineto{\pgfqpoint{7.671131in}{0.773588in}}%
\pgfpathlineto{\pgfqpoint{7.741421in}{0.773588in}}%
\pgfpathlineto{\pgfqpoint{7.812286in}{0.773588in}}%
\pgfpathlineto{\pgfqpoint{7.887441in}{0.773588in}}%
\pgfpathlineto{\pgfqpoint{7.957986in}{0.773588in}}%
\pgfpathlineto{\pgfqpoint{8.028258in}{0.773588in}}%
\pgfpathlineto{\pgfqpoint{8.100457in}{0.773588in}}%
\pgfpathlineto{\pgfqpoint{8.168934in}{0.773588in}}%
\pgfpathlineto{\pgfqpoint{8.237831in}{0.773588in}}%
\pgfpathlineto{\pgfqpoint{8.308718in}{0.773588in}}%
\pgfpathlineto{\pgfqpoint{8.376454in}{0.773588in}}%
\pgfpathlineto{\pgfqpoint{8.444285in}{0.773588in}}%
\pgfpathlineto{\pgfqpoint{8.515085in}{0.773588in}}%
\pgfpathlineto{\pgfqpoint{8.583601in}{0.773588in}}%
\pgfpathlineto{\pgfqpoint{8.651757in}{0.773588in}}%
\pgfpathlineto{\pgfqpoint{8.720615in}{0.773588in}}%
\pgfpathlineto{\pgfqpoint{8.787638in}{0.773588in}}%
\pgfpathlineto{\pgfqpoint{8.856098in}{0.773588in}}%
\pgfpathlineto{\pgfqpoint{8.924860in}{0.773588in}}%
\pgfpathlineto{\pgfqpoint{8.991646in}{0.773588in}}%
\pgfpathlineto{\pgfqpoint{9.059607in}{0.773588in}}%
\pgfpathlineto{\pgfqpoint{9.130518in}{0.773588in}}%
\pgfpathlineto{\pgfqpoint{9.198477in}{0.773588in}}%
\pgfpathlineto{\pgfqpoint{9.265262in}{0.773588in}}%
\pgfpathlineto{\pgfqpoint{9.334728in}{0.773588in}}%
\pgfpathlineto{\pgfqpoint{9.402138in}{0.773588in}}%
\pgfpathlineto{\pgfqpoint{9.467314in}{0.773588in}}%
\pgfpathlineto{\pgfqpoint{9.536145in}{0.773588in}}%
\pgfpathlineto{\pgfqpoint{9.603748in}{0.773588in}}%
\pgfpathlineto{\pgfqpoint{9.670768in}{0.773588in}}%
\pgfpathlineto{\pgfqpoint{9.740273in}{0.773588in}}%
\pgfpathlineto{\pgfqpoint{9.808815in}{0.773588in}}%
\pgfpathlineto{\pgfqpoint{9.876148in}{0.773588in}}%
\pgfpathlineto{\pgfqpoint{9.944615in}{0.773588in}}%
\pgfpathlineto{\pgfqpoint{10.012687in}{0.773588in}}%
\pgfpathlineto{\pgfqpoint{10.081602in}{0.773588in}}%
\pgfpathlineto{\pgfqpoint{10.153891in}{0.773588in}}%
\pgfpathlineto{\pgfqpoint{10.222102in}{0.773588in}}%
\pgfpathlineto{\pgfqpoint{10.290696in}{0.773588in}}%
\pgfpathlineto{\pgfqpoint{10.360853in}{0.773588in}}%
\pgfpathlineto{\pgfqpoint{10.428999in}{0.773588in}}%
\pgfpathlineto{\pgfqpoint{10.498528in}{0.773588in}}%
\pgfpathlineto{\pgfqpoint{10.569898in}{0.773588in}}%
\pgfpathlineto{\pgfqpoint{10.639006in}{0.773588in}}%
\pgfpathlineto{\pgfqpoint{10.708017in}{0.773588in}}%
\pgfpathlineto{\pgfqpoint{10.777925in}{0.773588in}}%
\pgfpathlineto{\pgfqpoint{10.846851in}{0.773588in}}%
\pgfpathlineto{\pgfqpoint{10.916008in}{0.773588in}}%
\pgfpathlineto{\pgfqpoint{10.986941in}{0.773588in}}%
\pgfpathlineto{\pgfqpoint{11.055389in}{0.773588in}}%
\pgfpathlineto{\pgfqpoint{11.125133in}{0.773588in}}%
\pgfpathlineto{\pgfqpoint{11.194984in}{0.773588in}}%
\pgfpathlineto{\pgfqpoint{11.261777in}{0.773588in}}%
\pgfpathlineto{\pgfqpoint{11.328677in}{0.773588in}}%
\pgfpathlineto{\pgfqpoint{11.398136in}{0.773588in}}%
\pgfpathlineto{\pgfqpoint{11.465294in}{0.773588in}}%
\pgfpathlineto{\pgfqpoint{11.532877in}{0.773588in}}%
\pgfpathlineto{\pgfqpoint{11.602395in}{0.773588in}}%
\pgfpathlineto{\pgfqpoint{11.669969in}{0.773588in}}%
\pgfpathlineto{\pgfqpoint{11.736702in}{0.773588in}}%
\pgfpathlineto{\pgfqpoint{11.804695in}{0.773588in}}%
\pgfpathlineto{\pgfqpoint{11.871239in}{0.773588in}}%
\pgfpathlineto{\pgfqpoint{11.938397in}{0.773588in}}%
\pgfpathlineto{\pgfqpoint{12.008843in}{0.773588in}}%
\pgfpathlineto{\pgfqpoint{12.076638in}{0.773588in}}%
\pgfpathlineto{\pgfqpoint{12.144815in}{0.773588in}}%
\pgfpathlineto{\pgfqpoint{12.213872in}{0.773588in}}%
\pgfpathlineto{\pgfqpoint{12.280761in}{0.773588in}}%
\pgfpathlineto{\pgfqpoint{12.347008in}{0.773588in}}%
\pgfpathlineto{\pgfqpoint{12.416448in}{0.773588in}}%
\pgfpathlineto{\pgfqpoint{12.484090in}{0.773588in}}%
\pgfpathlineto{\pgfqpoint{12.552597in}{0.773588in}}%
\pgfpathlineto{\pgfqpoint{12.623527in}{0.773588in}}%
\pgfpathlineto{\pgfqpoint{12.692271in}{0.773588in}}%
\pgfpathlineto{\pgfqpoint{12.759967in}{0.773588in}}%
\pgfpathlineto{\pgfqpoint{12.830499in}{0.773588in}}%
\pgfpathlineto{\pgfqpoint{12.900232in}{0.773588in}}%
\pgfpathlineto{\pgfqpoint{12.969036in}{0.773588in}}%
\pgfpathlineto{\pgfqpoint{13.041078in}{0.773588in}}%
\pgfpathlineto{\pgfqpoint{13.109710in}{0.773588in}}%
\pgfpathlineto{\pgfqpoint{13.178279in}{0.773588in}}%
\pgfpathlineto{\pgfqpoint{13.250526in}{0.773588in}}%
\pgfpathlineto{\pgfqpoint{13.319414in}{0.773588in}}%
\pgfpathlineto{\pgfqpoint{13.387091in}{0.773588in}}%
\pgfpathlineto{\pgfqpoint{13.457833in}{0.773588in}}%
\pgfpathlineto{\pgfqpoint{13.526360in}{0.773588in}}%
\pgfpathlineto{\pgfqpoint{13.595052in}{0.773588in}}%
\pgfpathlineto{\pgfqpoint{13.666880in}{0.773588in}}%
\pgfpathlineto{\pgfqpoint{13.735499in}{0.773588in}}%
\pgfpathlineto{\pgfqpoint{13.804228in}{0.773588in}}%
\pgfpathlineto{\pgfqpoint{13.876785in}{0.773588in}}%
\pgfpathlineto{\pgfqpoint{13.944783in}{0.773588in}}%
\pgfpathlineto{\pgfqpoint{14.012639in}{0.773588in}}%
\pgfpathlineto{\pgfqpoint{14.081564in}{0.773588in}}%
\pgfpathlineto{\pgfqpoint{14.149562in}{0.773588in}}%
\pgfpathlineto{\pgfqpoint{14.216102in}{0.773588in}}%
\pgfpathlineto{\pgfqpoint{14.286685in}{0.773588in}}%
\pgfpathlineto{\pgfqpoint{14.355979in}{0.773588in}}%
\pgfpathlineto{\pgfqpoint{14.424138in}{0.773588in}}%
\pgfpathlineto{\pgfqpoint{14.494110in}{0.773588in}}%
\pgfpathlineto{\pgfqpoint{14.562284in}{0.773588in}}%
\pgfpathlineto{\pgfqpoint{14.630523in}{0.773588in}}%
\pgfpathlineto{\pgfqpoint{14.700003in}{0.773588in}}%
\pgfpathlineto{\pgfqpoint{14.768225in}{0.773588in}}%
\pgfpathlineto{\pgfqpoint{14.836607in}{0.773588in}}%
\pgfpathlineto{\pgfqpoint{14.908199in}{0.773588in}}%
\pgfpathlineto{\pgfqpoint{14.976372in}{0.773588in}}%
\pgfpathlineto{\pgfqpoint{15.044439in}{0.773588in}}%
\pgfpathlineto{\pgfqpoint{15.115140in}{0.773588in}}%
\pgfpathlineto{\pgfqpoint{15.182512in}{0.773588in}}%
\pgfpathlineto{\pgfqpoint{15.249597in}{0.773588in}}%
\pgfpathlineto{\pgfqpoint{15.320168in}{0.773588in}}%
\pgfpathlineto{\pgfqpoint{15.388582in}{0.773588in}}%
\pgfpathlineto{\pgfqpoint{15.457342in}{0.773588in}}%
\pgfpathlineto{\pgfqpoint{15.527961in}{0.773588in}}%
\pgfpathlineto{\pgfqpoint{15.595878in}{0.773588in}}%
\pgfpathlineto{\pgfqpoint{15.665480in}{0.773588in}}%
\pgfpathlineto{\pgfqpoint{15.737902in}{0.773588in}}%
\pgfpathlineto{\pgfqpoint{15.807467in}{0.773588in}}%
\pgfpathlineto{\pgfqpoint{15.878678in}{0.773588in}}%
\pgfpathlineto{\pgfqpoint{15.949677in}{0.773588in}}%
\pgfpathlineto{\pgfqpoint{16.019598in}{0.773588in}}%
\pgfpathlineto{\pgfqpoint{16.088678in}{0.773588in}}%
\pgfpathlineto{\pgfqpoint{16.160188in}{0.773588in}}%
\pgfpathlineto{\pgfqpoint{16.229236in}{0.773588in}}%
\pgfpathlineto{\pgfqpoint{16.299236in}{0.773588in}}%
\pgfpathlineto{\pgfqpoint{16.373054in}{0.773588in}}%
\pgfpathlineto{\pgfqpoint{16.445981in}{0.773588in}}%
\pgfpathlineto{\pgfqpoint{16.519393in}{0.773588in}}%
\pgfpathlineto{\pgfqpoint{16.592371in}{0.773588in}}%
\pgfpathlineto{\pgfqpoint{16.662798in}{0.773588in}}%
\pgfpathlineto{\pgfqpoint{16.731829in}{0.773588in}}%
\pgfpathlineto{\pgfqpoint{16.803402in}{0.773588in}}%
\pgfpathlineto{\pgfqpoint{16.872974in}{0.773588in}}%
\pgfpathlineto{\pgfqpoint{16.942299in}{0.773588in}}%
\pgfpathlineto{\pgfqpoint{17.013249in}{0.773588in}}%
\pgfpathlineto{\pgfqpoint{17.081113in}{0.773588in}}%
\pgfpathlineto{\pgfqpoint{17.150280in}{0.773588in}}%
\pgfpathlineto{\pgfqpoint{17.220960in}{0.773588in}}%
\pgfpathlineto{\pgfqpoint{17.288800in}{0.773588in}}%
\pgfpathlineto{\pgfqpoint{17.356523in}{0.773588in}}%
\pgfpathlineto{\pgfqpoint{17.426570in}{0.773588in}}%
\pgfpathlineto{\pgfqpoint{17.494859in}{0.773588in}}%
\pgfpathlineto{\pgfqpoint{17.563254in}{0.773588in}}%
\pgfpathlineto{\pgfqpoint{17.634782in}{0.773588in}}%
\pgfpathlineto{\pgfqpoint{17.703480in}{0.773588in}}%
\pgfpathlineto{\pgfqpoint{17.772898in}{0.773588in}}%
\pgfpathlineto{\pgfqpoint{17.844148in}{0.773588in}}%
\pgfpathlineto{\pgfqpoint{17.912627in}{0.773588in}}%
\pgfpathlineto{\pgfqpoint{17.983524in}{0.773588in}}%
\pgfpathlineto{\pgfqpoint{18.056086in}{0.773588in}}%
\pgfpathlineto{\pgfqpoint{18.125960in}{0.773588in}}%
\pgfpathlineto{\pgfqpoint{18.195268in}{0.773588in}}%
\pgfpathlineto{\pgfqpoint{18.266780in}{0.773588in}}%
\pgfpathlineto{\pgfqpoint{18.336223in}{0.773588in}}%
\pgfpathlineto{\pgfqpoint{18.406899in}{0.773588in}}%
\pgfpathlineto{\pgfqpoint{18.479143in}{0.773588in}}%
\pgfpathlineto{\pgfqpoint{18.549509in}{0.773588in}}%
\pgfpathlineto{\pgfqpoint{18.619849in}{0.773588in}}%
\pgfpathlineto{\pgfqpoint{18.692839in}{0.773588in}}%
\pgfpathlineto{\pgfqpoint{18.763558in}{0.773588in}}%
\pgfpathlineto{\pgfqpoint{18.834141in}{0.773588in}}%
\pgfpathlineto{\pgfqpoint{18.906958in}{0.773588in}}%
\pgfpathlineto{\pgfqpoint{18.977585in}{0.773588in}}%
\pgfpathlineto{\pgfqpoint{19.048835in}{0.773588in}}%
\pgfpathlineto{\pgfqpoint{19.123240in}{0.773588in}}%
\pgfpathlineto{\pgfqpoint{19.195134in}{0.773588in}}%
\pgfpathlineto{\pgfqpoint{19.266961in}{0.773588in}}%
\pgfpathlineto{\pgfqpoint{19.340700in}{0.773588in}}%
\pgfpathlineto{\pgfqpoint{19.412352in}{0.773588in}}%
\pgfpathlineto{\pgfqpoint{19.482791in}{0.773588in}}%
\pgfpathlineto{\pgfqpoint{19.553641in}{0.773588in}}%
\pgfpathlineto{\pgfqpoint{19.623855in}{0.773588in}}%
\pgfpathlineto{\pgfqpoint{19.693765in}{0.773588in}}%
\pgfpathlineto{\pgfqpoint{19.766060in}{0.773588in}}%
\pgfpathlineto{\pgfqpoint{19.835928in}{0.773588in}}%
\pgfpathlineto{\pgfqpoint{19.907049in}{0.773588in}}%
\pgfpathlineto{\pgfqpoint{19.981199in}{0.773588in}}%
\pgfpathlineto{\pgfqpoint{20.052071in}{0.773588in}}%
\pgfpathlineto{\pgfqpoint{20.121885in}{0.773588in}}%
\pgfpathlineto{\pgfqpoint{20.193561in}{0.773588in}}%
\pgfpathlineto{\pgfqpoint{20.263963in}{0.773588in}}%
\pgfpathlineto{\pgfqpoint{20.334605in}{0.773588in}}%
\pgfpathlineto{\pgfqpoint{20.407276in}{0.773588in}}%
\pgfpathlineto{\pgfqpoint{20.476961in}{0.773588in}}%
\pgfpathlineto{\pgfqpoint{20.547172in}{0.773588in}}%
\pgfpathlineto{\pgfqpoint{20.618428in}{0.773588in}}%
\pgfpathlineto{\pgfqpoint{20.688980in}{0.773588in}}%
\pgfpathlineto{\pgfqpoint{20.758814in}{0.773588in}}%
\pgfpathlineto{\pgfqpoint{20.830586in}{0.773588in}}%
\pgfpathlineto{\pgfqpoint{20.899587in}{0.773588in}}%
\pgfpathlineto{\pgfqpoint{20.969373in}{0.773588in}}%
\pgfpathlineto{\pgfqpoint{21.040864in}{0.773588in}}%
\pgfpathlineto{\pgfqpoint{21.110656in}{0.773588in}}%
\pgfpathlineto{\pgfqpoint{21.181233in}{0.773588in}}%
\pgfpathlineto{\pgfqpoint{21.254645in}{0.773588in}}%
\pgfpathlineto{\pgfqpoint{21.324498in}{0.773588in}}%
\pgfpathlineto{\pgfqpoint{21.394839in}{0.773588in}}%
\pgfpathlineto{\pgfqpoint{21.467741in}{0.773588in}}%
\pgfpathlineto{\pgfqpoint{21.539601in}{0.773588in}}%
\pgfpathlineto{\pgfqpoint{21.610878in}{0.773588in}}%
\pgfpathlineto{\pgfqpoint{21.683994in}{0.773588in}}%
\pgfpathlineto{\pgfqpoint{21.756227in}{0.773588in}}%
\pgfpathlineto{\pgfqpoint{21.828555in}{0.773588in}}%
\pgfpathlineto{\pgfqpoint{21.903868in}{0.773588in}}%
\pgfpathlineto{\pgfqpoint{21.976836in}{0.773588in}}%
\pgfpathlineto{\pgfqpoint{22.048040in}{0.773588in}}%
\pgfpathlineto{\pgfqpoint{22.122462in}{0.773588in}}%
\pgfpathlineto{\pgfqpoint{22.195707in}{0.773588in}}%
\pgfpathlineto{\pgfqpoint{22.268824in}{0.773588in}}%
\pgfpathlineto{\pgfqpoint{22.343331in}{0.773588in}}%
\pgfpathlineto{\pgfqpoint{22.413449in}{0.773588in}}%
\pgfpathlineto{\pgfqpoint{22.482516in}{0.773588in}}%
\pgfpathlineto{\pgfqpoint{22.553535in}{0.773588in}}%
\pgfpathlineto{\pgfqpoint{22.624114in}{0.773588in}}%
\pgfpathlineto{\pgfqpoint{22.694016in}{0.773588in}}%
\pgfpathlineto{\pgfqpoint{22.764651in}{0.773588in}}%
\pgfpathlineto{\pgfqpoint{22.833628in}{0.773588in}}%
\pgfpathlineto{\pgfqpoint{22.902896in}{0.773588in}}%
\pgfpathlineto{\pgfqpoint{22.973868in}{0.773588in}}%
\pgfpathlineto{\pgfqpoint{23.043397in}{0.773588in}}%
\pgfpathlineto{\pgfqpoint{23.113267in}{0.773588in}}%
\pgfpathlineto{\pgfqpoint{23.184270in}{0.773588in}}%
\pgfpathlineto{\pgfqpoint{23.253484in}{0.773588in}}%
\pgfpathlineto{\pgfqpoint{23.323995in}{0.773588in}}%
\pgfpathlineto{\pgfqpoint{23.396126in}{0.773588in}}%
\pgfpathlineto{\pgfqpoint{23.467323in}{0.773588in}}%
\pgfpathlineto{\pgfqpoint{23.537850in}{0.773588in}}%
\pgfpathlineto{\pgfqpoint{23.610036in}{0.773588in}}%
\pgfpathlineto{\pgfqpoint{23.681448in}{0.773588in}}%
\pgfpathlineto{\pgfqpoint{23.752361in}{0.773588in}}%
\pgfpathlineto{\pgfqpoint{23.824951in}{0.773588in}}%
\pgfpathlineto{\pgfqpoint{23.895213in}{0.773588in}}%
\pgfpathlineto{\pgfqpoint{23.966722in}{0.773588in}}%
\pgfpathlineto{\pgfqpoint{24.039255in}{0.773588in}}%
\pgfpathlineto{\pgfqpoint{24.111784in}{0.773588in}}%
\pgfpathlineto{\pgfqpoint{24.183899in}{0.773588in}}%
\pgfpathlineto{\pgfqpoint{24.257307in}{0.773588in}}%
\pgfpathlineto{\pgfqpoint{24.329090in}{0.773588in}}%
\pgfpathlineto{\pgfqpoint{24.400328in}{0.773588in}}%
\pgfpathlineto{\pgfqpoint{24.476339in}{0.773588in}}%
\pgfpathlineto{\pgfqpoint{24.548639in}{0.773588in}}%
\pgfpathlineto{\pgfqpoint{24.618678in}{0.773588in}}%
\pgfpathlineto{\pgfqpoint{24.691660in}{0.773588in}}%
\pgfpathlineto{\pgfqpoint{24.764742in}{0.773588in}}%
\pgfpathlineto{\pgfqpoint{24.836992in}{0.773588in}}%
\pgfpathlineto{\pgfqpoint{24.911741in}{0.773588in}}%
\pgfpathlineto{\pgfqpoint{24.983525in}{0.773588in}}%
\pgfpathlineto{\pgfqpoint{25.055567in}{0.773588in}}%
\pgfpathlineto{\pgfqpoint{25.131109in}{0.773588in}}%
\pgfpathlineto{\pgfqpoint{25.203216in}{0.773588in}}%
\pgfpathlineto{\pgfqpoint{25.273349in}{0.773588in}}%
\pgfpathlineto{\pgfqpoint{25.347124in}{0.773588in}}%
\pgfpathlineto{\pgfqpoint{25.417047in}{0.773588in}}%
\pgfpathlineto{\pgfqpoint{25.487573in}{0.773588in}}%
\pgfpathlineto{\pgfqpoint{25.560110in}{0.773588in}}%
\pgfpathlineto{\pgfqpoint{25.631022in}{0.773588in}}%
\pgfpathlineto{\pgfqpoint{25.702341in}{0.773588in}}%
\pgfpathlineto{\pgfqpoint{25.775695in}{0.773588in}}%
\pgfpathlineto{\pgfqpoint{25.845667in}{0.773588in}}%
\pgfpathlineto{\pgfqpoint{25.916551in}{0.773588in}}%
\pgfpathlineto{\pgfqpoint{25.988588in}{0.773588in}}%
\pgfpathlineto{\pgfqpoint{26.058621in}{0.773588in}}%
\pgfpathlineto{\pgfqpoint{26.130346in}{0.773588in}}%
\pgfpathlineto{\pgfqpoint{26.203572in}{0.773588in}}%
\pgfpathlineto{\pgfqpoint{26.274267in}{0.773588in}}%
\pgfpathlineto{\pgfqpoint{26.344920in}{0.773588in}}%
\pgfpathlineto{\pgfqpoint{26.417231in}{0.773588in}}%
\pgfpathlineto{\pgfqpoint{26.487420in}{0.773588in}}%
\pgfpathlineto{\pgfqpoint{26.557235in}{0.773588in}}%
\pgfpathlineto{\pgfqpoint{26.629572in}{0.773588in}}%
\pgfpathlineto{\pgfqpoint{26.699584in}{0.773588in}}%
\pgfpathlineto{\pgfqpoint{26.769271in}{0.773588in}}%
\pgfpathlineto{\pgfqpoint{26.841234in}{0.773588in}}%
\pgfpathlineto{\pgfqpoint{26.912667in}{0.773588in}}%
\pgfpathlineto{\pgfqpoint{26.983641in}{0.773588in}}%
\pgfpathlineto{\pgfqpoint{27.056835in}{0.773588in}}%
\pgfpathlineto{\pgfqpoint{27.128948in}{0.773588in}}%
\pgfpathlineto{\pgfqpoint{27.201477in}{0.773588in}}%
\pgfpathlineto{\pgfqpoint{27.277488in}{0.773588in}}%
\pgfpathlineto{\pgfqpoint{27.350990in}{0.773588in}}%
\pgfpathlineto{\pgfqpoint{27.423884in}{0.773588in}}%
\pgfpathlineto{\pgfqpoint{27.500063in}{0.773588in}}%
\pgfpathlineto{\pgfqpoint{27.574929in}{0.773588in}}%
\pgfpathlineto{\pgfqpoint{27.649072in}{0.773588in}}%
\pgfpathlineto{\pgfqpoint{27.724006in}{0.773588in}}%
\pgfpathlineto{\pgfqpoint{27.795343in}{0.773588in}}%
\pgfpathlineto{\pgfqpoint{27.868117in}{0.773588in}}%
\pgfpathlineto{\pgfqpoint{27.943911in}{0.773588in}}%
\pgfpathlineto{\pgfqpoint{28.018234in}{0.773588in}}%
\pgfpathlineto{\pgfqpoint{28.090360in}{0.773588in}}%
\pgfpathlineto{\pgfqpoint{28.163352in}{0.773588in}}%
\pgfpathlineto{\pgfqpoint{28.234559in}{0.773588in}}%
\pgfpathlineto{\pgfqpoint{28.306604in}{0.773588in}}%
\pgfpathlineto{\pgfqpoint{28.380501in}{0.773588in}}%
\pgfpathlineto{\pgfqpoint{28.451943in}{0.773588in}}%
\pgfpathlineto{\pgfqpoint{28.522534in}{0.773588in}}%
\pgfpathlineto{\pgfqpoint{28.596041in}{0.773588in}}%
\pgfpathlineto{\pgfqpoint{28.668204in}{0.773588in}}%
\pgfpathlineto{\pgfqpoint{28.738605in}{0.773588in}}%
\pgfpathlineto{\pgfqpoint{28.811911in}{0.773588in}}%
\pgfpathlineto{\pgfqpoint{28.885217in}{0.773588in}}%
\pgfpathlineto{\pgfqpoint{28.956832in}{0.773588in}}%
\pgfpathlineto{\pgfqpoint{29.029887in}{0.773588in}}%
\pgfpathlineto{\pgfqpoint{29.100748in}{0.773588in}}%
\pgfpathlineto{\pgfqpoint{29.173201in}{0.773588in}}%
\pgfpathlineto{\pgfqpoint{29.248973in}{0.773588in}}%
\pgfpathlineto{\pgfqpoint{29.320759in}{0.773588in}}%
\pgfpathlineto{\pgfqpoint{29.393660in}{0.773588in}}%
\pgfpathlineto{\pgfqpoint{29.467898in}{0.773588in}}%
\pgfpathlineto{\pgfqpoint{29.540420in}{0.773588in}}%
\pgfpathlineto{\pgfqpoint{29.611700in}{0.773588in}}%
\pgfpathlineto{\pgfqpoint{29.684427in}{0.773588in}}%
\pgfpathlineto{\pgfqpoint{29.755113in}{0.773588in}}%
\pgfpathlineto{\pgfqpoint{29.827132in}{0.773588in}}%
\pgfpathlineto{\pgfqpoint{29.901656in}{0.773588in}}%
\pgfpathlineto{\pgfqpoint{29.974646in}{0.773588in}}%
\pgfpathlineto{\pgfqpoint{30.048252in}{0.773588in}}%
\pgfpathlineto{\pgfqpoint{30.122796in}{0.773588in}}%
\pgfpathlineto{\pgfqpoint{30.195443in}{0.773588in}}%
\pgfpathlineto{\pgfqpoint{30.269036in}{0.773588in}}%
\pgfpathlineto{\pgfqpoint{30.344328in}{0.773588in}}%
\pgfpathlineto{\pgfqpoint{30.417098in}{0.773588in}}%
\pgfpathlineto{\pgfqpoint{30.488991in}{0.773588in}}%
\pgfpathlineto{\pgfqpoint{30.562714in}{0.773588in}}%
\pgfpathlineto{\pgfqpoint{30.634099in}{0.773588in}}%
\pgfpathlineto{\pgfqpoint{30.707828in}{0.773588in}}%
\pgfpathlineto{\pgfqpoint{30.782249in}{0.773588in}}%
\pgfpathlineto{\pgfqpoint{30.854115in}{0.773588in}}%
\pgfpathlineto{\pgfqpoint{30.928305in}{0.773588in}}%
\pgfpathlineto{\pgfqpoint{31.002514in}{0.773588in}}%
\pgfpathlineto{\pgfqpoint{31.074452in}{0.773588in}}%
\pgfpathlineto{\pgfqpoint{31.147740in}{0.773588in}}%
\pgfpathlineto{\pgfqpoint{31.222913in}{0.773588in}}%
\pgfpathlineto{\pgfqpoint{31.294777in}{0.773588in}}%
\pgfpathlineto{\pgfqpoint{31.366613in}{0.773588in}}%
\pgfpathlineto{\pgfqpoint{31.439415in}{0.773588in}}%
\pgfpathlineto{\pgfqpoint{31.510140in}{0.773588in}}%
\pgfpathlineto{\pgfqpoint{31.582282in}{0.773588in}}%
\pgfpathlineto{\pgfqpoint{31.656180in}{0.773588in}}%
\pgfpathlineto{\pgfqpoint{31.728521in}{0.773588in}}%
\pgfpathlineto{\pgfqpoint{31.800877in}{0.773588in}}%
\pgfpathlineto{\pgfqpoint{31.873539in}{0.773588in}}%
\pgfpathlineto{\pgfqpoint{31.943734in}{0.773588in}}%
\pgfpathlineto{\pgfqpoint{32.015122in}{0.773588in}}%
\pgfpathlineto{\pgfqpoint{32.089684in}{0.773588in}}%
\pgfpathlineto{\pgfqpoint{32.161504in}{0.773588in}}%
\pgfpathlineto{\pgfqpoint{32.231773in}{0.773588in}}%
\pgfpathlineto{\pgfqpoint{32.305440in}{0.773588in}}%
\pgfpathlineto{\pgfqpoint{32.377016in}{0.773588in}}%
\pgfpathlineto{\pgfqpoint{32.447439in}{0.773588in}}%
\pgfpathlineto{\pgfqpoint{32.520401in}{0.773588in}}%
\pgfpathlineto{\pgfqpoint{32.590674in}{0.773588in}}%
\pgfpathlineto{\pgfqpoint{32.663709in}{0.773588in}}%
\pgfpathlineto{\pgfqpoint{32.740263in}{0.773588in}}%
\pgfpathlineto{\pgfqpoint{32.813546in}{0.773588in}}%
\pgfpathlineto{\pgfqpoint{32.887492in}{0.773588in}}%
\pgfpathlineto{\pgfqpoint{32.963168in}{0.773588in}}%
\pgfpathlineto{\pgfqpoint{33.037794in}{0.773588in}}%
\pgfpathlineto{\pgfqpoint{33.110479in}{0.773588in}}%
\pgfpathlineto{\pgfqpoint{33.185787in}{0.773588in}}%
\pgfpathlineto{\pgfqpoint{33.259507in}{0.773588in}}%
\pgfpathlineto{\pgfqpoint{33.333311in}{0.773588in}}%
\pgfpathlineto{\pgfqpoint{33.409286in}{0.773588in}}%
\pgfpathlineto{\pgfqpoint{33.483328in}{0.773588in}}%
\pgfpathlineto{\pgfqpoint{33.557012in}{0.773588in}}%
\pgfpathlineto{\pgfqpoint{33.631884in}{0.773588in}}%
\pgfpathlineto{\pgfqpoint{33.703848in}{0.773588in}}%
\pgfpathlineto{\pgfqpoint{33.776888in}{0.773588in}}%
\pgfpathlineto{\pgfqpoint{33.852393in}{0.773588in}}%
\pgfpathlineto{\pgfqpoint{33.923536in}{0.773588in}}%
\pgfpathlineto{\pgfqpoint{33.994648in}{0.773588in}}%
\pgfpathlineto{\pgfqpoint{34.067999in}{0.773588in}}%
\pgfpathlineto{\pgfqpoint{34.138346in}{0.773588in}}%
\pgfpathlineto{\pgfqpoint{34.210760in}{0.773588in}}%
\pgfpathlineto{\pgfqpoint{34.284339in}{0.773588in}}%
\pgfpathlineto{\pgfqpoint{34.354648in}{0.773588in}}%
\pgfpathlineto{\pgfqpoint{34.425604in}{0.773588in}}%
\pgfpathlineto{\pgfqpoint{34.499162in}{0.773588in}}%
\pgfpathlineto{\pgfqpoint{34.571449in}{0.773588in}}%
\pgfpathlineto{\pgfqpoint{34.643977in}{0.773588in}}%
\pgfpathlineto{\pgfqpoint{34.718731in}{0.773588in}}%
\pgfpathlineto{\pgfqpoint{34.789698in}{0.773588in}}%
\pgfpathlineto{\pgfqpoint{34.862212in}{0.773588in}}%
\pgfpathlineto{\pgfqpoint{34.936943in}{0.773588in}}%
\pgfpathlineto{\pgfqpoint{35.007838in}{0.773588in}}%
\pgfpathlineto{\pgfqpoint{35.080154in}{0.773588in}}%
\pgfpathlineto{\pgfqpoint{35.155466in}{0.773588in}}%
\pgfpathlineto{\pgfqpoint{35.227201in}{0.773588in}}%
\pgfpathlineto{\pgfqpoint{35.298174in}{0.773588in}}%
\pgfpathlineto{\pgfqpoint{35.372990in}{0.773588in}}%
\pgfpathlineto{\pgfqpoint{35.451774in}{0.773588in}}%
\pgfpathlineto{\pgfqpoint{35.574549in}{0.773588in}}%
\pgfpathlineto{\pgfqpoint{35.663523in}{0.773588in}}%
\pgfpathlineto{\pgfqpoint{35.741519in}{0.773588in}}%
\pgfpathlineto{\pgfqpoint{35.805568in}{0.773588in}}%
\pgfpathlineto{\pgfqpoint{35.870813in}{0.773588in}}%
\pgfpathlineto{\pgfqpoint{35.942832in}{0.773588in}}%
\pgfpathlineto{\pgfqpoint{36.012796in}{0.773588in}}%
\pgfpathlineto{\pgfqpoint{36.085094in}{0.773588in}}%
\pgfpathlineto{\pgfqpoint{36.154695in}{0.773588in}}%
\pgfpathlineto{\pgfqpoint{36.223624in}{0.773588in}}%
\pgfpathlineto{\pgfqpoint{36.293479in}{0.773588in}}%
\pgfpathlineto{\pgfqpoint{36.360634in}{0.773588in}}%
\pgfpathlineto{\pgfqpoint{36.428206in}{0.773588in}}%
\pgfpathlineto{\pgfqpoint{36.497087in}{0.773588in}}%
\pgfpathlineto{\pgfqpoint{36.563097in}{0.773588in}}%
\pgfpathlineto{\pgfqpoint{36.628950in}{0.773588in}}%
\pgfpathlineto{\pgfqpoint{36.696952in}{0.773588in}}%
\pgfpathlineto{\pgfqpoint{36.761893in}{0.773588in}}%
\pgfpathlineto{\pgfqpoint{36.827337in}{0.773588in}}%
\pgfpathlineto{\pgfqpoint{36.893714in}{0.773588in}}%
\pgfpathlineto{\pgfqpoint{36.957470in}{0.773588in}}%
\pgfpathlineto{\pgfqpoint{37.022217in}{0.773588in}}%
\pgfpathlineto{\pgfqpoint{37.088015in}{0.773588in}}%
\pgfpathlineto{\pgfqpoint{37.151827in}{0.773588in}}%
\pgfpathlineto{\pgfqpoint{37.151827in}{2.457661in}}%
\pgfpathlineto{\pgfqpoint{37.151827in}{2.457661in}}%
\pgfpathlineto{\pgfqpoint{37.088015in}{2.514045in}}%
\pgfpathlineto{\pgfqpoint{37.022217in}{2.426405in}}%
\pgfpathlineto{\pgfqpoint{36.957470in}{2.464622in}}%
\pgfpathlineto{\pgfqpoint{36.893714in}{2.468865in}}%
\pgfpathlineto{\pgfqpoint{36.827337in}{2.434812in}}%
\pgfpathlineto{\pgfqpoint{36.761893in}{2.464952in}}%
\pgfpathlineto{\pgfqpoint{36.696952in}{2.436993in}}%
\pgfpathlineto{\pgfqpoint{36.628950in}{2.376225in}}%
\pgfpathlineto{\pgfqpoint{36.563097in}{2.486664in}}%
\pgfpathlineto{\pgfqpoint{36.497087in}{2.404583in}}%
\pgfpathlineto{\pgfqpoint{36.428206in}{2.435547in}}%
\pgfpathlineto{\pgfqpoint{36.360634in}{2.349474in}}%
\pgfpathlineto{\pgfqpoint{36.293479in}{2.395888in}}%
\pgfpathlineto{\pgfqpoint{36.223624in}{2.332436in}}%
\pgfpathlineto{\pgfqpoint{36.154695in}{2.303111in}}%
\pgfpathlineto{\pgfqpoint{36.085094in}{2.286938in}}%
\pgfpathlineto{\pgfqpoint{36.012796in}{2.321125in}}%
\pgfpathlineto{\pgfqpoint{35.942832in}{2.269015in}}%
\pgfpathlineto{\pgfqpoint{35.870813in}{2.216022in}}%
\pgfpathlineto{\pgfqpoint{35.805568in}{1.249553in}}%
\pgfpathlineto{\pgfqpoint{35.741519in}{0.773588in}}%
\pgfpathlineto{\pgfqpoint{35.663523in}{0.773588in}}%
\pgfpathlineto{\pgfqpoint{35.574549in}{0.773588in}}%
\pgfpathlineto{\pgfqpoint{35.451774in}{0.773588in}}%
\pgfpathlineto{\pgfqpoint{35.372990in}{0.773588in}}%
\pgfpathlineto{\pgfqpoint{35.298174in}{0.773588in}}%
\pgfpathlineto{\pgfqpoint{35.227201in}{0.773588in}}%
\pgfpathlineto{\pgfqpoint{35.155466in}{0.773588in}}%
\pgfpathlineto{\pgfqpoint{35.080154in}{0.773588in}}%
\pgfpathlineto{\pgfqpoint{35.007838in}{0.773588in}}%
\pgfpathlineto{\pgfqpoint{34.936943in}{0.773588in}}%
\pgfpathlineto{\pgfqpoint{34.862212in}{0.773588in}}%
\pgfpathlineto{\pgfqpoint{34.789698in}{0.773588in}}%
\pgfpathlineto{\pgfqpoint{34.718731in}{0.773588in}}%
\pgfpathlineto{\pgfqpoint{34.643977in}{0.773588in}}%
\pgfpathlineto{\pgfqpoint{34.571449in}{0.773588in}}%
\pgfpathlineto{\pgfqpoint{34.499162in}{0.773588in}}%
\pgfpathlineto{\pgfqpoint{34.425604in}{0.773588in}}%
\pgfpathlineto{\pgfqpoint{34.354648in}{0.773588in}}%
\pgfpathlineto{\pgfqpoint{34.284339in}{0.773588in}}%
\pgfpathlineto{\pgfqpoint{34.210760in}{0.773588in}}%
\pgfpathlineto{\pgfqpoint{34.138346in}{0.773588in}}%
\pgfpathlineto{\pgfqpoint{34.067999in}{0.773588in}}%
\pgfpathlineto{\pgfqpoint{33.994648in}{0.773588in}}%
\pgfpathlineto{\pgfqpoint{33.923536in}{0.773588in}}%
\pgfpathlineto{\pgfqpoint{33.852393in}{0.773588in}}%
\pgfpathlineto{\pgfqpoint{33.776888in}{0.773588in}}%
\pgfpathlineto{\pgfqpoint{33.703848in}{0.773588in}}%
\pgfpathlineto{\pgfqpoint{33.631884in}{0.773588in}}%
\pgfpathlineto{\pgfqpoint{33.557012in}{0.773588in}}%
\pgfpathlineto{\pgfqpoint{33.483328in}{0.773588in}}%
\pgfpathlineto{\pgfqpoint{33.409286in}{0.773588in}}%
\pgfpathlineto{\pgfqpoint{33.333311in}{0.773588in}}%
\pgfpathlineto{\pgfqpoint{33.259507in}{0.773588in}}%
\pgfpathlineto{\pgfqpoint{33.185787in}{0.773588in}}%
\pgfpathlineto{\pgfqpoint{33.110479in}{0.773588in}}%
\pgfpathlineto{\pgfqpoint{33.037794in}{0.773588in}}%
\pgfpathlineto{\pgfqpoint{32.963168in}{0.773588in}}%
\pgfpathlineto{\pgfqpoint{32.887492in}{0.773588in}}%
\pgfpathlineto{\pgfqpoint{32.813546in}{0.773588in}}%
\pgfpathlineto{\pgfqpoint{32.740263in}{0.773588in}}%
\pgfpathlineto{\pgfqpoint{32.663709in}{0.773588in}}%
\pgfpathlineto{\pgfqpoint{32.590674in}{0.773588in}}%
\pgfpathlineto{\pgfqpoint{32.520401in}{0.773588in}}%
\pgfpathlineto{\pgfqpoint{32.447439in}{0.773588in}}%
\pgfpathlineto{\pgfqpoint{32.377016in}{0.773588in}}%
\pgfpathlineto{\pgfqpoint{32.305440in}{0.773588in}}%
\pgfpathlineto{\pgfqpoint{32.231773in}{0.773588in}}%
\pgfpathlineto{\pgfqpoint{32.161504in}{0.773588in}}%
\pgfpathlineto{\pgfqpoint{32.089684in}{0.773588in}}%
\pgfpathlineto{\pgfqpoint{32.015122in}{0.773588in}}%
\pgfpathlineto{\pgfqpoint{31.943734in}{0.773588in}}%
\pgfpathlineto{\pgfqpoint{31.873539in}{0.773588in}}%
\pgfpathlineto{\pgfqpoint{31.800877in}{0.773588in}}%
\pgfpathlineto{\pgfqpoint{31.728521in}{0.773588in}}%
\pgfpathlineto{\pgfqpoint{31.656180in}{0.773588in}}%
\pgfpathlineto{\pgfqpoint{31.582282in}{0.773588in}}%
\pgfpathlineto{\pgfqpoint{31.510140in}{0.773588in}}%
\pgfpathlineto{\pgfqpoint{31.439415in}{0.773588in}}%
\pgfpathlineto{\pgfqpoint{31.366613in}{0.773588in}}%
\pgfpathlineto{\pgfqpoint{31.294777in}{0.773588in}}%
\pgfpathlineto{\pgfqpoint{31.222913in}{0.773588in}}%
\pgfpathlineto{\pgfqpoint{31.147740in}{0.773588in}}%
\pgfpathlineto{\pgfqpoint{31.074452in}{0.773588in}}%
\pgfpathlineto{\pgfqpoint{31.002514in}{0.773588in}}%
\pgfpathlineto{\pgfqpoint{30.928305in}{0.773588in}}%
\pgfpathlineto{\pgfqpoint{30.854115in}{0.773588in}}%
\pgfpathlineto{\pgfqpoint{30.782249in}{0.773588in}}%
\pgfpathlineto{\pgfqpoint{30.707828in}{0.773588in}}%
\pgfpathlineto{\pgfqpoint{30.634099in}{0.773588in}}%
\pgfpathlineto{\pgfqpoint{30.562714in}{0.773588in}}%
\pgfpathlineto{\pgfqpoint{30.488991in}{0.773588in}}%
\pgfpathlineto{\pgfqpoint{30.417098in}{0.773588in}}%
\pgfpathlineto{\pgfqpoint{30.344328in}{0.773588in}}%
\pgfpathlineto{\pgfqpoint{30.269036in}{0.773588in}}%
\pgfpathlineto{\pgfqpoint{30.195443in}{0.773588in}}%
\pgfpathlineto{\pgfqpoint{30.122796in}{0.773588in}}%
\pgfpathlineto{\pgfqpoint{30.048252in}{0.773588in}}%
\pgfpathlineto{\pgfqpoint{29.974646in}{0.773588in}}%
\pgfpathlineto{\pgfqpoint{29.901656in}{0.773588in}}%
\pgfpathlineto{\pgfqpoint{29.827132in}{0.773588in}}%
\pgfpathlineto{\pgfqpoint{29.755113in}{0.773588in}}%
\pgfpathlineto{\pgfqpoint{29.684427in}{0.773588in}}%
\pgfpathlineto{\pgfqpoint{29.611700in}{0.773588in}}%
\pgfpathlineto{\pgfqpoint{29.540420in}{0.773588in}}%
\pgfpathlineto{\pgfqpoint{29.467898in}{0.773588in}}%
\pgfpathlineto{\pgfqpoint{29.393660in}{0.773588in}}%
\pgfpathlineto{\pgfqpoint{29.320759in}{0.773588in}}%
\pgfpathlineto{\pgfqpoint{29.248973in}{0.773588in}}%
\pgfpathlineto{\pgfqpoint{29.173201in}{0.773588in}}%
\pgfpathlineto{\pgfqpoint{29.100748in}{0.773588in}}%
\pgfpathlineto{\pgfqpoint{29.029887in}{0.773588in}}%
\pgfpathlineto{\pgfqpoint{28.956832in}{0.773588in}}%
\pgfpathlineto{\pgfqpoint{28.885217in}{0.773588in}}%
\pgfpathlineto{\pgfqpoint{28.811911in}{0.773588in}}%
\pgfpathlineto{\pgfqpoint{28.738605in}{0.773588in}}%
\pgfpathlineto{\pgfqpoint{28.668204in}{0.773588in}}%
\pgfpathlineto{\pgfqpoint{28.596041in}{0.773588in}}%
\pgfpathlineto{\pgfqpoint{28.522534in}{0.773588in}}%
\pgfpathlineto{\pgfqpoint{28.451943in}{0.773588in}}%
\pgfpathlineto{\pgfqpoint{28.380501in}{0.773588in}}%
\pgfpathlineto{\pgfqpoint{28.306604in}{0.773588in}}%
\pgfpathlineto{\pgfqpoint{28.234559in}{0.773588in}}%
\pgfpathlineto{\pgfqpoint{28.163352in}{0.773588in}}%
\pgfpathlineto{\pgfqpoint{28.090360in}{0.773588in}}%
\pgfpathlineto{\pgfqpoint{28.018234in}{0.773588in}}%
\pgfpathlineto{\pgfqpoint{27.943911in}{0.773588in}}%
\pgfpathlineto{\pgfqpoint{27.868117in}{0.773588in}}%
\pgfpathlineto{\pgfqpoint{27.795343in}{0.773588in}}%
\pgfpathlineto{\pgfqpoint{27.724006in}{0.773588in}}%
\pgfpathlineto{\pgfqpoint{27.649072in}{0.773588in}}%
\pgfpathlineto{\pgfqpoint{27.574929in}{0.773588in}}%
\pgfpathlineto{\pgfqpoint{27.500063in}{0.773588in}}%
\pgfpathlineto{\pgfqpoint{27.423884in}{0.773588in}}%
\pgfpathlineto{\pgfqpoint{27.350990in}{0.773588in}}%
\pgfpathlineto{\pgfqpoint{27.277488in}{0.773588in}}%
\pgfpathlineto{\pgfqpoint{27.201477in}{0.773588in}}%
\pgfpathlineto{\pgfqpoint{27.128948in}{0.773588in}}%
\pgfpathlineto{\pgfqpoint{27.056835in}{0.773588in}}%
\pgfpathlineto{\pgfqpoint{26.983641in}{0.773588in}}%
\pgfpathlineto{\pgfqpoint{26.912667in}{0.773588in}}%
\pgfpathlineto{\pgfqpoint{26.841234in}{0.773588in}}%
\pgfpathlineto{\pgfqpoint{26.769271in}{0.773588in}}%
\pgfpathlineto{\pgfqpoint{26.699584in}{0.773588in}}%
\pgfpathlineto{\pgfqpoint{26.629572in}{0.773588in}}%
\pgfpathlineto{\pgfqpoint{26.557235in}{0.773588in}}%
\pgfpathlineto{\pgfqpoint{26.487420in}{0.773588in}}%
\pgfpathlineto{\pgfqpoint{26.417231in}{0.773588in}}%
\pgfpathlineto{\pgfqpoint{26.344920in}{0.773588in}}%
\pgfpathlineto{\pgfqpoint{26.274267in}{0.773588in}}%
\pgfpathlineto{\pgfqpoint{26.203572in}{0.773588in}}%
\pgfpathlineto{\pgfqpoint{26.130346in}{0.773588in}}%
\pgfpathlineto{\pgfqpoint{26.058621in}{0.773588in}}%
\pgfpathlineto{\pgfqpoint{25.988588in}{0.773588in}}%
\pgfpathlineto{\pgfqpoint{25.916551in}{0.773588in}}%
\pgfpathlineto{\pgfqpoint{25.845667in}{0.773588in}}%
\pgfpathlineto{\pgfqpoint{25.775695in}{0.773588in}}%
\pgfpathlineto{\pgfqpoint{25.702341in}{0.773588in}}%
\pgfpathlineto{\pgfqpoint{25.631022in}{0.773588in}}%
\pgfpathlineto{\pgfqpoint{25.560110in}{0.773588in}}%
\pgfpathlineto{\pgfqpoint{25.487573in}{0.773588in}}%
\pgfpathlineto{\pgfqpoint{25.417047in}{0.773588in}}%
\pgfpathlineto{\pgfqpoint{25.347124in}{0.773588in}}%
\pgfpathlineto{\pgfqpoint{25.273349in}{0.773588in}}%
\pgfpathlineto{\pgfqpoint{25.203216in}{0.773588in}}%
\pgfpathlineto{\pgfqpoint{25.131109in}{0.773588in}}%
\pgfpathlineto{\pgfqpoint{25.055567in}{0.773588in}}%
\pgfpathlineto{\pgfqpoint{24.983525in}{0.773588in}}%
\pgfpathlineto{\pgfqpoint{24.911741in}{0.773588in}}%
\pgfpathlineto{\pgfqpoint{24.836992in}{0.773588in}}%
\pgfpathlineto{\pgfqpoint{24.764742in}{0.773588in}}%
\pgfpathlineto{\pgfqpoint{24.691660in}{0.773588in}}%
\pgfpathlineto{\pgfqpoint{24.618678in}{0.773588in}}%
\pgfpathlineto{\pgfqpoint{24.548639in}{0.773588in}}%
\pgfpathlineto{\pgfqpoint{24.476339in}{0.773588in}}%
\pgfpathlineto{\pgfqpoint{24.400328in}{0.773588in}}%
\pgfpathlineto{\pgfqpoint{24.329090in}{0.773588in}}%
\pgfpathlineto{\pgfqpoint{24.257307in}{0.773588in}}%
\pgfpathlineto{\pgfqpoint{24.183899in}{0.773588in}}%
\pgfpathlineto{\pgfqpoint{24.111784in}{0.773588in}}%
\pgfpathlineto{\pgfqpoint{24.039255in}{0.773588in}}%
\pgfpathlineto{\pgfqpoint{23.966722in}{0.773588in}}%
\pgfpathlineto{\pgfqpoint{23.895213in}{0.773588in}}%
\pgfpathlineto{\pgfqpoint{23.824951in}{0.773588in}}%
\pgfpathlineto{\pgfqpoint{23.752361in}{0.773588in}}%
\pgfpathlineto{\pgfqpoint{23.681448in}{0.773588in}}%
\pgfpathlineto{\pgfqpoint{23.610036in}{0.773588in}}%
\pgfpathlineto{\pgfqpoint{23.537850in}{0.773588in}}%
\pgfpathlineto{\pgfqpoint{23.467323in}{0.773588in}}%
\pgfpathlineto{\pgfqpoint{23.396126in}{0.773588in}}%
\pgfpathlineto{\pgfqpoint{23.323995in}{0.773588in}}%
\pgfpathlineto{\pgfqpoint{23.253484in}{0.773588in}}%
\pgfpathlineto{\pgfqpoint{23.184270in}{0.773588in}}%
\pgfpathlineto{\pgfqpoint{23.113267in}{0.773588in}}%
\pgfpathlineto{\pgfqpoint{23.043397in}{0.773588in}}%
\pgfpathlineto{\pgfqpoint{22.973868in}{0.773588in}}%
\pgfpathlineto{\pgfqpoint{22.902896in}{0.773588in}}%
\pgfpathlineto{\pgfqpoint{22.833628in}{0.773588in}}%
\pgfpathlineto{\pgfqpoint{22.764651in}{0.773588in}}%
\pgfpathlineto{\pgfqpoint{22.694016in}{0.773588in}}%
\pgfpathlineto{\pgfqpoint{22.624114in}{0.773588in}}%
\pgfpathlineto{\pgfqpoint{22.553535in}{0.773588in}}%
\pgfpathlineto{\pgfqpoint{22.482516in}{0.773588in}}%
\pgfpathlineto{\pgfqpoint{22.413449in}{0.773588in}}%
\pgfpathlineto{\pgfqpoint{22.343331in}{0.773588in}}%
\pgfpathlineto{\pgfqpoint{22.268824in}{0.773588in}}%
\pgfpathlineto{\pgfqpoint{22.195707in}{0.773588in}}%
\pgfpathlineto{\pgfqpoint{22.122462in}{0.773588in}}%
\pgfpathlineto{\pgfqpoint{22.048040in}{0.773588in}}%
\pgfpathlineto{\pgfqpoint{21.976836in}{0.773588in}}%
\pgfpathlineto{\pgfqpoint{21.903868in}{0.773588in}}%
\pgfpathlineto{\pgfqpoint{21.828555in}{0.773588in}}%
\pgfpathlineto{\pgfqpoint{21.756227in}{0.773588in}}%
\pgfpathlineto{\pgfqpoint{21.683994in}{0.773588in}}%
\pgfpathlineto{\pgfqpoint{21.610878in}{0.773588in}}%
\pgfpathlineto{\pgfqpoint{21.539601in}{0.773588in}}%
\pgfpathlineto{\pgfqpoint{21.467741in}{0.773588in}}%
\pgfpathlineto{\pgfqpoint{21.394839in}{0.773588in}}%
\pgfpathlineto{\pgfqpoint{21.324498in}{0.773588in}}%
\pgfpathlineto{\pgfqpoint{21.254645in}{0.773588in}}%
\pgfpathlineto{\pgfqpoint{21.181233in}{0.773588in}}%
\pgfpathlineto{\pgfqpoint{21.110656in}{0.773588in}}%
\pgfpathlineto{\pgfqpoint{21.040864in}{0.773588in}}%
\pgfpathlineto{\pgfqpoint{20.969373in}{0.773588in}}%
\pgfpathlineto{\pgfqpoint{20.899587in}{0.773588in}}%
\pgfpathlineto{\pgfqpoint{20.830586in}{0.773588in}}%
\pgfpathlineto{\pgfqpoint{20.758814in}{0.773588in}}%
\pgfpathlineto{\pgfqpoint{20.688980in}{0.773588in}}%
\pgfpathlineto{\pgfqpoint{20.618428in}{0.773588in}}%
\pgfpathlineto{\pgfqpoint{20.547172in}{0.773588in}}%
\pgfpathlineto{\pgfqpoint{20.476961in}{0.773588in}}%
\pgfpathlineto{\pgfqpoint{20.407276in}{0.773588in}}%
\pgfpathlineto{\pgfqpoint{20.334605in}{0.773588in}}%
\pgfpathlineto{\pgfqpoint{20.263963in}{0.773588in}}%
\pgfpathlineto{\pgfqpoint{20.193561in}{0.773588in}}%
\pgfpathlineto{\pgfqpoint{20.121885in}{0.773588in}}%
\pgfpathlineto{\pgfqpoint{20.052071in}{0.773588in}}%
\pgfpathlineto{\pgfqpoint{19.981199in}{0.773588in}}%
\pgfpathlineto{\pgfqpoint{19.907049in}{0.773588in}}%
\pgfpathlineto{\pgfqpoint{19.835928in}{0.773588in}}%
\pgfpathlineto{\pgfqpoint{19.766060in}{0.773588in}}%
\pgfpathlineto{\pgfqpoint{19.693765in}{0.773588in}}%
\pgfpathlineto{\pgfqpoint{19.623855in}{0.773588in}}%
\pgfpathlineto{\pgfqpoint{19.553641in}{0.773588in}}%
\pgfpathlineto{\pgfqpoint{19.482791in}{0.773588in}}%
\pgfpathlineto{\pgfqpoint{19.412352in}{0.773588in}}%
\pgfpathlineto{\pgfqpoint{19.340700in}{0.773588in}}%
\pgfpathlineto{\pgfqpoint{19.266961in}{0.773588in}}%
\pgfpathlineto{\pgfqpoint{19.195134in}{0.773588in}}%
\pgfpathlineto{\pgfqpoint{19.123240in}{0.773588in}}%
\pgfpathlineto{\pgfqpoint{19.048835in}{0.773588in}}%
\pgfpathlineto{\pgfqpoint{18.977585in}{0.773588in}}%
\pgfpathlineto{\pgfqpoint{18.906958in}{0.773588in}}%
\pgfpathlineto{\pgfqpoint{18.834141in}{0.773588in}}%
\pgfpathlineto{\pgfqpoint{18.763558in}{0.773588in}}%
\pgfpathlineto{\pgfqpoint{18.692839in}{0.773588in}}%
\pgfpathlineto{\pgfqpoint{18.619849in}{0.773588in}}%
\pgfpathlineto{\pgfqpoint{18.549509in}{0.773588in}}%
\pgfpathlineto{\pgfqpoint{18.479143in}{0.773588in}}%
\pgfpathlineto{\pgfqpoint{18.406899in}{0.773588in}}%
\pgfpathlineto{\pgfqpoint{18.336223in}{0.773588in}}%
\pgfpathlineto{\pgfqpoint{18.266780in}{0.773588in}}%
\pgfpathlineto{\pgfqpoint{18.195268in}{0.773588in}}%
\pgfpathlineto{\pgfqpoint{18.125960in}{0.773588in}}%
\pgfpathlineto{\pgfqpoint{18.056086in}{0.773588in}}%
\pgfpathlineto{\pgfqpoint{17.983524in}{0.773588in}}%
\pgfpathlineto{\pgfqpoint{17.912627in}{0.773588in}}%
\pgfpathlineto{\pgfqpoint{17.844148in}{0.773588in}}%
\pgfpathlineto{\pgfqpoint{17.772898in}{0.773588in}}%
\pgfpathlineto{\pgfqpoint{17.703480in}{0.773588in}}%
\pgfpathlineto{\pgfqpoint{17.634782in}{0.773588in}}%
\pgfpathlineto{\pgfqpoint{17.563254in}{0.773588in}}%
\pgfpathlineto{\pgfqpoint{17.494859in}{0.773588in}}%
\pgfpathlineto{\pgfqpoint{17.426570in}{0.773588in}}%
\pgfpathlineto{\pgfqpoint{17.356523in}{0.773588in}}%
\pgfpathlineto{\pgfqpoint{17.288800in}{0.773588in}}%
\pgfpathlineto{\pgfqpoint{17.220960in}{0.773588in}}%
\pgfpathlineto{\pgfqpoint{17.150280in}{0.773588in}}%
\pgfpathlineto{\pgfqpoint{17.081113in}{0.773588in}}%
\pgfpathlineto{\pgfqpoint{17.013249in}{0.773588in}}%
\pgfpathlineto{\pgfqpoint{16.942299in}{0.773588in}}%
\pgfpathlineto{\pgfqpoint{16.872974in}{0.773588in}}%
\pgfpathlineto{\pgfqpoint{16.803402in}{0.773588in}}%
\pgfpathlineto{\pgfqpoint{16.731829in}{0.773588in}}%
\pgfpathlineto{\pgfqpoint{16.662798in}{0.773588in}}%
\pgfpathlineto{\pgfqpoint{16.592371in}{0.773588in}}%
\pgfpathlineto{\pgfqpoint{16.519393in}{0.773588in}}%
\pgfpathlineto{\pgfqpoint{16.445981in}{0.773588in}}%
\pgfpathlineto{\pgfqpoint{16.373054in}{0.773588in}}%
\pgfpathlineto{\pgfqpoint{16.299236in}{0.773588in}}%
\pgfpathlineto{\pgfqpoint{16.229236in}{0.773588in}}%
\pgfpathlineto{\pgfqpoint{16.160188in}{0.773588in}}%
\pgfpathlineto{\pgfqpoint{16.088678in}{0.773588in}}%
\pgfpathlineto{\pgfqpoint{16.019598in}{0.773588in}}%
\pgfpathlineto{\pgfqpoint{15.949677in}{0.773588in}}%
\pgfpathlineto{\pgfqpoint{15.878678in}{0.773588in}}%
\pgfpathlineto{\pgfqpoint{15.807467in}{0.773588in}}%
\pgfpathlineto{\pgfqpoint{15.737902in}{0.773588in}}%
\pgfpathlineto{\pgfqpoint{15.665480in}{0.773588in}}%
\pgfpathlineto{\pgfqpoint{15.595878in}{0.773588in}}%
\pgfpathlineto{\pgfqpoint{15.527961in}{0.773588in}}%
\pgfpathlineto{\pgfqpoint{15.457342in}{0.773588in}}%
\pgfpathlineto{\pgfqpoint{15.388582in}{0.773588in}}%
\pgfpathlineto{\pgfqpoint{15.320168in}{0.773588in}}%
\pgfpathlineto{\pgfqpoint{15.249597in}{0.773588in}}%
\pgfpathlineto{\pgfqpoint{15.182512in}{0.773588in}}%
\pgfpathlineto{\pgfqpoint{15.115140in}{0.773588in}}%
\pgfpathlineto{\pgfqpoint{15.044439in}{0.773588in}}%
\pgfpathlineto{\pgfqpoint{14.976372in}{0.773588in}}%
\pgfpathlineto{\pgfqpoint{14.908199in}{0.773588in}}%
\pgfpathlineto{\pgfqpoint{14.836607in}{0.773588in}}%
\pgfpathlineto{\pgfqpoint{14.768225in}{0.773588in}}%
\pgfpathlineto{\pgfqpoint{14.700003in}{0.773588in}}%
\pgfpathlineto{\pgfqpoint{14.630523in}{0.773588in}}%
\pgfpathlineto{\pgfqpoint{14.562284in}{0.773588in}}%
\pgfpathlineto{\pgfqpoint{14.494110in}{0.773588in}}%
\pgfpathlineto{\pgfqpoint{14.424138in}{0.773588in}}%
\pgfpathlineto{\pgfqpoint{14.355979in}{0.773588in}}%
\pgfpathlineto{\pgfqpoint{14.286685in}{0.773588in}}%
\pgfpathlineto{\pgfqpoint{14.216102in}{0.773588in}}%
\pgfpathlineto{\pgfqpoint{14.149562in}{0.773588in}}%
\pgfpathlineto{\pgfqpoint{14.081564in}{0.773588in}}%
\pgfpathlineto{\pgfqpoint{14.012639in}{0.773588in}}%
\pgfpathlineto{\pgfqpoint{13.944783in}{0.773588in}}%
\pgfpathlineto{\pgfqpoint{13.876785in}{0.773588in}}%
\pgfpathlineto{\pgfqpoint{13.804228in}{0.773588in}}%
\pgfpathlineto{\pgfqpoint{13.735499in}{0.773588in}}%
\pgfpathlineto{\pgfqpoint{13.666880in}{0.773588in}}%
\pgfpathlineto{\pgfqpoint{13.595052in}{0.773588in}}%
\pgfpathlineto{\pgfqpoint{13.526360in}{0.773588in}}%
\pgfpathlineto{\pgfqpoint{13.457833in}{0.773588in}}%
\pgfpathlineto{\pgfqpoint{13.387091in}{0.773588in}}%
\pgfpathlineto{\pgfqpoint{13.319414in}{0.773588in}}%
\pgfpathlineto{\pgfqpoint{13.250526in}{0.773588in}}%
\pgfpathlineto{\pgfqpoint{13.178279in}{0.773588in}}%
\pgfpathlineto{\pgfqpoint{13.109710in}{0.773588in}}%
\pgfpathlineto{\pgfqpoint{13.041078in}{0.773588in}}%
\pgfpathlineto{\pgfqpoint{12.969036in}{0.773588in}}%
\pgfpathlineto{\pgfqpoint{12.900232in}{0.773588in}}%
\pgfpathlineto{\pgfqpoint{12.830499in}{0.773588in}}%
\pgfpathlineto{\pgfqpoint{12.759967in}{0.773588in}}%
\pgfpathlineto{\pgfqpoint{12.692271in}{0.773588in}}%
\pgfpathlineto{\pgfqpoint{12.623527in}{0.773588in}}%
\pgfpathlineto{\pgfqpoint{12.552597in}{0.773588in}}%
\pgfpathlineto{\pgfqpoint{12.484090in}{0.773588in}}%
\pgfpathlineto{\pgfqpoint{12.416448in}{0.773588in}}%
\pgfpathlineto{\pgfqpoint{12.347008in}{0.773588in}}%
\pgfpathlineto{\pgfqpoint{12.280761in}{0.773588in}}%
\pgfpathlineto{\pgfqpoint{12.213872in}{0.773588in}}%
\pgfpathlineto{\pgfqpoint{12.144815in}{0.773588in}}%
\pgfpathlineto{\pgfqpoint{12.076638in}{0.773588in}}%
\pgfpathlineto{\pgfqpoint{12.008843in}{0.773588in}}%
\pgfpathlineto{\pgfqpoint{11.938397in}{0.773588in}}%
\pgfpathlineto{\pgfqpoint{11.871239in}{0.773588in}}%
\pgfpathlineto{\pgfqpoint{11.804695in}{0.773588in}}%
\pgfpathlineto{\pgfqpoint{11.736702in}{0.773588in}}%
\pgfpathlineto{\pgfqpoint{11.669969in}{0.773588in}}%
\pgfpathlineto{\pgfqpoint{11.602395in}{0.773588in}}%
\pgfpathlineto{\pgfqpoint{11.532877in}{0.773588in}}%
\pgfpathlineto{\pgfqpoint{11.465294in}{0.773588in}}%
\pgfpathlineto{\pgfqpoint{11.398136in}{0.773588in}}%
\pgfpathlineto{\pgfqpoint{11.328677in}{0.773588in}}%
\pgfpathlineto{\pgfqpoint{11.261777in}{0.773588in}}%
\pgfpathlineto{\pgfqpoint{11.194984in}{0.773588in}}%
\pgfpathlineto{\pgfqpoint{11.125133in}{0.773588in}}%
\pgfpathlineto{\pgfqpoint{11.055389in}{0.773588in}}%
\pgfpathlineto{\pgfqpoint{10.986941in}{0.773588in}}%
\pgfpathlineto{\pgfqpoint{10.916008in}{0.773588in}}%
\pgfpathlineto{\pgfqpoint{10.846851in}{0.773588in}}%
\pgfpathlineto{\pgfqpoint{10.777925in}{0.773588in}}%
\pgfpathlineto{\pgfqpoint{10.708017in}{0.773588in}}%
\pgfpathlineto{\pgfqpoint{10.639006in}{0.773588in}}%
\pgfpathlineto{\pgfqpoint{10.569898in}{0.773588in}}%
\pgfpathlineto{\pgfqpoint{10.498528in}{0.773588in}}%
\pgfpathlineto{\pgfqpoint{10.428999in}{0.773588in}}%
\pgfpathlineto{\pgfqpoint{10.360853in}{0.773588in}}%
\pgfpathlineto{\pgfqpoint{10.290696in}{0.773588in}}%
\pgfpathlineto{\pgfqpoint{10.222102in}{0.773588in}}%
\pgfpathlineto{\pgfqpoint{10.153891in}{0.773588in}}%
\pgfpathlineto{\pgfqpoint{10.081602in}{0.773588in}}%
\pgfpathlineto{\pgfqpoint{10.012687in}{0.773588in}}%
\pgfpathlineto{\pgfqpoint{9.944615in}{0.773588in}}%
\pgfpathlineto{\pgfqpoint{9.876148in}{0.773588in}}%
\pgfpathlineto{\pgfqpoint{9.808815in}{0.773588in}}%
\pgfpathlineto{\pgfqpoint{9.740273in}{0.773588in}}%
\pgfpathlineto{\pgfqpoint{9.670768in}{0.773588in}}%
\pgfpathlineto{\pgfqpoint{9.603748in}{0.773588in}}%
\pgfpathlineto{\pgfqpoint{9.536145in}{0.773588in}}%
\pgfpathlineto{\pgfqpoint{9.467314in}{0.773588in}}%
\pgfpathlineto{\pgfqpoint{9.402138in}{0.773588in}}%
\pgfpathlineto{\pgfqpoint{9.334728in}{0.773588in}}%
\pgfpathlineto{\pgfqpoint{9.265262in}{0.773588in}}%
\pgfpathlineto{\pgfqpoint{9.198477in}{0.773588in}}%
\pgfpathlineto{\pgfqpoint{9.130518in}{0.773588in}}%
\pgfpathlineto{\pgfqpoint{9.059607in}{0.773588in}}%
\pgfpathlineto{\pgfqpoint{8.991646in}{0.773588in}}%
\pgfpathlineto{\pgfqpoint{8.924860in}{0.773588in}}%
\pgfpathlineto{\pgfqpoint{8.856098in}{0.773588in}}%
\pgfpathlineto{\pgfqpoint{8.787638in}{0.773588in}}%
\pgfpathlineto{\pgfqpoint{8.720615in}{0.773588in}}%
\pgfpathlineto{\pgfqpoint{8.651757in}{0.773588in}}%
\pgfpathlineto{\pgfqpoint{8.583601in}{0.773588in}}%
\pgfpathlineto{\pgfqpoint{8.515085in}{0.773588in}}%
\pgfpathlineto{\pgfqpoint{8.444285in}{0.773588in}}%
\pgfpathlineto{\pgfqpoint{8.376454in}{0.773588in}}%
\pgfpathlineto{\pgfqpoint{8.308718in}{0.773588in}}%
\pgfpathlineto{\pgfqpoint{8.237831in}{0.773588in}}%
\pgfpathlineto{\pgfqpoint{8.168934in}{0.773588in}}%
\pgfpathlineto{\pgfqpoint{8.100457in}{0.773588in}}%
\pgfpathlineto{\pgfqpoint{8.028258in}{0.773588in}}%
\pgfpathlineto{\pgfqpoint{7.957986in}{0.773588in}}%
\pgfpathlineto{\pgfqpoint{7.887441in}{0.773588in}}%
\pgfpathlineto{\pgfqpoint{7.812286in}{0.773588in}}%
\pgfpathlineto{\pgfqpoint{7.741421in}{0.773588in}}%
\pgfpathlineto{\pgfqpoint{7.671131in}{0.773588in}}%
\pgfpathlineto{\pgfqpoint{7.598690in}{0.773588in}}%
\pgfpathlineto{\pgfqpoint{7.527107in}{0.773588in}}%
\pgfpathlineto{\pgfqpoint{7.455306in}{0.773588in}}%
\pgfpathlineto{\pgfqpoint{7.379810in}{0.773588in}}%
\pgfpathlineto{\pgfqpoint{7.308295in}{0.773588in}}%
\pgfpathlineto{\pgfqpoint{7.238831in}{0.773588in}}%
\pgfpathlineto{\pgfqpoint{7.171034in}{0.773588in}}%
\pgfpathlineto{\pgfqpoint{7.106301in}{0.773588in}}%
\pgfpathlineto{\pgfqpoint{7.040805in}{0.773588in}}%
\pgfpathlineto{\pgfqpoint{6.973333in}{0.773588in}}%
\pgfpathlineto{\pgfqpoint{6.906544in}{0.773588in}}%
\pgfpathlineto{\pgfqpoint{6.839348in}{0.773588in}}%
\pgfpathlineto{\pgfqpoint{6.770581in}{0.773588in}}%
\pgfpathlineto{\pgfqpoint{6.704271in}{0.773588in}}%
\pgfpathlineto{\pgfqpoint{6.636218in}{0.773588in}}%
\pgfpathlineto{\pgfqpoint{6.568004in}{0.773588in}}%
\pgfpathlineto{\pgfqpoint{6.502885in}{0.773588in}}%
\pgfpathlineto{\pgfqpoint{6.437563in}{0.773588in}}%
\pgfpathlineto{\pgfqpoint{6.369842in}{0.773588in}}%
\pgfpathlineto{\pgfqpoint{6.303479in}{0.773588in}}%
\pgfpathlineto{\pgfqpoint{6.236861in}{0.773588in}}%
\pgfpathlineto{\pgfqpoint{6.169451in}{0.773588in}}%
\pgfpathlineto{\pgfqpoint{6.103260in}{0.773588in}}%
\pgfpathlineto{\pgfqpoint{6.036866in}{0.773588in}}%
\pgfpathlineto{\pgfqpoint{5.968494in}{0.773588in}}%
\pgfpathlineto{\pgfqpoint{5.901623in}{0.773588in}}%
\pgfpathlineto{\pgfqpoint{5.834669in}{0.773588in}}%
\pgfpathlineto{\pgfqpoint{5.766709in}{0.773588in}}%
\pgfpathlineto{\pgfqpoint{5.700467in}{0.773588in}}%
\pgfpathlineto{\pgfqpoint{5.633136in}{0.773588in}}%
\pgfpathlineto{\pgfqpoint{5.561464in}{0.773588in}}%
\pgfpathlineto{\pgfqpoint{5.492693in}{0.773588in}}%
\pgfpathlineto{\pgfqpoint{5.424203in}{0.773588in}}%
\pgfpathlineto{\pgfqpoint{5.353942in}{0.773588in}}%
\pgfpathlineto{\pgfqpoint{5.283251in}{0.773588in}}%
\pgfpathlineto{\pgfqpoint{5.213575in}{0.773588in}}%
\pgfpathlineto{\pgfqpoint{5.142206in}{0.773588in}}%
\pgfpathlineto{\pgfqpoint{5.073090in}{0.773588in}}%
\pgfpathlineto{\pgfqpoint{5.005143in}{0.773588in}}%
\pgfpathlineto{\pgfqpoint{4.935251in}{0.773588in}}%
\pgfpathlineto{\pgfqpoint{4.866311in}{0.773588in}}%
\pgfpathlineto{\pgfqpoint{4.796564in}{0.773588in}}%
\pgfpathlineto{\pgfqpoint{4.726503in}{0.773588in}}%
\pgfpathlineto{\pgfqpoint{4.658575in}{0.773588in}}%
\pgfpathlineto{\pgfqpoint{4.590503in}{0.773588in}}%
\pgfpathlineto{\pgfqpoint{4.520559in}{0.773588in}}%
\pgfpathlineto{\pgfqpoint{4.454265in}{0.773588in}}%
\pgfpathlineto{\pgfqpoint{4.387175in}{0.773588in}}%
\pgfpathlineto{\pgfqpoint{4.319299in}{0.773588in}}%
\pgfpathlineto{\pgfqpoint{4.253047in}{0.773588in}}%
\pgfpathlineto{\pgfqpoint{4.186150in}{0.773588in}}%
\pgfpathlineto{\pgfqpoint{4.116566in}{0.773588in}}%
\pgfpathlineto{\pgfqpoint{4.048987in}{0.773588in}}%
\pgfpathlineto{\pgfqpoint{3.981945in}{0.773588in}}%
\pgfpathlineto{\pgfqpoint{3.914312in}{0.773588in}}%
\pgfpathlineto{\pgfqpoint{3.848285in}{0.773588in}}%
\pgfpathlineto{\pgfqpoint{3.781205in}{0.773588in}}%
\pgfpathlineto{\pgfqpoint{3.712535in}{0.773588in}}%
\pgfpathlineto{\pgfqpoint{3.643958in}{0.773588in}}%
\pgfpathlineto{\pgfqpoint{3.575543in}{0.773588in}}%
\pgfpathlineto{\pgfqpoint{3.503039in}{0.773588in}}%
\pgfpathlineto{\pgfqpoint{3.430573in}{0.773588in}}%
\pgfpathlineto{\pgfqpoint{3.356465in}{0.773588in}}%
\pgfpathlineto{\pgfqpoint{3.277599in}{0.773588in}}%
\pgfpathlineto{\pgfqpoint{3.195433in}{0.773588in}}%
\pgfpathlineto{\pgfqpoint{3.123114in}{0.773588in}}%
\pgfpathlineto{\pgfqpoint{3.047640in}{0.773588in}}%
\pgfpathlineto{\pgfqpoint{2.975015in}{0.773588in}}%
\pgfpathlineto{\pgfqpoint{2.902674in}{0.773588in}}%
\pgfpathlineto{\pgfqpoint{2.826501in}{0.773588in}}%
\pgfpathlineto{\pgfqpoint{2.753491in}{0.773588in}}%
\pgfpathlineto{\pgfqpoint{2.681331in}{0.773588in}}%
\pgfpathlineto{\pgfqpoint{2.606023in}{0.773588in}}%
\pgfpathlineto{\pgfqpoint{2.534117in}{0.773588in}}%
\pgfpathlineto{\pgfqpoint{2.462769in}{0.773588in}}%
\pgfpathlineto{\pgfqpoint{2.387948in}{0.773588in}}%
\pgfpathlineto{\pgfqpoint{2.317152in}{0.773588in}}%
\pgfpathlineto{\pgfqpoint{2.248172in}{0.773588in}}%
\pgfpathlineto{\pgfqpoint{2.177337in}{0.773588in}}%
\pgfpathlineto{\pgfqpoint{2.108883in}{0.773588in}}%
\pgfpathlineto{\pgfqpoint{2.041179in}{0.773588in}}%
\pgfpathlineto{\pgfqpoint{1.970951in}{0.773588in}}%
\pgfpathlineto{\pgfqpoint{1.904458in}{0.773588in}}%
\pgfpathlineto{\pgfqpoint{1.835890in}{0.773588in}}%
\pgfpathlineto{\pgfqpoint{1.766402in}{0.773588in}}%
\pgfpathlineto{\pgfqpoint{1.698662in}{0.773588in}}%
\pgfpathlineto{\pgfqpoint{1.628862in}{0.773588in}}%
\pgfpathlineto{\pgfqpoint{1.557461in}{0.773588in}}%
\pgfpathlineto{\pgfqpoint{1.489306in}{0.773588in}}%
\pgfpathlineto{\pgfqpoint{1.421095in}{0.773588in}}%
\pgfpathlineto{\pgfqpoint{1.349373in}{0.773588in}}%
\pgfpathlineto{\pgfqpoint{1.283036in}{0.773588in}}%
\pgfpathlineto{\pgfqpoint{1.216322in}{0.773588in}}%
\pgfpathlineto{\pgfqpoint{1.147369in}{0.773588in}}%
\pgfpathlineto{\pgfqpoint{1.079942in}{0.773588in}}%
\pgfpathlineto{\pgfqpoint{1.012853in}{0.773588in}}%
\pgfpathlineto{\pgfqpoint{0.942110in}{0.773588in}}%
\pgfpathlineto{\pgfqpoint{0.875335in}{0.773588in}}%
\pgfpathlineto{\pgfqpoint{0.807094in}{0.773588in}}%
\pgfpathclose%
\pgfusepath{fill}%
\end{pgfscope}%
\begin{pgfscope}%
\pgfpathrectangle{\pgfqpoint{0.781402in}{0.773588in}}{\pgfqpoint{2.110351in}{5.415119in}}%
\pgfusepath{clip}%
\pgfsetbuttcap%
\pgfsetroundjoin%
\definecolor{currentfill}{rgb}{1.000000,0.498039,0.054902}%
\pgfsetfillcolor{currentfill}%
\pgfsetlinewidth{0.000000pt}%
\definecolor{currentstroke}{rgb}{0.000000,0.000000,0.000000}%
\pgfsetstrokecolor{currentstroke}%
\pgfsetdash{}{0pt}%
\pgfpathmoveto{\pgfqpoint{0.807094in}{0.773588in}}%
\pgfpathlineto{\pgfqpoint{0.807094in}{0.773588in}}%
\pgfpathlineto{\pgfqpoint{0.875335in}{0.773588in}}%
\pgfpathlineto{\pgfqpoint{0.942110in}{0.773588in}}%
\pgfpathlineto{\pgfqpoint{1.012853in}{0.773588in}}%
\pgfpathlineto{\pgfqpoint{1.079942in}{0.773588in}}%
\pgfpathlineto{\pgfqpoint{1.147369in}{0.773588in}}%
\pgfpathlineto{\pgfqpoint{1.216322in}{0.773588in}}%
\pgfpathlineto{\pgfqpoint{1.283036in}{0.773588in}}%
\pgfpathlineto{\pgfqpoint{1.349373in}{0.773588in}}%
\pgfpathlineto{\pgfqpoint{1.421095in}{0.773588in}}%
\pgfpathlineto{\pgfqpoint{1.489306in}{0.773588in}}%
\pgfpathlineto{\pgfqpoint{1.557461in}{0.773588in}}%
\pgfpathlineto{\pgfqpoint{1.628862in}{0.773588in}}%
\pgfpathlineto{\pgfqpoint{1.698662in}{0.773588in}}%
\pgfpathlineto{\pgfqpoint{1.766402in}{0.773588in}}%
\pgfpathlineto{\pgfqpoint{1.835890in}{0.773588in}}%
\pgfpathlineto{\pgfqpoint{1.904458in}{0.773588in}}%
\pgfpathlineto{\pgfqpoint{1.970951in}{0.773588in}}%
\pgfpathlineto{\pgfqpoint{2.041179in}{0.773588in}}%
\pgfpathlineto{\pgfqpoint{2.108883in}{0.773588in}}%
\pgfpathlineto{\pgfqpoint{2.177337in}{0.773588in}}%
\pgfpathlineto{\pgfqpoint{2.248172in}{0.773588in}}%
\pgfpathlineto{\pgfqpoint{2.317152in}{0.773588in}}%
\pgfpathlineto{\pgfqpoint{2.387948in}{0.773588in}}%
\pgfpathlineto{\pgfqpoint{2.462769in}{0.773588in}}%
\pgfpathlineto{\pgfqpoint{2.534117in}{0.773588in}}%
\pgfpathlineto{\pgfqpoint{2.606023in}{0.773588in}}%
\pgfpathlineto{\pgfqpoint{2.681331in}{0.773588in}}%
\pgfpathlineto{\pgfqpoint{2.753491in}{0.773588in}}%
\pgfpathlineto{\pgfqpoint{2.826501in}{0.773588in}}%
\pgfpathlineto{\pgfqpoint{2.902674in}{0.773588in}}%
\pgfpathlineto{\pgfqpoint{2.975015in}{0.773588in}}%
\pgfpathlineto{\pgfqpoint{3.047640in}{0.773588in}}%
\pgfpathlineto{\pgfqpoint{3.123114in}{0.773588in}}%
\pgfpathlineto{\pgfqpoint{3.195433in}{0.773588in}}%
\pgfpathlineto{\pgfqpoint{3.277599in}{0.773588in}}%
\pgfpathlineto{\pgfqpoint{3.356465in}{0.773588in}}%
\pgfpathlineto{\pgfqpoint{3.430573in}{0.773588in}}%
\pgfpathlineto{\pgfqpoint{3.503039in}{0.773588in}}%
\pgfpathlineto{\pgfqpoint{3.575543in}{0.773588in}}%
\pgfpathlineto{\pgfqpoint{3.643958in}{0.773588in}}%
\pgfpathlineto{\pgfqpoint{3.712535in}{0.773588in}}%
\pgfpathlineto{\pgfqpoint{3.781205in}{0.773588in}}%
\pgfpathlineto{\pgfqpoint{3.848285in}{0.773588in}}%
\pgfpathlineto{\pgfqpoint{3.914312in}{0.773588in}}%
\pgfpathlineto{\pgfqpoint{3.981945in}{0.773588in}}%
\pgfpathlineto{\pgfqpoint{4.048987in}{0.773588in}}%
\pgfpathlineto{\pgfqpoint{4.116566in}{0.773588in}}%
\pgfpathlineto{\pgfqpoint{4.186150in}{0.773588in}}%
\pgfpathlineto{\pgfqpoint{4.253047in}{0.773588in}}%
\pgfpathlineto{\pgfqpoint{4.319299in}{0.773588in}}%
\pgfpathlineto{\pgfqpoint{4.387175in}{0.773588in}}%
\pgfpathlineto{\pgfqpoint{4.454265in}{0.773588in}}%
\pgfpathlineto{\pgfqpoint{4.520559in}{0.773588in}}%
\pgfpathlineto{\pgfqpoint{4.590503in}{0.773588in}}%
\pgfpathlineto{\pgfqpoint{4.658575in}{0.773588in}}%
\pgfpathlineto{\pgfqpoint{4.726503in}{0.773588in}}%
\pgfpathlineto{\pgfqpoint{4.796564in}{0.773588in}}%
\pgfpathlineto{\pgfqpoint{4.866311in}{0.773588in}}%
\pgfpathlineto{\pgfqpoint{4.935251in}{0.773588in}}%
\pgfpathlineto{\pgfqpoint{5.005143in}{0.773588in}}%
\pgfpathlineto{\pgfqpoint{5.073090in}{0.773588in}}%
\pgfpathlineto{\pgfqpoint{5.142206in}{0.773588in}}%
\pgfpathlineto{\pgfqpoint{5.213575in}{0.773588in}}%
\pgfpathlineto{\pgfqpoint{5.283251in}{0.773588in}}%
\pgfpathlineto{\pgfqpoint{5.353942in}{0.773588in}}%
\pgfpathlineto{\pgfqpoint{5.424203in}{0.773588in}}%
\pgfpathlineto{\pgfqpoint{5.492693in}{0.773588in}}%
\pgfpathlineto{\pgfqpoint{5.561464in}{0.773588in}}%
\pgfpathlineto{\pgfqpoint{5.633136in}{0.773588in}}%
\pgfpathlineto{\pgfqpoint{5.700467in}{0.773588in}}%
\pgfpathlineto{\pgfqpoint{5.766709in}{0.773588in}}%
\pgfpathlineto{\pgfqpoint{5.834669in}{0.773588in}}%
\pgfpathlineto{\pgfqpoint{5.901623in}{0.773588in}}%
\pgfpathlineto{\pgfqpoint{5.968494in}{0.773588in}}%
\pgfpathlineto{\pgfqpoint{6.036866in}{0.773588in}}%
\pgfpathlineto{\pgfqpoint{6.103260in}{0.773588in}}%
\pgfpathlineto{\pgfqpoint{6.169451in}{0.773588in}}%
\pgfpathlineto{\pgfqpoint{6.236861in}{0.773588in}}%
\pgfpathlineto{\pgfqpoint{6.303479in}{0.773588in}}%
\pgfpathlineto{\pgfqpoint{6.369842in}{0.773588in}}%
\pgfpathlineto{\pgfqpoint{6.437563in}{0.773588in}}%
\pgfpathlineto{\pgfqpoint{6.502885in}{0.773588in}}%
\pgfpathlineto{\pgfqpoint{6.568004in}{0.773588in}}%
\pgfpathlineto{\pgfqpoint{6.636218in}{0.773588in}}%
\pgfpathlineto{\pgfqpoint{6.704271in}{0.773588in}}%
\pgfpathlineto{\pgfqpoint{6.770581in}{0.773588in}}%
\pgfpathlineto{\pgfqpoint{6.839348in}{0.773588in}}%
\pgfpathlineto{\pgfqpoint{6.906544in}{0.773588in}}%
\pgfpathlineto{\pgfqpoint{6.973333in}{0.773588in}}%
\pgfpathlineto{\pgfqpoint{7.040805in}{0.773588in}}%
\pgfpathlineto{\pgfqpoint{7.106301in}{0.773588in}}%
\pgfpathlineto{\pgfqpoint{7.171034in}{0.773588in}}%
\pgfpathlineto{\pgfqpoint{7.238831in}{0.773588in}}%
\pgfpathlineto{\pgfqpoint{7.308295in}{0.773588in}}%
\pgfpathlineto{\pgfqpoint{7.379810in}{0.773588in}}%
\pgfpathlineto{\pgfqpoint{7.455306in}{0.773588in}}%
\pgfpathlineto{\pgfqpoint{7.527107in}{0.773588in}}%
\pgfpathlineto{\pgfqpoint{7.598690in}{0.773588in}}%
\pgfpathlineto{\pgfqpoint{7.671131in}{0.773588in}}%
\pgfpathlineto{\pgfqpoint{7.741421in}{0.773588in}}%
\pgfpathlineto{\pgfqpoint{7.812286in}{0.773588in}}%
\pgfpathlineto{\pgfqpoint{7.887441in}{0.773588in}}%
\pgfpathlineto{\pgfqpoint{7.957986in}{0.773588in}}%
\pgfpathlineto{\pgfqpoint{8.028258in}{0.773588in}}%
\pgfpathlineto{\pgfqpoint{8.100457in}{0.773588in}}%
\pgfpathlineto{\pgfqpoint{8.168934in}{0.773588in}}%
\pgfpathlineto{\pgfqpoint{8.237831in}{0.773588in}}%
\pgfpathlineto{\pgfqpoint{8.308718in}{0.773588in}}%
\pgfpathlineto{\pgfqpoint{8.376454in}{0.773588in}}%
\pgfpathlineto{\pgfqpoint{8.444285in}{0.773588in}}%
\pgfpathlineto{\pgfqpoint{8.515085in}{0.773588in}}%
\pgfpathlineto{\pgfqpoint{8.583601in}{0.773588in}}%
\pgfpathlineto{\pgfqpoint{8.651757in}{0.773588in}}%
\pgfpathlineto{\pgfqpoint{8.720615in}{0.773588in}}%
\pgfpathlineto{\pgfqpoint{8.787638in}{0.773588in}}%
\pgfpathlineto{\pgfqpoint{8.856098in}{0.773588in}}%
\pgfpathlineto{\pgfqpoint{8.924860in}{0.773588in}}%
\pgfpathlineto{\pgfqpoint{8.991646in}{0.773588in}}%
\pgfpathlineto{\pgfqpoint{9.059607in}{0.773588in}}%
\pgfpathlineto{\pgfqpoint{9.130518in}{0.773588in}}%
\pgfpathlineto{\pgfqpoint{9.198477in}{0.773588in}}%
\pgfpathlineto{\pgfqpoint{9.265262in}{0.773588in}}%
\pgfpathlineto{\pgfqpoint{9.334728in}{0.773588in}}%
\pgfpathlineto{\pgfqpoint{9.402138in}{0.773588in}}%
\pgfpathlineto{\pgfqpoint{9.467314in}{0.773588in}}%
\pgfpathlineto{\pgfqpoint{9.536145in}{0.773588in}}%
\pgfpathlineto{\pgfqpoint{9.603748in}{0.773588in}}%
\pgfpathlineto{\pgfqpoint{9.670768in}{0.773588in}}%
\pgfpathlineto{\pgfqpoint{9.740273in}{0.773588in}}%
\pgfpathlineto{\pgfqpoint{9.808815in}{0.773588in}}%
\pgfpathlineto{\pgfqpoint{9.876148in}{0.773588in}}%
\pgfpathlineto{\pgfqpoint{9.944615in}{0.773588in}}%
\pgfpathlineto{\pgfqpoint{10.012687in}{0.773588in}}%
\pgfpathlineto{\pgfqpoint{10.081602in}{0.773588in}}%
\pgfpathlineto{\pgfqpoint{10.153891in}{0.773588in}}%
\pgfpathlineto{\pgfqpoint{10.222102in}{0.773588in}}%
\pgfpathlineto{\pgfqpoint{10.290696in}{0.773588in}}%
\pgfpathlineto{\pgfqpoint{10.360853in}{0.773588in}}%
\pgfpathlineto{\pgfqpoint{10.428999in}{0.773588in}}%
\pgfpathlineto{\pgfqpoint{10.498528in}{0.773588in}}%
\pgfpathlineto{\pgfqpoint{10.569898in}{0.773588in}}%
\pgfpathlineto{\pgfqpoint{10.639006in}{0.773588in}}%
\pgfpathlineto{\pgfqpoint{10.708017in}{0.773588in}}%
\pgfpathlineto{\pgfqpoint{10.777925in}{0.773588in}}%
\pgfpathlineto{\pgfqpoint{10.846851in}{0.773588in}}%
\pgfpathlineto{\pgfqpoint{10.916008in}{0.773588in}}%
\pgfpathlineto{\pgfqpoint{10.986941in}{0.773588in}}%
\pgfpathlineto{\pgfqpoint{11.055389in}{0.773588in}}%
\pgfpathlineto{\pgfqpoint{11.125133in}{0.773588in}}%
\pgfpathlineto{\pgfqpoint{11.194984in}{0.773588in}}%
\pgfpathlineto{\pgfqpoint{11.261777in}{0.773588in}}%
\pgfpathlineto{\pgfqpoint{11.328677in}{0.773588in}}%
\pgfpathlineto{\pgfqpoint{11.398136in}{0.773588in}}%
\pgfpathlineto{\pgfqpoint{11.465294in}{0.773588in}}%
\pgfpathlineto{\pgfqpoint{11.532877in}{0.773588in}}%
\pgfpathlineto{\pgfqpoint{11.602395in}{0.773588in}}%
\pgfpathlineto{\pgfqpoint{11.669969in}{0.773588in}}%
\pgfpathlineto{\pgfqpoint{11.736702in}{0.773588in}}%
\pgfpathlineto{\pgfqpoint{11.804695in}{0.773588in}}%
\pgfpathlineto{\pgfqpoint{11.871239in}{0.773588in}}%
\pgfpathlineto{\pgfqpoint{11.938397in}{0.773588in}}%
\pgfpathlineto{\pgfqpoint{12.008843in}{0.773588in}}%
\pgfpathlineto{\pgfqpoint{12.076638in}{0.773588in}}%
\pgfpathlineto{\pgfqpoint{12.144815in}{0.773588in}}%
\pgfpathlineto{\pgfqpoint{12.213872in}{0.773588in}}%
\pgfpathlineto{\pgfqpoint{12.280761in}{0.773588in}}%
\pgfpathlineto{\pgfqpoint{12.347008in}{0.773588in}}%
\pgfpathlineto{\pgfqpoint{12.416448in}{0.773588in}}%
\pgfpathlineto{\pgfqpoint{12.484090in}{0.773588in}}%
\pgfpathlineto{\pgfqpoint{12.552597in}{0.773588in}}%
\pgfpathlineto{\pgfqpoint{12.623527in}{0.773588in}}%
\pgfpathlineto{\pgfqpoint{12.692271in}{0.773588in}}%
\pgfpathlineto{\pgfqpoint{12.759967in}{0.773588in}}%
\pgfpathlineto{\pgfqpoint{12.830499in}{0.773588in}}%
\pgfpathlineto{\pgfqpoint{12.900232in}{0.773588in}}%
\pgfpathlineto{\pgfqpoint{12.969036in}{0.773588in}}%
\pgfpathlineto{\pgfqpoint{13.041078in}{0.773588in}}%
\pgfpathlineto{\pgfqpoint{13.109710in}{0.773588in}}%
\pgfpathlineto{\pgfqpoint{13.178279in}{0.773588in}}%
\pgfpathlineto{\pgfqpoint{13.250526in}{0.773588in}}%
\pgfpathlineto{\pgfqpoint{13.319414in}{0.773588in}}%
\pgfpathlineto{\pgfqpoint{13.387091in}{0.773588in}}%
\pgfpathlineto{\pgfqpoint{13.457833in}{0.773588in}}%
\pgfpathlineto{\pgfqpoint{13.526360in}{0.773588in}}%
\pgfpathlineto{\pgfqpoint{13.595052in}{0.773588in}}%
\pgfpathlineto{\pgfqpoint{13.666880in}{0.773588in}}%
\pgfpathlineto{\pgfqpoint{13.735499in}{0.773588in}}%
\pgfpathlineto{\pgfqpoint{13.804228in}{0.773588in}}%
\pgfpathlineto{\pgfqpoint{13.876785in}{0.773588in}}%
\pgfpathlineto{\pgfqpoint{13.944783in}{0.773588in}}%
\pgfpathlineto{\pgfqpoint{14.012639in}{0.773588in}}%
\pgfpathlineto{\pgfqpoint{14.081564in}{0.773588in}}%
\pgfpathlineto{\pgfqpoint{14.149562in}{0.773588in}}%
\pgfpathlineto{\pgfqpoint{14.216102in}{0.773588in}}%
\pgfpathlineto{\pgfqpoint{14.286685in}{0.773588in}}%
\pgfpathlineto{\pgfqpoint{14.355979in}{0.773588in}}%
\pgfpathlineto{\pgfqpoint{14.424138in}{0.773588in}}%
\pgfpathlineto{\pgfqpoint{14.494110in}{0.773588in}}%
\pgfpathlineto{\pgfqpoint{14.562284in}{0.773588in}}%
\pgfpathlineto{\pgfqpoint{14.630523in}{0.773588in}}%
\pgfpathlineto{\pgfqpoint{14.700003in}{0.773588in}}%
\pgfpathlineto{\pgfqpoint{14.768225in}{0.773588in}}%
\pgfpathlineto{\pgfqpoint{14.836607in}{0.773588in}}%
\pgfpathlineto{\pgfqpoint{14.908199in}{0.773588in}}%
\pgfpathlineto{\pgfqpoint{14.976372in}{0.773588in}}%
\pgfpathlineto{\pgfqpoint{15.044439in}{0.773588in}}%
\pgfpathlineto{\pgfqpoint{15.115140in}{0.773588in}}%
\pgfpathlineto{\pgfqpoint{15.182512in}{0.773588in}}%
\pgfpathlineto{\pgfqpoint{15.249597in}{0.773588in}}%
\pgfpathlineto{\pgfqpoint{15.320168in}{0.773588in}}%
\pgfpathlineto{\pgfqpoint{15.388582in}{0.773588in}}%
\pgfpathlineto{\pgfqpoint{15.457342in}{0.773588in}}%
\pgfpathlineto{\pgfqpoint{15.527961in}{0.773588in}}%
\pgfpathlineto{\pgfqpoint{15.595878in}{0.773588in}}%
\pgfpathlineto{\pgfqpoint{15.665480in}{0.773588in}}%
\pgfpathlineto{\pgfqpoint{15.737902in}{0.773588in}}%
\pgfpathlineto{\pgfqpoint{15.807467in}{0.773588in}}%
\pgfpathlineto{\pgfqpoint{15.878678in}{0.773588in}}%
\pgfpathlineto{\pgfqpoint{15.949677in}{0.773588in}}%
\pgfpathlineto{\pgfqpoint{16.019598in}{0.773588in}}%
\pgfpathlineto{\pgfqpoint{16.088678in}{0.773588in}}%
\pgfpathlineto{\pgfqpoint{16.160188in}{0.773588in}}%
\pgfpathlineto{\pgfqpoint{16.229236in}{0.773588in}}%
\pgfpathlineto{\pgfqpoint{16.299236in}{0.773588in}}%
\pgfpathlineto{\pgfqpoint{16.373054in}{0.773588in}}%
\pgfpathlineto{\pgfqpoint{16.445981in}{0.773588in}}%
\pgfpathlineto{\pgfqpoint{16.519393in}{0.773588in}}%
\pgfpathlineto{\pgfqpoint{16.592371in}{0.773588in}}%
\pgfpathlineto{\pgfqpoint{16.662798in}{0.773588in}}%
\pgfpathlineto{\pgfqpoint{16.731829in}{0.773588in}}%
\pgfpathlineto{\pgfqpoint{16.803402in}{0.773588in}}%
\pgfpathlineto{\pgfqpoint{16.872974in}{0.773588in}}%
\pgfpathlineto{\pgfqpoint{16.942299in}{0.773588in}}%
\pgfpathlineto{\pgfqpoint{17.013249in}{0.773588in}}%
\pgfpathlineto{\pgfqpoint{17.081113in}{0.773588in}}%
\pgfpathlineto{\pgfqpoint{17.150280in}{0.773588in}}%
\pgfpathlineto{\pgfqpoint{17.220960in}{0.773588in}}%
\pgfpathlineto{\pgfqpoint{17.288800in}{0.773588in}}%
\pgfpathlineto{\pgfqpoint{17.356523in}{0.773588in}}%
\pgfpathlineto{\pgfqpoint{17.426570in}{0.773588in}}%
\pgfpathlineto{\pgfqpoint{17.494859in}{0.773588in}}%
\pgfpathlineto{\pgfqpoint{17.563254in}{0.773588in}}%
\pgfpathlineto{\pgfqpoint{17.634782in}{0.773588in}}%
\pgfpathlineto{\pgfqpoint{17.703480in}{0.773588in}}%
\pgfpathlineto{\pgfqpoint{17.772898in}{0.773588in}}%
\pgfpathlineto{\pgfqpoint{17.844148in}{0.773588in}}%
\pgfpathlineto{\pgfqpoint{17.912627in}{0.773588in}}%
\pgfpathlineto{\pgfqpoint{17.983524in}{0.773588in}}%
\pgfpathlineto{\pgfqpoint{18.056086in}{0.773588in}}%
\pgfpathlineto{\pgfqpoint{18.125960in}{0.773588in}}%
\pgfpathlineto{\pgfqpoint{18.195268in}{0.773588in}}%
\pgfpathlineto{\pgfqpoint{18.266780in}{0.773588in}}%
\pgfpathlineto{\pgfqpoint{18.336223in}{0.773588in}}%
\pgfpathlineto{\pgfqpoint{18.406899in}{0.773588in}}%
\pgfpathlineto{\pgfqpoint{18.479143in}{0.773588in}}%
\pgfpathlineto{\pgfqpoint{18.549509in}{0.773588in}}%
\pgfpathlineto{\pgfqpoint{18.619849in}{0.773588in}}%
\pgfpathlineto{\pgfqpoint{18.692839in}{0.773588in}}%
\pgfpathlineto{\pgfqpoint{18.763558in}{0.773588in}}%
\pgfpathlineto{\pgfqpoint{18.834141in}{0.773588in}}%
\pgfpathlineto{\pgfqpoint{18.906958in}{0.773588in}}%
\pgfpathlineto{\pgfqpoint{18.977585in}{0.773588in}}%
\pgfpathlineto{\pgfqpoint{19.048835in}{0.773588in}}%
\pgfpathlineto{\pgfqpoint{19.123240in}{0.773588in}}%
\pgfpathlineto{\pgfqpoint{19.195134in}{0.773588in}}%
\pgfpathlineto{\pgfqpoint{19.266961in}{0.773588in}}%
\pgfpathlineto{\pgfqpoint{19.340700in}{0.773588in}}%
\pgfpathlineto{\pgfqpoint{19.412352in}{0.773588in}}%
\pgfpathlineto{\pgfqpoint{19.482791in}{0.773588in}}%
\pgfpathlineto{\pgfqpoint{19.553641in}{0.773588in}}%
\pgfpathlineto{\pgfqpoint{19.623855in}{0.773588in}}%
\pgfpathlineto{\pgfqpoint{19.693765in}{0.773588in}}%
\pgfpathlineto{\pgfqpoint{19.766060in}{0.773588in}}%
\pgfpathlineto{\pgfqpoint{19.835928in}{0.773588in}}%
\pgfpathlineto{\pgfqpoint{19.907049in}{0.773588in}}%
\pgfpathlineto{\pgfqpoint{19.981199in}{0.773588in}}%
\pgfpathlineto{\pgfqpoint{20.052071in}{0.773588in}}%
\pgfpathlineto{\pgfqpoint{20.121885in}{0.773588in}}%
\pgfpathlineto{\pgfqpoint{20.193561in}{0.773588in}}%
\pgfpathlineto{\pgfqpoint{20.263963in}{0.773588in}}%
\pgfpathlineto{\pgfqpoint{20.334605in}{0.773588in}}%
\pgfpathlineto{\pgfqpoint{20.407276in}{0.773588in}}%
\pgfpathlineto{\pgfqpoint{20.476961in}{0.773588in}}%
\pgfpathlineto{\pgfqpoint{20.547172in}{0.773588in}}%
\pgfpathlineto{\pgfqpoint{20.618428in}{0.773588in}}%
\pgfpathlineto{\pgfqpoint{20.688980in}{0.773588in}}%
\pgfpathlineto{\pgfqpoint{20.758814in}{0.773588in}}%
\pgfpathlineto{\pgfqpoint{20.830586in}{0.773588in}}%
\pgfpathlineto{\pgfqpoint{20.899587in}{0.773588in}}%
\pgfpathlineto{\pgfqpoint{20.969373in}{0.773588in}}%
\pgfpathlineto{\pgfqpoint{21.040864in}{0.773588in}}%
\pgfpathlineto{\pgfqpoint{21.110656in}{0.773588in}}%
\pgfpathlineto{\pgfqpoint{21.181233in}{0.773588in}}%
\pgfpathlineto{\pgfqpoint{21.254645in}{0.773588in}}%
\pgfpathlineto{\pgfqpoint{21.324498in}{0.773588in}}%
\pgfpathlineto{\pgfqpoint{21.394839in}{0.773588in}}%
\pgfpathlineto{\pgfqpoint{21.467741in}{0.773588in}}%
\pgfpathlineto{\pgfqpoint{21.539601in}{0.773588in}}%
\pgfpathlineto{\pgfqpoint{21.610878in}{0.773588in}}%
\pgfpathlineto{\pgfqpoint{21.683994in}{0.773588in}}%
\pgfpathlineto{\pgfqpoint{21.756227in}{0.773588in}}%
\pgfpathlineto{\pgfqpoint{21.828555in}{0.773588in}}%
\pgfpathlineto{\pgfqpoint{21.903868in}{0.773588in}}%
\pgfpathlineto{\pgfqpoint{21.976836in}{0.773588in}}%
\pgfpathlineto{\pgfqpoint{22.048040in}{0.773588in}}%
\pgfpathlineto{\pgfqpoint{22.122462in}{0.773588in}}%
\pgfpathlineto{\pgfqpoint{22.195707in}{0.773588in}}%
\pgfpathlineto{\pgfqpoint{22.268824in}{0.773588in}}%
\pgfpathlineto{\pgfqpoint{22.343331in}{0.773588in}}%
\pgfpathlineto{\pgfqpoint{22.413449in}{0.773588in}}%
\pgfpathlineto{\pgfqpoint{22.482516in}{0.773588in}}%
\pgfpathlineto{\pgfqpoint{22.553535in}{0.773588in}}%
\pgfpathlineto{\pgfqpoint{22.624114in}{0.773588in}}%
\pgfpathlineto{\pgfqpoint{22.694016in}{0.773588in}}%
\pgfpathlineto{\pgfqpoint{22.764651in}{0.773588in}}%
\pgfpathlineto{\pgfqpoint{22.833628in}{0.773588in}}%
\pgfpathlineto{\pgfqpoint{22.902896in}{0.773588in}}%
\pgfpathlineto{\pgfqpoint{22.973868in}{0.773588in}}%
\pgfpathlineto{\pgfqpoint{23.043397in}{0.773588in}}%
\pgfpathlineto{\pgfqpoint{23.113267in}{0.773588in}}%
\pgfpathlineto{\pgfqpoint{23.184270in}{0.773588in}}%
\pgfpathlineto{\pgfqpoint{23.253484in}{0.773588in}}%
\pgfpathlineto{\pgfqpoint{23.323995in}{0.773588in}}%
\pgfpathlineto{\pgfqpoint{23.396126in}{0.773588in}}%
\pgfpathlineto{\pgfqpoint{23.467323in}{0.773588in}}%
\pgfpathlineto{\pgfqpoint{23.537850in}{0.773588in}}%
\pgfpathlineto{\pgfqpoint{23.610036in}{0.773588in}}%
\pgfpathlineto{\pgfqpoint{23.681448in}{0.773588in}}%
\pgfpathlineto{\pgfqpoint{23.752361in}{0.773588in}}%
\pgfpathlineto{\pgfqpoint{23.824951in}{0.773588in}}%
\pgfpathlineto{\pgfqpoint{23.895213in}{0.773588in}}%
\pgfpathlineto{\pgfqpoint{23.966722in}{0.773588in}}%
\pgfpathlineto{\pgfqpoint{24.039255in}{0.773588in}}%
\pgfpathlineto{\pgfqpoint{24.111784in}{0.773588in}}%
\pgfpathlineto{\pgfqpoint{24.183899in}{0.773588in}}%
\pgfpathlineto{\pgfqpoint{24.257307in}{0.773588in}}%
\pgfpathlineto{\pgfqpoint{24.329090in}{0.773588in}}%
\pgfpathlineto{\pgfqpoint{24.400328in}{0.773588in}}%
\pgfpathlineto{\pgfqpoint{24.476339in}{0.773588in}}%
\pgfpathlineto{\pgfqpoint{24.548639in}{0.773588in}}%
\pgfpathlineto{\pgfqpoint{24.618678in}{0.773588in}}%
\pgfpathlineto{\pgfqpoint{24.691660in}{0.773588in}}%
\pgfpathlineto{\pgfqpoint{24.764742in}{0.773588in}}%
\pgfpathlineto{\pgfqpoint{24.836992in}{0.773588in}}%
\pgfpathlineto{\pgfqpoint{24.911741in}{0.773588in}}%
\pgfpathlineto{\pgfqpoint{24.983525in}{0.773588in}}%
\pgfpathlineto{\pgfqpoint{25.055567in}{0.773588in}}%
\pgfpathlineto{\pgfqpoint{25.131109in}{0.773588in}}%
\pgfpathlineto{\pgfqpoint{25.203216in}{0.773588in}}%
\pgfpathlineto{\pgfqpoint{25.273349in}{0.773588in}}%
\pgfpathlineto{\pgfqpoint{25.347124in}{0.773588in}}%
\pgfpathlineto{\pgfqpoint{25.417047in}{0.773588in}}%
\pgfpathlineto{\pgfqpoint{25.487573in}{0.773588in}}%
\pgfpathlineto{\pgfqpoint{25.560110in}{0.773588in}}%
\pgfpathlineto{\pgfqpoint{25.631022in}{0.773588in}}%
\pgfpathlineto{\pgfqpoint{25.702341in}{0.773588in}}%
\pgfpathlineto{\pgfqpoint{25.775695in}{0.773588in}}%
\pgfpathlineto{\pgfqpoint{25.845667in}{0.773588in}}%
\pgfpathlineto{\pgfqpoint{25.916551in}{0.773588in}}%
\pgfpathlineto{\pgfqpoint{25.988588in}{0.773588in}}%
\pgfpathlineto{\pgfqpoint{26.058621in}{0.773588in}}%
\pgfpathlineto{\pgfqpoint{26.130346in}{0.773588in}}%
\pgfpathlineto{\pgfqpoint{26.203572in}{0.773588in}}%
\pgfpathlineto{\pgfqpoint{26.274267in}{0.773588in}}%
\pgfpathlineto{\pgfqpoint{26.344920in}{0.773588in}}%
\pgfpathlineto{\pgfqpoint{26.417231in}{0.773588in}}%
\pgfpathlineto{\pgfqpoint{26.487420in}{0.773588in}}%
\pgfpathlineto{\pgfqpoint{26.557235in}{0.773588in}}%
\pgfpathlineto{\pgfqpoint{26.629572in}{0.773588in}}%
\pgfpathlineto{\pgfqpoint{26.699584in}{0.773588in}}%
\pgfpathlineto{\pgfqpoint{26.769271in}{0.773588in}}%
\pgfpathlineto{\pgfqpoint{26.841234in}{0.773588in}}%
\pgfpathlineto{\pgfqpoint{26.912667in}{0.773588in}}%
\pgfpathlineto{\pgfqpoint{26.983641in}{0.773588in}}%
\pgfpathlineto{\pgfqpoint{27.056835in}{0.773588in}}%
\pgfpathlineto{\pgfqpoint{27.128948in}{0.773588in}}%
\pgfpathlineto{\pgfqpoint{27.201477in}{0.773588in}}%
\pgfpathlineto{\pgfqpoint{27.277488in}{0.773588in}}%
\pgfpathlineto{\pgfqpoint{27.350990in}{0.773588in}}%
\pgfpathlineto{\pgfqpoint{27.423884in}{0.773588in}}%
\pgfpathlineto{\pgfqpoint{27.500063in}{0.773588in}}%
\pgfpathlineto{\pgfqpoint{27.574929in}{0.773588in}}%
\pgfpathlineto{\pgfqpoint{27.649072in}{0.773588in}}%
\pgfpathlineto{\pgfqpoint{27.724006in}{0.773588in}}%
\pgfpathlineto{\pgfqpoint{27.795343in}{0.773588in}}%
\pgfpathlineto{\pgfqpoint{27.868117in}{0.773588in}}%
\pgfpathlineto{\pgfqpoint{27.943911in}{0.773588in}}%
\pgfpathlineto{\pgfqpoint{28.018234in}{0.773588in}}%
\pgfpathlineto{\pgfqpoint{28.090360in}{0.773588in}}%
\pgfpathlineto{\pgfqpoint{28.163352in}{0.773588in}}%
\pgfpathlineto{\pgfqpoint{28.234559in}{0.773588in}}%
\pgfpathlineto{\pgfqpoint{28.306604in}{0.773588in}}%
\pgfpathlineto{\pgfqpoint{28.380501in}{0.773588in}}%
\pgfpathlineto{\pgfqpoint{28.451943in}{0.773588in}}%
\pgfpathlineto{\pgfqpoint{28.522534in}{0.773588in}}%
\pgfpathlineto{\pgfqpoint{28.596041in}{0.773588in}}%
\pgfpathlineto{\pgfqpoint{28.668204in}{0.773588in}}%
\pgfpathlineto{\pgfqpoint{28.738605in}{0.773588in}}%
\pgfpathlineto{\pgfqpoint{28.811911in}{0.773588in}}%
\pgfpathlineto{\pgfqpoint{28.885217in}{0.773588in}}%
\pgfpathlineto{\pgfqpoint{28.956832in}{0.773588in}}%
\pgfpathlineto{\pgfqpoint{29.029887in}{0.773588in}}%
\pgfpathlineto{\pgfqpoint{29.100748in}{0.773588in}}%
\pgfpathlineto{\pgfqpoint{29.173201in}{0.773588in}}%
\pgfpathlineto{\pgfqpoint{29.248973in}{0.773588in}}%
\pgfpathlineto{\pgfqpoint{29.320759in}{0.773588in}}%
\pgfpathlineto{\pgfqpoint{29.393660in}{0.773588in}}%
\pgfpathlineto{\pgfqpoint{29.467898in}{0.773588in}}%
\pgfpathlineto{\pgfqpoint{29.540420in}{0.773588in}}%
\pgfpathlineto{\pgfqpoint{29.611700in}{0.773588in}}%
\pgfpathlineto{\pgfqpoint{29.684427in}{0.773588in}}%
\pgfpathlineto{\pgfqpoint{29.755113in}{0.773588in}}%
\pgfpathlineto{\pgfqpoint{29.827132in}{0.773588in}}%
\pgfpathlineto{\pgfqpoint{29.901656in}{0.773588in}}%
\pgfpathlineto{\pgfqpoint{29.974646in}{0.773588in}}%
\pgfpathlineto{\pgfqpoint{30.048252in}{0.773588in}}%
\pgfpathlineto{\pgfqpoint{30.122796in}{0.773588in}}%
\pgfpathlineto{\pgfqpoint{30.195443in}{0.773588in}}%
\pgfpathlineto{\pgfqpoint{30.269036in}{0.773588in}}%
\pgfpathlineto{\pgfqpoint{30.344328in}{0.773588in}}%
\pgfpathlineto{\pgfqpoint{30.417098in}{0.773588in}}%
\pgfpathlineto{\pgfqpoint{30.488991in}{0.773588in}}%
\pgfpathlineto{\pgfqpoint{30.562714in}{0.773588in}}%
\pgfpathlineto{\pgfqpoint{30.634099in}{0.773588in}}%
\pgfpathlineto{\pgfqpoint{30.707828in}{0.773588in}}%
\pgfpathlineto{\pgfqpoint{30.782249in}{0.773588in}}%
\pgfpathlineto{\pgfqpoint{30.854115in}{0.773588in}}%
\pgfpathlineto{\pgfqpoint{30.928305in}{0.773588in}}%
\pgfpathlineto{\pgfqpoint{31.002514in}{0.773588in}}%
\pgfpathlineto{\pgfqpoint{31.074452in}{0.773588in}}%
\pgfpathlineto{\pgfqpoint{31.147740in}{0.773588in}}%
\pgfpathlineto{\pgfqpoint{31.222913in}{0.773588in}}%
\pgfpathlineto{\pgfqpoint{31.294777in}{0.773588in}}%
\pgfpathlineto{\pgfqpoint{31.366613in}{0.773588in}}%
\pgfpathlineto{\pgfqpoint{31.439415in}{0.773588in}}%
\pgfpathlineto{\pgfqpoint{31.510140in}{0.773588in}}%
\pgfpathlineto{\pgfqpoint{31.582282in}{0.773588in}}%
\pgfpathlineto{\pgfqpoint{31.656180in}{0.773588in}}%
\pgfpathlineto{\pgfqpoint{31.728521in}{0.773588in}}%
\pgfpathlineto{\pgfqpoint{31.800877in}{0.773588in}}%
\pgfpathlineto{\pgfqpoint{31.873539in}{0.773588in}}%
\pgfpathlineto{\pgfqpoint{31.943734in}{0.773588in}}%
\pgfpathlineto{\pgfqpoint{32.015122in}{0.773588in}}%
\pgfpathlineto{\pgfqpoint{32.089684in}{0.773588in}}%
\pgfpathlineto{\pgfqpoint{32.161504in}{0.773588in}}%
\pgfpathlineto{\pgfqpoint{32.231773in}{0.773588in}}%
\pgfpathlineto{\pgfqpoint{32.305440in}{0.773588in}}%
\pgfpathlineto{\pgfqpoint{32.377016in}{0.773588in}}%
\pgfpathlineto{\pgfqpoint{32.447439in}{0.773588in}}%
\pgfpathlineto{\pgfqpoint{32.520401in}{0.773588in}}%
\pgfpathlineto{\pgfqpoint{32.590674in}{0.773588in}}%
\pgfpathlineto{\pgfqpoint{32.663709in}{0.773588in}}%
\pgfpathlineto{\pgfqpoint{32.740263in}{0.773588in}}%
\pgfpathlineto{\pgfqpoint{32.813546in}{0.773588in}}%
\pgfpathlineto{\pgfqpoint{32.887492in}{0.773588in}}%
\pgfpathlineto{\pgfqpoint{32.963168in}{0.773588in}}%
\pgfpathlineto{\pgfqpoint{33.037794in}{0.773588in}}%
\pgfpathlineto{\pgfqpoint{33.110479in}{0.773588in}}%
\pgfpathlineto{\pgfqpoint{33.185787in}{0.773588in}}%
\pgfpathlineto{\pgfqpoint{33.259507in}{0.773588in}}%
\pgfpathlineto{\pgfqpoint{33.333311in}{0.773588in}}%
\pgfpathlineto{\pgfqpoint{33.409286in}{0.773588in}}%
\pgfpathlineto{\pgfqpoint{33.483328in}{0.773588in}}%
\pgfpathlineto{\pgfqpoint{33.557012in}{0.773588in}}%
\pgfpathlineto{\pgfqpoint{33.631884in}{0.773588in}}%
\pgfpathlineto{\pgfqpoint{33.703848in}{0.773588in}}%
\pgfpathlineto{\pgfqpoint{33.776888in}{0.773588in}}%
\pgfpathlineto{\pgfqpoint{33.852393in}{0.773588in}}%
\pgfpathlineto{\pgfqpoint{33.923536in}{0.773588in}}%
\pgfpathlineto{\pgfqpoint{33.994648in}{0.773588in}}%
\pgfpathlineto{\pgfqpoint{34.067999in}{0.773588in}}%
\pgfpathlineto{\pgfqpoint{34.138346in}{0.773588in}}%
\pgfpathlineto{\pgfqpoint{34.210760in}{0.773588in}}%
\pgfpathlineto{\pgfqpoint{34.284339in}{0.773588in}}%
\pgfpathlineto{\pgfqpoint{34.354648in}{0.773588in}}%
\pgfpathlineto{\pgfqpoint{34.425604in}{0.773588in}}%
\pgfpathlineto{\pgfqpoint{34.499162in}{0.773588in}}%
\pgfpathlineto{\pgfqpoint{34.571449in}{0.773588in}}%
\pgfpathlineto{\pgfqpoint{34.643977in}{0.773588in}}%
\pgfpathlineto{\pgfqpoint{34.718731in}{0.773588in}}%
\pgfpathlineto{\pgfqpoint{34.789698in}{0.773588in}}%
\pgfpathlineto{\pgfqpoint{34.862212in}{0.773588in}}%
\pgfpathlineto{\pgfqpoint{34.936943in}{0.773588in}}%
\pgfpathlineto{\pgfqpoint{35.007838in}{0.773588in}}%
\pgfpathlineto{\pgfqpoint{35.080154in}{0.773588in}}%
\pgfpathlineto{\pgfqpoint{35.155466in}{0.773588in}}%
\pgfpathlineto{\pgfqpoint{35.227201in}{0.773588in}}%
\pgfpathlineto{\pgfqpoint{35.298174in}{0.773588in}}%
\pgfpathlineto{\pgfqpoint{35.372990in}{0.773588in}}%
\pgfpathlineto{\pgfqpoint{35.451774in}{0.773588in}}%
\pgfpathlineto{\pgfqpoint{35.574549in}{0.773588in}}%
\pgfpathlineto{\pgfqpoint{35.663523in}{0.773588in}}%
\pgfpathlineto{\pgfqpoint{35.741519in}{0.773588in}}%
\pgfpathlineto{\pgfqpoint{35.805568in}{1.249553in}}%
\pgfpathlineto{\pgfqpoint{35.870813in}{2.216022in}}%
\pgfpathlineto{\pgfqpoint{35.942832in}{2.269015in}}%
\pgfpathlineto{\pgfqpoint{36.012796in}{2.321125in}}%
\pgfpathlineto{\pgfqpoint{36.085094in}{2.286938in}}%
\pgfpathlineto{\pgfqpoint{36.154695in}{2.303111in}}%
\pgfpathlineto{\pgfqpoint{36.223624in}{2.332436in}}%
\pgfpathlineto{\pgfqpoint{36.293479in}{2.395888in}}%
\pgfpathlineto{\pgfqpoint{36.360634in}{2.349474in}}%
\pgfpathlineto{\pgfqpoint{36.428206in}{2.435547in}}%
\pgfpathlineto{\pgfqpoint{36.497087in}{2.404583in}}%
\pgfpathlineto{\pgfqpoint{36.563097in}{2.486664in}}%
\pgfpathlineto{\pgfqpoint{36.628950in}{2.376225in}}%
\pgfpathlineto{\pgfqpoint{36.696952in}{2.436993in}}%
\pgfpathlineto{\pgfqpoint{36.761893in}{2.464952in}}%
\pgfpathlineto{\pgfqpoint{36.827337in}{2.434812in}}%
\pgfpathlineto{\pgfqpoint{36.893714in}{2.468865in}}%
\pgfpathlineto{\pgfqpoint{36.957470in}{2.464622in}}%
\pgfpathlineto{\pgfqpoint{37.022217in}{2.426405in}}%
\pgfpathlineto{\pgfqpoint{37.088015in}{2.514045in}}%
\pgfpathlineto{\pgfqpoint{37.151827in}{2.457661in}}%
\pgfpathlineto{\pgfqpoint{37.151827in}{4.172457in}}%
\pgfpathlineto{\pgfqpoint{37.151827in}{4.172457in}}%
\pgfpathlineto{\pgfqpoint{37.088015in}{4.257509in}}%
\pgfpathlineto{\pgfqpoint{37.022217in}{4.115512in}}%
\pgfpathlineto{\pgfqpoint{36.957470in}{4.207643in}}%
\pgfpathlineto{\pgfqpoint{36.893714in}{4.215547in}}%
\pgfpathlineto{\pgfqpoint{36.827337in}{4.095111in}}%
\pgfpathlineto{\pgfqpoint{36.761893in}{4.151117in}}%
\pgfpathlineto{\pgfqpoint{36.696952in}{4.115543in}}%
\pgfpathlineto{\pgfqpoint{36.628950in}{4.085256in}}%
\pgfpathlineto{\pgfqpoint{36.563097in}{4.110835in}}%
\pgfpathlineto{\pgfqpoint{36.497087in}{3.951551in}}%
\pgfpathlineto{\pgfqpoint{36.428206in}{4.007261in}}%
\pgfpathlineto{\pgfqpoint{36.360634in}{4.001718in}}%
\pgfpathlineto{\pgfqpoint{36.293479in}{3.999837in}}%
\pgfpathlineto{\pgfqpoint{36.223624in}{3.948574in}}%
\pgfpathlineto{\pgfqpoint{36.154695in}{3.912811in}}%
\pgfpathlineto{\pgfqpoint{36.085094in}{3.810697in}}%
\pgfpathlineto{\pgfqpoint{36.012796in}{3.977733in}}%
\pgfpathlineto{\pgfqpoint{35.942832in}{3.811738in}}%
\pgfpathlineto{\pgfqpoint{35.870813in}{3.700065in}}%
\pgfpathlineto{\pgfqpoint{35.805568in}{1.591997in}}%
\pgfpathlineto{\pgfqpoint{35.741519in}{0.773588in}}%
\pgfpathlineto{\pgfqpoint{35.663523in}{0.773588in}}%
\pgfpathlineto{\pgfqpoint{35.574549in}{0.773588in}}%
\pgfpathlineto{\pgfqpoint{35.451774in}{0.773588in}}%
\pgfpathlineto{\pgfqpoint{35.372990in}{0.773588in}}%
\pgfpathlineto{\pgfqpoint{35.298174in}{0.773588in}}%
\pgfpathlineto{\pgfqpoint{35.227201in}{0.773588in}}%
\pgfpathlineto{\pgfqpoint{35.155466in}{0.773588in}}%
\pgfpathlineto{\pgfqpoint{35.080154in}{0.773588in}}%
\pgfpathlineto{\pgfqpoint{35.007838in}{0.773588in}}%
\pgfpathlineto{\pgfqpoint{34.936943in}{0.773588in}}%
\pgfpathlineto{\pgfqpoint{34.862212in}{0.773588in}}%
\pgfpathlineto{\pgfqpoint{34.789698in}{0.773588in}}%
\pgfpathlineto{\pgfqpoint{34.718731in}{0.773588in}}%
\pgfpathlineto{\pgfqpoint{34.643977in}{0.773588in}}%
\pgfpathlineto{\pgfqpoint{34.571449in}{0.773588in}}%
\pgfpathlineto{\pgfqpoint{34.499162in}{0.773588in}}%
\pgfpathlineto{\pgfqpoint{34.425604in}{0.773588in}}%
\pgfpathlineto{\pgfqpoint{34.354648in}{0.773588in}}%
\pgfpathlineto{\pgfqpoint{34.284339in}{0.773588in}}%
\pgfpathlineto{\pgfqpoint{34.210760in}{0.773588in}}%
\pgfpathlineto{\pgfqpoint{34.138346in}{0.773588in}}%
\pgfpathlineto{\pgfqpoint{34.067999in}{0.773588in}}%
\pgfpathlineto{\pgfqpoint{33.994648in}{0.773588in}}%
\pgfpathlineto{\pgfqpoint{33.923536in}{0.773588in}}%
\pgfpathlineto{\pgfqpoint{33.852393in}{0.773588in}}%
\pgfpathlineto{\pgfqpoint{33.776888in}{0.773588in}}%
\pgfpathlineto{\pgfqpoint{33.703848in}{0.773588in}}%
\pgfpathlineto{\pgfqpoint{33.631884in}{0.773588in}}%
\pgfpathlineto{\pgfqpoint{33.557012in}{0.773588in}}%
\pgfpathlineto{\pgfqpoint{33.483328in}{0.773588in}}%
\pgfpathlineto{\pgfqpoint{33.409286in}{0.773588in}}%
\pgfpathlineto{\pgfqpoint{33.333311in}{0.773588in}}%
\pgfpathlineto{\pgfqpoint{33.259507in}{0.773588in}}%
\pgfpathlineto{\pgfqpoint{33.185787in}{0.773588in}}%
\pgfpathlineto{\pgfqpoint{33.110479in}{0.773588in}}%
\pgfpathlineto{\pgfqpoint{33.037794in}{0.773588in}}%
\pgfpathlineto{\pgfqpoint{32.963168in}{0.773588in}}%
\pgfpathlineto{\pgfqpoint{32.887492in}{0.773588in}}%
\pgfpathlineto{\pgfqpoint{32.813546in}{0.773588in}}%
\pgfpathlineto{\pgfqpoint{32.740263in}{0.773588in}}%
\pgfpathlineto{\pgfqpoint{32.663709in}{0.773588in}}%
\pgfpathlineto{\pgfqpoint{32.590674in}{0.773588in}}%
\pgfpathlineto{\pgfqpoint{32.520401in}{0.773588in}}%
\pgfpathlineto{\pgfqpoint{32.447439in}{0.773588in}}%
\pgfpathlineto{\pgfqpoint{32.377016in}{0.773588in}}%
\pgfpathlineto{\pgfqpoint{32.305440in}{0.773588in}}%
\pgfpathlineto{\pgfqpoint{32.231773in}{0.773588in}}%
\pgfpathlineto{\pgfqpoint{32.161504in}{0.773588in}}%
\pgfpathlineto{\pgfqpoint{32.089684in}{0.773588in}}%
\pgfpathlineto{\pgfqpoint{32.015122in}{0.773588in}}%
\pgfpathlineto{\pgfqpoint{31.943734in}{0.773588in}}%
\pgfpathlineto{\pgfqpoint{31.873539in}{0.773588in}}%
\pgfpathlineto{\pgfqpoint{31.800877in}{0.773588in}}%
\pgfpathlineto{\pgfqpoint{31.728521in}{0.773588in}}%
\pgfpathlineto{\pgfqpoint{31.656180in}{0.773588in}}%
\pgfpathlineto{\pgfqpoint{31.582282in}{0.773588in}}%
\pgfpathlineto{\pgfqpoint{31.510140in}{0.773588in}}%
\pgfpathlineto{\pgfqpoint{31.439415in}{0.773588in}}%
\pgfpathlineto{\pgfqpoint{31.366613in}{0.773588in}}%
\pgfpathlineto{\pgfqpoint{31.294777in}{0.773588in}}%
\pgfpathlineto{\pgfqpoint{31.222913in}{0.773588in}}%
\pgfpathlineto{\pgfqpoint{31.147740in}{0.773588in}}%
\pgfpathlineto{\pgfqpoint{31.074452in}{0.773588in}}%
\pgfpathlineto{\pgfqpoint{31.002514in}{0.773588in}}%
\pgfpathlineto{\pgfqpoint{30.928305in}{0.773588in}}%
\pgfpathlineto{\pgfqpoint{30.854115in}{0.773588in}}%
\pgfpathlineto{\pgfqpoint{30.782249in}{0.773588in}}%
\pgfpathlineto{\pgfqpoint{30.707828in}{0.773588in}}%
\pgfpathlineto{\pgfqpoint{30.634099in}{0.773588in}}%
\pgfpathlineto{\pgfqpoint{30.562714in}{0.773588in}}%
\pgfpathlineto{\pgfqpoint{30.488991in}{0.773588in}}%
\pgfpathlineto{\pgfqpoint{30.417098in}{0.773588in}}%
\pgfpathlineto{\pgfqpoint{30.344328in}{0.773588in}}%
\pgfpathlineto{\pgfqpoint{30.269036in}{0.773588in}}%
\pgfpathlineto{\pgfqpoint{30.195443in}{0.773588in}}%
\pgfpathlineto{\pgfqpoint{30.122796in}{0.773588in}}%
\pgfpathlineto{\pgfqpoint{30.048252in}{0.773588in}}%
\pgfpathlineto{\pgfqpoint{29.974646in}{0.773588in}}%
\pgfpathlineto{\pgfqpoint{29.901656in}{0.773588in}}%
\pgfpathlineto{\pgfqpoint{29.827132in}{0.773588in}}%
\pgfpathlineto{\pgfqpoint{29.755113in}{0.773588in}}%
\pgfpathlineto{\pgfqpoint{29.684427in}{0.773588in}}%
\pgfpathlineto{\pgfqpoint{29.611700in}{0.773588in}}%
\pgfpathlineto{\pgfqpoint{29.540420in}{0.773588in}}%
\pgfpathlineto{\pgfqpoint{29.467898in}{0.773588in}}%
\pgfpathlineto{\pgfqpoint{29.393660in}{0.773588in}}%
\pgfpathlineto{\pgfqpoint{29.320759in}{0.773588in}}%
\pgfpathlineto{\pgfqpoint{29.248973in}{0.773588in}}%
\pgfpathlineto{\pgfqpoint{29.173201in}{0.773588in}}%
\pgfpathlineto{\pgfqpoint{29.100748in}{0.773588in}}%
\pgfpathlineto{\pgfqpoint{29.029887in}{0.773588in}}%
\pgfpathlineto{\pgfqpoint{28.956832in}{0.773588in}}%
\pgfpathlineto{\pgfqpoint{28.885217in}{0.773588in}}%
\pgfpathlineto{\pgfqpoint{28.811911in}{0.773588in}}%
\pgfpathlineto{\pgfqpoint{28.738605in}{0.773588in}}%
\pgfpathlineto{\pgfqpoint{28.668204in}{0.773588in}}%
\pgfpathlineto{\pgfqpoint{28.596041in}{0.773588in}}%
\pgfpathlineto{\pgfqpoint{28.522534in}{0.773588in}}%
\pgfpathlineto{\pgfqpoint{28.451943in}{0.773588in}}%
\pgfpathlineto{\pgfqpoint{28.380501in}{0.773588in}}%
\pgfpathlineto{\pgfqpoint{28.306604in}{0.773588in}}%
\pgfpathlineto{\pgfqpoint{28.234559in}{0.773588in}}%
\pgfpathlineto{\pgfqpoint{28.163352in}{0.773588in}}%
\pgfpathlineto{\pgfqpoint{28.090360in}{0.773588in}}%
\pgfpathlineto{\pgfqpoint{28.018234in}{0.773588in}}%
\pgfpathlineto{\pgfqpoint{27.943911in}{0.773588in}}%
\pgfpathlineto{\pgfqpoint{27.868117in}{0.773588in}}%
\pgfpathlineto{\pgfqpoint{27.795343in}{0.773588in}}%
\pgfpathlineto{\pgfqpoint{27.724006in}{0.773588in}}%
\pgfpathlineto{\pgfqpoint{27.649072in}{0.773588in}}%
\pgfpathlineto{\pgfqpoint{27.574929in}{0.773588in}}%
\pgfpathlineto{\pgfqpoint{27.500063in}{0.773588in}}%
\pgfpathlineto{\pgfqpoint{27.423884in}{0.773588in}}%
\pgfpathlineto{\pgfqpoint{27.350990in}{0.773588in}}%
\pgfpathlineto{\pgfqpoint{27.277488in}{0.773588in}}%
\pgfpathlineto{\pgfqpoint{27.201477in}{0.773588in}}%
\pgfpathlineto{\pgfqpoint{27.128948in}{0.773588in}}%
\pgfpathlineto{\pgfqpoint{27.056835in}{0.773588in}}%
\pgfpathlineto{\pgfqpoint{26.983641in}{0.773588in}}%
\pgfpathlineto{\pgfqpoint{26.912667in}{0.773588in}}%
\pgfpathlineto{\pgfqpoint{26.841234in}{0.773588in}}%
\pgfpathlineto{\pgfqpoint{26.769271in}{0.773588in}}%
\pgfpathlineto{\pgfqpoint{26.699584in}{0.773588in}}%
\pgfpathlineto{\pgfqpoint{26.629572in}{0.773588in}}%
\pgfpathlineto{\pgfqpoint{26.557235in}{0.773588in}}%
\pgfpathlineto{\pgfqpoint{26.487420in}{0.773588in}}%
\pgfpathlineto{\pgfqpoint{26.417231in}{0.773588in}}%
\pgfpathlineto{\pgfqpoint{26.344920in}{0.773588in}}%
\pgfpathlineto{\pgfqpoint{26.274267in}{0.773588in}}%
\pgfpathlineto{\pgfqpoint{26.203572in}{0.773588in}}%
\pgfpathlineto{\pgfqpoint{26.130346in}{0.773588in}}%
\pgfpathlineto{\pgfqpoint{26.058621in}{0.773588in}}%
\pgfpathlineto{\pgfqpoint{25.988588in}{0.773588in}}%
\pgfpathlineto{\pgfqpoint{25.916551in}{0.773588in}}%
\pgfpathlineto{\pgfqpoint{25.845667in}{0.773588in}}%
\pgfpathlineto{\pgfqpoint{25.775695in}{0.773588in}}%
\pgfpathlineto{\pgfqpoint{25.702341in}{0.773588in}}%
\pgfpathlineto{\pgfqpoint{25.631022in}{0.773588in}}%
\pgfpathlineto{\pgfqpoint{25.560110in}{0.773588in}}%
\pgfpathlineto{\pgfqpoint{25.487573in}{0.773588in}}%
\pgfpathlineto{\pgfqpoint{25.417047in}{0.773588in}}%
\pgfpathlineto{\pgfqpoint{25.347124in}{0.773588in}}%
\pgfpathlineto{\pgfqpoint{25.273349in}{0.773588in}}%
\pgfpathlineto{\pgfqpoint{25.203216in}{0.773588in}}%
\pgfpathlineto{\pgfqpoint{25.131109in}{0.773588in}}%
\pgfpathlineto{\pgfqpoint{25.055567in}{0.773588in}}%
\pgfpathlineto{\pgfqpoint{24.983525in}{0.773588in}}%
\pgfpathlineto{\pgfqpoint{24.911741in}{0.773588in}}%
\pgfpathlineto{\pgfqpoint{24.836992in}{0.773588in}}%
\pgfpathlineto{\pgfqpoint{24.764742in}{0.773588in}}%
\pgfpathlineto{\pgfqpoint{24.691660in}{0.773588in}}%
\pgfpathlineto{\pgfqpoint{24.618678in}{0.773588in}}%
\pgfpathlineto{\pgfqpoint{24.548639in}{0.773588in}}%
\pgfpathlineto{\pgfqpoint{24.476339in}{0.773588in}}%
\pgfpathlineto{\pgfqpoint{24.400328in}{0.773588in}}%
\pgfpathlineto{\pgfqpoint{24.329090in}{0.773588in}}%
\pgfpathlineto{\pgfqpoint{24.257307in}{0.773588in}}%
\pgfpathlineto{\pgfqpoint{24.183899in}{0.773588in}}%
\pgfpathlineto{\pgfqpoint{24.111784in}{0.773588in}}%
\pgfpathlineto{\pgfqpoint{24.039255in}{0.773588in}}%
\pgfpathlineto{\pgfqpoint{23.966722in}{0.773588in}}%
\pgfpathlineto{\pgfqpoint{23.895213in}{0.773588in}}%
\pgfpathlineto{\pgfqpoint{23.824951in}{0.773588in}}%
\pgfpathlineto{\pgfqpoint{23.752361in}{0.773588in}}%
\pgfpathlineto{\pgfqpoint{23.681448in}{0.773588in}}%
\pgfpathlineto{\pgfqpoint{23.610036in}{0.773588in}}%
\pgfpathlineto{\pgfqpoint{23.537850in}{0.773588in}}%
\pgfpathlineto{\pgfqpoint{23.467323in}{0.773588in}}%
\pgfpathlineto{\pgfqpoint{23.396126in}{0.773588in}}%
\pgfpathlineto{\pgfqpoint{23.323995in}{0.773588in}}%
\pgfpathlineto{\pgfqpoint{23.253484in}{0.773588in}}%
\pgfpathlineto{\pgfqpoint{23.184270in}{0.773588in}}%
\pgfpathlineto{\pgfqpoint{23.113267in}{0.773588in}}%
\pgfpathlineto{\pgfqpoint{23.043397in}{0.773588in}}%
\pgfpathlineto{\pgfqpoint{22.973868in}{0.773588in}}%
\pgfpathlineto{\pgfqpoint{22.902896in}{0.773588in}}%
\pgfpathlineto{\pgfqpoint{22.833628in}{0.773588in}}%
\pgfpathlineto{\pgfqpoint{22.764651in}{0.773588in}}%
\pgfpathlineto{\pgfqpoint{22.694016in}{0.773588in}}%
\pgfpathlineto{\pgfqpoint{22.624114in}{0.773588in}}%
\pgfpathlineto{\pgfqpoint{22.553535in}{0.773588in}}%
\pgfpathlineto{\pgfqpoint{22.482516in}{0.773588in}}%
\pgfpathlineto{\pgfqpoint{22.413449in}{0.773588in}}%
\pgfpathlineto{\pgfqpoint{22.343331in}{0.773588in}}%
\pgfpathlineto{\pgfqpoint{22.268824in}{0.773588in}}%
\pgfpathlineto{\pgfqpoint{22.195707in}{0.773588in}}%
\pgfpathlineto{\pgfqpoint{22.122462in}{0.773588in}}%
\pgfpathlineto{\pgfqpoint{22.048040in}{0.773588in}}%
\pgfpathlineto{\pgfqpoint{21.976836in}{0.773588in}}%
\pgfpathlineto{\pgfqpoint{21.903868in}{0.773588in}}%
\pgfpathlineto{\pgfqpoint{21.828555in}{0.773588in}}%
\pgfpathlineto{\pgfqpoint{21.756227in}{0.773588in}}%
\pgfpathlineto{\pgfqpoint{21.683994in}{0.773588in}}%
\pgfpathlineto{\pgfqpoint{21.610878in}{0.773588in}}%
\pgfpathlineto{\pgfqpoint{21.539601in}{0.773588in}}%
\pgfpathlineto{\pgfqpoint{21.467741in}{0.773588in}}%
\pgfpathlineto{\pgfqpoint{21.394839in}{0.773588in}}%
\pgfpathlineto{\pgfqpoint{21.324498in}{0.773588in}}%
\pgfpathlineto{\pgfqpoint{21.254645in}{0.773588in}}%
\pgfpathlineto{\pgfqpoint{21.181233in}{0.773588in}}%
\pgfpathlineto{\pgfqpoint{21.110656in}{0.773588in}}%
\pgfpathlineto{\pgfqpoint{21.040864in}{0.773588in}}%
\pgfpathlineto{\pgfqpoint{20.969373in}{0.773588in}}%
\pgfpathlineto{\pgfqpoint{20.899587in}{0.773588in}}%
\pgfpathlineto{\pgfqpoint{20.830586in}{0.773588in}}%
\pgfpathlineto{\pgfqpoint{20.758814in}{0.773588in}}%
\pgfpathlineto{\pgfqpoint{20.688980in}{0.773588in}}%
\pgfpathlineto{\pgfqpoint{20.618428in}{0.773588in}}%
\pgfpathlineto{\pgfqpoint{20.547172in}{0.773588in}}%
\pgfpathlineto{\pgfqpoint{20.476961in}{0.773588in}}%
\pgfpathlineto{\pgfqpoint{20.407276in}{0.773588in}}%
\pgfpathlineto{\pgfqpoint{20.334605in}{0.773588in}}%
\pgfpathlineto{\pgfqpoint{20.263963in}{0.773588in}}%
\pgfpathlineto{\pgfqpoint{20.193561in}{0.773588in}}%
\pgfpathlineto{\pgfqpoint{20.121885in}{0.773588in}}%
\pgfpathlineto{\pgfqpoint{20.052071in}{0.773588in}}%
\pgfpathlineto{\pgfqpoint{19.981199in}{0.773588in}}%
\pgfpathlineto{\pgfqpoint{19.907049in}{0.773588in}}%
\pgfpathlineto{\pgfqpoint{19.835928in}{0.773588in}}%
\pgfpathlineto{\pgfqpoint{19.766060in}{0.773588in}}%
\pgfpathlineto{\pgfqpoint{19.693765in}{0.773588in}}%
\pgfpathlineto{\pgfqpoint{19.623855in}{0.773588in}}%
\pgfpathlineto{\pgfqpoint{19.553641in}{0.773588in}}%
\pgfpathlineto{\pgfqpoint{19.482791in}{0.773588in}}%
\pgfpathlineto{\pgfqpoint{19.412352in}{0.773588in}}%
\pgfpathlineto{\pgfqpoint{19.340700in}{0.773588in}}%
\pgfpathlineto{\pgfqpoint{19.266961in}{0.773588in}}%
\pgfpathlineto{\pgfqpoint{19.195134in}{0.773588in}}%
\pgfpathlineto{\pgfqpoint{19.123240in}{0.773588in}}%
\pgfpathlineto{\pgfqpoint{19.048835in}{0.773588in}}%
\pgfpathlineto{\pgfqpoint{18.977585in}{0.773588in}}%
\pgfpathlineto{\pgfqpoint{18.906958in}{0.773588in}}%
\pgfpathlineto{\pgfqpoint{18.834141in}{0.773588in}}%
\pgfpathlineto{\pgfqpoint{18.763558in}{0.773588in}}%
\pgfpathlineto{\pgfqpoint{18.692839in}{0.773588in}}%
\pgfpathlineto{\pgfqpoint{18.619849in}{0.773588in}}%
\pgfpathlineto{\pgfqpoint{18.549509in}{0.773588in}}%
\pgfpathlineto{\pgfqpoint{18.479143in}{0.773588in}}%
\pgfpathlineto{\pgfqpoint{18.406899in}{0.773588in}}%
\pgfpathlineto{\pgfqpoint{18.336223in}{0.773588in}}%
\pgfpathlineto{\pgfqpoint{18.266780in}{0.773588in}}%
\pgfpathlineto{\pgfqpoint{18.195268in}{0.773588in}}%
\pgfpathlineto{\pgfqpoint{18.125960in}{0.773588in}}%
\pgfpathlineto{\pgfqpoint{18.056086in}{0.773588in}}%
\pgfpathlineto{\pgfqpoint{17.983524in}{0.773588in}}%
\pgfpathlineto{\pgfqpoint{17.912627in}{0.773588in}}%
\pgfpathlineto{\pgfqpoint{17.844148in}{0.773588in}}%
\pgfpathlineto{\pgfqpoint{17.772898in}{0.773588in}}%
\pgfpathlineto{\pgfqpoint{17.703480in}{0.773588in}}%
\pgfpathlineto{\pgfqpoint{17.634782in}{0.773588in}}%
\pgfpathlineto{\pgfqpoint{17.563254in}{0.773588in}}%
\pgfpathlineto{\pgfqpoint{17.494859in}{0.773588in}}%
\pgfpathlineto{\pgfqpoint{17.426570in}{0.773588in}}%
\pgfpathlineto{\pgfqpoint{17.356523in}{0.773588in}}%
\pgfpathlineto{\pgfqpoint{17.288800in}{0.773588in}}%
\pgfpathlineto{\pgfqpoint{17.220960in}{0.773588in}}%
\pgfpathlineto{\pgfqpoint{17.150280in}{0.773588in}}%
\pgfpathlineto{\pgfqpoint{17.081113in}{0.773588in}}%
\pgfpathlineto{\pgfqpoint{17.013249in}{0.773588in}}%
\pgfpathlineto{\pgfqpoint{16.942299in}{0.773588in}}%
\pgfpathlineto{\pgfqpoint{16.872974in}{0.773588in}}%
\pgfpathlineto{\pgfqpoint{16.803402in}{0.773588in}}%
\pgfpathlineto{\pgfqpoint{16.731829in}{0.773588in}}%
\pgfpathlineto{\pgfqpoint{16.662798in}{0.773588in}}%
\pgfpathlineto{\pgfqpoint{16.592371in}{0.773588in}}%
\pgfpathlineto{\pgfqpoint{16.519393in}{0.773588in}}%
\pgfpathlineto{\pgfqpoint{16.445981in}{0.773588in}}%
\pgfpathlineto{\pgfqpoint{16.373054in}{0.773588in}}%
\pgfpathlineto{\pgfqpoint{16.299236in}{0.773588in}}%
\pgfpathlineto{\pgfqpoint{16.229236in}{0.773588in}}%
\pgfpathlineto{\pgfqpoint{16.160188in}{0.773588in}}%
\pgfpathlineto{\pgfqpoint{16.088678in}{0.773588in}}%
\pgfpathlineto{\pgfqpoint{16.019598in}{0.773588in}}%
\pgfpathlineto{\pgfqpoint{15.949677in}{0.773588in}}%
\pgfpathlineto{\pgfqpoint{15.878678in}{0.773588in}}%
\pgfpathlineto{\pgfqpoint{15.807467in}{0.773588in}}%
\pgfpathlineto{\pgfqpoint{15.737902in}{0.773588in}}%
\pgfpathlineto{\pgfqpoint{15.665480in}{0.773588in}}%
\pgfpathlineto{\pgfqpoint{15.595878in}{0.773588in}}%
\pgfpathlineto{\pgfqpoint{15.527961in}{0.773588in}}%
\pgfpathlineto{\pgfqpoint{15.457342in}{0.773588in}}%
\pgfpathlineto{\pgfqpoint{15.388582in}{0.773588in}}%
\pgfpathlineto{\pgfqpoint{15.320168in}{0.773588in}}%
\pgfpathlineto{\pgfqpoint{15.249597in}{0.773588in}}%
\pgfpathlineto{\pgfqpoint{15.182512in}{0.773588in}}%
\pgfpathlineto{\pgfqpoint{15.115140in}{0.773588in}}%
\pgfpathlineto{\pgfqpoint{15.044439in}{0.773588in}}%
\pgfpathlineto{\pgfqpoint{14.976372in}{0.773588in}}%
\pgfpathlineto{\pgfqpoint{14.908199in}{0.773588in}}%
\pgfpathlineto{\pgfqpoint{14.836607in}{0.773588in}}%
\pgfpathlineto{\pgfqpoint{14.768225in}{0.773588in}}%
\pgfpathlineto{\pgfqpoint{14.700003in}{0.773588in}}%
\pgfpathlineto{\pgfqpoint{14.630523in}{0.773588in}}%
\pgfpathlineto{\pgfqpoint{14.562284in}{0.773588in}}%
\pgfpathlineto{\pgfqpoint{14.494110in}{0.773588in}}%
\pgfpathlineto{\pgfqpoint{14.424138in}{0.773588in}}%
\pgfpathlineto{\pgfqpoint{14.355979in}{0.773588in}}%
\pgfpathlineto{\pgfqpoint{14.286685in}{0.773588in}}%
\pgfpathlineto{\pgfqpoint{14.216102in}{0.773588in}}%
\pgfpathlineto{\pgfqpoint{14.149562in}{0.773588in}}%
\pgfpathlineto{\pgfqpoint{14.081564in}{0.773588in}}%
\pgfpathlineto{\pgfqpoint{14.012639in}{0.773588in}}%
\pgfpathlineto{\pgfqpoint{13.944783in}{0.773588in}}%
\pgfpathlineto{\pgfqpoint{13.876785in}{0.773588in}}%
\pgfpathlineto{\pgfqpoint{13.804228in}{0.773588in}}%
\pgfpathlineto{\pgfqpoint{13.735499in}{0.773588in}}%
\pgfpathlineto{\pgfqpoint{13.666880in}{0.773588in}}%
\pgfpathlineto{\pgfqpoint{13.595052in}{0.773588in}}%
\pgfpathlineto{\pgfqpoint{13.526360in}{0.773588in}}%
\pgfpathlineto{\pgfqpoint{13.457833in}{0.773588in}}%
\pgfpathlineto{\pgfqpoint{13.387091in}{0.773588in}}%
\pgfpathlineto{\pgfqpoint{13.319414in}{0.773588in}}%
\pgfpathlineto{\pgfqpoint{13.250526in}{0.773588in}}%
\pgfpathlineto{\pgfqpoint{13.178279in}{0.773588in}}%
\pgfpathlineto{\pgfqpoint{13.109710in}{0.773588in}}%
\pgfpathlineto{\pgfqpoint{13.041078in}{0.773588in}}%
\pgfpathlineto{\pgfqpoint{12.969036in}{0.773588in}}%
\pgfpathlineto{\pgfqpoint{12.900232in}{0.773588in}}%
\pgfpathlineto{\pgfqpoint{12.830499in}{0.773588in}}%
\pgfpathlineto{\pgfqpoint{12.759967in}{0.773588in}}%
\pgfpathlineto{\pgfqpoint{12.692271in}{0.773588in}}%
\pgfpathlineto{\pgfqpoint{12.623527in}{0.773588in}}%
\pgfpathlineto{\pgfqpoint{12.552597in}{0.773588in}}%
\pgfpathlineto{\pgfqpoint{12.484090in}{0.773588in}}%
\pgfpathlineto{\pgfqpoint{12.416448in}{0.773588in}}%
\pgfpathlineto{\pgfqpoint{12.347008in}{0.773588in}}%
\pgfpathlineto{\pgfqpoint{12.280761in}{0.773588in}}%
\pgfpathlineto{\pgfqpoint{12.213872in}{0.773588in}}%
\pgfpathlineto{\pgfqpoint{12.144815in}{0.773588in}}%
\pgfpathlineto{\pgfqpoint{12.076638in}{0.773588in}}%
\pgfpathlineto{\pgfqpoint{12.008843in}{0.773588in}}%
\pgfpathlineto{\pgfqpoint{11.938397in}{0.773588in}}%
\pgfpathlineto{\pgfqpoint{11.871239in}{0.773588in}}%
\pgfpathlineto{\pgfqpoint{11.804695in}{0.773588in}}%
\pgfpathlineto{\pgfqpoint{11.736702in}{0.773588in}}%
\pgfpathlineto{\pgfqpoint{11.669969in}{0.773588in}}%
\pgfpathlineto{\pgfqpoint{11.602395in}{0.773588in}}%
\pgfpathlineto{\pgfqpoint{11.532877in}{0.773588in}}%
\pgfpathlineto{\pgfqpoint{11.465294in}{0.773588in}}%
\pgfpathlineto{\pgfqpoint{11.398136in}{0.773588in}}%
\pgfpathlineto{\pgfqpoint{11.328677in}{0.773588in}}%
\pgfpathlineto{\pgfqpoint{11.261777in}{0.773588in}}%
\pgfpathlineto{\pgfqpoint{11.194984in}{0.773588in}}%
\pgfpathlineto{\pgfqpoint{11.125133in}{0.773588in}}%
\pgfpathlineto{\pgfqpoint{11.055389in}{0.773588in}}%
\pgfpathlineto{\pgfqpoint{10.986941in}{0.773588in}}%
\pgfpathlineto{\pgfqpoint{10.916008in}{0.773588in}}%
\pgfpathlineto{\pgfqpoint{10.846851in}{0.773588in}}%
\pgfpathlineto{\pgfqpoint{10.777925in}{0.773588in}}%
\pgfpathlineto{\pgfqpoint{10.708017in}{0.773588in}}%
\pgfpathlineto{\pgfqpoint{10.639006in}{0.773588in}}%
\pgfpathlineto{\pgfqpoint{10.569898in}{0.773588in}}%
\pgfpathlineto{\pgfqpoint{10.498528in}{0.773588in}}%
\pgfpathlineto{\pgfqpoint{10.428999in}{0.773588in}}%
\pgfpathlineto{\pgfqpoint{10.360853in}{0.773588in}}%
\pgfpathlineto{\pgfqpoint{10.290696in}{0.773588in}}%
\pgfpathlineto{\pgfqpoint{10.222102in}{0.773588in}}%
\pgfpathlineto{\pgfqpoint{10.153891in}{0.773588in}}%
\pgfpathlineto{\pgfqpoint{10.081602in}{0.773588in}}%
\pgfpathlineto{\pgfqpoint{10.012687in}{0.773588in}}%
\pgfpathlineto{\pgfqpoint{9.944615in}{0.773588in}}%
\pgfpathlineto{\pgfqpoint{9.876148in}{0.773588in}}%
\pgfpathlineto{\pgfqpoint{9.808815in}{0.773588in}}%
\pgfpathlineto{\pgfqpoint{9.740273in}{0.773588in}}%
\pgfpathlineto{\pgfqpoint{9.670768in}{0.773588in}}%
\pgfpathlineto{\pgfqpoint{9.603748in}{0.773588in}}%
\pgfpathlineto{\pgfqpoint{9.536145in}{0.773588in}}%
\pgfpathlineto{\pgfqpoint{9.467314in}{0.773588in}}%
\pgfpathlineto{\pgfqpoint{9.402138in}{0.773588in}}%
\pgfpathlineto{\pgfqpoint{9.334728in}{0.773588in}}%
\pgfpathlineto{\pgfqpoint{9.265262in}{0.773588in}}%
\pgfpathlineto{\pgfqpoint{9.198477in}{0.773588in}}%
\pgfpathlineto{\pgfqpoint{9.130518in}{0.773588in}}%
\pgfpathlineto{\pgfqpoint{9.059607in}{0.773588in}}%
\pgfpathlineto{\pgfqpoint{8.991646in}{0.773588in}}%
\pgfpathlineto{\pgfqpoint{8.924860in}{0.773588in}}%
\pgfpathlineto{\pgfqpoint{8.856098in}{0.773588in}}%
\pgfpathlineto{\pgfqpoint{8.787638in}{0.773588in}}%
\pgfpathlineto{\pgfqpoint{8.720615in}{0.773588in}}%
\pgfpathlineto{\pgfqpoint{8.651757in}{0.773588in}}%
\pgfpathlineto{\pgfqpoint{8.583601in}{0.773588in}}%
\pgfpathlineto{\pgfqpoint{8.515085in}{0.773588in}}%
\pgfpathlineto{\pgfqpoint{8.444285in}{0.773588in}}%
\pgfpathlineto{\pgfqpoint{8.376454in}{0.773588in}}%
\pgfpathlineto{\pgfqpoint{8.308718in}{0.773588in}}%
\pgfpathlineto{\pgfqpoint{8.237831in}{0.773588in}}%
\pgfpathlineto{\pgfqpoint{8.168934in}{0.773588in}}%
\pgfpathlineto{\pgfqpoint{8.100457in}{0.773588in}}%
\pgfpathlineto{\pgfqpoint{8.028258in}{0.773588in}}%
\pgfpathlineto{\pgfqpoint{7.957986in}{0.773588in}}%
\pgfpathlineto{\pgfqpoint{7.887441in}{0.773588in}}%
\pgfpathlineto{\pgfqpoint{7.812286in}{0.773588in}}%
\pgfpathlineto{\pgfqpoint{7.741421in}{0.773588in}}%
\pgfpathlineto{\pgfqpoint{7.671131in}{0.773588in}}%
\pgfpathlineto{\pgfqpoint{7.598690in}{0.773588in}}%
\pgfpathlineto{\pgfqpoint{7.527107in}{0.773588in}}%
\pgfpathlineto{\pgfqpoint{7.455306in}{0.773588in}}%
\pgfpathlineto{\pgfqpoint{7.379810in}{0.773588in}}%
\pgfpathlineto{\pgfqpoint{7.308295in}{0.773588in}}%
\pgfpathlineto{\pgfqpoint{7.238831in}{0.773588in}}%
\pgfpathlineto{\pgfqpoint{7.171034in}{0.773588in}}%
\pgfpathlineto{\pgfqpoint{7.106301in}{0.773588in}}%
\pgfpathlineto{\pgfqpoint{7.040805in}{0.773588in}}%
\pgfpathlineto{\pgfqpoint{6.973333in}{0.773588in}}%
\pgfpathlineto{\pgfqpoint{6.906544in}{0.773588in}}%
\pgfpathlineto{\pgfqpoint{6.839348in}{0.773588in}}%
\pgfpathlineto{\pgfqpoint{6.770581in}{0.773588in}}%
\pgfpathlineto{\pgfqpoint{6.704271in}{0.773588in}}%
\pgfpathlineto{\pgfqpoint{6.636218in}{0.773588in}}%
\pgfpathlineto{\pgfqpoint{6.568004in}{0.773588in}}%
\pgfpathlineto{\pgfqpoint{6.502885in}{0.773588in}}%
\pgfpathlineto{\pgfqpoint{6.437563in}{0.773588in}}%
\pgfpathlineto{\pgfqpoint{6.369842in}{0.773588in}}%
\pgfpathlineto{\pgfqpoint{6.303479in}{0.773588in}}%
\pgfpathlineto{\pgfqpoint{6.236861in}{0.773588in}}%
\pgfpathlineto{\pgfqpoint{6.169451in}{0.773588in}}%
\pgfpathlineto{\pgfqpoint{6.103260in}{0.773588in}}%
\pgfpathlineto{\pgfqpoint{6.036866in}{0.773588in}}%
\pgfpathlineto{\pgfqpoint{5.968494in}{0.773588in}}%
\pgfpathlineto{\pgfqpoint{5.901623in}{0.773588in}}%
\pgfpathlineto{\pgfqpoint{5.834669in}{0.773588in}}%
\pgfpathlineto{\pgfqpoint{5.766709in}{0.773588in}}%
\pgfpathlineto{\pgfqpoint{5.700467in}{0.773588in}}%
\pgfpathlineto{\pgfqpoint{5.633136in}{0.773588in}}%
\pgfpathlineto{\pgfqpoint{5.561464in}{0.773588in}}%
\pgfpathlineto{\pgfqpoint{5.492693in}{0.773588in}}%
\pgfpathlineto{\pgfqpoint{5.424203in}{0.773588in}}%
\pgfpathlineto{\pgfqpoint{5.353942in}{0.773588in}}%
\pgfpathlineto{\pgfqpoint{5.283251in}{0.773588in}}%
\pgfpathlineto{\pgfqpoint{5.213575in}{0.773588in}}%
\pgfpathlineto{\pgfqpoint{5.142206in}{0.773588in}}%
\pgfpathlineto{\pgfqpoint{5.073090in}{0.773588in}}%
\pgfpathlineto{\pgfqpoint{5.005143in}{0.773588in}}%
\pgfpathlineto{\pgfqpoint{4.935251in}{0.773588in}}%
\pgfpathlineto{\pgfqpoint{4.866311in}{0.773588in}}%
\pgfpathlineto{\pgfqpoint{4.796564in}{0.773588in}}%
\pgfpathlineto{\pgfqpoint{4.726503in}{0.773588in}}%
\pgfpathlineto{\pgfqpoint{4.658575in}{0.773588in}}%
\pgfpathlineto{\pgfqpoint{4.590503in}{0.773588in}}%
\pgfpathlineto{\pgfqpoint{4.520559in}{0.773588in}}%
\pgfpathlineto{\pgfqpoint{4.454265in}{0.773588in}}%
\pgfpathlineto{\pgfqpoint{4.387175in}{0.773588in}}%
\pgfpathlineto{\pgfqpoint{4.319299in}{0.773588in}}%
\pgfpathlineto{\pgfqpoint{4.253047in}{0.773588in}}%
\pgfpathlineto{\pgfqpoint{4.186150in}{0.773588in}}%
\pgfpathlineto{\pgfqpoint{4.116566in}{0.773588in}}%
\pgfpathlineto{\pgfqpoint{4.048987in}{0.773588in}}%
\pgfpathlineto{\pgfqpoint{3.981945in}{0.773588in}}%
\pgfpathlineto{\pgfqpoint{3.914312in}{0.773588in}}%
\pgfpathlineto{\pgfqpoint{3.848285in}{0.773588in}}%
\pgfpathlineto{\pgfqpoint{3.781205in}{0.773588in}}%
\pgfpathlineto{\pgfqpoint{3.712535in}{0.773588in}}%
\pgfpathlineto{\pgfqpoint{3.643958in}{0.773588in}}%
\pgfpathlineto{\pgfqpoint{3.575543in}{0.773588in}}%
\pgfpathlineto{\pgfqpoint{3.503039in}{0.773588in}}%
\pgfpathlineto{\pgfqpoint{3.430573in}{0.773588in}}%
\pgfpathlineto{\pgfqpoint{3.356465in}{0.773588in}}%
\pgfpathlineto{\pgfqpoint{3.277599in}{0.773588in}}%
\pgfpathlineto{\pgfqpoint{3.195433in}{0.773588in}}%
\pgfpathlineto{\pgfqpoint{3.123114in}{0.773588in}}%
\pgfpathlineto{\pgfqpoint{3.047640in}{0.773588in}}%
\pgfpathlineto{\pgfqpoint{2.975015in}{0.773588in}}%
\pgfpathlineto{\pgfqpoint{2.902674in}{0.773588in}}%
\pgfpathlineto{\pgfqpoint{2.826501in}{0.773588in}}%
\pgfpathlineto{\pgfqpoint{2.753491in}{0.773588in}}%
\pgfpathlineto{\pgfqpoint{2.681331in}{0.773588in}}%
\pgfpathlineto{\pgfqpoint{2.606023in}{0.773588in}}%
\pgfpathlineto{\pgfqpoint{2.534117in}{0.773588in}}%
\pgfpathlineto{\pgfqpoint{2.462769in}{0.773588in}}%
\pgfpathlineto{\pgfqpoint{2.387948in}{0.773588in}}%
\pgfpathlineto{\pgfqpoint{2.317152in}{0.773588in}}%
\pgfpathlineto{\pgfqpoint{2.248172in}{0.773588in}}%
\pgfpathlineto{\pgfqpoint{2.177337in}{0.773588in}}%
\pgfpathlineto{\pgfqpoint{2.108883in}{0.773588in}}%
\pgfpathlineto{\pgfqpoint{2.041179in}{0.773588in}}%
\pgfpathlineto{\pgfqpoint{1.970951in}{0.773588in}}%
\pgfpathlineto{\pgfqpoint{1.904458in}{0.773588in}}%
\pgfpathlineto{\pgfqpoint{1.835890in}{0.773588in}}%
\pgfpathlineto{\pgfqpoint{1.766402in}{0.773588in}}%
\pgfpathlineto{\pgfqpoint{1.698662in}{0.773588in}}%
\pgfpathlineto{\pgfqpoint{1.628862in}{0.773588in}}%
\pgfpathlineto{\pgfqpoint{1.557461in}{0.773588in}}%
\pgfpathlineto{\pgfqpoint{1.489306in}{0.773588in}}%
\pgfpathlineto{\pgfqpoint{1.421095in}{0.773588in}}%
\pgfpathlineto{\pgfqpoint{1.349373in}{0.773588in}}%
\pgfpathlineto{\pgfqpoint{1.283036in}{0.773588in}}%
\pgfpathlineto{\pgfqpoint{1.216322in}{0.773588in}}%
\pgfpathlineto{\pgfqpoint{1.147369in}{0.773588in}}%
\pgfpathlineto{\pgfqpoint{1.079942in}{0.773588in}}%
\pgfpathlineto{\pgfqpoint{1.012853in}{0.773588in}}%
\pgfpathlineto{\pgfqpoint{0.942110in}{0.773588in}}%
\pgfpathlineto{\pgfqpoint{0.875335in}{0.773588in}}%
\pgfpathlineto{\pgfqpoint{0.807094in}{0.773588in}}%
\pgfpathclose%
\pgfusepath{fill}%
\end{pgfscope}%
\begin{pgfscope}%
\pgfpathrectangle{\pgfqpoint{0.781402in}{0.773588in}}{\pgfqpoint{2.110351in}{5.415119in}}%
\pgfusepath{clip}%
\pgfsetbuttcap%
\pgfsetroundjoin%
\definecolor{currentfill}{rgb}{0.172549,0.627451,0.172549}%
\pgfsetfillcolor{currentfill}%
\pgfsetlinewidth{0.000000pt}%
\definecolor{currentstroke}{rgb}{0.000000,0.000000,0.000000}%
\pgfsetstrokecolor{currentstroke}%
\pgfsetdash{}{0pt}%
\pgfpathmoveto{\pgfqpoint{0.807094in}{0.773588in}}%
\pgfpathlineto{\pgfqpoint{0.807094in}{0.773588in}}%
\pgfpathlineto{\pgfqpoint{0.875335in}{0.773588in}}%
\pgfpathlineto{\pgfqpoint{0.942110in}{0.773588in}}%
\pgfpathlineto{\pgfqpoint{1.012853in}{0.773588in}}%
\pgfpathlineto{\pgfqpoint{1.079942in}{0.773588in}}%
\pgfpathlineto{\pgfqpoint{1.147369in}{0.773588in}}%
\pgfpathlineto{\pgfqpoint{1.216322in}{0.773588in}}%
\pgfpathlineto{\pgfqpoint{1.283036in}{0.773588in}}%
\pgfpathlineto{\pgfqpoint{1.349373in}{0.773588in}}%
\pgfpathlineto{\pgfqpoint{1.421095in}{0.773588in}}%
\pgfpathlineto{\pgfqpoint{1.489306in}{0.773588in}}%
\pgfpathlineto{\pgfqpoint{1.557461in}{0.773588in}}%
\pgfpathlineto{\pgfqpoint{1.628862in}{0.773588in}}%
\pgfpathlineto{\pgfqpoint{1.698662in}{0.773588in}}%
\pgfpathlineto{\pgfqpoint{1.766402in}{0.773588in}}%
\pgfpathlineto{\pgfqpoint{1.835890in}{0.773588in}}%
\pgfpathlineto{\pgfqpoint{1.904458in}{0.773588in}}%
\pgfpathlineto{\pgfqpoint{1.970951in}{0.773588in}}%
\pgfpathlineto{\pgfqpoint{2.041179in}{0.773588in}}%
\pgfpathlineto{\pgfqpoint{2.108883in}{0.773588in}}%
\pgfpathlineto{\pgfqpoint{2.177337in}{0.773588in}}%
\pgfpathlineto{\pgfqpoint{2.248172in}{0.773588in}}%
\pgfpathlineto{\pgfqpoint{2.317152in}{0.773588in}}%
\pgfpathlineto{\pgfqpoint{2.387948in}{0.773588in}}%
\pgfpathlineto{\pgfqpoint{2.462769in}{0.773588in}}%
\pgfpathlineto{\pgfqpoint{2.534117in}{0.773588in}}%
\pgfpathlineto{\pgfqpoint{2.606023in}{0.773588in}}%
\pgfpathlineto{\pgfqpoint{2.681331in}{0.773588in}}%
\pgfpathlineto{\pgfqpoint{2.753491in}{0.773588in}}%
\pgfpathlineto{\pgfqpoint{2.826501in}{0.773588in}}%
\pgfpathlineto{\pgfqpoint{2.902674in}{0.773588in}}%
\pgfpathlineto{\pgfqpoint{2.975015in}{0.773588in}}%
\pgfpathlineto{\pgfqpoint{3.047640in}{0.773588in}}%
\pgfpathlineto{\pgfqpoint{3.123114in}{0.773588in}}%
\pgfpathlineto{\pgfqpoint{3.195433in}{0.773588in}}%
\pgfpathlineto{\pgfqpoint{3.277599in}{0.773588in}}%
\pgfpathlineto{\pgfqpoint{3.356465in}{0.773588in}}%
\pgfpathlineto{\pgfqpoint{3.430573in}{0.773588in}}%
\pgfpathlineto{\pgfqpoint{3.503039in}{0.773588in}}%
\pgfpathlineto{\pgfqpoint{3.575543in}{0.773588in}}%
\pgfpathlineto{\pgfqpoint{3.643958in}{0.773588in}}%
\pgfpathlineto{\pgfqpoint{3.712535in}{0.773588in}}%
\pgfpathlineto{\pgfqpoint{3.781205in}{0.773588in}}%
\pgfpathlineto{\pgfqpoint{3.848285in}{0.773588in}}%
\pgfpathlineto{\pgfqpoint{3.914312in}{0.773588in}}%
\pgfpathlineto{\pgfqpoint{3.981945in}{0.773588in}}%
\pgfpathlineto{\pgfqpoint{4.048987in}{0.773588in}}%
\pgfpathlineto{\pgfqpoint{4.116566in}{0.773588in}}%
\pgfpathlineto{\pgfqpoint{4.186150in}{0.773588in}}%
\pgfpathlineto{\pgfqpoint{4.253047in}{0.773588in}}%
\pgfpathlineto{\pgfqpoint{4.319299in}{0.773588in}}%
\pgfpathlineto{\pgfqpoint{4.387175in}{0.773588in}}%
\pgfpathlineto{\pgfqpoint{4.454265in}{0.773588in}}%
\pgfpathlineto{\pgfqpoint{4.520559in}{0.773588in}}%
\pgfpathlineto{\pgfqpoint{4.590503in}{0.773588in}}%
\pgfpathlineto{\pgfqpoint{4.658575in}{0.773588in}}%
\pgfpathlineto{\pgfqpoint{4.726503in}{0.773588in}}%
\pgfpathlineto{\pgfqpoint{4.796564in}{0.773588in}}%
\pgfpathlineto{\pgfqpoint{4.866311in}{0.773588in}}%
\pgfpathlineto{\pgfqpoint{4.935251in}{0.773588in}}%
\pgfpathlineto{\pgfqpoint{5.005143in}{0.773588in}}%
\pgfpathlineto{\pgfqpoint{5.073090in}{0.773588in}}%
\pgfpathlineto{\pgfqpoint{5.142206in}{0.773588in}}%
\pgfpathlineto{\pgfqpoint{5.213575in}{0.773588in}}%
\pgfpathlineto{\pgfqpoint{5.283251in}{0.773588in}}%
\pgfpathlineto{\pgfqpoint{5.353942in}{0.773588in}}%
\pgfpathlineto{\pgfqpoint{5.424203in}{0.773588in}}%
\pgfpathlineto{\pgfqpoint{5.492693in}{0.773588in}}%
\pgfpathlineto{\pgfqpoint{5.561464in}{0.773588in}}%
\pgfpathlineto{\pgfqpoint{5.633136in}{0.773588in}}%
\pgfpathlineto{\pgfqpoint{5.700467in}{0.773588in}}%
\pgfpathlineto{\pgfqpoint{5.766709in}{0.773588in}}%
\pgfpathlineto{\pgfqpoint{5.834669in}{0.773588in}}%
\pgfpathlineto{\pgfqpoint{5.901623in}{0.773588in}}%
\pgfpathlineto{\pgfqpoint{5.968494in}{0.773588in}}%
\pgfpathlineto{\pgfqpoint{6.036866in}{0.773588in}}%
\pgfpathlineto{\pgfqpoint{6.103260in}{0.773588in}}%
\pgfpathlineto{\pgfqpoint{6.169451in}{0.773588in}}%
\pgfpathlineto{\pgfqpoint{6.236861in}{0.773588in}}%
\pgfpathlineto{\pgfqpoint{6.303479in}{0.773588in}}%
\pgfpathlineto{\pgfqpoint{6.369842in}{0.773588in}}%
\pgfpathlineto{\pgfqpoint{6.437563in}{0.773588in}}%
\pgfpathlineto{\pgfqpoint{6.502885in}{0.773588in}}%
\pgfpathlineto{\pgfqpoint{6.568004in}{0.773588in}}%
\pgfpathlineto{\pgfqpoint{6.636218in}{0.773588in}}%
\pgfpathlineto{\pgfqpoint{6.704271in}{0.773588in}}%
\pgfpathlineto{\pgfqpoint{6.770581in}{0.773588in}}%
\pgfpathlineto{\pgfqpoint{6.839348in}{0.773588in}}%
\pgfpathlineto{\pgfqpoint{6.906544in}{0.773588in}}%
\pgfpathlineto{\pgfqpoint{6.973333in}{0.773588in}}%
\pgfpathlineto{\pgfqpoint{7.040805in}{0.773588in}}%
\pgfpathlineto{\pgfqpoint{7.106301in}{0.773588in}}%
\pgfpathlineto{\pgfqpoint{7.171034in}{0.773588in}}%
\pgfpathlineto{\pgfqpoint{7.238831in}{0.773588in}}%
\pgfpathlineto{\pgfqpoint{7.308295in}{0.773588in}}%
\pgfpathlineto{\pgfqpoint{7.379810in}{0.773588in}}%
\pgfpathlineto{\pgfqpoint{7.455306in}{0.773588in}}%
\pgfpathlineto{\pgfqpoint{7.527107in}{0.773588in}}%
\pgfpathlineto{\pgfqpoint{7.598690in}{0.773588in}}%
\pgfpathlineto{\pgfqpoint{7.671131in}{0.773588in}}%
\pgfpathlineto{\pgfqpoint{7.741421in}{0.773588in}}%
\pgfpathlineto{\pgfqpoint{7.812286in}{0.773588in}}%
\pgfpathlineto{\pgfqpoint{7.887441in}{0.773588in}}%
\pgfpathlineto{\pgfqpoint{7.957986in}{0.773588in}}%
\pgfpathlineto{\pgfqpoint{8.028258in}{0.773588in}}%
\pgfpathlineto{\pgfqpoint{8.100457in}{0.773588in}}%
\pgfpathlineto{\pgfqpoint{8.168934in}{0.773588in}}%
\pgfpathlineto{\pgfqpoint{8.237831in}{0.773588in}}%
\pgfpathlineto{\pgfqpoint{8.308718in}{0.773588in}}%
\pgfpathlineto{\pgfqpoint{8.376454in}{0.773588in}}%
\pgfpathlineto{\pgfqpoint{8.444285in}{0.773588in}}%
\pgfpathlineto{\pgfqpoint{8.515085in}{0.773588in}}%
\pgfpathlineto{\pgfqpoint{8.583601in}{0.773588in}}%
\pgfpathlineto{\pgfqpoint{8.651757in}{0.773588in}}%
\pgfpathlineto{\pgfqpoint{8.720615in}{0.773588in}}%
\pgfpathlineto{\pgfqpoint{8.787638in}{0.773588in}}%
\pgfpathlineto{\pgfqpoint{8.856098in}{0.773588in}}%
\pgfpathlineto{\pgfqpoint{8.924860in}{0.773588in}}%
\pgfpathlineto{\pgfqpoint{8.991646in}{0.773588in}}%
\pgfpathlineto{\pgfqpoint{9.059607in}{0.773588in}}%
\pgfpathlineto{\pgfqpoint{9.130518in}{0.773588in}}%
\pgfpathlineto{\pgfqpoint{9.198477in}{0.773588in}}%
\pgfpathlineto{\pgfqpoint{9.265262in}{0.773588in}}%
\pgfpathlineto{\pgfqpoint{9.334728in}{0.773588in}}%
\pgfpathlineto{\pgfqpoint{9.402138in}{0.773588in}}%
\pgfpathlineto{\pgfqpoint{9.467314in}{0.773588in}}%
\pgfpathlineto{\pgfqpoint{9.536145in}{0.773588in}}%
\pgfpathlineto{\pgfqpoint{9.603748in}{0.773588in}}%
\pgfpathlineto{\pgfqpoint{9.670768in}{0.773588in}}%
\pgfpathlineto{\pgfqpoint{9.740273in}{0.773588in}}%
\pgfpathlineto{\pgfqpoint{9.808815in}{0.773588in}}%
\pgfpathlineto{\pgfqpoint{9.876148in}{0.773588in}}%
\pgfpathlineto{\pgfqpoint{9.944615in}{0.773588in}}%
\pgfpathlineto{\pgfqpoint{10.012687in}{0.773588in}}%
\pgfpathlineto{\pgfqpoint{10.081602in}{0.773588in}}%
\pgfpathlineto{\pgfqpoint{10.153891in}{0.773588in}}%
\pgfpathlineto{\pgfqpoint{10.222102in}{0.773588in}}%
\pgfpathlineto{\pgfqpoint{10.290696in}{0.773588in}}%
\pgfpathlineto{\pgfqpoint{10.360853in}{0.773588in}}%
\pgfpathlineto{\pgfqpoint{10.428999in}{0.773588in}}%
\pgfpathlineto{\pgfqpoint{10.498528in}{0.773588in}}%
\pgfpathlineto{\pgfqpoint{10.569898in}{0.773588in}}%
\pgfpathlineto{\pgfqpoint{10.639006in}{0.773588in}}%
\pgfpathlineto{\pgfqpoint{10.708017in}{0.773588in}}%
\pgfpathlineto{\pgfqpoint{10.777925in}{0.773588in}}%
\pgfpathlineto{\pgfqpoint{10.846851in}{0.773588in}}%
\pgfpathlineto{\pgfqpoint{10.916008in}{0.773588in}}%
\pgfpathlineto{\pgfqpoint{10.986941in}{0.773588in}}%
\pgfpathlineto{\pgfqpoint{11.055389in}{0.773588in}}%
\pgfpathlineto{\pgfqpoint{11.125133in}{0.773588in}}%
\pgfpathlineto{\pgfqpoint{11.194984in}{0.773588in}}%
\pgfpathlineto{\pgfqpoint{11.261777in}{0.773588in}}%
\pgfpathlineto{\pgfqpoint{11.328677in}{0.773588in}}%
\pgfpathlineto{\pgfqpoint{11.398136in}{0.773588in}}%
\pgfpathlineto{\pgfqpoint{11.465294in}{0.773588in}}%
\pgfpathlineto{\pgfqpoint{11.532877in}{0.773588in}}%
\pgfpathlineto{\pgfqpoint{11.602395in}{0.773588in}}%
\pgfpathlineto{\pgfqpoint{11.669969in}{0.773588in}}%
\pgfpathlineto{\pgfqpoint{11.736702in}{0.773588in}}%
\pgfpathlineto{\pgfqpoint{11.804695in}{0.773588in}}%
\pgfpathlineto{\pgfqpoint{11.871239in}{0.773588in}}%
\pgfpathlineto{\pgfqpoint{11.938397in}{0.773588in}}%
\pgfpathlineto{\pgfqpoint{12.008843in}{0.773588in}}%
\pgfpathlineto{\pgfqpoint{12.076638in}{0.773588in}}%
\pgfpathlineto{\pgfqpoint{12.144815in}{0.773588in}}%
\pgfpathlineto{\pgfqpoint{12.213872in}{0.773588in}}%
\pgfpathlineto{\pgfqpoint{12.280761in}{0.773588in}}%
\pgfpathlineto{\pgfqpoint{12.347008in}{0.773588in}}%
\pgfpathlineto{\pgfqpoint{12.416448in}{0.773588in}}%
\pgfpathlineto{\pgfqpoint{12.484090in}{0.773588in}}%
\pgfpathlineto{\pgfqpoint{12.552597in}{0.773588in}}%
\pgfpathlineto{\pgfqpoint{12.623527in}{0.773588in}}%
\pgfpathlineto{\pgfqpoint{12.692271in}{0.773588in}}%
\pgfpathlineto{\pgfqpoint{12.759967in}{0.773588in}}%
\pgfpathlineto{\pgfqpoint{12.830499in}{0.773588in}}%
\pgfpathlineto{\pgfqpoint{12.900232in}{0.773588in}}%
\pgfpathlineto{\pgfqpoint{12.969036in}{0.773588in}}%
\pgfpathlineto{\pgfqpoint{13.041078in}{0.773588in}}%
\pgfpathlineto{\pgfqpoint{13.109710in}{0.773588in}}%
\pgfpathlineto{\pgfqpoint{13.178279in}{0.773588in}}%
\pgfpathlineto{\pgfqpoint{13.250526in}{0.773588in}}%
\pgfpathlineto{\pgfqpoint{13.319414in}{0.773588in}}%
\pgfpathlineto{\pgfqpoint{13.387091in}{0.773588in}}%
\pgfpathlineto{\pgfqpoint{13.457833in}{0.773588in}}%
\pgfpathlineto{\pgfqpoint{13.526360in}{0.773588in}}%
\pgfpathlineto{\pgfqpoint{13.595052in}{0.773588in}}%
\pgfpathlineto{\pgfqpoint{13.666880in}{0.773588in}}%
\pgfpathlineto{\pgfqpoint{13.735499in}{0.773588in}}%
\pgfpathlineto{\pgfqpoint{13.804228in}{0.773588in}}%
\pgfpathlineto{\pgfqpoint{13.876785in}{0.773588in}}%
\pgfpathlineto{\pgfqpoint{13.944783in}{0.773588in}}%
\pgfpathlineto{\pgfqpoint{14.012639in}{0.773588in}}%
\pgfpathlineto{\pgfqpoint{14.081564in}{0.773588in}}%
\pgfpathlineto{\pgfqpoint{14.149562in}{0.773588in}}%
\pgfpathlineto{\pgfqpoint{14.216102in}{0.773588in}}%
\pgfpathlineto{\pgfqpoint{14.286685in}{0.773588in}}%
\pgfpathlineto{\pgfqpoint{14.355979in}{0.773588in}}%
\pgfpathlineto{\pgfqpoint{14.424138in}{0.773588in}}%
\pgfpathlineto{\pgfqpoint{14.494110in}{0.773588in}}%
\pgfpathlineto{\pgfqpoint{14.562284in}{0.773588in}}%
\pgfpathlineto{\pgfqpoint{14.630523in}{0.773588in}}%
\pgfpathlineto{\pgfqpoint{14.700003in}{0.773588in}}%
\pgfpathlineto{\pgfqpoint{14.768225in}{0.773588in}}%
\pgfpathlineto{\pgfqpoint{14.836607in}{0.773588in}}%
\pgfpathlineto{\pgfqpoint{14.908199in}{0.773588in}}%
\pgfpathlineto{\pgfqpoint{14.976372in}{0.773588in}}%
\pgfpathlineto{\pgfqpoint{15.044439in}{0.773588in}}%
\pgfpathlineto{\pgfqpoint{15.115140in}{0.773588in}}%
\pgfpathlineto{\pgfqpoint{15.182512in}{0.773588in}}%
\pgfpathlineto{\pgfqpoint{15.249597in}{0.773588in}}%
\pgfpathlineto{\pgfqpoint{15.320168in}{0.773588in}}%
\pgfpathlineto{\pgfqpoint{15.388582in}{0.773588in}}%
\pgfpathlineto{\pgfqpoint{15.457342in}{0.773588in}}%
\pgfpathlineto{\pgfqpoint{15.527961in}{0.773588in}}%
\pgfpathlineto{\pgfqpoint{15.595878in}{0.773588in}}%
\pgfpathlineto{\pgfqpoint{15.665480in}{0.773588in}}%
\pgfpathlineto{\pgfqpoint{15.737902in}{0.773588in}}%
\pgfpathlineto{\pgfqpoint{15.807467in}{0.773588in}}%
\pgfpathlineto{\pgfqpoint{15.878678in}{0.773588in}}%
\pgfpathlineto{\pgfqpoint{15.949677in}{0.773588in}}%
\pgfpathlineto{\pgfqpoint{16.019598in}{0.773588in}}%
\pgfpathlineto{\pgfqpoint{16.088678in}{0.773588in}}%
\pgfpathlineto{\pgfqpoint{16.160188in}{0.773588in}}%
\pgfpathlineto{\pgfqpoint{16.229236in}{0.773588in}}%
\pgfpathlineto{\pgfqpoint{16.299236in}{0.773588in}}%
\pgfpathlineto{\pgfqpoint{16.373054in}{0.773588in}}%
\pgfpathlineto{\pgfqpoint{16.445981in}{0.773588in}}%
\pgfpathlineto{\pgfqpoint{16.519393in}{0.773588in}}%
\pgfpathlineto{\pgfqpoint{16.592371in}{0.773588in}}%
\pgfpathlineto{\pgfqpoint{16.662798in}{0.773588in}}%
\pgfpathlineto{\pgfqpoint{16.731829in}{0.773588in}}%
\pgfpathlineto{\pgfqpoint{16.803402in}{0.773588in}}%
\pgfpathlineto{\pgfqpoint{16.872974in}{0.773588in}}%
\pgfpathlineto{\pgfqpoint{16.942299in}{0.773588in}}%
\pgfpathlineto{\pgfqpoint{17.013249in}{0.773588in}}%
\pgfpathlineto{\pgfqpoint{17.081113in}{0.773588in}}%
\pgfpathlineto{\pgfqpoint{17.150280in}{0.773588in}}%
\pgfpathlineto{\pgfqpoint{17.220960in}{0.773588in}}%
\pgfpathlineto{\pgfqpoint{17.288800in}{0.773588in}}%
\pgfpathlineto{\pgfqpoint{17.356523in}{0.773588in}}%
\pgfpathlineto{\pgfqpoint{17.426570in}{0.773588in}}%
\pgfpathlineto{\pgfqpoint{17.494859in}{0.773588in}}%
\pgfpathlineto{\pgfqpoint{17.563254in}{0.773588in}}%
\pgfpathlineto{\pgfqpoint{17.634782in}{0.773588in}}%
\pgfpathlineto{\pgfqpoint{17.703480in}{0.773588in}}%
\pgfpathlineto{\pgfqpoint{17.772898in}{0.773588in}}%
\pgfpathlineto{\pgfqpoint{17.844148in}{0.773588in}}%
\pgfpathlineto{\pgfqpoint{17.912627in}{0.773588in}}%
\pgfpathlineto{\pgfqpoint{17.983524in}{0.773588in}}%
\pgfpathlineto{\pgfqpoint{18.056086in}{0.773588in}}%
\pgfpathlineto{\pgfqpoint{18.125960in}{0.773588in}}%
\pgfpathlineto{\pgfqpoint{18.195268in}{0.773588in}}%
\pgfpathlineto{\pgfqpoint{18.266780in}{0.773588in}}%
\pgfpathlineto{\pgfqpoint{18.336223in}{0.773588in}}%
\pgfpathlineto{\pgfqpoint{18.406899in}{0.773588in}}%
\pgfpathlineto{\pgfqpoint{18.479143in}{0.773588in}}%
\pgfpathlineto{\pgfqpoint{18.549509in}{0.773588in}}%
\pgfpathlineto{\pgfqpoint{18.619849in}{0.773588in}}%
\pgfpathlineto{\pgfqpoint{18.692839in}{0.773588in}}%
\pgfpathlineto{\pgfqpoint{18.763558in}{0.773588in}}%
\pgfpathlineto{\pgfqpoint{18.834141in}{0.773588in}}%
\pgfpathlineto{\pgfqpoint{18.906958in}{0.773588in}}%
\pgfpathlineto{\pgfqpoint{18.977585in}{0.773588in}}%
\pgfpathlineto{\pgfqpoint{19.048835in}{0.773588in}}%
\pgfpathlineto{\pgfqpoint{19.123240in}{0.773588in}}%
\pgfpathlineto{\pgfqpoint{19.195134in}{0.773588in}}%
\pgfpathlineto{\pgfqpoint{19.266961in}{0.773588in}}%
\pgfpathlineto{\pgfqpoint{19.340700in}{0.773588in}}%
\pgfpathlineto{\pgfqpoint{19.412352in}{0.773588in}}%
\pgfpathlineto{\pgfqpoint{19.482791in}{0.773588in}}%
\pgfpathlineto{\pgfqpoint{19.553641in}{0.773588in}}%
\pgfpathlineto{\pgfqpoint{19.623855in}{0.773588in}}%
\pgfpathlineto{\pgfqpoint{19.693765in}{0.773588in}}%
\pgfpathlineto{\pgfqpoint{19.766060in}{0.773588in}}%
\pgfpathlineto{\pgfqpoint{19.835928in}{0.773588in}}%
\pgfpathlineto{\pgfqpoint{19.907049in}{0.773588in}}%
\pgfpathlineto{\pgfqpoint{19.981199in}{0.773588in}}%
\pgfpathlineto{\pgfqpoint{20.052071in}{0.773588in}}%
\pgfpathlineto{\pgfqpoint{20.121885in}{0.773588in}}%
\pgfpathlineto{\pgfqpoint{20.193561in}{0.773588in}}%
\pgfpathlineto{\pgfqpoint{20.263963in}{0.773588in}}%
\pgfpathlineto{\pgfqpoint{20.334605in}{0.773588in}}%
\pgfpathlineto{\pgfqpoint{20.407276in}{0.773588in}}%
\pgfpathlineto{\pgfqpoint{20.476961in}{0.773588in}}%
\pgfpathlineto{\pgfqpoint{20.547172in}{0.773588in}}%
\pgfpathlineto{\pgfqpoint{20.618428in}{0.773588in}}%
\pgfpathlineto{\pgfqpoint{20.688980in}{0.773588in}}%
\pgfpathlineto{\pgfqpoint{20.758814in}{0.773588in}}%
\pgfpathlineto{\pgfqpoint{20.830586in}{0.773588in}}%
\pgfpathlineto{\pgfqpoint{20.899587in}{0.773588in}}%
\pgfpathlineto{\pgfqpoint{20.969373in}{0.773588in}}%
\pgfpathlineto{\pgfqpoint{21.040864in}{0.773588in}}%
\pgfpathlineto{\pgfqpoint{21.110656in}{0.773588in}}%
\pgfpathlineto{\pgfqpoint{21.181233in}{0.773588in}}%
\pgfpathlineto{\pgfqpoint{21.254645in}{0.773588in}}%
\pgfpathlineto{\pgfqpoint{21.324498in}{0.773588in}}%
\pgfpathlineto{\pgfqpoint{21.394839in}{0.773588in}}%
\pgfpathlineto{\pgfqpoint{21.467741in}{0.773588in}}%
\pgfpathlineto{\pgfqpoint{21.539601in}{0.773588in}}%
\pgfpathlineto{\pgfqpoint{21.610878in}{0.773588in}}%
\pgfpathlineto{\pgfqpoint{21.683994in}{0.773588in}}%
\pgfpathlineto{\pgfqpoint{21.756227in}{0.773588in}}%
\pgfpathlineto{\pgfqpoint{21.828555in}{0.773588in}}%
\pgfpathlineto{\pgfqpoint{21.903868in}{0.773588in}}%
\pgfpathlineto{\pgfqpoint{21.976836in}{0.773588in}}%
\pgfpathlineto{\pgfqpoint{22.048040in}{0.773588in}}%
\pgfpathlineto{\pgfqpoint{22.122462in}{0.773588in}}%
\pgfpathlineto{\pgfqpoint{22.195707in}{0.773588in}}%
\pgfpathlineto{\pgfqpoint{22.268824in}{0.773588in}}%
\pgfpathlineto{\pgfqpoint{22.343331in}{0.773588in}}%
\pgfpathlineto{\pgfqpoint{22.413449in}{0.773588in}}%
\pgfpathlineto{\pgfqpoint{22.482516in}{0.773588in}}%
\pgfpathlineto{\pgfqpoint{22.553535in}{0.773588in}}%
\pgfpathlineto{\pgfqpoint{22.624114in}{0.773588in}}%
\pgfpathlineto{\pgfqpoint{22.694016in}{0.773588in}}%
\pgfpathlineto{\pgfqpoint{22.764651in}{0.773588in}}%
\pgfpathlineto{\pgfqpoint{22.833628in}{0.773588in}}%
\pgfpathlineto{\pgfqpoint{22.902896in}{0.773588in}}%
\pgfpathlineto{\pgfqpoint{22.973868in}{0.773588in}}%
\pgfpathlineto{\pgfqpoint{23.043397in}{0.773588in}}%
\pgfpathlineto{\pgfqpoint{23.113267in}{0.773588in}}%
\pgfpathlineto{\pgfqpoint{23.184270in}{0.773588in}}%
\pgfpathlineto{\pgfqpoint{23.253484in}{0.773588in}}%
\pgfpathlineto{\pgfqpoint{23.323995in}{0.773588in}}%
\pgfpathlineto{\pgfqpoint{23.396126in}{0.773588in}}%
\pgfpathlineto{\pgfqpoint{23.467323in}{0.773588in}}%
\pgfpathlineto{\pgfqpoint{23.537850in}{0.773588in}}%
\pgfpathlineto{\pgfqpoint{23.610036in}{0.773588in}}%
\pgfpathlineto{\pgfqpoint{23.681448in}{0.773588in}}%
\pgfpathlineto{\pgfqpoint{23.752361in}{0.773588in}}%
\pgfpathlineto{\pgfqpoint{23.824951in}{0.773588in}}%
\pgfpathlineto{\pgfqpoint{23.895213in}{0.773588in}}%
\pgfpathlineto{\pgfqpoint{23.966722in}{0.773588in}}%
\pgfpathlineto{\pgfqpoint{24.039255in}{0.773588in}}%
\pgfpathlineto{\pgfqpoint{24.111784in}{0.773588in}}%
\pgfpathlineto{\pgfqpoint{24.183899in}{0.773588in}}%
\pgfpathlineto{\pgfqpoint{24.257307in}{0.773588in}}%
\pgfpathlineto{\pgfqpoint{24.329090in}{0.773588in}}%
\pgfpathlineto{\pgfqpoint{24.400328in}{0.773588in}}%
\pgfpathlineto{\pgfqpoint{24.476339in}{0.773588in}}%
\pgfpathlineto{\pgfqpoint{24.548639in}{0.773588in}}%
\pgfpathlineto{\pgfqpoint{24.618678in}{0.773588in}}%
\pgfpathlineto{\pgfqpoint{24.691660in}{0.773588in}}%
\pgfpathlineto{\pgfqpoint{24.764742in}{0.773588in}}%
\pgfpathlineto{\pgfqpoint{24.836992in}{0.773588in}}%
\pgfpathlineto{\pgfqpoint{24.911741in}{0.773588in}}%
\pgfpathlineto{\pgfqpoint{24.983525in}{0.773588in}}%
\pgfpathlineto{\pgfqpoint{25.055567in}{0.773588in}}%
\pgfpathlineto{\pgfqpoint{25.131109in}{0.773588in}}%
\pgfpathlineto{\pgfqpoint{25.203216in}{0.773588in}}%
\pgfpathlineto{\pgfqpoint{25.273349in}{0.773588in}}%
\pgfpathlineto{\pgfqpoint{25.347124in}{0.773588in}}%
\pgfpathlineto{\pgfqpoint{25.417047in}{0.773588in}}%
\pgfpathlineto{\pgfqpoint{25.487573in}{0.773588in}}%
\pgfpathlineto{\pgfqpoint{25.560110in}{0.773588in}}%
\pgfpathlineto{\pgfqpoint{25.631022in}{0.773588in}}%
\pgfpathlineto{\pgfqpoint{25.702341in}{0.773588in}}%
\pgfpathlineto{\pgfqpoint{25.775695in}{0.773588in}}%
\pgfpathlineto{\pgfqpoint{25.845667in}{0.773588in}}%
\pgfpathlineto{\pgfqpoint{25.916551in}{0.773588in}}%
\pgfpathlineto{\pgfqpoint{25.988588in}{0.773588in}}%
\pgfpathlineto{\pgfqpoint{26.058621in}{0.773588in}}%
\pgfpathlineto{\pgfqpoint{26.130346in}{0.773588in}}%
\pgfpathlineto{\pgfqpoint{26.203572in}{0.773588in}}%
\pgfpathlineto{\pgfqpoint{26.274267in}{0.773588in}}%
\pgfpathlineto{\pgfqpoint{26.344920in}{0.773588in}}%
\pgfpathlineto{\pgfqpoint{26.417231in}{0.773588in}}%
\pgfpathlineto{\pgfqpoint{26.487420in}{0.773588in}}%
\pgfpathlineto{\pgfqpoint{26.557235in}{0.773588in}}%
\pgfpathlineto{\pgfqpoint{26.629572in}{0.773588in}}%
\pgfpathlineto{\pgfqpoint{26.699584in}{0.773588in}}%
\pgfpathlineto{\pgfqpoint{26.769271in}{0.773588in}}%
\pgfpathlineto{\pgfqpoint{26.841234in}{0.773588in}}%
\pgfpathlineto{\pgfqpoint{26.912667in}{0.773588in}}%
\pgfpathlineto{\pgfqpoint{26.983641in}{0.773588in}}%
\pgfpathlineto{\pgfqpoint{27.056835in}{0.773588in}}%
\pgfpathlineto{\pgfqpoint{27.128948in}{0.773588in}}%
\pgfpathlineto{\pgfqpoint{27.201477in}{0.773588in}}%
\pgfpathlineto{\pgfqpoint{27.277488in}{0.773588in}}%
\pgfpathlineto{\pgfqpoint{27.350990in}{0.773588in}}%
\pgfpathlineto{\pgfqpoint{27.423884in}{0.773588in}}%
\pgfpathlineto{\pgfqpoint{27.500063in}{0.773588in}}%
\pgfpathlineto{\pgfqpoint{27.574929in}{0.773588in}}%
\pgfpathlineto{\pgfqpoint{27.649072in}{0.773588in}}%
\pgfpathlineto{\pgfqpoint{27.724006in}{0.773588in}}%
\pgfpathlineto{\pgfqpoint{27.795343in}{0.773588in}}%
\pgfpathlineto{\pgfqpoint{27.868117in}{0.773588in}}%
\pgfpathlineto{\pgfqpoint{27.943911in}{0.773588in}}%
\pgfpathlineto{\pgfqpoint{28.018234in}{0.773588in}}%
\pgfpathlineto{\pgfqpoint{28.090360in}{0.773588in}}%
\pgfpathlineto{\pgfqpoint{28.163352in}{0.773588in}}%
\pgfpathlineto{\pgfqpoint{28.234559in}{0.773588in}}%
\pgfpathlineto{\pgfqpoint{28.306604in}{0.773588in}}%
\pgfpathlineto{\pgfqpoint{28.380501in}{0.773588in}}%
\pgfpathlineto{\pgfqpoint{28.451943in}{0.773588in}}%
\pgfpathlineto{\pgfqpoint{28.522534in}{0.773588in}}%
\pgfpathlineto{\pgfqpoint{28.596041in}{0.773588in}}%
\pgfpathlineto{\pgfqpoint{28.668204in}{0.773588in}}%
\pgfpathlineto{\pgfqpoint{28.738605in}{0.773588in}}%
\pgfpathlineto{\pgfqpoint{28.811911in}{0.773588in}}%
\pgfpathlineto{\pgfqpoint{28.885217in}{0.773588in}}%
\pgfpathlineto{\pgfqpoint{28.956832in}{0.773588in}}%
\pgfpathlineto{\pgfqpoint{29.029887in}{0.773588in}}%
\pgfpathlineto{\pgfqpoint{29.100748in}{0.773588in}}%
\pgfpathlineto{\pgfqpoint{29.173201in}{0.773588in}}%
\pgfpathlineto{\pgfqpoint{29.248973in}{0.773588in}}%
\pgfpathlineto{\pgfqpoint{29.320759in}{0.773588in}}%
\pgfpathlineto{\pgfqpoint{29.393660in}{0.773588in}}%
\pgfpathlineto{\pgfqpoint{29.467898in}{0.773588in}}%
\pgfpathlineto{\pgfqpoint{29.540420in}{0.773588in}}%
\pgfpathlineto{\pgfqpoint{29.611700in}{0.773588in}}%
\pgfpathlineto{\pgfqpoint{29.684427in}{0.773588in}}%
\pgfpathlineto{\pgfqpoint{29.755113in}{0.773588in}}%
\pgfpathlineto{\pgfqpoint{29.827132in}{0.773588in}}%
\pgfpathlineto{\pgfqpoint{29.901656in}{0.773588in}}%
\pgfpathlineto{\pgfqpoint{29.974646in}{0.773588in}}%
\pgfpathlineto{\pgfqpoint{30.048252in}{0.773588in}}%
\pgfpathlineto{\pgfqpoint{30.122796in}{0.773588in}}%
\pgfpathlineto{\pgfqpoint{30.195443in}{0.773588in}}%
\pgfpathlineto{\pgfqpoint{30.269036in}{0.773588in}}%
\pgfpathlineto{\pgfqpoint{30.344328in}{0.773588in}}%
\pgfpathlineto{\pgfqpoint{30.417098in}{0.773588in}}%
\pgfpathlineto{\pgfqpoint{30.488991in}{0.773588in}}%
\pgfpathlineto{\pgfqpoint{30.562714in}{0.773588in}}%
\pgfpathlineto{\pgfqpoint{30.634099in}{0.773588in}}%
\pgfpathlineto{\pgfqpoint{30.707828in}{0.773588in}}%
\pgfpathlineto{\pgfqpoint{30.782249in}{0.773588in}}%
\pgfpathlineto{\pgfqpoint{30.854115in}{0.773588in}}%
\pgfpathlineto{\pgfqpoint{30.928305in}{0.773588in}}%
\pgfpathlineto{\pgfqpoint{31.002514in}{0.773588in}}%
\pgfpathlineto{\pgfqpoint{31.074452in}{0.773588in}}%
\pgfpathlineto{\pgfqpoint{31.147740in}{0.773588in}}%
\pgfpathlineto{\pgfqpoint{31.222913in}{0.773588in}}%
\pgfpathlineto{\pgfqpoint{31.294777in}{0.773588in}}%
\pgfpathlineto{\pgfqpoint{31.366613in}{0.773588in}}%
\pgfpathlineto{\pgfqpoint{31.439415in}{0.773588in}}%
\pgfpathlineto{\pgfqpoint{31.510140in}{0.773588in}}%
\pgfpathlineto{\pgfqpoint{31.582282in}{0.773588in}}%
\pgfpathlineto{\pgfqpoint{31.656180in}{0.773588in}}%
\pgfpathlineto{\pgfqpoint{31.728521in}{0.773588in}}%
\pgfpathlineto{\pgfqpoint{31.800877in}{0.773588in}}%
\pgfpathlineto{\pgfqpoint{31.873539in}{0.773588in}}%
\pgfpathlineto{\pgfqpoint{31.943734in}{0.773588in}}%
\pgfpathlineto{\pgfqpoint{32.015122in}{0.773588in}}%
\pgfpathlineto{\pgfqpoint{32.089684in}{0.773588in}}%
\pgfpathlineto{\pgfqpoint{32.161504in}{0.773588in}}%
\pgfpathlineto{\pgfqpoint{32.231773in}{0.773588in}}%
\pgfpathlineto{\pgfqpoint{32.305440in}{0.773588in}}%
\pgfpathlineto{\pgfqpoint{32.377016in}{0.773588in}}%
\pgfpathlineto{\pgfqpoint{32.447439in}{0.773588in}}%
\pgfpathlineto{\pgfqpoint{32.520401in}{0.773588in}}%
\pgfpathlineto{\pgfqpoint{32.590674in}{0.773588in}}%
\pgfpathlineto{\pgfqpoint{32.663709in}{0.773588in}}%
\pgfpathlineto{\pgfqpoint{32.740263in}{0.773588in}}%
\pgfpathlineto{\pgfqpoint{32.813546in}{0.773588in}}%
\pgfpathlineto{\pgfqpoint{32.887492in}{0.773588in}}%
\pgfpathlineto{\pgfqpoint{32.963168in}{0.773588in}}%
\pgfpathlineto{\pgfqpoint{33.037794in}{0.773588in}}%
\pgfpathlineto{\pgfqpoint{33.110479in}{0.773588in}}%
\pgfpathlineto{\pgfqpoint{33.185787in}{0.773588in}}%
\pgfpathlineto{\pgfqpoint{33.259507in}{0.773588in}}%
\pgfpathlineto{\pgfqpoint{33.333311in}{0.773588in}}%
\pgfpathlineto{\pgfqpoint{33.409286in}{0.773588in}}%
\pgfpathlineto{\pgfqpoint{33.483328in}{0.773588in}}%
\pgfpathlineto{\pgfqpoint{33.557012in}{0.773588in}}%
\pgfpathlineto{\pgfqpoint{33.631884in}{0.773588in}}%
\pgfpathlineto{\pgfqpoint{33.703848in}{0.773588in}}%
\pgfpathlineto{\pgfqpoint{33.776888in}{0.773588in}}%
\pgfpathlineto{\pgfqpoint{33.852393in}{0.773588in}}%
\pgfpathlineto{\pgfqpoint{33.923536in}{0.773588in}}%
\pgfpathlineto{\pgfqpoint{33.994648in}{0.773588in}}%
\pgfpathlineto{\pgfqpoint{34.067999in}{0.773588in}}%
\pgfpathlineto{\pgfqpoint{34.138346in}{0.773588in}}%
\pgfpathlineto{\pgfqpoint{34.210760in}{0.773588in}}%
\pgfpathlineto{\pgfqpoint{34.284339in}{0.773588in}}%
\pgfpathlineto{\pgfqpoint{34.354648in}{0.773588in}}%
\pgfpathlineto{\pgfqpoint{34.425604in}{0.773588in}}%
\pgfpathlineto{\pgfqpoint{34.499162in}{0.773588in}}%
\pgfpathlineto{\pgfqpoint{34.571449in}{0.773588in}}%
\pgfpathlineto{\pgfqpoint{34.643977in}{0.773588in}}%
\pgfpathlineto{\pgfqpoint{34.718731in}{0.773588in}}%
\pgfpathlineto{\pgfqpoint{34.789698in}{0.773588in}}%
\pgfpathlineto{\pgfqpoint{34.862212in}{0.773588in}}%
\pgfpathlineto{\pgfqpoint{34.936943in}{0.773588in}}%
\pgfpathlineto{\pgfqpoint{35.007838in}{0.773588in}}%
\pgfpathlineto{\pgfqpoint{35.080154in}{0.773588in}}%
\pgfpathlineto{\pgfqpoint{35.155466in}{0.773588in}}%
\pgfpathlineto{\pgfqpoint{35.227201in}{0.773588in}}%
\pgfpathlineto{\pgfqpoint{35.298174in}{0.773588in}}%
\pgfpathlineto{\pgfqpoint{35.372990in}{0.773588in}}%
\pgfpathlineto{\pgfqpoint{35.451774in}{0.773588in}}%
\pgfpathlineto{\pgfqpoint{35.574549in}{0.773588in}}%
\pgfpathlineto{\pgfqpoint{35.663523in}{0.773588in}}%
\pgfpathlineto{\pgfqpoint{35.741519in}{0.773588in}}%
\pgfpathlineto{\pgfqpoint{35.805568in}{1.591997in}}%
\pgfpathlineto{\pgfqpoint{35.870813in}{3.700065in}}%
\pgfpathlineto{\pgfqpoint{35.942832in}{3.811738in}}%
\pgfpathlineto{\pgfqpoint{36.012796in}{3.977733in}}%
\pgfpathlineto{\pgfqpoint{36.085094in}{3.810697in}}%
\pgfpathlineto{\pgfqpoint{36.154695in}{3.912811in}}%
\pgfpathlineto{\pgfqpoint{36.223624in}{3.948574in}}%
\pgfpathlineto{\pgfqpoint{36.293479in}{3.999837in}}%
\pgfpathlineto{\pgfqpoint{36.360634in}{4.001718in}}%
\pgfpathlineto{\pgfqpoint{36.428206in}{4.007261in}}%
\pgfpathlineto{\pgfqpoint{36.497087in}{3.951551in}}%
\pgfpathlineto{\pgfqpoint{36.563097in}{4.110835in}}%
\pgfpathlineto{\pgfqpoint{36.628950in}{4.085256in}}%
\pgfpathlineto{\pgfqpoint{36.696952in}{4.115543in}}%
\pgfpathlineto{\pgfqpoint{36.761893in}{4.151117in}}%
\pgfpathlineto{\pgfqpoint{36.827337in}{4.095111in}}%
\pgfpathlineto{\pgfqpoint{36.893714in}{4.215547in}}%
\pgfpathlineto{\pgfqpoint{36.957470in}{4.207643in}}%
\pgfpathlineto{\pgfqpoint{37.022217in}{4.115512in}}%
\pgfpathlineto{\pgfqpoint{37.088015in}{4.257509in}}%
\pgfpathlineto{\pgfqpoint{37.151827in}{4.172457in}}%
\pgfpathlineto{\pgfqpoint{37.151827in}{5.872359in}}%
\pgfpathlineto{\pgfqpoint{37.151827in}{5.872359in}}%
\pgfpathlineto{\pgfqpoint{37.088015in}{5.930845in}}%
\pgfpathlineto{\pgfqpoint{37.022217in}{5.814953in}}%
\pgfpathlineto{\pgfqpoint{36.957470in}{5.879596in}}%
\pgfpathlineto{\pgfqpoint{36.893714in}{5.875626in}}%
\pgfpathlineto{\pgfqpoint{36.827337in}{5.772285in}}%
\pgfpathlineto{\pgfqpoint{36.761893in}{5.793255in}}%
\pgfpathlineto{\pgfqpoint{36.696952in}{5.790749in}}%
\pgfpathlineto{\pgfqpoint{36.628950in}{5.732424in}}%
\pgfpathlineto{\pgfqpoint{36.563097in}{5.748599in}}%
\pgfpathlineto{\pgfqpoint{36.497087in}{5.625548in}}%
\pgfpathlineto{\pgfqpoint{36.428206in}{5.661488in}}%
\pgfpathlineto{\pgfqpoint{36.360634in}{5.619869in}}%
\pgfpathlineto{\pgfqpoint{36.293479in}{5.638191in}}%
\pgfpathlineto{\pgfqpoint{36.223624in}{5.539128in}}%
\pgfpathlineto{\pgfqpoint{36.154695in}{5.488048in}}%
\pgfpathlineto{\pgfqpoint{36.085094in}{5.399336in}}%
\pgfpathlineto{\pgfqpoint{36.012796in}{5.506716in}}%
\pgfpathlineto{\pgfqpoint{35.942832in}{5.319244in}}%
\pgfpathlineto{\pgfqpoint{35.870813in}{5.187514in}}%
\pgfpathlineto{\pgfqpoint{35.805568in}{1.724813in}}%
\pgfpathlineto{\pgfqpoint{35.741519in}{0.773588in}}%
\pgfpathlineto{\pgfqpoint{35.663523in}{0.773588in}}%
\pgfpathlineto{\pgfqpoint{35.574549in}{0.773588in}}%
\pgfpathlineto{\pgfqpoint{35.451774in}{0.773588in}}%
\pgfpathlineto{\pgfqpoint{35.372990in}{0.773588in}}%
\pgfpathlineto{\pgfqpoint{35.298174in}{0.773588in}}%
\pgfpathlineto{\pgfqpoint{35.227201in}{0.773588in}}%
\pgfpathlineto{\pgfqpoint{35.155466in}{0.773588in}}%
\pgfpathlineto{\pgfqpoint{35.080154in}{0.773588in}}%
\pgfpathlineto{\pgfqpoint{35.007838in}{0.773588in}}%
\pgfpathlineto{\pgfqpoint{34.936943in}{0.773588in}}%
\pgfpathlineto{\pgfqpoint{34.862212in}{0.773588in}}%
\pgfpathlineto{\pgfqpoint{34.789698in}{0.773588in}}%
\pgfpathlineto{\pgfqpoint{34.718731in}{0.773588in}}%
\pgfpathlineto{\pgfqpoint{34.643977in}{0.773588in}}%
\pgfpathlineto{\pgfqpoint{34.571449in}{0.773588in}}%
\pgfpathlineto{\pgfqpoint{34.499162in}{0.773588in}}%
\pgfpathlineto{\pgfqpoint{34.425604in}{0.773588in}}%
\pgfpathlineto{\pgfqpoint{34.354648in}{0.773588in}}%
\pgfpathlineto{\pgfqpoint{34.284339in}{0.773588in}}%
\pgfpathlineto{\pgfqpoint{34.210760in}{0.773588in}}%
\pgfpathlineto{\pgfqpoint{34.138346in}{0.773588in}}%
\pgfpathlineto{\pgfqpoint{34.067999in}{0.773588in}}%
\pgfpathlineto{\pgfqpoint{33.994648in}{0.773588in}}%
\pgfpathlineto{\pgfqpoint{33.923536in}{0.773588in}}%
\pgfpathlineto{\pgfqpoint{33.852393in}{0.773588in}}%
\pgfpathlineto{\pgfqpoint{33.776888in}{0.773588in}}%
\pgfpathlineto{\pgfqpoint{33.703848in}{0.773588in}}%
\pgfpathlineto{\pgfqpoint{33.631884in}{0.773588in}}%
\pgfpathlineto{\pgfqpoint{33.557012in}{0.773588in}}%
\pgfpathlineto{\pgfqpoint{33.483328in}{0.773588in}}%
\pgfpathlineto{\pgfqpoint{33.409286in}{0.773588in}}%
\pgfpathlineto{\pgfqpoint{33.333311in}{0.773588in}}%
\pgfpathlineto{\pgfqpoint{33.259507in}{0.773588in}}%
\pgfpathlineto{\pgfqpoint{33.185787in}{0.773588in}}%
\pgfpathlineto{\pgfqpoint{33.110479in}{0.773588in}}%
\pgfpathlineto{\pgfqpoint{33.037794in}{0.773588in}}%
\pgfpathlineto{\pgfqpoint{32.963168in}{0.773588in}}%
\pgfpathlineto{\pgfqpoint{32.887492in}{0.773588in}}%
\pgfpathlineto{\pgfqpoint{32.813546in}{0.773588in}}%
\pgfpathlineto{\pgfqpoint{32.740263in}{0.773588in}}%
\pgfpathlineto{\pgfqpoint{32.663709in}{0.773588in}}%
\pgfpathlineto{\pgfqpoint{32.590674in}{0.773588in}}%
\pgfpathlineto{\pgfqpoint{32.520401in}{0.773588in}}%
\pgfpathlineto{\pgfqpoint{32.447439in}{0.773588in}}%
\pgfpathlineto{\pgfqpoint{32.377016in}{0.773588in}}%
\pgfpathlineto{\pgfqpoint{32.305440in}{0.773588in}}%
\pgfpathlineto{\pgfqpoint{32.231773in}{0.773588in}}%
\pgfpathlineto{\pgfqpoint{32.161504in}{0.773588in}}%
\pgfpathlineto{\pgfqpoint{32.089684in}{0.773588in}}%
\pgfpathlineto{\pgfqpoint{32.015122in}{0.773588in}}%
\pgfpathlineto{\pgfqpoint{31.943734in}{0.773588in}}%
\pgfpathlineto{\pgfqpoint{31.873539in}{0.773588in}}%
\pgfpathlineto{\pgfqpoint{31.800877in}{0.773588in}}%
\pgfpathlineto{\pgfqpoint{31.728521in}{0.773588in}}%
\pgfpathlineto{\pgfqpoint{31.656180in}{0.773588in}}%
\pgfpathlineto{\pgfqpoint{31.582282in}{0.773588in}}%
\pgfpathlineto{\pgfqpoint{31.510140in}{0.773588in}}%
\pgfpathlineto{\pgfqpoint{31.439415in}{0.773588in}}%
\pgfpathlineto{\pgfqpoint{31.366613in}{0.773588in}}%
\pgfpathlineto{\pgfqpoint{31.294777in}{0.773588in}}%
\pgfpathlineto{\pgfqpoint{31.222913in}{0.773588in}}%
\pgfpathlineto{\pgfqpoint{31.147740in}{0.773588in}}%
\pgfpathlineto{\pgfqpoint{31.074452in}{0.773588in}}%
\pgfpathlineto{\pgfqpoint{31.002514in}{0.773588in}}%
\pgfpathlineto{\pgfqpoint{30.928305in}{0.773588in}}%
\pgfpathlineto{\pgfqpoint{30.854115in}{0.773588in}}%
\pgfpathlineto{\pgfqpoint{30.782249in}{0.773588in}}%
\pgfpathlineto{\pgfqpoint{30.707828in}{0.773588in}}%
\pgfpathlineto{\pgfqpoint{30.634099in}{0.773588in}}%
\pgfpathlineto{\pgfqpoint{30.562714in}{0.773588in}}%
\pgfpathlineto{\pgfqpoint{30.488991in}{0.773588in}}%
\pgfpathlineto{\pgfqpoint{30.417098in}{0.773588in}}%
\pgfpathlineto{\pgfqpoint{30.344328in}{0.773588in}}%
\pgfpathlineto{\pgfqpoint{30.269036in}{0.773588in}}%
\pgfpathlineto{\pgfqpoint{30.195443in}{0.773588in}}%
\pgfpathlineto{\pgfqpoint{30.122796in}{0.773588in}}%
\pgfpathlineto{\pgfqpoint{30.048252in}{0.773588in}}%
\pgfpathlineto{\pgfqpoint{29.974646in}{0.773588in}}%
\pgfpathlineto{\pgfqpoint{29.901656in}{0.773588in}}%
\pgfpathlineto{\pgfqpoint{29.827132in}{0.773588in}}%
\pgfpathlineto{\pgfqpoint{29.755113in}{0.773588in}}%
\pgfpathlineto{\pgfqpoint{29.684427in}{0.773588in}}%
\pgfpathlineto{\pgfqpoint{29.611700in}{0.773588in}}%
\pgfpathlineto{\pgfqpoint{29.540420in}{0.773588in}}%
\pgfpathlineto{\pgfqpoint{29.467898in}{0.773588in}}%
\pgfpathlineto{\pgfqpoint{29.393660in}{0.773588in}}%
\pgfpathlineto{\pgfqpoint{29.320759in}{0.773588in}}%
\pgfpathlineto{\pgfqpoint{29.248973in}{0.773588in}}%
\pgfpathlineto{\pgfqpoint{29.173201in}{0.773588in}}%
\pgfpathlineto{\pgfqpoint{29.100748in}{0.773588in}}%
\pgfpathlineto{\pgfqpoint{29.029887in}{0.773588in}}%
\pgfpathlineto{\pgfqpoint{28.956832in}{0.773588in}}%
\pgfpathlineto{\pgfqpoint{28.885217in}{0.773588in}}%
\pgfpathlineto{\pgfqpoint{28.811911in}{0.773588in}}%
\pgfpathlineto{\pgfqpoint{28.738605in}{0.773588in}}%
\pgfpathlineto{\pgfqpoint{28.668204in}{0.773588in}}%
\pgfpathlineto{\pgfqpoint{28.596041in}{0.773588in}}%
\pgfpathlineto{\pgfqpoint{28.522534in}{0.773588in}}%
\pgfpathlineto{\pgfqpoint{28.451943in}{0.773588in}}%
\pgfpathlineto{\pgfqpoint{28.380501in}{0.773588in}}%
\pgfpathlineto{\pgfqpoint{28.306604in}{0.773588in}}%
\pgfpathlineto{\pgfqpoint{28.234559in}{0.773588in}}%
\pgfpathlineto{\pgfqpoint{28.163352in}{0.773588in}}%
\pgfpathlineto{\pgfqpoint{28.090360in}{0.773588in}}%
\pgfpathlineto{\pgfqpoint{28.018234in}{0.773588in}}%
\pgfpathlineto{\pgfqpoint{27.943911in}{0.773588in}}%
\pgfpathlineto{\pgfqpoint{27.868117in}{0.773588in}}%
\pgfpathlineto{\pgfqpoint{27.795343in}{0.773588in}}%
\pgfpathlineto{\pgfqpoint{27.724006in}{0.773588in}}%
\pgfpathlineto{\pgfqpoint{27.649072in}{0.773588in}}%
\pgfpathlineto{\pgfqpoint{27.574929in}{0.773588in}}%
\pgfpathlineto{\pgfqpoint{27.500063in}{0.773588in}}%
\pgfpathlineto{\pgfqpoint{27.423884in}{0.773588in}}%
\pgfpathlineto{\pgfqpoint{27.350990in}{0.773588in}}%
\pgfpathlineto{\pgfqpoint{27.277488in}{0.773588in}}%
\pgfpathlineto{\pgfqpoint{27.201477in}{0.773588in}}%
\pgfpathlineto{\pgfqpoint{27.128948in}{0.773588in}}%
\pgfpathlineto{\pgfqpoint{27.056835in}{0.773588in}}%
\pgfpathlineto{\pgfqpoint{26.983641in}{0.773588in}}%
\pgfpathlineto{\pgfqpoint{26.912667in}{0.773588in}}%
\pgfpathlineto{\pgfqpoint{26.841234in}{0.773588in}}%
\pgfpathlineto{\pgfqpoint{26.769271in}{0.773588in}}%
\pgfpathlineto{\pgfqpoint{26.699584in}{0.773588in}}%
\pgfpathlineto{\pgfqpoint{26.629572in}{0.773588in}}%
\pgfpathlineto{\pgfqpoint{26.557235in}{0.773588in}}%
\pgfpathlineto{\pgfqpoint{26.487420in}{0.773588in}}%
\pgfpathlineto{\pgfqpoint{26.417231in}{0.773588in}}%
\pgfpathlineto{\pgfqpoint{26.344920in}{0.773588in}}%
\pgfpathlineto{\pgfqpoint{26.274267in}{0.773588in}}%
\pgfpathlineto{\pgfqpoint{26.203572in}{0.773588in}}%
\pgfpathlineto{\pgfqpoint{26.130346in}{0.773588in}}%
\pgfpathlineto{\pgfqpoint{26.058621in}{0.773588in}}%
\pgfpathlineto{\pgfqpoint{25.988588in}{0.773588in}}%
\pgfpathlineto{\pgfqpoint{25.916551in}{0.773588in}}%
\pgfpathlineto{\pgfqpoint{25.845667in}{0.773588in}}%
\pgfpathlineto{\pgfqpoint{25.775695in}{0.773588in}}%
\pgfpathlineto{\pgfqpoint{25.702341in}{0.773588in}}%
\pgfpathlineto{\pgfqpoint{25.631022in}{0.773588in}}%
\pgfpathlineto{\pgfqpoint{25.560110in}{0.773588in}}%
\pgfpathlineto{\pgfqpoint{25.487573in}{0.773588in}}%
\pgfpathlineto{\pgfqpoint{25.417047in}{0.773588in}}%
\pgfpathlineto{\pgfqpoint{25.347124in}{0.773588in}}%
\pgfpathlineto{\pgfqpoint{25.273349in}{0.773588in}}%
\pgfpathlineto{\pgfqpoint{25.203216in}{0.773588in}}%
\pgfpathlineto{\pgfqpoint{25.131109in}{0.773588in}}%
\pgfpathlineto{\pgfqpoint{25.055567in}{0.773588in}}%
\pgfpathlineto{\pgfqpoint{24.983525in}{0.773588in}}%
\pgfpathlineto{\pgfqpoint{24.911741in}{0.773588in}}%
\pgfpathlineto{\pgfqpoint{24.836992in}{0.773588in}}%
\pgfpathlineto{\pgfqpoint{24.764742in}{0.773588in}}%
\pgfpathlineto{\pgfqpoint{24.691660in}{0.773588in}}%
\pgfpathlineto{\pgfqpoint{24.618678in}{0.773588in}}%
\pgfpathlineto{\pgfqpoint{24.548639in}{0.773588in}}%
\pgfpathlineto{\pgfqpoint{24.476339in}{0.773588in}}%
\pgfpathlineto{\pgfqpoint{24.400328in}{0.773588in}}%
\pgfpathlineto{\pgfqpoint{24.329090in}{0.773588in}}%
\pgfpathlineto{\pgfqpoint{24.257307in}{0.773588in}}%
\pgfpathlineto{\pgfqpoint{24.183899in}{0.773588in}}%
\pgfpathlineto{\pgfqpoint{24.111784in}{0.773588in}}%
\pgfpathlineto{\pgfqpoint{24.039255in}{0.773588in}}%
\pgfpathlineto{\pgfqpoint{23.966722in}{0.773588in}}%
\pgfpathlineto{\pgfqpoint{23.895213in}{0.773588in}}%
\pgfpathlineto{\pgfqpoint{23.824951in}{0.773588in}}%
\pgfpathlineto{\pgfqpoint{23.752361in}{0.773588in}}%
\pgfpathlineto{\pgfqpoint{23.681448in}{0.773588in}}%
\pgfpathlineto{\pgfqpoint{23.610036in}{0.773588in}}%
\pgfpathlineto{\pgfqpoint{23.537850in}{0.773588in}}%
\pgfpathlineto{\pgfqpoint{23.467323in}{0.773588in}}%
\pgfpathlineto{\pgfqpoint{23.396126in}{0.773588in}}%
\pgfpathlineto{\pgfqpoint{23.323995in}{0.773588in}}%
\pgfpathlineto{\pgfqpoint{23.253484in}{0.773588in}}%
\pgfpathlineto{\pgfqpoint{23.184270in}{0.773588in}}%
\pgfpathlineto{\pgfqpoint{23.113267in}{0.773588in}}%
\pgfpathlineto{\pgfqpoint{23.043397in}{0.773588in}}%
\pgfpathlineto{\pgfqpoint{22.973868in}{0.773588in}}%
\pgfpathlineto{\pgfqpoint{22.902896in}{0.773588in}}%
\pgfpathlineto{\pgfqpoint{22.833628in}{0.773588in}}%
\pgfpathlineto{\pgfqpoint{22.764651in}{0.773588in}}%
\pgfpathlineto{\pgfqpoint{22.694016in}{0.773588in}}%
\pgfpathlineto{\pgfqpoint{22.624114in}{0.773588in}}%
\pgfpathlineto{\pgfqpoint{22.553535in}{0.773588in}}%
\pgfpathlineto{\pgfqpoint{22.482516in}{0.773588in}}%
\pgfpathlineto{\pgfqpoint{22.413449in}{0.773588in}}%
\pgfpathlineto{\pgfqpoint{22.343331in}{0.773588in}}%
\pgfpathlineto{\pgfqpoint{22.268824in}{0.773588in}}%
\pgfpathlineto{\pgfqpoint{22.195707in}{0.773588in}}%
\pgfpathlineto{\pgfqpoint{22.122462in}{0.773588in}}%
\pgfpathlineto{\pgfqpoint{22.048040in}{0.773588in}}%
\pgfpathlineto{\pgfqpoint{21.976836in}{0.773588in}}%
\pgfpathlineto{\pgfqpoint{21.903868in}{0.773588in}}%
\pgfpathlineto{\pgfqpoint{21.828555in}{0.773588in}}%
\pgfpathlineto{\pgfqpoint{21.756227in}{0.773588in}}%
\pgfpathlineto{\pgfqpoint{21.683994in}{0.773588in}}%
\pgfpathlineto{\pgfqpoint{21.610878in}{0.773588in}}%
\pgfpathlineto{\pgfqpoint{21.539601in}{0.773588in}}%
\pgfpathlineto{\pgfqpoint{21.467741in}{0.773588in}}%
\pgfpathlineto{\pgfqpoint{21.394839in}{0.773588in}}%
\pgfpathlineto{\pgfqpoint{21.324498in}{0.773588in}}%
\pgfpathlineto{\pgfqpoint{21.254645in}{0.773588in}}%
\pgfpathlineto{\pgfqpoint{21.181233in}{0.773588in}}%
\pgfpathlineto{\pgfqpoint{21.110656in}{0.773588in}}%
\pgfpathlineto{\pgfqpoint{21.040864in}{0.773588in}}%
\pgfpathlineto{\pgfqpoint{20.969373in}{0.773588in}}%
\pgfpathlineto{\pgfqpoint{20.899587in}{0.773588in}}%
\pgfpathlineto{\pgfqpoint{20.830586in}{0.773588in}}%
\pgfpathlineto{\pgfqpoint{20.758814in}{0.773588in}}%
\pgfpathlineto{\pgfqpoint{20.688980in}{0.773588in}}%
\pgfpathlineto{\pgfqpoint{20.618428in}{0.773588in}}%
\pgfpathlineto{\pgfqpoint{20.547172in}{0.773588in}}%
\pgfpathlineto{\pgfqpoint{20.476961in}{0.773588in}}%
\pgfpathlineto{\pgfqpoint{20.407276in}{0.773588in}}%
\pgfpathlineto{\pgfqpoint{20.334605in}{0.773588in}}%
\pgfpathlineto{\pgfqpoint{20.263963in}{0.773588in}}%
\pgfpathlineto{\pgfqpoint{20.193561in}{0.773588in}}%
\pgfpathlineto{\pgfqpoint{20.121885in}{0.773588in}}%
\pgfpathlineto{\pgfqpoint{20.052071in}{0.773588in}}%
\pgfpathlineto{\pgfqpoint{19.981199in}{0.773588in}}%
\pgfpathlineto{\pgfqpoint{19.907049in}{0.773588in}}%
\pgfpathlineto{\pgfqpoint{19.835928in}{0.773588in}}%
\pgfpathlineto{\pgfqpoint{19.766060in}{0.773588in}}%
\pgfpathlineto{\pgfqpoint{19.693765in}{0.773588in}}%
\pgfpathlineto{\pgfqpoint{19.623855in}{0.773588in}}%
\pgfpathlineto{\pgfqpoint{19.553641in}{0.773588in}}%
\pgfpathlineto{\pgfqpoint{19.482791in}{0.773588in}}%
\pgfpathlineto{\pgfqpoint{19.412352in}{0.773588in}}%
\pgfpathlineto{\pgfqpoint{19.340700in}{0.773588in}}%
\pgfpathlineto{\pgfqpoint{19.266961in}{0.773588in}}%
\pgfpathlineto{\pgfqpoint{19.195134in}{0.773588in}}%
\pgfpathlineto{\pgfqpoint{19.123240in}{0.773588in}}%
\pgfpathlineto{\pgfqpoint{19.048835in}{0.773588in}}%
\pgfpathlineto{\pgfqpoint{18.977585in}{0.773588in}}%
\pgfpathlineto{\pgfqpoint{18.906958in}{0.773588in}}%
\pgfpathlineto{\pgfqpoint{18.834141in}{0.773588in}}%
\pgfpathlineto{\pgfqpoint{18.763558in}{0.773588in}}%
\pgfpathlineto{\pgfqpoint{18.692839in}{0.773588in}}%
\pgfpathlineto{\pgfqpoint{18.619849in}{0.773588in}}%
\pgfpathlineto{\pgfqpoint{18.549509in}{0.773588in}}%
\pgfpathlineto{\pgfqpoint{18.479143in}{0.773588in}}%
\pgfpathlineto{\pgfqpoint{18.406899in}{0.773588in}}%
\pgfpathlineto{\pgfqpoint{18.336223in}{0.773588in}}%
\pgfpathlineto{\pgfqpoint{18.266780in}{0.773588in}}%
\pgfpathlineto{\pgfqpoint{18.195268in}{0.773588in}}%
\pgfpathlineto{\pgfqpoint{18.125960in}{0.773588in}}%
\pgfpathlineto{\pgfqpoint{18.056086in}{0.773588in}}%
\pgfpathlineto{\pgfqpoint{17.983524in}{0.773588in}}%
\pgfpathlineto{\pgfqpoint{17.912627in}{0.773588in}}%
\pgfpathlineto{\pgfqpoint{17.844148in}{0.773588in}}%
\pgfpathlineto{\pgfqpoint{17.772898in}{0.773588in}}%
\pgfpathlineto{\pgfqpoint{17.703480in}{0.773588in}}%
\pgfpathlineto{\pgfqpoint{17.634782in}{0.773588in}}%
\pgfpathlineto{\pgfqpoint{17.563254in}{0.773588in}}%
\pgfpathlineto{\pgfqpoint{17.494859in}{0.773588in}}%
\pgfpathlineto{\pgfqpoint{17.426570in}{0.773588in}}%
\pgfpathlineto{\pgfqpoint{17.356523in}{0.773588in}}%
\pgfpathlineto{\pgfqpoint{17.288800in}{0.773588in}}%
\pgfpathlineto{\pgfqpoint{17.220960in}{0.773588in}}%
\pgfpathlineto{\pgfqpoint{17.150280in}{0.773588in}}%
\pgfpathlineto{\pgfqpoint{17.081113in}{0.773588in}}%
\pgfpathlineto{\pgfqpoint{17.013249in}{0.773588in}}%
\pgfpathlineto{\pgfqpoint{16.942299in}{0.773588in}}%
\pgfpathlineto{\pgfqpoint{16.872974in}{0.773588in}}%
\pgfpathlineto{\pgfqpoint{16.803402in}{0.773588in}}%
\pgfpathlineto{\pgfqpoint{16.731829in}{0.773588in}}%
\pgfpathlineto{\pgfqpoint{16.662798in}{0.773588in}}%
\pgfpathlineto{\pgfqpoint{16.592371in}{0.773588in}}%
\pgfpathlineto{\pgfqpoint{16.519393in}{0.773588in}}%
\pgfpathlineto{\pgfqpoint{16.445981in}{0.773588in}}%
\pgfpathlineto{\pgfqpoint{16.373054in}{0.773588in}}%
\pgfpathlineto{\pgfqpoint{16.299236in}{0.773588in}}%
\pgfpathlineto{\pgfqpoint{16.229236in}{0.773588in}}%
\pgfpathlineto{\pgfqpoint{16.160188in}{0.773588in}}%
\pgfpathlineto{\pgfqpoint{16.088678in}{0.773588in}}%
\pgfpathlineto{\pgfqpoint{16.019598in}{0.773588in}}%
\pgfpathlineto{\pgfqpoint{15.949677in}{0.773588in}}%
\pgfpathlineto{\pgfqpoint{15.878678in}{0.773588in}}%
\pgfpathlineto{\pgfqpoint{15.807467in}{0.773588in}}%
\pgfpathlineto{\pgfqpoint{15.737902in}{0.773588in}}%
\pgfpathlineto{\pgfqpoint{15.665480in}{0.773588in}}%
\pgfpathlineto{\pgfqpoint{15.595878in}{0.773588in}}%
\pgfpathlineto{\pgfqpoint{15.527961in}{0.773588in}}%
\pgfpathlineto{\pgfqpoint{15.457342in}{0.773588in}}%
\pgfpathlineto{\pgfqpoint{15.388582in}{0.773588in}}%
\pgfpathlineto{\pgfqpoint{15.320168in}{0.773588in}}%
\pgfpathlineto{\pgfqpoint{15.249597in}{0.773588in}}%
\pgfpathlineto{\pgfqpoint{15.182512in}{0.773588in}}%
\pgfpathlineto{\pgfqpoint{15.115140in}{0.773588in}}%
\pgfpathlineto{\pgfqpoint{15.044439in}{0.773588in}}%
\pgfpathlineto{\pgfqpoint{14.976372in}{0.773588in}}%
\pgfpathlineto{\pgfqpoint{14.908199in}{0.773588in}}%
\pgfpathlineto{\pgfqpoint{14.836607in}{0.773588in}}%
\pgfpathlineto{\pgfqpoint{14.768225in}{0.773588in}}%
\pgfpathlineto{\pgfqpoint{14.700003in}{0.773588in}}%
\pgfpathlineto{\pgfqpoint{14.630523in}{0.773588in}}%
\pgfpathlineto{\pgfqpoint{14.562284in}{0.773588in}}%
\pgfpathlineto{\pgfqpoint{14.494110in}{0.773588in}}%
\pgfpathlineto{\pgfqpoint{14.424138in}{0.773588in}}%
\pgfpathlineto{\pgfqpoint{14.355979in}{0.773588in}}%
\pgfpathlineto{\pgfqpoint{14.286685in}{0.773588in}}%
\pgfpathlineto{\pgfqpoint{14.216102in}{0.773588in}}%
\pgfpathlineto{\pgfqpoint{14.149562in}{0.773588in}}%
\pgfpathlineto{\pgfqpoint{14.081564in}{0.773588in}}%
\pgfpathlineto{\pgfqpoint{14.012639in}{0.773588in}}%
\pgfpathlineto{\pgfqpoint{13.944783in}{0.773588in}}%
\pgfpathlineto{\pgfqpoint{13.876785in}{0.773588in}}%
\pgfpathlineto{\pgfqpoint{13.804228in}{0.773588in}}%
\pgfpathlineto{\pgfqpoint{13.735499in}{0.773588in}}%
\pgfpathlineto{\pgfqpoint{13.666880in}{0.773588in}}%
\pgfpathlineto{\pgfqpoint{13.595052in}{0.773588in}}%
\pgfpathlineto{\pgfqpoint{13.526360in}{0.773588in}}%
\pgfpathlineto{\pgfqpoint{13.457833in}{0.773588in}}%
\pgfpathlineto{\pgfqpoint{13.387091in}{0.773588in}}%
\pgfpathlineto{\pgfqpoint{13.319414in}{0.773588in}}%
\pgfpathlineto{\pgfqpoint{13.250526in}{0.773588in}}%
\pgfpathlineto{\pgfqpoint{13.178279in}{0.773588in}}%
\pgfpathlineto{\pgfqpoint{13.109710in}{0.773588in}}%
\pgfpathlineto{\pgfqpoint{13.041078in}{0.773588in}}%
\pgfpathlineto{\pgfqpoint{12.969036in}{0.773588in}}%
\pgfpathlineto{\pgfqpoint{12.900232in}{0.773588in}}%
\pgfpathlineto{\pgfqpoint{12.830499in}{0.773588in}}%
\pgfpathlineto{\pgfqpoint{12.759967in}{0.773588in}}%
\pgfpathlineto{\pgfqpoint{12.692271in}{0.773588in}}%
\pgfpathlineto{\pgfqpoint{12.623527in}{0.773588in}}%
\pgfpathlineto{\pgfqpoint{12.552597in}{0.773588in}}%
\pgfpathlineto{\pgfqpoint{12.484090in}{0.773588in}}%
\pgfpathlineto{\pgfqpoint{12.416448in}{0.773588in}}%
\pgfpathlineto{\pgfqpoint{12.347008in}{0.773588in}}%
\pgfpathlineto{\pgfqpoint{12.280761in}{0.773588in}}%
\pgfpathlineto{\pgfqpoint{12.213872in}{0.773588in}}%
\pgfpathlineto{\pgfqpoint{12.144815in}{0.773588in}}%
\pgfpathlineto{\pgfqpoint{12.076638in}{0.773588in}}%
\pgfpathlineto{\pgfqpoint{12.008843in}{0.773588in}}%
\pgfpathlineto{\pgfqpoint{11.938397in}{0.773588in}}%
\pgfpathlineto{\pgfqpoint{11.871239in}{0.773588in}}%
\pgfpathlineto{\pgfqpoint{11.804695in}{0.773588in}}%
\pgfpathlineto{\pgfqpoint{11.736702in}{0.773588in}}%
\pgfpathlineto{\pgfqpoint{11.669969in}{0.773588in}}%
\pgfpathlineto{\pgfqpoint{11.602395in}{0.773588in}}%
\pgfpathlineto{\pgfqpoint{11.532877in}{0.773588in}}%
\pgfpathlineto{\pgfqpoint{11.465294in}{0.773588in}}%
\pgfpathlineto{\pgfqpoint{11.398136in}{0.773588in}}%
\pgfpathlineto{\pgfqpoint{11.328677in}{0.773588in}}%
\pgfpathlineto{\pgfqpoint{11.261777in}{0.773588in}}%
\pgfpathlineto{\pgfqpoint{11.194984in}{0.773588in}}%
\pgfpathlineto{\pgfqpoint{11.125133in}{0.773588in}}%
\pgfpathlineto{\pgfqpoint{11.055389in}{0.773588in}}%
\pgfpathlineto{\pgfqpoint{10.986941in}{0.773588in}}%
\pgfpathlineto{\pgfqpoint{10.916008in}{0.773588in}}%
\pgfpathlineto{\pgfqpoint{10.846851in}{0.773588in}}%
\pgfpathlineto{\pgfqpoint{10.777925in}{0.773588in}}%
\pgfpathlineto{\pgfqpoint{10.708017in}{0.773588in}}%
\pgfpathlineto{\pgfqpoint{10.639006in}{0.773588in}}%
\pgfpathlineto{\pgfqpoint{10.569898in}{0.773588in}}%
\pgfpathlineto{\pgfqpoint{10.498528in}{0.773588in}}%
\pgfpathlineto{\pgfqpoint{10.428999in}{0.773588in}}%
\pgfpathlineto{\pgfqpoint{10.360853in}{0.773588in}}%
\pgfpathlineto{\pgfqpoint{10.290696in}{0.773588in}}%
\pgfpathlineto{\pgfqpoint{10.222102in}{0.773588in}}%
\pgfpathlineto{\pgfqpoint{10.153891in}{0.773588in}}%
\pgfpathlineto{\pgfqpoint{10.081602in}{0.773588in}}%
\pgfpathlineto{\pgfqpoint{10.012687in}{0.773588in}}%
\pgfpathlineto{\pgfqpoint{9.944615in}{0.773588in}}%
\pgfpathlineto{\pgfqpoint{9.876148in}{0.773588in}}%
\pgfpathlineto{\pgfqpoint{9.808815in}{0.773588in}}%
\pgfpathlineto{\pgfqpoint{9.740273in}{0.773588in}}%
\pgfpathlineto{\pgfqpoint{9.670768in}{0.773588in}}%
\pgfpathlineto{\pgfqpoint{9.603748in}{0.773588in}}%
\pgfpathlineto{\pgfqpoint{9.536145in}{0.773588in}}%
\pgfpathlineto{\pgfqpoint{9.467314in}{0.773588in}}%
\pgfpathlineto{\pgfqpoint{9.402138in}{0.773588in}}%
\pgfpathlineto{\pgfqpoint{9.334728in}{0.773588in}}%
\pgfpathlineto{\pgfqpoint{9.265262in}{0.773588in}}%
\pgfpathlineto{\pgfqpoint{9.198477in}{0.773588in}}%
\pgfpathlineto{\pgfqpoint{9.130518in}{0.773588in}}%
\pgfpathlineto{\pgfqpoint{9.059607in}{0.773588in}}%
\pgfpathlineto{\pgfqpoint{8.991646in}{0.773588in}}%
\pgfpathlineto{\pgfqpoint{8.924860in}{0.773588in}}%
\pgfpathlineto{\pgfqpoint{8.856098in}{0.773588in}}%
\pgfpathlineto{\pgfqpoint{8.787638in}{0.773588in}}%
\pgfpathlineto{\pgfqpoint{8.720615in}{0.773588in}}%
\pgfpathlineto{\pgfqpoint{8.651757in}{0.773588in}}%
\pgfpathlineto{\pgfqpoint{8.583601in}{0.773588in}}%
\pgfpathlineto{\pgfqpoint{8.515085in}{0.773588in}}%
\pgfpathlineto{\pgfqpoint{8.444285in}{0.773588in}}%
\pgfpathlineto{\pgfqpoint{8.376454in}{0.773588in}}%
\pgfpathlineto{\pgfqpoint{8.308718in}{0.773588in}}%
\pgfpathlineto{\pgfqpoint{8.237831in}{0.773588in}}%
\pgfpathlineto{\pgfqpoint{8.168934in}{0.773588in}}%
\pgfpathlineto{\pgfqpoint{8.100457in}{0.773588in}}%
\pgfpathlineto{\pgfqpoint{8.028258in}{0.773588in}}%
\pgfpathlineto{\pgfqpoint{7.957986in}{0.773588in}}%
\pgfpathlineto{\pgfqpoint{7.887441in}{0.773588in}}%
\pgfpathlineto{\pgfqpoint{7.812286in}{0.773588in}}%
\pgfpathlineto{\pgfqpoint{7.741421in}{0.773588in}}%
\pgfpathlineto{\pgfqpoint{7.671131in}{0.773588in}}%
\pgfpathlineto{\pgfqpoint{7.598690in}{0.773588in}}%
\pgfpathlineto{\pgfqpoint{7.527107in}{0.773588in}}%
\pgfpathlineto{\pgfqpoint{7.455306in}{0.773588in}}%
\pgfpathlineto{\pgfqpoint{7.379810in}{0.773588in}}%
\pgfpathlineto{\pgfqpoint{7.308295in}{0.773588in}}%
\pgfpathlineto{\pgfqpoint{7.238831in}{0.773588in}}%
\pgfpathlineto{\pgfqpoint{7.171034in}{0.773588in}}%
\pgfpathlineto{\pgfqpoint{7.106301in}{0.773588in}}%
\pgfpathlineto{\pgfqpoint{7.040805in}{0.773588in}}%
\pgfpathlineto{\pgfqpoint{6.973333in}{0.773588in}}%
\pgfpathlineto{\pgfqpoint{6.906544in}{0.773588in}}%
\pgfpathlineto{\pgfqpoint{6.839348in}{0.773588in}}%
\pgfpathlineto{\pgfqpoint{6.770581in}{0.773588in}}%
\pgfpathlineto{\pgfqpoint{6.704271in}{0.773588in}}%
\pgfpathlineto{\pgfqpoint{6.636218in}{0.773588in}}%
\pgfpathlineto{\pgfqpoint{6.568004in}{0.773588in}}%
\pgfpathlineto{\pgfqpoint{6.502885in}{0.773588in}}%
\pgfpathlineto{\pgfqpoint{6.437563in}{0.773588in}}%
\pgfpathlineto{\pgfqpoint{6.369842in}{0.773588in}}%
\pgfpathlineto{\pgfqpoint{6.303479in}{0.773588in}}%
\pgfpathlineto{\pgfqpoint{6.236861in}{0.773588in}}%
\pgfpathlineto{\pgfqpoint{6.169451in}{0.773588in}}%
\pgfpathlineto{\pgfqpoint{6.103260in}{0.773588in}}%
\pgfpathlineto{\pgfqpoint{6.036866in}{0.773588in}}%
\pgfpathlineto{\pgfqpoint{5.968494in}{0.773588in}}%
\pgfpathlineto{\pgfqpoint{5.901623in}{0.773588in}}%
\pgfpathlineto{\pgfqpoint{5.834669in}{0.773588in}}%
\pgfpathlineto{\pgfqpoint{5.766709in}{0.773588in}}%
\pgfpathlineto{\pgfqpoint{5.700467in}{0.773588in}}%
\pgfpathlineto{\pgfqpoint{5.633136in}{0.773588in}}%
\pgfpathlineto{\pgfqpoint{5.561464in}{0.773588in}}%
\pgfpathlineto{\pgfqpoint{5.492693in}{0.773588in}}%
\pgfpathlineto{\pgfqpoint{5.424203in}{0.773588in}}%
\pgfpathlineto{\pgfqpoint{5.353942in}{0.773588in}}%
\pgfpathlineto{\pgfqpoint{5.283251in}{0.773588in}}%
\pgfpathlineto{\pgfqpoint{5.213575in}{0.773588in}}%
\pgfpathlineto{\pgfqpoint{5.142206in}{0.773588in}}%
\pgfpathlineto{\pgfqpoint{5.073090in}{0.773588in}}%
\pgfpathlineto{\pgfqpoint{5.005143in}{0.773588in}}%
\pgfpathlineto{\pgfqpoint{4.935251in}{0.773588in}}%
\pgfpathlineto{\pgfqpoint{4.866311in}{0.773588in}}%
\pgfpathlineto{\pgfqpoint{4.796564in}{0.773588in}}%
\pgfpathlineto{\pgfqpoint{4.726503in}{0.773588in}}%
\pgfpathlineto{\pgfqpoint{4.658575in}{0.773588in}}%
\pgfpathlineto{\pgfqpoint{4.590503in}{0.773588in}}%
\pgfpathlineto{\pgfqpoint{4.520559in}{0.773588in}}%
\pgfpathlineto{\pgfqpoint{4.454265in}{0.773588in}}%
\pgfpathlineto{\pgfqpoint{4.387175in}{0.773588in}}%
\pgfpathlineto{\pgfqpoint{4.319299in}{0.773588in}}%
\pgfpathlineto{\pgfqpoint{4.253047in}{0.773588in}}%
\pgfpathlineto{\pgfqpoint{4.186150in}{0.773588in}}%
\pgfpathlineto{\pgfqpoint{4.116566in}{0.773588in}}%
\pgfpathlineto{\pgfqpoint{4.048987in}{0.773588in}}%
\pgfpathlineto{\pgfqpoint{3.981945in}{0.773588in}}%
\pgfpathlineto{\pgfqpoint{3.914312in}{0.773588in}}%
\pgfpathlineto{\pgfqpoint{3.848285in}{0.773588in}}%
\pgfpathlineto{\pgfqpoint{3.781205in}{0.773588in}}%
\pgfpathlineto{\pgfqpoint{3.712535in}{0.773588in}}%
\pgfpathlineto{\pgfqpoint{3.643958in}{0.773588in}}%
\pgfpathlineto{\pgfqpoint{3.575543in}{0.773588in}}%
\pgfpathlineto{\pgfqpoint{3.503039in}{0.773588in}}%
\pgfpathlineto{\pgfqpoint{3.430573in}{0.773588in}}%
\pgfpathlineto{\pgfqpoint{3.356465in}{0.773588in}}%
\pgfpathlineto{\pgfqpoint{3.277599in}{0.773588in}}%
\pgfpathlineto{\pgfqpoint{3.195433in}{0.773588in}}%
\pgfpathlineto{\pgfqpoint{3.123114in}{0.773588in}}%
\pgfpathlineto{\pgfqpoint{3.047640in}{0.773588in}}%
\pgfpathlineto{\pgfqpoint{2.975015in}{0.773588in}}%
\pgfpathlineto{\pgfqpoint{2.902674in}{0.773588in}}%
\pgfpathlineto{\pgfqpoint{2.826501in}{0.773588in}}%
\pgfpathlineto{\pgfqpoint{2.753491in}{0.773588in}}%
\pgfpathlineto{\pgfqpoint{2.681331in}{0.773588in}}%
\pgfpathlineto{\pgfqpoint{2.606023in}{0.773588in}}%
\pgfpathlineto{\pgfqpoint{2.534117in}{0.773588in}}%
\pgfpathlineto{\pgfqpoint{2.462769in}{0.773588in}}%
\pgfpathlineto{\pgfqpoint{2.387948in}{0.773588in}}%
\pgfpathlineto{\pgfqpoint{2.317152in}{0.773588in}}%
\pgfpathlineto{\pgfqpoint{2.248172in}{0.773588in}}%
\pgfpathlineto{\pgfqpoint{2.177337in}{0.773588in}}%
\pgfpathlineto{\pgfqpoint{2.108883in}{0.773588in}}%
\pgfpathlineto{\pgfqpoint{2.041179in}{0.773588in}}%
\pgfpathlineto{\pgfqpoint{1.970951in}{0.773588in}}%
\pgfpathlineto{\pgfqpoint{1.904458in}{0.773588in}}%
\pgfpathlineto{\pgfqpoint{1.835890in}{0.773588in}}%
\pgfpathlineto{\pgfqpoint{1.766402in}{0.773588in}}%
\pgfpathlineto{\pgfqpoint{1.698662in}{0.773588in}}%
\pgfpathlineto{\pgfqpoint{1.628862in}{0.773588in}}%
\pgfpathlineto{\pgfqpoint{1.557461in}{0.773588in}}%
\pgfpathlineto{\pgfqpoint{1.489306in}{0.773588in}}%
\pgfpathlineto{\pgfqpoint{1.421095in}{0.773588in}}%
\pgfpathlineto{\pgfqpoint{1.349373in}{0.773588in}}%
\pgfpathlineto{\pgfqpoint{1.283036in}{0.773588in}}%
\pgfpathlineto{\pgfqpoint{1.216322in}{0.773588in}}%
\pgfpathlineto{\pgfqpoint{1.147369in}{0.773588in}}%
\pgfpathlineto{\pgfqpoint{1.079942in}{0.773588in}}%
\pgfpathlineto{\pgfqpoint{1.012853in}{0.773588in}}%
\pgfpathlineto{\pgfqpoint{0.942110in}{0.773588in}}%
\pgfpathlineto{\pgfqpoint{0.875335in}{0.773588in}}%
\pgfpathlineto{\pgfqpoint{0.807094in}{0.773588in}}%
\pgfpathclose%
\pgfusepath{fill}%
\end{pgfscope}%
\begin{pgfscope}%
\pgfpathrectangle{\pgfqpoint{0.781402in}{0.773588in}}{\pgfqpoint{2.110351in}{5.415119in}}%
\pgfusepath{clip}%
\pgfsetbuttcap%
\pgfsetroundjoin%
\definecolor{currentfill}{rgb}{0.839216,0.152941,0.156863}%
\pgfsetfillcolor{currentfill}%
\pgfsetlinewidth{0.000000pt}%
\definecolor{currentstroke}{rgb}{0.000000,0.000000,0.000000}%
\pgfsetstrokecolor{currentstroke}%
\pgfsetdash{}{0pt}%
\pgfpathmoveto{\pgfqpoint{0.807094in}{1.307859in}}%
\pgfpathlineto{\pgfqpoint{0.807094in}{0.773588in}}%
\pgfpathlineto{\pgfqpoint{0.875335in}{0.773588in}}%
\pgfpathlineto{\pgfqpoint{0.942110in}{0.773588in}}%
\pgfpathlineto{\pgfqpoint{1.012853in}{0.773588in}}%
\pgfpathlineto{\pgfqpoint{1.079942in}{0.773588in}}%
\pgfpathlineto{\pgfqpoint{1.147369in}{0.773588in}}%
\pgfpathlineto{\pgfqpoint{1.216322in}{0.773588in}}%
\pgfpathlineto{\pgfqpoint{1.283036in}{0.773588in}}%
\pgfpathlineto{\pgfqpoint{1.349373in}{0.773588in}}%
\pgfpathlineto{\pgfqpoint{1.421095in}{0.773588in}}%
\pgfpathlineto{\pgfqpoint{1.489306in}{0.773588in}}%
\pgfpathlineto{\pgfqpoint{1.557461in}{0.773588in}}%
\pgfpathlineto{\pgfqpoint{1.628862in}{0.773588in}}%
\pgfpathlineto{\pgfqpoint{1.698662in}{0.773588in}}%
\pgfpathlineto{\pgfqpoint{1.766402in}{0.773588in}}%
\pgfpathlineto{\pgfqpoint{1.835890in}{0.773588in}}%
\pgfpathlineto{\pgfqpoint{1.904458in}{0.773588in}}%
\pgfpathlineto{\pgfqpoint{1.970951in}{0.773588in}}%
\pgfpathlineto{\pgfqpoint{2.041179in}{0.773588in}}%
\pgfpathlineto{\pgfqpoint{2.108883in}{0.773588in}}%
\pgfpathlineto{\pgfqpoint{2.177337in}{0.773588in}}%
\pgfpathlineto{\pgfqpoint{2.248172in}{0.773588in}}%
\pgfpathlineto{\pgfqpoint{2.317152in}{0.773588in}}%
\pgfpathlineto{\pgfqpoint{2.387948in}{0.773588in}}%
\pgfpathlineto{\pgfqpoint{2.462769in}{0.773588in}}%
\pgfpathlineto{\pgfqpoint{2.534117in}{0.773588in}}%
\pgfpathlineto{\pgfqpoint{2.606023in}{0.773588in}}%
\pgfpathlineto{\pgfqpoint{2.681331in}{0.773588in}}%
\pgfpathlineto{\pgfqpoint{2.753491in}{0.773588in}}%
\pgfpathlineto{\pgfqpoint{2.826501in}{0.773588in}}%
\pgfpathlineto{\pgfqpoint{2.902674in}{0.773588in}}%
\pgfpathlineto{\pgfqpoint{2.975015in}{0.773588in}}%
\pgfpathlineto{\pgfqpoint{3.047640in}{0.773588in}}%
\pgfpathlineto{\pgfqpoint{3.123114in}{0.773588in}}%
\pgfpathlineto{\pgfqpoint{3.195433in}{0.773588in}}%
\pgfpathlineto{\pgfqpoint{3.277599in}{0.773588in}}%
\pgfpathlineto{\pgfqpoint{3.356465in}{0.773588in}}%
\pgfpathlineto{\pgfqpoint{3.430573in}{0.773588in}}%
\pgfpathlineto{\pgfqpoint{3.503039in}{0.773588in}}%
\pgfpathlineto{\pgfqpoint{3.575543in}{0.773588in}}%
\pgfpathlineto{\pgfqpoint{3.643958in}{0.773588in}}%
\pgfpathlineto{\pgfqpoint{3.712535in}{0.773588in}}%
\pgfpathlineto{\pgfqpoint{3.781205in}{0.773588in}}%
\pgfpathlineto{\pgfqpoint{3.848285in}{0.773588in}}%
\pgfpathlineto{\pgfqpoint{3.914312in}{0.773588in}}%
\pgfpathlineto{\pgfqpoint{3.981945in}{0.773588in}}%
\pgfpathlineto{\pgfqpoint{4.048987in}{0.773588in}}%
\pgfpathlineto{\pgfqpoint{4.116566in}{0.773588in}}%
\pgfpathlineto{\pgfqpoint{4.186150in}{0.773588in}}%
\pgfpathlineto{\pgfqpoint{4.253047in}{0.773588in}}%
\pgfpathlineto{\pgfqpoint{4.319299in}{0.773588in}}%
\pgfpathlineto{\pgfqpoint{4.387175in}{0.773588in}}%
\pgfpathlineto{\pgfqpoint{4.454265in}{0.773588in}}%
\pgfpathlineto{\pgfqpoint{4.520559in}{0.773588in}}%
\pgfpathlineto{\pgfqpoint{4.590503in}{0.773588in}}%
\pgfpathlineto{\pgfqpoint{4.658575in}{0.773588in}}%
\pgfpathlineto{\pgfqpoint{4.726503in}{0.773588in}}%
\pgfpathlineto{\pgfqpoint{4.796564in}{0.773588in}}%
\pgfpathlineto{\pgfqpoint{4.866311in}{0.773588in}}%
\pgfpathlineto{\pgfqpoint{4.935251in}{0.773588in}}%
\pgfpathlineto{\pgfqpoint{5.005143in}{0.773588in}}%
\pgfpathlineto{\pgfqpoint{5.073090in}{0.773588in}}%
\pgfpathlineto{\pgfqpoint{5.142206in}{0.773588in}}%
\pgfpathlineto{\pgfqpoint{5.213575in}{0.773588in}}%
\pgfpathlineto{\pgfqpoint{5.283251in}{0.773588in}}%
\pgfpathlineto{\pgfqpoint{5.353942in}{0.773588in}}%
\pgfpathlineto{\pgfqpoint{5.424203in}{0.773588in}}%
\pgfpathlineto{\pgfqpoint{5.492693in}{0.773588in}}%
\pgfpathlineto{\pgfqpoint{5.561464in}{0.773588in}}%
\pgfpathlineto{\pgfqpoint{5.633136in}{0.773588in}}%
\pgfpathlineto{\pgfqpoint{5.700467in}{0.773588in}}%
\pgfpathlineto{\pgfqpoint{5.766709in}{0.773588in}}%
\pgfpathlineto{\pgfqpoint{5.834669in}{0.773588in}}%
\pgfpathlineto{\pgfqpoint{5.901623in}{0.773588in}}%
\pgfpathlineto{\pgfqpoint{5.968494in}{0.773588in}}%
\pgfpathlineto{\pgfqpoint{6.036866in}{0.773588in}}%
\pgfpathlineto{\pgfqpoint{6.103260in}{0.773588in}}%
\pgfpathlineto{\pgfqpoint{6.169451in}{0.773588in}}%
\pgfpathlineto{\pgfqpoint{6.236861in}{0.773588in}}%
\pgfpathlineto{\pgfqpoint{6.303479in}{0.773588in}}%
\pgfpathlineto{\pgfqpoint{6.369842in}{0.773588in}}%
\pgfpathlineto{\pgfqpoint{6.437563in}{0.773588in}}%
\pgfpathlineto{\pgfqpoint{6.502885in}{0.773588in}}%
\pgfpathlineto{\pgfqpoint{6.568004in}{0.773588in}}%
\pgfpathlineto{\pgfqpoint{6.636218in}{0.773588in}}%
\pgfpathlineto{\pgfqpoint{6.704271in}{0.773588in}}%
\pgfpathlineto{\pgfqpoint{6.770581in}{0.773588in}}%
\pgfpathlineto{\pgfqpoint{6.839348in}{0.773588in}}%
\pgfpathlineto{\pgfqpoint{6.906544in}{0.773588in}}%
\pgfpathlineto{\pgfqpoint{6.973333in}{0.773588in}}%
\pgfpathlineto{\pgfqpoint{7.040805in}{0.773588in}}%
\pgfpathlineto{\pgfqpoint{7.106301in}{0.773588in}}%
\pgfpathlineto{\pgfqpoint{7.171034in}{0.773588in}}%
\pgfpathlineto{\pgfqpoint{7.238831in}{0.773588in}}%
\pgfpathlineto{\pgfqpoint{7.308295in}{0.773588in}}%
\pgfpathlineto{\pgfqpoint{7.379810in}{0.773588in}}%
\pgfpathlineto{\pgfqpoint{7.455306in}{0.773588in}}%
\pgfpathlineto{\pgfqpoint{7.527107in}{0.773588in}}%
\pgfpathlineto{\pgfqpoint{7.598690in}{0.773588in}}%
\pgfpathlineto{\pgfqpoint{7.671131in}{0.773588in}}%
\pgfpathlineto{\pgfqpoint{7.741421in}{0.773588in}}%
\pgfpathlineto{\pgfqpoint{7.812286in}{0.773588in}}%
\pgfpathlineto{\pgfqpoint{7.887441in}{0.773588in}}%
\pgfpathlineto{\pgfqpoint{7.957986in}{0.773588in}}%
\pgfpathlineto{\pgfqpoint{8.028258in}{0.773588in}}%
\pgfpathlineto{\pgfqpoint{8.100457in}{0.773588in}}%
\pgfpathlineto{\pgfqpoint{8.168934in}{0.773588in}}%
\pgfpathlineto{\pgfqpoint{8.237831in}{0.773588in}}%
\pgfpathlineto{\pgfqpoint{8.308718in}{0.773588in}}%
\pgfpathlineto{\pgfqpoint{8.376454in}{0.773588in}}%
\pgfpathlineto{\pgfqpoint{8.444285in}{0.773588in}}%
\pgfpathlineto{\pgfqpoint{8.515085in}{0.773588in}}%
\pgfpathlineto{\pgfqpoint{8.583601in}{0.773588in}}%
\pgfpathlineto{\pgfqpoint{8.651757in}{0.773588in}}%
\pgfpathlineto{\pgfqpoint{8.720615in}{0.773588in}}%
\pgfpathlineto{\pgfqpoint{8.787638in}{0.773588in}}%
\pgfpathlineto{\pgfqpoint{8.856098in}{0.773588in}}%
\pgfpathlineto{\pgfqpoint{8.924860in}{0.773588in}}%
\pgfpathlineto{\pgfqpoint{8.991646in}{0.773588in}}%
\pgfpathlineto{\pgfqpoint{9.059607in}{0.773588in}}%
\pgfpathlineto{\pgfqpoint{9.130518in}{0.773588in}}%
\pgfpathlineto{\pgfqpoint{9.198477in}{0.773588in}}%
\pgfpathlineto{\pgfqpoint{9.265262in}{0.773588in}}%
\pgfpathlineto{\pgfqpoint{9.334728in}{0.773588in}}%
\pgfpathlineto{\pgfqpoint{9.402138in}{0.773588in}}%
\pgfpathlineto{\pgfqpoint{9.467314in}{0.773588in}}%
\pgfpathlineto{\pgfqpoint{9.536145in}{0.773588in}}%
\pgfpathlineto{\pgfqpoint{9.603748in}{0.773588in}}%
\pgfpathlineto{\pgfqpoint{9.670768in}{0.773588in}}%
\pgfpathlineto{\pgfqpoint{9.740273in}{0.773588in}}%
\pgfpathlineto{\pgfqpoint{9.808815in}{0.773588in}}%
\pgfpathlineto{\pgfqpoint{9.876148in}{0.773588in}}%
\pgfpathlineto{\pgfqpoint{9.944615in}{0.773588in}}%
\pgfpathlineto{\pgfqpoint{10.012687in}{0.773588in}}%
\pgfpathlineto{\pgfqpoint{10.081602in}{0.773588in}}%
\pgfpathlineto{\pgfqpoint{10.153891in}{0.773588in}}%
\pgfpathlineto{\pgfqpoint{10.222102in}{0.773588in}}%
\pgfpathlineto{\pgfqpoint{10.290696in}{0.773588in}}%
\pgfpathlineto{\pgfqpoint{10.360853in}{0.773588in}}%
\pgfpathlineto{\pgfqpoint{10.428999in}{0.773588in}}%
\pgfpathlineto{\pgfqpoint{10.498528in}{0.773588in}}%
\pgfpathlineto{\pgfqpoint{10.569898in}{0.773588in}}%
\pgfpathlineto{\pgfqpoint{10.639006in}{0.773588in}}%
\pgfpathlineto{\pgfqpoint{10.708017in}{0.773588in}}%
\pgfpathlineto{\pgfqpoint{10.777925in}{0.773588in}}%
\pgfpathlineto{\pgfqpoint{10.846851in}{0.773588in}}%
\pgfpathlineto{\pgfqpoint{10.916008in}{0.773588in}}%
\pgfpathlineto{\pgfqpoint{10.986941in}{0.773588in}}%
\pgfpathlineto{\pgfqpoint{11.055389in}{0.773588in}}%
\pgfpathlineto{\pgfqpoint{11.125133in}{0.773588in}}%
\pgfpathlineto{\pgfqpoint{11.194984in}{0.773588in}}%
\pgfpathlineto{\pgfqpoint{11.261777in}{0.773588in}}%
\pgfpathlineto{\pgfqpoint{11.328677in}{0.773588in}}%
\pgfpathlineto{\pgfqpoint{11.398136in}{0.773588in}}%
\pgfpathlineto{\pgfqpoint{11.465294in}{0.773588in}}%
\pgfpathlineto{\pgfqpoint{11.532877in}{0.773588in}}%
\pgfpathlineto{\pgfqpoint{11.602395in}{0.773588in}}%
\pgfpathlineto{\pgfqpoint{11.669969in}{0.773588in}}%
\pgfpathlineto{\pgfqpoint{11.736702in}{0.773588in}}%
\pgfpathlineto{\pgfqpoint{11.804695in}{0.773588in}}%
\pgfpathlineto{\pgfqpoint{11.871239in}{0.773588in}}%
\pgfpathlineto{\pgfqpoint{11.938397in}{0.773588in}}%
\pgfpathlineto{\pgfqpoint{12.008843in}{0.773588in}}%
\pgfpathlineto{\pgfqpoint{12.076638in}{0.773588in}}%
\pgfpathlineto{\pgfqpoint{12.144815in}{0.773588in}}%
\pgfpathlineto{\pgfqpoint{12.213872in}{0.773588in}}%
\pgfpathlineto{\pgfqpoint{12.280761in}{0.773588in}}%
\pgfpathlineto{\pgfqpoint{12.347008in}{0.773588in}}%
\pgfpathlineto{\pgfqpoint{12.416448in}{0.773588in}}%
\pgfpathlineto{\pgfqpoint{12.484090in}{0.773588in}}%
\pgfpathlineto{\pgfqpoint{12.552597in}{0.773588in}}%
\pgfpathlineto{\pgfqpoint{12.623527in}{0.773588in}}%
\pgfpathlineto{\pgfqpoint{12.692271in}{0.773588in}}%
\pgfpathlineto{\pgfqpoint{12.759967in}{0.773588in}}%
\pgfpathlineto{\pgfqpoint{12.830499in}{0.773588in}}%
\pgfpathlineto{\pgfqpoint{12.900232in}{0.773588in}}%
\pgfpathlineto{\pgfqpoint{12.969036in}{0.773588in}}%
\pgfpathlineto{\pgfqpoint{13.041078in}{0.773588in}}%
\pgfpathlineto{\pgfqpoint{13.109710in}{0.773588in}}%
\pgfpathlineto{\pgfqpoint{13.178279in}{0.773588in}}%
\pgfpathlineto{\pgfqpoint{13.250526in}{0.773588in}}%
\pgfpathlineto{\pgfqpoint{13.319414in}{0.773588in}}%
\pgfpathlineto{\pgfqpoint{13.387091in}{0.773588in}}%
\pgfpathlineto{\pgfqpoint{13.457833in}{0.773588in}}%
\pgfpathlineto{\pgfqpoint{13.526360in}{0.773588in}}%
\pgfpathlineto{\pgfqpoint{13.595052in}{0.773588in}}%
\pgfpathlineto{\pgfqpoint{13.666880in}{0.773588in}}%
\pgfpathlineto{\pgfqpoint{13.735499in}{0.773588in}}%
\pgfpathlineto{\pgfqpoint{13.804228in}{0.773588in}}%
\pgfpathlineto{\pgfqpoint{13.876785in}{0.773588in}}%
\pgfpathlineto{\pgfqpoint{13.944783in}{0.773588in}}%
\pgfpathlineto{\pgfqpoint{14.012639in}{0.773588in}}%
\pgfpathlineto{\pgfqpoint{14.081564in}{0.773588in}}%
\pgfpathlineto{\pgfqpoint{14.149562in}{0.773588in}}%
\pgfpathlineto{\pgfqpoint{14.216102in}{0.773588in}}%
\pgfpathlineto{\pgfqpoint{14.286685in}{0.773588in}}%
\pgfpathlineto{\pgfqpoint{14.355979in}{0.773588in}}%
\pgfpathlineto{\pgfqpoint{14.424138in}{0.773588in}}%
\pgfpathlineto{\pgfqpoint{14.494110in}{0.773588in}}%
\pgfpathlineto{\pgfqpoint{14.562284in}{0.773588in}}%
\pgfpathlineto{\pgfqpoint{14.630523in}{0.773588in}}%
\pgfpathlineto{\pgfqpoint{14.700003in}{0.773588in}}%
\pgfpathlineto{\pgfqpoint{14.768225in}{0.773588in}}%
\pgfpathlineto{\pgfqpoint{14.836607in}{0.773588in}}%
\pgfpathlineto{\pgfqpoint{14.908199in}{0.773588in}}%
\pgfpathlineto{\pgfqpoint{14.976372in}{0.773588in}}%
\pgfpathlineto{\pgfqpoint{15.044439in}{0.773588in}}%
\pgfpathlineto{\pgfqpoint{15.115140in}{0.773588in}}%
\pgfpathlineto{\pgfqpoint{15.182512in}{0.773588in}}%
\pgfpathlineto{\pgfqpoint{15.249597in}{0.773588in}}%
\pgfpathlineto{\pgfqpoint{15.320168in}{0.773588in}}%
\pgfpathlineto{\pgfqpoint{15.388582in}{0.773588in}}%
\pgfpathlineto{\pgfqpoint{15.457342in}{0.773588in}}%
\pgfpathlineto{\pgfqpoint{15.527961in}{0.773588in}}%
\pgfpathlineto{\pgfqpoint{15.595878in}{0.773588in}}%
\pgfpathlineto{\pgfqpoint{15.665480in}{0.773588in}}%
\pgfpathlineto{\pgfqpoint{15.737902in}{0.773588in}}%
\pgfpathlineto{\pgfqpoint{15.807467in}{0.773588in}}%
\pgfpathlineto{\pgfqpoint{15.878678in}{0.773588in}}%
\pgfpathlineto{\pgfqpoint{15.949677in}{0.773588in}}%
\pgfpathlineto{\pgfqpoint{16.019598in}{0.773588in}}%
\pgfpathlineto{\pgfqpoint{16.088678in}{0.773588in}}%
\pgfpathlineto{\pgfqpoint{16.160188in}{0.773588in}}%
\pgfpathlineto{\pgfqpoint{16.229236in}{0.773588in}}%
\pgfpathlineto{\pgfqpoint{16.299236in}{0.773588in}}%
\pgfpathlineto{\pgfqpoint{16.373054in}{0.773588in}}%
\pgfpathlineto{\pgfqpoint{16.445981in}{0.773588in}}%
\pgfpathlineto{\pgfqpoint{16.519393in}{0.773588in}}%
\pgfpathlineto{\pgfqpoint{16.592371in}{0.773588in}}%
\pgfpathlineto{\pgfqpoint{16.662798in}{0.773588in}}%
\pgfpathlineto{\pgfqpoint{16.731829in}{0.773588in}}%
\pgfpathlineto{\pgfqpoint{16.803402in}{0.773588in}}%
\pgfpathlineto{\pgfqpoint{16.872974in}{0.773588in}}%
\pgfpathlineto{\pgfqpoint{16.942299in}{0.773588in}}%
\pgfpathlineto{\pgfqpoint{17.013249in}{0.773588in}}%
\pgfpathlineto{\pgfqpoint{17.081113in}{0.773588in}}%
\pgfpathlineto{\pgfqpoint{17.150280in}{0.773588in}}%
\pgfpathlineto{\pgfqpoint{17.220960in}{0.773588in}}%
\pgfpathlineto{\pgfqpoint{17.288800in}{0.773588in}}%
\pgfpathlineto{\pgfqpoint{17.356523in}{0.773588in}}%
\pgfpathlineto{\pgfqpoint{17.426570in}{0.773588in}}%
\pgfpathlineto{\pgfqpoint{17.494859in}{0.773588in}}%
\pgfpathlineto{\pgfqpoint{17.563254in}{0.773588in}}%
\pgfpathlineto{\pgfqpoint{17.634782in}{0.773588in}}%
\pgfpathlineto{\pgfqpoint{17.703480in}{0.773588in}}%
\pgfpathlineto{\pgfqpoint{17.772898in}{0.773588in}}%
\pgfpathlineto{\pgfqpoint{17.844148in}{0.773588in}}%
\pgfpathlineto{\pgfqpoint{17.912627in}{0.773588in}}%
\pgfpathlineto{\pgfqpoint{17.983524in}{0.773588in}}%
\pgfpathlineto{\pgfqpoint{18.056086in}{0.773588in}}%
\pgfpathlineto{\pgfqpoint{18.125960in}{0.773588in}}%
\pgfpathlineto{\pgfqpoint{18.195268in}{0.773588in}}%
\pgfpathlineto{\pgfqpoint{18.266780in}{0.773588in}}%
\pgfpathlineto{\pgfqpoint{18.336223in}{0.773588in}}%
\pgfpathlineto{\pgfqpoint{18.406899in}{0.773588in}}%
\pgfpathlineto{\pgfqpoint{18.479143in}{0.773588in}}%
\pgfpathlineto{\pgfqpoint{18.549509in}{0.773588in}}%
\pgfpathlineto{\pgfqpoint{18.619849in}{0.773588in}}%
\pgfpathlineto{\pgfqpoint{18.692839in}{0.773588in}}%
\pgfpathlineto{\pgfqpoint{18.763558in}{0.773588in}}%
\pgfpathlineto{\pgfqpoint{18.834141in}{0.773588in}}%
\pgfpathlineto{\pgfqpoint{18.906958in}{0.773588in}}%
\pgfpathlineto{\pgfqpoint{18.977585in}{0.773588in}}%
\pgfpathlineto{\pgfqpoint{19.048835in}{0.773588in}}%
\pgfpathlineto{\pgfqpoint{19.123240in}{0.773588in}}%
\pgfpathlineto{\pgfqpoint{19.195134in}{0.773588in}}%
\pgfpathlineto{\pgfqpoint{19.266961in}{0.773588in}}%
\pgfpathlineto{\pgfqpoint{19.340700in}{0.773588in}}%
\pgfpathlineto{\pgfqpoint{19.412352in}{0.773588in}}%
\pgfpathlineto{\pgfqpoint{19.482791in}{0.773588in}}%
\pgfpathlineto{\pgfqpoint{19.553641in}{0.773588in}}%
\pgfpathlineto{\pgfqpoint{19.623855in}{0.773588in}}%
\pgfpathlineto{\pgfqpoint{19.693765in}{0.773588in}}%
\pgfpathlineto{\pgfqpoint{19.766060in}{0.773588in}}%
\pgfpathlineto{\pgfqpoint{19.835928in}{0.773588in}}%
\pgfpathlineto{\pgfqpoint{19.907049in}{0.773588in}}%
\pgfpathlineto{\pgfqpoint{19.981199in}{0.773588in}}%
\pgfpathlineto{\pgfqpoint{20.052071in}{0.773588in}}%
\pgfpathlineto{\pgfqpoint{20.121885in}{0.773588in}}%
\pgfpathlineto{\pgfqpoint{20.193561in}{0.773588in}}%
\pgfpathlineto{\pgfqpoint{20.263963in}{0.773588in}}%
\pgfpathlineto{\pgfqpoint{20.334605in}{0.773588in}}%
\pgfpathlineto{\pgfqpoint{20.407276in}{0.773588in}}%
\pgfpathlineto{\pgfqpoint{20.476961in}{0.773588in}}%
\pgfpathlineto{\pgfqpoint{20.547172in}{0.773588in}}%
\pgfpathlineto{\pgfqpoint{20.618428in}{0.773588in}}%
\pgfpathlineto{\pgfqpoint{20.688980in}{0.773588in}}%
\pgfpathlineto{\pgfqpoint{20.758814in}{0.773588in}}%
\pgfpathlineto{\pgfqpoint{20.830586in}{0.773588in}}%
\pgfpathlineto{\pgfqpoint{20.899587in}{0.773588in}}%
\pgfpathlineto{\pgfqpoint{20.969373in}{0.773588in}}%
\pgfpathlineto{\pgfqpoint{21.040864in}{0.773588in}}%
\pgfpathlineto{\pgfqpoint{21.110656in}{0.773588in}}%
\pgfpathlineto{\pgfqpoint{21.181233in}{0.773588in}}%
\pgfpathlineto{\pgfqpoint{21.254645in}{0.773588in}}%
\pgfpathlineto{\pgfqpoint{21.324498in}{0.773588in}}%
\pgfpathlineto{\pgfqpoint{21.394839in}{0.773588in}}%
\pgfpathlineto{\pgfqpoint{21.467741in}{0.773588in}}%
\pgfpathlineto{\pgfqpoint{21.539601in}{0.773588in}}%
\pgfpathlineto{\pgfqpoint{21.610878in}{0.773588in}}%
\pgfpathlineto{\pgfqpoint{21.683994in}{0.773588in}}%
\pgfpathlineto{\pgfqpoint{21.756227in}{0.773588in}}%
\pgfpathlineto{\pgfqpoint{21.828555in}{0.773588in}}%
\pgfpathlineto{\pgfqpoint{21.903868in}{0.773588in}}%
\pgfpathlineto{\pgfqpoint{21.976836in}{0.773588in}}%
\pgfpathlineto{\pgfqpoint{22.048040in}{0.773588in}}%
\pgfpathlineto{\pgfqpoint{22.122462in}{0.773588in}}%
\pgfpathlineto{\pgfqpoint{22.195707in}{0.773588in}}%
\pgfpathlineto{\pgfqpoint{22.268824in}{0.773588in}}%
\pgfpathlineto{\pgfqpoint{22.343331in}{0.773588in}}%
\pgfpathlineto{\pgfqpoint{22.413449in}{0.773588in}}%
\pgfpathlineto{\pgfqpoint{22.482516in}{0.773588in}}%
\pgfpathlineto{\pgfqpoint{22.553535in}{0.773588in}}%
\pgfpathlineto{\pgfqpoint{22.624114in}{0.773588in}}%
\pgfpathlineto{\pgfqpoint{22.694016in}{0.773588in}}%
\pgfpathlineto{\pgfqpoint{22.764651in}{0.773588in}}%
\pgfpathlineto{\pgfqpoint{22.833628in}{0.773588in}}%
\pgfpathlineto{\pgfqpoint{22.902896in}{0.773588in}}%
\pgfpathlineto{\pgfqpoint{22.973868in}{0.773588in}}%
\pgfpathlineto{\pgfqpoint{23.043397in}{0.773588in}}%
\pgfpathlineto{\pgfqpoint{23.113267in}{0.773588in}}%
\pgfpathlineto{\pgfqpoint{23.184270in}{0.773588in}}%
\pgfpathlineto{\pgfqpoint{23.253484in}{0.773588in}}%
\pgfpathlineto{\pgfqpoint{23.323995in}{0.773588in}}%
\pgfpathlineto{\pgfqpoint{23.396126in}{0.773588in}}%
\pgfpathlineto{\pgfqpoint{23.467323in}{0.773588in}}%
\pgfpathlineto{\pgfqpoint{23.537850in}{0.773588in}}%
\pgfpathlineto{\pgfqpoint{23.610036in}{0.773588in}}%
\pgfpathlineto{\pgfqpoint{23.681448in}{0.773588in}}%
\pgfpathlineto{\pgfqpoint{23.752361in}{0.773588in}}%
\pgfpathlineto{\pgfqpoint{23.824951in}{0.773588in}}%
\pgfpathlineto{\pgfqpoint{23.895213in}{0.773588in}}%
\pgfpathlineto{\pgfqpoint{23.966722in}{0.773588in}}%
\pgfpathlineto{\pgfqpoint{24.039255in}{0.773588in}}%
\pgfpathlineto{\pgfqpoint{24.111784in}{0.773588in}}%
\pgfpathlineto{\pgfqpoint{24.183899in}{0.773588in}}%
\pgfpathlineto{\pgfqpoint{24.257307in}{0.773588in}}%
\pgfpathlineto{\pgfqpoint{24.329090in}{0.773588in}}%
\pgfpathlineto{\pgfqpoint{24.400328in}{0.773588in}}%
\pgfpathlineto{\pgfqpoint{24.476339in}{0.773588in}}%
\pgfpathlineto{\pgfqpoint{24.548639in}{0.773588in}}%
\pgfpathlineto{\pgfqpoint{24.618678in}{0.773588in}}%
\pgfpathlineto{\pgfqpoint{24.691660in}{0.773588in}}%
\pgfpathlineto{\pgfqpoint{24.764742in}{0.773588in}}%
\pgfpathlineto{\pgfqpoint{24.836992in}{0.773588in}}%
\pgfpathlineto{\pgfqpoint{24.911741in}{0.773588in}}%
\pgfpathlineto{\pgfqpoint{24.983525in}{0.773588in}}%
\pgfpathlineto{\pgfqpoint{25.055567in}{0.773588in}}%
\pgfpathlineto{\pgfqpoint{25.131109in}{0.773588in}}%
\pgfpathlineto{\pgfqpoint{25.203216in}{0.773588in}}%
\pgfpathlineto{\pgfqpoint{25.273349in}{0.773588in}}%
\pgfpathlineto{\pgfqpoint{25.347124in}{0.773588in}}%
\pgfpathlineto{\pgfqpoint{25.417047in}{0.773588in}}%
\pgfpathlineto{\pgfqpoint{25.487573in}{0.773588in}}%
\pgfpathlineto{\pgfqpoint{25.560110in}{0.773588in}}%
\pgfpathlineto{\pgfqpoint{25.631022in}{0.773588in}}%
\pgfpathlineto{\pgfqpoint{25.702341in}{0.773588in}}%
\pgfpathlineto{\pgfqpoint{25.775695in}{0.773588in}}%
\pgfpathlineto{\pgfqpoint{25.845667in}{0.773588in}}%
\pgfpathlineto{\pgfqpoint{25.916551in}{0.773588in}}%
\pgfpathlineto{\pgfqpoint{25.988588in}{0.773588in}}%
\pgfpathlineto{\pgfqpoint{26.058621in}{0.773588in}}%
\pgfpathlineto{\pgfqpoint{26.130346in}{0.773588in}}%
\pgfpathlineto{\pgfqpoint{26.203572in}{0.773588in}}%
\pgfpathlineto{\pgfqpoint{26.274267in}{0.773588in}}%
\pgfpathlineto{\pgfqpoint{26.344920in}{0.773588in}}%
\pgfpathlineto{\pgfqpoint{26.417231in}{0.773588in}}%
\pgfpathlineto{\pgfqpoint{26.487420in}{0.773588in}}%
\pgfpathlineto{\pgfqpoint{26.557235in}{0.773588in}}%
\pgfpathlineto{\pgfqpoint{26.629572in}{0.773588in}}%
\pgfpathlineto{\pgfqpoint{26.699584in}{0.773588in}}%
\pgfpathlineto{\pgfqpoint{26.769271in}{0.773588in}}%
\pgfpathlineto{\pgfqpoint{26.841234in}{0.773588in}}%
\pgfpathlineto{\pgfqpoint{26.912667in}{0.773588in}}%
\pgfpathlineto{\pgfqpoint{26.983641in}{0.773588in}}%
\pgfpathlineto{\pgfqpoint{27.056835in}{0.773588in}}%
\pgfpathlineto{\pgfqpoint{27.128948in}{0.773588in}}%
\pgfpathlineto{\pgfqpoint{27.201477in}{0.773588in}}%
\pgfpathlineto{\pgfqpoint{27.277488in}{0.773588in}}%
\pgfpathlineto{\pgfqpoint{27.350990in}{0.773588in}}%
\pgfpathlineto{\pgfqpoint{27.423884in}{0.773588in}}%
\pgfpathlineto{\pgfqpoint{27.500063in}{0.773588in}}%
\pgfpathlineto{\pgfqpoint{27.574929in}{0.773588in}}%
\pgfpathlineto{\pgfqpoint{27.649072in}{0.773588in}}%
\pgfpathlineto{\pgfqpoint{27.724006in}{0.773588in}}%
\pgfpathlineto{\pgfqpoint{27.795343in}{0.773588in}}%
\pgfpathlineto{\pgfqpoint{27.868117in}{0.773588in}}%
\pgfpathlineto{\pgfqpoint{27.943911in}{0.773588in}}%
\pgfpathlineto{\pgfqpoint{28.018234in}{0.773588in}}%
\pgfpathlineto{\pgfqpoint{28.090360in}{0.773588in}}%
\pgfpathlineto{\pgfqpoint{28.163352in}{0.773588in}}%
\pgfpathlineto{\pgfqpoint{28.234559in}{0.773588in}}%
\pgfpathlineto{\pgfqpoint{28.306604in}{0.773588in}}%
\pgfpathlineto{\pgfqpoint{28.380501in}{0.773588in}}%
\pgfpathlineto{\pgfqpoint{28.451943in}{0.773588in}}%
\pgfpathlineto{\pgfqpoint{28.522534in}{0.773588in}}%
\pgfpathlineto{\pgfqpoint{28.596041in}{0.773588in}}%
\pgfpathlineto{\pgfqpoint{28.668204in}{0.773588in}}%
\pgfpathlineto{\pgfqpoint{28.738605in}{0.773588in}}%
\pgfpathlineto{\pgfqpoint{28.811911in}{0.773588in}}%
\pgfpathlineto{\pgfqpoint{28.885217in}{0.773588in}}%
\pgfpathlineto{\pgfqpoint{28.956832in}{0.773588in}}%
\pgfpathlineto{\pgfqpoint{29.029887in}{0.773588in}}%
\pgfpathlineto{\pgfqpoint{29.100748in}{0.773588in}}%
\pgfpathlineto{\pgfqpoint{29.173201in}{0.773588in}}%
\pgfpathlineto{\pgfqpoint{29.248973in}{0.773588in}}%
\pgfpathlineto{\pgfqpoint{29.320759in}{0.773588in}}%
\pgfpathlineto{\pgfqpoint{29.393660in}{0.773588in}}%
\pgfpathlineto{\pgfqpoint{29.467898in}{0.773588in}}%
\pgfpathlineto{\pgfqpoint{29.540420in}{0.773588in}}%
\pgfpathlineto{\pgfqpoint{29.611700in}{0.773588in}}%
\pgfpathlineto{\pgfqpoint{29.684427in}{0.773588in}}%
\pgfpathlineto{\pgfqpoint{29.755113in}{0.773588in}}%
\pgfpathlineto{\pgfqpoint{29.827132in}{0.773588in}}%
\pgfpathlineto{\pgfqpoint{29.901656in}{0.773588in}}%
\pgfpathlineto{\pgfqpoint{29.974646in}{0.773588in}}%
\pgfpathlineto{\pgfqpoint{30.048252in}{0.773588in}}%
\pgfpathlineto{\pgfqpoint{30.122796in}{0.773588in}}%
\pgfpathlineto{\pgfqpoint{30.195443in}{0.773588in}}%
\pgfpathlineto{\pgfqpoint{30.269036in}{0.773588in}}%
\pgfpathlineto{\pgfqpoint{30.344328in}{0.773588in}}%
\pgfpathlineto{\pgfqpoint{30.417098in}{0.773588in}}%
\pgfpathlineto{\pgfqpoint{30.488991in}{0.773588in}}%
\pgfpathlineto{\pgfqpoint{30.562714in}{0.773588in}}%
\pgfpathlineto{\pgfqpoint{30.634099in}{0.773588in}}%
\pgfpathlineto{\pgfqpoint{30.707828in}{0.773588in}}%
\pgfpathlineto{\pgfqpoint{30.782249in}{0.773588in}}%
\pgfpathlineto{\pgfqpoint{30.854115in}{0.773588in}}%
\pgfpathlineto{\pgfqpoint{30.928305in}{0.773588in}}%
\pgfpathlineto{\pgfqpoint{31.002514in}{0.773588in}}%
\pgfpathlineto{\pgfqpoint{31.074452in}{0.773588in}}%
\pgfpathlineto{\pgfqpoint{31.147740in}{0.773588in}}%
\pgfpathlineto{\pgfqpoint{31.222913in}{0.773588in}}%
\pgfpathlineto{\pgfqpoint{31.294777in}{0.773588in}}%
\pgfpathlineto{\pgfqpoint{31.366613in}{0.773588in}}%
\pgfpathlineto{\pgfqpoint{31.439415in}{0.773588in}}%
\pgfpathlineto{\pgfqpoint{31.510140in}{0.773588in}}%
\pgfpathlineto{\pgfqpoint{31.582282in}{0.773588in}}%
\pgfpathlineto{\pgfqpoint{31.656180in}{0.773588in}}%
\pgfpathlineto{\pgfqpoint{31.728521in}{0.773588in}}%
\pgfpathlineto{\pgfqpoint{31.800877in}{0.773588in}}%
\pgfpathlineto{\pgfqpoint{31.873539in}{0.773588in}}%
\pgfpathlineto{\pgfqpoint{31.943734in}{0.773588in}}%
\pgfpathlineto{\pgfqpoint{32.015122in}{0.773588in}}%
\pgfpathlineto{\pgfqpoint{32.089684in}{0.773588in}}%
\pgfpathlineto{\pgfqpoint{32.161504in}{0.773588in}}%
\pgfpathlineto{\pgfqpoint{32.231773in}{0.773588in}}%
\pgfpathlineto{\pgfqpoint{32.305440in}{0.773588in}}%
\pgfpathlineto{\pgfqpoint{32.377016in}{0.773588in}}%
\pgfpathlineto{\pgfqpoint{32.447439in}{0.773588in}}%
\pgfpathlineto{\pgfqpoint{32.520401in}{0.773588in}}%
\pgfpathlineto{\pgfqpoint{32.590674in}{0.773588in}}%
\pgfpathlineto{\pgfqpoint{32.663709in}{0.773588in}}%
\pgfpathlineto{\pgfqpoint{32.740263in}{0.773588in}}%
\pgfpathlineto{\pgfqpoint{32.813546in}{0.773588in}}%
\pgfpathlineto{\pgfqpoint{32.887492in}{0.773588in}}%
\pgfpathlineto{\pgfqpoint{32.963168in}{0.773588in}}%
\pgfpathlineto{\pgfqpoint{33.037794in}{0.773588in}}%
\pgfpathlineto{\pgfqpoint{33.110479in}{0.773588in}}%
\pgfpathlineto{\pgfqpoint{33.185787in}{0.773588in}}%
\pgfpathlineto{\pgfqpoint{33.259507in}{0.773588in}}%
\pgfpathlineto{\pgfqpoint{33.333311in}{0.773588in}}%
\pgfpathlineto{\pgfqpoint{33.409286in}{0.773588in}}%
\pgfpathlineto{\pgfqpoint{33.483328in}{0.773588in}}%
\pgfpathlineto{\pgfqpoint{33.557012in}{0.773588in}}%
\pgfpathlineto{\pgfqpoint{33.631884in}{0.773588in}}%
\pgfpathlineto{\pgfqpoint{33.703848in}{0.773588in}}%
\pgfpathlineto{\pgfqpoint{33.776888in}{0.773588in}}%
\pgfpathlineto{\pgfqpoint{33.852393in}{0.773588in}}%
\pgfpathlineto{\pgfqpoint{33.923536in}{0.773588in}}%
\pgfpathlineto{\pgfqpoint{33.994648in}{0.773588in}}%
\pgfpathlineto{\pgfqpoint{34.067999in}{0.773588in}}%
\pgfpathlineto{\pgfqpoint{34.138346in}{0.773588in}}%
\pgfpathlineto{\pgfqpoint{34.210760in}{0.773588in}}%
\pgfpathlineto{\pgfqpoint{34.284339in}{0.773588in}}%
\pgfpathlineto{\pgfqpoint{34.354648in}{0.773588in}}%
\pgfpathlineto{\pgfqpoint{34.425604in}{0.773588in}}%
\pgfpathlineto{\pgfqpoint{34.499162in}{0.773588in}}%
\pgfpathlineto{\pgfqpoint{34.571449in}{0.773588in}}%
\pgfpathlineto{\pgfqpoint{34.643977in}{0.773588in}}%
\pgfpathlineto{\pgfqpoint{34.718731in}{0.773588in}}%
\pgfpathlineto{\pgfqpoint{34.789698in}{0.773588in}}%
\pgfpathlineto{\pgfqpoint{34.862212in}{0.773588in}}%
\pgfpathlineto{\pgfqpoint{34.936943in}{0.773588in}}%
\pgfpathlineto{\pgfqpoint{35.007838in}{0.773588in}}%
\pgfpathlineto{\pgfqpoint{35.080154in}{0.773588in}}%
\pgfpathlineto{\pgfqpoint{35.155466in}{0.773588in}}%
\pgfpathlineto{\pgfqpoint{35.227201in}{0.773588in}}%
\pgfpathlineto{\pgfqpoint{35.298174in}{0.773588in}}%
\pgfpathlineto{\pgfqpoint{35.372990in}{0.773588in}}%
\pgfpathlineto{\pgfqpoint{35.451774in}{0.773588in}}%
\pgfpathlineto{\pgfqpoint{35.574549in}{0.773588in}}%
\pgfpathlineto{\pgfqpoint{35.663523in}{0.773588in}}%
\pgfpathlineto{\pgfqpoint{35.741519in}{0.773588in}}%
\pgfpathlineto{\pgfqpoint{35.805568in}{1.724813in}}%
\pgfpathlineto{\pgfqpoint{35.870813in}{5.187514in}}%
\pgfpathlineto{\pgfqpoint{35.942832in}{5.319244in}}%
\pgfpathlineto{\pgfqpoint{36.012796in}{5.506716in}}%
\pgfpathlineto{\pgfqpoint{36.085094in}{5.399336in}}%
\pgfpathlineto{\pgfqpoint{36.154695in}{5.488048in}}%
\pgfpathlineto{\pgfqpoint{36.223624in}{5.539128in}}%
\pgfpathlineto{\pgfqpoint{36.293479in}{5.638191in}}%
\pgfpathlineto{\pgfqpoint{36.360634in}{5.619869in}}%
\pgfpathlineto{\pgfqpoint{36.428206in}{5.661488in}}%
\pgfpathlineto{\pgfqpoint{36.497087in}{5.625548in}}%
\pgfpathlineto{\pgfqpoint{36.563097in}{5.748599in}}%
\pgfpathlineto{\pgfqpoint{36.628950in}{5.732424in}}%
\pgfpathlineto{\pgfqpoint{36.696952in}{5.790749in}}%
\pgfpathlineto{\pgfqpoint{36.761893in}{5.793255in}}%
\pgfpathlineto{\pgfqpoint{36.827337in}{5.772285in}}%
\pgfpathlineto{\pgfqpoint{36.893714in}{5.875626in}}%
\pgfpathlineto{\pgfqpoint{36.957470in}{5.879596in}}%
\pgfpathlineto{\pgfqpoint{37.022217in}{5.814953in}}%
\pgfpathlineto{\pgfqpoint{37.088015in}{5.930845in}}%
\pgfpathlineto{\pgfqpoint{37.151827in}{5.872359in}}%
\pgfpathlineto{\pgfqpoint{37.151827in}{5.872359in}}%
\pgfpathlineto{\pgfqpoint{37.151827in}{5.872359in}}%
\pgfpathlineto{\pgfqpoint{37.088015in}{5.930845in}}%
\pgfpathlineto{\pgfqpoint{37.022217in}{5.814953in}}%
\pgfpathlineto{\pgfqpoint{36.957470in}{5.879596in}}%
\pgfpathlineto{\pgfqpoint{36.893714in}{5.875626in}}%
\pgfpathlineto{\pgfqpoint{36.827337in}{5.772285in}}%
\pgfpathlineto{\pgfqpoint{36.761893in}{5.793255in}}%
\pgfpathlineto{\pgfqpoint{36.696952in}{5.790749in}}%
\pgfpathlineto{\pgfqpoint{36.628950in}{5.732424in}}%
\pgfpathlineto{\pgfqpoint{36.563097in}{5.748599in}}%
\pgfpathlineto{\pgfqpoint{36.497087in}{5.625548in}}%
\pgfpathlineto{\pgfqpoint{36.428206in}{5.661488in}}%
\pgfpathlineto{\pgfqpoint{36.360634in}{5.619869in}}%
\pgfpathlineto{\pgfqpoint{36.293479in}{5.638191in}}%
\pgfpathlineto{\pgfqpoint{36.223624in}{5.539128in}}%
\pgfpathlineto{\pgfqpoint{36.154695in}{5.488048in}}%
\pgfpathlineto{\pgfqpoint{36.085094in}{5.399336in}}%
\pgfpathlineto{\pgfqpoint{36.012796in}{5.506716in}}%
\pgfpathlineto{\pgfqpoint{35.942832in}{5.319244in}}%
\pgfpathlineto{\pgfqpoint{35.870813in}{5.187514in}}%
\pgfpathlineto{\pgfqpoint{35.805568in}{1.724813in}}%
\pgfpathlineto{\pgfqpoint{35.741519in}{0.926398in}}%
\pgfpathlineto{\pgfqpoint{35.663523in}{1.497722in}}%
\pgfpathlineto{\pgfqpoint{35.574549in}{1.253992in}}%
\pgfpathlineto{\pgfqpoint{35.451774in}{1.457530in}}%
\pgfpathlineto{\pgfqpoint{35.372990in}{1.484551in}}%
\pgfpathlineto{\pgfqpoint{35.298174in}{1.575304in}}%
\pgfpathlineto{\pgfqpoint{35.227201in}{1.567696in}}%
\pgfpathlineto{\pgfqpoint{35.155466in}{1.498229in}}%
\pgfpathlineto{\pgfqpoint{35.080154in}{1.504964in}}%
\pgfpathlineto{\pgfqpoint{35.007838in}{1.565434in}}%
\pgfpathlineto{\pgfqpoint{34.936943in}{1.472512in}}%
\pgfpathlineto{\pgfqpoint{34.862212in}{1.539915in}}%
\pgfpathlineto{\pgfqpoint{34.789698in}{1.566490in}}%
\pgfpathlineto{\pgfqpoint{34.718731in}{1.510439in}}%
\pgfpathlineto{\pgfqpoint{34.643977in}{1.518304in}}%
\pgfpathlineto{\pgfqpoint{34.571449in}{1.509126in}}%
\pgfpathlineto{\pgfqpoint{34.499162in}{1.532760in}}%
\pgfpathlineto{\pgfqpoint{34.425604in}{1.536135in}}%
\pgfpathlineto{\pgfqpoint{34.354648in}{1.509098in}}%
\pgfpathlineto{\pgfqpoint{34.284339in}{1.501294in}}%
\pgfpathlineto{\pgfqpoint{34.210760in}{1.544412in}}%
\pgfpathlineto{\pgfqpoint{34.138346in}{1.579842in}}%
\pgfpathlineto{\pgfqpoint{34.067999in}{1.597485in}}%
\pgfpathlineto{\pgfqpoint{33.994648in}{1.506848in}}%
\pgfpathlineto{\pgfqpoint{33.923536in}{1.510628in}}%
\pgfpathlineto{\pgfqpoint{33.852393in}{1.523428in}}%
\pgfpathlineto{\pgfqpoint{33.776888in}{1.547229in}}%
\pgfpathlineto{\pgfqpoint{33.703848in}{1.541899in}}%
\pgfpathlineto{\pgfqpoint{33.631884in}{1.540189in}}%
\pgfpathlineto{\pgfqpoint{33.557012in}{1.493064in}}%
\pgfpathlineto{\pgfqpoint{33.483328in}{1.495765in}}%
\pgfpathlineto{\pgfqpoint{33.409286in}{1.496544in}}%
\pgfpathlineto{\pgfqpoint{33.333311in}{1.535470in}}%
\pgfpathlineto{\pgfqpoint{33.259507in}{1.540608in}}%
\pgfpathlineto{\pgfqpoint{33.185787in}{1.504590in}}%
\pgfpathlineto{\pgfqpoint{33.110479in}{1.573511in}}%
\pgfpathlineto{\pgfqpoint{33.037794in}{1.469339in}}%
\pgfpathlineto{\pgfqpoint{32.963168in}{1.535735in}}%
\pgfpathlineto{\pgfqpoint{32.887492in}{1.565013in}}%
\pgfpathlineto{\pgfqpoint{32.813546in}{1.550963in}}%
\pgfpathlineto{\pgfqpoint{32.740263in}{1.520863in}}%
\pgfpathlineto{\pgfqpoint{32.663709in}{1.514249in}}%
\pgfpathlineto{\pgfqpoint{32.590674in}{1.580135in}}%
\pgfpathlineto{\pgfqpoint{32.520401in}{1.609372in}}%
\pgfpathlineto{\pgfqpoint{32.447439in}{1.551309in}}%
\pgfpathlineto{\pgfqpoint{32.377016in}{1.548278in}}%
\pgfpathlineto{\pgfqpoint{32.305440in}{1.510067in}}%
\pgfpathlineto{\pgfqpoint{32.231773in}{1.534031in}}%
\pgfpathlineto{\pgfqpoint{32.161504in}{1.584371in}}%
\pgfpathlineto{\pgfqpoint{32.089684in}{1.475635in}}%
\pgfpathlineto{\pgfqpoint{32.015122in}{1.564938in}}%
\pgfpathlineto{\pgfqpoint{31.943734in}{1.539139in}}%
\pgfpathlineto{\pgfqpoint{31.873539in}{1.524414in}}%
\pgfpathlineto{\pgfqpoint{31.800877in}{1.526914in}}%
\pgfpathlineto{\pgfqpoint{31.728521in}{1.478907in}}%
\pgfpathlineto{\pgfqpoint{31.656180in}{1.507670in}}%
\pgfpathlineto{\pgfqpoint{31.582282in}{1.531156in}}%
\pgfpathlineto{\pgfqpoint{31.510140in}{1.546845in}}%
\pgfpathlineto{\pgfqpoint{31.439415in}{1.521427in}}%
\pgfpathlineto{\pgfqpoint{31.366613in}{1.549389in}}%
\pgfpathlineto{\pgfqpoint{31.294777in}{1.518785in}}%
\pgfpathlineto{\pgfqpoint{31.222913in}{1.554710in}}%
\pgfpathlineto{\pgfqpoint{31.147740in}{1.596386in}}%
\pgfpathlineto{\pgfqpoint{31.074452in}{1.548263in}}%
\pgfpathlineto{\pgfqpoint{31.002514in}{1.529259in}}%
\pgfpathlineto{\pgfqpoint{30.928305in}{1.493309in}}%
\pgfpathlineto{\pgfqpoint{30.854115in}{1.531777in}}%
\pgfpathlineto{\pgfqpoint{30.782249in}{1.546146in}}%
\pgfpathlineto{\pgfqpoint{30.707828in}{1.510561in}}%
\pgfpathlineto{\pgfqpoint{30.634099in}{1.535898in}}%
\pgfpathlineto{\pgfqpoint{30.562714in}{1.547390in}}%
\pgfpathlineto{\pgfqpoint{30.488991in}{1.584407in}}%
\pgfpathlineto{\pgfqpoint{30.417098in}{1.585140in}}%
\pgfpathlineto{\pgfqpoint{30.344328in}{1.529684in}}%
\pgfpathlineto{\pgfqpoint{30.269036in}{1.551759in}}%
\pgfpathlineto{\pgfqpoint{30.195443in}{1.496379in}}%
\pgfpathlineto{\pgfqpoint{30.122796in}{1.517802in}}%
\pgfpathlineto{\pgfqpoint{30.048252in}{1.500767in}}%
\pgfpathlineto{\pgfqpoint{29.974646in}{1.539047in}}%
\pgfpathlineto{\pgfqpoint{29.901656in}{1.496104in}}%
\pgfpathlineto{\pgfqpoint{29.827132in}{1.579406in}}%
\pgfpathlineto{\pgfqpoint{29.755113in}{1.559196in}}%
\pgfpathlineto{\pgfqpoint{29.684427in}{1.527152in}}%
\pgfpathlineto{\pgfqpoint{29.611700in}{1.571508in}}%
\pgfpathlineto{\pgfqpoint{29.540420in}{1.569026in}}%
\pgfpathlineto{\pgfqpoint{29.467898in}{1.524046in}}%
\pgfpathlineto{\pgfqpoint{29.393660in}{1.540183in}}%
\pgfpathlineto{\pgfqpoint{29.320759in}{1.484943in}}%
\pgfpathlineto{\pgfqpoint{29.248973in}{1.527257in}}%
\pgfpathlineto{\pgfqpoint{29.173201in}{1.524477in}}%
\pgfpathlineto{\pgfqpoint{29.100748in}{1.552146in}}%
\pgfpathlineto{\pgfqpoint{29.029887in}{1.583015in}}%
\pgfpathlineto{\pgfqpoint{28.956832in}{1.517961in}}%
\pgfpathlineto{\pgfqpoint{28.885217in}{1.469280in}}%
\pgfpathlineto{\pgfqpoint{28.811911in}{1.517836in}}%
\pgfpathlineto{\pgfqpoint{28.738605in}{1.560384in}}%
\pgfpathlineto{\pgfqpoint{28.668204in}{1.504888in}}%
\pgfpathlineto{\pgfqpoint{28.596041in}{1.503924in}}%
\pgfpathlineto{\pgfqpoint{28.522534in}{1.549251in}}%
\pgfpathlineto{\pgfqpoint{28.451943in}{1.448544in}}%
\pgfpathlineto{\pgfqpoint{28.380501in}{1.548452in}}%
\pgfpathlineto{\pgfqpoint{28.306604in}{1.510336in}}%
\pgfpathlineto{\pgfqpoint{28.234559in}{1.521556in}}%
\pgfpathlineto{\pgfqpoint{28.163352in}{1.544437in}}%
\pgfpathlineto{\pgfqpoint{28.090360in}{1.569511in}}%
\pgfpathlineto{\pgfqpoint{28.018234in}{1.509844in}}%
\pgfpathlineto{\pgfqpoint{27.943911in}{1.468393in}}%
\pgfpathlineto{\pgfqpoint{27.868117in}{1.459063in}}%
\pgfpathlineto{\pgfqpoint{27.795343in}{1.534409in}}%
\pgfpathlineto{\pgfqpoint{27.724006in}{1.492647in}}%
\pgfpathlineto{\pgfqpoint{27.649072in}{1.464448in}}%
\pgfpathlineto{\pgfqpoint{27.574929in}{1.520586in}}%
\pgfpathlineto{\pgfqpoint{27.500063in}{1.481939in}}%
\pgfpathlineto{\pgfqpoint{27.423884in}{1.532911in}}%
\pgfpathlineto{\pgfqpoint{27.350990in}{1.474921in}}%
\pgfpathlineto{\pgfqpoint{27.277488in}{1.495423in}}%
\pgfpathlineto{\pgfqpoint{27.201477in}{1.536899in}}%
\pgfpathlineto{\pgfqpoint{27.128948in}{1.517289in}}%
\pgfpathlineto{\pgfqpoint{27.056835in}{1.590285in}}%
\pgfpathlineto{\pgfqpoint{26.983641in}{1.507448in}}%
\pgfpathlineto{\pgfqpoint{26.912667in}{1.508803in}}%
\pgfpathlineto{\pgfqpoint{26.841234in}{1.531143in}}%
\pgfpathlineto{\pgfqpoint{26.769271in}{1.587070in}}%
\pgfpathlineto{\pgfqpoint{26.699584in}{1.536318in}}%
\pgfpathlineto{\pgfqpoint{26.629572in}{1.505684in}}%
\pgfpathlineto{\pgfqpoint{26.557235in}{1.577246in}}%
\pgfpathlineto{\pgfqpoint{26.487420in}{1.579015in}}%
\pgfpathlineto{\pgfqpoint{26.417231in}{1.554414in}}%
\pgfpathlineto{\pgfqpoint{26.344920in}{1.577262in}}%
\pgfpathlineto{\pgfqpoint{26.274267in}{1.546213in}}%
\pgfpathlineto{\pgfqpoint{26.203572in}{1.531879in}}%
\pgfpathlineto{\pgfqpoint{26.130346in}{1.485416in}}%
\pgfpathlineto{\pgfqpoint{26.058621in}{1.532726in}}%
\pgfpathlineto{\pgfqpoint{25.988588in}{1.637698in}}%
\pgfpathlineto{\pgfqpoint{25.916551in}{1.552118in}}%
\pgfpathlineto{\pgfqpoint{25.845667in}{1.553032in}}%
\pgfpathlineto{\pgfqpoint{25.775695in}{1.540403in}}%
\pgfpathlineto{\pgfqpoint{25.702341in}{1.515995in}}%
\pgfpathlineto{\pgfqpoint{25.631022in}{1.543851in}}%
\pgfpathlineto{\pgfqpoint{25.560110in}{1.507083in}}%
\pgfpathlineto{\pgfqpoint{25.487573in}{1.559183in}}%
\pgfpathlineto{\pgfqpoint{25.417047in}{1.598818in}}%
\pgfpathlineto{\pgfqpoint{25.347124in}{1.429075in}}%
\pgfpathlineto{\pgfqpoint{25.273349in}{1.578390in}}%
\pgfpathlineto{\pgfqpoint{25.203216in}{1.548758in}}%
\pgfpathlineto{\pgfqpoint{25.131109in}{1.493167in}}%
\pgfpathlineto{\pgfqpoint{25.055567in}{1.501851in}}%
\pgfpathlineto{\pgfqpoint{24.983525in}{1.542142in}}%
\pgfpathlineto{\pgfqpoint{24.911741in}{1.502131in}}%
\pgfpathlineto{\pgfqpoint{24.836992in}{1.503712in}}%
\pgfpathlineto{\pgfqpoint{24.764742in}{1.482019in}}%
\pgfpathlineto{\pgfqpoint{24.691660in}{1.550760in}}%
\pgfpathlineto{\pgfqpoint{24.618678in}{1.593510in}}%
\pgfpathlineto{\pgfqpoint{24.548639in}{1.522700in}}%
\pgfpathlineto{\pgfqpoint{24.476339in}{1.496645in}}%
\pgfpathlineto{\pgfqpoint{24.400328in}{1.524874in}}%
\pgfpathlineto{\pgfqpoint{24.329090in}{1.515672in}}%
\pgfpathlineto{\pgfqpoint{24.257307in}{1.551779in}}%
\pgfpathlineto{\pgfqpoint{24.183899in}{1.573078in}}%
\pgfpathlineto{\pgfqpoint{24.111784in}{1.480083in}}%
\pgfpathlineto{\pgfqpoint{24.039255in}{1.583860in}}%
\pgfpathlineto{\pgfqpoint{23.966722in}{1.482405in}}%
\pgfpathlineto{\pgfqpoint{23.895213in}{1.559116in}}%
\pgfpathlineto{\pgfqpoint{23.824951in}{1.565118in}}%
\pgfpathlineto{\pgfqpoint{23.752361in}{1.598700in}}%
\pgfpathlineto{\pgfqpoint{23.681448in}{1.532899in}}%
\pgfpathlineto{\pgfqpoint{23.610036in}{1.604120in}}%
\pgfpathlineto{\pgfqpoint{23.537850in}{1.584190in}}%
\pgfpathlineto{\pgfqpoint{23.467323in}{1.504832in}}%
\pgfpathlineto{\pgfqpoint{23.396126in}{1.546275in}}%
\pgfpathlineto{\pgfqpoint{23.323995in}{1.506883in}}%
\pgfpathlineto{\pgfqpoint{23.253484in}{1.547590in}}%
\pgfpathlineto{\pgfqpoint{23.184270in}{1.562385in}}%
\pgfpathlineto{\pgfqpoint{23.113267in}{1.521624in}}%
\pgfpathlineto{\pgfqpoint{23.043397in}{1.575295in}}%
\pgfpathlineto{\pgfqpoint{22.973868in}{1.560943in}}%
\pgfpathlineto{\pgfqpoint{22.902896in}{1.599819in}}%
\pgfpathlineto{\pgfqpoint{22.833628in}{1.614311in}}%
\pgfpathlineto{\pgfqpoint{22.764651in}{1.627833in}}%
\pgfpathlineto{\pgfqpoint{22.694016in}{1.591324in}}%
\pgfpathlineto{\pgfqpoint{22.624114in}{1.585803in}}%
\pgfpathlineto{\pgfqpoint{22.553535in}{1.587263in}}%
\pgfpathlineto{\pgfqpoint{22.482516in}{1.558749in}}%
\pgfpathlineto{\pgfqpoint{22.413449in}{1.500385in}}%
\pgfpathlineto{\pgfqpoint{22.343331in}{1.557458in}}%
\pgfpathlineto{\pgfqpoint{22.268824in}{1.501061in}}%
\pgfpathlineto{\pgfqpoint{22.195707in}{1.470709in}}%
\pgfpathlineto{\pgfqpoint{22.122462in}{1.531987in}}%
\pgfpathlineto{\pgfqpoint{22.048040in}{1.514715in}}%
\pgfpathlineto{\pgfqpoint{21.976836in}{1.478797in}}%
\pgfpathlineto{\pgfqpoint{21.903868in}{1.524106in}}%
\pgfpathlineto{\pgfqpoint{21.828555in}{1.545626in}}%
\pgfpathlineto{\pgfqpoint{21.756227in}{1.500466in}}%
\pgfpathlineto{\pgfqpoint{21.683994in}{1.562266in}}%
\pgfpathlineto{\pgfqpoint{21.610878in}{1.535035in}}%
\pgfpathlineto{\pgfqpoint{21.539601in}{1.476574in}}%
\pgfpathlineto{\pgfqpoint{21.467741in}{1.536204in}}%
\pgfpathlineto{\pgfqpoint{21.394839in}{1.545125in}}%
\pgfpathlineto{\pgfqpoint{21.324498in}{1.540504in}}%
\pgfpathlineto{\pgfqpoint{21.254645in}{1.536177in}}%
\pgfpathlineto{\pgfqpoint{21.181233in}{1.498560in}}%
\pgfpathlineto{\pgfqpoint{21.110656in}{1.507756in}}%
\pgfpathlineto{\pgfqpoint{21.040864in}{1.557813in}}%
\pgfpathlineto{\pgfqpoint{20.969373in}{1.475010in}}%
\pgfpathlineto{\pgfqpoint{20.899587in}{1.557287in}}%
\pgfpathlineto{\pgfqpoint{20.830586in}{1.600105in}}%
\pgfpathlineto{\pgfqpoint{20.758814in}{1.518855in}}%
\pgfpathlineto{\pgfqpoint{20.688980in}{1.523452in}}%
\pgfpathlineto{\pgfqpoint{20.618428in}{1.575582in}}%
\pgfpathlineto{\pgfqpoint{20.547172in}{1.479274in}}%
\pgfpathlineto{\pgfqpoint{20.476961in}{1.568820in}}%
\pgfpathlineto{\pgfqpoint{20.407276in}{1.589136in}}%
\pgfpathlineto{\pgfqpoint{20.334605in}{1.512026in}}%
\pgfpathlineto{\pgfqpoint{20.263963in}{1.553784in}}%
\pgfpathlineto{\pgfqpoint{20.193561in}{1.528517in}}%
\pgfpathlineto{\pgfqpoint{20.121885in}{1.569264in}}%
\pgfpathlineto{\pgfqpoint{20.052071in}{1.524114in}}%
\pgfpathlineto{\pgfqpoint{19.981199in}{1.576325in}}%
\pgfpathlineto{\pgfqpoint{19.907049in}{1.535340in}}%
\pgfpathlineto{\pgfqpoint{19.835928in}{1.578664in}}%
\pgfpathlineto{\pgfqpoint{19.766060in}{1.590515in}}%
\pgfpathlineto{\pgfqpoint{19.693765in}{1.551767in}}%
\pgfpathlineto{\pgfqpoint{19.623855in}{1.582011in}}%
\pgfpathlineto{\pgfqpoint{19.553641in}{1.547421in}}%
\pgfpathlineto{\pgfqpoint{19.482791in}{1.558572in}}%
\pgfpathlineto{\pgfqpoint{19.412352in}{1.526749in}}%
\pgfpathlineto{\pgfqpoint{19.340700in}{1.530595in}}%
\pgfpathlineto{\pgfqpoint{19.266961in}{1.484903in}}%
\pgfpathlineto{\pgfqpoint{19.195134in}{1.542659in}}%
\pgfpathlineto{\pgfqpoint{19.123240in}{1.474807in}}%
\pgfpathlineto{\pgfqpoint{19.048835in}{1.558845in}}%
\pgfpathlineto{\pgfqpoint{18.977585in}{1.536842in}}%
\pgfpathlineto{\pgfqpoint{18.906958in}{1.556952in}}%
\pgfpathlineto{\pgfqpoint{18.834141in}{1.497654in}}%
\pgfpathlineto{\pgfqpoint{18.763558in}{1.548196in}}%
\pgfpathlineto{\pgfqpoint{18.692839in}{1.573012in}}%
\pgfpathlineto{\pgfqpoint{18.619849in}{1.522647in}}%
\pgfpathlineto{\pgfqpoint{18.549509in}{1.554154in}}%
\pgfpathlineto{\pgfqpoint{18.479143in}{1.530036in}}%
\pgfpathlineto{\pgfqpoint{18.406899in}{1.576170in}}%
\pgfpathlineto{\pgfqpoint{18.336223in}{1.592736in}}%
\pgfpathlineto{\pgfqpoint{18.266780in}{1.551346in}}%
\pgfpathlineto{\pgfqpoint{18.195268in}{1.555759in}}%
\pgfpathlineto{\pgfqpoint{18.125960in}{1.550138in}}%
\pgfpathlineto{\pgfqpoint{18.056086in}{1.567174in}}%
\pgfpathlineto{\pgfqpoint{17.983524in}{1.490930in}}%
\pgfpathlineto{\pgfqpoint{17.912627in}{1.606337in}}%
\pgfpathlineto{\pgfqpoint{17.844148in}{1.568296in}}%
\pgfpathlineto{\pgfqpoint{17.772898in}{1.531912in}}%
\pgfpathlineto{\pgfqpoint{17.703480in}{1.569518in}}%
\pgfpathlineto{\pgfqpoint{17.634782in}{1.532540in}}%
\pgfpathlineto{\pgfqpoint{17.563254in}{1.551407in}}%
\pgfpathlineto{\pgfqpoint{17.494859in}{1.566279in}}%
\pgfpathlineto{\pgfqpoint{17.426570in}{1.614741in}}%
\pgfpathlineto{\pgfqpoint{17.356523in}{1.556708in}}%
\pgfpathlineto{\pgfqpoint{17.288800in}{1.576164in}}%
\pgfpathlineto{\pgfqpoint{17.220960in}{1.520709in}}%
\pgfpathlineto{\pgfqpoint{17.150280in}{1.549893in}}%
\pgfpathlineto{\pgfqpoint{17.081113in}{1.607636in}}%
\pgfpathlineto{\pgfqpoint{17.013249in}{1.579836in}}%
\pgfpathlineto{\pgfqpoint{16.942299in}{1.530455in}}%
\pgfpathlineto{\pgfqpoint{16.872974in}{1.567216in}}%
\pgfpathlineto{\pgfqpoint{16.803402in}{1.535742in}}%
\pgfpathlineto{\pgfqpoint{16.731829in}{1.545614in}}%
\pgfpathlineto{\pgfqpoint{16.662798in}{1.576556in}}%
\pgfpathlineto{\pgfqpoint{16.592371in}{1.568175in}}%
\pgfpathlineto{\pgfqpoint{16.519393in}{1.512666in}}%
\pgfpathlineto{\pgfqpoint{16.445981in}{1.479184in}}%
\pgfpathlineto{\pgfqpoint{16.373054in}{1.518566in}}%
\pgfpathlineto{\pgfqpoint{16.299236in}{1.547781in}}%
\pgfpathlineto{\pgfqpoint{16.229236in}{1.588895in}}%
\pgfpathlineto{\pgfqpoint{16.160188in}{1.615478in}}%
\pgfpathlineto{\pgfqpoint{16.088678in}{1.562144in}}%
\pgfpathlineto{\pgfqpoint{16.019598in}{1.533442in}}%
\pgfpathlineto{\pgfqpoint{15.949677in}{1.570939in}}%
\pgfpathlineto{\pgfqpoint{15.878678in}{1.537539in}}%
\pgfpathlineto{\pgfqpoint{15.807467in}{1.558776in}}%
\pgfpathlineto{\pgfqpoint{15.737902in}{1.546544in}}%
\pgfpathlineto{\pgfqpoint{15.665480in}{1.485292in}}%
\pgfpathlineto{\pgfqpoint{15.595878in}{1.576527in}}%
\pgfpathlineto{\pgfqpoint{15.527961in}{1.591577in}}%
\pgfpathlineto{\pgfqpoint{15.457342in}{1.518712in}}%
\pgfpathlineto{\pgfqpoint{15.388582in}{1.568504in}}%
\pgfpathlineto{\pgfqpoint{15.320168in}{1.594538in}}%
\pgfpathlineto{\pgfqpoint{15.249597in}{1.556145in}}%
\pgfpathlineto{\pgfqpoint{15.182512in}{1.625603in}}%
\pgfpathlineto{\pgfqpoint{15.115140in}{1.491634in}}%
\pgfpathlineto{\pgfqpoint{15.044439in}{1.563018in}}%
\pgfpathlineto{\pgfqpoint{14.976372in}{1.574084in}}%
\pgfpathlineto{\pgfqpoint{14.908199in}{1.487275in}}%
\pgfpathlineto{\pgfqpoint{14.836607in}{1.555255in}}%
\pgfpathlineto{\pgfqpoint{14.768225in}{1.564815in}}%
\pgfpathlineto{\pgfqpoint{14.700003in}{1.560771in}}%
\pgfpathlineto{\pgfqpoint{14.630523in}{1.647021in}}%
\pgfpathlineto{\pgfqpoint{14.562284in}{1.585704in}}%
\pgfpathlineto{\pgfqpoint{14.494110in}{1.591320in}}%
\pgfpathlineto{\pgfqpoint{14.424138in}{1.612428in}}%
\pgfpathlineto{\pgfqpoint{14.355979in}{1.562231in}}%
\pgfpathlineto{\pgfqpoint{14.286685in}{1.518898in}}%
\pgfpathlineto{\pgfqpoint{14.216102in}{1.565625in}}%
\pgfpathlineto{\pgfqpoint{14.149562in}{1.603040in}}%
\pgfpathlineto{\pgfqpoint{14.081564in}{1.596280in}}%
\pgfpathlineto{\pgfqpoint{14.012639in}{1.516863in}}%
\pgfpathlineto{\pgfqpoint{13.944783in}{1.625153in}}%
\pgfpathlineto{\pgfqpoint{13.876785in}{1.533353in}}%
\pgfpathlineto{\pgfqpoint{13.804228in}{1.555707in}}%
\pgfpathlineto{\pgfqpoint{13.735499in}{1.590527in}}%
\pgfpathlineto{\pgfqpoint{13.666880in}{1.566668in}}%
\pgfpathlineto{\pgfqpoint{13.595052in}{1.554583in}}%
\pgfpathlineto{\pgfqpoint{13.526360in}{1.562836in}}%
\pgfpathlineto{\pgfqpoint{13.457833in}{1.561703in}}%
\pgfpathlineto{\pgfqpoint{13.387091in}{1.577859in}}%
\pgfpathlineto{\pgfqpoint{13.319414in}{1.553188in}}%
\pgfpathlineto{\pgfqpoint{13.250526in}{1.514976in}}%
\pgfpathlineto{\pgfqpoint{13.178279in}{1.605883in}}%
\pgfpathlineto{\pgfqpoint{13.109710in}{1.539729in}}%
\pgfpathlineto{\pgfqpoint{13.041078in}{1.522245in}}%
\pgfpathlineto{\pgfqpoint{12.969036in}{1.532065in}}%
\pgfpathlineto{\pgfqpoint{12.900232in}{1.588449in}}%
\pgfpathlineto{\pgfqpoint{12.830499in}{1.556532in}}%
\pgfpathlineto{\pgfqpoint{12.759967in}{1.597326in}}%
\pgfpathlineto{\pgfqpoint{12.692271in}{1.569592in}}%
\pgfpathlineto{\pgfqpoint{12.623527in}{1.546145in}}%
\pgfpathlineto{\pgfqpoint{12.552597in}{1.583495in}}%
\pgfpathlineto{\pgfqpoint{12.484090in}{1.553558in}}%
\pgfpathlineto{\pgfqpoint{12.416448in}{1.590858in}}%
\pgfpathlineto{\pgfqpoint{12.347008in}{1.582946in}}%
\pgfpathlineto{\pgfqpoint{12.280761in}{1.595843in}}%
\pgfpathlineto{\pgfqpoint{12.213872in}{1.548130in}}%
\pgfpathlineto{\pgfqpoint{12.144815in}{1.548198in}}%
\pgfpathlineto{\pgfqpoint{12.076638in}{1.576044in}}%
\pgfpathlineto{\pgfqpoint{12.008843in}{1.542631in}}%
\pgfpathlineto{\pgfqpoint{11.938397in}{1.530226in}}%
\pgfpathlineto{\pgfqpoint{11.871239in}{1.623511in}}%
\pgfpathlineto{\pgfqpoint{11.804695in}{1.609101in}}%
\pgfpathlineto{\pgfqpoint{11.736702in}{1.591494in}}%
\pgfpathlineto{\pgfqpoint{11.669969in}{1.591651in}}%
\pgfpathlineto{\pgfqpoint{11.602395in}{1.588063in}}%
\pgfpathlineto{\pgfqpoint{11.532877in}{1.538342in}}%
\pgfpathlineto{\pgfqpoint{11.465294in}{1.651022in}}%
\pgfpathlineto{\pgfqpoint{11.398136in}{1.568638in}}%
\pgfpathlineto{\pgfqpoint{11.328677in}{1.640298in}}%
\pgfpathlineto{\pgfqpoint{11.261777in}{1.617115in}}%
\pgfpathlineto{\pgfqpoint{11.194984in}{1.563569in}}%
\pgfpathlineto{\pgfqpoint{11.125133in}{1.562163in}}%
\pgfpathlineto{\pgfqpoint{11.055389in}{1.597329in}}%
\pgfpathlineto{\pgfqpoint{10.986941in}{1.616858in}}%
\pgfpathlineto{\pgfqpoint{10.916008in}{1.558010in}}%
\pgfpathlineto{\pgfqpoint{10.846851in}{1.584449in}}%
\pgfpathlineto{\pgfqpoint{10.777925in}{1.672266in}}%
\pgfpathlineto{\pgfqpoint{10.708017in}{1.559245in}}%
\pgfpathlineto{\pgfqpoint{10.639006in}{1.565727in}}%
\pgfpathlineto{\pgfqpoint{10.569898in}{1.560845in}}%
\pgfpathlineto{\pgfqpoint{10.498528in}{1.575255in}}%
\pgfpathlineto{\pgfqpoint{10.428999in}{1.601026in}}%
\pgfpathlineto{\pgfqpoint{10.360853in}{1.609879in}}%
\pgfpathlineto{\pgfqpoint{10.290696in}{1.594108in}}%
\pgfpathlineto{\pgfqpoint{10.222102in}{1.577171in}}%
\pgfpathlineto{\pgfqpoint{10.153891in}{1.577726in}}%
\pgfpathlineto{\pgfqpoint{10.081602in}{1.579021in}}%
\pgfpathlineto{\pgfqpoint{10.012687in}{1.571830in}}%
\pgfpathlineto{\pgfqpoint{9.944615in}{1.607414in}}%
\pgfpathlineto{\pgfqpoint{9.876148in}{1.600879in}}%
\pgfpathlineto{\pgfqpoint{9.808815in}{1.632972in}}%
\pgfpathlineto{\pgfqpoint{9.740273in}{1.514664in}}%
\pgfpathlineto{\pgfqpoint{9.670768in}{1.605175in}}%
\pgfpathlineto{\pgfqpoint{9.603748in}{1.530400in}}%
\pgfpathlineto{\pgfqpoint{9.536145in}{1.584048in}}%
\pgfpathlineto{\pgfqpoint{9.467314in}{1.669498in}}%
\pgfpathlineto{\pgfqpoint{9.402138in}{1.616851in}}%
\pgfpathlineto{\pgfqpoint{9.334728in}{1.556183in}}%
\pgfpathlineto{\pgfqpoint{9.265262in}{1.608840in}}%
\pgfpathlineto{\pgfqpoint{9.198477in}{1.594816in}}%
\pgfpathlineto{\pgfqpoint{9.130518in}{1.567871in}}%
\pgfpathlineto{\pgfqpoint{9.059607in}{1.530704in}}%
\pgfpathlineto{\pgfqpoint{8.991646in}{1.616904in}}%
\pgfpathlineto{\pgfqpoint{8.924860in}{1.650890in}}%
\pgfpathlineto{\pgfqpoint{8.856098in}{1.573738in}}%
\pgfpathlineto{\pgfqpoint{8.787638in}{1.628773in}}%
\pgfpathlineto{\pgfqpoint{8.720615in}{1.653656in}}%
\pgfpathlineto{\pgfqpoint{8.651757in}{1.528788in}}%
\pgfpathlineto{\pgfqpoint{8.583601in}{1.564719in}}%
\pgfpathlineto{\pgfqpoint{8.515085in}{1.561282in}}%
\pgfpathlineto{\pgfqpoint{8.444285in}{1.618288in}}%
\pgfpathlineto{\pgfqpoint{8.376454in}{1.528210in}}%
\pgfpathlineto{\pgfqpoint{8.308718in}{1.612249in}}%
\pgfpathlineto{\pgfqpoint{8.237831in}{1.525261in}}%
\pgfpathlineto{\pgfqpoint{8.168934in}{1.584183in}}%
\pgfpathlineto{\pgfqpoint{8.100457in}{1.567280in}}%
\pgfpathlineto{\pgfqpoint{8.028258in}{1.528063in}}%
\pgfpathlineto{\pgfqpoint{7.957986in}{1.588459in}}%
\pgfpathlineto{\pgfqpoint{7.887441in}{1.543653in}}%
\pgfpathlineto{\pgfqpoint{7.812286in}{1.502950in}}%
\pgfpathlineto{\pgfqpoint{7.741421in}{1.595614in}}%
\pgfpathlineto{\pgfqpoint{7.671131in}{1.558132in}}%
\pgfpathlineto{\pgfqpoint{7.598690in}{1.571863in}}%
\pgfpathlineto{\pgfqpoint{7.527107in}{1.536944in}}%
\pgfpathlineto{\pgfqpoint{7.455306in}{1.513495in}}%
\pgfpathlineto{\pgfqpoint{7.379810in}{1.509290in}}%
\pgfpathlineto{\pgfqpoint{7.308295in}{1.475877in}}%
\pgfpathlineto{\pgfqpoint{7.238831in}{1.570075in}}%
\pgfpathlineto{\pgfqpoint{7.171034in}{1.649474in}}%
\pgfpathlineto{\pgfqpoint{7.106301in}{1.620571in}}%
\pgfpathlineto{\pgfqpoint{7.040805in}{1.612247in}}%
\pgfpathlineto{\pgfqpoint{6.973333in}{1.623196in}}%
\pgfpathlineto{\pgfqpoint{6.906544in}{1.609310in}}%
\pgfpathlineto{\pgfqpoint{6.839348in}{1.618981in}}%
\pgfpathlineto{\pgfqpoint{6.770581in}{1.588416in}}%
\pgfpathlineto{\pgfqpoint{6.704271in}{1.573417in}}%
\pgfpathlineto{\pgfqpoint{6.636218in}{1.538703in}}%
\pgfpathlineto{\pgfqpoint{6.568004in}{1.636712in}}%
\pgfpathlineto{\pgfqpoint{6.502885in}{1.600212in}}%
\pgfpathlineto{\pgfqpoint{6.437563in}{1.667753in}}%
\pgfpathlineto{\pgfqpoint{6.369842in}{1.645747in}}%
\pgfpathlineto{\pgfqpoint{6.303479in}{1.572865in}}%
\pgfpathlineto{\pgfqpoint{6.236861in}{1.638834in}}%
\pgfpathlineto{\pgfqpoint{6.169451in}{1.620713in}}%
\pgfpathlineto{\pgfqpoint{6.103260in}{1.567223in}}%
\pgfpathlineto{\pgfqpoint{6.036866in}{1.596194in}}%
\pgfpathlineto{\pgfqpoint{5.968494in}{1.598652in}}%
\pgfpathlineto{\pgfqpoint{5.901623in}{1.585460in}}%
\pgfpathlineto{\pgfqpoint{5.834669in}{1.669465in}}%
\pgfpathlineto{\pgfqpoint{5.766709in}{1.591992in}}%
\pgfpathlineto{\pgfqpoint{5.700467in}{1.635448in}}%
\pgfpathlineto{\pgfqpoint{5.633136in}{1.531670in}}%
\pgfpathlineto{\pgfqpoint{5.561464in}{1.566624in}}%
\pgfpathlineto{\pgfqpoint{5.492693in}{1.581814in}}%
\pgfpathlineto{\pgfqpoint{5.424203in}{1.539101in}}%
\pgfpathlineto{\pgfqpoint{5.353942in}{1.546209in}}%
\pgfpathlineto{\pgfqpoint{5.283251in}{1.578277in}}%
\pgfpathlineto{\pgfqpoint{5.213575in}{1.564353in}}%
\pgfpathlineto{\pgfqpoint{5.142206in}{1.514728in}}%
\pgfpathlineto{\pgfqpoint{5.073090in}{1.586937in}}%
\pgfpathlineto{\pgfqpoint{5.005143in}{1.541610in}}%
\pgfpathlineto{\pgfqpoint{4.935251in}{1.543441in}}%
\pgfpathlineto{\pgfqpoint{4.866311in}{1.551953in}}%
\pgfpathlineto{\pgfqpoint{4.796564in}{1.554831in}}%
\pgfpathlineto{\pgfqpoint{4.726503in}{1.646353in}}%
\pgfpathlineto{\pgfqpoint{4.658575in}{1.553886in}}%
\pgfpathlineto{\pgfqpoint{4.590503in}{1.619428in}}%
\pgfpathlineto{\pgfqpoint{4.520559in}{1.573513in}}%
\pgfpathlineto{\pgfqpoint{4.454265in}{1.573094in}}%
\pgfpathlineto{\pgfqpoint{4.387175in}{1.602589in}}%
\pgfpathlineto{\pgfqpoint{4.319299in}{1.631713in}}%
\pgfpathlineto{\pgfqpoint{4.253047in}{1.597809in}}%
\pgfpathlineto{\pgfqpoint{4.186150in}{1.577993in}}%
\pgfpathlineto{\pgfqpoint{4.116566in}{1.541750in}}%
\pgfpathlineto{\pgfqpoint{4.048987in}{1.536941in}}%
\pgfpathlineto{\pgfqpoint{3.981945in}{1.579845in}}%
\pgfpathlineto{\pgfqpoint{3.914312in}{1.648297in}}%
\pgfpathlineto{\pgfqpoint{3.848285in}{1.555661in}}%
\pgfpathlineto{\pgfqpoint{3.781205in}{1.641367in}}%
\pgfpathlineto{\pgfqpoint{3.712535in}{1.559422in}}%
\pgfpathlineto{\pgfqpoint{3.643958in}{1.549270in}}%
\pgfpathlineto{\pgfqpoint{3.575543in}{1.615504in}}%
\pgfpathlineto{\pgfqpoint{3.503039in}{1.512115in}}%
\pgfpathlineto{\pgfqpoint{3.430573in}{1.518470in}}%
\pgfpathlineto{\pgfqpoint{3.356465in}{1.481625in}}%
\pgfpathlineto{\pgfqpoint{3.277599in}{1.460074in}}%
\pgfpathlineto{\pgfqpoint{3.195433in}{1.535981in}}%
\pgfpathlineto{\pgfqpoint{3.123114in}{1.539362in}}%
\pgfpathlineto{\pgfqpoint{3.047640in}{1.497556in}}%
\pgfpathlineto{\pgfqpoint{2.975015in}{1.577858in}}%
\pgfpathlineto{\pgfqpoint{2.902674in}{1.503915in}}%
\pgfpathlineto{\pgfqpoint{2.826501in}{1.511302in}}%
\pgfpathlineto{\pgfqpoint{2.753491in}{1.526723in}}%
\pgfpathlineto{\pgfqpoint{2.681331in}{1.508459in}}%
\pgfpathlineto{\pgfqpoint{2.606023in}{1.539689in}}%
\pgfpathlineto{\pgfqpoint{2.534117in}{1.571750in}}%
\pgfpathlineto{\pgfqpoint{2.462769in}{1.533021in}}%
\pgfpathlineto{\pgfqpoint{2.387948in}{1.542608in}}%
\pgfpathlineto{\pgfqpoint{2.317152in}{1.565343in}}%
\pgfpathlineto{\pgfqpoint{2.248172in}{1.557431in}}%
\pgfpathlineto{\pgfqpoint{2.177337in}{1.515214in}}%
\pgfpathlineto{\pgfqpoint{2.108883in}{1.600714in}}%
\pgfpathlineto{\pgfqpoint{2.041179in}{1.615811in}}%
\pgfpathlineto{\pgfqpoint{1.970951in}{1.597714in}}%
\pgfpathlineto{\pgfqpoint{1.904458in}{1.607168in}}%
\pgfpathlineto{\pgfqpoint{1.835890in}{1.572583in}}%
\pgfpathlineto{\pgfqpoint{1.766402in}{1.607225in}}%
\pgfpathlineto{\pgfqpoint{1.698662in}{1.543801in}}%
\pgfpathlineto{\pgfqpoint{1.628862in}{1.552264in}}%
\pgfpathlineto{\pgfqpoint{1.557461in}{1.605381in}}%
\pgfpathlineto{\pgfqpoint{1.489306in}{1.586306in}}%
\pgfpathlineto{\pgfqpoint{1.421095in}{1.553948in}}%
\pgfpathlineto{\pgfqpoint{1.349373in}{1.576150in}}%
\pgfpathlineto{\pgfqpoint{1.283036in}{1.605899in}}%
\pgfpathlineto{\pgfqpoint{1.216322in}{1.566196in}}%
\pgfpathlineto{\pgfqpoint{1.147369in}{1.515255in}}%
\pgfpathlineto{\pgfqpoint{1.079942in}{1.615813in}}%
\pgfpathlineto{\pgfqpoint{1.012853in}{1.550858in}}%
\pgfpathlineto{\pgfqpoint{0.942110in}{1.586095in}}%
\pgfpathlineto{\pgfqpoint{0.875335in}{1.613006in}}%
\pgfpathlineto{\pgfqpoint{0.807094in}{1.307859in}}%
\pgfpathclose%
\pgfusepath{fill}%
\end{pgfscope}%
\begin{pgfscope}%
\pgfpathrectangle{\pgfqpoint{0.781402in}{0.773588in}}{\pgfqpoint{2.110351in}{5.415119in}}%
\pgfusepath{clip}%
\pgfsetbuttcap%
\pgfsetroundjoin%
\definecolor{currentfill}{rgb}{0.580392,0.403922,0.741176}%
\pgfsetfillcolor{currentfill}%
\pgfsetlinewidth{0.000000pt}%
\definecolor{currentstroke}{rgb}{0.000000,0.000000,0.000000}%
\pgfsetstrokecolor{currentstroke}%
\pgfsetdash{}{0pt}%
\pgfpathmoveto{\pgfqpoint{0.807094in}{1.768207in}}%
\pgfpathlineto{\pgfqpoint{0.807094in}{1.307859in}}%
\pgfpathlineto{\pgfqpoint{0.875335in}{1.613006in}}%
\pgfpathlineto{\pgfqpoint{0.942110in}{1.586095in}}%
\pgfpathlineto{\pgfqpoint{1.012853in}{1.550858in}}%
\pgfpathlineto{\pgfqpoint{1.079942in}{1.615813in}}%
\pgfpathlineto{\pgfqpoint{1.147369in}{1.515255in}}%
\pgfpathlineto{\pgfqpoint{1.216322in}{1.566196in}}%
\pgfpathlineto{\pgfqpoint{1.283036in}{1.605899in}}%
\pgfpathlineto{\pgfqpoint{1.349373in}{1.576150in}}%
\pgfpathlineto{\pgfqpoint{1.421095in}{1.553948in}}%
\pgfpathlineto{\pgfqpoint{1.489306in}{1.586306in}}%
\pgfpathlineto{\pgfqpoint{1.557461in}{1.605381in}}%
\pgfpathlineto{\pgfqpoint{1.628862in}{1.552264in}}%
\pgfpathlineto{\pgfqpoint{1.698662in}{1.543801in}}%
\pgfpathlineto{\pgfqpoint{1.766402in}{1.607225in}}%
\pgfpathlineto{\pgfqpoint{1.835890in}{1.572583in}}%
\pgfpathlineto{\pgfqpoint{1.904458in}{1.607168in}}%
\pgfpathlineto{\pgfqpoint{1.970951in}{1.597714in}}%
\pgfpathlineto{\pgfqpoint{2.041179in}{1.615811in}}%
\pgfpathlineto{\pgfqpoint{2.108883in}{1.600714in}}%
\pgfpathlineto{\pgfqpoint{2.177337in}{1.515214in}}%
\pgfpathlineto{\pgfqpoint{2.248172in}{1.557431in}}%
\pgfpathlineto{\pgfqpoint{2.317152in}{1.565343in}}%
\pgfpathlineto{\pgfqpoint{2.387948in}{1.542608in}}%
\pgfpathlineto{\pgfqpoint{2.462769in}{1.533021in}}%
\pgfpathlineto{\pgfqpoint{2.534117in}{1.571750in}}%
\pgfpathlineto{\pgfqpoint{2.606023in}{1.539689in}}%
\pgfpathlineto{\pgfqpoint{2.681331in}{1.508459in}}%
\pgfpathlineto{\pgfqpoint{2.753491in}{1.526723in}}%
\pgfpathlineto{\pgfqpoint{2.826501in}{1.511302in}}%
\pgfpathlineto{\pgfqpoint{2.902674in}{1.503915in}}%
\pgfpathlineto{\pgfqpoint{2.975015in}{1.577858in}}%
\pgfpathlineto{\pgfqpoint{3.047640in}{1.497556in}}%
\pgfpathlineto{\pgfqpoint{3.123114in}{1.539362in}}%
\pgfpathlineto{\pgfqpoint{3.195433in}{1.535981in}}%
\pgfpathlineto{\pgfqpoint{3.277599in}{1.460074in}}%
\pgfpathlineto{\pgfqpoint{3.356465in}{1.481625in}}%
\pgfpathlineto{\pgfqpoint{3.430573in}{1.518470in}}%
\pgfpathlineto{\pgfqpoint{3.503039in}{1.512115in}}%
\pgfpathlineto{\pgfqpoint{3.575543in}{1.615504in}}%
\pgfpathlineto{\pgfqpoint{3.643958in}{1.549270in}}%
\pgfpathlineto{\pgfqpoint{3.712535in}{1.559422in}}%
\pgfpathlineto{\pgfqpoint{3.781205in}{1.641367in}}%
\pgfpathlineto{\pgfqpoint{3.848285in}{1.555661in}}%
\pgfpathlineto{\pgfqpoint{3.914312in}{1.648297in}}%
\pgfpathlineto{\pgfqpoint{3.981945in}{1.579845in}}%
\pgfpathlineto{\pgfqpoint{4.048987in}{1.536941in}}%
\pgfpathlineto{\pgfqpoint{4.116566in}{1.541750in}}%
\pgfpathlineto{\pgfqpoint{4.186150in}{1.577993in}}%
\pgfpathlineto{\pgfqpoint{4.253047in}{1.597809in}}%
\pgfpathlineto{\pgfqpoint{4.319299in}{1.631713in}}%
\pgfpathlineto{\pgfqpoint{4.387175in}{1.602589in}}%
\pgfpathlineto{\pgfqpoint{4.454265in}{1.573094in}}%
\pgfpathlineto{\pgfqpoint{4.520559in}{1.573513in}}%
\pgfpathlineto{\pgfqpoint{4.590503in}{1.619428in}}%
\pgfpathlineto{\pgfqpoint{4.658575in}{1.553886in}}%
\pgfpathlineto{\pgfqpoint{4.726503in}{1.646353in}}%
\pgfpathlineto{\pgfqpoint{4.796564in}{1.554831in}}%
\pgfpathlineto{\pgfqpoint{4.866311in}{1.551953in}}%
\pgfpathlineto{\pgfqpoint{4.935251in}{1.543441in}}%
\pgfpathlineto{\pgfqpoint{5.005143in}{1.541610in}}%
\pgfpathlineto{\pgfqpoint{5.073090in}{1.586937in}}%
\pgfpathlineto{\pgfqpoint{5.142206in}{1.514728in}}%
\pgfpathlineto{\pgfqpoint{5.213575in}{1.564353in}}%
\pgfpathlineto{\pgfqpoint{5.283251in}{1.578277in}}%
\pgfpathlineto{\pgfqpoint{5.353942in}{1.546209in}}%
\pgfpathlineto{\pgfqpoint{5.424203in}{1.539101in}}%
\pgfpathlineto{\pgfqpoint{5.492693in}{1.581814in}}%
\pgfpathlineto{\pgfqpoint{5.561464in}{1.566624in}}%
\pgfpathlineto{\pgfqpoint{5.633136in}{1.531670in}}%
\pgfpathlineto{\pgfqpoint{5.700467in}{1.635448in}}%
\pgfpathlineto{\pgfqpoint{5.766709in}{1.591992in}}%
\pgfpathlineto{\pgfqpoint{5.834669in}{1.669465in}}%
\pgfpathlineto{\pgfqpoint{5.901623in}{1.585460in}}%
\pgfpathlineto{\pgfqpoint{5.968494in}{1.598652in}}%
\pgfpathlineto{\pgfqpoint{6.036866in}{1.596194in}}%
\pgfpathlineto{\pgfqpoint{6.103260in}{1.567223in}}%
\pgfpathlineto{\pgfqpoint{6.169451in}{1.620713in}}%
\pgfpathlineto{\pgfqpoint{6.236861in}{1.638834in}}%
\pgfpathlineto{\pgfqpoint{6.303479in}{1.572865in}}%
\pgfpathlineto{\pgfqpoint{6.369842in}{1.645747in}}%
\pgfpathlineto{\pgfqpoint{6.437563in}{1.667753in}}%
\pgfpathlineto{\pgfqpoint{6.502885in}{1.600212in}}%
\pgfpathlineto{\pgfqpoint{6.568004in}{1.636712in}}%
\pgfpathlineto{\pgfqpoint{6.636218in}{1.538703in}}%
\pgfpathlineto{\pgfqpoint{6.704271in}{1.573417in}}%
\pgfpathlineto{\pgfqpoint{6.770581in}{1.588416in}}%
\pgfpathlineto{\pgfqpoint{6.839348in}{1.618981in}}%
\pgfpathlineto{\pgfqpoint{6.906544in}{1.609310in}}%
\pgfpathlineto{\pgfqpoint{6.973333in}{1.623196in}}%
\pgfpathlineto{\pgfqpoint{7.040805in}{1.612247in}}%
\pgfpathlineto{\pgfqpoint{7.106301in}{1.620571in}}%
\pgfpathlineto{\pgfqpoint{7.171034in}{1.649474in}}%
\pgfpathlineto{\pgfqpoint{7.238831in}{1.570075in}}%
\pgfpathlineto{\pgfqpoint{7.308295in}{1.475877in}}%
\pgfpathlineto{\pgfqpoint{7.379810in}{1.509290in}}%
\pgfpathlineto{\pgfqpoint{7.455306in}{1.513495in}}%
\pgfpathlineto{\pgfqpoint{7.527107in}{1.536944in}}%
\pgfpathlineto{\pgfqpoint{7.598690in}{1.571863in}}%
\pgfpathlineto{\pgfqpoint{7.671131in}{1.558132in}}%
\pgfpathlineto{\pgfqpoint{7.741421in}{1.595614in}}%
\pgfpathlineto{\pgfqpoint{7.812286in}{1.502950in}}%
\pgfpathlineto{\pgfqpoint{7.887441in}{1.543653in}}%
\pgfpathlineto{\pgfqpoint{7.957986in}{1.588459in}}%
\pgfpathlineto{\pgfqpoint{8.028258in}{1.528063in}}%
\pgfpathlineto{\pgfqpoint{8.100457in}{1.567280in}}%
\pgfpathlineto{\pgfqpoint{8.168934in}{1.584183in}}%
\pgfpathlineto{\pgfqpoint{8.237831in}{1.525261in}}%
\pgfpathlineto{\pgfqpoint{8.308718in}{1.612249in}}%
\pgfpathlineto{\pgfqpoint{8.376454in}{1.528210in}}%
\pgfpathlineto{\pgfqpoint{8.444285in}{1.618288in}}%
\pgfpathlineto{\pgfqpoint{8.515085in}{1.561282in}}%
\pgfpathlineto{\pgfqpoint{8.583601in}{1.564719in}}%
\pgfpathlineto{\pgfqpoint{8.651757in}{1.528788in}}%
\pgfpathlineto{\pgfqpoint{8.720615in}{1.653656in}}%
\pgfpathlineto{\pgfqpoint{8.787638in}{1.628773in}}%
\pgfpathlineto{\pgfqpoint{8.856098in}{1.573738in}}%
\pgfpathlineto{\pgfqpoint{8.924860in}{1.650890in}}%
\pgfpathlineto{\pgfqpoint{8.991646in}{1.616904in}}%
\pgfpathlineto{\pgfqpoint{9.059607in}{1.530704in}}%
\pgfpathlineto{\pgfqpoint{9.130518in}{1.567871in}}%
\pgfpathlineto{\pgfqpoint{9.198477in}{1.594816in}}%
\pgfpathlineto{\pgfqpoint{9.265262in}{1.608840in}}%
\pgfpathlineto{\pgfqpoint{9.334728in}{1.556183in}}%
\pgfpathlineto{\pgfqpoint{9.402138in}{1.616851in}}%
\pgfpathlineto{\pgfqpoint{9.467314in}{1.669498in}}%
\pgfpathlineto{\pgfqpoint{9.536145in}{1.584048in}}%
\pgfpathlineto{\pgfqpoint{9.603748in}{1.530400in}}%
\pgfpathlineto{\pgfqpoint{9.670768in}{1.605175in}}%
\pgfpathlineto{\pgfqpoint{9.740273in}{1.514664in}}%
\pgfpathlineto{\pgfqpoint{9.808815in}{1.632972in}}%
\pgfpathlineto{\pgfqpoint{9.876148in}{1.600879in}}%
\pgfpathlineto{\pgfqpoint{9.944615in}{1.607414in}}%
\pgfpathlineto{\pgfqpoint{10.012687in}{1.571830in}}%
\pgfpathlineto{\pgfqpoint{10.081602in}{1.579021in}}%
\pgfpathlineto{\pgfqpoint{10.153891in}{1.577726in}}%
\pgfpathlineto{\pgfqpoint{10.222102in}{1.577171in}}%
\pgfpathlineto{\pgfqpoint{10.290696in}{1.594108in}}%
\pgfpathlineto{\pgfqpoint{10.360853in}{1.609879in}}%
\pgfpathlineto{\pgfqpoint{10.428999in}{1.601026in}}%
\pgfpathlineto{\pgfqpoint{10.498528in}{1.575255in}}%
\pgfpathlineto{\pgfqpoint{10.569898in}{1.560845in}}%
\pgfpathlineto{\pgfqpoint{10.639006in}{1.565727in}}%
\pgfpathlineto{\pgfqpoint{10.708017in}{1.559245in}}%
\pgfpathlineto{\pgfqpoint{10.777925in}{1.672266in}}%
\pgfpathlineto{\pgfqpoint{10.846851in}{1.584449in}}%
\pgfpathlineto{\pgfqpoint{10.916008in}{1.558010in}}%
\pgfpathlineto{\pgfqpoint{10.986941in}{1.616858in}}%
\pgfpathlineto{\pgfqpoint{11.055389in}{1.597329in}}%
\pgfpathlineto{\pgfqpoint{11.125133in}{1.562163in}}%
\pgfpathlineto{\pgfqpoint{11.194984in}{1.563569in}}%
\pgfpathlineto{\pgfqpoint{11.261777in}{1.617115in}}%
\pgfpathlineto{\pgfqpoint{11.328677in}{1.640298in}}%
\pgfpathlineto{\pgfqpoint{11.398136in}{1.568638in}}%
\pgfpathlineto{\pgfqpoint{11.465294in}{1.651022in}}%
\pgfpathlineto{\pgfqpoint{11.532877in}{1.538342in}}%
\pgfpathlineto{\pgfqpoint{11.602395in}{1.588063in}}%
\pgfpathlineto{\pgfqpoint{11.669969in}{1.591651in}}%
\pgfpathlineto{\pgfqpoint{11.736702in}{1.591494in}}%
\pgfpathlineto{\pgfqpoint{11.804695in}{1.609101in}}%
\pgfpathlineto{\pgfqpoint{11.871239in}{1.623511in}}%
\pgfpathlineto{\pgfqpoint{11.938397in}{1.530226in}}%
\pgfpathlineto{\pgfqpoint{12.008843in}{1.542631in}}%
\pgfpathlineto{\pgfqpoint{12.076638in}{1.576044in}}%
\pgfpathlineto{\pgfqpoint{12.144815in}{1.548198in}}%
\pgfpathlineto{\pgfqpoint{12.213872in}{1.548130in}}%
\pgfpathlineto{\pgfqpoint{12.280761in}{1.595843in}}%
\pgfpathlineto{\pgfqpoint{12.347008in}{1.582946in}}%
\pgfpathlineto{\pgfqpoint{12.416448in}{1.590858in}}%
\pgfpathlineto{\pgfqpoint{12.484090in}{1.553558in}}%
\pgfpathlineto{\pgfqpoint{12.552597in}{1.583495in}}%
\pgfpathlineto{\pgfqpoint{12.623527in}{1.546145in}}%
\pgfpathlineto{\pgfqpoint{12.692271in}{1.569592in}}%
\pgfpathlineto{\pgfqpoint{12.759967in}{1.597326in}}%
\pgfpathlineto{\pgfqpoint{12.830499in}{1.556532in}}%
\pgfpathlineto{\pgfqpoint{12.900232in}{1.588449in}}%
\pgfpathlineto{\pgfqpoint{12.969036in}{1.532065in}}%
\pgfpathlineto{\pgfqpoint{13.041078in}{1.522245in}}%
\pgfpathlineto{\pgfqpoint{13.109710in}{1.539729in}}%
\pgfpathlineto{\pgfqpoint{13.178279in}{1.605883in}}%
\pgfpathlineto{\pgfqpoint{13.250526in}{1.514976in}}%
\pgfpathlineto{\pgfqpoint{13.319414in}{1.553188in}}%
\pgfpathlineto{\pgfqpoint{13.387091in}{1.577859in}}%
\pgfpathlineto{\pgfqpoint{13.457833in}{1.561703in}}%
\pgfpathlineto{\pgfqpoint{13.526360in}{1.562836in}}%
\pgfpathlineto{\pgfqpoint{13.595052in}{1.554583in}}%
\pgfpathlineto{\pgfqpoint{13.666880in}{1.566668in}}%
\pgfpathlineto{\pgfqpoint{13.735499in}{1.590527in}}%
\pgfpathlineto{\pgfqpoint{13.804228in}{1.555707in}}%
\pgfpathlineto{\pgfqpoint{13.876785in}{1.533353in}}%
\pgfpathlineto{\pgfqpoint{13.944783in}{1.625153in}}%
\pgfpathlineto{\pgfqpoint{14.012639in}{1.516863in}}%
\pgfpathlineto{\pgfqpoint{14.081564in}{1.596280in}}%
\pgfpathlineto{\pgfqpoint{14.149562in}{1.603040in}}%
\pgfpathlineto{\pgfqpoint{14.216102in}{1.565625in}}%
\pgfpathlineto{\pgfqpoint{14.286685in}{1.518898in}}%
\pgfpathlineto{\pgfqpoint{14.355979in}{1.562231in}}%
\pgfpathlineto{\pgfqpoint{14.424138in}{1.612428in}}%
\pgfpathlineto{\pgfqpoint{14.494110in}{1.591320in}}%
\pgfpathlineto{\pgfqpoint{14.562284in}{1.585704in}}%
\pgfpathlineto{\pgfqpoint{14.630523in}{1.647021in}}%
\pgfpathlineto{\pgfqpoint{14.700003in}{1.560771in}}%
\pgfpathlineto{\pgfqpoint{14.768225in}{1.564815in}}%
\pgfpathlineto{\pgfqpoint{14.836607in}{1.555255in}}%
\pgfpathlineto{\pgfqpoint{14.908199in}{1.487275in}}%
\pgfpathlineto{\pgfqpoint{14.976372in}{1.574084in}}%
\pgfpathlineto{\pgfqpoint{15.044439in}{1.563018in}}%
\pgfpathlineto{\pgfqpoint{15.115140in}{1.491634in}}%
\pgfpathlineto{\pgfqpoint{15.182512in}{1.625603in}}%
\pgfpathlineto{\pgfqpoint{15.249597in}{1.556145in}}%
\pgfpathlineto{\pgfqpoint{15.320168in}{1.594538in}}%
\pgfpathlineto{\pgfqpoint{15.388582in}{1.568504in}}%
\pgfpathlineto{\pgfqpoint{15.457342in}{1.518712in}}%
\pgfpathlineto{\pgfqpoint{15.527961in}{1.591577in}}%
\pgfpathlineto{\pgfqpoint{15.595878in}{1.576527in}}%
\pgfpathlineto{\pgfqpoint{15.665480in}{1.485292in}}%
\pgfpathlineto{\pgfqpoint{15.737902in}{1.546544in}}%
\pgfpathlineto{\pgfqpoint{15.807467in}{1.558776in}}%
\pgfpathlineto{\pgfqpoint{15.878678in}{1.537539in}}%
\pgfpathlineto{\pgfqpoint{15.949677in}{1.570939in}}%
\pgfpathlineto{\pgfqpoint{16.019598in}{1.533442in}}%
\pgfpathlineto{\pgfqpoint{16.088678in}{1.562144in}}%
\pgfpathlineto{\pgfqpoint{16.160188in}{1.615478in}}%
\pgfpathlineto{\pgfqpoint{16.229236in}{1.588895in}}%
\pgfpathlineto{\pgfqpoint{16.299236in}{1.547781in}}%
\pgfpathlineto{\pgfqpoint{16.373054in}{1.518566in}}%
\pgfpathlineto{\pgfqpoint{16.445981in}{1.479184in}}%
\pgfpathlineto{\pgfqpoint{16.519393in}{1.512666in}}%
\pgfpathlineto{\pgfqpoint{16.592371in}{1.568175in}}%
\pgfpathlineto{\pgfqpoint{16.662798in}{1.576556in}}%
\pgfpathlineto{\pgfqpoint{16.731829in}{1.545614in}}%
\pgfpathlineto{\pgfqpoint{16.803402in}{1.535742in}}%
\pgfpathlineto{\pgfqpoint{16.872974in}{1.567216in}}%
\pgfpathlineto{\pgfqpoint{16.942299in}{1.530455in}}%
\pgfpathlineto{\pgfqpoint{17.013249in}{1.579836in}}%
\pgfpathlineto{\pgfqpoint{17.081113in}{1.607636in}}%
\pgfpathlineto{\pgfqpoint{17.150280in}{1.549893in}}%
\pgfpathlineto{\pgfqpoint{17.220960in}{1.520709in}}%
\pgfpathlineto{\pgfqpoint{17.288800in}{1.576164in}}%
\pgfpathlineto{\pgfqpoint{17.356523in}{1.556708in}}%
\pgfpathlineto{\pgfqpoint{17.426570in}{1.614741in}}%
\pgfpathlineto{\pgfqpoint{17.494859in}{1.566279in}}%
\pgfpathlineto{\pgfqpoint{17.563254in}{1.551407in}}%
\pgfpathlineto{\pgfqpoint{17.634782in}{1.532540in}}%
\pgfpathlineto{\pgfqpoint{17.703480in}{1.569518in}}%
\pgfpathlineto{\pgfqpoint{17.772898in}{1.531912in}}%
\pgfpathlineto{\pgfqpoint{17.844148in}{1.568296in}}%
\pgfpathlineto{\pgfqpoint{17.912627in}{1.606337in}}%
\pgfpathlineto{\pgfqpoint{17.983524in}{1.490930in}}%
\pgfpathlineto{\pgfqpoint{18.056086in}{1.567174in}}%
\pgfpathlineto{\pgfqpoint{18.125960in}{1.550138in}}%
\pgfpathlineto{\pgfqpoint{18.195268in}{1.555759in}}%
\pgfpathlineto{\pgfqpoint{18.266780in}{1.551346in}}%
\pgfpathlineto{\pgfqpoint{18.336223in}{1.592736in}}%
\pgfpathlineto{\pgfqpoint{18.406899in}{1.576170in}}%
\pgfpathlineto{\pgfqpoint{18.479143in}{1.530036in}}%
\pgfpathlineto{\pgfqpoint{18.549509in}{1.554154in}}%
\pgfpathlineto{\pgfqpoint{18.619849in}{1.522647in}}%
\pgfpathlineto{\pgfqpoint{18.692839in}{1.573012in}}%
\pgfpathlineto{\pgfqpoint{18.763558in}{1.548196in}}%
\pgfpathlineto{\pgfqpoint{18.834141in}{1.497654in}}%
\pgfpathlineto{\pgfqpoint{18.906958in}{1.556952in}}%
\pgfpathlineto{\pgfqpoint{18.977585in}{1.536842in}}%
\pgfpathlineto{\pgfqpoint{19.048835in}{1.558845in}}%
\pgfpathlineto{\pgfqpoint{19.123240in}{1.474807in}}%
\pgfpathlineto{\pgfqpoint{19.195134in}{1.542659in}}%
\pgfpathlineto{\pgfqpoint{19.266961in}{1.484903in}}%
\pgfpathlineto{\pgfqpoint{19.340700in}{1.530595in}}%
\pgfpathlineto{\pgfqpoint{19.412352in}{1.526749in}}%
\pgfpathlineto{\pgfqpoint{19.482791in}{1.558572in}}%
\pgfpathlineto{\pgfqpoint{19.553641in}{1.547421in}}%
\pgfpathlineto{\pgfqpoint{19.623855in}{1.582011in}}%
\pgfpathlineto{\pgfqpoint{19.693765in}{1.551767in}}%
\pgfpathlineto{\pgfqpoint{19.766060in}{1.590515in}}%
\pgfpathlineto{\pgfqpoint{19.835928in}{1.578664in}}%
\pgfpathlineto{\pgfqpoint{19.907049in}{1.535340in}}%
\pgfpathlineto{\pgfqpoint{19.981199in}{1.576325in}}%
\pgfpathlineto{\pgfqpoint{20.052071in}{1.524114in}}%
\pgfpathlineto{\pgfqpoint{20.121885in}{1.569264in}}%
\pgfpathlineto{\pgfqpoint{20.193561in}{1.528517in}}%
\pgfpathlineto{\pgfqpoint{20.263963in}{1.553784in}}%
\pgfpathlineto{\pgfqpoint{20.334605in}{1.512026in}}%
\pgfpathlineto{\pgfqpoint{20.407276in}{1.589136in}}%
\pgfpathlineto{\pgfqpoint{20.476961in}{1.568820in}}%
\pgfpathlineto{\pgfqpoint{20.547172in}{1.479274in}}%
\pgfpathlineto{\pgfqpoint{20.618428in}{1.575582in}}%
\pgfpathlineto{\pgfqpoint{20.688980in}{1.523452in}}%
\pgfpathlineto{\pgfqpoint{20.758814in}{1.518855in}}%
\pgfpathlineto{\pgfqpoint{20.830586in}{1.600105in}}%
\pgfpathlineto{\pgfqpoint{20.899587in}{1.557287in}}%
\pgfpathlineto{\pgfqpoint{20.969373in}{1.475010in}}%
\pgfpathlineto{\pgfqpoint{21.040864in}{1.557813in}}%
\pgfpathlineto{\pgfqpoint{21.110656in}{1.507756in}}%
\pgfpathlineto{\pgfqpoint{21.181233in}{1.498560in}}%
\pgfpathlineto{\pgfqpoint{21.254645in}{1.536177in}}%
\pgfpathlineto{\pgfqpoint{21.324498in}{1.540504in}}%
\pgfpathlineto{\pgfqpoint{21.394839in}{1.545125in}}%
\pgfpathlineto{\pgfqpoint{21.467741in}{1.536204in}}%
\pgfpathlineto{\pgfqpoint{21.539601in}{1.476574in}}%
\pgfpathlineto{\pgfqpoint{21.610878in}{1.535035in}}%
\pgfpathlineto{\pgfqpoint{21.683994in}{1.562266in}}%
\pgfpathlineto{\pgfqpoint{21.756227in}{1.500466in}}%
\pgfpathlineto{\pgfqpoint{21.828555in}{1.545626in}}%
\pgfpathlineto{\pgfqpoint{21.903868in}{1.524106in}}%
\pgfpathlineto{\pgfqpoint{21.976836in}{1.478797in}}%
\pgfpathlineto{\pgfqpoint{22.048040in}{1.514715in}}%
\pgfpathlineto{\pgfqpoint{22.122462in}{1.531987in}}%
\pgfpathlineto{\pgfqpoint{22.195707in}{1.470709in}}%
\pgfpathlineto{\pgfqpoint{22.268824in}{1.501061in}}%
\pgfpathlineto{\pgfqpoint{22.343331in}{1.557458in}}%
\pgfpathlineto{\pgfqpoint{22.413449in}{1.500385in}}%
\pgfpathlineto{\pgfqpoint{22.482516in}{1.558749in}}%
\pgfpathlineto{\pgfqpoint{22.553535in}{1.587263in}}%
\pgfpathlineto{\pgfqpoint{22.624114in}{1.585803in}}%
\pgfpathlineto{\pgfqpoint{22.694016in}{1.591324in}}%
\pgfpathlineto{\pgfqpoint{22.764651in}{1.627833in}}%
\pgfpathlineto{\pgfqpoint{22.833628in}{1.614311in}}%
\pgfpathlineto{\pgfqpoint{22.902896in}{1.599819in}}%
\pgfpathlineto{\pgfqpoint{22.973868in}{1.560943in}}%
\pgfpathlineto{\pgfqpoint{23.043397in}{1.575295in}}%
\pgfpathlineto{\pgfqpoint{23.113267in}{1.521624in}}%
\pgfpathlineto{\pgfqpoint{23.184270in}{1.562385in}}%
\pgfpathlineto{\pgfqpoint{23.253484in}{1.547590in}}%
\pgfpathlineto{\pgfqpoint{23.323995in}{1.506883in}}%
\pgfpathlineto{\pgfqpoint{23.396126in}{1.546275in}}%
\pgfpathlineto{\pgfqpoint{23.467323in}{1.504832in}}%
\pgfpathlineto{\pgfqpoint{23.537850in}{1.584190in}}%
\pgfpathlineto{\pgfqpoint{23.610036in}{1.604120in}}%
\pgfpathlineto{\pgfqpoint{23.681448in}{1.532899in}}%
\pgfpathlineto{\pgfqpoint{23.752361in}{1.598700in}}%
\pgfpathlineto{\pgfqpoint{23.824951in}{1.565118in}}%
\pgfpathlineto{\pgfqpoint{23.895213in}{1.559116in}}%
\pgfpathlineto{\pgfqpoint{23.966722in}{1.482405in}}%
\pgfpathlineto{\pgfqpoint{24.039255in}{1.583860in}}%
\pgfpathlineto{\pgfqpoint{24.111784in}{1.480083in}}%
\pgfpathlineto{\pgfqpoint{24.183899in}{1.573078in}}%
\pgfpathlineto{\pgfqpoint{24.257307in}{1.551779in}}%
\pgfpathlineto{\pgfqpoint{24.329090in}{1.515672in}}%
\pgfpathlineto{\pgfqpoint{24.400328in}{1.524874in}}%
\pgfpathlineto{\pgfqpoint{24.476339in}{1.496645in}}%
\pgfpathlineto{\pgfqpoint{24.548639in}{1.522700in}}%
\pgfpathlineto{\pgfqpoint{24.618678in}{1.593510in}}%
\pgfpathlineto{\pgfqpoint{24.691660in}{1.550760in}}%
\pgfpathlineto{\pgfqpoint{24.764742in}{1.482019in}}%
\pgfpathlineto{\pgfqpoint{24.836992in}{1.503712in}}%
\pgfpathlineto{\pgfqpoint{24.911741in}{1.502131in}}%
\pgfpathlineto{\pgfqpoint{24.983525in}{1.542142in}}%
\pgfpathlineto{\pgfqpoint{25.055567in}{1.501851in}}%
\pgfpathlineto{\pgfqpoint{25.131109in}{1.493167in}}%
\pgfpathlineto{\pgfqpoint{25.203216in}{1.548758in}}%
\pgfpathlineto{\pgfqpoint{25.273349in}{1.578390in}}%
\pgfpathlineto{\pgfqpoint{25.347124in}{1.429075in}}%
\pgfpathlineto{\pgfqpoint{25.417047in}{1.598818in}}%
\pgfpathlineto{\pgfqpoint{25.487573in}{1.559183in}}%
\pgfpathlineto{\pgfqpoint{25.560110in}{1.507083in}}%
\pgfpathlineto{\pgfqpoint{25.631022in}{1.543851in}}%
\pgfpathlineto{\pgfqpoint{25.702341in}{1.515995in}}%
\pgfpathlineto{\pgfqpoint{25.775695in}{1.540403in}}%
\pgfpathlineto{\pgfqpoint{25.845667in}{1.553032in}}%
\pgfpathlineto{\pgfqpoint{25.916551in}{1.552118in}}%
\pgfpathlineto{\pgfqpoint{25.988588in}{1.637698in}}%
\pgfpathlineto{\pgfqpoint{26.058621in}{1.532726in}}%
\pgfpathlineto{\pgfqpoint{26.130346in}{1.485416in}}%
\pgfpathlineto{\pgfqpoint{26.203572in}{1.531879in}}%
\pgfpathlineto{\pgfqpoint{26.274267in}{1.546213in}}%
\pgfpathlineto{\pgfqpoint{26.344920in}{1.577262in}}%
\pgfpathlineto{\pgfqpoint{26.417231in}{1.554414in}}%
\pgfpathlineto{\pgfqpoint{26.487420in}{1.579015in}}%
\pgfpathlineto{\pgfqpoint{26.557235in}{1.577246in}}%
\pgfpathlineto{\pgfqpoint{26.629572in}{1.505684in}}%
\pgfpathlineto{\pgfqpoint{26.699584in}{1.536318in}}%
\pgfpathlineto{\pgfqpoint{26.769271in}{1.587070in}}%
\pgfpathlineto{\pgfqpoint{26.841234in}{1.531143in}}%
\pgfpathlineto{\pgfqpoint{26.912667in}{1.508803in}}%
\pgfpathlineto{\pgfqpoint{26.983641in}{1.507448in}}%
\pgfpathlineto{\pgfqpoint{27.056835in}{1.590285in}}%
\pgfpathlineto{\pgfqpoint{27.128948in}{1.517289in}}%
\pgfpathlineto{\pgfqpoint{27.201477in}{1.536899in}}%
\pgfpathlineto{\pgfqpoint{27.277488in}{1.495423in}}%
\pgfpathlineto{\pgfqpoint{27.350990in}{1.474921in}}%
\pgfpathlineto{\pgfqpoint{27.423884in}{1.532911in}}%
\pgfpathlineto{\pgfqpoint{27.500063in}{1.481939in}}%
\pgfpathlineto{\pgfqpoint{27.574929in}{1.520586in}}%
\pgfpathlineto{\pgfqpoint{27.649072in}{1.464448in}}%
\pgfpathlineto{\pgfqpoint{27.724006in}{1.492647in}}%
\pgfpathlineto{\pgfqpoint{27.795343in}{1.534409in}}%
\pgfpathlineto{\pgfqpoint{27.868117in}{1.459063in}}%
\pgfpathlineto{\pgfqpoint{27.943911in}{1.468393in}}%
\pgfpathlineto{\pgfqpoint{28.018234in}{1.509844in}}%
\pgfpathlineto{\pgfqpoint{28.090360in}{1.569511in}}%
\pgfpathlineto{\pgfqpoint{28.163352in}{1.544437in}}%
\pgfpathlineto{\pgfqpoint{28.234559in}{1.521556in}}%
\pgfpathlineto{\pgfqpoint{28.306604in}{1.510336in}}%
\pgfpathlineto{\pgfqpoint{28.380501in}{1.548452in}}%
\pgfpathlineto{\pgfqpoint{28.451943in}{1.448544in}}%
\pgfpathlineto{\pgfqpoint{28.522534in}{1.549251in}}%
\pgfpathlineto{\pgfqpoint{28.596041in}{1.503924in}}%
\pgfpathlineto{\pgfqpoint{28.668204in}{1.504888in}}%
\pgfpathlineto{\pgfqpoint{28.738605in}{1.560384in}}%
\pgfpathlineto{\pgfqpoint{28.811911in}{1.517836in}}%
\pgfpathlineto{\pgfqpoint{28.885217in}{1.469280in}}%
\pgfpathlineto{\pgfqpoint{28.956832in}{1.517961in}}%
\pgfpathlineto{\pgfqpoint{29.029887in}{1.583015in}}%
\pgfpathlineto{\pgfqpoint{29.100748in}{1.552146in}}%
\pgfpathlineto{\pgfqpoint{29.173201in}{1.524477in}}%
\pgfpathlineto{\pgfqpoint{29.248973in}{1.527257in}}%
\pgfpathlineto{\pgfqpoint{29.320759in}{1.484943in}}%
\pgfpathlineto{\pgfqpoint{29.393660in}{1.540183in}}%
\pgfpathlineto{\pgfqpoint{29.467898in}{1.524046in}}%
\pgfpathlineto{\pgfqpoint{29.540420in}{1.569026in}}%
\pgfpathlineto{\pgfqpoint{29.611700in}{1.571508in}}%
\pgfpathlineto{\pgfqpoint{29.684427in}{1.527152in}}%
\pgfpathlineto{\pgfqpoint{29.755113in}{1.559196in}}%
\pgfpathlineto{\pgfqpoint{29.827132in}{1.579406in}}%
\pgfpathlineto{\pgfqpoint{29.901656in}{1.496104in}}%
\pgfpathlineto{\pgfqpoint{29.974646in}{1.539047in}}%
\pgfpathlineto{\pgfqpoint{30.048252in}{1.500767in}}%
\pgfpathlineto{\pgfqpoint{30.122796in}{1.517802in}}%
\pgfpathlineto{\pgfqpoint{30.195443in}{1.496379in}}%
\pgfpathlineto{\pgfqpoint{30.269036in}{1.551759in}}%
\pgfpathlineto{\pgfqpoint{30.344328in}{1.529684in}}%
\pgfpathlineto{\pgfqpoint{30.417098in}{1.585140in}}%
\pgfpathlineto{\pgfqpoint{30.488991in}{1.584407in}}%
\pgfpathlineto{\pgfqpoint{30.562714in}{1.547390in}}%
\pgfpathlineto{\pgfqpoint{30.634099in}{1.535898in}}%
\pgfpathlineto{\pgfqpoint{30.707828in}{1.510561in}}%
\pgfpathlineto{\pgfqpoint{30.782249in}{1.546146in}}%
\pgfpathlineto{\pgfqpoint{30.854115in}{1.531777in}}%
\pgfpathlineto{\pgfqpoint{30.928305in}{1.493309in}}%
\pgfpathlineto{\pgfqpoint{31.002514in}{1.529259in}}%
\pgfpathlineto{\pgfqpoint{31.074452in}{1.548263in}}%
\pgfpathlineto{\pgfqpoint{31.147740in}{1.596386in}}%
\pgfpathlineto{\pgfqpoint{31.222913in}{1.554710in}}%
\pgfpathlineto{\pgfqpoint{31.294777in}{1.518785in}}%
\pgfpathlineto{\pgfqpoint{31.366613in}{1.549389in}}%
\pgfpathlineto{\pgfqpoint{31.439415in}{1.521427in}}%
\pgfpathlineto{\pgfqpoint{31.510140in}{1.546845in}}%
\pgfpathlineto{\pgfqpoint{31.582282in}{1.531156in}}%
\pgfpathlineto{\pgfqpoint{31.656180in}{1.507670in}}%
\pgfpathlineto{\pgfqpoint{31.728521in}{1.478907in}}%
\pgfpathlineto{\pgfqpoint{31.800877in}{1.526914in}}%
\pgfpathlineto{\pgfqpoint{31.873539in}{1.524414in}}%
\pgfpathlineto{\pgfqpoint{31.943734in}{1.539139in}}%
\pgfpathlineto{\pgfqpoint{32.015122in}{1.564938in}}%
\pgfpathlineto{\pgfqpoint{32.089684in}{1.475635in}}%
\pgfpathlineto{\pgfqpoint{32.161504in}{1.584371in}}%
\pgfpathlineto{\pgfqpoint{32.231773in}{1.534031in}}%
\pgfpathlineto{\pgfqpoint{32.305440in}{1.510067in}}%
\pgfpathlineto{\pgfqpoint{32.377016in}{1.548278in}}%
\pgfpathlineto{\pgfqpoint{32.447439in}{1.551309in}}%
\pgfpathlineto{\pgfqpoint{32.520401in}{1.609372in}}%
\pgfpathlineto{\pgfqpoint{32.590674in}{1.580135in}}%
\pgfpathlineto{\pgfqpoint{32.663709in}{1.514249in}}%
\pgfpathlineto{\pgfqpoint{32.740263in}{1.520863in}}%
\pgfpathlineto{\pgfqpoint{32.813546in}{1.550963in}}%
\pgfpathlineto{\pgfqpoint{32.887492in}{1.565013in}}%
\pgfpathlineto{\pgfqpoint{32.963168in}{1.535735in}}%
\pgfpathlineto{\pgfqpoint{33.037794in}{1.469339in}}%
\pgfpathlineto{\pgfqpoint{33.110479in}{1.573511in}}%
\pgfpathlineto{\pgfqpoint{33.185787in}{1.504590in}}%
\pgfpathlineto{\pgfqpoint{33.259507in}{1.540608in}}%
\pgfpathlineto{\pgfqpoint{33.333311in}{1.535470in}}%
\pgfpathlineto{\pgfqpoint{33.409286in}{1.496544in}}%
\pgfpathlineto{\pgfqpoint{33.483328in}{1.495765in}}%
\pgfpathlineto{\pgfqpoint{33.557012in}{1.493064in}}%
\pgfpathlineto{\pgfqpoint{33.631884in}{1.540189in}}%
\pgfpathlineto{\pgfqpoint{33.703848in}{1.541899in}}%
\pgfpathlineto{\pgfqpoint{33.776888in}{1.547229in}}%
\pgfpathlineto{\pgfqpoint{33.852393in}{1.523428in}}%
\pgfpathlineto{\pgfqpoint{33.923536in}{1.510628in}}%
\pgfpathlineto{\pgfqpoint{33.994648in}{1.506848in}}%
\pgfpathlineto{\pgfqpoint{34.067999in}{1.597485in}}%
\pgfpathlineto{\pgfqpoint{34.138346in}{1.579842in}}%
\pgfpathlineto{\pgfqpoint{34.210760in}{1.544412in}}%
\pgfpathlineto{\pgfqpoint{34.284339in}{1.501294in}}%
\pgfpathlineto{\pgfqpoint{34.354648in}{1.509098in}}%
\pgfpathlineto{\pgfqpoint{34.425604in}{1.536135in}}%
\pgfpathlineto{\pgfqpoint{34.499162in}{1.532760in}}%
\pgfpathlineto{\pgfqpoint{34.571449in}{1.509126in}}%
\pgfpathlineto{\pgfqpoint{34.643977in}{1.518304in}}%
\pgfpathlineto{\pgfqpoint{34.718731in}{1.510439in}}%
\pgfpathlineto{\pgfqpoint{34.789698in}{1.566490in}}%
\pgfpathlineto{\pgfqpoint{34.862212in}{1.539915in}}%
\pgfpathlineto{\pgfqpoint{34.936943in}{1.472512in}}%
\pgfpathlineto{\pgfqpoint{35.007838in}{1.565434in}}%
\pgfpathlineto{\pgfqpoint{35.080154in}{1.504964in}}%
\pgfpathlineto{\pgfqpoint{35.155466in}{1.498229in}}%
\pgfpathlineto{\pgfqpoint{35.227201in}{1.567696in}}%
\pgfpathlineto{\pgfqpoint{35.298174in}{1.575304in}}%
\pgfpathlineto{\pgfqpoint{35.372990in}{1.484551in}}%
\pgfpathlineto{\pgfqpoint{35.451774in}{1.457530in}}%
\pgfpathlineto{\pgfqpoint{35.574549in}{1.253992in}}%
\pgfpathlineto{\pgfqpoint{35.663523in}{1.497722in}}%
\pgfpathlineto{\pgfqpoint{35.741519in}{0.926398in}}%
\pgfpathlineto{\pgfqpoint{35.805568in}{1.724813in}}%
\pgfpathlineto{\pgfqpoint{35.870813in}{5.187514in}}%
\pgfpathlineto{\pgfqpoint{35.942832in}{5.319244in}}%
\pgfpathlineto{\pgfqpoint{36.012796in}{5.506716in}}%
\pgfpathlineto{\pgfqpoint{36.085094in}{5.399336in}}%
\pgfpathlineto{\pgfqpoint{36.154695in}{5.488048in}}%
\pgfpathlineto{\pgfqpoint{36.223624in}{5.539128in}}%
\pgfpathlineto{\pgfqpoint{36.293479in}{5.638191in}}%
\pgfpathlineto{\pgfqpoint{36.360634in}{5.619869in}}%
\pgfpathlineto{\pgfqpoint{36.428206in}{5.661488in}}%
\pgfpathlineto{\pgfqpoint{36.497087in}{5.625548in}}%
\pgfpathlineto{\pgfqpoint{36.563097in}{5.748599in}}%
\pgfpathlineto{\pgfqpoint{36.628950in}{5.732424in}}%
\pgfpathlineto{\pgfqpoint{36.696952in}{5.790749in}}%
\pgfpathlineto{\pgfqpoint{36.761893in}{5.793255in}}%
\pgfpathlineto{\pgfqpoint{36.827337in}{5.772285in}}%
\pgfpathlineto{\pgfqpoint{36.893714in}{5.875626in}}%
\pgfpathlineto{\pgfqpoint{36.957470in}{5.879596in}}%
\pgfpathlineto{\pgfqpoint{37.022217in}{5.814953in}}%
\pgfpathlineto{\pgfqpoint{37.088015in}{5.930845in}}%
\pgfpathlineto{\pgfqpoint{37.151827in}{5.872359in}}%
\pgfpathlineto{\pgfqpoint{37.151827in}{5.872359in}}%
\pgfpathlineto{\pgfqpoint{37.151827in}{5.872359in}}%
\pgfpathlineto{\pgfqpoint{37.088015in}{5.930845in}}%
\pgfpathlineto{\pgfqpoint{37.022217in}{5.814953in}}%
\pgfpathlineto{\pgfqpoint{36.957470in}{5.879596in}}%
\pgfpathlineto{\pgfqpoint{36.893714in}{5.875626in}}%
\pgfpathlineto{\pgfqpoint{36.827337in}{5.772285in}}%
\pgfpathlineto{\pgfqpoint{36.761893in}{5.793255in}}%
\pgfpathlineto{\pgfqpoint{36.696952in}{5.790749in}}%
\pgfpathlineto{\pgfqpoint{36.628950in}{5.732424in}}%
\pgfpathlineto{\pgfqpoint{36.563097in}{5.748599in}}%
\pgfpathlineto{\pgfqpoint{36.497087in}{5.625548in}}%
\pgfpathlineto{\pgfqpoint{36.428206in}{5.661488in}}%
\pgfpathlineto{\pgfqpoint{36.360634in}{5.619869in}}%
\pgfpathlineto{\pgfqpoint{36.293479in}{5.638191in}}%
\pgfpathlineto{\pgfqpoint{36.223624in}{5.539128in}}%
\pgfpathlineto{\pgfqpoint{36.154695in}{5.488048in}}%
\pgfpathlineto{\pgfqpoint{36.085094in}{5.399336in}}%
\pgfpathlineto{\pgfqpoint{36.012796in}{5.506716in}}%
\pgfpathlineto{\pgfqpoint{35.942832in}{5.319244in}}%
\pgfpathlineto{\pgfqpoint{35.870813in}{5.187514in}}%
\pgfpathlineto{\pgfqpoint{35.805568in}{1.724813in}}%
\pgfpathlineto{\pgfqpoint{35.741519in}{0.926398in}}%
\pgfpathlineto{\pgfqpoint{35.663523in}{1.559351in}}%
\pgfpathlineto{\pgfqpoint{35.574549in}{1.695897in}}%
\pgfpathlineto{\pgfqpoint{35.451774in}{2.152523in}}%
\pgfpathlineto{\pgfqpoint{35.372990in}{2.229596in}}%
\pgfpathlineto{\pgfqpoint{35.298174in}{2.365524in}}%
\pgfpathlineto{\pgfqpoint{35.227201in}{2.323501in}}%
\pgfpathlineto{\pgfqpoint{35.155466in}{2.252098in}}%
\pgfpathlineto{\pgfqpoint{35.080154in}{2.265781in}}%
\pgfpathlineto{\pgfqpoint{35.007838in}{2.329202in}}%
\pgfpathlineto{\pgfqpoint{34.936943in}{2.228499in}}%
\pgfpathlineto{\pgfqpoint{34.862212in}{2.316665in}}%
\pgfpathlineto{\pgfqpoint{34.789698in}{2.375130in}}%
\pgfpathlineto{\pgfqpoint{34.718731in}{2.268673in}}%
\pgfpathlineto{\pgfqpoint{34.643977in}{2.263431in}}%
\pgfpathlineto{\pgfqpoint{34.571449in}{2.215727in}}%
\pgfpathlineto{\pgfqpoint{34.499162in}{2.315701in}}%
\pgfpathlineto{\pgfqpoint{34.425604in}{2.343943in}}%
\pgfpathlineto{\pgfqpoint{34.354648in}{2.265161in}}%
\pgfpathlineto{\pgfqpoint{34.284339in}{2.239019in}}%
\pgfpathlineto{\pgfqpoint{34.210760in}{2.335533in}}%
\pgfpathlineto{\pgfqpoint{34.138346in}{2.364491in}}%
\pgfpathlineto{\pgfqpoint{34.067999in}{2.399372in}}%
\pgfpathlineto{\pgfqpoint{33.994648in}{2.292558in}}%
\pgfpathlineto{\pgfqpoint{33.923536in}{2.304522in}}%
\pgfpathlineto{\pgfqpoint{33.852393in}{2.319907in}}%
\pgfpathlineto{\pgfqpoint{33.776888in}{2.315839in}}%
\pgfpathlineto{\pgfqpoint{33.703848in}{2.340502in}}%
\pgfpathlineto{\pgfqpoint{33.631884in}{2.326432in}}%
\pgfpathlineto{\pgfqpoint{33.557012in}{2.262254in}}%
\pgfpathlineto{\pgfqpoint{33.483328in}{2.250884in}}%
\pgfpathlineto{\pgfqpoint{33.409286in}{2.213105in}}%
\pgfpathlineto{\pgfqpoint{33.333311in}{2.263560in}}%
\pgfpathlineto{\pgfqpoint{33.259507in}{2.258974in}}%
\pgfpathlineto{\pgfqpoint{33.185787in}{2.261273in}}%
\pgfpathlineto{\pgfqpoint{33.110479in}{2.347512in}}%
\pgfpathlineto{\pgfqpoint{33.037794in}{2.220336in}}%
\pgfpathlineto{\pgfqpoint{32.963168in}{2.321530in}}%
\pgfpathlineto{\pgfqpoint{32.887492in}{2.259385in}}%
\pgfpathlineto{\pgfqpoint{32.813546in}{2.302499in}}%
\pgfpathlineto{\pgfqpoint{32.740263in}{2.198868in}}%
\pgfpathlineto{\pgfqpoint{32.663709in}{2.224849in}}%
\pgfpathlineto{\pgfqpoint{32.590674in}{2.343567in}}%
\pgfpathlineto{\pgfqpoint{32.520401in}{2.391312in}}%
\pgfpathlineto{\pgfqpoint{32.447439in}{2.373974in}}%
\pgfpathlineto{\pgfqpoint{32.377016in}{2.315904in}}%
\pgfpathlineto{\pgfqpoint{32.305440in}{2.296487in}}%
\pgfpathlineto{\pgfqpoint{32.231773in}{2.288985in}}%
\pgfpathlineto{\pgfqpoint{32.161504in}{2.355976in}}%
\pgfpathlineto{\pgfqpoint{32.089684in}{2.225068in}}%
\pgfpathlineto{\pgfqpoint{32.015122in}{2.327995in}}%
\pgfpathlineto{\pgfqpoint{31.943734in}{2.369077in}}%
\pgfpathlineto{\pgfqpoint{31.873539in}{2.331917in}}%
\pgfpathlineto{\pgfqpoint{31.800877in}{2.280462in}}%
\pgfpathlineto{\pgfqpoint{31.728521in}{2.251796in}}%
\pgfpathlineto{\pgfqpoint{31.656180in}{2.336838in}}%
\pgfpathlineto{\pgfqpoint{31.582282in}{2.268170in}}%
\pgfpathlineto{\pgfqpoint{31.510140in}{2.302230in}}%
\pgfpathlineto{\pgfqpoint{31.439415in}{2.321634in}}%
\pgfpathlineto{\pgfqpoint{31.366613in}{2.314323in}}%
\pgfpathlineto{\pgfqpoint{31.294777in}{2.280819in}}%
\pgfpathlineto{\pgfqpoint{31.222913in}{2.184436in}}%
\pgfpathlineto{\pgfqpoint{31.147740in}{2.263642in}}%
\pgfpathlineto{\pgfqpoint{31.074452in}{2.277852in}}%
\pgfpathlineto{\pgfqpoint{31.002514in}{2.291781in}}%
\pgfpathlineto{\pgfqpoint{30.928305in}{2.226503in}}%
\pgfpathlineto{\pgfqpoint{30.854115in}{2.345668in}}%
\pgfpathlineto{\pgfqpoint{30.782249in}{2.307196in}}%
\pgfpathlineto{\pgfqpoint{30.707828in}{2.262804in}}%
\pgfpathlineto{\pgfqpoint{30.634099in}{2.290693in}}%
\pgfpathlineto{\pgfqpoint{30.562714in}{2.342473in}}%
\pgfpathlineto{\pgfqpoint{30.488991in}{2.314096in}}%
\pgfpathlineto{\pgfqpoint{30.417098in}{2.297310in}}%
\pgfpathlineto{\pgfqpoint{30.344328in}{2.233525in}}%
\pgfpathlineto{\pgfqpoint{30.269036in}{2.240572in}}%
\pgfpathlineto{\pgfqpoint{30.195443in}{2.267018in}}%
\pgfpathlineto{\pgfqpoint{30.122796in}{2.302823in}}%
\pgfpathlineto{\pgfqpoint{30.048252in}{2.233145in}}%
\pgfpathlineto{\pgfqpoint{29.974646in}{2.294793in}}%
\pgfpathlineto{\pgfqpoint{29.901656in}{2.260473in}}%
\pgfpathlineto{\pgfqpoint{29.827132in}{2.325966in}}%
\pgfpathlineto{\pgfqpoint{29.755113in}{2.339174in}}%
\pgfpathlineto{\pgfqpoint{29.684427in}{2.314700in}}%
\pgfpathlineto{\pgfqpoint{29.611700in}{2.298618in}}%
\pgfpathlineto{\pgfqpoint{29.540420in}{2.308092in}}%
\pgfpathlineto{\pgfqpoint{29.467898in}{2.225859in}}%
\pgfpathlineto{\pgfqpoint{29.393660in}{2.293684in}}%
\pgfpathlineto{\pgfqpoint{29.320759in}{2.212543in}}%
\pgfpathlineto{\pgfqpoint{29.248973in}{2.266146in}}%
\pgfpathlineto{\pgfqpoint{29.173201in}{2.264222in}}%
\pgfpathlineto{\pgfqpoint{29.100748in}{2.328953in}}%
\pgfpathlineto{\pgfqpoint{29.029887in}{2.351888in}}%
\pgfpathlineto{\pgfqpoint{28.956832in}{2.301102in}}%
\pgfpathlineto{\pgfqpoint{28.885217in}{2.243646in}}%
\pgfpathlineto{\pgfqpoint{28.811911in}{2.244114in}}%
\pgfpathlineto{\pgfqpoint{28.738605in}{2.325439in}}%
\pgfpathlineto{\pgfqpoint{28.668204in}{2.281621in}}%
\pgfpathlineto{\pgfqpoint{28.596041in}{2.249022in}}%
\pgfpathlineto{\pgfqpoint{28.522534in}{2.371326in}}%
\pgfpathlineto{\pgfqpoint{28.451943in}{2.210001in}}%
\pgfpathlineto{\pgfqpoint{28.380501in}{2.251101in}}%
\pgfpathlineto{\pgfqpoint{28.306604in}{2.253161in}}%
\pgfpathlineto{\pgfqpoint{28.234559in}{2.300486in}}%
\pgfpathlineto{\pgfqpoint{28.163352in}{2.334604in}}%
\pgfpathlineto{\pgfqpoint{28.090360in}{2.334452in}}%
\pgfpathlineto{\pgfqpoint{28.018234in}{2.221174in}}%
\pgfpathlineto{\pgfqpoint{27.943911in}{2.189532in}}%
\pgfpathlineto{\pgfqpoint{27.868117in}{2.204306in}}%
\pgfpathlineto{\pgfqpoint{27.795343in}{2.299883in}}%
\pgfpathlineto{\pgfqpoint{27.724006in}{2.281051in}}%
\pgfpathlineto{\pgfqpoint{27.649072in}{2.226320in}}%
\pgfpathlineto{\pgfqpoint{27.574929in}{2.248437in}}%
\pgfpathlineto{\pgfqpoint{27.500063in}{2.202434in}}%
\pgfpathlineto{\pgfqpoint{27.423884in}{2.307017in}}%
\pgfpathlineto{\pgfqpoint{27.350990in}{2.267292in}}%
\pgfpathlineto{\pgfqpoint{27.277488in}{2.267224in}}%
\pgfpathlineto{\pgfqpoint{27.201477in}{2.293648in}}%
\pgfpathlineto{\pgfqpoint{27.128948in}{2.266617in}}%
\pgfpathlineto{\pgfqpoint{27.056835in}{2.356686in}}%
\pgfpathlineto{\pgfqpoint{26.983641in}{2.312856in}}%
\pgfpathlineto{\pgfqpoint{26.912667in}{2.325656in}}%
\pgfpathlineto{\pgfqpoint{26.841234in}{2.274754in}}%
\pgfpathlineto{\pgfqpoint{26.769271in}{2.375722in}}%
\pgfpathlineto{\pgfqpoint{26.699584in}{2.336737in}}%
\pgfpathlineto{\pgfqpoint{26.629572in}{2.283473in}}%
\pgfpathlineto{\pgfqpoint{26.557235in}{2.412854in}}%
\pgfpathlineto{\pgfqpoint{26.487420in}{2.378132in}}%
\pgfpathlineto{\pgfqpoint{26.417231in}{2.324241in}}%
\pgfpathlineto{\pgfqpoint{26.344920in}{2.323634in}}%
\pgfpathlineto{\pgfqpoint{26.274267in}{2.288707in}}%
\pgfpathlineto{\pgfqpoint{26.203572in}{2.256552in}}%
\pgfpathlineto{\pgfqpoint{26.130346in}{2.300808in}}%
\pgfpathlineto{\pgfqpoint{26.058621in}{2.295589in}}%
\pgfpathlineto{\pgfqpoint{25.988588in}{2.408263in}}%
\pgfpathlineto{\pgfqpoint{25.916551in}{2.342600in}}%
\pgfpathlineto{\pgfqpoint{25.845667in}{2.389794in}}%
\pgfpathlineto{\pgfqpoint{25.775695in}{2.302233in}}%
\pgfpathlineto{\pgfqpoint{25.702341in}{2.259177in}}%
\pgfpathlineto{\pgfqpoint{25.631022in}{2.327916in}}%
\pgfpathlineto{\pgfqpoint{25.560110in}{2.283593in}}%
\pgfpathlineto{\pgfqpoint{25.487573in}{2.303464in}}%
\pgfpathlineto{\pgfqpoint{25.417047in}{2.378551in}}%
\pgfpathlineto{\pgfqpoint{25.347124in}{2.209151in}}%
\pgfpathlineto{\pgfqpoint{25.273349in}{2.342925in}}%
\pgfpathlineto{\pgfqpoint{25.203216in}{2.321418in}}%
\pgfpathlineto{\pgfqpoint{25.131109in}{2.237827in}}%
\pgfpathlineto{\pgfqpoint{25.055567in}{2.265861in}}%
\pgfpathlineto{\pgfqpoint{24.983525in}{2.296342in}}%
\pgfpathlineto{\pgfqpoint{24.911741in}{2.253604in}}%
\pgfpathlineto{\pgfqpoint{24.836992in}{2.301860in}}%
\pgfpathlineto{\pgfqpoint{24.764742in}{2.218280in}}%
\pgfpathlineto{\pgfqpoint{24.691660in}{2.296715in}}%
\pgfpathlineto{\pgfqpoint{24.618678in}{2.393918in}}%
\pgfpathlineto{\pgfqpoint{24.548639in}{2.286684in}}%
\pgfpathlineto{\pgfqpoint{24.476339in}{2.176033in}}%
\pgfpathlineto{\pgfqpoint{24.400328in}{2.260240in}}%
\pgfpathlineto{\pgfqpoint{24.329090in}{2.306604in}}%
\pgfpathlineto{\pgfqpoint{24.257307in}{2.255320in}}%
\pgfpathlineto{\pgfqpoint{24.183899in}{2.348892in}}%
\pgfpathlineto{\pgfqpoint{24.111784in}{2.210574in}}%
\pgfpathlineto{\pgfqpoint{24.039255in}{2.386143in}}%
\pgfpathlineto{\pgfqpoint{23.966722in}{2.214818in}}%
\pgfpathlineto{\pgfqpoint{23.895213in}{2.345307in}}%
\pgfpathlineto{\pgfqpoint{23.824951in}{2.348982in}}%
\pgfpathlineto{\pgfqpoint{23.752361in}{2.328198in}}%
\pgfpathlineto{\pgfqpoint{23.681448in}{2.241376in}}%
\pgfpathlineto{\pgfqpoint{23.610036in}{2.363105in}}%
\pgfpathlineto{\pgfqpoint{23.537850in}{2.319420in}}%
\pgfpathlineto{\pgfqpoint{23.467323in}{2.253040in}}%
\pgfpathlineto{\pgfqpoint{23.396126in}{2.315437in}}%
\pgfpathlineto{\pgfqpoint{23.323995in}{2.269897in}}%
\pgfpathlineto{\pgfqpoint{23.253484in}{2.376160in}}%
\pgfpathlineto{\pgfqpoint{23.184270in}{2.386374in}}%
\pgfpathlineto{\pgfqpoint{23.113267in}{2.326259in}}%
\pgfpathlineto{\pgfqpoint{23.043397in}{2.293230in}}%
\pgfpathlineto{\pgfqpoint{22.973868in}{2.389573in}}%
\pgfpathlineto{\pgfqpoint{22.902896in}{2.397494in}}%
\pgfpathlineto{\pgfqpoint{22.833628in}{2.421459in}}%
\pgfpathlineto{\pgfqpoint{22.764651in}{2.482801in}}%
\pgfpathlineto{\pgfqpoint{22.694016in}{2.395373in}}%
\pgfpathlineto{\pgfqpoint{22.624114in}{2.348659in}}%
\pgfpathlineto{\pgfqpoint{22.553535in}{2.323391in}}%
\pgfpathlineto{\pgfqpoint{22.482516in}{2.398083in}}%
\pgfpathlineto{\pgfqpoint{22.413449in}{2.303340in}}%
\pgfpathlineto{\pgfqpoint{22.343331in}{2.348390in}}%
\pgfpathlineto{\pgfqpoint{22.268824in}{2.223836in}}%
\pgfpathlineto{\pgfqpoint{22.195707in}{2.170512in}}%
\pgfpathlineto{\pgfqpoint{22.122462in}{2.278881in}}%
\pgfpathlineto{\pgfqpoint{22.048040in}{2.330770in}}%
\pgfpathlineto{\pgfqpoint{21.976836in}{2.197014in}}%
\pgfpathlineto{\pgfqpoint{21.903868in}{2.291199in}}%
\pgfpathlineto{\pgfqpoint{21.828555in}{2.291422in}}%
\pgfpathlineto{\pgfqpoint{21.756227in}{2.259136in}}%
\pgfpathlineto{\pgfqpoint{21.683994in}{2.339697in}}%
\pgfpathlineto{\pgfqpoint{21.610878in}{2.303436in}}%
\pgfpathlineto{\pgfqpoint{21.539601in}{2.268173in}}%
\pgfpathlineto{\pgfqpoint{21.467741in}{2.288976in}}%
\pgfpathlineto{\pgfqpoint{21.394839in}{2.299263in}}%
\pgfpathlineto{\pgfqpoint{21.324498in}{2.355684in}}%
\pgfpathlineto{\pgfqpoint{21.254645in}{2.304203in}}%
\pgfpathlineto{\pgfqpoint{21.181233in}{2.235383in}}%
\pgfpathlineto{\pgfqpoint{21.110656in}{2.319899in}}%
\pgfpathlineto{\pgfqpoint{21.040864in}{2.379865in}}%
\pgfpathlineto{\pgfqpoint{20.969373in}{2.236644in}}%
\pgfpathlineto{\pgfqpoint{20.899587in}{2.361138in}}%
\pgfpathlineto{\pgfqpoint{20.830586in}{2.379466in}}%
\pgfpathlineto{\pgfqpoint{20.758814in}{2.293630in}}%
\pgfpathlineto{\pgfqpoint{20.688980in}{2.326817in}}%
\pgfpathlineto{\pgfqpoint{20.618428in}{2.387675in}}%
\pgfpathlineto{\pgfqpoint{20.547172in}{2.261538in}}%
\pgfpathlineto{\pgfqpoint{20.476961in}{2.308828in}}%
\pgfpathlineto{\pgfqpoint{20.407276in}{2.336694in}}%
\pgfpathlineto{\pgfqpoint{20.334605in}{2.289116in}}%
\pgfpathlineto{\pgfqpoint{20.263963in}{2.361737in}}%
\pgfpathlineto{\pgfqpoint{20.193561in}{2.294331in}}%
\pgfpathlineto{\pgfqpoint{20.121885in}{2.382111in}}%
\pgfpathlineto{\pgfqpoint{20.052071in}{2.291057in}}%
\pgfpathlineto{\pgfqpoint{19.981199in}{2.286178in}}%
\pgfpathlineto{\pgfqpoint{19.907049in}{2.285374in}}%
\pgfpathlineto{\pgfqpoint{19.835928in}{2.389725in}}%
\pgfpathlineto{\pgfqpoint{19.766060in}{2.363164in}}%
\pgfpathlineto{\pgfqpoint{19.693765in}{2.277444in}}%
\pgfpathlineto{\pgfqpoint{19.623855in}{2.419361in}}%
\pgfpathlineto{\pgfqpoint{19.553641in}{2.351876in}}%
\pgfpathlineto{\pgfqpoint{19.482791in}{2.360552in}}%
\pgfpathlineto{\pgfqpoint{19.412352in}{2.283026in}}%
\pgfpathlineto{\pgfqpoint{19.340700in}{2.296885in}}%
\pgfpathlineto{\pgfqpoint{19.266961in}{2.284268in}}%
\pgfpathlineto{\pgfqpoint{19.195134in}{2.289248in}}%
\pgfpathlineto{\pgfqpoint{19.123240in}{2.232984in}}%
\pgfpathlineto{\pgfqpoint{19.048835in}{2.271599in}}%
\pgfpathlineto{\pgfqpoint{18.977585in}{2.266246in}}%
\pgfpathlineto{\pgfqpoint{18.906958in}{2.330027in}}%
\pgfpathlineto{\pgfqpoint{18.834141in}{2.284510in}}%
\pgfpathlineto{\pgfqpoint{18.763558in}{2.293835in}}%
\pgfpathlineto{\pgfqpoint{18.692839in}{2.416355in}}%
\pgfpathlineto{\pgfqpoint{18.619849in}{2.333533in}}%
\pgfpathlineto{\pgfqpoint{18.549509in}{2.394626in}}%
\pgfpathlineto{\pgfqpoint{18.479143in}{2.296630in}}%
\pgfpathlineto{\pgfqpoint{18.406899in}{2.321433in}}%
\pgfpathlineto{\pgfqpoint{18.336223in}{2.426745in}}%
\pgfpathlineto{\pgfqpoint{18.266780in}{2.377914in}}%
\pgfpathlineto{\pgfqpoint{18.195268in}{2.353926in}}%
\pgfpathlineto{\pgfqpoint{18.125960in}{2.341081in}}%
\pgfpathlineto{\pgfqpoint{18.056086in}{2.293209in}}%
\pgfpathlineto{\pgfqpoint{17.983524in}{2.245730in}}%
\pgfpathlineto{\pgfqpoint{17.912627in}{2.376504in}}%
\pgfpathlineto{\pgfqpoint{17.844148in}{2.355139in}}%
\pgfpathlineto{\pgfqpoint{17.772898in}{2.315892in}}%
\pgfpathlineto{\pgfqpoint{17.703480in}{2.375946in}}%
\pgfpathlineto{\pgfqpoint{17.634782in}{2.313263in}}%
\pgfpathlineto{\pgfqpoint{17.563254in}{2.352403in}}%
\pgfpathlineto{\pgfqpoint{17.494859in}{2.295254in}}%
\pgfpathlineto{\pgfqpoint{17.426570in}{2.414408in}}%
\pgfpathlineto{\pgfqpoint{17.356523in}{2.344504in}}%
\pgfpathlineto{\pgfqpoint{17.288800in}{2.420723in}}%
\pgfpathlineto{\pgfqpoint{17.220960in}{2.310083in}}%
\pgfpathlineto{\pgfqpoint{17.150280in}{2.317372in}}%
\pgfpathlineto{\pgfqpoint{17.081113in}{2.450638in}}%
\pgfpathlineto{\pgfqpoint{17.013249in}{2.402088in}}%
\pgfpathlineto{\pgfqpoint{16.942299in}{2.319482in}}%
\pgfpathlineto{\pgfqpoint{16.872974in}{2.315337in}}%
\pgfpathlineto{\pgfqpoint{16.803402in}{2.334003in}}%
\pgfpathlineto{\pgfqpoint{16.731829in}{2.347150in}}%
\pgfpathlineto{\pgfqpoint{16.662798in}{2.362100in}}%
\pgfpathlineto{\pgfqpoint{16.592371in}{2.408326in}}%
\pgfpathlineto{\pgfqpoint{16.519393in}{2.237324in}}%
\pgfpathlineto{\pgfqpoint{16.445981in}{2.213369in}}%
\pgfpathlineto{\pgfqpoint{16.373054in}{2.296087in}}%
\pgfpathlineto{\pgfqpoint{16.299236in}{2.346079in}}%
\pgfpathlineto{\pgfqpoint{16.229236in}{2.416211in}}%
\pgfpathlineto{\pgfqpoint{16.160188in}{2.441511in}}%
\pgfpathlineto{\pgfqpoint{16.088678in}{2.378177in}}%
\pgfpathlineto{\pgfqpoint{16.019598in}{2.336319in}}%
\pgfpathlineto{\pgfqpoint{15.949677in}{2.389314in}}%
\pgfpathlineto{\pgfqpoint{15.878678in}{2.309932in}}%
\pgfpathlineto{\pgfqpoint{15.807467in}{2.316651in}}%
\pgfpathlineto{\pgfqpoint{15.737902in}{2.339944in}}%
\pgfpathlineto{\pgfqpoint{15.665480in}{2.298817in}}%
\pgfpathlineto{\pgfqpoint{15.595878in}{2.387626in}}%
\pgfpathlineto{\pgfqpoint{15.527961in}{2.337079in}}%
\pgfpathlineto{\pgfqpoint{15.457342in}{2.315581in}}%
\pgfpathlineto{\pgfqpoint{15.388582in}{2.367244in}}%
\pgfpathlineto{\pgfqpoint{15.320168in}{2.386687in}}%
\pgfpathlineto{\pgfqpoint{15.249597in}{2.366103in}}%
\pgfpathlineto{\pgfqpoint{15.182512in}{2.425014in}}%
\pgfpathlineto{\pgfqpoint{15.115140in}{2.334152in}}%
\pgfpathlineto{\pgfqpoint{15.044439in}{2.362664in}}%
\pgfpathlineto{\pgfqpoint{14.976372in}{2.414390in}}%
\pgfpathlineto{\pgfqpoint{14.908199in}{2.269316in}}%
\pgfpathlineto{\pgfqpoint{14.836607in}{2.352508in}}%
\pgfpathlineto{\pgfqpoint{14.768225in}{2.407033in}}%
\pgfpathlineto{\pgfqpoint{14.700003in}{2.378815in}}%
\pgfpathlineto{\pgfqpoint{14.630523in}{2.406563in}}%
\pgfpathlineto{\pgfqpoint{14.562284in}{2.368556in}}%
\pgfpathlineto{\pgfqpoint{14.494110in}{2.382148in}}%
\pgfpathlineto{\pgfqpoint{14.424138in}{2.394395in}}%
\pgfpathlineto{\pgfqpoint{14.355979in}{2.324319in}}%
\pgfpathlineto{\pgfqpoint{14.286685in}{2.344426in}}%
\pgfpathlineto{\pgfqpoint{14.216102in}{2.399801in}}%
\pgfpathlineto{\pgfqpoint{14.149562in}{2.396819in}}%
\pgfpathlineto{\pgfqpoint{14.081564in}{2.427453in}}%
\pgfpathlineto{\pgfqpoint{14.012639in}{2.322965in}}%
\pgfpathlineto{\pgfqpoint{13.944783in}{2.405673in}}%
\pgfpathlineto{\pgfqpoint{13.876785in}{2.318313in}}%
\pgfpathlineto{\pgfqpoint{13.804228in}{2.364455in}}%
\pgfpathlineto{\pgfqpoint{13.735499in}{2.360084in}}%
\pgfpathlineto{\pgfqpoint{13.666880in}{2.298388in}}%
\pgfpathlineto{\pgfqpoint{13.595052in}{2.404581in}}%
\pgfpathlineto{\pgfqpoint{13.526360in}{2.400693in}}%
\pgfpathlineto{\pgfqpoint{13.457833in}{2.329708in}}%
\pgfpathlineto{\pgfqpoint{13.387091in}{2.389023in}}%
\pgfpathlineto{\pgfqpoint{13.319414in}{2.338118in}}%
\pgfpathlineto{\pgfqpoint{13.250526in}{2.313702in}}%
\pgfpathlineto{\pgfqpoint{13.178279in}{2.321556in}}%
\pgfpathlineto{\pgfqpoint{13.109710in}{2.363697in}}%
\pgfpathlineto{\pgfqpoint{13.041078in}{2.275523in}}%
\pgfpathlineto{\pgfqpoint{12.969036in}{2.316575in}}%
\pgfpathlineto{\pgfqpoint{12.900232in}{2.345595in}}%
\pgfpathlineto{\pgfqpoint{12.830499in}{2.350302in}}%
\pgfpathlineto{\pgfqpoint{12.759967in}{2.460769in}}%
\pgfpathlineto{\pgfqpoint{12.692271in}{2.394316in}}%
\pgfpathlineto{\pgfqpoint{12.623527in}{2.382185in}}%
\pgfpathlineto{\pgfqpoint{12.552597in}{2.424792in}}%
\pgfpathlineto{\pgfqpoint{12.484090in}{2.329892in}}%
\pgfpathlineto{\pgfqpoint{12.416448in}{2.381820in}}%
\pgfpathlineto{\pgfqpoint{12.347008in}{2.445924in}}%
\pgfpathlineto{\pgfqpoint{12.280761in}{2.454963in}}%
\pgfpathlineto{\pgfqpoint{12.213872in}{2.340878in}}%
\pgfpathlineto{\pgfqpoint{12.144815in}{2.287899in}}%
\pgfpathlineto{\pgfqpoint{12.076638in}{2.412443in}}%
\pgfpathlineto{\pgfqpoint{12.008843in}{2.309380in}}%
\pgfpathlineto{\pgfqpoint{11.938397in}{2.312124in}}%
\pgfpathlineto{\pgfqpoint{11.871239in}{2.412976in}}%
\pgfpathlineto{\pgfqpoint{11.804695in}{2.447359in}}%
\pgfpathlineto{\pgfqpoint{11.736702in}{2.399387in}}%
\pgfpathlineto{\pgfqpoint{11.669969in}{2.418078in}}%
\pgfpathlineto{\pgfqpoint{11.602395in}{2.427688in}}%
\pgfpathlineto{\pgfqpoint{11.532877in}{2.379878in}}%
\pgfpathlineto{\pgfqpoint{11.465294in}{2.465238in}}%
\pgfpathlineto{\pgfqpoint{11.398136in}{2.391542in}}%
\pgfpathlineto{\pgfqpoint{11.328677in}{2.440319in}}%
\pgfpathlineto{\pgfqpoint{11.261777in}{2.462632in}}%
\pgfpathlineto{\pgfqpoint{11.194984in}{2.396114in}}%
\pgfpathlineto{\pgfqpoint{11.125133in}{2.352777in}}%
\pgfpathlineto{\pgfqpoint{11.055389in}{2.377512in}}%
\pgfpathlineto{\pgfqpoint{10.986941in}{2.445029in}}%
\pgfpathlineto{\pgfqpoint{10.916008in}{2.355493in}}%
\pgfpathlineto{\pgfqpoint{10.846851in}{2.362275in}}%
\pgfpathlineto{\pgfqpoint{10.777925in}{2.465290in}}%
\pgfpathlineto{\pgfqpoint{10.708017in}{2.385366in}}%
\pgfpathlineto{\pgfqpoint{10.639006in}{2.392576in}}%
\pgfpathlineto{\pgfqpoint{10.569898in}{2.317130in}}%
\pgfpathlineto{\pgfqpoint{10.498528in}{2.385212in}}%
\pgfpathlineto{\pgfqpoint{10.428999in}{2.375232in}}%
\pgfpathlineto{\pgfqpoint{10.360853in}{2.401431in}}%
\pgfpathlineto{\pgfqpoint{10.290696in}{2.378244in}}%
\pgfpathlineto{\pgfqpoint{10.222102in}{2.459106in}}%
\pgfpathlineto{\pgfqpoint{10.153891in}{2.387850in}}%
\pgfpathlineto{\pgfqpoint{10.081602in}{2.316230in}}%
\pgfpathlineto{\pgfqpoint{10.012687in}{2.404361in}}%
\pgfpathlineto{\pgfqpoint{9.944615in}{2.524428in}}%
\pgfpathlineto{\pgfqpoint{9.876148in}{2.370141in}}%
\pgfpathlineto{\pgfqpoint{9.808815in}{2.428756in}}%
\pgfpathlineto{\pgfqpoint{9.740273in}{2.326536in}}%
\pgfpathlineto{\pgfqpoint{9.670768in}{2.451919in}}%
\pgfpathlineto{\pgfqpoint{9.603748in}{2.351056in}}%
\pgfpathlineto{\pgfqpoint{9.536145in}{2.401471in}}%
\pgfpathlineto{\pgfqpoint{9.467314in}{2.503416in}}%
\pgfpathlineto{\pgfqpoint{9.402138in}{2.483410in}}%
\pgfpathlineto{\pgfqpoint{9.334728in}{2.332717in}}%
\pgfpathlineto{\pgfqpoint{9.265262in}{2.435613in}}%
\pgfpathlineto{\pgfqpoint{9.198477in}{2.374259in}}%
\pgfpathlineto{\pgfqpoint{9.130518in}{2.342457in}}%
\pgfpathlineto{\pgfqpoint{9.059607in}{2.324022in}}%
\pgfpathlineto{\pgfqpoint{8.991646in}{2.437245in}}%
\pgfpathlineto{\pgfqpoint{8.924860in}{2.492165in}}%
\pgfpathlineto{\pgfqpoint{8.856098in}{2.355079in}}%
\pgfpathlineto{\pgfqpoint{8.787638in}{2.418926in}}%
\pgfpathlineto{\pgfqpoint{8.720615in}{2.493017in}}%
\pgfpathlineto{\pgfqpoint{8.651757in}{2.343993in}}%
\pgfpathlineto{\pgfqpoint{8.583601in}{2.369451in}}%
\pgfpathlineto{\pgfqpoint{8.515085in}{2.347538in}}%
\pgfpathlineto{\pgfqpoint{8.444285in}{2.407523in}}%
\pgfpathlineto{\pgfqpoint{8.376454in}{2.376527in}}%
\pgfpathlineto{\pgfqpoint{8.308718in}{2.415966in}}%
\pgfpathlineto{\pgfqpoint{8.237831in}{2.274930in}}%
\pgfpathlineto{\pgfqpoint{8.168934in}{2.339596in}}%
\pgfpathlineto{\pgfqpoint{8.100457in}{2.350531in}}%
\pgfpathlineto{\pgfqpoint{8.028258in}{2.304644in}}%
\pgfpathlineto{\pgfqpoint{7.957986in}{2.310678in}}%
\pgfpathlineto{\pgfqpoint{7.887441in}{2.276659in}}%
\pgfpathlineto{\pgfqpoint{7.812286in}{2.263590in}}%
\pgfpathlineto{\pgfqpoint{7.741421in}{2.387356in}}%
\pgfpathlineto{\pgfqpoint{7.671131in}{2.329237in}}%
\pgfpathlineto{\pgfqpoint{7.598690in}{2.329323in}}%
\pgfpathlineto{\pgfqpoint{7.527107in}{2.320950in}}%
\pgfpathlineto{\pgfqpoint{7.455306in}{2.253215in}}%
\pgfpathlineto{\pgfqpoint{7.379810in}{2.272963in}}%
\pgfpathlineto{\pgfqpoint{7.308295in}{2.256817in}}%
\pgfpathlineto{\pgfqpoint{7.238831in}{2.444538in}}%
\pgfpathlineto{\pgfqpoint{7.171034in}{2.517744in}}%
\pgfpathlineto{\pgfqpoint{7.106301in}{2.463778in}}%
\pgfpathlineto{\pgfqpoint{7.040805in}{2.524475in}}%
\pgfpathlineto{\pgfqpoint{6.973333in}{2.473005in}}%
\pgfpathlineto{\pgfqpoint{6.906544in}{2.423239in}}%
\pgfpathlineto{\pgfqpoint{6.839348in}{2.407795in}}%
\pgfpathlineto{\pgfqpoint{6.770581in}{2.413310in}}%
\pgfpathlineto{\pgfqpoint{6.704271in}{2.381557in}}%
\pgfpathlineto{\pgfqpoint{6.636218in}{2.349013in}}%
\pgfpathlineto{\pgfqpoint{6.568004in}{2.521477in}}%
\pgfpathlineto{\pgfqpoint{6.502885in}{2.471125in}}%
\pgfpathlineto{\pgfqpoint{6.437563in}{2.483194in}}%
\pgfpathlineto{\pgfqpoint{6.369842in}{2.468254in}}%
\pgfpathlineto{\pgfqpoint{6.303479in}{2.413452in}}%
\pgfpathlineto{\pgfqpoint{6.236861in}{2.491128in}}%
\pgfpathlineto{\pgfqpoint{6.169451in}{2.396602in}}%
\pgfpathlineto{\pgfqpoint{6.103260in}{2.347344in}}%
\pgfpathlineto{\pgfqpoint{6.036866in}{2.454770in}}%
\pgfpathlineto{\pgfqpoint{5.968494in}{2.415040in}}%
\pgfpathlineto{\pgfqpoint{5.901623in}{2.422676in}}%
\pgfpathlineto{\pgfqpoint{5.834669in}{2.515321in}}%
\pgfpathlineto{\pgfqpoint{5.766709in}{2.438440in}}%
\pgfpathlineto{\pgfqpoint{5.700467in}{2.424659in}}%
\pgfpathlineto{\pgfqpoint{5.633136in}{2.340892in}}%
\pgfpathlineto{\pgfqpoint{5.561464in}{2.344049in}}%
\pgfpathlineto{\pgfqpoint{5.492693in}{2.353124in}}%
\pgfpathlineto{\pgfqpoint{5.424203in}{2.396953in}}%
\pgfpathlineto{\pgfqpoint{5.353942in}{2.320139in}}%
\pgfpathlineto{\pgfqpoint{5.283251in}{2.365402in}}%
\pgfpathlineto{\pgfqpoint{5.213575in}{2.370042in}}%
\pgfpathlineto{\pgfqpoint{5.142206in}{2.299670in}}%
\pgfpathlineto{\pgfqpoint{5.073090in}{2.395244in}}%
\pgfpathlineto{\pgfqpoint{5.005143in}{2.392115in}}%
\pgfpathlineto{\pgfqpoint{4.935251in}{2.344606in}}%
\pgfpathlineto{\pgfqpoint{4.866311in}{2.310785in}}%
\pgfpathlineto{\pgfqpoint{4.796564in}{2.338156in}}%
\pgfpathlineto{\pgfqpoint{4.726503in}{2.472786in}}%
\pgfpathlineto{\pgfqpoint{4.658575in}{2.369430in}}%
\pgfpathlineto{\pgfqpoint{4.590503in}{2.366629in}}%
\pgfpathlineto{\pgfqpoint{4.520559in}{2.389366in}}%
\pgfpathlineto{\pgfqpoint{4.454265in}{2.420274in}}%
\pgfpathlineto{\pgfqpoint{4.387175in}{2.441472in}}%
\pgfpathlineto{\pgfqpoint{4.319299in}{2.440327in}}%
\pgfpathlineto{\pgfqpoint{4.253047in}{2.432371in}}%
\pgfpathlineto{\pgfqpoint{4.186150in}{2.363470in}}%
\pgfpathlineto{\pgfqpoint{4.116566in}{2.371017in}}%
\pgfpathlineto{\pgfqpoint{4.048987in}{2.276365in}}%
\pgfpathlineto{\pgfqpoint{3.981945in}{2.406194in}}%
\pgfpathlineto{\pgfqpoint{3.914312in}{2.500852in}}%
\pgfpathlineto{\pgfqpoint{3.848285in}{2.400018in}}%
\pgfpathlineto{\pgfqpoint{3.781205in}{2.459129in}}%
\pgfpathlineto{\pgfqpoint{3.712535in}{2.399915in}}%
\pgfpathlineto{\pgfqpoint{3.643958in}{2.356364in}}%
\pgfpathlineto{\pgfqpoint{3.575543in}{2.415234in}}%
\pgfpathlineto{\pgfqpoint{3.503039in}{2.293216in}}%
\pgfpathlineto{\pgfqpoint{3.430573in}{2.273638in}}%
\pgfpathlineto{\pgfqpoint{3.356465in}{2.204051in}}%
\pgfpathlineto{\pgfqpoint{3.277599in}{2.117445in}}%
\pgfpathlineto{\pgfqpoint{3.195433in}{2.282530in}}%
\pgfpathlineto{\pgfqpoint{3.123114in}{2.295769in}}%
\pgfpathlineto{\pgfqpoint{3.047640in}{2.262722in}}%
\pgfpathlineto{\pgfqpoint{2.975015in}{2.336790in}}%
\pgfpathlineto{\pgfqpoint{2.902674in}{2.185241in}}%
\pgfpathlineto{\pgfqpoint{2.826501in}{2.273706in}}%
\pgfpathlineto{\pgfqpoint{2.753491in}{2.296750in}}%
\pgfpathlineto{\pgfqpoint{2.681331in}{2.252813in}}%
\pgfpathlineto{\pgfqpoint{2.606023in}{2.286411in}}%
\pgfpathlineto{\pgfqpoint{2.534117in}{2.313522in}}%
\pgfpathlineto{\pgfqpoint{2.462769in}{2.298530in}}%
\pgfpathlineto{\pgfqpoint{2.387948in}{2.273118in}}%
\pgfpathlineto{\pgfqpoint{2.317152in}{2.357001in}}%
\pgfpathlineto{\pgfqpoint{2.248172in}{2.352023in}}%
\pgfpathlineto{\pgfqpoint{2.177337in}{2.301583in}}%
\pgfpathlineto{\pgfqpoint{2.108883in}{2.404848in}}%
\pgfpathlineto{\pgfqpoint{2.041179in}{2.398977in}}%
\pgfpathlineto{\pgfqpoint{1.970951in}{2.404193in}}%
\pgfpathlineto{\pgfqpoint{1.904458in}{2.394984in}}%
\pgfpathlineto{\pgfqpoint{1.835890in}{2.409824in}}%
\pgfpathlineto{\pgfqpoint{1.766402in}{2.393356in}}%
\pgfpathlineto{\pgfqpoint{1.698662in}{2.334921in}}%
\pgfpathlineto{\pgfqpoint{1.628862in}{2.310003in}}%
\pgfpathlineto{\pgfqpoint{1.557461in}{2.341186in}}%
\pgfpathlineto{\pgfqpoint{1.489306in}{2.418612in}}%
\pgfpathlineto{\pgfqpoint{1.421095in}{2.337585in}}%
\pgfpathlineto{\pgfqpoint{1.349373in}{2.416360in}}%
\pgfpathlineto{\pgfqpoint{1.283036in}{2.417344in}}%
\pgfpathlineto{\pgfqpoint{1.216322in}{2.395361in}}%
\pgfpathlineto{\pgfqpoint{1.147369in}{2.342344in}}%
\pgfpathlineto{\pgfqpoint{1.079942in}{2.407828in}}%
\pgfpathlineto{\pgfqpoint{1.012853in}{2.339506in}}%
\pgfpathlineto{\pgfqpoint{0.942110in}{2.361029in}}%
\pgfpathlineto{\pgfqpoint{0.875335in}{2.447378in}}%
\pgfpathlineto{\pgfqpoint{0.807094in}{1.768207in}}%
\pgfpathclose%
\pgfusepath{fill}%
\end{pgfscope}%
\begin{pgfscope}%
\pgfpathrectangle{\pgfqpoint{0.781402in}{0.773588in}}{\pgfqpoint{2.110351in}{5.415119in}}%
\pgfusepath{clip}%
\pgfsetbuttcap%
\pgfsetroundjoin%
\definecolor{currentfill}{rgb}{0.549020,0.337255,0.294118}%
\pgfsetfillcolor{currentfill}%
\pgfsetlinewidth{0.000000pt}%
\definecolor{currentstroke}{rgb}{0.000000,0.000000,0.000000}%
\pgfsetstrokecolor{currentstroke}%
\pgfsetdash{}{0pt}%
\pgfpathmoveto{\pgfqpoint{0.807094in}{2.186873in}}%
\pgfpathlineto{\pgfqpoint{0.807094in}{1.768207in}}%
\pgfpathlineto{\pgfqpoint{0.875335in}{2.447378in}}%
\pgfpathlineto{\pgfqpoint{0.942110in}{2.361029in}}%
\pgfpathlineto{\pgfqpoint{1.012853in}{2.339506in}}%
\pgfpathlineto{\pgfqpoint{1.079942in}{2.407828in}}%
\pgfpathlineto{\pgfqpoint{1.147369in}{2.342344in}}%
\pgfpathlineto{\pgfqpoint{1.216322in}{2.395361in}}%
\pgfpathlineto{\pgfqpoint{1.283036in}{2.417344in}}%
\pgfpathlineto{\pgfqpoint{1.349373in}{2.416360in}}%
\pgfpathlineto{\pgfqpoint{1.421095in}{2.337585in}}%
\pgfpathlineto{\pgfqpoint{1.489306in}{2.418612in}}%
\pgfpathlineto{\pgfqpoint{1.557461in}{2.341186in}}%
\pgfpathlineto{\pgfqpoint{1.628862in}{2.310003in}}%
\pgfpathlineto{\pgfqpoint{1.698662in}{2.334921in}}%
\pgfpathlineto{\pgfqpoint{1.766402in}{2.393356in}}%
\pgfpathlineto{\pgfqpoint{1.835890in}{2.409824in}}%
\pgfpathlineto{\pgfqpoint{1.904458in}{2.394984in}}%
\pgfpathlineto{\pgfqpoint{1.970951in}{2.404193in}}%
\pgfpathlineto{\pgfqpoint{2.041179in}{2.398977in}}%
\pgfpathlineto{\pgfqpoint{2.108883in}{2.404848in}}%
\pgfpathlineto{\pgfqpoint{2.177337in}{2.301583in}}%
\pgfpathlineto{\pgfqpoint{2.248172in}{2.352023in}}%
\pgfpathlineto{\pgfqpoint{2.317152in}{2.357001in}}%
\pgfpathlineto{\pgfqpoint{2.387948in}{2.273118in}}%
\pgfpathlineto{\pgfqpoint{2.462769in}{2.298530in}}%
\pgfpathlineto{\pgfqpoint{2.534117in}{2.313522in}}%
\pgfpathlineto{\pgfqpoint{2.606023in}{2.286411in}}%
\pgfpathlineto{\pgfqpoint{2.681331in}{2.252813in}}%
\pgfpathlineto{\pgfqpoint{2.753491in}{2.296750in}}%
\pgfpathlineto{\pgfqpoint{2.826501in}{2.273706in}}%
\pgfpathlineto{\pgfqpoint{2.902674in}{2.185241in}}%
\pgfpathlineto{\pgfqpoint{2.975015in}{2.336790in}}%
\pgfpathlineto{\pgfqpoint{3.047640in}{2.262722in}}%
\pgfpathlineto{\pgfqpoint{3.123114in}{2.295769in}}%
\pgfpathlineto{\pgfqpoint{3.195433in}{2.282530in}}%
\pgfpathlineto{\pgfqpoint{3.277599in}{2.117445in}}%
\pgfpathlineto{\pgfqpoint{3.356465in}{2.204051in}}%
\pgfpathlineto{\pgfqpoint{3.430573in}{2.273638in}}%
\pgfpathlineto{\pgfqpoint{3.503039in}{2.293216in}}%
\pgfpathlineto{\pgfqpoint{3.575543in}{2.415234in}}%
\pgfpathlineto{\pgfqpoint{3.643958in}{2.356364in}}%
\pgfpathlineto{\pgfqpoint{3.712535in}{2.399915in}}%
\pgfpathlineto{\pgfqpoint{3.781205in}{2.459129in}}%
\pgfpathlineto{\pgfqpoint{3.848285in}{2.400018in}}%
\pgfpathlineto{\pgfqpoint{3.914312in}{2.500852in}}%
\pgfpathlineto{\pgfqpoint{3.981945in}{2.406194in}}%
\pgfpathlineto{\pgfqpoint{4.048987in}{2.276365in}}%
\pgfpathlineto{\pgfqpoint{4.116566in}{2.371017in}}%
\pgfpathlineto{\pgfqpoint{4.186150in}{2.363470in}}%
\pgfpathlineto{\pgfqpoint{4.253047in}{2.432371in}}%
\pgfpathlineto{\pgfqpoint{4.319299in}{2.440327in}}%
\pgfpathlineto{\pgfqpoint{4.387175in}{2.441472in}}%
\pgfpathlineto{\pgfqpoint{4.454265in}{2.420274in}}%
\pgfpathlineto{\pgfqpoint{4.520559in}{2.389366in}}%
\pgfpathlineto{\pgfqpoint{4.590503in}{2.366629in}}%
\pgfpathlineto{\pgfqpoint{4.658575in}{2.369430in}}%
\pgfpathlineto{\pgfqpoint{4.726503in}{2.472786in}}%
\pgfpathlineto{\pgfqpoint{4.796564in}{2.338156in}}%
\pgfpathlineto{\pgfqpoint{4.866311in}{2.310785in}}%
\pgfpathlineto{\pgfqpoint{4.935251in}{2.344606in}}%
\pgfpathlineto{\pgfqpoint{5.005143in}{2.392115in}}%
\pgfpathlineto{\pgfqpoint{5.073090in}{2.395244in}}%
\pgfpathlineto{\pgfqpoint{5.142206in}{2.299670in}}%
\pgfpathlineto{\pgfqpoint{5.213575in}{2.370042in}}%
\pgfpathlineto{\pgfqpoint{5.283251in}{2.365402in}}%
\pgfpathlineto{\pgfqpoint{5.353942in}{2.320139in}}%
\pgfpathlineto{\pgfqpoint{5.424203in}{2.396953in}}%
\pgfpathlineto{\pgfqpoint{5.492693in}{2.353124in}}%
\pgfpathlineto{\pgfqpoint{5.561464in}{2.344049in}}%
\pgfpathlineto{\pgfqpoint{5.633136in}{2.340892in}}%
\pgfpathlineto{\pgfqpoint{5.700467in}{2.424659in}}%
\pgfpathlineto{\pgfqpoint{5.766709in}{2.438440in}}%
\pgfpathlineto{\pgfqpoint{5.834669in}{2.515321in}}%
\pgfpathlineto{\pgfqpoint{5.901623in}{2.422676in}}%
\pgfpathlineto{\pgfqpoint{5.968494in}{2.415040in}}%
\pgfpathlineto{\pgfqpoint{6.036866in}{2.454770in}}%
\pgfpathlineto{\pgfqpoint{6.103260in}{2.347344in}}%
\pgfpathlineto{\pgfqpoint{6.169451in}{2.396602in}}%
\pgfpathlineto{\pgfqpoint{6.236861in}{2.491128in}}%
\pgfpathlineto{\pgfqpoint{6.303479in}{2.413452in}}%
\pgfpathlineto{\pgfqpoint{6.369842in}{2.468254in}}%
\pgfpathlineto{\pgfqpoint{6.437563in}{2.483194in}}%
\pgfpathlineto{\pgfqpoint{6.502885in}{2.471125in}}%
\pgfpathlineto{\pgfqpoint{6.568004in}{2.521477in}}%
\pgfpathlineto{\pgfqpoint{6.636218in}{2.349013in}}%
\pgfpathlineto{\pgfqpoint{6.704271in}{2.381557in}}%
\pgfpathlineto{\pgfqpoint{6.770581in}{2.413310in}}%
\pgfpathlineto{\pgfqpoint{6.839348in}{2.407795in}}%
\pgfpathlineto{\pgfqpoint{6.906544in}{2.423239in}}%
\pgfpathlineto{\pgfqpoint{6.973333in}{2.473005in}}%
\pgfpathlineto{\pgfqpoint{7.040805in}{2.524475in}}%
\pgfpathlineto{\pgfqpoint{7.106301in}{2.463778in}}%
\pgfpathlineto{\pgfqpoint{7.171034in}{2.517744in}}%
\pgfpathlineto{\pgfqpoint{7.238831in}{2.444538in}}%
\pgfpathlineto{\pgfqpoint{7.308295in}{2.256817in}}%
\pgfpathlineto{\pgfqpoint{7.379810in}{2.272963in}}%
\pgfpathlineto{\pgfqpoint{7.455306in}{2.253215in}}%
\pgfpathlineto{\pgfqpoint{7.527107in}{2.320950in}}%
\pgfpathlineto{\pgfqpoint{7.598690in}{2.329323in}}%
\pgfpathlineto{\pgfqpoint{7.671131in}{2.329237in}}%
\pgfpathlineto{\pgfqpoint{7.741421in}{2.387356in}}%
\pgfpathlineto{\pgfqpoint{7.812286in}{2.263590in}}%
\pgfpathlineto{\pgfqpoint{7.887441in}{2.276659in}}%
\pgfpathlineto{\pgfqpoint{7.957986in}{2.310678in}}%
\pgfpathlineto{\pgfqpoint{8.028258in}{2.304644in}}%
\pgfpathlineto{\pgfqpoint{8.100457in}{2.350531in}}%
\pgfpathlineto{\pgfqpoint{8.168934in}{2.339596in}}%
\pgfpathlineto{\pgfqpoint{8.237831in}{2.274930in}}%
\pgfpathlineto{\pgfqpoint{8.308718in}{2.415966in}}%
\pgfpathlineto{\pgfqpoint{8.376454in}{2.376527in}}%
\pgfpathlineto{\pgfqpoint{8.444285in}{2.407523in}}%
\pgfpathlineto{\pgfqpoint{8.515085in}{2.347538in}}%
\pgfpathlineto{\pgfqpoint{8.583601in}{2.369451in}}%
\pgfpathlineto{\pgfqpoint{8.651757in}{2.343993in}}%
\pgfpathlineto{\pgfqpoint{8.720615in}{2.493017in}}%
\pgfpathlineto{\pgfqpoint{8.787638in}{2.418926in}}%
\pgfpathlineto{\pgfqpoint{8.856098in}{2.355079in}}%
\pgfpathlineto{\pgfqpoint{8.924860in}{2.492165in}}%
\pgfpathlineto{\pgfqpoint{8.991646in}{2.437245in}}%
\pgfpathlineto{\pgfqpoint{9.059607in}{2.324022in}}%
\pgfpathlineto{\pgfqpoint{9.130518in}{2.342457in}}%
\pgfpathlineto{\pgfqpoint{9.198477in}{2.374259in}}%
\pgfpathlineto{\pgfqpoint{9.265262in}{2.435613in}}%
\pgfpathlineto{\pgfqpoint{9.334728in}{2.332717in}}%
\pgfpathlineto{\pgfqpoint{9.402138in}{2.483410in}}%
\pgfpathlineto{\pgfqpoint{9.467314in}{2.503416in}}%
\pgfpathlineto{\pgfqpoint{9.536145in}{2.401471in}}%
\pgfpathlineto{\pgfqpoint{9.603748in}{2.351056in}}%
\pgfpathlineto{\pgfqpoint{9.670768in}{2.451919in}}%
\pgfpathlineto{\pgfqpoint{9.740273in}{2.326536in}}%
\pgfpathlineto{\pgfqpoint{9.808815in}{2.428756in}}%
\pgfpathlineto{\pgfqpoint{9.876148in}{2.370141in}}%
\pgfpathlineto{\pgfqpoint{9.944615in}{2.524428in}}%
\pgfpathlineto{\pgfqpoint{10.012687in}{2.404361in}}%
\pgfpathlineto{\pgfqpoint{10.081602in}{2.316230in}}%
\pgfpathlineto{\pgfqpoint{10.153891in}{2.387850in}}%
\pgfpathlineto{\pgfqpoint{10.222102in}{2.459106in}}%
\pgfpathlineto{\pgfqpoint{10.290696in}{2.378244in}}%
\pgfpathlineto{\pgfqpoint{10.360853in}{2.401431in}}%
\pgfpathlineto{\pgfqpoint{10.428999in}{2.375232in}}%
\pgfpathlineto{\pgfqpoint{10.498528in}{2.385212in}}%
\pgfpathlineto{\pgfqpoint{10.569898in}{2.317130in}}%
\pgfpathlineto{\pgfqpoint{10.639006in}{2.392576in}}%
\pgfpathlineto{\pgfqpoint{10.708017in}{2.385366in}}%
\pgfpathlineto{\pgfqpoint{10.777925in}{2.465290in}}%
\pgfpathlineto{\pgfqpoint{10.846851in}{2.362275in}}%
\pgfpathlineto{\pgfqpoint{10.916008in}{2.355493in}}%
\pgfpathlineto{\pgfqpoint{10.986941in}{2.445029in}}%
\pgfpathlineto{\pgfqpoint{11.055389in}{2.377512in}}%
\pgfpathlineto{\pgfqpoint{11.125133in}{2.352777in}}%
\pgfpathlineto{\pgfqpoint{11.194984in}{2.396114in}}%
\pgfpathlineto{\pgfqpoint{11.261777in}{2.462632in}}%
\pgfpathlineto{\pgfqpoint{11.328677in}{2.440319in}}%
\pgfpathlineto{\pgfqpoint{11.398136in}{2.391542in}}%
\pgfpathlineto{\pgfqpoint{11.465294in}{2.465238in}}%
\pgfpathlineto{\pgfqpoint{11.532877in}{2.379878in}}%
\pgfpathlineto{\pgfqpoint{11.602395in}{2.427688in}}%
\pgfpathlineto{\pgfqpoint{11.669969in}{2.418078in}}%
\pgfpathlineto{\pgfqpoint{11.736702in}{2.399387in}}%
\pgfpathlineto{\pgfqpoint{11.804695in}{2.447359in}}%
\pgfpathlineto{\pgfqpoint{11.871239in}{2.412976in}}%
\pgfpathlineto{\pgfqpoint{11.938397in}{2.312124in}}%
\pgfpathlineto{\pgfqpoint{12.008843in}{2.309380in}}%
\pgfpathlineto{\pgfqpoint{12.076638in}{2.412443in}}%
\pgfpathlineto{\pgfqpoint{12.144815in}{2.287899in}}%
\pgfpathlineto{\pgfqpoint{12.213872in}{2.340878in}}%
\pgfpathlineto{\pgfqpoint{12.280761in}{2.454963in}}%
\pgfpathlineto{\pgfqpoint{12.347008in}{2.445924in}}%
\pgfpathlineto{\pgfqpoint{12.416448in}{2.381820in}}%
\pgfpathlineto{\pgfqpoint{12.484090in}{2.329892in}}%
\pgfpathlineto{\pgfqpoint{12.552597in}{2.424792in}}%
\pgfpathlineto{\pgfqpoint{12.623527in}{2.382185in}}%
\pgfpathlineto{\pgfqpoint{12.692271in}{2.394316in}}%
\pgfpathlineto{\pgfqpoint{12.759967in}{2.460769in}}%
\pgfpathlineto{\pgfqpoint{12.830499in}{2.350302in}}%
\pgfpathlineto{\pgfqpoint{12.900232in}{2.345595in}}%
\pgfpathlineto{\pgfqpoint{12.969036in}{2.316575in}}%
\pgfpathlineto{\pgfqpoint{13.041078in}{2.275523in}}%
\pgfpathlineto{\pgfqpoint{13.109710in}{2.363697in}}%
\pgfpathlineto{\pgfqpoint{13.178279in}{2.321556in}}%
\pgfpathlineto{\pgfqpoint{13.250526in}{2.313702in}}%
\pgfpathlineto{\pgfqpoint{13.319414in}{2.338118in}}%
\pgfpathlineto{\pgfqpoint{13.387091in}{2.389023in}}%
\pgfpathlineto{\pgfqpoint{13.457833in}{2.329708in}}%
\pgfpathlineto{\pgfqpoint{13.526360in}{2.400693in}}%
\pgfpathlineto{\pgfqpoint{13.595052in}{2.404581in}}%
\pgfpathlineto{\pgfqpoint{13.666880in}{2.298388in}}%
\pgfpathlineto{\pgfqpoint{13.735499in}{2.360084in}}%
\pgfpathlineto{\pgfqpoint{13.804228in}{2.364455in}}%
\pgfpathlineto{\pgfqpoint{13.876785in}{2.318313in}}%
\pgfpathlineto{\pgfqpoint{13.944783in}{2.405673in}}%
\pgfpathlineto{\pgfqpoint{14.012639in}{2.322965in}}%
\pgfpathlineto{\pgfqpoint{14.081564in}{2.427453in}}%
\pgfpathlineto{\pgfqpoint{14.149562in}{2.396819in}}%
\pgfpathlineto{\pgfqpoint{14.216102in}{2.399801in}}%
\pgfpathlineto{\pgfqpoint{14.286685in}{2.344426in}}%
\pgfpathlineto{\pgfqpoint{14.355979in}{2.324319in}}%
\pgfpathlineto{\pgfqpoint{14.424138in}{2.394395in}}%
\pgfpathlineto{\pgfqpoint{14.494110in}{2.382148in}}%
\pgfpathlineto{\pgfqpoint{14.562284in}{2.368556in}}%
\pgfpathlineto{\pgfqpoint{14.630523in}{2.406563in}}%
\pgfpathlineto{\pgfqpoint{14.700003in}{2.378815in}}%
\pgfpathlineto{\pgfqpoint{14.768225in}{2.407033in}}%
\pgfpathlineto{\pgfqpoint{14.836607in}{2.352508in}}%
\pgfpathlineto{\pgfqpoint{14.908199in}{2.269316in}}%
\pgfpathlineto{\pgfqpoint{14.976372in}{2.414390in}}%
\pgfpathlineto{\pgfqpoint{15.044439in}{2.362664in}}%
\pgfpathlineto{\pgfqpoint{15.115140in}{2.334152in}}%
\pgfpathlineto{\pgfqpoint{15.182512in}{2.425014in}}%
\pgfpathlineto{\pgfqpoint{15.249597in}{2.366103in}}%
\pgfpathlineto{\pgfqpoint{15.320168in}{2.386687in}}%
\pgfpathlineto{\pgfqpoint{15.388582in}{2.367244in}}%
\pgfpathlineto{\pgfqpoint{15.457342in}{2.315581in}}%
\pgfpathlineto{\pgfqpoint{15.527961in}{2.337079in}}%
\pgfpathlineto{\pgfqpoint{15.595878in}{2.387626in}}%
\pgfpathlineto{\pgfqpoint{15.665480in}{2.298817in}}%
\pgfpathlineto{\pgfqpoint{15.737902in}{2.339944in}}%
\pgfpathlineto{\pgfqpoint{15.807467in}{2.316651in}}%
\pgfpathlineto{\pgfqpoint{15.878678in}{2.309932in}}%
\pgfpathlineto{\pgfqpoint{15.949677in}{2.389314in}}%
\pgfpathlineto{\pgfqpoint{16.019598in}{2.336319in}}%
\pgfpathlineto{\pgfqpoint{16.088678in}{2.378177in}}%
\pgfpathlineto{\pgfqpoint{16.160188in}{2.441511in}}%
\pgfpathlineto{\pgfqpoint{16.229236in}{2.416211in}}%
\pgfpathlineto{\pgfqpoint{16.299236in}{2.346079in}}%
\pgfpathlineto{\pgfqpoint{16.373054in}{2.296087in}}%
\pgfpathlineto{\pgfqpoint{16.445981in}{2.213369in}}%
\pgfpathlineto{\pgfqpoint{16.519393in}{2.237324in}}%
\pgfpathlineto{\pgfqpoint{16.592371in}{2.408326in}}%
\pgfpathlineto{\pgfqpoint{16.662798in}{2.362100in}}%
\pgfpathlineto{\pgfqpoint{16.731829in}{2.347150in}}%
\pgfpathlineto{\pgfqpoint{16.803402in}{2.334003in}}%
\pgfpathlineto{\pgfqpoint{16.872974in}{2.315337in}}%
\pgfpathlineto{\pgfqpoint{16.942299in}{2.319482in}}%
\pgfpathlineto{\pgfqpoint{17.013249in}{2.402088in}}%
\pgfpathlineto{\pgfqpoint{17.081113in}{2.450638in}}%
\pgfpathlineto{\pgfqpoint{17.150280in}{2.317372in}}%
\pgfpathlineto{\pgfqpoint{17.220960in}{2.310083in}}%
\pgfpathlineto{\pgfqpoint{17.288800in}{2.420723in}}%
\pgfpathlineto{\pgfqpoint{17.356523in}{2.344504in}}%
\pgfpathlineto{\pgfqpoint{17.426570in}{2.414408in}}%
\pgfpathlineto{\pgfqpoint{17.494859in}{2.295254in}}%
\pgfpathlineto{\pgfqpoint{17.563254in}{2.352403in}}%
\pgfpathlineto{\pgfqpoint{17.634782in}{2.313263in}}%
\pgfpathlineto{\pgfqpoint{17.703480in}{2.375946in}}%
\pgfpathlineto{\pgfqpoint{17.772898in}{2.315892in}}%
\pgfpathlineto{\pgfqpoint{17.844148in}{2.355139in}}%
\pgfpathlineto{\pgfqpoint{17.912627in}{2.376504in}}%
\pgfpathlineto{\pgfqpoint{17.983524in}{2.245730in}}%
\pgfpathlineto{\pgfqpoint{18.056086in}{2.293209in}}%
\pgfpathlineto{\pgfqpoint{18.125960in}{2.341081in}}%
\pgfpathlineto{\pgfqpoint{18.195268in}{2.353926in}}%
\pgfpathlineto{\pgfqpoint{18.266780in}{2.377914in}}%
\pgfpathlineto{\pgfqpoint{18.336223in}{2.426745in}}%
\pgfpathlineto{\pgfqpoint{18.406899in}{2.321433in}}%
\pgfpathlineto{\pgfqpoint{18.479143in}{2.296630in}}%
\pgfpathlineto{\pgfqpoint{18.549509in}{2.394626in}}%
\pgfpathlineto{\pgfqpoint{18.619849in}{2.333533in}}%
\pgfpathlineto{\pgfqpoint{18.692839in}{2.416355in}}%
\pgfpathlineto{\pgfqpoint{18.763558in}{2.293835in}}%
\pgfpathlineto{\pgfqpoint{18.834141in}{2.284510in}}%
\pgfpathlineto{\pgfqpoint{18.906958in}{2.330027in}}%
\pgfpathlineto{\pgfqpoint{18.977585in}{2.266246in}}%
\pgfpathlineto{\pgfqpoint{19.048835in}{2.271599in}}%
\pgfpathlineto{\pgfqpoint{19.123240in}{2.232984in}}%
\pgfpathlineto{\pgfqpoint{19.195134in}{2.289248in}}%
\pgfpathlineto{\pgfqpoint{19.266961in}{2.284268in}}%
\pgfpathlineto{\pgfqpoint{19.340700in}{2.296885in}}%
\pgfpathlineto{\pgfqpoint{19.412352in}{2.283026in}}%
\pgfpathlineto{\pgfqpoint{19.482791in}{2.360552in}}%
\pgfpathlineto{\pgfqpoint{19.553641in}{2.351876in}}%
\pgfpathlineto{\pgfqpoint{19.623855in}{2.419361in}}%
\pgfpathlineto{\pgfqpoint{19.693765in}{2.277444in}}%
\pgfpathlineto{\pgfqpoint{19.766060in}{2.363164in}}%
\pgfpathlineto{\pgfqpoint{19.835928in}{2.389725in}}%
\pgfpathlineto{\pgfqpoint{19.907049in}{2.285374in}}%
\pgfpathlineto{\pgfqpoint{19.981199in}{2.286178in}}%
\pgfpathlineto{\pgfqpoint{20.052071in}{2.291057in}}%
\pgfpathlineto{\pgfqpoint{20.121885in}{2.382111in}}%
\pgfpathlineto{\pgfqpoint{20.193561in}{2.294331in}}%
\pgfpathlineto{\pgfqpoint{20.263963in}{2.361737in}}%
\pgfpathlineto{\pgfqpoint{20.334605in}{2.289116in}}%
\pgfpathlineto{\pgfqpoint{20.407276in}{2.336694in}}%
\pgfpathlineto{\pgfqpoint{20.476961in}{2.308828in}}%
\pgfpathlineto{\pgfqpoint{20.547172in}{2.261538in}}%
\pgfpathlineto{\pgfqpoint{20.618428in}{2.387675in}}%
\pgfpathlineto{\pgfqpoint{20.688980in}{2.326817in}}%
\pgfpathlineto{\pgfqpoint{20.758814in}{2.293630in}}%
\pgfpathlineto{\pgfqpoint{20.830586in}{2.379466in}}%
\pgfpathlineto{\pgfqpoint{20.899587in}{2.361138in}}%
\pgfpathlineto{\pgfqpoint{20.969373in}{2.236644in}}%
\pgfpathlineto{\pgfqpoint{21.040864in}{2.379865in}}%
\pgfpathlineto{\pgfqpoint{21.110656in}{2.319899in}}%
\pgfpathlineto{\pgfqpoint{21.181233in}{2.235383in}}%
\pgfpathlineto{\pgfqpoint{21.254645in}{2.304203in}}%
\pgfpathlineto{\pgfqpoint{21.324498in}{2.355684in}}%
\pgfpathlineto{\pgfqpoint{21.394839in}{2.299263in}}%
\pgfpathlineto{\pgfqpoint{21.467741in}{2.288976in}}%
\pgfpathlineto{\pgfqpoint{21.539601in}{2.268173in}}%
\pgfpathlineto{\pgfqpoint{21.610878in}{2.303436in}}%
\pgfpathlineto{\pgfqpoint{21.683994in}{2.339697in}}%
\pgfpathlineto{\pgfqpoint{21.756227in}{2.259136in}}%
\pgfpathlineto{\pgfqpoint{21.828555in}{2.291422in}}%
\pgfpathlineto{\pgfqpoint{21.903868in}{2.291199in}}%
\pgfpathlineto{\pgfqpoint{21.976836in}{2.197014in}}%
\pgfpathlineto{\pgfqpoint{22.048040in}{2.330770in}}%
\pgfpathlineto{\pgfqpoint{22.122462in}{2.278881in}}%
\pgfpathlineto{\pgfqpoint{22.195707in}{2.170512in}}%
\pgfpathlineto{\pgfqpoint{22.268824in}{2.223836in}}%
\pgfpathlineto{\pgfqpoint{22.343331in}{2.348390in}}%
\pgfpathlineto{\pgfqpoint{22.413449in}{2.303340in}}%
\pgfpathlineto{\pgfqpoint{22.482516in}{2.398083in}}%
\pgfpathlineto{\pgfqpoint{22.553535in}{2.323391in}}%
\pgfpathlineto{\pgfqpoint{22.624114in}{2.348659in}}%
\pgfpathlineto{\pgfqpoint{22.694016in}{2.395373in}}%
\pgfpathlineto{\pgfqpoint{22.764651in}{2.482801in}}%
\pgfpathlineto{\pgfqpoint{22.833628in}{2.421459in}}%
\pgfpathlineto{\pgfqpoint{22.902896in}{2.397494in}}%
\pgfpathlineto{\pgfqpoint{22.973868in}{2.389573in}}%
\pgfpathlineto{\pgfqpoint{23.043397in}{2.293230in}}%
\pgfpathlineto{\pgfqpoint{23.113267in}{2.326259in}}%
\pgfpathlineto{\pgfqpoint{23.184270in}{2.386374in}}%
\pgfpathlineto{\pgfqpoint{23.253484in}{2.376160in}}%
\pgfpathlineto{\pgfqpoint{23.323995in}{2.269897in}}%
\pgfpathlineto{\pgfqpoint{23.396126in}{2.315437in}}%
\pgfpathlineto{\pgfqpoint{23.467323in}{2.253040in}}%
\pgfpathlineto{\pgfqpoint{23.537850in}{2.319420in}}%
\pgfpathlineto{\pgfqpoint{23.610036in}{2.363105in}}%
\pgfpathlineto{\pgfqpoint{23.681448in}{2.241376in}}%
\pgfpathlineto{\pgfqpoint{23.752361in}{2.328198in}}%
\pgfpathlineto{\pgfqpoint{23.824951in}{2.348982in}}%
\pgfpathlineto{\pgfqpoint{23.895213in}{2.345307in}}%
\pgfpathlineto{\pgfqpoint{23.966722in}{2.214818in}}%
\pgfpathlineto{\pgfqpoint{24.039255in}{2.386143in}}%
\pgfpathlineto{\pgfqpoint{24.111784in}{2.210574in}}%
\pgfpathlineto{\pgfqpoint{24.183899in}{2.348892in}}%
\pgfpathlineto{\pgfqpoint{24.257307in}{2.255320in}}%
\pgfpathlineto{\pgfqpoint{24.329090in}{2.306604in}}%
\pgfpathlineto{\pgfqpoint{24.400328in}{2.260240in}}%
\pgfpathlineto{\pgfqpoint{24.476339in}{2.176033in}}%
\pgfpathlineto{\pgfqpoint{24.548639in}{2.286684in}}%
\pgfpathlineto{\pgfqpoint{24.618678in}{2.393918in}}%
\pgfpathlineto{\pgfqpoint{24.691660in}{2.296715in}}%
\pgfpathlineto{\pgfqpoint{24.764742in}{2.218280in}}%
\pgfpathlineto{\pgfqpoint{24.836992in}{2.301860in}}%
\pgfpathlineto{\pgfqpoint{24.911741in}{2.253604in}}%
\pgfpathlineto{\pgfqpoint{24.983525in}{2.296342in}}%
\pgfpathlineto{\pgfqpoint{25.055567in}{2.265861in}}%
\pgfpathlineto{\pgfqpoint{25.131109in}{2.237827in}}%
\pgfpathlineto{\pgfqpoint{25.203216in}{2.321418in}}%
\pgfpathlineto{\pgfqpoint{25.273349in}{2.342925in}}%
\pgfpathlineto{\pgfqpoint{25.347124in}{2.209151in}}%
\pgfpathlineto{\pgfqpoint{25.417047in}{2.378551in}}%
\pgfpathlineto{\pgfqpoint{25.487573in}{2.303464in}}%
\pgfpathlineto{\pgfqpoint{25.560110in}{2.283593in}}%
\pgfpathlineto{\pgfqpoint{25.631022in}{2.327916in}}%
\pgfpathlineto{\pgfqpoint{25.702341in}{2.259177in}}%
\pgfpathlineto{\pgfqpoint{25.775695in}{2.302233in}}%
\pgfpathlineto{\pgfqpoint{25.845667in}{2.389794in}}%
\pgfpathlineto{\pgfqpoint{25.916551in}{2.342600in}}%
\pgfpathlineto{\pgfqpoint{25.988588in}{2.408263in}}%
\pgfpathlineto{\pgfqpoint{26.058621in}{2.295589in}}%
\pgfpathlineto{\pgfqpoint{26.130346in}{2.300808in}}%
\pgfpathlineto{\pgfqpoint{26.203572in}{2.256552in}}%
\pgfpathlineto{\pgfqpoint{26.274267in}{2.288707in}}%
\pgfpathlineto{\pgfqpoint{26.344920in}{2.323634in}}%
\pgfpathlineto{\pgfqpoint{26.417231in}{2.324241in}}%
\pgfpathlineto{\pgfqpoint{26.487420in}{2.378132in}}%
\pgfpathlineto{\pgfqpoint{26.557235in}{2.412854in}}%
\pgfpathlineto{\pgfqpoint{26.629572in}{2.283473in}}%
\pgfpathlineto{\pgfqpoint{26.699584in}{2.336737in}}%
\pgfpathlineto{\pgfqpoint{26.769271in}{2.375722in}}%
\pgfpathlineto{\pgfqpoint{26.841234in}{2.274754in}}%
\pgfpathlineto{\pgfqpoint{26.912667in}{2.325656in}}%
\pgfpathlineto{\pgfqpoint{26.983641in}{2.312856in}}%
\pgfpathlineto{\pgfqpoint{27.056835in}{2.356686in}}%
\pgfpathlineto{\pgfqpoint{27.128948in}{2.266617in}}%
\pgfpathlineto{\pgfqpoint{27.201477in}{2.293648in}}%
\pgfpathlineto{\pgfqpoint{27.277488in}{2.267224in}}%
\pgfpathlineto{\pgfqpoint{27.350990in}{2.267292in}}%
\pgfpathlineto{\pgfqpoint{27.423884in}{2.307017in}}%
\pgfpathlineto{\pgfqpoint{27.500063in}{2.202434in}}%
\pgfpathlineto{\pgfqpoint{27.574929in}{2.248437in}}%
\pgfpathlineto{\pgfqpoint{27.649072in}{2.226320in}}%
\pgfpathlineto{\pgfqpoint{27.724006in}{2.281051in}}%
\pgfpathlineto{\pgfqpoint{27.795343in}{2.299883in}}%
\pgfpathlineto{\pgfqpoint{27.868117in}{2.204306in}}%
\pgfpathlineto{\pgfqpoint{27.943911in}{2.189532in}}%
\pgfpathlineto{\pgfqpoint{28.018234in}{2.221174in}}%
\pgfpathlineto{\pgfqpoint{28.090360in}{2.334452in}}%
\pgfpathlineto{\pgfqpoint{28.163352in}{2.334604in}}%
\pgfpathlineto{\pgfqpoint{28.234559in}{2.300486in}}%
\pgfpathlineto{\pgfqpoint{28.306604in}{2.253161in}}%
\pgfpathlineto{\pgfqpoint{28.380501in}{2.251101in}}%
\pgfpathlineto{\pgfqpoint{28.451943in}{2.210001in}}%
\pgfpathlineto{\pgfqpoint{28.522534in}{2.371326in}}%
\pgfpathlineto{\pgfqpoint{28.596041in}{2.249022in}}%
\pgfpathlineto{\pgfqpoint{28.668204in}{2.281621in}}%
\pgfpathlineto{\pgfqpoint{28.738605in}{2.325439in}}%
\pgfpathlineto{\pgfqpoint{28.811911in}{2.244114in}}%
\pgfpathlineto{\pgfqpoint{28.885217in}{2.243646in}}%
\pgfpathlineto{\pgfqpoint{28.956832in}{2.301102in}}%
\pgfpathlineto{\pgfqpoint{29.029887in}{2.351888in}}%
\pgfpathlineto{\pgfqpoint{29.100748in}{2.328953in}}%
\pgfpathlineto{\pgfqpoint{29.173201in}{2.264222in}}%
\pgfpathlineto{\pgfqpoint{29.248973in}{2.266146in}}%
\pgfpathlineto{\pgfqpoint{29.320759in}{2.212543in}}%
\pgfpathlineto{\pgfqpoint{29.393660in}{2.293684in}}%
\pgfpathlineto{\pgfqpoint{29.467898in}{2.225859in}}%
\pgfpathlineto{\pgfqpoint{29.540420in}{2.308092in}}%
\pgfpathlineto{\pgfqpoint{29.611700in}{2.298618in}}%
\pgfpathlineto{\pgfqpoint{29.684427in}{2.314700in}}%
\pgfpathlineto{\pgfqpoint{29.755113in}{2.339174in}}%
\pgfpathlineto{\pgfqpoint{29.827132in}{2.325966in}}%
\pgfpathlineto{\pgfqpoint{29.901656in}{2.260473in}}%
\pgfpathlineto{\pgfqpoint{29.974646in}{2.294793in}}%
\pgfpathlineto{\pgfqpoint{30.048252in}{2.233145in}}%
\pgfpathlineto{\pgfqpoint{30.122796in}{2.302823in}}%
\pgfpathlineto{\pgfqpoint{30.195443in}{2.267018in}}%
\pgfpathlineto{\pgfqpoint{30.269036in}{2.240572in}}%
\pgfpathlineto{\pgfqpoint{30.344328in}{2.233525in}}%
\pgfpathlineto{\pgfqpoint{30.417098in}{2.297310in}}%
\pgfpathlineto{\pgfqpoint{30.488991in}{2.314096in}}%
\pgfpathlineto{\pgfqpoint{30.562714in}{2.342473in}}%
\pgfpathlineto{\pgfqpoint{30.634099in}{2.290693in}}%
\pgfpathlineto{\pgfqpoint{30.707828in}{2.262804in}}%
\pgfpathlineto{\pgfqpoint{30.782249in}{2.307196in}}%
\pgfpathlineto{\pgfqpoint{30.854115in}{2.345668in}}%
\pgfpathlineto{\pgfqpoint{30.928305in}{2.226503in}}%
\pgfpathlineto{\pgfqpoint{31.002514in}{2.291781in}}%
\pgfpathlineto{\pgfqpoint{31.074452in}{2.277852in}}%
\pgfpathlineto{\pgfqpoint{31.147740in}{2.263642in}}%
\pgfpathlineto{\pgfqpoint{31.222913in}{2.184436in}}%
\pgfpathlineto{\pgfqpoint{31.294777in}{2.280819in}}%
\pgfpathlineto{\pgfqpoint{31.366613in}{2.314323in}}%
\pgfpathlineto{\pgfqpoint{31.439415in}{2.321634in}}%
\pgfpathlineto{\pgfqpoint{31.510140in}{2.302230in}}%
\pgfpathlineto{\pgfqpoint{31.582282in}{2.268170in}}%
\pgfpathlineto{\pgfqpoint{31.656180in}{2.336838in}}%
\pgfpathlineto{\pgfqpoint{31.728521in}{2.251796in}}%
\pgfpathlineto{\pgfqpoint{31.800877in}{2.280462in}}%
\pgfpathlineto{\pgfqpoint{31.873539in}{2.331917in}}%
\pgfpathlineto{\pgfqpoint{31.943734in}{2.369077in}}%
\pgfpathlineto{\pgfqpoint{32.015122in}{2.327995in}}%
\pgfpathlineto{\pgfqpoint{32.089684in}{2.225068in}}%
\pgfpathlineto{\pgfqpoint{32.161504in}{2.355976in}}%
\pgfpathlineto{\pgfqpoint{32.231773in}{2.288985in}}%
\pgfpathlineto{\pgfqpoint{32.305440in}{2.296487in}}%
\pgfpathlineto{\pgfqpoint{32.377016in}{2.315904in}}%
\pgfpathlineto{\pgfqpoint{32.447439in}{2.373974in}}%
\pgfpathlineto{\pgfqpoint{32.520401in}{2.391312in}}%
\pgfpathlineto{\pgfqpoint{32.590674in}{2.343567in}}%
\pgfpathlineto{\pgfqpoint{32.663709in}{2.224849in}}%
\pgfpathlineto{\pgfqpoint{32.740263in}{2.198868in}}%
\pgfpathlineto{\pgfqpoint{32.813546in}{2.302499in}}%
\pgfpathlineto{\pgfqpoint{32.887492in}{2.259385in}}%
\pgfpathlineto{\pgfqpoint{32.963168in}{2.321530in}}%
\pgfpathlineto{\pgfqpoint{33.037794in}{2.220336in}}%
\pgfpathlineto{\pgfqpoint{33.110479in}{2.347512in}}%
\pgfpathlineto{\pgfqpoint{33.185787in}{2.261273in}}%
\pgfpathlineto{\pgfqpoint{33.259507in}{2.258974in}}%
\pgfpathlineto{\pgfqpoint{33.333311in}{2.263560in}}%
\pgfpathlineto{\pgfqpoint{33.409286in}{2.213105in}}%
\pgfpathlineto{\pgfqpoint{33.483328in}{2.250884in}}%
\pgfpathlineto{\pgfqpoint{33.557012in}{2.262254in}}%
\pgfpathlineto{\pgfqpoint{33.631884in}{2.326432in}}%
\pgfpathlineto{\pgfqpoint{33.703848in}{2.340502in}}%
\pgfpathlineto{\pgfqpoint{33.776888in}{2.315839in}}%
\pgfpathlineto{\pgfqpoint{33.852393in}{2.319907in}}%
\pgfpathlineto{\pgfqpoint{33.923536in}{2.304522in}}%
\pgfpathlineto{\pgfqpoint{33.994648in}{2.292558in}}%
\pgfpathlineto{\pgfqpoint{34.067999in}{2.399372in}}%
\pgfpathlineto{\pgfqpoint{34.138346in}{2.364491in}}%
\pgfpathlineto{\pgfqpoint{34.210760in}{2.335533in}}%
\pgfpathlineto{\pgfqpoint{34.284339in}{2.239019in}}%
\pgfpathlineto{\pgfqpoint{34.354648in}{2.265161in}}%
\pgfpathlineto{\pgfqpoint{34.425604in}{2.343943in}}%
\pgfpathlineto{\pgfqpoint{34.499162in}{2.315701in}}%
\pgfpathlineto{\pgfqpoint{34.571449in}{2.215727in}}%
\pgfpathlineto{\pgfqpoint{34.643977in}{2.263431in}}%
\pgfpathlineto{\pgfqpoint{34.718731in}{2.268673in}}%
\pgfpathlineto{\pgfqpoint{34.789698in}{2.375130in}}%
\pgfpathlineto{\pgfqpoint{34.862212in}{2.316665in}}%
\pgfpathlineto{\pgfqpoint{34.936943in}{2.228499in}}%
\pgfpathlineto{\pgfqpoint{35.007838in}{2.329202in}}%
\pgfpathlineto{\pgfqpoint{35.080154in}{2.265781in}}%
\pgfpathlineto{\pgfqpoint{35.155466in}{2.252098in}}%
\pgfpathlineto{\pgfqpoint{35.227201in}{2.323501in}}%
\pgfpathlineto{\pgfqpoint{35.298174in}{2.365524in}}%
\pgfpathlineto{\pgfqpoint{35.372990in}{2.229596in}}%
\pgfpathlineto{\pgfqpoint{35.451774in}{2.152523in}}%
\pgfpathlineto{\pgfqpoint{35.574549in}{1.695897in}}%
\pgfpathlineto{\pgfqpoint{35.663523in}{1.559351in}}%
\pgfpathlineto{\pgfqpoint{35.741519in}{0.926398in}}%
\pgfpathlineto{\pgfqpoint{35.805568in}{1.724813in}}%
\pgfpathlineto{\pgfqpoint{35.870813in}{5.187514in}}%
\pgfpathlineto{\pgfqpoint{35.942832in}{5.319244in}}%
\pgfpathlineto{\pgfqpoint{36.012796in}{5.506716in}}%
\pgfpathlineto{\pgfqpoint{36.085094in}{5.399336in}}%
\pgfpathlineto{\pgfqpoint{36.154695in}{5.488048in}}%
\pgfpathlineto{\pgfqpoint{36.223624in}{5.539128in}}%
\pgfpathlineto{\pgfqpoint{36.293479in}{5.638191in}}%
\pgfpathlineto{\pgfqpoint{36.360634in}{5.619869in}}%
\pgfpathlineto{\pgfqpoint{36.428206in}{5.661488in}}%
\pgfpathlineto{\pgfqpoint{36.497087in}{5.625548in}}%
\pgfpathlineto{\pgfqpoint{36.563097in}{5.748599in}}%
\pgfpathlineto{\pgfqpoint{36.628950in}{5.732424in}}%
\pgfpathlineto{\pgfqpoint{36.696952in}{5.790749in}}%
\pgfpathlineto{\pgfqpoint{36.761893in}{5.793255in}}%
\pgfpathlineto{\pgfqpoint{36.827337in}{5.772285in}}%
\pgfpathlineto{\pgfqpoint{36.893714in}{5.875626in}}%
\pgfpathlineto{\pgfqpoint{36.957470in}{5.879596in}}%
\pgfpathlineto{\pgfqpoint{37.022217in}{5.814953in}}%
\pgfpathlineto{\pgfqpoint{37.088015in}{5.930845in}}%
\pgfpathlineto{\pgfqpoint{37.151827in}{5.872359in}}%
\pgfpathlineto{\pgfqpoint{37.151827in}{5.872359in}}%
\pgfpathlineto{\pgfqpoint{37.151827in}{5.872359in}}%
\pgfpathlineto{\pgfqpoint{37.088015in}{5.930845in}}%
\pgfpathlineto{\pgfqpoint{37.022217in}{5.814953in}}%
\pgfpathlineto{\pgfqpoint{36.957470in}{5.879596in}}%
\pgfpathlineto{\pgfqpoint{36.893714in}{5.875626in}}%
\pgfpathlineto{\pgfqpoint{36.827337in}{5.772285in}}%
\pgfpathlineto{\pgfqpoint{36.761893in}{5.793255in}}%
\pgfpathlineto{\pgfqpoint{36.696952in}{5.790749in}}%
\pgfpathlineto{\pgfqpoint{36.628950in}{5.732424in}}%
\pgfpathlineto{\pgfqpoint{36.563097in}{5.748599in}}%
\pgfpathlineto{\pgfqpoint{36.497087in}{5.625548in}}%
\pgfpathlineto{\pgfqpoint{36.428206in}{5.661488in}}%
\pgfpathlineto{\pgfqpoint{36.360634in}{5.619869in}}%
\pgfpathlineto{\pgfqpoint{36.293479in}{5.638191in}}%
\pgfpathlineto{\pgfqpoint{36.223624in}{5.539128in}}%
\pgfpathlineto{\pgfqpoint{36.154695in}{5.488048in}}%
\pgfpathlineto{\pgfqpoint{36.085094in}{5.399336in}}%
\pgfpathlineto{\pgfqpoint{36.012796in}{5.506716in}}%
\pgfpathlineto{\pgfqpoint{35.942832in}{5.319244in}}%
\pgfpathlineto{\pgfqpoint{35.870813in}{5.187514in}}%
\pgfpathlineto{\pgfqpoint{35.805568in}{1.724813in}}%
\pgfpathlineto{\pgfqpoint{35.741519in}{0.926398in}}%
\pgfpathlineto{\pgfqpoint{35.663523in}{1.559351in}}%
\pgfpathlineto{\pgfqpoint{35.574549in}{1.695897in}}%
\pgfpathlineto{\pgfqpoint{35.451774in}{2.152523in}}%
\pgfpathlineto{\pgfqpoint{35.372990in}{2.229596in}}%
\pgfpathlineto{\pgfqpoint{35.298174in}{2.365524in}}%
\pgfpathlineto{\pgfqpoint{35.227201in}{2.323501in}}%
\pgfpathlineto{\pgfqpoint{35.155466in}{2.252098in}}%
\pgfpathlineto{\pgfqpoint{35.080154in}{2.265781in}}%
\pgfpathlineto{\pgfqpoint{35.007838in}{2.329202in}}%
\pgfpathlineto{\pgfqpoint{34.936943in}{2.228499in}}%
\pgfpathlineto{\pgfqpoint{34.862212in}{2.316665in}}%
\pgfpathlineto{\pgfqpoint{34.789698in}{2.375130in}}%
\pgfpathlineto{\pgfqpoint{34.718731in}{2.268673in}}%
\pgfpathlineto{\pgfqpoint{34.643977in}{2.263431in}}%
\pgfpathlineto{\pgfqpoint{34.571449in}{2.215727in}}%
\pgfpathlineto{\pgfqpoint{34.499162in}{2.315701in}}%
\pgfpathlineto{\pgfqpoint{34.425604in}{2.343943in}}%
\pgfpathlineto{\pgfqpoint{34.354648in}{2.265161in}}%
\pgfpathlineto{\pgfqpoint{34.284339in}{2.239019in}}%
\pgfpathlineto{\pgfqpoint{34.210760in}{2.335533in}}%
\pgfpathlineto{\pgfqpoint{34.138346in}{2.364491in}}%
\pgfpathlineto{\pgfqpoint{34.067999in}{2.399372in}}%
\pgfpathlineto{\pgfqpoint{33.994648in}{2.292558in}}%
\pgfpathlineto{\pgfqpoint{33.923536in}{2.304522in}}%
\pgfpathlineto{\pgfqpoint{33.852393in}{2.319907in}}%
\pgfpathlineto{\pgfqpoint{33.776888in}{2.315839in}}%
\pgfpathlineto{\pgfqpoint{33.703848in}{2.340502in}}%
\pgfpathlineto{\pgfqpoint{33.631884in}{2.326432in}}%
\pgfpathlineto{\pgfqpoint{33.557012in}{2.262254in}}%
\pgfpathlineto{\pgfqpoint{33.483328in}{2.250884in}}%
\pgfpathlineto{\pgfqpoint{33.409286in}{2.213105in}}%
\pgfpathlineto{\pgfqpoint{33.333311in}{2.263560in}}%
\pgfpathlineto{\pgfqpoint{33.259507in}{2.258974in}}%
\pgfpathlineto{\pgfqpoint{33.185787in}{2.261273in}}%
\pgfpathlineto{\pgfqpoint{33.110479in}{2.347512in}}%
\pgfpathlineto{\pgfqpoint{33.037794in}{2.220336in}}%
\pgfpathlineto{\pgfqpoint{32.963168in}{2.321530in}}%
\pgfpathlineto{\pgfqpoint{32.887492in}{2.259385in}}%
\pgfpathlineto{\pgfqpoint{32.813546in}{2.302499in}}%
\pgfpathlineto{\pgfqpoint{32.740263in}{2.198868in}}%
\pgfpathlineto{\pgfqpoint{32.663709in}{2.224849in}}%
\pgfpathlineto{\pgfqpoint{32.590674in}{2.343567in}}%
\pgfpathlineto{\pgfqpoint{32.520401in}{2.391312in}}%
\pgfpathlineto{\pgfqpoint{32.447439in}{2.373974in}}%
\pgfpathlineto{\pgfqpoint{32.377016in}{2.315904in}}%
\pgfpathlineto{\pgfqpoint{32.305440in}{2.296487in}}%
\pgfpathlineto{\pgfqpoint{32.231773in}{2.288985in}}%
\pgfpathlineto{\pgfqpoint{32.161504in}{2.355976in}}%
\pgfpathlineto{\pgfqpoint{32.089684in}{2.225068in}}%
\pgfpathlineto{\pgfqpoint{32.015122in}{2.327995in}}%
\pgfpathlineto{\pgfqpoint{31.943734in}{2.369077in}}%
\pgfpathlineto{\pgfqpoint{31.873539in}{2.331917in}}%
\pgfpathlineto{\pgfqpoint{31.800877in}{2.280462in}}%
\pgfpathlineto{\pgfqpoint{31.728521in}{2.251796in}}%
\pgfpathlineto{\pgfqpoint{31.656180in}{2.336838in}}%
\pgfpathlineto{\pgfqpoint{31.582282in}{2.268170in}}%
\pgfpathlineto{\pgfqpoint{31.510140in}{2.302230in}}%
\pgfpathlineto{\pgfqpoint{31.439415in}{2.321634in}}%
\pgfpathlineto{\pgfqpoint{31.366613in}{2.314323in}}%
\pgfpathlineto{\pgfqpoint{31.294777in}{2.280819in}}%
\pgfpathlineto{\pgfqpoint{31.222913in}{2.184436in}}%
\pgfpathlineto{\pgfqpoint{31.147740in}{2.263642in}}%
\pgfpathlineto{\pgfqpoint{31.074452in}{2.277852in}}%
\pgfpathlineto{\pgfqpoint{31.002514in}{2.291781in}}%
\pgfpathlineto{\pgfqpoint{30.928305in}{2.226503in}}%
\pgfpathlineto{\pgfqpoint{30.854115in}{2.345668in}}%
\pgfpathlineto{\pgfqpoint{30.782249in}{2.307196in}}%
\pgfpathlineto{\pgfqpoint{30.707828in}{2.262804in}}%
\pgfpathlineto{\pgfqpoint{30.634099in}{2.290693in}}%
\pgfpathlineto{\pgfqpoint{30.562714in}{2.342473in}}%
\pgfpathlineto{\pgfqpoint{30.488991in}{2.314096in}}%
\pgfpathlineto{\pgfqpoint{30.417098in}{2.297310in}}%
\pgfpathlineto{\pgfqpoint{30.344328in}{2.233525in}}%
\pgfpathlineto{\pgfqpoint{30.269036in}{2.240572in}}%
\pgfpathlineto{\pgfqpoint{30.195443in}{2.267018in}}%
\pgfpathlineto{\pgfqpoint{30.122796in}{2.302823in}}%
\pgfpathlineto{\pgfqpoint{30.048252in}{2.233145in}}%
\pgfpathlineto{\pgfqpoint{29.974646in}{2.294793in}}%
\pgfpathlineto{\pgfqpoint{29.901656in}{2.260473in}}%
\pgfpathlineto{\pgfqpoint{29.827132in}{2.325966in}}%
\pgfpathlineto{\pgfqpoint{29.755113in}{2.339174in}}%
\pgfpathlineto{\pgfqpoint{29.684427in}{2.314700in}}%
\pgfpathlineto{\pgfqpoint{29.611700in}{2.298618in}}%
\pgfpathlineto{\pgfqpoint{29.540420in}{2.308092in}}%
\pgfpathlineto{\pgfqpoint{29.467898in}{2.225859in}}%
\pgfpathlineto{\pgfqpoint{29.393660in}{2.293684in}}%
\pgfpathlineto{\pgfqpoint{29.320759in}{2.212543in}}%
\pgfpathlineto{\pgfqpoint{29.248973in}{2.266146in}}%
\pgfpathlineto{\pgfqpoint{29.173201in}{2.264222in}}%
\pgfpathlineto{\pgfqpoint{29.100748in}{2.328953in}}%
\pgfpathlineto{\pgfqpoint{29.029887in}{2.351888in}}%
\pgfpathlineto{\pgfqpoint{28.956832in}{2.301102in}}%
\pgfpathlineto{\pgfqpoint{28.885217in}{2.243646in}}%
\pgfpathlineto{\pgfqpoint{28.811911in}{2.244114in}}%
\pgfpathlineto{\pgfqpoint{28.738605in}{2.325439in}}%
\pgfpathlineto{\pgfqpoint{28.668204in}{2.281621in}}%
\pgfpathlineto{\pgfqpoint{28.596041in}{2.249022in}}%
\pgfpathlineto{\pgfqpoint{28.522534in}{2.371326in}}%
\pgfpathlineto{\pgfqpoint{28.451943in}{2.210001in}}%
\pgfpathlineto{\pgfqpoint{28.380501in}{2.251101in}}%
\pgfpathlineto{\pgfqpoint{28.306604in}{2.253161in}}%
\pgfpathlineto{\pgfqpoint{28.234559in}{2.300486in}}%
\pgfpathlineto{\pgfqpoint{28.163352in}{2.334604in}}%
\pgfpathlineto{\pgfqpoint{28.090360in}{2.334452in}}%
\pgfpathlineto{\pgfqpoint{28.018234in}{2.221174in}}%
\pgfpathlineto{\pgfqpoint{27.943911in}{2.189532in}}%
\pgfpathlineto{\pgfqpoint{27.868117in}{2.204306in}}%
\pgfpathlineto{\pgfqpoint{27.795343in}{2.299883in}}%
\pgfpathlineto{\pgfqpoint{27.724006in}{2.281051in}}%
\pgfpathlineto{\pgfqpoint{27.649072in}{2.226320in}}%
\pgfpathlineto{\pgfqpoint{27.574929in}{2.248437in}}%
\pgfpathlineto{\pgfqpoint{27.500063in}{2.202434in}}%
\pgfpathlineto{\pgfqpoint{27.423884in}{2.307017in}}%
\pgfpathlineto{\pgfqpoint{27.350990in}{2.267292in}}%
\pgfpathlineto{\pgfqpoint{27.277488in}{2.267224in}}%
\pgfpathlineto{\pgfqpoint{27.201477in}{2.293648in}}%
\pgfpathlineto{\pgfqpoint{27.128948in}{2.266617in}}%
\pgfpathlineto{\pgfqpoint{27.056835in}{2.356686in}}%
\pgfpathlineto{\pgfqpoint{26.983641in}{2.312856in}}%
\pgfpathlineto{\pgfqpoint{26.912667in}{2.325656in}}%
\pgfpathlineto{\pgfqpoint{26.841234in}{2.274754in}}%
\pgfpathlineto{\pgfqpoint{26.769271in}{2.375722in}}%
\pgfpathlineto{\pgfqpoint{26.699584in}{2.336737in}}%
\pgfpathlineto{\pgfqpoint{26.629572in}{2.283473in}}%
\pgfpathlineto{\pgfqpoint{26.557235in}{2.412854in}}%
\pgfpathlineto{\pgfqpoint{26.487420in}{2.378132in}}%
\pgfpathlineto{\pgfqpoint{26.417231in}{2.324241in}}%
\pgfpathlineto{\pgfqpoint{26.344920in}{2.323634in}}%
\pgfpathlineto{\pgfqpoint{26.274267in}{2.288707in}}%
\pgfpathlineto{\pgfqpoint{26.203572in}{2.256552in}}%
\pgfpathlineto{\pgfqpoint{26.130346in}{2.300808in}}%
\pgfpathlineto{\pgfqpoint{26.058621in}{2.295589in}}%
\pgfpathlineto{\pgfqpoint{25.988588in}{2.408263in}}%
\pgfpathlineto{\pgfqpoint{25.916551in}{2.342600in}}%
\pgfpathlineto{\pgfqpoint{25.845667in}{2.389794in}}%
\pgfpathlineto{\pgfqpoint{25.775695in}{2.302233in}}%
\pgfpathlineto{\pgfqpoint{25.702341in}{2.259177in}}%
\pgfpathlineto{\pgfqpoint{25.631022in}{2.327916in}}%
\pgfpathlineto{\pgfqpoint{25.560110in}{2.283593in}}%
\pgfpathlineto{\pgfqpoint{25.487573in}{2.303464in}}%
\pgfpathlineto{\pgfqpoint{25.417047in}{2.378551in}}%
\pgfpathlineto{\pgfqpoint{25.347124in}{2.209151in}}%
\pgfpathlineto{\pgfqpoint{25.273349in}{2.342925in}}%
\pgfpathlineto{\pgfqpoint{25.203216in}{2.321418in}}%
\pgfpathlineto{\pgfqpoint{25.131109in}{2.237827in}}%
\pgfpathlineto{\pgfqpoint{25.055567in}{2.265861in}}%
\pgfpathlineto{\pgfqpoint{24.983525in}{2.296342in}}%
\pgfpathlineto{\pgfqpoint{24.911741in}{2.253604in}}%
\pgfpathlineto{\pgfqpoint{24.836992in}{2.301860in}}%
\pgfpathlineto{\pgfqpoint{24.764742in}{2.218280in}}%
\pgfpathlineto{\pgfqpoint{24.691660in}{2.296715in}}%
\pgfpathlineto{\pgfqpoint{24.618678in}{2.393918in}}%
\pgfpathlineto{\pgfqpoint{24.548639in}{2.286684in}}%
\pgfpathlineto{\pgfqpoint{24.476339in}{2.176033in}}%
\pgfpathlineto{\pgfqpoint{24.400328in}{2.260240in}}%
\pgfpathlineto{\pgfqpoint{24.329090in}{2.306604in}}%
\pgfpathlineto{\pgfqpoint{24.257307in}{2.255320in}}%
\pgfpathlineto{\pgfqpoint{24.183899in}{2.348892in}}%
\pgfpathlineto{\pgfqpoint{24.111784in}{2.210574in}}%
\pgfpathlineto{\pgfqpoint{24.039255in}{2.386143in}}%
\pgfpathlineto{\pgfqpoint{23.966722in}{2.214818in}}%
\pgfpathlineto{\pgfqpoint{23.895213in}{2.345307in}}%
\pgfpathlineto{\pgfqpoint{23.824951in}{2.348982in}}%
\pgfpathlineto{\pgfqpoint{23.752361in}{2.328198in}}%
\pgfpathlineto{\pgfqpoint{23.681448in}{2.241376in}}%
\pgfpathlineto{\pgfqpoint{23.610036in}{2.363105in}}%
\pgfpathlineto{\pgfqpoint{23.537850in}{2.319420in}}%
\pgfpathlineto{\pgfqpoint{23.467323in}{2.253040in}}%
\pgfpathlineto{\pgfqpoint{23.396126in}{2.315437in}}%
\pgfpathlineto{\pgfqpoint{23.323995in}{2.269897in}}%
\pgfpathlineto{\pgfqpoint{23.253484in}{2.376160in}}%
\pgfpathlineto{\pgfqpoint{23.184270in}{2.386374in}}%
\pgfpathlineto{\pgfqpoint{23.113267in}{2.326259in}}%
\pgfpathlineto{\pgfqpoint{23.043397in}{2.293230in}}%
\pgfpathlineto{\pgfqpoint{22.973868in}{2.389573in}}%
\pgfpathlineto{\pgfqpoint{22.902896in}{2.397494in}}%
\pgfpathlineto{\pgfqpoint{22.833628in}{2.421459in}}%
\pgfpathlineto{\pgfqpoint{22.764651in}{2.482801in}}%
\pgfpathlineto{\pgfqpoint{22.694016in}{2.395373in}}%
\pgfpathlineto{\pgfqpoint{22.624114in}{2.348659in}}%
\pgfpathlineto{\pgfqpoint{22.553535in}{2.323391in}}%
\pgfpathlineto{\pgfqpoint{22.482516in}{2.398083in}}%
\pgfpathlineto{\pgfqpoint{22.413449in}{2.303340in}}%
\pgfpathlineto{\pgfqpoint{22.343331in}{2.348390in}}%
\pgfpathlineto{\pgfqpoint{22.268824in}{2.223836in}}%
\pgfpathlineto{\pgfqpoint{22.195707in}{2.170512in}}%
\pgfpathlineto{\pgfqpoint{22.122462in}{2.278881in}}%
\pgfpathlineto{\pgfqpoint{22.048040in}{2.330770in}}%
\pgfpathlineto{\pgfqpoint{21.976836in}{2.197014in}}%
\pgfpathlineto{\pgfqpoint{21.903868in}{2.291199in}}%
\pgfpathlineto{\pgfqpoint{21.828555in}{2.291422in}}%
\pgfpathlineto{\pgfqpoint{21.756227in}{2.259136in}}%
\pgfpathlineto{\pgfqpoint{21.683994in}{2.339697in}}%
\pgfpathlineto{\pgfqpoint{21.610878in}{2.303436in}}%
\pgfpathlineto{\pgfqpoint{21.539601in}{2.268173in}}%
\pgfpathlineto{\pgfqpoint{21.467741in}{2.288976in}}%
\pgfpathlineto{\pgfqpoint{21.394839in}{2.299263in}}%
\pgfpathlineto{\pgfqpoint{21.324498in}{2.355684in}}%
\pgfpathlineto{\pgfqpoint{21.254645in}{2.304203in}}%
\pgfpathlineto{\pgfqpoint{21.181233in}{2.235383in}}%
\pgfpathlineto{\pgfqpoint{21.110656in}{2.319899in}}%
\pgfpathlineto{\pgfqpoint{21.040864in}{2.379865in}}%
\pgfpathlineto{\pgfqpoint{20.969373in}{2.236644in}}%
\pgfpathlineto{\pgfqpoint{20.899587in}{2.361138in}}%
\pgfpathlineto{\pgfqpoint{20.830586in}{2.379466in}}%
\pgfpathlineto{\pgfqpoint{20.758814in}{2.293630in}}%
\pgfpathlineto{\pgfqpoint{20.688980in}{2.326817in}}%
\pgfpathlineto{\pgfqpoint{20.618428in}{2.387675in}}%
\pgfpathlineto{\pgfqpoint{20.547172in}{2.261538in}}%
\pgfpathlineto{\pgfqpoint{20.476961in}{2.308828in}}%
\pgfpathlineto{\pgfqpoint{20.407276in}{2.336694in}}%
\pgfpathlineto{\pgfqpoint{20.334605in}{2.289116in}}%
\pgfpathlineto{\pgfqpoint{20.263963in}{2.361737in}}%
\pgfpathlineto{\pgfqpoint{20.193561in}{2.294331in}}%
\pgfpathlineto{\pgfqpoint{20.121885in}{2.382111in}}%
\pgfpathlineto{\pgfqpoint{20.052071in}{2.291057in}}%
\pgfpathlineto{\pgfqpoint{19.981199in}{2.286178in}}%
\pgfpathlineto{\pgfqpoint{19.907049in}{2.285374in}}%
\pgfpathlineto{\pgfqpoint{19.835928in}{2.389725in}}%
\pgfpathlineto{\pgfqpoint{19.766060in}{2.363164in}}%
\pgfpathlineto{\pgfqpoint{19.693765in}{2.277444in}}%
\pgfpathlineto{\pgfqpoint{19.623855in}{2.419361in}}%
\pgfpathlineto{\pgfqpoint{19.553641in}{2.351876in}}%
\pgfpathlineto{\pgfqpoint{19.482791in}{2.360552in}}%
\pgfpathlineto{\pgfqpoint{19.412352in}{2.283026in}}%
\pgfpathlineto{\pgfqpoint{19.340700in}{2.296885in}}%
\pgfpathlineto{\pgfqpoint{19.266961in}{2.284268in}}%
\pgfpathlineto{\pgfqpoint{19.195134in}{2.289248in}}%
\pgfpathlineto{\pgfqpoint{19.123240in}{2.232984in}}%
\pgfpathlineto{\pgfqpoint{19.048835in}{2.271599in}}%
\pgfpathlineto{\pgfqpoint{18.977585in}{2.266246in}}%
\pgfpathlineto{\pgfqpoint{18.906958in}{2.330027in}}%
\pgfpathlineto{\pgfqpoint{18.834141in}{2.284510in}}%
\pgfpathlineto{\pgfqpoint{18.763558in}{2.293835in}}%
\pgfpathlineto{\pgfqpoint{18.692839in}{2.416355in}}%
\pgfpathlineto{\pgfqpoint{18.619849in}{2.333533in}}%
\pgfpathlineto{\pgfqpoint{18.549509in}{2.394626in}}%
\pgfpathlineto{\pgfqpoint{18.479143in}{2.296630in}}%
\pgfpathlineto{\pgfqpoint{18.406899in}{2.321433in}}%
\pgfpathlineto{\pgfqpoint{18.336223in}{2.426745in}}%
\pgfpathlineto{\pgfqpoint{18.266780in}{2.377914in}}%
\pgfpathlineto{\pgfqpoint{18.195268in}{2.353926in}}%
\pgfpathlineto{\pgfqpoint{18.125960in}{2.341081in}}%
\pgfpathlineto{\pgfqpoint{18.056086in}{2.293209in}}%
\pgfpathlineto{\pgfqpoint{17.983524in}{2.245730in}}%
\pgfpathlineto{\pgfqpoint{17.912627in}{2.376504in}}%
\pgfpathlineto{\pgfqpoint{17.844148in}{2.355139in}}%
\pgfpathlineto{\pgfqpoint{17.772898in}{2.315892in}}%
\pgfpathlineto{\pgfqpoint{17.703480in}{2.375946in}}%
\pgfpathlineto{\pgfqpoint{17.634782in}{2.313263in}}%
\pgfpathlineto{\pgfqpoint{17.563254in}{2.352403in}}%
\pgfpathlineto{\pgfqpoint{17.494859in}{2.295254in}}%
\pgfpathlineto{\pgfqpoint{17.426570in}{2.414408in}}%
\pgfpathlineto{\pgfqpoint{17.356523in}{2.344504in}}%
\pgfpathlineto{\pgfqpoint{17.288800in}{2.420723in}}%
\pgfpathlineto{\pgfqpoint{17.220960in}{2.310083in}}%
\pgfpathlineto{\pgfqpoint{17.150280in}{2.317372in}}%
\pgfpathlineto{\pgfqpoint{17.081113in}{2.450638in}}%
\pgfpathlineto{\pgfqpoint{17.013249in}{2.402088in}}%
\pgfpathlineto{\pgfqpoint{16.942299in}{2.319482in}}%
\pgfpathlineto{\pgfqpoint{16.872974in}{2.315337in}}%
\pgfpathlineto{\pgfqpoint{16.803402in}{2.334003in}}%
\pgfpathlineto{\pgfqpoint{16.731829in}{2.347150in}}%
\pgfpathlineto{\pgfqpoint{16.662798in}{2.362100in}}%
\pgfpathlineto{\pgfqpoint{16.592371in}{2.408326in}}%
\pgfpathlineto{\pgfqpoint{16.519393in}{2.237324in}}%
\pgfpathlineto{\pgfqpoint{16.445981in}{2.213369in}}%
\pgfpathlineto{\pgfqpoint{16.373054in}{2.296087in}}%
\pgfpathlineto{\pgfqpoint{16.299236in}{2.346079in}}%
\pgfpathlineto{\pgfqpoint{16.229236in}{2.416211in}}%
\pgfpathlineto{\pgfqpoint{16.160188in}{2.441511in}}%
\pgfpathlineto{\pgfqpoint{16.088678in}{2.378177in}}%
\pgfpathlineto{\pgfqpoint{16.019598in}{2.336319in}}%
\pgfpathlineto{\pgfqpoint{15.949677in}{2.389314in}}%
\pgfpathlineto{\pgfqpoint{15.878678in}{2.309932in}}%
\pgfpathlineto{\pgfqpoint{15.807467in}{2.316651in}}%
\pgfpathlineto{\pgfqpoint{15.737902in}{2.339944in}}%
\pgfpathlineto{\pgfqpoint{15.665480in}{2.298817in}}%
\pgfpathlineto{\pgfqpoint{15.595878in}{2.387626in}}%
\pgfpathlineto{\pgfqpoint{15.527961in}{2.337079in}}%
\pgfpathlineto{\pgfqpoint{15.457342in}{2.315581in}}%
\pgfpathlineto{\pgfqpoint{15.388582in}{2.367244in}}%
\pgfpathlineto{\pgfqpoint{15.320168in}{2.386687in}}%
\pgfpathlineto{\pgfqpoint{15.249597in}{2.366103in}}%
\pgfpathlineto{\pgfqpoint{15.182512in}{2.425014in}}%
\pgfpathlineto{\pgfqpoint{15.115140in}{2.334152in}}%
\pgfpathlineto{\pgfqpoint{15.044439in}{2.362664in}}%
\pgfpathlineto{\pgfqpoint{14.976372in}{2.414390in}}%
\pgfpathlineto{\pgfqpoint{14.908199in}{2.269316in}}%
\pgfpathlineto{\pgfqpoint{14.836607in}{2.352508in}}%
\pgfpathlineto{\pgfqpoint{14.768225in}{2.407033in}}%
\pgfpathlineto{\pgfqpoint{14.700003in}{2.378815in}}%
\pgfpathlineto{\pgfqpoint{14.630523in}{2.406563in}}%
\pgfpathlineto{\pgfqpoint{14.562284in}{2.368556in}}%
\pgfpathlineto{\pgfqpoint{14.494110in}{2.382148in}}%
\pgfpathlineto{\pgfqpoint{14.424138in}{2.394395in}}%
\pgfpathlineto{\pgfqpoint{14.355979in}{2.324319in}}%
\pgfpathlineto{\pgfqpoint{14.286685in}{2.344426in}}%
\pgfpathlineto{\pgfqpoint{14.216102in}{2.399801in}}%
\pgfpathlineto{\pgfqpoint{14.149562in}{2.396819in}}%
\pgfpathlineto{\pgfqpoint{14.081564in}{2.427453in}}%
\pgfpathlineto{\pgfqpoint{14.012639in}{2.322965in}}%
\pgfpathlineto{\pgfqpoint{13.944783in}{2.405673in}}%
\pgfpathlineto{\pgfqpoint{13.876785in}{2.318313in}}%
\pgfpathlineto{\pgfqpoint{13.804228in}{2.364455in}}%
\pgfpathlineto{\pgfqpoint{13.735499in}{2.360084in}}%
\pgfpathlineto{\pgfqpoint{13.666880in}{2.298388in}}%
\pgfpathlineto{\pgfqpoint{13.595052in}{2.404581in}}%
\pgfpathlineto{\pgfqpoint{13.526360in}{2.400693in}}%
\pgfpathlineto{\pgfqpoint{13.457833in}{2.329708in}}%
\pgfpathlineto{\pgfqpoint{13.387091in}{2.389023in}}%
\pgfpathlineto{\pgfqpoint{13.319414in}{2.338118in}}%
\pgfpathlineto{\pgfqpoint{13.250526in}{2.313702in}}%
\pgfpathlineto{\pgfqpoint{13.178279in}{2.321556in}}%
\pgfpathlineto{\pgfqpoint{13.109710in}{2.363697in}}%
\pgfpathlineto{\pgfqpoint{13.041078in}{2.275523in}}%
\pgfpathlineto{\pgfqpoint{12.969036in}{2.316575in}}%
\pgfpathlineto{\pgfqpoint{12.900232in}{2.345595in}}%
\pgfpathlineto{\pgfqpoint{12.830499in}{2.350302in}}%
\pgfpathlineto{\pgfqpoint{12.759967in}{2.460769in}}%
\pgfpathlineto{\pgfqpoint{12.692271in}{2.394316in}}%
\pgfpathlineto{\pgfqpoint{12.623527in}{2.382185in}}%
\pgfpathlineto{\pgfqpoint{12.552597in}{2.424792in}}%
\pgfpathlineto{\pgfqpoint{12.484090in}{2.329892in}}%
\pgfpathlineto{\pgfqpoint{12.416448in}{2.381820in}}%
\pgfpathlineto{\pgfqpoint{12.347008in}{2.445924in}}%
\pgfpathlineto{\pgfqpoint{12.280761in}{2.454963in}}%
\pgfpathlineto{\pgfqpoint{12.213872in}{2.340878in}}%
\pgfpathlineto{\pgfqpoint{12.144815in}{2.287899in}}%
\pgfpathlineto{\pgfqpoint{12.076638in}{2.412443in}}%
\pgfpathlineto{\pgfqpoint{12.008843in}{2.309380in}}%
\pgfpathlineto{\pgfqpoint{11.938397in}{2.312124in}}%
\pgfpathlineto{\pgfqpoint{11.871239in}{2.412976in}}%
\pgfpathlineto{\pgfqpoint{11.804695in}{2.447359in}}%
\pgfpathlineto{\pgfqpoint{11.736702in}{2.399387in}}%
\pgfpathlineto{\pgfqpoint{11.669969in}{2.418078in}}%
\pgfpathlineto{\pgfqpoint{11.602395in}{2.427688in}}%
\pgfpathlineto{\pgfqpoint{11.532877in}{2.379878in}}%
\pgfpathlineto{\pgfqpoint{11.465294in}{2.465238in}}%
\pgfpathlineto{\pgfqpoint{11.398136in}{2.391542in}}%
\pgfpathlineto{\pgfqpoint{11.328677in}{2.440319in}}%
\pgfpathlineto{\pgfqpoint{11.261777in}{2.462632in}}%
\pgfpathlineto{\pgfqpoint{11.194984in}{2.396114in}}%
\pgfpathlineto{\pgfqpoint{11.125133in}{2.352777in}}%
\pgfpathlineto{\pgfqpoint{11.055389in}{2.377512in}}%
\pgfpathlineto{\pgfqpoint{10.986941in}{2.445029in}}%
\pgfpathlineto{\pgfqpoint{10.916008in}{2.355493in}}%
\pgfpathlineto{\pgfqpoint{10.846851in}{2.362275in}}%
\pgfpathlineto{\pgfqpoint{10.777925in}{2.465290in}}%
\pgfpathlineto{\pgfqpoint{10.708017in}{2.385366in}}%
\pgfpathlineto{\pgfqpoint{10.639006in}{2.392576in}}%
\pgfpathlineto{\pgfqpoint{10.569898in}{2.317130in}}%
\pgfpathlineto{\pgfqpoint{10.498528in}{2.385212in}}%
\pgfpathlineto{\pgfqpoint{10.428999in}{2.375232in}}%
\pgfpathlineto{\pgfqpoint{10.360853in}{2.401431in}}%
\pgfpathlineto{\pgfqpoint{10.290696in}{2.378244in}}%
\pgfpathlineto{\pgfqpoint{10.222102in}{2.459106in}}%
\pgfpathlineto{\pgfqpoint{10.153891in}{2.387850in}}%
\pgfpathlineto{\pgfqpoint{10.081602in}{2.316230in}}%
\pgfpathlineto{\pgfqpoint{10.012687in}{2.404361in}}%
\pgfpathlineto{\pgfqpoint{9.944615in}{2.524428in}}%
\pgfpathlineto{\pgfqpoint{9.876148in}{2.370141in}}%
\pgfpathlineto{\pgfqpoint{9.808815in}{2.428756in}}%
\pgfpathlineto{\pgfqpoint{9.740273in}{2.326536in}}%
\pgfpathlineto{\pgfqpoint{9.670768in}{2.451919in}}%
\pgfpathlineto{\pgfqpoint{9.603748in}{2.351056in}}%
\pgfpathlineto{\pgfqpoint{9.536145in}{2.401471in}}%
\pgfpathlineto{\pgfqpoint{9.467314in}{2.503416in}}%
\pgfpathlineto{\pgfqpoint{9.402138in}{2.483410in}}%
\pgfpathlineto{\pgfqpoint{9.334728in}{2.332717in}}%
\pgfpathlineto{\pgfqpoint{9.265262in}{2.435613in}}%
\pgfpathlineto{\pgfqpoint{9.198477in}{2.374259in}}%
\pgfpathlineto{\pgfqpoint{9.130518in}{2.342457in}}%
\pgfpathlineto{\pgfqpoint{9.059607in}{2.324022in}}%
\pgfpathlineto{\pgfqpoint{8.991646in}{2.437245in}}%
\pgfpathlineto{\pgfqpoint{8.924860in}{2.492165in}}%
\pgfpathlineto{\pgfqpoint{8.856098in}{2.355079in}}%
\pgfpathlineto{\pgfqpoint{8.787638in}{2.418926in}}%
\pgfpathlineto{\pgfqpoint{8.720615in}{2.493017in}}%
\pgfpathlineto{\pgfqpoint{8.651757in}{2.343993in}}%
\pgfpathlineto{\pgfqpoint{8.583601in}{2.369451in}}%
\pgfpathlineto{\pgfqpoint{8.515085in}{2.347538in}}%
\pgfpathlineto{\pgfqpoint{8.444285in}{2.407523in}}%
\pgfpathlineto{\pgfqpoint{8.376454in}{2.376527in}}%
\pgfpathlineto{\pgfqpoint{8.308718in}{2.415966in}}%
\pgfpathlineto{\pgfqpoint{8.237831in}{2.274930in}}%
\pgfpathlineto{\pgfqpoint{8.168934in}{2.339596in}}%
\pgfpathlineto{\pgfqpoint{8.100457in}{2.350531in}}%
\pgfpathlineto{\pgfqpoint{8.028258in}{2.304644in}}%
\pgfpathlineto{\pgfqpoint{7.957986in}{2.310678in}}%
\pgfpathlineto{\pgfqpoint{7.887441in}{2.276659in}}%
\pgfpathlineto{\pgfqpoint{7.812286in}{2.263590in}}%
\pgfpathlineto{\pgfqpoint{7.741421in}{2.387356in}}%
\pgfpathlineto{\pgfqpoint{7.671131in}{2.329237in}}%
\pgfpathlineto{\pgfqpoint{7.598690in}{2.329323in}}%
\pgfpathlineto{\pgfqpoint{7.527107in}{2.320950in}}%
\pgfpathlineto{\pgfqpoint{7.455306in}{2.253215in}}%
\pgfpathlineto{\pgfqpoint{7.379810in}{2.272963in}}%
\pgfpathlineto{\pgfqpoint{7.308295in}{2.256817in}}%
\pgfpathlineto{\pgfqpoint{7.238831in}{2.444538in}}%
\pgfpathlineto{\pgfqpoint{7.171034in}{2.517744in}}%
\pgfpathlineto{\pgfqpoint{7.106301in}{2.463778in}}%
\pgfpathlineto{\pgfqpoint{7.040805in}{2.524475in}}%
\pgfpathlineto{\pgfqpoint{6.973333in}{2.473005in}}%
\pgfpathlineto{\pgfqpoint{6.906544in}{2.423239in}}%
\pgfpathlineto{\pgfqpoint{6.839348in}{2.407795in}}%
\pgfpathlineto{\pgfqpoint{6.770581in}{2.413310in}}%
\pgfpathlineto{\pgfqpoint{6.704271in}{2.381557in}}%
\pgfpathlineto{\pgfqpoint{6.636218in}{2.349013in}}%
\pgfpathlineto{\pgfqpoint{6.568004in}{2.521477in}}%
\pgfpathlineto{\pgfqpoint{6.502885in}{2.471125in}}%
\pgfpathlineto{\pgfqpoint{6.437563in}{2.483194in}}%
\pgfpathlineto{\pgfqpoint{6.369842in}{2.468254in}}%
\pgfpathlineto{\pgfqpoint{6.303479in}{2.413452in}}%
\pgfpathlineto{\pgfqpoint{6.236861in}{2.491128in}}%
\pgfpathlineto{\pgfqpoint{6.169451in}{2.396602in}}%
\pgfpathlineto{\pgfqpoint{6.103260in}{2.347344in}}%
\pgfpathlineto{\pgfqpoint{6.036866in}{2.454770in}}%
\pgfpathlineto{\pgfqpoint{5.968494in}{2.415040in}}%
\pgfpathlineto{\pgfqpoint{5.901623in}{2.422676in}}%
\pgfpathlineto{\pgfqpoint{5.834669in}{2.515321in}}%
\pgfpathlineto{\pgfqpoint{5.766709in}{2.438440in}}%
\pgfpathlineto{\pgfqpoint{5.700467in}{2.424659in}}%
\pgfpathlineto{\pgfqpoint{5.633136in}{2.340892in}}%
\pgfpathlineto{\pgfqpoint{5.561464in}{2.344049in}}%
\pgfpathlineto{\pgfqpoint{5.492693in}{2.353124in}}%
\pgfpathlineto{\pgfqpoint{5.424203in}{2.396953in}}%
\pgfpathlineto{\pgfqpoint{5.353942in}{2.320139in}}%
\pgfpathlineto{\pgfqpoint{5.283251in}{2.365402in}}%
\pgfpathlineto{\pgfqpoint{5.213575in}{2.370042in}}%
\pgfpathlineto{\pgfqpoint{5.142206in}{2.299670in}}%
\pgfpathlineto{\pgfqpoint{5.073090in}{2.395244in}}%
\pgfpathlineto{\pgfqpoint{5.005143in}{2.392115in}}%
\pgfpathlineto{\pgfqpoint{4.935251in}{2.344606in}}%
\pgfpathlineto{\pgfqpoint{4.866311in}{2.310785in}}%
\pgfpathlineto{\pgfqpoint{4.796564in}{2.338156in}}%
\pgfpathlineto{\pgfqpoint{4.726503in}{2.472786in}}%
\pgfpathlineto{\pgfqpoint{4.658575in}{2.369430in}}%
\pgfpathlineto{\pgfqpoint{4.590503in}{2.366629in}}%
\pgfpathlineto{\pgfqpoint{4.520559in}{2.389366in}}%
\pgfpathlineto{\pgfqpoint{4.454265in}{2.420274in}}%
\pgfpathlineto{\pgfqpoint{4.387175in}{2.441472in}}%
\pgfpathlineto{\pgfqpoint{4.319299in}{2.440327in}}%
\pgfpathlineto{\pgfqpoint{4.253047in}{2.432371in}}%
\pgfpathlineto{\pgfqpoint{4.186150in}{2.363470in}}%
\pgfpathlineto{\pgfqpoint{4.116566in}{2.371017in}}%
\pgfpathlineto{\pgfqpoint{4.048987in}{2.276365in}}%
\pgfpathlineto{\pgfqpoint{3.981945in}{2.406194in}}%
\pgfpathlineto{\pgfqpoint{3.914312in}{2.500852in}}%
\pgfpathlineto{\pgfqpoint{3.848285in}{2.400018in}}%
\pgfpathlineto{\pgfqpoint{3.781205in}{2.459129in}}%
\pgfpathlineto{\pgfqpoint{3.712535in}{2.399915in}}%
\pgfpathlineto{\pgfqpoint{3.643958in}{2.356364in}}%
\pgfpathlineto{\pgfqpoint{3.575543in}{2.415234in}}%
\pgfpathlineto{\pgfqpoint{3.503039in}{2.293216in}}%
\pgfpathlineto{\pgfqpoint{3.430573in}{2.273638in}}%
\pgfpathlineto{\pgfqpoint{3.356465in}{2.204051in}}%
\pgfpathlineto{\pgfqpoint{3.277599in}{2.117445in}}%
\pgfpathlineto{\pgfqpoint{3.195433in}{2.282530in}}%
\pgfpathlineto{\pgfqpoint{3.123114in}{2.295769in}}%
\pgfpathlineto{\pgfqpoint{3.047640in}{2.262722in}}%
\pgfpathlineto{\pgfqpoint{2.975015in}{2.336790in}}%
\pgfpathlineto{\pgfqpoint{2.902674in}{2.185241in}}%
\pgfpathlineto{\pgfqpoint{2.826501in}{2.273706in}}%
\pgfpathlineto{\pgfqpoint{2.753491in}{2.296750in}}%
\pgfpathlineto{\pgfqpoint{2.681331in}{2.252813in}}%
\pgfpathlineto{\pgfqpoint{2.606023in}{2.286411in}}%
\pgfpathlineto{\pgfqpoint{2.534117in}{2.313522in}}%
\pgfpathlineto{\pgfqpoint{2.462769in}{2.298530in}}%
\pgfpathlineto{\pgfqpoint{2.387948in}{2.273118in}}%
\pgfpathlineto{\pgfqpoint{2.317152in}{2.357001in}}%
\pgfpathlineto{\pgfqpoint{2.248172in}{2.352023in}}%
\pgfpathlineto{\pgfqpoint{2.177337in}{2.994942in}}%
\pgfpathlineto{\pgfqpoint{2.108883in}{3.199197in}}%
\pgfpathlineto{\pgfqpoint{2.041179in}{3.226152in}}%
\pgfpathlineto{\pgfqpoint{1.970951in}{3.206483in}}%
\pgfpathlineto{\pgfqpoint{1.904458in}{3.206165in}}%
\pgfpathlineto{\pgfqpoint{1.835890in}{3.194228in}}%
\pgfpathlineto{\pgfqpoint{1.766402in}{3.210954in}}%
\pgfpathlineto{\pgfqpoint{1.698662in}{3.139991in}}%
\pgfpathlineto{\pgfqpoint{1.628862in}{3.080656in}}%
\pgfpathlineto{\pgfqpoint{1.557461in}{3.128937in}}%
\pgfpathlineto{\pgfqpoint{1.489306in}{3.199521in}}%
\pgfpathlineto{\pgfqpoint{1.421095in}{3.112266in}}%
\pgfpathlineto{\pgfqpoint{1.349373in}{3.185522in}}%
\pgfpathlineto{\pgfqpoint{1.283036in}{3.179317in}}%
\pgfpathlineto{\pgfqpoint{1.216322in}{3.189818in}}%
\pgfpathlineto{\pgfqpoint{1.147369in}{3.146493in}}%
\pgfpathlineto{\pgfqpoint{1.079942in}{3.268734in}}%
\pgfpathlineto{\pgfqpoint{1.012853in}{3.112906in}}%
\pgfpathlineto{\pgfqpoint{0.942110in}{3.203568in}}%
\pgfpathlineto{\pgfqpoint{0.875335in}{3.217388in}}%
\pgfpathlineto{\pgfqpoint{0.807094in}{2.186873in}}%
\pgfpathclose%
\pgfusepath{fill}%
\end{pgfscope}%
\begin{pgfscope}%
\pgfsetbuttcap%
\pgfsetroundjoin%
\definecolor{currentfill}{rgb}{0.000000,0.000000,0.000000}%
\pgfsetfillcolor{currentfill}%
\pgfsetlinewidth{0.803000pt}%
\definecolor{currentstroke}{rgb}{0.000000,0.000000,0.000000}%
\pgfsetstrokecolor{currentstroke}%
\pgfsetdash{}{0pt}%
\pgfsys@defobject{currentmarker}{\pgfqpoint{0.000000in}{-0.048611in}}{\pgfqpoint{0.000000in}{0.000000in}}{%
\pgfpathmoveto{\pgfqpoint{0.000000in}{0.000000in}}%
\pgfpathlineto{\pgfqpoint{0.000000in}{-0.048611in}}%
\pgfusepath{stroke,fill}%
}%
\begin{pgfscope}%
\pgfsys@transformshift{0.781402in}{0.773588in}%
\pgfsys@useobject{currentmarker}{}%
\end{pgfscope}%
\end{pgfscope}%
\begin{pgfscope}%
\definecolor{textcolor}{rgb}{0.000000,0.000000,0.000000}%
\pgfsetstrokecolor{textcolor}%
\pgfsetfillcolor{textcolor}%
\pgftext[x=0.781402in,y=0.676366in,,top]{\color{textcolor}\rmfamily\fontsize{10.000000}{12.000000}\selectfont \(\displaystyle {0}\)}%
\end{pgfscope}%
\begin{pgfscope}%
\pgfsetbuttcap%
\pgfsetroundjoin%
\definecolor{currentfill}{rgb}{0.000000,0.000000,0.000000}%
\pgfsetfillcolor{currentfill}%
\pgfsetlinewidth{0.803000pt}%
\definecolor{currentstroke}{rgb}{0.000000,0.000000,0.000000}%
\pgfsetstrokecolor{currentstroke}%
\pgfsetdash{}{0pt}%
\pgfsys@defobject{currentmarker}{\pgfqpoint{0.000000in}{-0.048611in}}{\pgfqpoint{0.000000in}{0.000000in}}{%
\pgfpathmoveto{\pgfqpoint{0.000000in}{0.000000in}}%
\pgfpathlineto{\pgfqpoint{0.000000in}{-0.048611in}}%
\pgfusepath{stroke,fill}%
}%
\begin{pgfscope}%
\pgfsys@transformshift{1.482344in}{0.773588in}%
\pgfsys@useobject{currentmarker}{}%
\end{pgfscope}%
\end{pgfscope}%
\begin{pgfscope}%
\definecolor{textcolor}{rgb}{0.000000,0.000000,0.000000}%
\pgfsetstrokecolor{textcolor}%
\pgfsetfillcolor{textcolor}%
\pgftext[x=1.482344in,y=0.676366in,,top]{\color{textcolor}\rmfamily\fontsize{10.000000}{12.000000}\selectfont \(\displaystyle {50}\)}%
\end{pgfscope}%
\begin{pgfscope}%
\pgfsetbuttcap%
\pgfsetroundjoin%
\definecolor{currentfill}{rgb}{0.000000,0.000000,0.000000}%
\pgfsetfillcolor{currentfill}%
\pgfsetlinewidth{0.803000pt}%
\definecolor{currentstroke}{rgb}{0.000000,0.000000,0.000000}%
\pgfsetstrokecolor{currentstroke}%
\pgfsetdash{}{0pt}%
\pgfsys@defobject{currentmarker}{\pgfqpoint{0.000000in}{-0.048611in}}{\pgfqpoint{0.000000in}{0.000000in}}{%
\pgfpathmoveto{\pgfqpoint{0.000000in}{0.000000in}}%
\pgfpathlineto{\pgfqpoint{0.000000in}{-0.048611in}}%
\pgfusepath{stroke,fill}%
}%
\begin{pgfscope}%
\pgfsys@transformshift{2.183286in}{0.773588in}%
\pgfsys@useobject{currentmarker}{}%
\end{pgfscope}%
\end{pgfscope}%
\begin{pgfscope}%
\definecolor{textcolor}{rgb}{0.000000,0.000000,0.000000}%
\pgfsetstrokecolor{textcolor}%
\pgfsetfillcolor{textcolor}%
\pgftext[x=2.183286in,y=0.676366in,,top]{\color{textcolor}\rmfamily\fontsize{10.000000}{12.000000}\selectfont \(\displaystyle {100}\)}%
\end{pgfscope}%
\begin{pgfscope}%
\pgfsetbuttcap%
\pgfsetroundjoin%
\definecolor{currentfill}{rgb}{0.000000,0.000000,0.000000}%
\pgfsetfillcolor{currentfill}%
\pgfsetlinewidth{0.803000pt}%
\definecolor{currentstroke}{rgb}{0.000000,0.000000,0.000000}%
\pgfsetstrokecolor{currentstroke}%
\pgfsetdash{}{0pt}%
\pgfsys@defobject{currentmarker}{\pgfqpoint{0.000000in}{-0.048611in}}{\pgfqpoint{0.000000in}{0.000000in}}{%
\pgfpathmoveto{\pgfqpoint{0.000000in}{0.000000in}}%
\pgfpathlineto{\pgfqpoint{0.000000in}{-0.048611in}}%
\pgfusepath{stroke,fill}%
}%
\begin{pgfscope}%
\pgfsys@transformshift{2.884228in}{0.773588in}%
\pgfsys@useobject{currentmarker}{}%
\end{pgfscope}%
\end{pgfscope}%
\begin{pgfscope}%
\definecolor{textcolor}{rgb}{0.000000,0.000000,0.000000}%
\pgfsetstrokecolor{textcolor}%
\pgfsetfillcolor{textcolor}%
\pgftext[x=2.884228in,y=0.676366in,,top]{\color{textcolor}\rmfamily\fontsize{10.000000}{12.000000}\selectfont \(\displaystyle {150}\)}%
\end{pgfscope}%
\begin{pgfscope}%
\pgfsetbuttcap%
\pgfsetroundjoin%
\definecolor{currentfill}{rgb}{0.000000,0.000000,0.000000}%
\pgfsetfillcolor{currentfill}%
\pgfsetlinewidth{0.803000pt}%
\definecolor{currentstroke}{rgb}{0.000000,0.000000,0.000000}%
\pgfsetstrokecolor{currentstroke}%
\pgfsetdash{}{0pt}%
\pgfsys@defobject{currentmarker}{\pgfqpoint{-0.048611in}{0.000000in}}{\pgfqpoint{-0.000000in}{0.000000in}}{%
\pgfpathmoveto{\pgfqpoint{-0.000000in}{0.000000in}}%
\pgfpathlineto{\pgfqpoint{-0.048611in}{0.000000in}}%
\pgfusepath{stroke,fill}%
}%
\begin{pgfscope}%
\pgfsys@transformshift{0.781402in}{0.773588in}%
\pgfsys@useobject{currentmarker}{}%
\end{pgfscope}%
\end{pgfscope}%
\begin{pgfscope}%
\definecolor{textcolor}{rgb}{0.000000,0.000000,0.000000}%
\pgfsetstrokecolor{textcolor}%
\pgfsetfillcolor{textcolor}%
\pgftext[x=0.506711in, y=0.725363in, left, base]{\color{textcolor}\rmfamily\fontsize{10.000000}{12.000000}\selectfont \(\displaystyle {0.0}\)}%
\end{pgfscope}%
\begin{pgfscope}%
\pgfsetbuttcap%
\pgfsetroundjoin%
\definecolor{currentfill}{rgb}{0.000000,0.000000,0.000000}%
\pgfsetfillcolor{currentfill}%
\pgfsetlinewidth{0.803000pt}%
\definecolor{currentstroke}{rgb}{0.000000,0.000000,0.000000}%
\pgfsetstrokecolor{currentstroke}%
\pgfsetdash{}{0pt}%
\pgfsys@defobject{currentmarker}{\pgfqpoint{-0.048611in}{0.000000in}}{\pgfqpoint{-0.000000in}{0.000000in}}{%
\pgfpathmoveto{\pgfqpoint{-0.000000in}{0.000000in}}%
\pgfpathlineto{\pgfqpoint{-0.048611in}{0.000000in}}%
\pgfusepath{stroke,fill}%
}%
\begin{pgfscope}%
\pgfsys@transformshift{0.781402in}{1.480442in}%
\pgfsys@useobject{currentmarker}{}%
\end{pgfscope}%
\end{pgfscope}%
\begin{pgfscope}%
\definecolor{textcolor}{rgb}{0.000000,0.000000,0.000000}%
\pgfsetstrokecolor{textcolor}%
\pgfsetfillcolor{textcolor}%
\pgftext[x=0.506711in, y=1.432216in, left, base]{\color{textcolor}\rmfamily\fontsize{10.000000}{12.000000}\selectfont \(\displaystyle {0.1}\)}%
\end{pgfscope}%
\begin{pgfscope}%
\pgfsetbuttcap%
\pgfsetroundjoin%
\definecolor{currentfill}{rgb}{0.000000,0.000000,0.000000}%
\pgfsetfillcolor{currentfill}%
\pgfsetlinewidth{0.803000pt}%
\definecolor{currentstroke}{rgb}{0.000000,0.000000,0.000000}%
\pgfsetstrokecolor{currentstroke}%
\pgfsetdash{}{0pt}%
\pgfsys@defobject{currentmarker}{\pgfqpoint{-0.048611in}{0.000000in}}{\pgfqpoint{-0.000000in}{0.000000in}}{%
\pgfpathmoveto{\pgfqpoint{-0.000000in}{0.000000in}}%
\pgfpathlineto{\pgfqpoint{-0.048611in}{0.000000in}}%
\pgfusepath{stroke,fill}%
}%
\begin{pgfscope}%
\pgfsys@transformshift{0.781402in}{2.187295in}%
\pgfsys@useobject{currentmarker}{}%
\end{pgfscope}%
\end{pgfscope}%
\begin{pgfscope}%
\definecolor{textcolor}{rgb}{0.000000,0.000000,0.000000}%
\pgfsetstrokecolor{textcolor}%
\pgfsetfillcolor{textcolor}%
\pgftext[x=0.506711in, y=2.139070in, left, base]{\color{textcolor}\rmfamily\fontsize{10.000000}{12.000000}\selectfont \(\displaystyle {0.2}\)}%
\end{pgfscope}%
\begin{pgfscope}%
\pgfsetbuttcap%
\pgfsetroundjoin%
\definecolor{currentfill}{rgb}{0.000000,0.000000,0.000000}%
\pgfsetfillcolor{currentfill}%
\pgfsetlinewidth{0.803000pt}%
\definecolor{currentstroke}{rgb}{0.000000,0.000000,0.000000}%
\pgfsetstrokecolor{currentstroke}%
\pgfsetdash{}{0pt}%
\pgfsys@defobject{currentmarker}{\pgfqpoint{-0.048611in}{0.000000in}}{\pgfqpoint{-0.000000in}{0.000000in}}{%
\pgfpathmoveto{\pgfqpoint{-0.000000in}{0.000000in}}%
\pgfpathlineto{\pgfqpoint{-0.048611in}{0.000000in}}%
\pgfusepath{stroke,fill}%
}%
\begin{pgfscope}%
\pgfsys@transformshift{0.781402in}{2.894148in}%
\pgfsys@useobject{currentmarker}{}%
\end{pgfscope}%
\end{pgfscope}%
\begin{pgfscope}%
\definecolor{textcolor}{rgb}{0.000000,0.000000,0.000000}%
\pgfsetstrokecolor{textcolor}%
\pgfsetfillcolor{textcolor}%
\pgftext[x=0.506711in, y=2.845923in, left, base]{\color{textcolor}\rmfamily\fontsize{10.000000}{12.000000}\selectfont \(\displaystyle {0.3}\)}%
\end{pgfscope}%
\begin{pgfscope}%
\pgfsetbuttcap%
\pgfsetroundjoin%
\definecolor{currentfill}{rgb}{0.000000,0.000000,0.000000}%
\pgfsetfillcolor{currentfill}%
\pgfsetlinewidth{0.803000pt}%
\definecolor{currentstroke}{rgb}{0.000000,0.000000,0.000000}%
\pgfsetstrokecolor{currentstroke}%
\pgfsetdash{}{0pt}%
\pgfsys@defobject{currentmarker}{\pgfqpoint{-0.048611in}{0.000000in}}{\pgfqpoint{-0.000000in}{0.000000in}}{%
\pgfpathmoveto{\pgfqpoint{-0.000000in}{0.000000in}}%
\pgfpathlineto{\pgfqpoint{-0.048611in}{0.000000in}}%
\pgfusepath{stroke,fill}%
}%
\begin{pgfscope}%
\pgfsys@transformshift{0.781402in}{3.601002in}%
\pgfsys@useobject{currentmarker}{}%
\end{pgfscope}%
\end{pgfscope}%
\begin{pgfscope}%
\definecolor{textcolor}{rgb}{0.000000,0.000000,0.000000}%
\pgfsetstrokecolor{textcolor}%
\pgfsetfillcolor{textcolor}%
\pgftext[x=0.506711in, y=3.552776in, left, base]{\color{textcolor}\rmfamily\fontsize{10.000000}{12.000000}\selectfont \(\displaystyle {0.4}\)}%
\end{pgfscope}%
\begin{pgfscope}%
\pgfsetbuttcap%
\pgfsetroundjoin%
\definecolor{currentfill}{rgb}{0.000000,0.000000,0.000000}%
\pgfsetfillcolor{currentfill}%
\pgfsetlinewidth{0.803000pt}%
\definecolor{currentstroke}{rgb}{0.000000,0.000000,0.000000}%
\pgfsetstrokecolor{currentstroke}%
\pgfsetdash{}{0pt}%
\pgfsys@defobject{currentmarker}{\pgfqpoint{-0.048611in}{0.000000in}}{\pgfqpoint{-0.000000in}{0.000000in}}{%
\pgfpathmoveto{\pgfqpoint{-0.000000in}{0.000000in}}%
\pgfpathlineto{\pgfqpoint{-0.048611in}{0.000000in}}%
\pgfusepath{stroke,fill}%
}%
\begin{pgfscope}%
\pgfsys@transformshift{0.781402in}{4.307855in}%
\pgfsys@useobject{currentmarker}{}%
\end{pgfscope}%
\end{pgfscope}%
\begin{pgfscope}%
\definecolor{textcolor}{rgb}{0.000000,0.000000,0.000000}%
\pgfsetstrokecolor{textcolor}%
\pgfsetfillcolor{textcolor}%
\pgftext[x=0.506711in, y=4.259630in, left, base]{\color{textcolor}\rmfamily\fontsize{10.000000}{12.000000}\selectfont \(\displaystyle {0.5}\)}%
\end{pgfscope}%
\begin{pgfscope}%
\pgfsetbuttcap%
\pgfsetroundjoin%
\definecolor{currentfill}{rgb}{0.000000,0.000000,0.000000}%
\pgfsetfillcolor{currentfill}%
\pgfsetlinewidth{0.803000pt}%
\definecolor{currentstroke}{rgb}{0.000000,0.000000,0.000000}%
\pgfsetstrokecolor{currentstroke}%
\pgfsetdash{}{0pt}%
\pgfsys@defobject{currentmarker}{\pgfqpoint{-0.048611in}{0.000000in}}{\pgfqpoint{-0.000000in}{0.000000in}}{%
\pgfpathmoveto{\pgfqpoint{-0.000000in}{0.000000in}}%
\pgfpathlineto{\pgfqpoint{-0.048611in}{0.000000in}}%
\pgfusepath{stroke,fill}%
}%
\begin{pgfscope}%
\pgfsys@transformshift{0.781402in}{5.014708in}%
\pgfsys@useobject{currentmarker}{}%
\end{pgfscope}%
\end{pgfscope}%
\begin{pgfscope}%
\definecolor{textcolor}{rgb}{0.000000,0.000000,0.000000}%
\pgfsetstrokecolor{textcolor}%
\pgfsetfillcolor{textcolor}%
\pgftext[x=0.506711in, y=4.966483in, left, base]{\color{textcolor}\rmfamily\fontsize{10.000000}{12.000000}\selectfont \(\displaystyle {0.6}\)}%
\end{pgfscope}%
\begin{pgfscope}%
\pgfsetbuttcap%
\pgfsetroundjoin%
\definecolor{currentfill}{rgb}{0.000000,0.000000,0.000000}%
\pgfsetfillcolor{currentfill}%
\pgfsetlinewidth{0.803000pt}%
\definecolor{currentstroke}{rgb}{0.000000,0.000000,0.000000}%
\pgfsetstrokecolor{currentstroke}%
\pgfsetdash{}{0pt}%
\pgfsys@defobject{currentmarker}{\pgfqpoint{-0.048611in}{0.000000in}}{\pgfqpoint{-0.000000in}{0.000000in}}{%
\pgfpathmoveto{\pgfqpoint{-0.000000in}{0.000000in}}%
\pgfpathlineto{\pgfqpoint{-0.048611in}{0.000000in}}%
\pgfusepath{stroke,fill}%
}%
\begin{pgfscope}%
\pgfsys@transformshift{0.781402in}{5.721562in}%
\pgfsys@useobject{currentmarker}{}%
\end{pgfscope}%
\end{pgfscope}%
\begin{pgfscope}%
\definecolor{textcolor}{rgb}{0.000000,0.000000,0.000000}%
\pgfsetstrokecolor{textcolor}%
\pgfsetfillcolor{textcolor}%
\pgftext[x=0.506711in, y=5.673336in, left, base]{\color{textcolor}\rmfamily\fontsize{10.000000}{12.000000}\selectfont \(\displaystyle {0.7}\)}%
\end{pgfscope}%
\begin{pgfscope}%
\pgfsetrectcap%
\pgfsetroundjoin%
\pgfsetlinewidth{0.803000pt}%
\definecolor{currentstroke}{rgb}{0.000000,0.000000,0.000000}%
\pgfsetstrokecolor{currentstroke}%
\pgfsetdash{}{0pt}%
\pgfpathmoveto{\pgfqpoint{2.825449in}{0.707284in}}%
\pgfpathlineto{\pgfqpoint{2.958057in}{0.839893in}}%
\pgfusepath{stroke}%
\end{pgfscope}%
\begin{pgfscope}%
\pgfpathrectangle{\pgfqpoint{0.781402in}{0.773588in}}{\pgfqpoint{2.110351in}{5.415119in}}%
\pgfusepath{clip}%
\pgfsetrectcap%
\pgfsetroundjoin%
\pgfsetlinewidth{1.505625pt}%
\definecolor{currentstroke}{rgb}{0.000000,0.000000,1.000000}%
\pgfsetstrokecolor{currentstroke}%
\pgfsetdash{}{0pt}%
\pgfpathmoveto{\pgfqpoint{2.190812in}{0.773588in}}%
\pgfpathlineto{\pgfqpoint{2.190812in}{6.188708in}}%
\pgfusepath{stroke}%
\end{pgfscope}%
\begin{pgfscope}%
\pgfpathrectangle{\pgfqpoint{0.781402in}{0.773588in}}{\pgfqpoint{2.110351in}{5.415119in}}%
\pgfusepath{clip}%
\pgfsetrectcap%
\pgfsetroundjoin%
\pgfsetlinewidth{1.505625pt}%
\definecolor{currentstroke}{rgb}{0.750000,0.750000,0.000000}%
\pgfsetstrokecolor{currentstroke}%
\pgfsetdash{}{0pt}%
\pgfusepath{stroke}%
\end{pgfscope}%
\begin{pgfscope}%
\pgfpathrectangle{\pgfqpoint{0.781402in}{0.773588in}}{\pgfqpoint{2.110351in}{5.415119in}}%
\pgfusepath{clip}%
\pgfsetrectcap%
\pgfsetroundjoin%
\pgfsetlinewidth{1.505625pt}%
\definecolor{currentstroke}{rgb}{0.750000,0.000000,0.750000}%
\pgfsetstrokecolor{currentstroke}%
\pgfsetdash{}{0pt}%
\pgfusepath{stroke}%
\end{pgfscope}%
\begin{pgfscope}%
\pgfpathrectangle{\pgfqpoint{0.781402in}{0.773588in}}{\pgfqpoint{2.110351in}{5.415119in}}%
\pgfusepath{clip}%
\pgfsetrectcap%
\pgfsetroundjoin%
\pgfsetlinewidth{1.505625pt}%
\definecolor{currentstroke}{rgb}{1.000000,0.000000,0.000000}%
\pgfsetstrokecolor{currentstroke}%
\pgfsetdash{}{0pt}%
\pgfusepath{stroke}%
\end{pgfscope}%
\begin{pgfscope}%
\pgfpathrectangle{\pgfqpoint{0.781402in}{0.773588in}}{\pgfqpoint{2.110351in}{5.415119in}}%
\pgfusepath{clip}%
\pgfsetrectcap%
\pgfsetroundjoin%
\pgfsetlinewidth{1.505625pt}%
\definecolor{currentstroke}{rgb}{0.000000,0.500000,0.000000}%
\pgfsetstrokecolor{currentstroke}%
\pgfsetdash{}{0pt}%
\pgfusepath{stroke}%
\end{pgfscope}%
\begin{pgfscope}%
\pgfpathrectangle{\pgfqpoint{0.781402in}{0.773588in}}{\pgfqpoint{2.110351in}{5.415119in}}%
\pgfusepath{clip}%
\pgfsetrectcap%
\pgfsetroundjoin%
\pgfsetlinewidth{1.505625pt}%
\definecolor{currentstroke}{rgb}{0.000000,0.750000,0.750000}%
\pgfsetstrokecolor{currentstroke}%
\pgfsetdash{}{0pt}%
\pgfusepath{stroke}%
\end{pgfscope}%
\begin{pgfscope}%
\pgfpathrectangle{\pgfqpoint{0.781402in}{0.773588in}}{\pgfqpoint{2.110351in}{5.415119in}}%
\pgfusepath{clip}%
\pgfsetrectcap%
\pgfsetroundjoin%
\pgfsetlinewidth{1.505625pt}%
\definecolor{currentstroke}{rgb}{0.750000,0.000000,0.750000}%
\pgfsetstrokecolor{currentstroke}%
\pgfsetdash{}{0pt}%
\pgfusepath{stroke}%
\end{pgfscope}%
\begin{pgfscope}%
\pgfsetrectcap%
\pgfsetmiterjoin%
\pgfsetlinewidth{0.803000pt}%
\definecolor{currentstroke}{rgb}{0.000000,0.000000,0.000000}%
\pgfsetstrokecolor{currentstroke}%
\pgfsetdash{}{0pt}%
\pgfpathmoveto{\pgfqpoint{0.781402in}{0.773588in}}%
\pgfpathlineto{\pgfqpoint{0.781402in}{6.188708in}}%
\pgfusepath{stroke}%
\end{pgfscope}%
\begin{pgfscope}%
\pgfsetrectcap%
\pgfsetmiterjoin%
\pgfsetlinewidth{0.803000pt}%
\definecolor{currentstroke}{rgb}{0.000000,0.000000,0.000000}%
\pgfsetstrokecolor{currentstroke}%
\pgfsetdash{}{0pt}%
\pgfpathmoveto{\pgfqpoint{0.781402in}{0.773588in}}%
\pgfpathlineto{\pgfqpoint{2.891753in}{0.773588in}}%
\pgfusepath{stroke}%
\end{pgfscope}%
\begin{pgfscope}%
\pgfsetbuttcap%
\pgfsetmiterjoin%
\definecolor{currentfill}{rgb}{1.000000,1.000000,1.000000}%
\pgfsetfillcolor{currentfill}%
\pgfsetlinewidth{0.000000pt}%
\definecolor{currentstroke}{rgb}{0.000000,0.000000,0.000000}%
\pgfsetstrokecolor{currentstroke}%
\pgfsetstrokeopacity{0.000000}%
\pgfsetdash{}{0pt}%
\pgfpathmoveto{\pgfqpoint{3.332180in}{0.773588in}}%
\pgfpathlineto{\pgfqpoint{5.626098in}{0.773588in}}%
\pgfpathlineto{\pgfqpoint{5.626098in}{6.188708in}}%
\pgfpathlineto{\pgfqpoint{3.332180in}{6.188708in}}%
\pgfpathclose%
\pgfusepath{fill}%
\end{pgfscope}%
\begin{pgfscope}%
\pgfpathrectangle{\pgfqpoint{3.332180in}{0.773588in}}{\pgfqpoint{2.293918in}{5.415119in}}%
\pgfusepath{clip}%
\pgfsetbuttcap%
\pgfsetroundjoin%
\definecolor{currentfill}{rgb}{0.121569,0.466667,0.705882}%
\pgfsetfillcolor{currentfill}%
\pgfsetlinewidth{0.000000pt}%
\definecolor{currentstroke}{rgb}{0.000000,0.000000,0.000000}%
\pgfsetstrokecolor{currentstroke}%
\pgfsetdash{}{0pt}%
\pgfpathmoveto{\pgfqpoint{-30.745143in}{0.773588in}}%
\pgfpathlineto{\pgfqpoint{-30.745143in}{0.773588in}}%
\pgfpathlineto{\pgfqpoint{-30.676901in}{0.773588in}}%
\pgfpathlineto{\pgfqpoint{-30.610127in}{0.773588in}}%
\pgfpathlineto{\pgfqpoint{-30.539383in}{0.773588in}}%
\pgfpathlineto{\pgfqpoint{-30.472294in}{0.773588in}}%
\pgfpathlineto{\pgfqpoint{-30.404868in}{0.773588in}}%
\pgfpathlineto{\pgfqpoint{-30.335915in}{0.773588in}}%
\pgfpathlineto{\pgfqpoint{-30.269201in}{0.773588in}}%
\pgfpathlineto{\pgfqpoint{-30.202864in}{0.773588in}}%
\pgfpathlineto{\pgfqpoint{-30.131142in}{0.773588in}}%
\pgfpathlineto{\pgfqpoint{-30.062931in}{0.773588in}}%
\pgfpathlineto{\pgfqpoint{-29.994776in}{0.773588in}}%
\pgfpathlineto{\pgfqpoint{-29.923374in}{0.773588in}}%
\pgfpathlineto{\pgfqpoint{-29.853575in}{0.773588in}}%
\pgfpathlineto{\pgfqpoint{-29.785834in}{0.773588in}}%
\pgfpathlineto{\pgfqpoint{-29.716346in}{0.773588in}}%
\pgfpathlineto{\pgfqpoint{-29.647778in}{0.773588in}}%
\pgfpathlineto{\pgfqpoint{-29.581285in}{0.773588in}}%
\pgfpathlineto{\pgfqpoint{-29.511058in}{0.773588in}}%
\pgfpathlineto{\pgfqpoint{-29.443353in}{0.773588in}}%
\pgfpathlineto{\pgfqpoint{-29.374900in}{0.773588in}}%
\pgfpathlineto{\pgfqpoint{-29.304065in}{0.773588in}}%
\pgfpathlineto{\pgfqpoint{-29.235084in}{0.773588in}}%
\pgfpathlineto{\pgfqpoint{-29.164288in}{0.773588in}}%
\pgfpathlineto{\pgfqpoint{-29.089467in}{0.773588in}}%
\pgfpathlineto{\pgfqpoint{-29.018119in}{0.773588in}}%
\pgfpathlineto{\pgfqpoint{-28.946213in}{0.773588in}}%
\pgfpathlineto{\pgfqpoint{-28.870906in}{0.773588in}}%
\pgfpathlineto{\pgfqpoint{-28.798746in}{0.773588in}}%
\pgfpathlineto{\pgfqpoint{-28.725736in}{0.773588in}}%
\pgfpathlineto{\pgfqpoint{-28.649563in}{0.773588in}}%
\pgfpathlineto{\pgfqpoint{-28.577221in}{0.773588in}}%
\pgfpathlineto{\pgfqpoint{-28.504596in}{0.773588in}}%
\pgfpathlineto{\pgfqpoint{-28.429123in}{0.773588in}}%
\pgfpathlineto{\pgfqpoint{-28.356803in}{0.773588in}}%
\pgfpathlineto{\pgfqpoint{-28.274638in}{0.773588in}}%
\pgfpathlineto{\pgfqpoint{-28.195772in}{0.773588in}}%
\pgfpathlineto{\pgfqpoint{-28.121663in}{0.773588in}}%
\pgfpathlineto{\pgfqpoint{-28.049197in}{0.773588in}}%
\pgfpathlineto{\pgfqpoint{-27.976694in}{0.773588in}}%
\pgfpathlineto{\pgfqpoint{-27.908279in}{0.773588in}}%
\pgfpathlineto{\pgfqpoint{-27.839702in}{0.773588in}}%
\pgfpathlineto{\pgfqpoint{-27.771032in}{0.773588in}}%
\pgfpathlineto{\pgfqpoint{-27.703951in}{0.773588in}}%
\pgfpathlineto{\pgfqpoint{-27.637925in}{0.773588in}}%
\pgfpathlineto{\pgfqpoint{-27.570291in}{0.773588in}}%
\pgfpathlineto{\pgfqpoint{-27.503250in}{0.773588in}}%
\pgfpathlineto{\pgfqpoint{-27.435671in}{0.773588in}}%
\pgfpathlineto{\pgfqpoint{-27.366087in}{0.773588in}}%
\pgfpathlineto{\pgfqpoint{-27.299189in}{0.773588in}}%
\pgfpathlineto{\pgfqpoint{-27.232937in}{0.773588in}}%
\pgfpathlineto{\pgfqpoint{-27.165061in}{0.773588in}}%
\pgfpathlineto{\pgfqpoint{-27.097972in}{0.773588in}}%
\pgfpathlineto{\pgfqpoint{-27.031677in}{0.773588in}}%
\pgfpathlineto{\pgfqpoint{-26.961733in}{0.773588in}}%
\pgfpathlineto{\pgfqpoint{-26.893662in}{0.773588in}}%
\pgfpathlineto{\pgfqpoint{-26.825734in}{0.773588in}}%
\pgfpathlineto{\pgfqpoint{-26.755672in}{0.773588in}}%
\pgfpathlineto{\pgfqpoint{-26.685925in}{0.773588in}}%
\pgfpathlineto{\pgfqpoint{-26.616985in}{0.773588in}}%
\pgfpathlineto{\pgfqpoint{-26.547093in}{0.773588in}}%
\pgfpathlineto{\pgfqpoint{-26.479146in}{0.773588in}}%
\pgfpathlineto{\pgfqpoint{-26.410031in}{0.773588in}}%
\pgfpathlineto{\pgfqpoint{-26.338662in}{0.773588in}}%
\pgfpathlineto{\pgfqpoint{-26.268985in}{0.773588in}}%
\pgfpathlineto{\pgfqpoint{-26.198294in}{0.773588in}}%
\pgfpathlineto{\pgfqpoint{-26.128033in}{0.773588in}}%
\pgfpathlineto{\pgfqpoint{-26.059544in}{0.773588in}}%
\pgfpathlineto{\pgfqpoint{-25.990773in}{0.773588in}}%
\pgfpathlineto{\pgfqpoint{-25.919100in}{0.773588in}}%
\pgfpathlineto{\pgfqpoint{-25.851769in}{0.773588in}}%
\pgfpathlineto{\pgfqpoint{-25.785528in}{0.773588in}}%
\pgfpathlineto{\pgfqpoint{-25.717568in}{0.773588in}}%
\pgfpathlineto{\pgfqpoint{-25.650614in}{0.773588in}}%
\pgfpathlineto{\pgfqpoint{-25.583742in}{0.773588in}}%
\pgfpathlineto{\pgfqpoint{-25.515370in}{0.773588in}}%
\pgfpathlineto{\pgfqpoint{-25.448976in}{0.773588in}}%
\pgfpathlineto{\pgfqpoint{-25.382786in}{0.773588in}}%
\pgfpathlineto{\pgfqpoint{-25.315375in}{0.773588in}}%
\pgfpathlineto{\pgfqpoint{-25.248757in}{0.773588in}}%
\pgfpathlineto{\pgfqpoint{-25.182394in}{0.773588in}}%
\pgfpathlineto{\pgfqpoint{-25.114674in}{0.773588in}}%
\pgfpathlineto{\pgfqpoint{-25.049351in}{0.773588in}}%
\pgfpathlineto{\pgfqpoint{-24.984232in}{0.773588in}}%
\pgfpathlineto{\pgfqpoint{-24.916019in}{0.773588in}}%
\pgfpathlineto{\pgfqpoint{-24.847966in}{0.773588in}}%
\pgfpathlineto{\pgfqpoint{-24.781655in}{0.773588in}}%
\pgfpathlineto{\pgfqpoint{-24.712889in}{0.773588in}}%
\pgfpathlineto{\pgfqpoint{-24.645693in}{0.773588in}}%
\pgfpathlineto{\pgfqpoint{-24.578904in}{0.773588in}}%
\pgfpathlineto{\pgfqpoint{-24.511431in}{0.773588in}}%
\pgfpathlineto{\pgfqpoint{-24.445935in}{0.773588in}}%
\pgfpathlineto{\pgfqpoint{-24.381203in}{0.773588in}}%
\pgfpathlineto{\pgfqpoint{-24.313406in}{0.773588in}}%
\pgfpathlineto{\pgfqpoint{-24.243942in}{0.773588in}}%
\pgfpathlineto{\pgfqpoint{-24.172427in}{0.773588in}}%
\pgfpathlineto{\pgfqpoint{-24.096931in}{0.773588in}}%
\pgfpathlineto{\pgfqpoint{-24.025129in}{0.773588in}}%
\pgfpathlineto{\pgfqpoint{-23.953546in}{0.773588in}}%
\pgfpathlineto{\pgfqpoint{-23.881105in}{0.773588in}}%
\pgfpathlineto{\pgfqpoint{-23.810815in}{0.773588in}}%
\pgfpathlineto{\pgfqpoint{-23.739951in}{0.773588in}}%
\pgfpathlineto{\pgfqpoint{-23.664795in}{0.773588in}}%
\pgfpathlineto{\pgfqpoint{-23.594251in}{0.773588in}}%
\pgfpathlineto{\pgfqpoint{-23.523979in}{0.773588in}}%
\pgfpathlineto{\pgfqpoint{-23.451780in}{0.773588in}}%
\pgfpathlineto{\pgfqpoint{-23.383303in}{0.773588in}}%
\pgfpathlineto{\pgfqpoint{-23.314406in}{0.773588in}}%
\pgfpathlineto{\pgfqpoint{-23.243519in}{0.773588in}}%
\pgfpathlineto{\pgfqpoint{-23.175783in}{0.773588in}}%
\pgfpathlineto{\pgfqpoint{-23.107951in}{0.773588in}}%
\pgfpathlineto{\pgfqpoint{-23.037151in}{0.773588in}}%
\pgfpathlineto{\pgfqpoint{-22.968635in}{0.773588in}}%
\pgfpathlineto{\pgfqpoint{-22.900479in}{0.773588in}}%
\pgfpathlineto{\pgfqpoint{-22.831621in}{0.773588in}}%
\pgfpathlineto{\pgfqpoint{-22.764599in}{0.773588in}}%
\pgfpathlineto{\pgfqpoint{-22.696138in}{0.773588in}}%
\pgfpathlineto{\pgfqpoint{-22.627376in}{0.773588in}}%
\pgfpathlineto{\pgfqpoint{-22.560590in}{0.773588in}}%
\pgfpathlineto{\pgfqpoint{-22.492630in}{0.773588in}}%
\pgfpathlineto{\pgfqpoint{-22.421718in}{0.773588in}}%
\pgfpathlineto{\pgfqpoint{-22.353759in}{0.773588in}}%
\pgfpathlineto{\pgfqpoint{-22.286974in}{0.773588in}}%
\pgfpathlineto{\pgfqpoint{-22.217508in}{0.773588in}}%
\pgfpathlineto{\pgfqpoint{-22.150098in}{0.773588in}}%
\pgfpathlineto{\pgfqpoint{-22.084923in}{0.773588in}}%
\pgfpathlineto{\pgfqpoint{-22.016091in}{0.773588in}}%
\pgfpathlineto{\pgfqpoint{-21.948488in}{0.773588in}}%
\pgfpathlineto{\pgfqpoint{-21.881469in}{0.773588in}}%
\pgfpathlineto{\pgfqpoint{-21.811964in}{0.773588in}}%
\pgfpathlineto{\pgfqpoint{-21.743422in}{0.773588in}}%
\pgfpathlineto{\pgfqpoint{-21.676089in}{0.773588in}}%
\pgfpathlineto{\pgfqpoint{-21.607622in}{0.773588in}}%
\pgfpathlineto{\pgfqpoint{-21.539549in}{0.773588in}}%
\pgfpathlineto{\pgfqpoint{-21.470634in}{0.773588in}}%
\pgfpathlineto{\pgfqpoint{-21.398346in}{0.773588in}}%
\pgfpathlineto{\pgfqpoint{-21.330134in}{0.773588in}}%
\pgfpathlineto{\pgfqpoint{-21.261541in}{0.773588in}}%
\pgfpathlineto{\pgfqpoint{-21.191384in}{0.773588in}}%
\pgfpathlineto{\pgfqpoint{-21.123238in}{0.773588in}}%
\pgfpathlineto{\pgfqpoint{-21.053709in}{0.773588in}}%
\pgfpathlineto{\pgfqpoint{-20.982338in}{0.773588in}}%
\pgfpathlineto{\pgfqpoint{-20.913231in}{0.773588in}}%
\pgfpathlineto{\pgfqpoint{-20.844220in}{0.773588in}}%
\pgfpathlineto{\pgfqpoint{-20.774312in}{0.773588in}}%
\pgfpathlineto{\pgfqpoint{-20.705385in}{0.773588in}}%
\pgfpathlineto{\pgfqpoint{-20.636228in}{0.773588in}}%
\pgfpathlineto{\pgfqpoint{-20.565295in}{0.773588in}}%
\pgfpathlineto{\pgfqpoint{-20.496847in}{0.773588in}}%
\pgfpathlineto{\pgfqpoint{-20.427103in}{0.773588in}}%
\pgfpathlineto{\pgfqpoint{-20.357252in}{0.773588in}}%
\pgfpathlineto{\pgfqpoint{-20.290460in}{0.773588in}}%
\pgfpathlineto{\pgfqpoint{-20.223560in}{0.773588in}}%
\pgfpathlineto{\pgfqpoint{-20.154101in}{0.773588in}}%
\pgfpathlineto{\pgfqpoint{-20.086943in}{0.773588in}}%
\pgfpathlineto{\pgfqpoint{-20.019359in}{0.773588in}}%
\pgfpathlineto{\pgfqpoint{-19.949841in}{0.773588in}}%
\pgfpathlineto{\pgfqpoint{-19.882267in}{0.773588in}}%
\pgfpathlineto{\pgfqpoint{-19.815535in}{0.773588in}}%
\pgfpathlineto{\pgfqpoint{-19.747542in}{0.773588in}}%
\pgfpathlineto{\pgfqpoint{-19.680997in}{0.773588in}}%
\pgfpathlineto{\pgfqpoint{-19.613839in}{0.773588in}}%
\pgfpathlineto{\pgfqpoint{-19.543393in}{0.773588in}}%
\pgfpathlineto{\pgfqpoint{-19.475599in}{0.773588in}}%
\pgfpathlineto{\pgfqpoint{-19.407422in}{0.773588in}}%
\pgfpathlineto{\pgfqpoint{-19.338364in}{0.773588in}}%
\pgfpathlineto{\pgfqpoint{-19.271476in}{0.773588in}}%
\pgfpathlineto{\pgfqpoint{-19.205228in}{0.773588in}}%
\pgfpathlineto{\pgfqpoint{-19.135788in}{0.773588in}}%
\pgfpathlineto{\pgfqpoint{-19.068147in}{0.773588in}}%
\pgfpathlineto{\pgfqpoint{-18.999640in}{0.773588in}}%
\pgfpathlineto{\pgfqpoint{-18.928709in}{0.773588in}}%
\pgfpathlineto{\pgfqpoint{-18.859965in}{0.773588in}}%
\pgfpathlineto{\pgfqpoint{-18.792270in}{0.773588in}}%
\pgfpathlineto{\pgfqpoint{-18.721737in}{0.773588in}}%
\pgfpathlineto{\pgfqpoint{-18.652004in}{0.773588in}}%
\pgfpathlineto{\pgfqpoint{-18.583201in}{0.773588in}}%
\pgfpathlineto{\pgfqpoint{-18.511159in}{0.773588in}}%
\pgfpathlineto{\pgfqpoint{-18.442527in}{0.773588in}}%
\pgfpathlineto{\pgfqpoint{-18.373957in}{0.773588in}}%
\pgfpathlineto{\pgfqpoint{-18.301710in}{0.773588in}}%
\pgfpathlineto{\pgfqpoint{-18.232823in}{0.773588in}}%
\pgfpathlineto{\pgfqpoint{-18.165145in}{0.773588in}}%
\pgfpathlineto{\pgfqpoint{-18.094404in}{0.773588in}}%
\pgfpathlineto{\pgfqpoint{-18.025876in}{0.773588in}}%
\pgfpathlineto{\pgfqpoint{-17.957185in}{0.773588in}}%
\pgfpathlineto{\pgfqpoint{-17.885356in}{0.773588in}}%
\pgfpathlineto{\pgfqpoint{-17.816737in}{0.773588in}}%
\pgfpathlineto{\pgfqpoint{-17.748008in}{0.773588in}}%
\pgfpathlineto{\pgfqpoint{-17.675451in}{0.773588in}}%
\pgfpathlineto{\pgfqpoint{-17.607453in}{0.773588in}}%
\pgfpathlineto{\pgfqpoint{-17.539598in}{0.773588in}}%
\pgfpathlineto{\pgfqpoint{-17.470673in}{0.773588in}}%
\pgfpathlineto{\pgfqpoint{-17.402675in}{0.773588in}}%
\pgfpathlineto{\pgfqpoint{-17.336134in}{0.773588in}}%
\pgfpathlineto{\pgfqpoint{-17.265552in}{0.773588in}}%
\pgfpathlineto{\pgfqpoint{-17.196258in}{0.773588in}}%
\pgfpathlineto{\pgfqpoint{-17.128098in}{0.773588in}}%
\pgfpathlineto{\pgfqpoint{-17.058127in}{0.773588in}}%
\pgfpathlineto{\pgfqpoint{-16.989953in}{0.773588in}}%
\pgfpathlineto{\pgfqpoint{-16.921714in}{0.773588in}}%
\pgfpathlineto{\pgfqpoint{-16.852234in}{0.773588in}}%
\pgfpathlineto{\pgfqpoint{-16.784012in}{0.773588in}}%
\pgfpathlineto{\pgfqpoint{-16.715630in}{0.773588in}}%
\pgfpathlineto{\pgfqpoint{-16.644038in}{0.773588in}}%
\pgfpathlineto{\pgfqpoint{-16.575864in}{0.773588in}}%
\pgfpathlineto{\pgfqpoint{-16.507797in}{0.773588in}}%
\pgfpathlineto{\pgfqpoint{-16.437097in}{0.773588in}}%
\pgfpathlineto{\pgfqpoint{-16.369724in}{0.773588in}}%
\pgfpathlineto{\pgfqpoint{-16.302639in}{0.773588in}}%
\pgfpathlineto{\pgfqpoint{-16.232068in}{0.773588in}}%
\pgfpathlineto{\pgfqpoint{-16.163654in}{0.773588in}}%
\pgfpathlineto{\pgfqpoint{-16.094895in}{0.773588in}}%
\pgfpathlineto{\pgfqpoint{-16.024276in}{0.773588in}}%
\pgfpathlineto{\pgfqpoint{-15.956358in}{0.773588in}}%
\pgfpathlineto{\pgfqpoint{-15.886757in}{0.773588in}}%
\pgfpathlineto{\pgfqpoint{-15.814335in}{0.773588in}}%
\pgfpathlineto{\pgfqpoint{-15.744769in}{0.773588in}}%
\pgfpathlineto{\pgfqpoint{-15.673559in}{0.773588in}}%
\pgfpathlineto{\pgfqpoint{-15.602559in}{0.773588in}}%
\pgfpathlineto{\pgfqpoint{-15.532639in}{0.773588in}}%
\pgfpathlineto{\pgfqpoint{-15.463558in}{0.773588in}}%
\pgfpathlineto{\pgfqpoint{-15.392048in}{0.773588in}}%
\pgfpathlineto{\pgfqpoint{-15.323000in}{0.773588in}}%
\pgfpathlineto{\pgfqpoint{-15.253001in}{0.773588in}}%
\pgfpathlineto{\pgfqpoint{-15.179182in}{0.773588in}}%
\pgfpathlineto{\pgfqpoint{-15.106256in}{0.773588in}}%
\pgfpathlineto{\pgfqpoint{-15.032843in}{0.773588in}}%
\pgfpathlineto{\pgfqpoint{-14.959865in}{0.773588in}}%
\pgfpathlineto{\pgfqpoint{-14.889438in}{0.773588in}}%
\pgfpathlineto{\pgfqpoint{-14.820408in}{0.773588in}}%
\pgfpathlineto{\pgfqpoint{-14.748835in}{0.773588in}}%
\pgfpathlineto{\pgfqpoint{-14.679263in}{0.773588in}}%
\pgfpathlineto{\pgfqpoint{-14.609938in}{0.773588in}}%
\pgfpathlineto{\pgfqpoint{-14.538988in}{0.773588in}}%
\pgfpathlineto{\pgfqpoint{-14.471123in}{0.773588in}}%
\pgfpathlineto{\pgfqpoint{-14.401957in}{0.773588in}}%
\pgfpathlineto{\pgfqpoint{-14.331277in}{0.773588in}}%
\pgfpathlineto{\pgfqpoint{-14.263436in}{0.773588in}}%
\pgfpathlineto{\pgfqpoint{-14.195714in}{0.773588in}}%
\pgfpathlineto{\pgfqpoint{-14.125666in}{0.773588in}}%
\pgfpathlineto{\pgfqpoint{-14.057377in}{0.773588in}}%
\pgfpathlineto{\pgfqpoint{-13.988983in}{0.773588in}}%
\pgfpathlineto{\pgfqpoint{-13.917455in}{0.773588in}}%
\pgfpathlineto{\pgfqpoint{-13.848756in}{0.773588in}}%
\pgfpathlineto{\pgfqpoint{-13.779339in}{0.773588in}}%
\pgfpathlineto{\pgfqpoint{-13.708089in}{0.773588in}}%
\pgfpathlineto{\pgfqpoint{-13.639609in}{0.773588in}}%
\pgfpathlineto{\pgfqpoint{-13.568713in}{0.773588in}}%
\pgfpathlineto{\pgfqpoint{-13.496151in}{0.773588in}}%
\pgfpathlineto{\pgfqpoint{-13.426276in}{0.773588in}}%
\pgfpathlineto{\pgfqpoint{-13.356968in}{0.773588in}}%
\pgfpathlineto{\pgfqpoint{-13.285456in}{0.773588in}}%
\pgfpathlineto{\pgfqpoint{-13.216013in}{0.773588in}}%
\pgfpathlineto{\pgfqpoint{-13.145337in}{0.773588in}}%
\pgfpathlineto{\pgfqpoint{-13.073093in}{0.773588in}}%
\pgfpathlineto{\pgfqpoint{-13.002727in}{0.773588in}}%
\pgfpathlineto{\pgfqpoint{-12.932388in}{0.773588in}}%
\pgfpathlineto{\pgfqpoint{-12.859398in}{0.773588in}}%
\pgfpathlineto{\pgfqpoint{-12.788679in}{0.773588in}}%
\pgfpathlineto{\pgfqpoint{-12.718096in}{0.773588in}}%
\pgfpathlineto{\pgfqpoint{-12.645279in}{0.773588in}}%
\pgfpathlineto{\pgfqpoint{-12.574652in}{0.773588in}}%
\pgfpathlineto{\pgfqpoint{-12.503401in}{0.773588in}}%
\pgfpathlineto{\pgfqpoint{-12.428997in}{0.773588in}}%
\pgfpathlineto{\pgfqpoint{-12.357103in}{0.773588in}}%
\pgfpathlineto{\pgfqpoint{-12.285275in}{0.773588in}}%
\pgfpathlineto{\pgfqpoint{-12.211536in}{0.773588in}}%
\pgfpathlineto{\pgfqpoint{-12.139885in}{0.773588in}}%
\pgfpathlineto{\pgfqpoint{-12.069446in}{0.773588in}}%
\pgfpathlineto{\pgfqpoint{-11.998596in}{0.773588in}}%
\pgfpathlineto{\pgfqpoint{-11.928381in}{0.773588in}}%
\pgfpathlineto{\pgfqpoint{-11.858471in}{0.773588in}}%
\pgfpathlineto{\pgfqpoint{-11.786176in}{0.773588in}}%
\pgfpathlineto{\pgfqpoint{-11.716309in}{0.773588in}}%
\pgfpathlineto{\pgfqpoint{-11.645187in}{0.773588in}}%
\pgfpathlineto{\pgfqpoint{-11.571037in}{0.773588in}}%
\pgfpathlineto{\pgfqpoint{-11.500166in}{0.773588in}}%
\pgfpathlineto{\pgfqpoint{-11.430351in}{0.773588in}}%
\pgfpathlineto{\pgfqpoint{-11.358676in}{0.773588in}}%
\pgfpathlineto{\pgfqpoint{-11.288273in}{0.773588in}}%
\pgfpathlineto{\pgfqpoint{-11.217631in}{0.773588in}}%
\pgfpathlineto{\pgfqpoint{-11.144960in}{0.773588in}}%
\pgfpathlineto{\pgfqpoint{-11.075276in}{0.773588in}}%
\pgfpathlineto{\pgfqpoint{-11.005065in}{0.773588in}}%
\pgfpathlineto{\pgfqpoint{-10.933808in}{0.773588in}}%
\pgfpathlineto{\pgfqpoint{-10.863257in}{0.773588in}}%
\pgfpathlineto{\pgfqpoint{-10.793422in}{0.773588in}}%
\pgfpathlineto{\pgfqpoint{-10.721651in}{0.773588in}}%
\pgfpathlineto{\pgfqpoint{-10.652649in}{0.773588in}}%
\pgfpathlineto{\pgfqpoint{-10.582864in}{0.773588in}}%
\pgfpathlineto{\pgfqpoint{-10.511372in}{0.773588in}}%
\pgfpathlineto{\pgfqpoint{-10.441581in}{0.773588in}}%
\pgfpathlineto{\pgfqpoint{-10.371004in}{0.773588in}}%
\pgfpathlineto{\pgfqpoint{-10.297591in}{0.773588in}}%
\pgfpathlineto{\pgfqpoint{-10.227739in}{0.773588in}}%
\pgfpathlineto{\pgfqpoint{-10.157398in}{0.773588in}}%
\pgfpathlineto{\pgfqpoint{-10.084496in}{0.773588in}}%
\pgfpathlineto{\pgfqpoint{-10.012635in}{0.773588in}}%
\pgfpathlineto{\pgfqpoint{-9.941359in}{0.773588in}}%
\pgfpathlineto{\pgfqpoint{-9.868243in}{0.773588in}}%
\pgfpathlineto{\pgfqpoint{-9.796010in}{0.773588in}}%
\pgfpathlineto{\pgfqpoint{-9.723682in}{0.773588in}}%
\pgfpathlineto{\pgfqpoint{-9.648369in}{0.773588in}}%
\pgfpathlineto{\pgfqpoint{-9.575400in}{0.773588in}}%
\pgfpathlineto{\pgfqpoint{-9.504196in}{0.773588in}}%
\pgfpathlineto{\pgfqpoint{-9.429774in}{0.773588in}}%
\pgfpathlineto{\pgfqpoint{-9.356529in}{0.773588in}}%
\pgfpathlineto{\pgfqpoint{-9.283412in}{0.773588in}}%
\pgfpathlineto{\pgfqpoint{-9.208906in}{0.773588in}}%
\pgfpathlineto{\pgfqpoint{-9.138788in}{0.773588in}}%
\pgfpathlineto{\pgfqpoint{-9.069721in}{0.773588in}}%
\pgfpathlineto{\pgfqpoint{-8.998701in}{0.773588in}}%
\pgfpathlineto{\pgfqpoint{-8.928123in}{0.773588in}}%
\pgfpathlineto{\pgfqpoint{-8.858221in}{0.773588in}}%
\pgfpathlineto{\pgfqpoint{-8.787586in}{0.773588in}}%
\pgfpathlineto{\pgfqpoint{-8.718609in}{0.773588in}}%
\pgfpathlineto{\pgfqpoint{-8.649341in}{0.773588in}}%
\pgfpathlineto{\pgfqpoint{-8.578369in}{0.773588in}}%
\pgfpathlineto{\pgfqpoint{-8.508839in}{0.773588in}}%
\pgfpathlineto{\pgfqpoint{-8.438970in}{0.773588in}}%
\pgfpathlineto{\pgfqpoint{-8.367967in}{0.773588in}}%
\pgfpathlineto{\pgfqpoint{-8.298753in}{0.773588in}}%
\pgfpathlineto{\pgfqpoint{-8.228241in}{0.773588in}}%
\pgfpathlineto{\pgfqpoint{-8.156111in}{0.773588in}}%
\pgfpathlineto{\pgfqpoint{-8.084913in}{0.773588in}}%
\pgfpathlineto{\pgfqpoint{-8.014387in}{0.773588in}}%
\pgfpathlineto{\pgfqpoint{-7.942200in}{0.773588in}}%
\pgfpathlineto{\pgfqpoint{-7.870788in}{0.773588in}}%
\pgfpathlineto{\pgfqpoint{-7.799875in}{0.773588in}}%
\pgfpathlineto{\pgfqpoint{-7.727286in}{0.773588in}}%
\pgfpathlineto{\pgfqpoint{-7.657024in}{0.773588in}}%
\pgfpathlineto{\pgfqpoint{-7.585515in}{0.773588in}}%
\pgfpathlineto{\pgfqpoint{-7.512981in}{0.773588in}}%
\pgfpathlineto{\pgfqpoint{-7.440452in}{0.773588in}}%
\pgfpathlineto{\pgfqpoint{-7.368338in}{0.773588in}}%
\pgfpathlineto{\pgfqpoint{-7.294929in}{0.773588in}}%
\pgfpathlineto{\pgfqpoint{-7.223147in}{0.773588in}}%
\pgfpathlineto{\pgfqpoint{-7.151909in}{0.773588in}}%
\pgfpathlineto{\pgfqpoint{-7.075897in}{0.773588in}}%
\pgfpathlineto{\pgfqpoint{-7.003598in}{0.773588in}}%
\pgfpathlineto{\pgfqpoint{-6.933558in}{0.773588in}}%
\pgfpathlineto{\pgfqpoint{-6.860576in}{0.773588in}}%
\pgfpathlineto{\pgfqpoint{-6.787495in}{0.773588in}}%
\pgfpathlineto{\pgfqpoint{-6.715245in}{0.773588in}}%
\pgfpathlineto{\pgfqpoint{-6.640496in}{0.773588in}}%
\pgfpathlineto{\pgfqpoint{-6.568712in}{0.773588in}}%
\pgfpathlineto{\pgfqpoint{-6.496669in}{0.773588in}}%
\pgfpathlineto{\pgfqpoint{-6.421127in}{0.773588in}}%
\pgfpathlineto{\pgfqpoint{-6.349020in}{0.773588in}}%
\pgfpathlineto{\pgfqpoint{-6.278887in}{0.773588in}}%
\pgfpathlineto{\pgfqpoint{-6.205113in}{0.773588in}}%
\pgfpathlineto{\pgfqpoint{-6.135189in}{0.773588in}}%
\pgfpathlineto{\pgfqpoint{-6.064664in}{0.773588in}}%
\pgfpathlineto{\pgfqpoint{-5.992127in}{0.773588in}}%
\pgfpathlineto{\pgfqpoint{-5.921215in}{0.773588in}}%
\pgfpathlineto{\pgfqpoint{-5.849895in}{0.773588in}}%
\pgfpathlineto{\pgfqpoint{-5.776542in}{0.773588in}}%
\pgfpathlineto{\pgfqpoint{-5.706570in}{0.773588in}}%
\pgfpathlineto{\pgfqpoint{-5.635685in}{0.773588in}}%
\pgfpathlineto{\pgfqpoint{-5.563648in}{0.773588in}}%
\pgfpathlineto{\pgfqpoint{-5.493615in}{0.773588in}}%
\pgfpathlineto{\pgfqpoint{-5.421890in}{0.773588in}}%
\pgfpathlineto{\pgfqpoint{-5.348665in}{0.773588in}}%
\pgfpathlineto{\pgfqpoint{-5.277970in}{0.773588in}}%
\pgfpathlineto{\pgfqpoint{-5.207317in}{0.773588in}}%
\pgfpathlineto{\pgfqpoint{-5.135005in}{0.773588in}}%
\pgfpathlineto{\pgfqpoint{-5.064817in}{0.773588in}}%
\pgfpathlineto{\pgfqpoint{-4.995002in}{0.773588in}}%
\pgfpathlineto{\pgfqpoint{-4.922664in}{0.773588in}}%
\pgfpathlineto{\pgfqpoint{-4.852652in}{0.773588in}}%
\pgfpathlineto{\pgfqpoint{-4.782965in}{0.773588in}}%
\pgfpathlineto{\pgfqpoint{-4.711002in}{0.773588in}}%
\pgfpathlineto{\pgfqpoint{-4.639570in}{0.773588in}}%
\pgfpathlineto{\pgfqpoint{-4.568595in}{0.773588in}}%
\pgfpathlineto{\pgfqpoint{-4.495401in}{0.773588in}}%
\pgfpathlineto{\pgfqpoint{-4.423289in}{0.773588in}}%
\pgfpathlineto{\pgfqpoint{-4.350759in}{0.773588in}}%
\pgfpathlineto{\pgfqpoint{-4.274748in}{0.773588in}}%
\pgfpathlineto{\pgfqpoint{-4.201246in}{0.773588in}}%
\pgfpathlineto{\pgfqpoint{-4.128352in}{0.773588in}}%
\pgfpathlineto{\pgfqpoint{-4.052173in}{0.773588in}}%
\pgfpathlineto{\pgfqpoint{-3.977308in}{0.773588in}}%
\pgfpathlineto{\pgfqpoint{-3.903164in}{0.773588in}}%
\pgfpathlineto{\pgfqpoint{-3.828230in}{0.773588in}}%
\pgfpathlineto{\pgfqpoint{-3.756893in}{0.773588in}}%
\pgfpathlineto{\pgfqpoint{-3.684119in}{0.773588in}}%
\pgfpathlineto{\pgfqpoint{-3.608325in}{0.773588in}}%
\pgfpathlineto{\pgfqpoint{-3.534002in}{0.773588in}}%
\pgfpathlineto{\pgfqpoint{-3.461876in}{0.773588in}}%
\pgfpathlineto{\pgfqpoint{-3.388885in}{0.773588in}}%
\pgfpathlineto{\pgfqpoint{-3.317677in}{0.773588in}}%
\pgfpathlineto{\pgfqpoint{-3.245632in}{0.773588in}}%
\pgfpathlineto{\pgfqpoint{-3.171735in}{0.773588in}}%
\pgfpathlineto{\pgfqpoint{-3.100293in}{0.773588in}}%
\pgfpathlineto{\pgfqpoint{-3.029702in}{0.773588in}}%
\pgfpathlineto{\pgfqpoint{-2.956196in}{0.773588in}}%
\pgfpathlineto{\pgfqpoint{-2.884033in}{0.773588in}}%
\pgfpathlineto{\pgfqpoint{-2.813631in}{0.773588in}}%
\pgfpathlineto{\pgfqpoint{-2.740325in}{0.773588in}}%
\pgfpathlineto{\pgfqpoint{-2.667020in}{0.773588in}}%
\pgfpathlineto{\pgfqpoint{-2.595405in}{0.773588in}}%
\pgfpathlineto{\pgfqpoint{-2.522349in}{0.773588in}}%
\pgfpathlineto{\pgfqpoint{-2.451489in}{0.773588in}}%
\pgfpathlineto{\pgfqpoint{-2.379035in}{0.773588in}}%
\pgfpathlineto{\pgfqpoint{-2.303263in}{0.773588in}}%
\pgfpathlineto{\pgfqpoint{-2.231477in}{0.773588in}}%
\pgfpathlineto{\pgfqpoint{-2.158577in}{0.773588in}}%
\pgfpathlineto{\pgfqpoint{-2.084338in}{0.773588in}}%
\pgfpathlineto{\pgfqpoint{-2.011817in}{0.773588in}}%
\pgfpathlineto{\pgfqpoint{-1.940536in}{0.773588in}}%
\pgfpathlineto{\pgfqpoint{-1.867809in}{0.773588in}}%
\pgfpathlineto{\pgfqpoint{-1.797123in}{0.773588in}}%
\pgfpathlineto{\pgfqpoint{-1.725105in}{0.773588in}}%
\pgfpathlineto{\pgfqpoint{-1.650581in}{0.773588in}}%
\pgfpathlineto{\pgfqpoint{-1.577590in}{0.773588in}}%
\pgfpathlineto{\pgfqpoint{-1.503984in}{0.773588in}}%
\pgfpathlineto{\pgfqpoint{-1.429441in}{0.773588in}}%
\pgfpathlineto{\pgfqpoint{-1.356794in}{0.773588in}}%
\pgfpathlineto{\pgfqpoint{-1.283200in}{0.773588in}}%
\pgfpathlineto{\pgfqpoint{-1.207908in}{0.773588in}}%
\pgfpathlineto{\pgfqpoint{-1.135138in}{0.773588in}}%
\pgfpathlineto{\pgfqpoint{-1.063245in}{0.773588in}}%
\pgfpathlineto{\pgfqpoint{-0.989522in}{0.773588in}}%
\pgfpathlineto{\pgfqpoint{-0.918138in}{0.773588in}}%
\pgfpathlineto{\pgfqpoint{-0.844408in}{0.773588in}}%
\pgfpathlineto{\pgfqpoint{-0.769987in}{0.773588in}}%
\pgfpathlineto{\pgfqpoint{-0.698122in}{0.773588in}}%
\pgfpathlineto{\pgfqpoint{-0.623932in}{0.773588in}}%
\pgfpathlineto{\pgfqpoint{-0.549722in}{0.773588in}}%
\pgfpathlineto{\pgfqpoint{-0.477784in}{0.773588in}}%
\pgfpathlineto{\pgfqpoint{-0.404497in}{0.773588in}}%
\pgfpathlineto{\pgfqpoint{-0.329323in}{0.773588in}}%
\pgfpathlineto{\pgfqpoint{-0.257459in}{0.773588in}}%
\pgfpathlineto{\pgfqpoint{-0.185623in}{0.773588in}}%
\pgfpathlineto{\pgfqpoint{-0.112821in}{0.773588in}}%
\pgfpathlineto{\pgfqpoint{-0.042097in}{0.773588in}}%
\pgfpathlineto{\pgfqpoint{0.030045in}{0.773588in}}%
\pgfpathlineto{\pgfqpoint{0.103944in}{0.773588in}}%
\pgfpathlineto{\pgfqpoint{0.176284in}{0.773588in}}%
\pgfpathlineto{\pgfqpoint{0.248641in}{0.773588in}}%
\pgfpathlineto{\pgfqpoint{0.321303in}{0.773588in}}%
\pgfpathlineto{\pgfqpoint{0.391498in}{0.773588in}}%
\pgfpathlineto{\pgfqpoint{0.462885in}{0.773588in}}%
\pgfpathlineto{\pgfqpoint{0.537447in}{0.773588in}}%
\pgfpathlineto{\pgfqpoint{0.609267in}{0.773588in}}%
\pgfpathlineto{\pgfqpoint{0.679536in}{0.773588in}}%
\pgfpathlineto{\pgfqpoint{0.753203in}{0.773588in}}%
\pgfpathlineto{\pgfqpoint{0.824780in}{0.773588in}}%
\pgfpathlineto{\pgfqpoint{0.895203in}{0.773588in}}%
\pgfpathlineto{\pgfqpoint{0.968165in}{0.773588in}}%
\pgfpathlineto{\pgfqpoint{1.038437in}{0.773588in}}%
\pgfpathlineto{\pgfqpoint{1.111472in}{0.773588in}}%
\pgfpathlineto{\pgfqpoint{1.188027in}{0.773588in}}%
\pgfpathlineto{\pgfqpoint{1.261309in}{0.773588in}}%
\pgfpathlineto{\pgfqpoint{1.335256in}{0.773588in}}%
\pgfpathlineto{\pgfqpoint{1.410931in}{0.773588in}}%
\pgfpathlineto{\pgfqpoint{1.485557in}{0.773588in}}%
\pgfpathlineto{\pgfqpoint{1.558242in}{0.773588in}}%
\pgfpathlineto{\pgfqpoint{1.633550in}{0.773588in}}%
\pgfpathlineto{\pgfqpoint{1.707271in}{0.773588in}}%
\pgfpathlineto{\pgfqpoint{1.781074in}{0.773588in}}%
\pgfpathlineto{\pgfqpoint{1.857049in}{0.773588in}}%
\pgfpathlineto{\pgfqpoint{1.931091in}{0.773588in}}%
\pgfpathlineto{\pgfqpoint{2.004776in}{0.773588in}}%
\pgfpathlineto{\pgfqpoint{2.079647in}{0.773588in}}%
\pgfpathlineto{\pgfqpoint{2.151611in}{0.773588in}}%
\pgfpathlineto{\pgfqpoint{2.224651in}{0.773588in}}%
\pgfpathlineto{\pgfqpoint{2.300156in}{0.773588in}}%
\pgfpathlineto{\pgfqpoint{2.371299in}{0.773588in}}%
\pgfpathlineto{\pgfqpoint{2.442412in}{0.773588in}}%
\pgfpathlineto{\pgfqpoint{2.515763in}{0.773588in}}%
\pgfpathlineto{\pgfqpoint{2.586109in}{0.773588in}}%
\pgfpathlineto{\pgfqpoint{2.658524in}{0.773588in}}%
\pgfpathlineto{\pgfqpoint{2.732102in}{0.773588in}}%
\pgfpathlineto{\pgfqpoint{2.802412in}{0.773588in}}%
\pgfpathlineto{\pgfqpoint{2.873367in}{0.773588in}}%
\pgfpathlineto{\pgfqpoint{2.946925in}{0.773588in}}%
\pgfpathlineto{\pgfqpoint{3.019212in}{0.773588in}}%
\pgfpathlineto{\pgfqpoint{3.091740in}{0.773588in}}%
\pgfpathlineto{\pgfqpoint{3.166494in}{0.773588in}}%
\pgfpathlineto{\pgfqpoint{3.237461in}{0.773588in}}%
\pgfpathlineto{\pgfqpoint{3.309976in}{0.773588in}}%
\pgfpathlineto{\pgfqpoint{3.384706in}{0.773588in}}%
\pgfpathlineto{\pgfqpoint{3.455601in}{0.773588in}}%
\pgfpathlineto{\pgfqpoint{3.527918in}{0.773588in}}%
\pgfpathlineto{\pgfqpoint{3.603230in}{0.773588in}}%
\pgfpathlineto{\pgfqpoint{3.674965in}{0.773588in}}%
\pgfpathlineto{\pgfqpoint{3.745938in}{0.773588in}}%
\pgfpathlineto{\pgfqpoint{3.820753in}{0.773588in}}%
\pgfpathlineto{\pgfqpoint{3.899537in}{0.773588in}}%
\pgfpathlineto{\pgfqpoint{4.022313in}{0.773588in}}%
\pgfpathlineto{\pgfqpoint{4.111286in}{0.773588in}}%
\pgfpathlineto{\pgfqpoint{4.189282in}{0.773588in}}%
\pgfpathlineto{\pgfqpoint{4.253332in}{0.773588in}}%
\pgfpathlineto{\pgfqpoint{4.318577in}{0.773588in}}%
\pgfpathlineto{\pgfqpoint{4.390596in}{0.773588in}}%
\pgfpathlineto{\pgfqpoint{4.460559in}{0.773588in}}%
\pgfpathlineto{\pgfqpoint{4.532858in}{0.773588in}}%
\pgfpathlineto{\pgfqpoint{4.602459in}{0.773588in}}%
\pgfpathlineto{\pgfqpoint{4.671388in}{0.773588in}}%
\pgfpathlineto{\pgfqpoint{4.741243in}{0.773588in}}%
\pgfpathlineto{\pgfqpoint{4.808397in}{0.773588in}}%
\pgfpathlineto{\pgfqpoint{4.875969in}{0.773588in}}%
\pgfpathlineto{\pgfqpoint{4.944850in}{0.773588in}}%
\pgfpathlineto{\pgfqpoint{5.010860in}{0.773588in}}%
\pgfpathlineto{\pgfqpoint{5.076713in}{0.773588in}}%
\pgfpathlineto{\pgfqpoint{5.144715in}{0.773588in}}%
\pgfpathlineto{\pgfqpoint{5.209656in}{0.773588in}}%
\pgfpathlineto{\pgfqpoint{5.275100in}{0.773588in}}%
\pgfpathlineto{\pgfqpoint{5.341477in}{0.773588in}}%
\pgfpathlineto{\pgfqpoint{5.405234in}{0.773588in}}%
\pgfpathlineto{\pgfqpoint{5.469981in}{0.773588in}}%
\pgfpathlineto{\pgfqpoint{5.535779in}{0.773588in}}%
\pgfpathlineto{\pgfqpoint{5.599590in}{0.773588in}}%
\pgfpathlineto{\pgfqpoint{5.599590in}{2.457661in}}%
\pgfpathlineto{\pgfqpoint{5.599590in}{2.457661in}}%
\pgfpathlineto{\pgfqpoint{5.535779in}{2.514045in}}%
\pgfpathlineto{\pgfqpoint{5.469981in}{2.426405in}}%
\pgfpathlineto{\pgfqpoint{5.405234in}{2.464622in}}%
\pgfpathlineto{\pgfqpoint{5.341477in}{2.468865in}}%
\pgfpathlineto{\pgfqpoint{5.275100in}{2.434812in}}%
\pgfpathlineto{\pgfqpoint{5.209656in}{2.464952in}}%
\pgfpathlineto{\pgfqpoint{5.144715in}{2.436993in}}%
\pgfpathlineto{\pgfqpoint{5.076713in}{2.376225in}}%
\pgfpathlineto{\pgfqpoint{5.010860in}{2.486664in}}%
\pgfpathlineto{\pgfqpoint{4.944850in}{2.404583in}}%
\pgfpathlineto{\pgfqpoint{4.875969in}{2.435547in}}%
\pgfpathlineto{\pgfqpoint{4.808397in}{2.349474in}}%
\pgfpathlineto{\pgfqpoint{4.741243in}{2.395888in}}%
\pgfpathlineto{\pgfqpoint{4.671388in}{2.332436in}}%
\pgfpathlineto{\pgfqpoint{4.602459in}{2.303111in}}%
\pgfpathlineto{\pgfqpoint{4.532858in}{2.286938in}}%
\pgfpathlineto{\pgfqpoint{4.460559in}{2.321125in}}%
\pgfpathlineto{\pgfqpoint{4.390596in}{2.269015in}}%
\pgfpathlineto{\pgfqpoint{4.318577in}{2.216022in}}%
\pgfpathlineto{\pgfqpoint{4.253332in}{1.249553in}}%
\pgfpathlineto{\pgfqpoint{4.189282in}{0.773588in}}%
\pgfpathlineto{\pgfqpoint{4.111286in}{0.773588in}}%
\pgfpathlineto{\pgfqpoint{4.022313in}{0.773588in}}%
\pgfpathlineto{\pgfqpoint{3.899537in}{0.773588in}}%
\pgfpathlineto{\pgfqpoint{3.820753in}{0.773588in}}%
\pgfpathlineto{\pgfqpoint{3.745938in}{0.773588in}}%
\pgfpathlineto{\pgfqpoint{3.674965in}{0.773588in}}%
\pgfpathlineto{\pgfqpoint{3.603230in}{0.773588in}}%
\pgfpathlineto{\pgfqpoint{3.527918in}{0.773588in}}%
\pgfpathlineto{\pgfqpoint{3.455601in}{0.773588in}}%
\pgfpathlineto{\pgfqpoint{3.384706in}{0.773588in}}%
\pgfpathlineto{\pgfqpoint{3.309976in}{0.773588in}}%
\pgfpathlineto{\pgfqpoint{3.237461in}{0.773588in}}%
\pgfpathlineto{\pgfqpoint{3.166494in}{0.773588in}}%
\pgfpathlineto{\pgfqpoint{3.091740in}{0.773588in}}%
\pgfpathlineto{\pgfqpoint{3.019212in}{0.773588in}}%
\pgfpathlineto{\pgfqpoint{2.946925in}{0.773588in}}%
\pgfpathlineto{\pgfqpoint{2.873367in}{0.773588in}}%
\pgfpathlineto{\pgfqpoint{2.802412in}{0.773588in}}%
\pgfpathlineto{\pgfqpoint{2.732102in}{0.773588in}}%
\pgfpathlineto{\pgfqpoint{2.658524in}{0.773588in}}%
\pgfpathlineto{\pgfqpoint{2.586109in}{0.773588in}}%
\pgfpathlineto{\pgfqpoint{2.515763in}{0.773588in}}%
\pgfpathlineto{\pgfqpoint{2.442412in}{0.773588in}}%
\pgfpathlineto{\pgfqpoint{2.371299in}{0.773588in}}%
\pgfpathlineto{\pgfqpoint{2.300156in}{0.773588in}}%
\pgfpathlineto{\pgfqpoint{2.224651in}{0.773588in}}%
\pgfpathlineto{\pgfqpoint{2.151611in}{0.773588in}}%
\pgfpathlineto{\pgfqpoint{2.079647in}{0.773588in}}%
\pgfpathlineto{\pgfqpoint{2.004776in}{0.773588in}}%
\pgfpathlineto{\pgfqpoint{1.931091in}{0.773588in}}%
\pgfpathlineto{\pgfqpoint{1.857049in}{0.773588in}}%
\pgfpathlineto{\pgfqpoint{1.781074in}{0.773588in}}%
\pgfpathlineto{\pgfqpoint{1.707271in}{0.773588in}}%
\pgfpathlineto{\pgfqpoint{1.633550in}{0.773588in}}%
\pgfpathlineto{\pgfqpoint{1.558242in}{0.773588in}}%
\pgfpathlineto{\pgfqpoint{1.485557in}{0.773588in}}%
\pgfpathlineto{\pgfqpoint{1.410931in}{0.773588in}}%
\pgfpathlineto{\pgfqpoint{1.335256in}{0.773588in}}%
\pgfpathlineto{\pgfqpoint{1.261309in}{0.773588in}}%
\pgfpathlineto{\pgfqpoint{1.188027in}{0.773588in}}%
\pgfpathlineto{\pgfqpoint{1.111472in}{0.773588in}}%
\pgfpathlineto{\pgfqpoint{1.038437in}{0.773588in}}%
\pgfpathlineto{\pgfqpoint{0.968165in}{0.773588in}}%
\pgfpathlineto{\pgfqpoint{0.895203in}{0.773588in}}%
\pgfpathlineto{\pgfqpoint{0.824780in}{0.773588in}}%
\pgfpathlineto{\pgfqpoint{0.753203in}{0.773588in}}%
\pgfpathlineto{\pgfqpoint{0.679536in}{0.773588in}}%
\pgfpathlineto{\pgfqpoint{0.609267in}{0.773588in}}%
\pgfpathlineto{\pgfqpoint{0.537447in}{0.773588in}}%
\pgfpathlineto{\pgfqpoint{0.462885in}{0.773588in}}%
\pgfpathlineto{\pgfqpoint{0.391498in}{0.773588in}}%
\pgfpathlineto{\pgfqpoint{0.321303in}{0.773588in}}%
\pgfpathlineto{\pgfqpoint{0.248641in}{0.773588in}}%
\pgfpathlineto{\pgfqpoint{0.176284in}{0.773588in}}%
\pgfpathlineto{\pgfqpoint{0.103944in}{0.773588in}}%
\pgfpathlineto{\pgfqpoint{0.030045in}{0.773588in}}%
\pgfpathlineto{\pgfqpoint{-0.042097in}{0.773588in}}%
\pgfpathlineto{\pgfqpoint{-0.112821in}{0.773588in}}%
\pgfpathlineto{\pgfqpoint{-0.185623in}{0.773588in}}%
\pgfpathlineto{\pgfqpoint{-0.257459in}{0.773588in}}%
\pgfpathlineto{\pgfqpoint{-0.329323in}{0.773588in}}%
\pgfpathlineto{\pgfqpoint{-0.404497in}{0.773588in}}%
\pgfpathlineto{\pgfqpoint{-0.477784in}{0.773588in}}%
\pgfpathlineto{\pgfqpoint{-0.549722in}{0.773588in}}%
\pgfpathlineto{\pgfqpoint{-0.623932in}{0.773588in}}%
\pgfpathlineto{\pgfqpoint{-0.698122in}{0.773588in}}%
\pgfpathlineto{\pgfqpoint{-0.769987in}{0.773588in}}%
\pgfpathlineto{\pgfqpoint{-0.844408in}{0.773588in}}%
\pgfpathlineto{\pgfqpoint{-0.918138in}{0.773588in}}%
\pgfpathlineto{\pgfqpoint{-0.989522in}{0.773588in}}%
\pgfpathlineto{\pgfqpoint{-1.063245in}{0.773588in}}%
\pgfpathlineto{\pgfqpoint{-1.135138in}{0.773588in}}%
\pgfpathlineto{\pgfqpoint{-1.207908in}{0.773588in}}%
\pgfpathlineto{\pgfqpoint{-1.283200in}{0.773588in}}%
\pgfpathlineto{\pgfqpoint{-1.356794in}{0.773588in}}%
\pgfpathlineto{\pgfqpoint{-1.429441in}{0.773588in}}%
\pgfpathlineto{\pgfqpoint{-1.503984in}{0.773588in}}%
\pgfpathlineto{\pgfqpoint{-1.577590in}{0.773588in}}%
\pgfpathlineto{\pgfqpoint{-1.650581in}{0.773588in}}%
\pgfpathlineto{\pgfqpoint{-1.725105in}{0.773588in}}%
\pgfpathlineto{\pgfqpoint{-1.797123in}{0.773588in}}%
\pgfpathlineto{\pgfqpoint{-1.867809in}{0.773588in}}%
\pgfpathlineto{\pgfqpoint{-1.940536in}{0.773588in}}%
\pgfpathlineto{\pgfqpoint{-2.011817in}{0.773588in}}%
\pgfpathlineto{\pgfqpoint{-2.084338in}{0.773588in}}%
\pgfpathlineto{\pgfqpoint{-2.158577in}{0.773588in}}%
\pgfpathlineto{\pgfqpoint{-2.231477in}{0.773588in}}%
\pgfpathlineto{\pgfqpoint{-2.303263in}{0.773588in}}%
\pgfpathlineto{\pgfqpoint{-2.379035in}{0.773588in}}%
\pgfpathlineto{\pgfqpoint{-2.451489in}{0.773588in}}%
\pgfpathlineto{\pgfqpoint{-2.522349in}{0.773588in}}%
\pgfpathlineto{\pgfqpoint{-2.595405in}{0.773588in}}%
\pgfpathlineto{\pgfqpoint{-2.667020in}{0.773588in}}%
\pgfpathlineto{\pgfqpoint{-2.740325in}{0.773588in}}%
\pgfpathlineto{\pgfqpoint{-2.813631in}{0.773588in}}%
\pgfpathlineto{\pgfqpoint{-2.884033in}{0.773588in}}%
\pgfpathlineto{\pgfqpoint{-2.956196in}{0.773588in}}%
\pgfpathlineto{\pgfqpoint{-3.029702in}{0.773588in}}%
\pgfpathlineto{\pgfqpoint{-3.100293in}{0.773588in}}%
\pgfpathlineto{\pgfqpoint{-3.171735in}{0.773588in}}%
\pgfpathlineto{\pgfqpoint{-3.245632in}{0.773588in}}%
\pgfpathlineto{\pgfqpoint{-3.317677in}{0.773588in}}%
\pgfpathlineto{\pgfqpoint{-3.388885in}{0.773588in}}%
\pgfpathlineto{\pgfqpoint{-3.461876in}{0.773588in}}%
\pgfpathlineto{\pgfqpoint{-3.534002in}{0.773588in}}%
\pgfpathlineto{\pgfqpoint{-3.608325in}{0.773588in}}%
\pgfpathlineto{\pgfqpoint{-3.684119in}{0.773588in}}%
\pgfpathlineto{\pgfqpoint{-3.756893in}{0.773588in}}%
\pgfpathlineto{\pgfqpoint{-3.828230in}{0.773588in}}%
\pgfpathlineto{\pgfqpoint{-3.903164in}{0.773588in}}%
\pgfpathlineto{\pgfqpoint{-3.977308in}{0.773588in}}%
\pgfpathlineto{\pgfqpoint{-4.052173in}{0.773588in}}%
\pgfpathlineto{\pgfqpoint{-4.128352in}{0.773588in}}%
\pgfpathlineto{\pgfqpoint{-4.201246in}{0.773588in}}%
\pgfpathlineto{\pgfqpoint{-4.274748in}{0.773588in}}%
\pgfpathlineto{\pgfqpoint{-4.350759in}{0.773588in}}%
\pgfpathlineto{\pgfqpoint{-4.423289in}{0.773588in}}%
\pgfpathlineto{\pgfqpoint{-4.495401in}{0.773588in}}%
\pgfpathlineto{\pgfqpoint{-4.568595in}{0.773588in}}%
\pgfpathlineto{\pgfqpoint{-4.639570in}{0.773588in}}%
\pgfpathlineto{\pgfqpoint{-4.711002in}{0.773588in}}%
\pgfpathlineto{\pgfqpoint{-4.782965in}{0.773588in}}%
\pgfpathlineto{\pgfqpoint{-4.852652in}{0.773588in}}%
\pgfpathlineto{\pgfqpoint{-4.922664in}{0.773588in}}%
\pgfpathlineto{\pgfqpoint{-4.995002in}{0.773588in}}%
\pgfpathlineto{\pgfqpoint{-5.064817in}{0.773588in}}%
\pgfpathlineto{\pgfqpoint{-5.135005in}{0.773588in}}%
\pgfpathlineto{\pgfqpoint{-5.207317in}{0.773588in}}%
\pgfpathlineto{\pgfqpoint{-5.277970in}{0.773588in}}%
\pgfpathlineto{\pgfqpoint{-5.348665in}{0.773588in}}%
\pgfpathlineto{\pgfqpoint{-5.421890in}{0.773588in}}%
\pgfpathlineto{\pgfqpoint{-5.493615in}{0.773588in}}%
\pgfpathlineto{\pgfqpoint{-5.563648in}{0.773588in}}%
\pgfpathlineto{\pgfqpoint{-5.635685in}{0.773588in}}%
\pgfpathlineto{\pgfqpoint{-5.706570in}{0.773588in}}%
\pgfpathlineto{\pgfqpoint{-5.776542in}{0.773588in}}%
\pgfpathlineto{\pgfqpoint{-5.849895in}{0.773588in}}%
\pgfpathlineto{\pgfqpoint{-5.921215in}{0.773588in}}%
\pgfpathlineto{\pgfqpoint{-5.992127in}{0.773588in}}%
\pgfpathlineto{\pgfqpoint{-6.064664in}{0.773588in}}%
\pgfpathlineto{\pgfqpoint{-6.135189in}{0.773588in}}%
\pgfpathlineto{\pgfqpoint{-6.205113in}{0.773588in}}%
\pgfpathlineto{\pgfqpoint{-6.278887in}{0.773588in}}%
\pgfpathlineto{\pgfqpoint{-6.349020in}{0.773588in}}%
\pgfpathlineto{\pgfqpoint{-6.421127in}{0.773588in}}%
\pgfpathlineto{\pgfqpoint{-6.496669in}{0.773588in}}%
\pgfpathlineto{\pgfqpoint{-6.568712in}{0.773588in}}%
\pgfpathlineto{\pgfqpoint{-6.640496in}{0.773588in}}%
\pgfpathlineto{\pgfqpoint{-6.715245in}{0.773588in}}%
\pgfpathlineto{\pgfqpoint{-6.787495in}{0.773588in}}%
\pgfpathlineto{\pgfqpoint{-6.860576in}{0.773588in}}%
\pgfpathlineto{\pgfqpoint{-6.933558in}{0.773588in}}%
\pgfpathlineto{\pgfqpoint{-7.003598in}{0.773588in}}%
\pgfpathlineto{\pgfqpoint{-7.075897in}{0.773588in}}%
\pgfpathlineto{\pgfqpoint{-7.151909in}{0.773588in}}%
\pgfpathlineto{\pgfqpoint{-7.223147in}{0.773588in}}%
\pgfpathlineto{\pgfqpoint{-7.294929in}{0.773588in}}%
\pgfpathlineto{\pgfqpoint{-7.368338in}{0.773588in}}%
\pgfpathlineto{\pgfqpoint{-7.440452in}{0.773588in}}%
\pgfpathlineto{\pgfqpoint{-7.512981in}{0.773588in}}%
\pgfpathlineto{\pgfqpoint{-7.585515in}{0.773588in}}%
\pgfpathlineto{\pgfqpoint{-7.657024in}{0.773588in}}%
\pgfpathlineto{\pgfqpoint{-7.727286in}{0.773588in}}%
\pgfpathlineto{\pgfqpoint{-7.799875in}{0.773588in}}%
\pgfpathlineto{\pgfqpoint{-7.870788in}{0.773588in}}%
\pgfpathlineto{\pgfqpoint{-7.942200in}{0.773588in}}%
\pgfpathlineto{\pgfqpoint{-8.014387in}{0.773588in}}%
\pgfpathlineto{\pgfqpoint{-8.084913in}{0.773588in}}%
\pgfpathlineto{\pgfqpoint{-8.156111in}{0.773588in}}%
\pgfpathlineto{\pgfqpoint{-8.228241in}{0.773588in}}%
\pgfpathlineto{\pgfqpoint{-8.298753in}{0.773588in}}%
\pgfpathlineto{\pgfqpoint{-8.367967in}{0.773588in}}%
\pgfpathlineto{\pgfqpoint{-8.438970in}{0.773588in}}%
\pgfpathlineto{\pgfqpoint{-8.508839in}{0.773588in}}%
\pgfpathlineto{\pgfqpoint{-8.578369in}{0.773588in}}%
\pgfpathlineto{\pgfqpoint{-8.649341in}{0.773588in}}%
\pgfpathlineto{\pgfqpoint{-8.718609in}{0.773588in}}%
\pgfpathlineto{\pgfqpoint{-8.787586in}{0.773588in}}%
\pgfpathlineto{\pgfqpoint{-8.858221in}{0.773588in}}%
\pgfpathlineto{\pgfqpoint{-8.928123in}{0.773588in}}%
\pgfpathlineto{\pgfqpoint{-8.998701in}{0.773588in}}%
\pgfpathlineto{\pgfqpoint{-9.069721in}{0.773588in}}%
\pgfpathlineto{\pgfqpoint{-9.138788in}{0.773588in}}%
\pgfpathlineto{\pgfqpoint{-9.208906in}{0.773588in}}%
\pgfpathlineto{\pgfqpoint{-9.283412in}{0.773588in}}%
\pgfpathlineto{\pgfqpoint{-9.356529in}{0.773588in}}%
\pgfpathlineto{\pgfqpoint{-9.429774in}{0.773588in}}%
\pgfpathlineto{\pgfqpoint{-9.504196in}{0.773588in}}%
\pgfpathlineto{\pgfqpoint{-9.575400in}{0.773588in}}%
\pgfpathlineto{\pgfqpoint{-9.648369in}{0.773588in}}%
\pgfpathlineto{\pgfqpoint{-9.723682in}{0.773588in}}%
\pgfpathlineto{\pgfqpoint{-9.796010in}{0.773588in}}%
\pgfpathlineto{\pgfqpoint{-9.868243in}{0.773588in}}%
\pgfpathlineto{\pgfqpoint{-9.941359in}{0.773588in}}%
\pgfpathlineto{\pgfqpoint{-10.012635in}{0.773588in}}%
\pgfpathlineto{\pgfqpoint{-10.084496in}{0.773588in}}%
\pgfpathlineto{\pgfqpoint{-10.157398in}{0.773588in}}%
\pgfpathlineto{\pgfqpoint{-10.227739in}{0.773588in}}%
\pgfpathlineto{\pgfqpoint{-10.297591in}{0.773588in}}%
\pgfpathlineto{\pgfqpoint{-10.371004in}{0.773588in}}%
\pgfpathlineto{\pgfqpoint{-10.441581in}{0.773588in}}%
\pgfpathlineto{\pgfqpoint{-10.511372in}{0.773588in}}%
\pgfpathlineto{\pgfqpoint{-10.582864in}{0.773588in}}%
\pgfpathlineto{\pgfqpoint{-10.652649in}{0.773588in}}%
\pgfpathlineto{\pgfqpoint{-10.721651in}{0.773588in}}%
\pgfpathlineto{\pgfqpoint{-10.793422in}{0.773588in}}%
\pgfpathlineto{\pgfqpoint{-10.863257in}{0.773588in}}%
\pgfpathlineto{\pgfqpoint{-10.933808in}{0.773588in}}%
\pgfpathlineto{\pgfqpoint{-11.005065in}{0.773588in}}%
\pgfpathlineto{\pgfqpoint{-11.075276in}{0.773588in}}%
\pgfpathlineto{\pgfqpoint{-11.144960in}{0.773588in}}%
\pgfpathlineto{\pgfqpoint{-11.217631in}{0.773588in}}%
\pgfpathlineto{\pgfqpoint{-11.288273in}{0.773588in}}%
\pgfpathlineto{\pgfqpoint{-11.358676in}{0.773588in}}%
\pgfpathlineto{\pgfqpoint{-11.430351in}{0.773588in}}%
\pgfpathlineto{\pgfqpoint{-11.500166in}{0.773588in}}%
\pgfpathlineto{\pgfqpoint{-11.571037in}{0.773588in}}%
\pgfpathlineto{\pgfqpoint{-11.645187in}{0.773588in}}%
\pgfpathlineto{\pgfqpoint{-11.716309in}{0.773588in}}%
\pgfpathlineto{\pgfqpoint{-11.786176in}{0.773588in}}%
\pgfpathlineto{\pgfqpoint{-11.858471in}{0.773588in}}%
\pgfpathlineto{\pgfqpoint{-11.928381in}{0.773588in}}%
\pgfpathlineto{\pgfqpoint{-11.998596in}{0.773588in}}%
\pgfpathlineto{\pgfqpoint{-12.069446in}{0.773588in}}%
\pgfpathlineto{\pgfqpoint{-12.139885in}{0.773588in}}%
\pgfpathlineto{\pgfqpoint{-12.211536in}{0.773588in}}%
\pgfpathlineto{\pgfqpoint{-12.285275in}{0.773588in}}%
\pgfpathlineto{\pgfqpoint{-12.357103in}{0.773588in}}%
\pgfpathlineto{\pgfqpoint{-12.428997in}{0.773588in}}%
\pgfpathlineto{\pgfqpoint{-12.503401in}{0.773588in}}%
\pgfpathlineto{\pgfqpoint{-12.574652in}{0.773588in}}%
\pgfpathlineto{\pgfqpoint{-12.645279in}{0.773588in}}%
\pgfpathlineto{\pgfqpoint{-12.718096in}{0.773588in}}%
\pgfpathlineto{\pgfqpoint{-12.788679in}{0.773588in}}%
\pgfpathlineto{\pgfqpoint{-12.859398in}{0.773588in}}%
\pgfpathlineto{\pgfqpoint{-12.932388in}{0.773588in}}%
\pgfpathlineto{\pgfqpoint{-13.002727in}{0.773588in}}%
\pgfpathlineto{\pgfqpoint{-13.073093in}{0.773588in}}%
\pgfpathlineto{\pgfqpoint{-13.145337in}{0.773588in}}%
\pgfpathlineto{\pgfqpoint{-13.216013in}{0.773588in}}%
\pgfpathlineto{\pgfqpoint{-13.285456in}{0.773588in}}%
\pgfpathlineto{\pgfqpoint{-13.356968in}{0.773588in}}%
\pgfpathlineto{\pgfqpoint{-13.426276in}{0.773588in}}%
\pgfpathlineto{\pgfqpoint{-13.496151in}{0.773588in}}%
\pgfpathlineto{\pgfqpoint{-13.568713in}{0.773588in}}%
\pgfpathlineto{\pgfqpoint{-13.639609in}{0.773588in}}%
\pgfpathlineto{\pgfqpoint{-13.708089in}{0.773588in}}%
\pgfpathlineto{\pgfqpoint{-13.779339in}{0.773588in}}%
\pgfpathlineto{\pgfqpoint{-13.848756in}{0.773588in}}%
\pgfpathlineto{\pgfqpoint{-13.917455in}{0.773588in}}%
\pgfpathlineto{\pgfqpoint{-13.988983in}{0.773588in}}%
\pgfpathlineto{\pgfqpoint{-14.057377in}{0.773588in}}%
\pgfpathlineto{\pgfqpoint{-14.125666in}{0.773588in}}%
\pgfpathlineto{\pgfqpoint{-14.195714in}{0.773588in}}%
\pgfpathlineto{\pgfqpoint{-14.263436in}{0.773588in}}%
\pgfpathlineto{\pgfqpoint{-14.331277in}{0.773588in}}%
\pgfpathlineto{\pgfqpoint{-14.401957in}{0.773588in}}%
\pgfpathlineto{\pgfqpoint{-14.471123in}{0.773588in}}%
\pgfpathlineto{\pgfqpoint{-14.538988in}{0.773588in}}%
\pgfpathlineto{\pgfqpoint{-14.609938in}{0.773588in}}%
\pgfpathlineto{\pgfqpoint{-14.679263in}{0.773588in}}%
\pgfpathlineto{\pgfqpoint{-14.748835in}{0.773588in}}%
\pgfpathlineto{\pgfqpoint{-14.820408in}{0.773588in}}%
\pgfpathlineto{\pgfqpoint{-14.889438in}{0.773588in}}%
\pgfpathlineto{\pgfqpoint{-14.959865in}{0.773588in}}%
\pgfpathlineto{\pgfqpoint{-15.032843in}{0.773588in}}%
\pgfpathlineto{\pgfqpoint{-15.106256in}{0.773588in}}%
\pgfpathlineto{\pgfqpoint{-15.179182in}{0.773588in}}%
\pgfpathlineto{\pgfqpoint{-15.253001in}{0.773588in}}%
\pgfpathlineto{\pgfqpoint{-15.323000in}{0.773588in}}%
\pgfpathlineto{\pgfqpoint{-15.392048in}{0.773588in}}%
\pgfpathlineto{\pgfqpoint{-15.463558in}{0.773588in}}%
\pgfpathlineto{\pgfqpoint{-15.532639in}{0.773588in}}%
\pgfpathlineto{\pgfqpoint{-15.602559in}{0.773588in}}%
\pgfpathlineto{\pgfqpoint{-15.673559in}{0.773588in}}%
\pgfpathlineto{\pgfqpoint{-15.744769in}{0.773588in}}%
\pgfpathlineto{\pgfqpoint{-15.814335in}{0.773588in}}%
\pgfpathlineto{\pgfqpoint{-15.886757in}{0.773588in}}%
\pgfpathlineto{\pgfqpoint{-15.956358in}{0.773588in}}%
\pgfpathlineto{\pgfqpoint{-16.024276in}{0.773588in}}%
\pgfpathlineto{\pgfqpoint{-16.094895in}{0.773588in}}%
\pgfpathlineto{\pgfqpoint{-16.163654in}{0.773588in}}%
\pgfpathlineto{\pgfqpoint{-16.232068in}{0.773588in}}%
\pgfpathlineto{\pgfqpoint{-16.302639in}{0.773588in}}%
\pgfpathlineto{\pgfqpoint{-16.369724in}{0.773588in}}%
\pgfpathlineto{\pgfqpoint{-16.437097in}{0.773588in}}%
\pgfpathlineto{\pgfqpoint{-16.507797in}{0.773588in}}%
\pgfpathlineto{\pgfqpoint{-16.575864in}{0.773588in}}%
\pgfpathlineto{\pgfqpoint{-16.644038in}{0.773588in}}%
\pgfpathlineto{\pgfqpoint{-16.715630in}{0.773588in}}%
\pgfpathlineto{\pgfqpoint{-16.784012in}{0.773588in}}%
\pgfpathlineto{\pgfqpoint{-16.852234in}{0.773588in}}%
\pgfpathlineto{\pgfqpoint{-16.921714in}{0.773588in}}%
\pgfpathlineto{\pgfqpoint{-16.989953in}{0.773588in}}%
\pgfpathlineto{\pgfqpoint{-17.058127in}{0.773588in}}%
\pgfpathlineto{\pgfqpoint{-17.128098in}{0.773588in}}%
\pgfpathlineto{\pgfqpoint{-17.196258in}{0.773588in}}%
\pgfpathlineto{\pgfqpoint{-17.265552in}{0.773588in}}%
\pgfpathlineto{\pgfqpoint{-17.336134in}{0.773588in}}%
\pgfpathlineto{\pgfqpoint{-17.402675in}{0.773588in}}%
\pgfpathlineto{\pgfqpoint{-17.470673in}{0.773588in}}%
\pgfpathlineto{\pgfqpoint{-17.539598in}{0.773588in}}%
\pgfpathlineto{\pgfqpoint{-17.607453in}{0.773588in}}%
\pgfpathlineto{\pgfqpoint{-17.675451in}{0.773588in}}%
\pgfpathlineto{\pgfqpoint{-17.748008in}{0.773588in}}%
\pgfpathlineto{\pgfqpoint{-17.816737in}{0.773588in}}%
\pgfpathlineto{\pgfqpoint{-17.885356in}{0.773588in}}%
\pgfpathlineto{\pgfqpoint{-17.957185in}{0.773588in}}%
\pgfpathlineto{\pgfqpoint{-18.025876in}{0.773588in}}%
\pgfpathlineto{\pgfqpoint{-18.094404in}{0.773588in}}%
\pgfpathlineto{\pgfqpoint{-18.165145in}{0.773588in}}%
\pgfpathlineto{\pgfqpoint{-18.232823in}{0.773588in}}%
\pgfpathlineto{\pgfqpoint{-18.301710in}{0.773588in}}%
\pgfpathlineto{\pgfqpoint{-18.373957in}{0.773588in}}%
\pgfpathlineto{\pgfqpoint{-18.442527in}{0.773588in}}%
\pgfpathlineto{\pgfqpoint{-18.511159in}{0.773588in}}%
\pgfpathlineto{\pgfqpoint{-18.583201in}{0.773588in}}%
\pgfpathlineto{\pgfqpoint{-18.652004in}{0.773588in}}%
\pgfpathlineto{\pgfqpoint{-18.721737in}{0.773588in}}%
\pgfpathlineto{\pgfqpoint{-18.792270in}{0.773588in}}%
\pgfpathlineto{\pgfqpoint{-18.859965in}{0.773588in}}%
\pgfpathlineto{\pgfqpoint{-18.928709in}{0.773588in}}%
\pgfpathlineto{\pgfqpoint{-18.999640in}{0.773588in}}%
\pgfpathlineto{\pgfqpoint{-19.068147in}{0.773588in}}%
\pgfpathlineto{\pgfqpoint{-19.135788in}{0.773588in}}%
\pgfpathlineto{\pgfqpoint{-19.205228in}{0.773588in}}%
\pgfpathlineto{\pgfqpoint{-19.271476in}{0.773588in}}%
\pgfpathlineto{\pgfqpoint{-19.338364in}{0.773588in}}%
\pgfpathlineto{\pgfqpoint{-19.407422in}{0.773588in}}%
\pgfpathlineto{\pgfqpoint{-19.475599in}{0.773588in}}%
\pgfpathlineto{\pgfqpoint{-19.543393in}{0.773588in}}%
\pgfpathlineto{\pgfqpoint{-19.613839in}{0.773588in}}%
\pgfpathlineto{\pgfqpoint{-19.680997in}{0.773588in}}%
\pgfpathlineto{\pgfqpoint{-19.747542in}{0.773588in}}%
\pgfpathlineto{\pgfqpoint{-19.815535in}{0.773588in}}%
\pgfpathlineto{\pgfqpoint{-19.882267in}{0.773588in}}%
\pgfpathlineto{\pgfqpoint{-19.949841in}{0.773588in}}%
\pgfpathlineto{\pgfqpoint{-20.019359in}{0.773588in}}%
\pgfpathlineto{\pgfqpoint{-20.086943in}{0.773588in}}%
\pgfpathlineto{\pgfqpoint{-20.154101in}{0.773588in}}%
\pgfpathlineto{\pgfqpoint{-20.223560in}{0.773588in}}%
\pgfpathlineto{\pgfqpoint{-20.290460in}{0.773588in}}%
\pgfpathlineto{\pgfqpoint{-20.357252in}{0.773588in}}%
\pgfpathlineto{\pgfqpoint{-20.427103in}{0.773588in}}%
\pgfpathlineto{\pgfqpoint{-20.496847in}{0.773588in}}%
\pgfpathlineto{\pgfqpoint{-20.565295in}{0.773588in}}%
\pgfpathlineto{\pgfqpoint{-20.636228in}{0.773588in}}%
\pgfpathlineto{\pgfqpoint{-20.705385in}{0.773588in}}%
\pgfpathlineto{\pgfqpoint{-20.774312in}{0.773588in}}%
\pgfpathlineto{\pgfqpoint{-20.844220in}{0.773588in}}%
\pgfpathlineto{\pgfqpoint{-20.913231in}{0.773588in}}%
\pgfpathlineto{\pgfqpoint{-20.982338in}{0.773588in}}%
\pgfpathlineto{\pgfqpoint{-21.053709in}{0.773588in}}%
\pgfpathlineto{\pgfqpoint{-21.123238in}{0.773588in}}%
\pgfpathlineto{\pgfqpoint{-21.191384in}{0.773588in}}%
\pgfpathlineto{\pgfqpoint{-21.261541in}{0.773588in}}%
\pgfpathlineto{\pgfqpoint{-21.330134in}{0.773588in}}%
\pgfpathlineto{\pgfqpoint{-21.398346in}{0.773588in}}%
\pgfpathlineto{\pgfqpoint{-21.470634in}{0.773588in}}%
\pgfpathlineto{\pgfqpoint{-21.539549in}{0.773588in}}%
\pgfpathlineto{\pgfqpoint{-21.607622in}{0.773588in}}%
\pgfpathlineto{\pgfqpoint{-21.676089in}{0.773588in}}%
\pgfpathlineto{\pgfqpoint{-21.743422in}{0.773588in}}%
\pgfpathlineto{\pgfqpoint{-21.811964in}{0.773588in}}%
\pgfpathlineto{\pgfqpoint{-21.881469in}{0.773588in}}%
\pgfpathlineto{\pgfqpoint{-21.948488in}{0.773588in}}%
\pgfpathlineto{\pgfqpoint{-22.016091in}{0.773588in}}%
\pgfpathlineto{\pgfqpoint{-22.084923in}{0.773588in}}%
\pgfpathlineto{\pgfqpoint{-22.150098in}{0.773588in}}%
\pgfpathlineto{\pgfqpoint{-22.217508in}{0.773588in}}%
\pgfpathlineto{\pgfqpoint{-22.286974in}{0.773588in}}%
\pgfpathlineto{\pgfqpoint{-22.353759in}{0.773588in}}%
\pgfpathlineto{\pgfqpoint{-22.421718in}{0.773588in}}%
\pgfpathlineto{\pgfqpoint{-22.492630in}{0.773588in}}%
\pgfpathlineto{\pgfqpoint{-22.560590in}{0.773588in}}%
\pgfpathlineto{\pgfqpoint{-22.627376in}{0.773588in}}%
\pgfpathlineto{\pgfqpoint{-22.696138in}{0.773588in}}%
\pgfpathlineto{\pgfqpoint{-22.764599in}{0.773588in}}%
\pgfpathlineto{\pgfqpoint{-22.831621in}{0.773588in}}%
\pgfpathlineto{\pgfqpoint{-22.900479in}{0.773588in}}%
\pgfpathlineto{\pgfqpoint{-22.968635in}{0.773588in}}%
\pgfpathlineto{\pgfqpoint{-23.037151in}{0.773588in}}%
\pgfpathlineto{\pgfqpoint{-23.107951in}{0.773588in}}%
\pgfpathlineto{\pgfqpoint{-23.175783in}{0.773588in}}%
\pgfpathlineto{\pgfqpoint{-23.243519in}{0.773588in}}%
\pgfpathlineto{\pgfqpoint{-23.314406in}{0.773588in}}%
\pgfpathlineto{\pgfqpoint{-23.383303in}{0.773588in}}%
\pgfpathlineto{\pgfqpoint{-23.451780in}{0.773588in}}%
\pgfpathlineto{\pgfqpoint{-23.523979in}{0.773588in}}%
\pgfpathlineto{\pgfqpoint{-23.594251in}{0.773588in}}%
\pgfpathlineto{\pgfqpoint{-23.664795in}{0.773588in}}%
\pgfpathlineto{\pgfqpoint{-23.739951in}{0.773588in}}%
\pgfpathlineto{\pgfqpoint{-23.810815in}{0.773588in}}%
\pgfpathlineto{\pgfqpoint{-23.881105in}{0.773588in}}%
\pgfpathlineto{\pgfqpoint{-23.953546in}{0.773588in}}%
\pgfpathlineto{\pgfqpoint{-24.025129in}{0.773588in}}%
\pgfpathlineto{\pgfqpoint{-24.096931in}{0.773588in}}%
\pgfpathlineto{\pgfqpoint{-24.172427in}{0.773588in}}%
\pgfpathlineto{\pgfqpoint{-24.243942in}{0.773588in}}%
\pgfpathlineto{\pgfqpoint{-24.313406in}{0.773588in}}%
\pgfpathlineto{\pgfqpoint{-24.381203in}{0.773588in}}%
\pgfpathlineto{\pgfqpoint{-24.445935in}{0.773588in}}%
\pgfpathlineto{\pgfqpoint{-24.511431in}{0.773588in}}%
\pgfpathlineto{\pgfqpoint{-24.578904in}{0.773588in}}%
\pgfpathlineto{\pgfqpoint{-24.645693in}{0.773588in}}%
\pgfpathlineto{\pgfqpoint{-24.712889in}{0.773588in}}%
\pgfpathlineto{\pgfqpoint{-24.781655in}{0.773588in}}%
\pgfpathlineto{\pgfqpoint{-24.847966in}{0.773588in}}%
\pgfpathlineto{\pgfqpoint{-24.916019in}{0.773588in}}%
\pgfpathlineto{\pgfqpoint{-24.984232in}{0.773588in}}%
\pgfpathlineto{\pgfqpoint{-25.049351in}{0.773588in}}%
\pgfpathlineto{\pgfqpoint{-25.114674in}{0.773588in}}%
\pgfpathlineto{\pgfqpoint{-25.182394in}{0.773588in}}%
\pgfpathlineto{\pgfqpoint{-25.248757in}{0.773588in}}%
\pgfpathlineto{\pgfqpoint{-25.315375in}{0.773588in}}%
\pgfpathlineto{\pgfqpoint{-25.382786in}{0.773588in}}%
\pgfpathlineto{\pgfqpoint{-25.448976in}{0.773588in}}%
\pgfpathlineto{\pgfqpoint{-25.515370in}{0.773588in}}%
\pgfpathlineto{\pgfqpoint{-25.583742in}{0.773588in}}%
\pgfpathlineto{\pgfqpoint{-25.650614in}{0.773588in}}%
\pgfpathlineto{\pgfqpoint{-25.717568in}{0.773588in}}%
\pgfpathlineto{\pgfqpoint{-25.785528in}{0.773588in}}%
\pgfpathlineto{\pgfqpoint{-25.851769in}{0.773588in}}%
\pgfpathlineto{\pgfqpoint{-25.919100in}{0.773588in}}%
\pgfpathlineto{\pgfqpoint{-25.990773in}{0.773588in}}%
\pgfpathlineto{\pgfqpoint{-26.059544in}{0.773588in}}%
\pgfpathlineto{\pgfqpoint{-26.128033in}{0.773588in}}%
\pgfpathlineto{\pgfqpoint{-26.198294in}{0.773588in}}%
\pgfpathlineto{\pgfqpoint{-26.268985in}{0.773588in}}%
\pgfpathlineto{\pgfqpoint{-26.338662in}{0.773588in}}%
\pgfpathlineto{\pgfqpoint{-26.410031in}{0.773588in}}%
\pgfpathlineto{\pgfqpoint{-26.479146in}{0.773588in}}%
\pgfpathlineto{\pgfqpoint{-26.547093in}{0.773588in}}%
\pgfpathlineto{\pgfqpoint{-26.616985in}{0.773588in}}%
\pgfpathlineto{\pgfqpoint{-26.685925in}{0.773588in}}%
\pgfpathlineto{\pgfqpoint{-26.755672in}{0.773588in}}%
\pgfpathlineto{\pgfqpoint{-26.825734in}{0.773588in}}%
\pgfpathlineto{\pgfqpoint{-26.893662in}{0.773588in}}%
\pgfpathlineto{\pgfqpoint{-26.961733in}{0.773588in}}%
\pgfpathlineto{\pgfqpoint{-27.031677in}{0.773588in}}%
\pgfpathlineto{\pgfqpoint{-27.097972in}{0.773588in}}%
\pgfpathlineto{\pgfqpoint{-27.165061in}{0.773588in}}%
\pgfpathlineto{\pgfqpoint{-27.232937in}{0.773588in}}%
\pgfpathlineto{\pgfqpoint{-27.299189in}{0.773588in}}%
\pgfpathlineto{\pgfqpoint{-27.366087in}{0.773588in}}%
\pgfpathlineto{\pgfqpoint{-27.435671in}{0.773588in}}%
\pgfpathlineto{\pgfqpoint{-27.503250in}{0.773588in}}%
\pgfpathlineto{\pgfqpoint{-27.570291in}{0.773588in}}%
\pgfpathlineto{\pgfqpoint{-27.637925in}{0.773588in}}%
\pgfpathlineto{\pgfqpoint{-27.703951in}{0.773588in}}%
\pgfpathlineto{\pgfqpoint{-27.771032in}{0.773588in}}%
\pgfpathlineto{\pgfqpoint{-27.839702in}{0.773588in}}%
\pgfpathlineto{\pgfqpoint{-27.908279in}{0.773588in}}%
\pgfpathlineto{\pgfqpoint{-27.976694in}{0.773588in}}%
\pgfpathlineto{\pgfqpoint{-28.049197in}{0.773588in}}%
\pgfpathlineto{\pgfqpoint{-28.121663in}{0.773588in}}%
\pgfpathlineto{\pgfqpoint{-28.195772in}{0.773588in}}%
\pgfpathlineto{\pgfqpoint{-28.274638in}{0.773588in}}%
\pgfpathlineto{\pgfqpoint{-28.356803in}{0.773588in}}%
\pgfpathlineto{\pgfqpoint{-28.429123in}{0.773588in}}%
\pgfpathlineto{\pgfqpoint{-28.504596in}{0.773588in}}%
\pgfpathlineto{\pgfqpoint{-28.577221in}{0.773588in}}%
\pgfpathlineto{\pgfqpoint{-28.649563in}{0.773588in}}%
\pgfpathlineto{\pgfqpoint{-28.725736in}{0.773588in}}%
\pgfpathlineto{\pgfqpoint{-28.798746in}{0.773588in}}%
\pgfpathlineto{\pgfqpoint{-28.870906in}{0.773588in}}%
\pgfpathlineto{\pgfqpoint{-28.946213in}{0.773588in}}%
\pgfpathlineto{\pgfqpoint{-29.018119in}{0.773588in}}%
\pgfpathlineto{\pgfqpoint{-29.089467in}{0.773588in}}%
\pgfpathlineto{\pgfqpoint{-29.164288in}{0.773588in}}%
\pgfpathlineto{\pgfqpoint{-29.235084in}{0.773588in}}%
\pgfpathlineto{\pgfqpoint{-29.304065in}{0.773588in}}%
\pgfpathlineto{\pgfqpoint{-29.374900in}{0.773588in}}%
\pgfpathlineto{\pgfqpoint{-29.443353in}{0.773588in}}%
\pgfpathlineto{\pgfqpoint{-29.511058in}{0.773588in}}%
\pgfpathlineto{\pgfqpoint{-29.581285in}{0.773588in}}%
\pgfpathlineto{\pgfqpoint{-29.647778in}{0.773588in}}%
\pgfpathlineto{\pgfqpoint{-29.716346in}{0.773588in}}%
\pgfpathlineto{\pgfqpoint{-29.785834in}{0.773588in}}%
\pgfpathlineto{\pgfqpoint{-29.853575in}{0.773588in}}%
\pgfpathlineto{\pgfqpoint{-29.923374in}{0.773588in}}%
\pgfpathlineto{\pgfqpoint{-29.994776in}{0.773588in}}%
\pgfpathlineto{\pgfqpoint{-30.062931in}{0.773588in}}%
\pgfpathlineto{\pgfqpoint{-30.131142in}{0.773588in}}%
\pgfpathlineto{\pgfqpoint{-30.202864in}{0.773588in}}%
\pgfpathlineto{\pgfqpoint{-30.269201in}{0.773588in}}%
\pgfpathlineto{\pgfqpoint{-30.335915in}{0.773588in}}%
\pgfpathlineto{\pgfqpoint{-30.404868in}{0.773588in}}%
\pgfpathlineto{\pgfqpoint{-30.472294in}{0.773588in}}%
\pgfpathlineto{\pgfqpoint{-30.539383in}{0.773588in}}%
\pgfpathlineto{\pgfqpoint{-30.610127in}{0.773588in}}%
\pgfpathlineto{\pgfqpoint{-30.676901in}{0.773588in}}%
\pgfpathlineto{\pgfqpoint{-30.745143in}{0.773588in}}%
\pgfpathclose%
\pgfusepath{fill}%
\end{pgfscope}%
\begin{pgfscope}%
\pgfpathrectangle{\pgfqpoint{3.332180in}{0.773588in}}{\pgfqpoint{2.293918in}{5.415119in}}%
\pgfusepath{clip}%
\pgfsetbuttcap%
\pgfsetroundjoin%
\definecolor{currentfill}{rgb}{1.000000,0.498039,0.054902}%
\pgfsetfillcolor{currentfill}%
\pgfsetlinewidth{0.000000pt}%
\definecolor{currentstroke}{rgb}{0.000000,0.000000,0.000000}%
\pgfsetstrokecolor{currentstroke}%
\pgfsetdash{}{0pt}%
\pgfpathmoveto{\pgfqpoint{-30.745143in}{0.773588in}}%
\pgfpathlineto{\pgfqpoint{-30.745143in}{0.773588in}}%
\pgfpathlineto{\pgfqpoint{-30.676901in}{0.773588in}}%
\pgfpathlineto{\pgfqpoint{-30.610127in}{0.773588in}}%
\pgfpathlineto{\pgfqpoint{-30.539383in}{0.773588in}}%
\pgfpathlineto{\pgfqpoint{-30.472294in}{0.773588in}}%
\pgfpathlineto{\pgfqpoint{-30.404868in}{0.773588in}}%
\pgfpathlineto{\pgfqpoint{-30.335915in}{0.773588in}}%
\pgfpathlineto{\pgfqpoint{-30.269201in}{0.773588in}}%
\pgfpathlineto{\pgfqpoint{-30.202864in}{0.773588in}}%
\pgfpathlineto{\pgfqpoint{-30.131142in}{0.773588in}}%
\pgfpathlineto{\pgfqpoint{-30.062931in}{0.773588in}}%
\pgfpathlineto{\pgfqpoint{-29.994776in}{0.773588in}}%
\pgfpathlineto{\pgfqpoint{-29.923374in}{0.773588in}}%
\pgfpathlineto{\pgfqpoint{-29.853575in}{0.773588in}}%
\pgfpathlineto{\pgfqpoint{-29.785834in}{0.773588in}}%
\pgfpathlineto{\pgfqpoint{-29.716346in}{0.773588in}}%
\pgfpathlineto{\pgfqpoint{-29.647778in}{0.773588in}}%
\pgfpathlineto{\pgfqpoint{-29.581285in}{0.773588in}}%
\pgfpathlineto{\pgfqpoint{-29.511058in}{0.773588in}}%
\pgfpathlineto{\pgfqpoint{-29.443353in}{0.773588in}}%
\pgfpathlineto{\pgfqpoint{-29.374900in}{0.773588in}}%
\pgfpathlineto{\pgfqpoint{-29.304065in}{0.773588in}}%
\pgfpathlineto{\pgfqpoint{-29.235084in}{0.773588in}}%
\pgfpathlineto{\pgfqpoint{-29.164288in}{0.773588in}}%
\pgfpathlineto{\pgfqpoint{-29.089467in}{0.773588in}}%
\pgfpathlineto{\pgfqpoint{-29.018119in}{0.773588in}}%
\pgfpathlineto{\pgfqpoint{-28.946213in}{0.773588in}}%
\pgfpathlineto{\pgfqpoint{-28.870906in}{0.773588in}}%
\pgfpathlineto{\pgfqpoint{-28.798746in}{0.773588in}}%
\pgfpathlineto{\pgfqpoint{-28.725736in}{0.773588in}}%
\pgfpathlineto{\pgfqpoint{-28.649563in}{0.773588in}}%
\pgfpathlineto{\pgfqpoint{-28.577221in}{0.773588in}}%
\pgfpathlineto{\pgfqpoint{-28.504596in}{0.773588in}}%
\pgfpathlineto{\pgfqpoint{-28.429123in}{0.773588in}}%
\pgfpathlineto{\pgfqpoint{-28.356803in}{0.773588in}}%
\pgfpathlineto{\pgfqpoint{-28.274638in}{0.773588in}}%
\pgfpathlineto{\pgfqpoint{-28.195772in}{0.773588in}}%
\pgfpathlineto{\pgfqpoint{-28.121663in}{0.773588in}}%
\pgfpathlineto{\pgfqpoint{-28.049197in}{0.773588in}}%
\pgfpathlineto{\pgfqpoint{-27.976694in}{0.773588in}}%
\pgfpathlineto{\pgfqpoint{-27.908279in}{0.773588in}}%
\pgfpathlineto{\pgfqpoint{-27.839702in}{0.773588in}}%
\pgfpathlineto{\pgfqpoint{-27.771032in}{0.773588in}}%
\pgfpathlineto{\pgfqpoint{-27.703951in}{0.773588in}}%
\pgfpathlineto{\pgfqpoint{-27.637925in}{0.773588in}}%
\pgfpathlineto{\pgfqpoint{-27.570291in}{0.773588in}}%
\pgfpathlineto{\pgfqpoint{-27.503250in}{0.773588in}}%
\pgfpathlineto{\pgfqpoint{-27.435671in}{0.773588in}}%
\pgfpathlineto{\pgfqpoint{-27.366087in}{0.773588in}}%
\pgfpathlineto{\pgfqpoint{-27.299189in}{0.773588in}}%
\pgfpathlineto{\pgfqpoint{-27.232937in}{0.773588in}}%
\pgfpathlineto{\pgfqpoint{-27.165061in}{0.773588in}}%
\pgfpathlineto{\pgfqpoint{-27.097972in}{0.773588in}}%
\pgfpathlineto{\pgfqpoint{-27.031677in}{0.773588in}}%
\pgfpathlineto{\pgfqpoint{-26.961733in}{0.773588in}}%
\pgfpathlineto{\pgfqpoint{-26.893662in}{0.773588in}}%
\pgfpathlineto{\pgfqpoint{-26.825734in}{0.773588in}}%
\pgfpathlineto{\pgfqpoint{-26.755672in}{0.773588in}}%
\pgfpathlineto{\pgfqpoint{-26.685925in}{0.773588in}}%
\pgfpathlineto{\pgfqpoint{-26.616985in}{0.773588in}}%
\pgfpathlineto{\pgfqpoint{-26.547093in}{0.773588in}}%
\pgfpathlineto{\pgfqpoint{-26.479146in}{0.773588in}}%
\pgfpathlineto{\pgfqpoint{-26.410031in}{0.773588in}}%
\pgfpathlineto{\pgfqpoint{-26.338662in}{0.773588in}}%
\pgfpathlineto{\pgfqpoint{-26.268985in}{0.773588in}}%
\pgfpathlineto{\pgfqpoint{-26.198294in}{0.773588in}}%
\pgfpathlineto{\pgfqpoint{-26.128033in}{0.773588in}}%
\pgfpathlineto{\pgfqpoint{-26.059544in}{0.773588in}}%
\pgfpathlineto{\pgfqpoint{-25.990773in}{0.773588in}}%
\pgfpathlineto{\pgfqpoint{-25.919100in}{0.773588in}}%
\pgfpathlineto{\pgfqpoint{-25.851769in}{0.773588in}}%
\pgfpathlineto{\pgfqpoint{-25.785528in}{0.773588in}}%
\pgfpathlineto{\pgfqpoint{-25.717568in}{0.773588in}}%
\pgfpathlineto{\pgfqpoint{-25.650614in}{0.773588in}}%
\pgfpathlineto{\pgfqpoint{-25.583742in}{0.773588in}}%
\pgfpathlineto{\pgfqpoint{-25.515370in}{0.773588in}}%
\pgfpathlineto{\pgfqpoint{-25.448976in}{0.773588in}}%
\pgfpathlineto{\pgfqpoint{-25.382786in}{0.773588in}}%
\pgfpathlineto{\pgfqpoint{-25.315375in}{0.773588in}}%
\pgfpathlineto{\pgfqpoint{-25.248757in}{0.773588in}}%
\pgfpathlineto{\pgfqpoint{-25.182394in}{0.773588in}}%
\pgfpathlineto{\pgfqpoint{-25.114674in}{0.773588in}}%
\pgfpathlineto{\pgfqpoint{-25.049351in}{0.773588in}}%
\pgfpathlineto{\pgfqpoint{-24.984232in}{0.773588in}}%
\pgfpathlineto{\pgfqpoint{-24.916019in}{0.773588in}}%
\pgfpathlineto{\pgfqpoint{-24.847966in}{0.773588in}}%
\pgfpathlineto{\pgfqpoint{-24.781655in}{0.773588in}}%
\pgfpathlineto{\pgfqpoint{-24.712889in}{0.773588in}}%
\pgfpathlineto{\pgfqpoint{-24.645693in}{0.773588in}}%
\pgfpathlineto{\pgfqpoint{-24.578904in}{0.773588in}}%
\pgfpathlineto{\pgfqpoint{-24.511431in}{0.773588in}}%
\pgfpathlineto{\pgfqpoint{-24.445935in}{0.773588in}}%
\pgfpathlineto{\pgfqpoint{-24.381203in}{0.773588in}}%
\pgfpathlineto{\pgfqpoint{-24.313406in}{0.773588in}}%
\pgfpathlineto{\pgfqpoint{-24.243942in}{0.773588in}}%
\pgfpathlineto{\pgfqpoint{-24.172427in}{0.773588in}}%
\pgfpathlineto{\pgfqpoint{-24.096931in}{0.773588in}}%
\pgfpathlineto{\pgfqpoint{-24.025129in}{0.773588in}}%
\pgfpathlineto{\pgfqpoint{-23.953546in}{0.773588in}}%
\pgfpathlineto{\pgfqpoint{-23.881105in}{0.773588in}}%
\pgfpathlineto{\pgfqpoint{-23.810815in}{0.773588in}}%
\pgfpathlineto{\pgfqpoint{-23.739951in}{0.773588in}}%
\pgfpathlineto{\pgfqpoint{-23.664795in}{0.773588in}}%
\pgfpathlineto{\pgfqpoint{-23.594251in}{0.773588in}}%
\pgfpathlineto{\pgfqpoint{-23.523979in}{0.773588in}}%
\pgfpathlineto{\pgfqpoint{-23.451780in}{0.773588in}}%
\pgfpathlineto{\pgfqpoint{-23.383303in}{0.773588in}}%
\pgfpathlineto{\pgfqpoint{-23.314406in}{0.773588in}}%
\pgfpathlineto{\pgfqpoint{-23.243519in}{0.773588in}}%
\pgfpathlineto{\pgfqpoint{-23.175783in}{0.773588in}}%
\pgfpathlineto{\pgfqpoint{-23.107951in}{0.773588in}}%
\pgfpathlineto{\pgfqpoint{-23.037151in}{0.773588in}}%
\pgfpathlineto{\pgfqpoint{-22.968635in}{0.773588in}}%
\pgfpathlineto{\pgfqpoint{-22.900479in}{0.773588in}}%
\pgfpathlineto{\pgfqpoint{-22.831621in}{0.773588in}}%
\pgfpathlineto{\pgfqpoint{-22.764599in}{0.773588in}}%
\pgfpathlineto{\pgfqpoint{-22.696138in}{0.773588in}}%
\pgfpathlineto{\pgfqpoint{-22.627376in}{0.773588in}}%
\pgfpathlineto{\pgfqpoint{-22.560590in}{0.773588in}}%
\pgfpathlineto{\pgfqpoint{-22.492630in}{0.773588in}}%
\pgfpathlineto{\pgfqpoint{-22.421718in}{0.773588in}}%
\pgfpathlineto{\pgfqpoint{-22.353759in}{0.773588in}}%
\pgfpathlineto{\pgfqpoint{-22.286974in}{0.773588in}}%
\pgfpathlineto{\pgfqpoint{-22.217508in}{0.773588in}}%
\pgfpathlineto{\pgfqpoint{-22.150098in}{0.773588in}}%
\pgfpathlineto{\pgfqpoint{-22.084923in}{0.773588in}}%
\pgfpathlineto{\pgfqpoint{-22.016091in}{0.773588in}}%
\pgfpathlineto{\pgfqpoint{-21.948488in}{0.773588in}}%
\pgfpathlineto{\pgfqpoint{-21.881469in}{0.773588in}}%
\pgfpathlineto{\pgfqpoint{-21.811964in}{0.773588in}}%
\pgfpathlineto{\pgfqpoint{-21.743422in}{0.773588in}}%
\pgfpathlineto{\pgfqpoint{-21.676089in}{0.773588in}}%
\pgfpathlineto{\pgfqpoint{-21.607622in}{0.773588in}}%
\pgfpathlineto{\pgfqpoint{-21.539549in}{0.773588in}}%
\pgfpathlineto{\pgfqpoint{-21.470634in}{0.773588in}}%
\pgfpathlineto{\pgfqpoint{-21.398346in}{0.773588in}}%
\pgfpathlineto{\pgfqpoint{-21.330134in}{0.773588in}}%
\pgfpathlineto{\pgfqpoint{-21.261541in}{0.773588in}}%
\pgfpathlineto{\pgfqpoint{-21.191384in}{0.773588in}}%
\pgfpathlineto{\pgfqpoint{-21.123238in}{0.773588in}}%
\pgfpathlineto{\pgfqpoint{-21.053709in}{0.773588in}}%
\pgfpathlineto{\pgfqpoint{-20.982338in}{0.773588in}}%
\pgfpathlineto{\pgfqpoint{-20.913231in}{0.773588in}}%
\pgfpathlineto{\pgfqpoint{-20.844220in}{0.773588in}}%
\pgfpathlineto{\pgfqpoint{-20.774312in}{0.773588in}}%
\pgfpathlineto{\pgfqpoint{-20.705385in}{0.773588in}}%
\pgfpathlineto{\pgfqpoint{-20.636228in}{0.773588in}}%
\pgfpathlineto{\pgfqpoint{-20.565295in}{0.773588in}}%
\pgfpathlineto{\pgfqpoint{-20.496847in}{0.773588in}}%
\pgfpathlineto{\pgfqpoint{-20.427103in}{0.773588in}}%
\pgfpathlineto{\pgfqpoint{-20.357252in}{0.773588in}}%
\pgfpathlineto{\pgfqpoint{-20.290460in}{0.773588in}}%
\pgfpathlineto{\pgfqpoint{-20.223560in}{0.773588in}}%
\pgfpathlineto{\pgfqpoint{-20.154101in}{0.773588in}}%
\pgfpathlineto{\pgfqpoint{-20.086943in}{0.773588in}}%
\pgfpathlineto{\pgfqpoint{-20.019359in}{0.773588in}}%
\pgfpathlineto{\pgfqpoint{-19.949841in}{0.773588in}}%
\pgfpathlineto{\pgfqpoint{-19.882267in}{0.773588in}}%
\pgfpathlineto{\pgfqpoint{-19.815535in}{0.773588in}}%
\pgfpathlineto{\pgfqpoint{-19.747542in}{0.773588in}}%
\pgfpathlineto{\pgfqpoint{-19.680997in}{0.773588in}}%
\pgfpathlineto{\pgfqpoint{-19.613839in}{0.773588in}}%
\pgfpathlineto{\pgfqpoint{-19.543393in}{0.773588in}}%
\pgfpathlineto{\pgfqpoint{-19.475599in}{0.773588in}}%
\pgfpathlineto{\pgfqpoint{-19.407422in}{0.773588in}}%
\pgfpathlineto{\pgfqpoint{-19.338364in}{0.773588in}}%
\pgfpathlineto{\pgfqpoint{-19.271476in}{0.773588in}}%
\pgfpathlineto{\pgfqpoint{-19.205228in}{0.773588in}}%
\pgfpathlineto{\pgfqpoint{-19.135788in}{0.773588in}}%
\pgfpathlineto{\pgfqpoint{-19.068147in}{0.773588in}}%
\pgfpathlineto{\pgfqpoint{-18.999640in}{0.773588in}}%
\pgfpathlineto{\pgfqpoint{-18.928709in}{0.773588in}}%
\pgfpathlineto{\pgfqpoint{-18.859965in}{0.773588in}}%
\pgfpathlineto{\pgfqpoint{-18.792270in}{0.773588in}}%
\pgfpathlineto{\pgfqpoint{-18.721737in}{0.773588in}}%
\pgfpathlineto{\pgfqpoint{-18.652004in}{0.773588in}}%
\pgfpathlineto{\pgfqpoint{-18.583201in}{0.773588in}}%
\pgfpathlineto{\pgfqpoint{-18.511159in}{0.773588in}}%
\pgfpathlineto{\pgfqpoint{-18.442527in}{0.773588in}}%
\pgfpathlineto{\pgfqpoint{-18.373957in}{0.773588in}}%
\pgfpathlineto{\pgfqpoint{-18.301710in}{0.773588in}}%
\pgfpathlineto{\pgfqpoint{-18.232823in}{0.773588in}}%
\pgfpathlineto{\pgfqpoint{-18.165145in}{0.773588in}}%
\pgfpathlineto{\pgfqpoint{-18.094404in}{0.773588in}}%
\pgfpathlineto{\pgfqpoint{-18.025876in}{0.773588in}}%
\pgfpathlineto{\pgfqpoint{-17.957185in}{0.773588in}}%
\pgfpathlineto{\pgfqpoint{-17.885356in}{0.773588in}}%
\pgfpathlineto{\pgfqpoint{-17.816737in}{0.773588in}}%
\pgfpathlineto{\pgfqpoint{-17.748008in}{0.773588in}}%
\pgfpathlineto{\pgfqpoint{-17.675451in}{0.773588in}}%
\pgfpathlineto{\pgfqpoint{-17.607453in}{0.773588in}}%
\pgfpathlineto{\pgfqpoint{-17.539598in}{0.773588in}}%
\pgfpathlineto{\pgfqpoint{-17.470673in}{0.773588in}}%
\pgfpathlineto{\pgfqpoint{-17.402675in}{0.773588in}}%
\pgfpathlineto{\pgfqpoint{-17.336134in}{0.773588in}}%
\pgfpathlineto{\pgfqpoint{-17.265552in}{0.773588in}}%
\pgfpathlineto{\pgfqpoint{-17.196258in}{0.773588in}}%
\pgfpathlineto{\pgfqpoint{-17.128098in}{0.773588in}}%
\pgfpathlineto{\pgfqpoint{-17.058127in}{0.773588in}}%
\pgfpathlineto{\pgfqpoint{-16.989953in}{0.773588in}}%
\pgfpathlineto{\pgfqpoint{-16.921714in}{0.773588in}}%
\pgfpathlineto{\pgfqpoint{-16.852234in}{0.773588in}}%
\pgfpathlineto{\pgfqpoint{-16.784012in}{0.773588in}}%
\pgfpathlineto{\pgfqpoint{-16.715630in}{0.773588in}}%
\pgfpathlineto{\pgfqpoint{-16.644038in}{0.773588in}}%
\pgfpathlineto{\pgfqpoint{-16.575864in}{0.773588in}}%
\pgfpathlineto{\pgfqpoint{-16.507797in}{0.773588in}}%
\pgfpathlineto{\pgfqpoint{-16.437097in}{0.773588in}}%
\pgfpathlineto{\pgfqpoint{-16.369724in}{0.773588in}}%
\pgfpathlineto{\pgfqpoint{-16.302639in}{0.773588in}}%
\pgfpathlineto{\pgfqpoint{-16.232068in}{0.773588in}}%
\pgfpathlineto{\pgfqpoint{-16.163654in}{0.773588in}}%
\pgfpathlineto{\pgfqpoint{-16.094895in}{0.773588in}}%
\pgfpathlineto{\pgfqpoint{-16.024276in}{0.773588in}}%
\pgfpathlineto{\pgfqpoint{-15.956358in}{0.773588in}}%
\pgfpathlineto{\pgfqpoint{-15.886757in}{0.773588in}}%
\pgfpathlineto{\pgfqpoint{-15.814335in}{0.773588in}}%
\pgfpathlineto{\pgfqpoint{-15.744769in}{0.773588in}}%
\pgfpathlineto{\pgfqpoint{-15.673559in}{0.773588in}}%
\pgfpathlineto{\pgfqpoint{-15.602559in}{0.773588in}}%
\pgfpathlineto{\pgfqpoint{-15.532639in}{0.773588in}}%
\pgfpathlineto{\pgfqpoint{-15.463558in}{0.773588in}}%
\pgfpathlineto{\pgfqpoint{-15.392048in}{0.773588in}}%
\pgfpathlineto{\pgfqpoint{-15.323000in}{0.773588in}}%
\pgfpathlineto{\pgfqpoint{-15.253001in}{0.773588in}}%
\pgfpathlineto{\pgfqpoint{-15.179182in}{0.773588in}}%
\pgfpathlineto{\pgfqpoint{-15.106256in}{0.773588in}}%
\pgfpathlineto{\pgfqpoint{-15.032843in}{0.773588in}}%
\pgfpathlineto{\pgfqpoint{-14.959865in}{0.773588in}}%
\pgfpathlineto{\pgfqpoint{-14.889438in}{0.773588in}}%
\pgfpathlineto{\pgfqpoint{-14.820408in}{0.773588in}}%
\pgfpathlineto{\pgfqpoint{-14.748835in}{0.773588in}}%
\pgfpathlineto{\pgfqpoint{-14.679263in}{0.773588in}}%
\pgfpathlineto{\pgfqpoint{-14.609938in}{0.773588in}}%
\pgfpathlineto{\pgfqpoint{-14.538988in}{0.773588in}}%
\pgfpathlineto{\pgfqpoint{-14.471123in}{0.773588in}}%
\pgfpathlineto{\pgfqpoint{-14.401957in}{0.773588in}}%
\pgfpathlineto{\pgfqpoint{-14.331277in}{0.773588in}}%
\pgfpathlineto{\pgfqpoint{-14.263436in}{0.773588in}}%
\pgfpathlineto{\pgfqpoint{-14.195714in}{0.773588in}}%
\pgfpathlineto{\pgfqpoint{-14.125666in}{0.773588in}}%
\pgfpathlineto{\pgfqpoint{-14.057377in}{0.773588in}}%
\pgfpathlineto{\pgfqpoint{-13.988983in}{0.773588in}}%
\pgfpathlineto{\pgfqpoint{-13.917455in}{0.773588in}}%
\pgfpathlineto{\pgfqpoint{-13.848756in}{0.773588in}}%
\pgfpathlineto{\pgfqpoint{-13.779339in}{0.773588in}}%
\pgfpathlineto{\pgfqpoint{-13.708089in}{0.773588in}}%
\pgfpathlineto{\pgfqpoint{-13.639609in}{0.773588in}}%
\pgfpathlineto{\pgfqpoint{-13.568713in}{0.773588in}}%
\pgfpathlineto{\pgfqpoint{-13.496151in}{0.773588in}}%
\pgfpathlineto{\pgfqpoint{-13.426276in}{0.773588in}}%
\pgfpathlineto{\pgfqpoint{-13.356968in}{0.773588in}}%
\pgfpathlineto{\pgfqpoint{-13.285456in}{0.773588in}}%
\pgfpathlineto{\pgfqpoint{-13.216013in}{0.773588in}}%
\pgfpathlineto{\pgfqpoint{-13.145337in}{0.773588in}}%
\pgfpathlineto{\pgfqpoint{-13.073093in}{0.773588in}}%
\pgfpathlineto{\pgfqpoint{-13.002727in}{0.773588in}}%
\pgfpathlineto{\pgfqpoint{-12.932388in}{0.773588in}}%
\pgfpathlineto{\pgfqpoint{-12.859398in}{0.773588in}}%
\pgfpathlineto{\pgfqpoint{-12.788679in}{0.773588in}}%
\pgfpathlineto{\pgfqpoint{-12.718096in}{0.773588in}}%
\pgfpathlineto{\pgfqpoint{-12.645279in}{0.773588in}}%
\pgfpathlineto{\pgfqpoint{-12.574652in}{0.773588in}}%
\pgfpathlineto{\pgfqpoint{-12.503401in}{0.773588in}}%
\pgfpathlineto{\pgfqpoint{-12.428997in}{0.773588in}}%
\pgfpathlineto{\pgfqpoint{-12.357103in}{0.773588in}}%
\pgfpathlineto{\pgfqpoint{-12.285275in}{0.773588in}}%
\pgfpathlineto{\pgfqpoint{-12.211536in}{0.773588in}}%
\pgfpathlineto{\pgfqpoint{-12.139885in}{0.773588in}}%
\pgfpathlineto{\pgfqpoint{-12.069446in}{0.773588in}}%
\pgfpathlineto{\pgfqpoint{-11.998596in}{0.773588in}}%
\pgfpathlineto{\pgfqpoint{-11.928381in}{0.773588in}}%
\pgfpathlineto{\pgfqpoint{-11.858471in}{0.773588in}}%
\pgfpathlineto{\pgfqpoint{-11.786176in}{0.773588in}}%
\pgfpathlineto{\pgfqpoint{-11.716309in}{0.773588in}}%
\pgfpathlineto{\pgfqpoint{-11.645187in}{0.773588in}}%
\pgfpathlineto{\pgfqpoint{-11.571037in}{0.773588in}}%
\pgfpathlineto{\pgfqpoint{-11.500166in}{0.773588in}}%
\pgfpathlineto{\pgfqpoint{-11.430351in}{0.773588in}}%
\pgfpathlineto{\pgfqpoint{-11.358676in}{0.773588in}}%
\pgfpathlineto{\pgfqpoint{-11.288273in}{0.773588in}}%
\pgfpathlineto{\pgfqpoint{-11.217631in}{0.773588in}}%
\pgfpathlineto{\pgfqpoint{-11.144960in}{0.773588in}}%
\pgfpathlineto{\pgfqpoint{-11.075276in}{0.773588in}}%
\pgfpathlineto{\pgfqpoint{-11.005065in}{0.773588in}}%
\pgfpathlineto{\pgfqpoint{-10.933808in}{0.773588in}}%
\pgfpathlineto{\pgfqpoint{-10.863257in}{0.773588in}}%
\pgfpathlineto{\pgfqpoint{-10.793422in}{0.773588in}}%
\pgfpathlineto{\pgfqpoint{-10.721651in}{0.773588in}}%
\pgfpathlineto{\pgfqpoint{-10.652649in}{0.773588in}}%
\pgfpathlineto{\pgfqpoint{-10.582864in}{0.773588in}}%
\pgfpathlineto{\pgfqpoint{-10.511372in}{0.773588in}}%
\pgfpathlineto{\pgfqpoint{-10.441581in}{0.773588in}}%
\pgfpathlineto{\pgfqpoint{-10.371004in}{0.773588in}}%
\pgfpathlineto{\pgfqpoint{-10.297591in}{0.773588in}}%
\pgfpathlineto{\pgfqpoint{-10.227739in}{0.773588in}}%
\pgfpathlineto{\pgfqpoint{-10.157398in}{0.773588in}}%
\pgfpathlineto{\pgfqpoint{-10.084496in}{0.773588in}}%
\pgfpathlineto{\pgfqpoint{-10.012635in}{0.773588in}}%
\pgfpathlineto{\pgfqpoint{-9.941359in}{0.773588in}}%
\pgfpathlineto{\pgfqpoint{-9.868243in}{0.773588in}}%
\pgfpathlineto{\pgfqpoint{-9.796010in}{0.773588in}}%
\pgfpathlineto{\pgfqpoint{-9.723682in}{0.773588in}}%
\pgfpathlineto{\pgfqpoint{-9.648369in}{0.773588in}}%
\pgfpathlineto{\pgfqpoint{-9.575400in}{0.773588in}}%
\pgfpathlineto{\pgfqpoint{-9.504196in}{0.773588in}}%
\pgfpathlineto{\pgfqpoint{-9.429774in}{0.773588in}}%
\pgfpathlineto{\pgfqpoint{-9.356529in}{0.773588in}}%
\pgfpathlineto{\pgfqpoint{-9.283412in}{0.773588in}}%
\pgfpathlineto{\pgfqpoint{-9.208906in}{0.773588in}}%
\pgfpathlineto{\pgfqpoint{-9.138788in}{0.773588in}}%
\pgfpathlineto{\pgfqpoint{-9.069721in}{0.773588in}}%
\pgfpathlineto{\pgfqpoint{-8.998701in}{0.773588in}}%
\pgfpathlineto{\pgfqpoint{-8.928123in}{0.773588in}}%
\pgfpathlineto{\pgfqpoint{-8.858221in}{0.773588in}}%
\pgfpathlineto{\pgfqpoint{-8.787586in}{0.773588in}}%
\pgfpathlineto{\pgfqpoint{-8.718609in}{0.773588in}}%
\pgfpathlineto{\pgfqpoint{-8.649341in}{0.773588in}}%
\pgfpathlineto{\pgfqpoint{-8.578369in}{0.773588in}}%
\pgfpathlineto{\pgfqpoint{-8.508839in}{0.773588in}}%
\pgfpathlineto{\pgfqpoint{-8.438970in}{0.773588in}}%
\pgfpathlineto{\pgfqpoint{-8.367967in}{0.773588in}}%
\pgfpathlineto{\pgfqpoint{-8.298753in}{0.773588in}}%
\pgfpathlineto{\pgfqpoint{-8.228241in}{0.773588in}}%
\pgfpathlineto{\pgfqpoint{-8.156111in}{0.773588in}}%
\pgfpathlineto{\pgfqpoint{-8.084913in}{0.773588in}}%
\pgfpathlineto{\pgfqpoint{-8.014387in}{0.773588in}}%
\pgfpathlineto{\pgfqpoint{-7.942200in}{0.773588in}}%
\pgfpathlineto{\pgfqpoint{-7.870788in}{0.773588in}}%
\pgfpathlineto{\pgfqpoint{-7.799875in}{0.773588in}}%
\pgfpathlineto{\pgfqpoint{-7.727286in}{0.773588in}}%
\pgfpathlineto{\pgfqpoint{-7.657024in}{0.773588in}}%
\pgfpathlineto{\pgfqpoint{-7.585515in}{0.773588in}}%
\pgfpathlineto{\pgfqpoint{-7.512981in}{0.773588in}}%
\pgfpathlineto{\pgfqpoint{-7.440452in}{0.773588in}}%
\pgfpathlineto{\pgfqpoint{-7.368338in}{0.773588in}}%
\pgfpathlineto{\pgfqpoint{-7.294929in}{0.773588in}}%
\pgfpathlineto{\pgfqpoint{-7.223147in}{0.773588in}}%
\pgfpathlineto{\pgfqpoint{-7.151909in}{0.773588in}}%
\pgfpathlineto{\pgfqpoint{-7.075897in}{0.773588in}}%
\pgfpathlineto{\pgfqpoint{-7.003598in}{0.773588in}}%
\pgfpathlineto{\pgfqpoint{-6.933558in}{0.773588in}}%
\pgfpathlineto{\pgfqpoint{-6.860576in}{0.773588in}}%
\pgfpathlineto{\pgfqpoint{-6.787495in}{0.773588in}}%
\pgfpathlineto{\pgfqpoint{-6.715245in}{0.773588in}}%
\pgfpathlineto{\pgfqpoint{-6.640496in}{0.773588in}}%
\pgfpathlineto{\pgfqpoint{-6.568712in}{0.773588in}}%
\pgfpathlineto{\pgfqpoint{-6.496669in}{0.773588in}}%
\pgfpathlineto{\pgfqpoint{-6.421127in}{0.773588in}}%
\pgfpathlineto{\pgfqpoint{-6.349020in}{0.773588in}}%
\pgfpathlineto{\pgfqpoint{-6.278887in}{0.773588in}}%
\pgfpathlineto{\pgfqpoint{-6.205113in}{0.773588in}}%
\pgfpathlineto{\pgfqpoint{-6.135189in}{0.773588in}}%
\pgfpathlineto{\pgfqpoint{-6.064664in}{0.773588in}}%
\pgfpathlineto{\pgfqpoint{-5.992127in}{0.773588in}}%
\pgfpathlineto{\pgfqpoint{-5.921215in}{0.773588in}}%
\pgfpathlineto{\pgfqpoint{-5.849895in}{0.773588in}}%
\pgfpathlineto{\pgfqpoint{-5.776542in}{0.773588in}}%
\pgfpathlineto{\pgfqpoint{-5.706570in}{0.773588in}}%
\pgfpathlineto{\pgfqpoint{-5.635685in}{0.773588in}}%
\pgfpathlineto{\pgfqpoint{-5.563648in}{0.773588in}}%
\pgfpathlineto{\pgfqpoint{-5.493615in}{0.773588in}}%
\pgfpathlineto{\pgfqpoint{-5.421890in}{0.773588in}}%
\pgfpathlineto{\pgfqpoint{-5.348665in}{0.773588in}}%
\pgfpathlineto{\pgfqpoint{-5.277970in}{0.773588in}}%
\pgfpathlineto{\pgfqpoint{-5.207317in}{0.773588in}}%
\pgfpathlineto{\pgfqpoint{-5.135005in}{0.773588in}}%
\pgfpathlineto{\pgfqpoint{-5.064817in}{0.773588in}}%
\pgfpathlineto{\pgfqpoint{-4.995002in}{0.773588in}}%
\pgfpathlineto{\pgfqpoint{-4.922664in}{0.773588in}}%
\pgfpathlineto{\pgfqpoint{-4.852652in}{0.773588in}}%
\pgfpathlineto{\pgfqpoint{-4.782965in}{0.773588in}}%
\pgfpathlineto{\pgfqpoint{-4.711002in}{0.773588in}}%
\pgfpathlineto{\pgfqpoint{-4.639570in}{0.773588in}}%
\pgfpathlineto{\pgfqpoint{-4.568595in}{0.773588in}}%
\pgfpathlineto{\pgfqpoint{-4.495401in}{0.773588in}}%
\pgfpathlineto{\pgfqpoint{-4.423289in}{0.773588in}}%
\pgfpathlineto{\pgfqpoint{-4.350759in}{0.773588in}}%
\pgfpathlineto{\pgfqpoint{-4.274748in}{0.773588in}}%
\pgfpathlineto{\pgfqpoint{-4.201246in}{0.773588in}}%
\pgfpathlineto{\pgfqpoint{-4.128352in}{0.773588in}}%
\pgfpathlineto{\pgfqpoint{-4.052173in}{0.773588in}}%
\pgfpathlineto{\pgfqpoint{-3.977308in}{0.773588in}}%
\pgfpathlineto{\pgfqpoint{-3.903164in}{0.773588in}}%
\pgfpathlineto{\pgfqpoint{-3.828230in}{0.773588in}}%
\pgfpathlineto{\pgfqpoint{-3.756893in}{0.773588in}}%
\pgfpathlineto{\pgfqpoint{-3.684119in}{0.773588in}}%
\pgfpathlineto{\pgfqpoint{-3.608325in}{0.773588in}}%
\pgfpathlineto{\pgfqpoint{-3.534002in}{0.773588in}}%
\pgfpathlineto{\pgfqpoint{-3.461876in}{0.773588in}}%
\pgfpathlineto{\pgfqpoint{-3.388885in}{0.773588in}}%
\pgfpathlineto{\pgfqpoint{-3.317677in}{0.773588in}}%
\pgfpathlineto{\pgfqpoint{-3.245632in}{0.773588in}}%
\pgfpathlineto{\pgfqpoint{-3.171735in}{0.773588in}}%
\pgfpathlineto{\pgfqpoint{-3.100293in}{0.773588in}}%
\pgfpathlineto{\pgfqpoint{-3.029702in}{0.773588in}}%
\pgfpathlineto{\pgfqpoint{-2.956196in}{0.773588in}}%
\pgfpathlineto{\pgfqpoint{-2.884033in}{0.773588in}}%
\pgfpathlineto{\pgfqpoint{-2.813631in}{0.773588in}}%
\pgfpathlineto{\pgfqpoint{-2.740325in}{0.773588in}}%
\pgfpathlineto{\pgfqpoint{-2.667020in}{0.773588in}}%
\pgfpathlineto{\pgfqpoint{-2.595405in}{0.773588in}}%
\pgfpathlineto{\pgfqpoint{-2.522349in}{0.773588in}}%
\pgfpathlineto{\pgfqpoint{-2.451489in}{0.773588in}}%
\pgfpathlineto{\pgfqpoint{-2.379035in}{0.773588in}}%
\pgfpathlineto{\pgfqpoint{-2.303263in}{0.773588in}}%
\pgfpathlineto{\pgfqpoint{-2.231477in}{0.773588in}}%
\pgfpathlineto{\pgfqpoint{-2.158577in}{0.773588in}}%
\pgfpathlineto{\pgfqpoint{-2.084338in}{0.773588in}}%
\pgfpathlineto{\pgfqpoint{-2.011817in}{0.773588in}}%
\pgfpathlineto{\pgfqpoint{-1.940536in}{0.773588in}}%
\pgfpathlineto{\pgfqpoint{-1.867809in}{0.773588in}}%
\pgfpathlineto{\pgfqpoint{-1.797123in}{0.773588in}}%
\pgfpathlineto{\pgfqpoint{-1.725105in}{0.773588in}}%
\pgfpathlineto{\pgfqpoint{-1.650581in}{0.773588in}}%
\pgfpathlineto{\pgfqpoint{-1.577590in}{0.773588in}}%
\pgfpathlineto{\pgfqpoint{-1.503984in}{0.773588in}}%
\pgfpathlineto{\pgfqpoint{-1.429441in}{0.773588in}}%
\pgfpathlineto{\pgfqpoint{-1.356794in}{0.773588in}}%
\pgfpathlineto{\pgfqpoint{-1.283200in}{0.773588in}}%
\pgfpathlineto{\pgfqpoint{-1.207908in}{0.773588in}}%
\pgfpathlineto{\pgfqpoint{-1.135138in}{0.773588in}}%
\pgfpathlineto{\pgfqpoint{-1.063245in}{0.773588in}}%
\pgfpathlineto{\pgfqpoint{-0.989522in}{0.773588in}}%
\pgfpathlineto{\pgfqpoint{-0.918138in}{0.773588in}}%
\pgfpathlineto{\pgfqpoint{-0.844408in}{0.773588in}}%
\pgfpathlineto{\pgfqpoint{-0.769987in}{0.773588in}}%
\pgfpathlineto{\pgfqpoint{-0.698122in}{0.773588in}}%
\pgfpathlineto{\pgfqpoint{-0.623932in}{0.773588in}}%
\pgfpathlineto{\pgfqpoint{-0.549722in}{0.773588in}}%
\pgfpathlineto{\pgfqpoint{-0.477784in}{0.773588in}}%
\pgfpathlineto{\pgfqpoint{-0.404497in}{0.773588in}}%
\pgfpathlineto{\pgfqpoint{-0.329323in}{0.773588in}}%
\pgfpathlineto{\pgfqpoint{-0.257459in}{0.773588in}}%
\pgfpathlineto{\pgfqpoint{-0.185623in}{0.773588in}}%
\pgfpathlineto{\pgfqpoint{-0.112821in}{0.773588in}}%
\pgfpathlineto{\pgfqpoint{-0.042097in}{0.773588in}}%
\pgfpathlineto{\pgfqpoint{0.030045in}{0.773588in}}%
\pgfpathlineto{\pgfqpoint{0.103944in}{0.773588in}}%
\pgfpathlineto{\pgfqpoint{0.176284in}{0.773588in}}%
\pgfpathlineto{\pgfqpoint{0.248641in}{0.773588in}}%
\pgfpathlineto{\pgfqpoint{0.321303in}{0.773588in}}%
\pgfpathlineto{\pgfqpoint{0.391498in}{0.773588in}}%
\pgfpathlineto{\pgfqpoint{0.462885in}{0.773588in}}%
\pgfpathlineto{\pgfqpoint{0.537447in}{0.773588in}}%
\pgfpathlineto{\pgfqpoint{0.609267in}{0.773588in}}%
\pgfpathlineto{\pgfqpoint{0.679536in}{0.773588in}}%
\pgfpathlineto{\pgfqpoint{0.753203in}{0.773588in}}%
\pgfpathlineto{\pgfqpoint{0.824780in}{0.773588in}}%
\pgfpathlineto{\pgfqpoint{0.895203in}{0.773588in}}%
\pgfpathlineto{\pgfqpoint{0.968165in}{0.773588in}}%
\pgfpathlineto{\pgfqpoint{1.038437in}{0.773588in}}%
\pgfpathlineto{\pgfqpoint{1.111472in}{0.773588in}}%
\pgfpathlineto{\pgfqpoint{1.188027in}{0.773588in}}%
\pgfpathlineto{\pgfqpoint{1.261309in}{0.773588in}}%
\pgfpathlineto{\pgfqpoint{1.335256in}{0.773588in}}%
\pgfpathlineto{\pgfqpoint{1.410931in}{0.773588in}}%
\pgfpathlineto{\pgfqpoint{1.485557in}{0.773588in}}%
\pgfpathlineto{\pgfqpoint{1.558242in}{0.773588in}}%
\pgfpathlineto{\pgfqpoint{1.633550in}{0.773588in}}%
\pgfpathlineto{\pgfqpoint{1.707271in}{0.773588in}}%
\pgfpathlineto{\pgfqpoint{1.781074in}{0.773588in}}%
\pgfpathlineto{\pgfqpoint{1.857049in}{0.773588in}}%
\pgfpathlineto{\pgfqpoint{1.931091in}{0.773588in}}%
\pgfpathlineto{\pgfqpoint{2.004776in}{0.773588in}}%
\pgfpathlineto{\pgfqpoint{2.079647in}{0.773588in}}%
\pgfpathlineto{\pgfqpoint{2.151611in}{0.773588in}}%
\pgfpathlineto{\pgfqpoint{2.224651in}{0.773588in}}%
\pgfpathlineto{\pgfqpoint{2.300156in}{0.773588in}}%
\pgfpathlineto{\pgfqpoint{2.371299in}{0.773588in}}%
\pgfpathlineto{\pgfqpoint{2.442412in}{0.773588in}}%
\pgfpathlineto{\pgfqpoint{2.515763in}{0.773588in}}%
\pgfpathlineto{\pgfqpoint{2.586109in}{0.773588in}}%
\pgfpathlineto{\pgfqpoint{2.658524in}{0.773588in}}%
\pgfpathlineto{\pgfqpoint{2.732102in}{0.773588in}}%
\pgfpathlineto{\pgfqpoint{2.802412in}{0.773588in}}%
\pgfpathlineto{\pgfqpoint{2.873367in}{0.773588in}}%
\pgfpathlineto{\pgfqpoint{2.946925in}{0.773588in}}%
\pgfpathlineto{\pgfqpoint{3.019212in}{0.773588in}}%
\pgfpathlineto{\pgfqpoint{3.091740in}{0.773588in}}%
\pgfpathlineto{\pgfqpoint{3.166494in}{0.773588in}}%
\pgfpathlineto{\pgfqpoint{3.237461in}{0.773588in}}%
\pgfpathlineto{\pgfqpoint{3.309976in}{0.773588in}}%
\pgfpathlineto{\pgfqpoint{3.384706in}{0.773588in}}%
\pgfpathlineto{\pgfqpoint{3.455601in}{0.773588in}}%
\pgfpathlineto{\pgfqpoint{3.527918in}{0.773588in}}%
\pgfpathlineto{\pgfqpoint{3.603230in}{0.773588in}}%
\pgfpathlineto{\pgfqpoint{3.674965in}{0.773588in}}%
\pgfpathlineto{\pgfqpoint{3.745938in}{0.773588in}}%
\pgfpathlineto{\pgfqpoint{3.820753in}{0.773588in}}%
\pgfpathlineto{\pgfqpoint{3.899537in}{0.773588in}}%
\pgfpathlineto{\pgfqpoint{4.022313in}{0.773588in}}%
\pgfpathlineto{\pgfqpoint{4.111286in}{0.773588in}}%
\pgfpathlineto{\pgfqpoint{4.189282in}{0.773588in}}%
\pgfpathlineto{\pgfqpoint{4.253332in}{1.249553in}}%
\pgfpathlineto{\pgfqpoint{4.318577in}{2.216022in}}%
\pgfpathlineto{\pgfqpoint{4.390596in}{2.269015in}}%
\pgfpathlineto{\pgfqpoint{4.460559in}{2.321125in}}%
\pgfpathlineto{\pgfqpoint{4.532858in}{2.286938in}}%
\pgfpathlineto{\pgfqpoint{4.602459in}{2.303111in}}%
\pgfpathlineto{\pgfqpoint{4.671388in}{2.332436in}}%
\pgfpathlineto{\pgfqpoint{4.741243in}{2.395888in}}%
\pgfpathlineto{\pgfqpoint{4.808397in}{2.349474in}}%
\pgfpathlineto{\pgfqpoint{4.875969in}{2.435547in}}%
\pgfpathlineto{\pgfqpoint{4.944850in}{2.404583in}}%
\pgfpathlineto{\pgfqpoint{5.010860in}{2.486664in}}%
\pgfpathlineto{\pgfqpoint{5.076713in}{2.376225in}}%
\pgfpathlineto{\pgfqpoint{5.144715in}{2.436993in}}%
\pgfpathlineto{\pgfqpoint{5.209656in}{2.464952in}}%
\pgfpathlineto{\pgfqpoint{5.275100in}{2.434812in}}%
\pgfpathlineto{\pgfqpoint{5.341477in}{2.468865in}}%
\pgfpathlineto{\pgfqpoint{5.405234in}{2.464622in}}%
\pgfpathlineto{\pgfqpoint{5.469981in}{2.426405in}}%
\pgfpathlineto{\pgfqpoint{5.535779in}{2.514045in}}%
\pgfpathlineto{\pgfqpoint{5.599590in}{2.457661in}}%
\pgfpathlineto{\pgfqpoint{5.599590in}{4.172457in}}%
\pgfpathlineto{\pgfqpoint{5.599590in}{4.172457in}}%
\pgfpathlineto{\pgfqpoint{5.535779in}{4.257509in}}%
\pgfpathlineto{\pgfqpoint{5.469981in}{4.115512in}}%
\pgfpathlineto{\pgfqpoint{5.405234in}{4.207643in}}%
\pgfpathlineto{\pgfqpoint{5.341477in}{4.215547in}}%
\pgfpathlineto{\pgfqpoint{5.275100in}{4.095111in}}%
\pgfpathlineto{\pgfqpoint{5.209656in}{4.151117in}}%
\pgfpathlineto{\pgfqpoint{5.144715in}{4.115543in}}%
\pgfpathlineto{\pgfqpoint{5.076713in}{4.085256in}}%
\pgfpathlineto{\pgfqpoint{5.010860in}{4.110835in}}%
\pgfpathlineto{\pgfqpoint{4.944850in}{3.951551in}}%
\pgfpathlineto{\pgfqpoint{4.875969in}{4.007261in}}%
\pgfpathlineto{\pgfqpoint{4.808397in}{4.001718in}}%
\pgfpathlineto{\pgfqpoint{4.741243in}{3.999837in}}%
\pgfpathlineto{\pgfqpoint{4.671388in}{3.948574in}}%
\pgfpathlineto{\pgfqpoint{4.602459in}{3.912811in}}%
\pgfpathlineto{\pgfqpoint{4.532858in}{3.810697in}}%
\pgfpathlineto{\pgfqpoint{4.460559in}{3.977733in}}%
\pgfpathlineto{\pgfqpoint{4.390596in}{3.811738in}}%
\pgfpathlineto{\pgfqpoint{4.318577in}{3.700065in}}%
\pgfpathlineto{\pgfqpoint{4.253332in}{1.591997in}}%
\pgfpathlineto{\pgfqpoint{4.189282in}{0.773588in}}%
\pgfpathlineto{\pgfqpoint{4.111286in}{0.773588in}}%
\pgfpathlineto{\pgfqpoint{4.022313in}{0.773588in}}%
\pgfpathlineto{\pgfqpoint{3.899537in}{0.773588in}}%
\pgfpathlineto{\pgfqpoint{3.820753in}{0.773588in}}%
\pgfpathlineto{\pgfqpoint{3.745938in}{0.773588in}}%
\pgfpathlineto{\pgfqpoint{3.674965in}{0.773588in}}%
\pgfpathlineto{\pgfqpoint{3.603230in}{0.773588in}}%
\pgfpathlineto{\pgfqpoint{3.527918in}{0.773588in}}%
\pgfpathlineto{\pgfqpoint{3.455601in}{0.773588in}}%
\pgfpathlineto{\pgfqpoint{3.384706in}{0.773588in}}%
\pgfpathlineto{\pgfqpoint{3.309976in}{0.773588in}}%
\pgfpathlineto{\pgfqpoint{3.237461in}{0.773588in}}%
\pgfpathlineto{\pgfqpoint{3.166494in}{0.773588in}}%
\pgfpathlineto{\pgfqpoint{3.091740in}{0.773588in}}%
\pgfpathlineto{\pgfqpoint{3.019212in}{0.773588in}}%
\pgfpathlineto{\pgfqpoint{2.946925in}{0.773588in}}%
\pgfpathlineto{\pgfqpoint{2.873367in}{0.773588in}}%
\pgfpathlineto{\pgfqpoint{2.802412in}{0.773588in}}%
\pgfpathlineto{\pgfqpoint{2.732102in}{0.773588in}}%
\pgfpathlineto{\pgfqpoint{2.658524in}{0.773588in}}%
\pgfpathlineto{\pgfqpoint{2.586109in}{0.773588in}}%
\pgfpathlineto{\pgfqpoint{2.515763in}{0.773588in}}%
\pgfpathlineto{\pgfqpoint{2.442412in}{0.773588in}}%
\pgfpathlineto{\pgfqpoint{2.371299in}{0.773588in}}%
\pgfpathlineto{\pgfqpoint{2.300156in}{0.773588in}}%
\pgfpathlineto{\pgfqpoint{2.224651in}{0.773588in}}%
\pgfpathlineto{\pgfqpoint{2.151611in}{0.773588in}}%
\pgfpathlineto{\pgfqpoint{2.079647in}{0.773588in}}%
\pgfpathlineto{\pgfqpoint{2.004776in}{0.773588in}}%
\pgfpathlineto{\pgfqpoint{1.931091in}{0.773588in}}%
\pgfpathlineto{\pgfqpoint{1.857049in}{0.773588in}}%
\pgfpathlineto{\pgfqpoint{1.781074in}{0.773588in}}%
\pgfpathlineto{\pgfqpoint{1.707271in}{0.773588in}}%
\pgfpathlineto{\pgfqpoint{1.633550in}{0.773588in}}%
\pgfpathlineto{\pgfqpoint{1.558242in}{0.773588in}}%
\pgfpathlineto{\pgfqpoint{1.485557in}{0.773588in}}%
\pgfpathlineto{\pgfqpoint{1.410931in}{0.773588in}}%
\pgfpathlineto{\pgfqpoint{1.335256in}{0.773588in}}%
\pgfpathlineto{\pgfqpoint{1.261309in}{0.773588in}}%
\pgfpathlineto{\pgfqpoint{1.188027in}{0.773588in}}%
\pgfpathlineto{\pgfqpoint{1.111472in}{0.773588in}}%
\pgfpathlineto{\pgfqpoint{1.038437in}{0.773588in}}%
\pgfpathlineto{\pgfqpoint{0.968165in}{0.773588in}}%
\pgfpathlineto{\pgfqpoint{0.895203in}{0.773588in}}%
\pgfpathlineto{\pgfqpoint{0.824780in}{0.773588in}}%
\pgfpathlineto{\pgfqpoint{0.753203in}{0.773588in}}%
\pgfpathlineto{\pgfqpoint{0.679536in}{0.773588in}}%
\pgfpathlineto{\pgfqpoint{0.609267in}{0.773588in}}%
\pgfpathlineto{\pgfqpoint{0.537447in}{0.773588in}}%
\pgfpathlineto{\pgfqpoint{0.462885in}{0.773588in}}%
\pgfpathlineto{\pgfqpoint{0.391498in}{0.773588in}}%
\pgfpathlineto{\pgfqpoint{0.321303in}{0.773588in}}%
\pgfpathlineto{\pgfqpoint{0.248641in}{0.773588in}}%
\pgfpathlineto{\pgfqpoint{0.176284in}{0.773588in}}%
\pgfpathlineto{\pgfqpoint{0.103944in}{0.773588in}}%
\pgfpathlineto{\pgfqpoint{0.030045in}{0.773588in}}%
\pgfpathlineto{\pgfqpoint{-0.042097in}{0.773588in}}%
\pgfpathlineto{\pgfqpoint{-0.112821in}{0.773588in}}%
\pgfpathlineto{\pgfqpoint{-0.185623in}{0.773588in}}%
\pgfpathlineto{\pgfqpoint{-0.257459in}{0.773588in}}%
\pgfpathlineto{\pgfqpoint{-0.329323in}{0.773588in}}%
\pgfpathlineto{\pgfqpoint{-0.404497in}{0.773588in}}%
\pgfpathlineto{\pgfqpoint{-0.477784in}{0.773588in}}%
\pgfpathlineto{\pgfqpoint{-0.549722in}{0.773588in}}%
\pgfpathlineto{\pgfqpoint{-0.623932in}{0.773588in}}%
\pgfpathlineto{\pgfqpoint{-0.698122in}{0.773588in}}%
\pgfpathlineto{\pgfqpoint{-0.769987in}{0.773588in}}%
\pgfpathlineto{\pgfqpoint{-0.844408in}{0.773588in}}%
\pgfpathlineto{\pgfqpoint{-0.918138in}{0.773588in}}%
\pgfpathlineto{\pgfqpoint{-0.989522in}{0.773588in}}%
\pgfpathlineto{\pgfqpoint{-1.063245in}{0.773588in}}%
\pgfpathlineto{\pgfqpoint{-1.135138in}{0.773588in}}%
\pgfpathlineto{\pgfqpoint{-1.207908in}{0.773588in}}%
\pgfpathlineto{\pgfqpoint{-1.283200in}{0.773588in}}%
\pgfpathlineto{\pgfqpoint{-1.356794in}{0.773588in}}%
\pgfpathlineto{\pgfqpoint{-1.429441in}{0.773588in}}%
\pgfpathlineto{\pgfqpoint{-1.503984in}{0.773588in}}%
\pgfpathlineto{\pgfqpoint{-1.577590in}{0.773588in}}%
\pgfpathlineto{\pgfqpoint{-1.650581in}{0.773588in}}%
\pgfpathlineto{\pgfqpoint{-1.725105in}{0.773588in}}%
\pgfpathlineto{\pgfqpoint{-1.797123in}{0.773588in}}%
\pgfpathlineto{\pgfqpoint{-1.867809in}{0.773588in}}%
\pgfpathlineto{\pgfqpoint{-1.940536in}{0.773588in}}%
\pgfpathlineto{\pgfqpoint{-2.011817in}{0.773588in}}%
\pgfpathlineto{\pgfqpoint{-2.084338in}{0.773588in}}%
\pgfpathlineto{\pgfqpoint{-2.158577in}{0.773588in}}%
\pgfpathlineto{\pgfqpoint{-2.231477in}{0.773588in}}%
\pgfpathlineto{\pgfqpoint{-2.303263in}{0.773588in}}%
\pgfpathlineto{\pgfqpoint{-2.379035in}{0.773588in}}%
\pgfpathlineto{\pgfqpoint{-2.451489in}{0.773588in}}%
\pgfpathlineto{\pgfqpoint{-2.522349in}{0.773588in}}%
\pgfpathlineto{\pgfqpoint{-2.595405in}{0.773588in}}%
\pgfpathlineto{\pgfqpoint{-2.667020in}{0.773588in}}%
\pgfpathlineto{\pgfqpoint{-2.740325in}{0.773588in}}%
\pgfpathlineto{\pgfqpoint{-2.813631in}{0.773588in}}%
\pgfpathlineto{\pgfqpoint{-2.884033in}{0.773588in}}%
\pgfpathlineto{\pgfqpoint{-2.956196in}{0.773588in}}%
\pgfpathlineto{\pgfqpoint{-3.029702in}{0.773588in}}%
\pgfpathlineto{\pgfqpoint{-3.100293in}{0.773588in}}%
\pgfpathlineto{\pgfqpoint{-3.171735in}{0.773588in}}%
\pgfpathlineto{\pgfqpoint{-3.245632in}{0.773588in}}%
\pgfpathlineto{\pgfqpoint{-3.317677in}{0.773588in}}%
\pgfpathlineto{\pgfqpoint{-3.388885in}{0.773588in}}%
\pgfpathlineto{\pgfqpoint{-3.461876in}{0.773588in}}%
\pgfpathlineto{\pgfqpoint{-3.534002in}{0.773588in}}%
\pgfpathlineto{\pgfqpoint{-3.608325in}{0.773588in}}%
\pgfpathlineto{\pgfqpoint{-3.684119in}{0.773588in}}%
\pgfpathlineto{\pgfqpoint{-3.756893in}{0.773588in}}%
\pgfpathlineto{\pgfqpoint{-3.828230in}{0.773588in}}%
\pgfpathlineto{\pgfqpoint{-3.903164in}{0.773588in}}%
\pgfpathlineto{\pgfqpoint{-3.977308in}{0.773588in}}%
\pgfpathlineto{\pgfqpoint{-4.052173in}{0.773588in}}%
\pgfpathlineto{\pgfqpoint{-4.128352in}{0.773588in}}%
\pgfpathlineto{\pgfqpoint{-4.201246in}{0.773588in}}%
\pgfpathlineto{\pgfqpoint{-4.274748in}{0.773588in}}%
\pgfpathlineto{\pgfqpoint{-4.350759in}{0.773588in}}%
\pgfpathlineto{\pgfqpoint{-4.423289in}{0.773588in}}%
\pgfpathlineto{\pgfqpoint{-4.495401in}{0.773588in}}%
\pgfpathlineto{\pgfqpoint{-4.568595in}{0.773588in}}%
\pgfpathlineto{\pgfqpoint{-4.639570in}{0.773588in}}%
\pgfpathlineto{\pgfqpoint{-4.711002in}{0.773588in}}%
\pgfpathlineto{\pgfqpoint{-4.782965in}{0.773588in}}%
\pgfpathlineto{\pgfqpoint{-4.852652in}{0.773588in}}%
\pgfpathlineto{\pgfqpoint{-4.922664in}{0.773588in}}%
\pgfpathlineto{\pgfqpoint{-4.995002in}{0.773588in}}%
\pgfpathlineto{\pgfqpoint{-5.064817in}{0.773588in}}%
\pgfpathlineto{\pgfqpoint{-5.135005in}{0.773588in}}%
\pgfpathlineto{\pgfqpoint{-5.207317in}{0.773588in}}%
\pgfpathlineto{\pgfqpoint{-5.277970in}{0.773588in}}%
\pgfpathlineto{\pgfqpoint{-5.348665in}{0.773588in}}%
\pgfpathlineto{\pgfqpoint{-5.421890in}{0.773588in}}%
\pgfpathlineto{\pgfqpoint{-5.493615in}{0.773588in}}%
\pgfpathlineto{\pgfqpoint{-5.563648in}{0.773588in}}%
\pgfpathlineto{\pgfqpoint{-5.635685in}{0.773588in}}%
\pgfpathlineto{\pgfqpoint{-5.706570in}{0.773588in}}%
\pgfpathlineto{\pgfqpoint{-5.776542in}{0.773588in}}%
\pgfpathlineto{\pgfqpoint{-5.849895in}{0.773588in}}%
\pgfpathlineto{\pgfqpoint{-5.921215in}{0.773588in}}%
\pgfpathlineto{\pgfqpoint{-5.992127in}{0.773588in}}%
\pgfpathlineto{\pgfqpoint{-6.064664in}{0.773588in}}%
\pgfpathlineto{\pgfqpoint{-6.135189in}{0.773588in}}%
\pgfpathlineto{\pgfqpoint{-6.205113in}{0.773588in}}%
\pgfpathlineto{\pgfqpoint{-6.278887in}{0.773588in}}%
\pgfpathlineto{\pgfqpoint{-6.349020in}{0.773588in}}%
\pgfpathlineto{\pgfqpoint{-6.421127in}{0.773588in}}%
\pgfpathlineto{\pgfqpoint{-6.496669in}{0.773588in}}%
\pgfpathlineto{\pgfqpoint{-6.568712in}{0.773588in}}%
\pgfpathlineto{\pgfqpoint{-6.640496in}{0.773588in}}%
\pgfpathlineto{\pgfqpoint{-6.715245in}{0.773588in}}%
\pgfpathlineto{\pgfqpoint{-6.787495in}{0.773588in}}%
\pgfpathlineto{\pgfqpoint{-6.860576in}{0.773588in}}%
\pgfpathlineto{\pgfqpoint{-6.933558in}{0.773588in}}%
\pgfpathlineto{\pgfqpoint{-7.003598in}{0.773588in}}%
\pgfpathlineto{\pgfqpoint{-7.075897in}{0.773588in}}%
\pgfpathlineto{\pgfqpoint{-7.151909in}{0.773588in}}%
\pgfpathlineto{\pgfqpoint{-7.223147in}{0.773588in}}%
\pgfpathlineto{\pgfqpoint{-7.294929in}{0.773588in}}%
\pgfpathlineto{\pgfqpoint{-7.368338in}{0.773588in}}%
\pgfpathlineto{\pgfqpoint{-7.440452in}{0.773588in}}%
\pgfpathlineto{\pgfqpoint{-7.512981in}{0.773588in}}%
\pgfpathlineto{\pgfqpoint{-7.585515in}{0.773588in}}%
\pgfpathlineto{\pgfqpoint{-7.657024in}{0.773588in}}%
\pgfpathlineto{\pgfqpoint{-7.727286in}{0.773588in}}%
\pgfpathlineto{\pgfqpoint{-7.799875in}{0.773588in}}%
\pgfpathlineto{\pgfqpoint{-7.870788in}{0.773588in}}%
\pgfpathlineto{\pgfqpoint{-7.942200in}{0.773588in}}%
\pgfpathlineto{\pgfqpoint{-8.014387in}{0.773588in}}%
\pgfpathlineto{\pgfqpoint{-8.084913in}{0.773588in}}%
\pgfpathlineto{\pgfqpoint{-8.156111in}{0.773588in}}%
\pgfpathlineto{\pgfqpoint{-8.228241in}{0.773588in}}%
\pgfpathlineto{\pgfqpoint{-8.298753in}{0.773588in}}%
\pgfpathlineto{\pgfqpoint{-8.367967in}{0.773588in}}%
\pgfpathlineto{\pgfqpoint{-8.438970in}{0.773588in}}%
\pgfpathlineto{\pgfqpoint{-8.508839in}{0.773588in}}%
\pgfpathlineto{\pgfqpoint{-8.578369in}{0.773588in}}%
\pgfpathlineto{\pgfqpoint{-8.649341in}{0.773588in}}%
\pgfpathlineto{\pgfqpoint{-8.718609in}{0.773588in}}%
\pgfpathlineto{\pgfqpoint{-8.787586in}{0.773588in}}%
\pgfpathlineto{\pgfqpoint{-8.858221in}{0.773588in}}%
\pgfpathlineto{\pgfqpoint{-8.928123in}{0.773588in}}%
\pgfpathlineto{\pgfqpoint{-8.998701in}{0.773588in}}%
\pgfpathlineto{\pgfqpoint{-9.069721in}{0.773588in}}%
\pgfpathlineto{\pgfqpoint{-9.138788in}{0.773588in}}%
\pgfpathlineto{\pgfqpoint{-9.208906in}{0.773588in}}%
\pgfpathlineto{\pgfqpoint{-9.283412in}{0.773588in}}%
\pgfpathlineto{\pgfqpoint{-9.356529in}{0.773588in}}%
\pgfpathlineto{\pgfqpoint{-9.429774in}{0.773588in}}%
\pgfpathlineto{\pgfqpoint{-9.504196in}{0.773588in}}%
\pgfpathlineto{\pgfqpoint{-9.575400in}{0.773588in}}%
\pgfpathlineto{\pgfqpoint{-9.648369in}{0.773588in}}%
\pgfpathlineto{\pgfqpoint{-9.723682in}{0.773588in}}%
\pgfpathlineto{\pgfqpoint{-9.796010in}{0.773588in}}%
\pgfpathlineto{\pgfqpoint{-9.868243in}{0.773588in}}%
\pgfpathlineto{\pgfqpoint{-9.941359in}{0.773588in}}%
\pgfpathlineto{\pgfqpoint{-10.012635in}{0.773588in}}%
\pgfpathlineto{\pgfqpoint{-10.084496in}{0.773588in}}%
\pgfpathlineto{\pgfqpoint{-10.157398in}{0.773588in}}%
\pgfpathlineto{\pgfqpoint{-10.227739in}{0.773588in}}%
\pgfpathlineto{\pgfqpoint{-10.297591in}{0.773588in}}%
\pgfpathlineto{\pgfqpoint{-10.371004in}{0.773588in}}%
\pgfpathlineto{\pgfqpoint{-10.441581in}{0.773588in}}%
\pgfpathlineto{\pgfqpoint{-10.511372in}{0.773588in}}%
\pgfpathlineto{\pgfqpoint{-10.582864in}{0.773588in}}%
\pgfpathlineto{\pgfqpoint{-10.652649in}{0.773588in}}%
\pgfpathlineto{\pgfqpoint{-10.721651in}{0.773588in}}%
\pgfpathlineto{\pgfqpoint{-10.793422in}{0.773588in}}%
\pgfpathlineto{\pgfqpoint{-10.863257in}{0.773588in}}%
\pgfpathlineto{\pgfqpoint{-10.933808in}{0.773588in}}%
\pgfpathlineto{\pgfqpoint{-11.005065in}{0.773588in}}%
\pgfpathlineto{\pgfqpoint{-11.075276in}{0.773588in}}%
\pgfpathlineto{\pgfqpoint{-11.144960in}{0.773588in}}%
\pgfpathlineto{\pgfqpoint{-11.217631in}{0.773588in}}%
\pgfpathlineto{\pgfqpoint{-11.288273in}{0.773588in}}%
\pgfpathlineto{\pgfqpoint{-11.358676in}{0.773588in}}%
\pgfpathlineto{\pgfqpoint{-11.430351in}{0.773588in}}%
\pgfpathlineto{\pgfqpoint{-11.500166in}{0.773588in}}%
\pgfpathlineto{\pgfqpoint{-11.571037in}{0.773588in}}%
\pgfpathlineto{\pgfqpoint{-11.645187in}{0.773588in}}%
\pgfpathlineto{\pgfqpoint{-11.716309in}{0.773588in}}%
\pgfpathlineto{\pgfqpoint{-11.786176in}{0.773588in}}%
\pgfpathlineto{\pgfqpoint{-11.858471in}{0.773588in}}%
\pgfpathlineto{\pgfqpoint{-11.928381in}{0.773588in}}%
\pgfpathlineto{\pgfqpoint{-11.998596in}{0.773588in}}%
\pgfpathlineto{\pgfqpoint{-12.069446in}{0.773588in}}%
\pgfpathlineto{\pgfqpoint{-12.139885in}{0.773588in}}%
\pgfpathlineto{\pgfqpoint{-12.211536in}{0.773588in}}%
\pgfpathlineto{\pgfqpoint{-12.285275in}{0.773588in}}%
\pgfpathlineto{\pgfqpoint{-12.357103in}{0.773588in}}%
\pgfpathlineto{\pgfqpoint{-12.428997in}{0.773588in}}%
\pgfpathlineto{\pgfqpoint{-12.503401in}{0.773588in}}%
\pgfpathlineto{\pgfqpoint{-12.574652in}{0.773588in}}%
\pgfpathlineto{\pgfqpoint{-12.645279in}{0.773588in}}%
\pgfpathlineto{\pgfqpoint{-12.718096in}{0.773588in}}%
\pgfpathlineto{\pgfqpoint{-12.788679in}{0.773588in}}%
\pgfpathlineto{\pgfqpoint{-12.859398in}{0.773588in}}%
\pgfpathlineto{\pgfqpoint{-12.932388in}{0.773588in}}%
\pgfpathlineto{\pgfqpoint{-13.002727in}{0.773588in}}%
\pgfpathlineto{\pgfqpoint{-13.073093in}{0.773588in}}%
\pgfpathlineto{\pgfqpoint{-13.145337in}{0.773588in}}%
\pgfpathlineto{\pgfqpoint{-13.216013in}{0.773588in}}%
\pgfpathlineto{\pgfqpoint{-13.285456in}{0.773588in}}%
\pgfpathlineto{\pgfqpoint{-13.356968in}{0.773588in}}%
\pgfpathlineto{\pgfqpoint{-13.426276in}{0.773588in}}%
\pgfpathlineto{\pgfqpoint{-13.496151in}{0.773588in}}%
\pgfpathlineto{\pgfqpoint{-13.568713in}{0.773588in}}%
\pgfpathlineto{\pgfqpoint{-13.639609in}{0.773588in}}%
\pgfpathlineto{\pgfqpoint{-13.708089in}{0.773588in}}%
\pgfpathlineto{\pgfqpoint{-13.779339in}{0.773588in}}%
\pgfpathlineto{\pgfqpoint{-13.848756in}{0.773588in}}%
\pgfpathlineto{\pgfqpoint{-13.917455in}{0.773588in}}%
\pgfpathlineto{\pgfqpoint{-13.988983in}{0.773588in}}%
\pgfpathlineto{\pgfqpoint{-14.057377in}{0.773588in}}%
\pgfpathlineto{\pgfqpoint{-14.125666in}{0.773588in}}%
\pgfpathlineto{\pgfqpoint{-14.195714in}{0.773588in}}%
\pgfpathlineto{\pgfqpoint{-14.263436in}{0.773588in}}%
\pgfpathlineto{\pgfqpoint{-14.331277in}{0.773588in}}%
\pgfpathlineto{\pgfqpoint{-14.401957in}{0.773588in}}%
\pgfpathlineto{\pgfqpoint{-14.471123in}{0.773588in}}%
\pgfpathlineto{\pgfqpoint{-14.538988in}{0.773588in}}%
\pgfpathlineto{\pgfqpoint{-14.609938in}{0.773588in}}%
\pgfpathlineto{\pgfqpoint{-14.679263in}{0.773588in}}%
\pgfpathlineto{\pgfqpoint{-14.748835in}{0.773588in}}%
\pgfpathlineto{\pgfqpoint{-14.820408in}{0.773588in}}%
\pgfpathlineto{\pgfqpoint{-14.889438in}{0.773588in}}%
\pgfpathlineto{\pgfqpoint{-14.959865in}{0.773588in}}%
\pgfpathlineto{\pgfqpoint{-15.032843in}{0.773588in}}%
\pgfpathlineto{\pgfqpoint{-15.106256in}{0.773588in}}%
\pgfpathlineto{\pgfqpoint{-15.179182in}{0.773588in}}%
\pgfpathlineto{\pgfqpoint{-15.253001in}{0.773588in}}%
\pgfpathlineto{\pgfqpoint{-15.323000in}{0.773588in}}%
\pgfpathlineto{\pgfqpoint{-15.392048in}{0.773588in}}%
\pgfpathlineto{\pgfqpoint{-15.463558in}{0.773588in}}%
\pgfpathlineto{\pgfqpoint{-15.532639in}{0.773588in}}%
\pgfpathlineto{\pgfqpoint{-15.602559in}{0.773588in}}%
\pgfpathlineto{\pgfqpoint{-15.673559in}{0.773588in}}%
\pgfpathlineto{\pgfqpoint{-15.744769in}{0.773588in}}%
\pgfpathlineto{\pgfqpoint{-15.814335in}{0.773588in}}%
\pgfpathlineto{\pgfqpoint{-15.886757in}{0.773588in}}%
\pgfpathlineto{\pgfqpoint{-15.956358in}{0.773588in}}%
\pgfpathlineto{\pgfqpoint{-16.024276in}{0.773588in}}%
\pgfpathlineto{\pgfqpoint{-16.094895in}{0.773588in}}%
\pgfpathlineto{\pgfqpoint{-16.163654in}{0.773588in}}%
\pgfpathlineto{\pgfqpoint{-16.232068in}{0.773588in}}%
\pgfpathlineto{\pgfqpoint{-16.302639in}{0.773588in}}%
\pgfpathlineto{\pgfqpoint{-16.369724in}{0.773588in}}%
\pgfpathlineto{\pgfqpoint{-16.437097in}{0.773588in}}%
\pgfpathlineto{\pgfqpoint{-16.507797in}{0.773588in}}%
\pgfpathlineto{\pgfqpoint{-16.575864in}{0.773588in}}%
\pgfpathlineto{\pgfqpoint{-16.644038in}{0.773588in}}%
\pgfpathlineto{\pgfqpoint{-16.715630in}{0.773588in}}%
\pgfpathlineto{\pgfqpoint{-16.784012in}{0.773588in}}%
\pgfpathlineto{\pgfqpoint{-16.852234in}{0.773588in}}%
\pgfpathlineto{\pgfqpoint{-16.921714in}{0.773588in}}%
\pgfpathlineto{\pgfqpoint{-16.989953in}{0.773588in}}%
\pgfpathlineto{\pgfqpoint{-17.058127in}{0.773588in}}%
\pgfpathlineto{\pgfqpoint{-17.128098in}{0.773588in}}%
\pgfpathlineto{\pgfqpoint{-17.196258in}{0.773588in}}%
\pgfpathlineto{\pgfqpoint{-17.265552in}{0.773588in}}%
\pgfpathlineto{\pgfqpoint{-17.336134in}{0.773588in}}%
\pgfpathlineto{\pgfqpoint{-17.402675in}{0.773588in}}%
\pgfpathlineto{\pgfqpoint{-17.470673in}{0.773588in}}%
\pgfpathlineto{\pgfqpoint{-17.539598in}{0.773588in}}%
\pgfpathlineto{\pgfqpoint{-17.607453in}{0.773588in}}%
\pgfpathlineto{\pgfqpoint{-17.675451in}{0.773588in}}%
\pgfpathlineto{\pgfqpoint{-17.748008in}{0.773588in}}%
\pgfpathlineto{\pgfqpoint{-17.816737in}{0.773588in}}%
\pgfpathlineto{\pgfqpoint{-17.885356in}{0.773588in}}%
\pgfpathlineto{\pgfqpoint{-17.957185in}{0.773588in}}%
\pgfpathlineto{\pgfqpoint{-18.025876in}{0.773588in}}%
\pgfpathlineto{\pgfqpoint{-18.094404in}{0.773588in}}%
\pgfpathlineto{\pgfqpoint{-18.165145in}{0.773588in}}%
\pgfpathlineto{\pgfqpoint{-18.232823in}{0.773588in}}%
\pgfpathlineto{\pgfqpoint{-18.301710in}{0.773588in}}%
\pgfpathlineto{\pgfqpoint{-18.373957in}{0.773588in}}%
\pgfpathlineto{\pgfqpoint{-18.442527in}{0.773588in}}%
\pgfpathlineto{\pgfqpoint{-18.511159in}{0.773588in}}%
\pgfpathlineto{\pgfqpoint{-18.583201in}{0.773588in}}%
\pgfpathlineto{\pgfqpoint{-18.652004in}{0.773588in}}%
\pgfpathlineto{\pgfqpoint{-18.721737in}{0.773588in}}%
\pgfpathlineto{\pgfqpoint{-18.792270in}{0.773588in}}%
\pgfpathlineto{\pgfqpoint{-18.859965in}{0.773588in}}%
\pgfpathlineto{\pgfqpoint{-18.928709in}{0.773588in}}%
\pgfpathlineto{\pgfqpoint{-18.999640in}{0.773588in}}%
\pgfpathlineto{\pgfqpoint{-19.068147in}{0.773588in}}%
\pgfpathlineto{\pgfqpoint{-19.135788in}{0.773588in}}%
\pgfpathlineto{\pgfqpoint{-19.205228in}{0.773588in}}%
\pgfpathlineto{\pgfqpoint{-19.271476in}{0.773588in}}%
\pgfpathlineto{\pgfqpoint{-19.338364in}{0.773588in}}%
\pgfpathlineto{\pgfqpoint{-19.407422in}{0.773588in}}%
\pgfpathlineto{\pgfqpoint{-19.475599in}{0.773588in}}%
\pgfpathlineto{\pgfqpoint{-19.543393in}{0.773588in}}%
\pgfpathlineto{\pgfqpoint{-19.613839in}{0.773588in}}%
\pgfpathlineto{\pgfqpoint{-19.680997in}{0.773588in}}%
\pgfpathlineto{\pgfqpoint{-19.747542in}{0.773588in}}%
\pgfpathlineto{\pgfqpoint{-19.815535in}{0.773588in}}%
\pgfpathlineto{\pgfqpoint{-19.882267in}{0.773588in}}%
\pgfpathlineto{\pgfqpoint{-19.949841in}{0.773588in}}%
\pgfpathlineto{\pgfqpoint{-20.019359in}{0.773588in}}%
\pgfpathlineto{\pgfqpoint{-20.086943in}{0.773588in}}%
\pgfpathlineto{\pgfqpoint{-20.154101in}{0.773588in}}%
\pgfpathlineto{\pgfqpoint{-20.223560in}{0.773588in}}%
\pgfpathlineto{\pgfqpoint{-20.290460in}{0.773588in}}%
\pgfpathlineto{\pgfqpoint{-20.357252in}{0.773588in}}%
\pgfpathlineto{\pgfqpoint{-20.427103in}{0.773588in}}%
\pgfpathlineto{\pgfqpoint{-20.496847in}{0.773588in}}%
\pgfpathlineto{\pgfqpoint{-20.565295in}{0.773588in}}%
\pgfpathlineto{\pgfqpoint{-20.636228in}{0.773588in}}%
\pgfpathlineto{\pgfqpoint{-20.705385in}{0.773588in}}%
\pgfpathlineto{\pgfqpoint{-20.774312in}{0.773588in}}%
\pgfpathlineto{\pgfqpoint{-20.844220in}{0.773588in}}%
\pgfpathlineto{\pgfqpoint{-20.913231in}{0.773588in}}%
\pgfpathlineto{\pgfqpoint{-20.982338in}{0.773588in}}%
\pgfpathlineto{\pgfqpoint{-21.053709in}{0.773588in}}%
\pgfpathlineto{\pgfqpoint{-21.123238in}{0.773588in}}%
\pgfpathlineto{\pgfqpoint{-21.191384in}{0.773588in}}%
\pgfpathlineto{\pgfqpoint{-21.261541in}{0.773588in}}%
\pgfpathlineto{\pgfqpoint{-21.330134in}{0.773588in}}%
\pgfpathlineto{\pgfqpoint{-21.398346in}{0.773588in}}%
\pgfpathlineto{\pgfqpoint{-21.470634in}{0.773588in}}%
\pgfpathlineto{\pgfqpoint{-21.539549in}{0.773588in}}%
\pgfpathlineto{\pgfqpoint{-21.607622in}{0.773588in}}%
\pgfpathlineto{\pgfqpoint{-21.676089in}{0.773588in}}%
\pgfpathlineto{\pgfqpoint{-21.743422in}{0.773588in}}%
\pgfpathlineto{\pgfqpoint{-21.811964in}{0.773588in}}%
\pgfpathlineto{\pgfqpoint{-21.881469in}{0.773588in}}%
\pgfpathlineto{\pgfqpoint{-21.948488in}{0.773588in}}%
\pgfpathlineto{\pgfqpoint{-22.016091in}{0.773588in}}%
\pgfpathlineto{\pgfqpoint{-22.084923in}{0.773588in}}%
\pgfpathlineto{\pgfqpoint{-22.150098in}{0.773588in}}%
\pgfpathlineto{\pgfqpoint{-22.217508in}{0.773588in}}%
\pgfpathlineto{\pgfqpoint{-22.286974in}{0.773588in}}%
\pgfpathlineto{\pgfqpoint{-22.353759in}{0.773588in}}%
\pgfpathlineto{\pgfqpoint{-22.421718in}{0.773588in}}%
\pgfpathlineto{\pgfqpoint{-22.492630in}{0.773588in}}%
\pgfpathlineto{\pgfqpoint{-22.560590in}{0.773588in}}%
\pgfpathlineto{\pgfqpoint{-22.627376in}{0.773588in}}%
\pgfpathlineto{\pgfqpoint{-22.696138in}{0.773588in}}%
\pgfpathlineto{\pgfqpoint{-22.764599in}{0.773588in}}%
\pgfpathlineto{\pgfqpoint{-22.831621in}{0.773588in}}%
\pgfpathlineto{\pgfqpoint{-22.900479in}{0.773588in}}%
\pgfpathlineto{\pgfqpoint{-22.968635in}{0.773588in}}%
\pgfpathlineto{\pgfqpoint{-23.037151in}{0.773588in}}%
\pgfpathlineto{\pgfqpoint{-23.107951in}{0.773588in}}%
\pgfpathlineto{\pgfqpoint{-23.175783in}{0.773588in}}%
\pgfpathlineto{\pgfqpoint{-23.243519in}{0.773588in}}%
\pgfpathlineto{\pgfqpoint{-23.314406in}{0.773588in}}%
\pgfpathlineto{\pgfqpoint{-23.383303in}{0.773588in}}%
\pgfpathlineto{\pgfqpoint{-23.451780in}{0.773588in}}%
\pgfpathlineto{\pgfqpoint{-23.523979in}{0.773588in}}%
\pgfpathlineto{\pgfqpoint{-23.594251in}{0.773588in}}%
\pgfpathlineto{\pgfqpoint{-23.664795in}{0.773588in}}%
\pgfpathlineto{\pgfqpoint{-23.739951in}{0.773588in}}%
\pgfpathlineto{\pgfqpoint{-23.810815in}{0.773588in}}%
\pgfpathlineto{\pgfqpoint{-23.881105in}{0.773588in}}%
\pgfpathlineto{\pgfqpoint{-23.953546in}{0.773588in}}%
\pgfpathlineto{\pgfqpoint{-24.025129in}{0.773588in}}%
\pgfpathlineto{\pgfqpoint{-24.096931in}{0.773588in}}%
\pgfpathlineto{\pgfqpoint{-24.172427in}{0.773588in}}%
\pgfpathlineto{\pgfqpoint{-24.243942in}{0.773588in}}%
\pgfpathlineto{\pgfqpoint{-24.313406in}{0.773588in}}%
\pgfpathlineto{\pgfqpoint{-24.381203in}{0.773588in}}%
\pgfpathlineto{\pgfqpoint{-24.445935in}{0.773588in}}%
\pgfpathlineto{\pgfqpoint{-24.511431in}{0.773588in}}%
\pgfpathlineto{\pgfqpoint{-24.578904in}{0.773588in}}%
\pgfpathlineto{\pgfqpoint{-24.645693in}{0.773588in}}%
\pgfpathlineto{\pgfqpoint{-24.712889in}{0.773588in}}%
\pgfpathlineto{\pgfqpoint{-24.781655in}{0.773588in}}%
\pgfpathlineto{\pgfqpoint{-24.847966in}{0.773588in}}%
\pgfpathlineto{\pgfqpoint{-24.916019in}{0.773588in}}%
\pgfpathlineto{\pgfqpoint{-24.984232in}{0.773588in}}%
\pgfpathlineto{\pgfqpoint{-25.049351in}{0.773588in}}%
\pgfpathlineto{\pgfqpoint{-25.114674in}{0.773588in}}%
\pgfpathlineto{\pgfqpoint{-25.182394in}{0.773588in}}%
\pgfpathlineto{\pgfqpoint{-25.248757in}{0.773588in}}%
\pgfpathlineto{\pgfqpoint{-25.315375in}{0.773588in}}%
\pgfpathlineto{\pgfqpoint{-25.382786in}{0.773588in}}%
\pgfpathlineto{\pgfqpoint{-25.448976in}{0.773588in}}%
\pgfpathlineto{\pgfqpoint{-25.515370in}{0.773588in}}%
\pgfpathlineto{\pgfqpoint{-25.583742in}{0.773588in}}%
\pgfpathlineto{\pgfqpoint{-25.650614in}{0.773588in}}%
\pgfpathlineto{\pgfqpoint{-25.717568in}{0.773588in}}%
\pgfpathlineto{\pgfqpoint{-25.785528in}{0.773588in}}%
\pgfpathlineto{\pgfqpoint{-25.851769in}{0.773588in}}%
\pgfpathlineto{\pgfqpoint{-25.919100in}{0.773588in}}%
\pgfpathlineto{\pgfqpoint{-25.990773in}{0.773588in}}%
\pgfpathlineto{\pgfqpoint{-26.059544in}{0.773588in}}%
\pgfpathlineto{\pgfqpoint{-26.128033in}{0.773588in}}%
\pgfpathlineto{\pgfqpoint{-26.198294in}{0.773588in}}%
\pgfpathlineto{\pgfqpoint{-26.268985in}{0.773588in}}%
\pgfpathlineto{\pgfqpoint{-26.338662in}{0.773588in}}%
\pgfpathlineto{\pgfqpoint{-26.410031in}{0.773588in}}%
\pgfpathlineto{\pgfqpoint{-26.479146in}{0.773588in}}%
\pgfpathlineto{\pgfqpoint{-26.547093in}{0.773588in}}%
\pgfpathlineto{\pgfqpoint{-26.616985in}{0.773588in}}%
\pgfpathlineto{\pgfqpoint{-26.685925in}{0.773588in}}%
\pgfpathlineto{\pgfqpoint{-26.755672in}{0.773588in}}%
\pgfpathlineto{\pgfqpoint{-26.825734in}{0.773588in}}%
\pgfpathlineto{\pgfqpoint{-26.893662in}{0.773588in}}%
\pgfpathlineto{\pgfqpoint{-26.961733in}{0.773588in}}%
\pgfpathlineto{\pgfqpoint{-27.031677in}{0.773588in}}%
\pgfpathlineto{\pgfqpoint{-27.097972in}{0.773588in}}%
\pgfpathlineto{\pgfqpoint{-27.165061in}{0.773588in}}%
\pgfpathlineto{\pgfqpoint{-27.232937in}{0.773588in}}%
\pgfpathlineto{\pgfqpoint{-27.299189in}{0.773588in}}%
\pgfpathlineto{\pgfqpoint{-27.366087in}{0.773588in}}%
\pgfpathlineto{\pgfqpoint{-27.435671in}{0.773588in}}%
\pgfpathlineto{\pgfqpoint{-27.503250in}{0.773588in}}%
\pgfpathlineto{\pgfqpoint{-27.570291in}{0.773588in}}%
\pgfpathlineto{\pgfqpoint{-27.637925in}{0.773588in}}%
\pgfpathlineto{\pgfqpoint{-27.703951in}{0.773588in}}%
\pgfpathlineto{\pgfqpoint{-27.771032in}{0.773588in}}%
\pgfpathlineto{\pgfqpoint{-27.839702in}{0.773588in}}%
\pgfpathlineto{\pgfqpoint{-27.908279in}{0.773588in}}%
\pgfpathlineto{\pgfqpoint{-27.976694in}{0.773588in}}%
\pgfpathlineto{\pgfqpoint{-28.049197in}{0.773588in}}%
\pgfpathlineto{\pgfqpoint{-28.121663in}{0.773588in}}%
\pgfpathlineto{\pgfqpoint{-28.195772in}{0.773588in}}%
\pgfpathlineto{\pgfqpoint{-28.274638in}{0.773588in}}%
\pgfpathlineto{\pgfqpoint{-28.356803in}{0.773588in}}%
\pgfpathlineto{\pgfqpoint{-28.429123in}{0.773588in}}%
\pgfpathlineto{\pgfqpoint{-28.504596in}{0.773588in}}%
\pgfpathlineto{\pgfqpoint{-28.577221in}{0.773588in}}%
\pgfpathlineto{\pgfqpoint{-28.649563in}{0.773588in}}%
\pgfpathlineto{\pgfqpoint{-28.725736in}{0.773588in}}%
\pgfpathlineto{\pgfqpoint{-28.798746in}{0.773588in}}%
\pgfpathlineto{\pgfqpoint{-28.870906in}{0.773588in}}%
\pgfpathlineto{\pgfqpoint{-28.946213in}{0.773588in}}%
\pgfpathlineto{\pgfqpoint{-29.018119in}{0.773588in}}%
\pgfpathlineto{\pgfqpoint{-29.089467in}{0.773588in}}%
\pgfpathlineto{\pgfqpoint{-29.164288in}{0.773588in}}%
\pgfpathlineto{\pgfqpoint{-29.235084in}{0.773588in}}%
\pgfpathlineto{\pgfqpoint{-29.304065in}{0.773588in}}%
\pgfpathlineto{\pgfqpoint{-29.374900in}{0.773588in}}%
\pgfpathlineto{\pgfqpoint{-29.443353in}{0.773588in}}%
\pgfpathlineto{\pgfqpoint{-29.511058in}{0.773588in}}%
\pgfpathlineto{\pgfqpoint{-29.581285in}{0.773588in}}%
\pgfpathlineto{\pgfqpoint{-29.647778in}{0.773588in}}%
\pgfpathlineto{\pgfqpoint{-29.716346in}{0.773588in}}%
\pgfpathlineto{\pgfqpoint{-29.785834in}{0.773588in}}%
\pgfpathlineto{\pgfqpoint{-29.853575in}{0.773588in}}%
\pgfpathlineto{\pgfqpoint{-29.923374in}{0.773588in}}%
\pgfpathlineto{\pgfqpoint{-29.994776in}{0.773588in}}%
\pgfpathlineto{\pgfqpoint{-30.062931in}{0.773588in}}%
\pgfpathlineto{\pgfqpoint{-30.131142in}{0.773588in}}%
\pgfpathlineto{\pgfqpoint{-30.202864in}{0.773588in}}%
\pgfpathlineto{\pgfqpoint{-30.269201in}{0.773588in}}%
\pgfpathlineto{\pgfqpoint{-30.335915in}{0.773588in}}%
\pgfpathlineto{\pgfqpoint{-30.404868in}{0.773588in}}%
\pgfpathlineto{\pgfqpoint{-30.472294in}{0.773588in}}%
\pgfpathlineto{\pgfqpoint{-30.539383in}{0.773588in}}%
\pgfpathlineto{\pgfqpoint{-30.610127in}{0.773588in}}%
\pgfpathlineto{\pgfqpoint{-30.676901in}{0.773588in}}%
\pgfpathlineto{\pgfqpoint{-30.745143in}{0.773588in}}%
\pgfpathclose%
\pgfusepath{fill}%
\end{pgfscope}%
\begin{pgfscope}%
\pgfpathrectangle{\pgfqpoint{3.332180in}{0.773588in}}{\pgfqpoint{2.293918in}{5.415119in}}%
\pgfusepath{clip}%
\pgfsetbuttcap%
\pgfsetroundjoin%
\definecolor{currentfill}{rgb}{0.172549,0.627451,0.172549}%
\pgfsetfillcolor{currentfill}%
\pgfsetlinewidth{0.000000pt}%
\definecolor{currentstroke}{rgb}{0.000000,0.000000,0.000000}%
\pgfsetstrokecolor{currentstroke}%
\pgfsetdash{}{0pt}%
\pgfpathmoveto{\pgfqpoint{-30.745143in}{0.773588in}}%
\pgfpathlineto{\pgfqpoint{-30.745143in}{0.773588in}}%
\pgfpathlineto{\pgfqpoint{-30.676901in}{0.773588in}}%
\pgfpathlineto{\pgfqpoint{-30.610127in}{0.773588in}}%
\pgfpathlineto{\pgfqpoint{-30.539383in}{0.773588in}}%
\pgfpathlineto{\pgfqpoint{-30.472294in}{0.773588in}}%
\pgfpathlineto{\pgfqpoint{-30.404868in}{0.773588in}}%
\pgfpathlineto{\pgfqpoint{-30.335915in}{0.773588in}}%
\pgfpathlineto{\pgfqpoint{-30.269201in}{0.773588in}}%
\pgfpathlineto{\pgfqpoint{-30.202864in}{0.773588in}}%
\pgfpathlineto{\pgfqpoint{-30.131142in}{0.773588in}}%
\pgfpathlineto{\pgfqpoint{-30.062931in}{0.773588in}}%
\pgfpathlineto{\pgfqpoint{-29.994776in}{0.773588in}}%
\pgfpathlineto{\pgfqpoint{-29.923374in}{0.773588in}}%
\pgfpathlineto{\pgfqpoint{-29.853575in}{0.773588in}}%
\pgfpathlineto{\pgfqpoint{-29.785834in}{0.773588in}}%
\pgfpathlineto{\pgfqpoint{-29.716346in}{0.773588in}}%
\pgfpathlineto{\pgfqpoint{-29.647778in}{0.773588in}}%
\pgfpathlineto{\pgfqpoint{-29.581285in}{0.773588in}}%
\pgfpathlineto{\pgfqpoint{-29.511058in}{0.773588in}}%
\pgfpathlineto{\pgfqpoint{-29.443353in}{0.773588in}}%
\pgfpathlineto{\pgfqpoint{-29.374900in}{0.773588in}}%
\pgfpathlineto{\pgfqpoint{-29.304065in}{0.773588in}}%
\pgfpathlineto{\pgfqpoint{-29.235084in}{0.773588in}}%
\pgfpathlineto{\pgfqpoint{-29.164288in}{0.773588in}}%
\pgfpathlineto{\pgfqpoint{-29.089467in}{0.773588in}}%
\pgfpathlineto{\pgfqpoint{-29.018119in}{0.773588in}}%
\pgfpathlineto{\pgfqpoint{-28.946213in}{0.773588in}}%
\pgfpathlineto{\pgfqpoint{-28.870906in}{0.773588in}}%
\pgfpathlineto{\pgfqpoint{-28.798746in}{0.773588in}}%
\pgfpathlineto{\pgfqpoint{-28.725736in}{0.773588in}}%
\pgfpathlineto{\pgfqpoint{-28.649563in}{0.773588in}}%
\pgfpathlineto{\pgfqpoint{-28.577221in}{0.773588in}}%
\pgfpathlineto{\pgfqpoint{-28.504596in}{0.773588in}}%
\pgfpathlineto{\pgfqpoint{-28.429123in}{0.773588in}}%
\pgfpathlineto{\pgfqpoint{-28.356803in}{0.773588in}}%
\pgfpathlineto{\pgfqpoint{-28.274638in}{0.773588in}}%
\pgfpathlineto{\pgfqpoint{-28.195772in}{0.773588in}}%
\pgfpathlineto{\pgfqpoint{-28.121663in}{0.773588in}}%
\pgfpathlineto{\pgfqpoint{-28.049197in}{0.773588in}}%
\pgfpathlineto{\pgfqpoint{-27.976694in}{0.773588in}}%
\pgfpathlineto{\pgfqpoint{-27.908279in}{0.773588in}}%
\pgfpathlineto{\pgfqpoint{-27.839702in}{0.773588in}}%
\pgfpathlineto{\pgfqpoint{-27.771032in}{0.773588in}}%
\pgfpathlineto{\pgfqpoint{-27.703951in}{0.773588in}}%
\pgfpathlineto{\pgfqpoint{-27.637925in}{0.773588in}}%
\pgfpathlineto{\pgfqpoint{-27.570291in}{0.773588in}}%
\pgfpathlineto{\pgfqpoint{-27.503250in}{0.773588in}}%
\pgfpathlineto{\pgfqpoint{-27.435671in}{0.773588in}}%
\pgfpathlineto{\pgfqpoint{-27.366087in}{0.773588in}}%
\pgfpathlineto{\pgfqpoint{-27.299189in}{0.773588in}}%
\pgfpathlineto{\pgfqpoint{-27.232937in}{0.773588in}}%
\pgfpathlineto{\pgfqpoint{-27.165061in}{0.773588in}}%
\pgfpathlineto{\pgfqpoint{-27.097972in}{0.773588in}}%
\pgfpathlineto{\pgfqpoint{-27.031677in}{0.773588in}}%
\pgfpathlineto{\pgfqpoint{-26.961733in}{0.773588in}}%
\pgfpathlineto{\pgfqpoint{-26.893662in}{0.773588in}}%
\pgfpathlineto{\pgfqpoint{-26.825734in}{0.773588in}}%
\pgfpathlineto{\pgfqpoint{-26.755672in}{0.773588in}}%
\pgfpathlineto{\pgfqpoint{-26.685925in}{0.773588in}}%
\pgfpathlineto{\pgfqpoint{-26.616985in}{0.773588in}}%
\pgfpathlineto{\pgfqpoint{-26.547093in}{0.773588in}}%
\pgfpathlineto{\pgfqpoint{-26.479146in}{0.773588in}}%
\pgfpathlineto{\pgfqpoint{-26.410031in}{0.773588in}}%
\pgfpathlineto{\pgfqpoint{-26.338662in}{0.773588in}}%
\pgfpathlineto{\pgfqpoint{-26.268985in}{0.773588in}}%
\pgfpathlineto{\pgfqpoint{-26.198294in}{0.773588in}}%
\pgfpathlineto{\pgfqpoint{-26.128033in}{0.773588in}}%
\pgfpathlineto{\pgfqpoint{-26.059544in}{0.773588in}}%
\pgfpathlineto{\pgfqpoint{-25.990773in}{0.773588in}}%
\pgfpathlineto{\pgfqpoint{-25.919100in}{0.773588in}}%
\pgfpathlineto{\pgfqpoint{-25.851769in}{0.773588in}}%
\pgfpathlineto{\pgfqpoint{-25.785528in}{0.773588in}}%
\pgfpathlineto{\pgfqpoint{-25.717568in}{0.773588in}}%
\pgfpathlineto{\pgfqpoint{-25.650614in}{0.773588in}}%
\pgfpathlineto{\pgfqpoint{-25.583742in}{0.773588in}}%
\pgfpathlineto{\pgfqpoint{-25.515370in}{0.773588in}}%
\pgfpathlineto{\pgfqpoint{-25.448976in}{0.773588in}}%
\pgfpathlineto{\pgfqpoint{-25.382786in}{0.773588in}}%
\pgfpathlineto{\pgfqpoint{-25.315375in}{0.773588in}}%
\pgfpathlineto{\pgfqpoint{-25.248757in}{0.773588in}}%
\pgfpathlineto{\pgfqpoint{-25.182394in}{0.773588in}}%
\pgfpathlineto{\pgfqpoint{-25.114674in}{0.773588in}}%
\pgfpathlineto{\pgfqpoint{-25.049351in}{0.773588in}}%
\pgfpathlineto{\pgfqpoint{-24.984232in}{0.773588in}}%
\pgfpathlineto{\pgfqpoint{-24.916019in}{0.773588in}}%
\pgfpathlineto{\pgfqpoint{-24.847966in}{0.773588in}}%
\pgfpathlineto{\pgfqpoint{-24.781655in}{0.773588in}}%
\pgfpathlineto{\pgfqpoint{-24.712889in}{0.773588in}}%
\pgfpathlineto{\pgfqpoint{-24.645693in}{0.773588in}}%
\pgfpathlineto{\pgfqpoint{-24.578904in}{0.773588in}}%
\pgfpathlineto{\pgfqpoint{-24.511431in}{0.773588in}}%
\pgfpathlineto{\pgfqpoint{-24.445935in}{0.773588in}}%
\pgfpathlineto{\pgfqpoint{-24.381203in}{0.773588in}}%
\pgfpathlineto{\pgfqpoint{-24.313406in}{0.773588in}}%
\pgfpathlineto{\pgfqpoint{-24.243942in}{0.773588in}}%
\pgfpathlineto{\pgfqpoint{-24.172427in}{0.773588in}}%
\pgfpathlineto{\pgfqpoint{-24.096931in}{0.773588in}}%
\pgfpathlineto{\pgfqpoint{-24.025129in}{0.773588in}}%
\pgfpathlineto{\pgfqpoint{-23.953546in}{0.773588in}}%
\pgfpathlineto{\pgfqpoint{-23.881105in}{0.773588in}}%
\pgfpathlineto{\pgfqpoint{-23.810815in}{0.773588in}}%
\pgfpathlineto{\pgfqpoint{-23.739951in}{0.773588in}}%
\pgfpathlineto{\pgfqpoint{-23.664795in}{0.773588in}}%
\pgfpathlineto{\pgfqpoint{-23.594251in}{0.773588in}}%
\pgfpathlineto{\pgfqpoint{-23.523979in}{0.773588in}}%
\pgfpathlineto{\pgfqpoint{-23.451780in}{0.773588in}}%
\pgfpathlineto{\pgfqpoint{-23.383303in}{0.773588in}}%
\pgfpathlineto{\pgfqpoint{-23.314406in}{0.773588in}}%
\pgfpathlineto{\pgfqpoint{-23.243519in}{0.773588in}}%
\pgfpathlineto{\pgfqpoint{-23.175783in}{0.773588in}}%
\pgfpathlineto{\pgfqpoint{-23.107951in}{0.773588in}}%
\pgfpathlineto{\pgfqpoint{-23.037151in}{0.773588in}}%
\pgfpathlineto{\pgfqpoint{-22.968635in}{0.773588in}}%
\pgfpathlineto{\pgfqpoint{-22.900479in}{0.773588in}}%
\pgfpathlineto{\pgfqpoint{-22.831621in}{0.773588in}}%
\pgfpathlineto{\pgfqpoint{-22.764599in}{0.773588in}}%
\pgfpathlineto{\pgfqpoint{-22.696138in}{0.773588in}}%
\pgfpathlineto{\pgfqpoint{-22.627376in}{0.773588in}}%
\pgfpathlineto{\pgfqpoint{-22.560590in}{0.773588in}}%
\pgfpathlineto{\pgfqpoint{-22.492630in}{0.773588in}}%
\pgfpathlineto{\pgfqpoint{-22.421718in}{0.773588in}}%
\pgfpathlineto{\pgfqpoint{-22.353759in}{0.773588in}}%
\pgfpathlineto{\pgfqpoint{-22.286974in}{0.773588in}}%
\pgfpathlineto{\pgfqpoint{-22.217508in}{0.773588in}}%
\pgfpathlineto{\pgfqpoint{-22.150098in}{0.773588in}}%
\pgfpathlineto{\pgfqpoint{-22.084923in}{0.773588in}}%
\pgfpathlineto{\pgfqpoint{-22.016091in}{0.773588in}}%
\pgfpathlineto{\pgfqpoint{-21.948488in}{0.773588in}}%
\pgfpathlineto{\pgfqpoint{-21.881469in}{0.773588in}}%
\pgfpathlineto{\pgfqpoint{-21.811964in}{0.773588in}}%
\pgfpathlineto{\pgfqpoint{-21.743422in}{0.773588in}}%
\pgfpathlineto{\pgfqpoint{-21.676089in}{0.773588in}}%
\pgfpathlineto{\pgfqpoint{-21.607622in}{0.773588in}}%
\pgfpathlineto{\pgfqpoint{-21.539549in}{0.773588in}}%
\pgfpathlineto{\pgfqpoint{-21.470634in}{0.773588in}}%
\pgfpathlineto{\pgfqpoint{-21.398346in}{0.773588in}}%
\pgfpathlineto{\pgfqpoint{-21.330134in}{0.773588in}}%
\pgfpathlineto{\pgfqpoint{-21.261541in}{0.773588in}}%
\pgfpathlineto{\pgfqpoint{-21.191384in}{0.773588in}}%
\pgfpathlineto{\pgfqpoint{-21.123238in}{0.773588in}}%
\pgfpathlineto{\pgfqpoint{-21.053709in}{0.773588in}}%
\pgfpathlineto{\pgfqpoint{-20.982338in}{0.773588in}}%
\pgfpathlineto{\pgfqpoint{-20.913231in}{0.773588in}}%
\pgfpathlineto{\pgfqpoint{-20.844220in}{0.773588in}}%
\pgfpathlineto{\pgfqpoint{-20.774312in}{0.773588in}}%
\pgfpathlineto{\pgfqpoint{-20.705385in}{0.773588in}}%
\pgfpathlineto{\pgfqpoint{-20.636228in}{0.773588in}}%
\pgfpathlineto{\pgfqpoint{-20.565295in}{0.773588in}}%
\pgfpathlineto{\pgfqpoint{-20.496847in}{0.773588in}}%
\pgfpathlineto{\pgfqpoint{-20.427103in}{0.773588in}}%
\pgfpathlineto{\pgfqpoint{-20.357252in}{0.773588in}}%
\pgfpathlineto{\pgfqpoint{-20.290460in}{0.773588in}}%
\pgfpathlineto{\pgfqpoint{-20.223560in}{0.773588in}}%
\pgfpathlineto{\pgfqpoint{-20.154101in}{0.773588in}}%
\pgfpathlineto{\pgfqpoint{-20.086943in}{0.773588in}}%
\pgfpathlineto{\pgfqpoint{-20.019359in}{0.773588in}}%
\pgfpathlineto{\pgfqpoint{-19.949841in}{0.773588in}}%
\pgfpathlineto{\pgfqpoint{-19.882267in}{0.773588in}}%
\pgfpathlineto{\pgfqpoint{-19.815535in}{0.773588in}}%
\pgfpathlineto{\pgfqpoint{-19.747542in}{0.773588in}}%
\pgfpathlineto{\pgfqpoint{-19.680997in}{0.773588in}}%
\pgfpathlineto{\pgfqpoint{-19.613839in}{0.773588in}}%
\pgfpathlineto{\pgfqpoint{-19.543393in}{0.773588in}}%
\pgfpathlineto{\pgfqpoint{-19.475599in}{0.773588in}}%
\pgfpathlineto{\pgfqpoint{-19.407422in}{0.773588in}}%
\pgfpathlineto{\pgfqpoint{-19.338364in}{0.773588in}}%
\pgfpathlineto{\pgfqpoint{-19.271476in}{0.773588in}}%
\pgfpathlineto{\pgfqpoint{-19.205228in}{0.773588in}}%
\pgfpathlineto{\pgfqpoint{-19.135788in}{0.773588in}}%
\pgfpathlineto{\pgfqpoint{-19.068147in}{0.773588in}}%
\pgfpathlineto{\pgfqpoint{-18.999640in}{0.773588in}}%
\pgfpathlineto{\pgfqpoint{-18.928709in}{0.773588in}}%
\pgfpathlineto{\pgfqpoint{-18.859965in}{0.773588in}}%
\pgfpathlineto{\pgfqpoint{-18.792270in}{0.773588in}}%
\pgfpathlineto{\pgfqpoint{-18.721737in}{0.773588in}}%
\pgfpathlineto{\pgfqpoint{-18.652004in}{0.773588in}}%
\pgfpathlineto{\pgfqpoint{-18.583201in}{0.773588in}}%
\pgfpathlineto{\pgfqpoint{-18.511159in}{0.773588in}}%
\pgfpathlineto{\pgfqpoint{-18.442527in}{0.773588in}}%
\pgfpathlineto{\pgfqpoint{-18.373957in}{0.773588in}}%
\pgfpathlineto{\pgfqpoint{-18.301710in}{0.773588in}}%
\pgfpathlineto{\pgfqpoint{-18.232823in}{0.773588in}}%
\pgfpathlineto{\pgfqpoint{-18.165145in}{0.773588in}}%
\pgfpathlineto{\pgfqpoint{-18.094404in}{0.773588in}}%
\pgfpathlineto{\pgfqpoint{-18.025876in}{0.773588in}}%
\pgfpathlineto{\pgfqpoint{-17.957185in}{0.773588in}}%
\pgfpathlineto{\pgfqpoint{-17.885356in}{0.773588in}}%
\pgfpathlineto{\pgfqpoint{-17.816737in}{0.773588in}}%
\pgfpathlineto{\pgfqpoint{-17.748008in}{0.773588in}}%
\pgfpathlineto{\pgfqpoint{-17.675451in}{0.773588in}}%
\pgfpathlineto{\pgfqpoint{-17.607453in}{0.773588in}}%
\pgfpathlineto{\pgfqpoint{-17.539598in}{0.773588in}}%
\pgfpathlineto{\pgfqpoint{-17.470673in}{0.773588in}}%
\pgfpathlineto{\pgfqpoint{-17.402675in}{0.773588in}}%
\pgfpathlineto{\pgfqpoint{-17.336134in}{0.773588in}}%
\pgfpathlineto{\pgfqpoint{-17.265552in}{0.773588in}}%
\pgfpathlineto{\pgfqpoint{-17.196258in}{0.773588in}}%
\pgfpathlineto{\pgfqpoint{-17.128098in}{0.773588in}}%
\pgfpathlineto{\pgfqpoint{-17.058127in}{0.773588in}}%
\pgfpathlineto{\pgfqpoint{-16.989953in}{0.773588in}}%
\pgfpathlineto{\pgfqpoint{-16.921714in}{0.773588in}}%
\pgfpathlineto{\pgfqpoint{-16.852234in}{0.773588in}}%
\pgfpathlineto{\pgfqpoint{-16.784012in}{0.773588in}}%
\pgfpathlineto{\pgfqpoint{-16.715630in}{0.773588in}}%
\pgfpathlineto{\pgfqpoint{-16.644038in}{0.773588in}}%
\pgfpathlineto{\pgfqpoint{-16.575864in}{0.773588in}}%
\pgfpathlineto{\pgfqpoint{-16.507797in}{0.773588in}}%
\pgfpathlineto{\pgfqpoint{-16.437097in}{0.773588in}}%
\pgfpathlineto{\pgfqpoint{-16.369724in}{0.773588in}}%
\pgfpathlineto{\pgfqpoint{-16.302639in}{0.773588in}}%
\pgfpathlineto{\pgfqpoint{-16.232068in}{0.773588in}}%
\pgfpathlineto{\pgfqpoint{-16.163654in}{0.773588in}}%
\pgfpathlineto{\pgfqpoint{-16.094895in}{0.773588in}}%
\pgfpathlineto{\pgfqpoint{-16.024276in}{0.773588in}}%
\pgfpathlineto{\pgfqpoint{-15.956358in}{0.773588in}}%
\pgfpathlineto{\pgfqpoint{-15.886757in}{0.773588in}}%
\pgfpathlineto{\pgfqpoint{-15.814335in}{0.773588in}}%
\pgfpathlineto{\pgfqpoint{-15.744769in}{0.773588in}}%
\pgfpathlineto{\pgfqpoint{-15.673559in}{0.773588in}}%
\pgfpathlineto{\pgfqpoint{-15.602559in}{0.773588in}}%
\pgfpathlineto{\pgfqpoint{-15.532639in}{0.773588in}}%
\pgfpathlineto{\pgfqpoint{-15.463558in}{0.773588in}}%
\pgfpathlineto{\pgfqpoint{-15.392048in}{0.773588in}}%
\pgfpathlineto{\pgfqpoint{-15.323000in}{0.773588in}}%
\pgfpathlineto{\pgfqpoint{-15.253001in}{0.773588in}}%
\pgfpathlineto{\pgfqpoint{-15.179182in}{0.773588in}}%
\pgfpathlineto{\pgfqpoint{-15.106256in}{0.773588in}}%
\pgfpathlineto{\pgfqpoint{-15.032843in}{0.773588in}}%
\pgfpathlineto{\pgfqpoint{-14.959865in}{0.773588in}}%
\pgfpathlineto{\pgfqpoint{-14.889438in}{0.773588in}}%
\pgfpathlineto{\pgfqpoint{-14.820408in}{0.773588in}}%
\pgfpathlineto{\pgfqpoint{-14.748835in}{0.773588in}}%
\pgfpathlineto{\pgfqpoint{-14.679263in}{0.773588in}}%
\pgfpathlineto{\pgfqpoint{-14.609938in}{0.773588in}}%
\pgfpathlineto{\pgfqpoint{-14.538988in}{0.773588in}}%
\pgfpathlineto{\pgfqpoint{-14.471123in}{0.773588in}}%
\pgfpathlineto{\pgfqpoint{-14.401957in}{0.773588in}}%
\pgfpathlineto{\pgfqpoint{-14.331277in}{0.773588in}}%
\pgfpathlineto{\pgfqpoint{-14.263436in}{0.773588in}}%
\pgfpathlineto{\pgfqpoint{-14.195714in}{0.773588in}}%
\pgfpathlineto{\pgfqpoint{-14.125666in}{0.773588in}}%
\pgfpathlineto{\pgfqpoint{-14.057377in}{0.773588in}}%
\pgfpathlineto{\pgfqpoint{-13.988983in}{0.773588in}}%
\pgfpathlineto{\pgfqpoint{-13.917455in}{0.773588in}}%
\pgfpathlineto{\pgfqpoint{-13.848756in}{0.773588in}}%
\pgfpathlineto{\pgfqpoint{-13.779339in}{0.773588in}}%
\pgfpathlineto{\pgfqpoint{-13.708089in}{0.773588in}}%
\pgfpathlineto{\pgfqpoint{-13.639609in}{0.773588in}}%
\pgfpathlineto{\pgfqpoint{-13.568713in}{0.773588in}}%
\pgfpathlineto{\pgfqpoint{-13.496151in}{0.773588in}}%
\pgfpathlineto{\pgfqpoint{-13.426276in}{0.773588in}}%
\pgfpathlineto{\pgfqpoint{-13.356968in}{0.773588in}}%
\pgfpathlineto{\pgfqpoint{-13.285456in}{0.773588in}}%
\pgfpathlineto{\pgfqpoint{-13.216013in}{0.773588in}}%
\pgfpathlineto{\pgfqpoint{-13.145337in}{0.773588in}}%
\pgfpathlineto{\pgfqpoint{-13.073093in}{0.773588in}}%
\pgfpathlineto{\pgfqpoint{-13.002727in}{0.773588in}}%
\pgfpathlineto{\pgfqpoint{-12.932388in}{0.773588in}}%
\pgfpathlineto{\pgfqpoint{-12.859398in}{0.773588in}}%
\pgfpathlineto{\pgfqpoint{-12.788679in}{0.773588in}}%
\pgfpathlineto{\pgfqpoint{-12.718096in}{0.773588in}}%
\pgfpathlineto{\pgfqpoint{-12.645279in}{0.773588in}}%
\pgfpathlineto{\pgfqpoint{-12.574652in}{0.773588in}}%
\pgfpathlineto{\pgfqpoint{-12.503401in}{0.773588in}}%
\pgfpathlineto{\pgfqpoint{-12.428997in}{0.773588in}}%
\pgfpathlineto{\pgfqpoint{-12.357103in}{0.773588in}}%
\pgfpathlineto{\pgfqpoint{-12.285275in}{0.773588in}}%
\pgfpathlineto{\pgfqpoint{-12.211536in}{0.773588in}}%
\pgfpathlineto{\pgfqpoint{-12.139885in}{0.773588in}}%
\pgfpathlineto{\pgfqpoint{-12.069446in}{0.773588in}}%
\pgfpathlineto{\pgfqpoint{-11.998596in}{0.773588in}}%
\pgfpathlineto{\pgfqpoint{-11.928381in}{0.773588in}}%
\pgfpathlineto{\pgfqpoint{-11.858471in}{0.773588in}}%
\pgfpathlineto{\pgfqpoint{-11.786176in}{0.773588in}}%
\pgfpathlineto{\pgfqpoint{-11.716309in}{0.773588in}}%
\pgfpathlineto{\pgfqpoint{-11.645187in}{0.773588in}}%
\pgfpathlineto{\pgfqpoint{-11.571037in}{0.773588in}}%
\pgfpathlineto{\pgfqpoint{-11.500166in}{0.773588in}}%
\pgfpathlineto{\pgfqpoint{-11.430351in}{0.773588in}}%
\pgfpathlineto{\pgfqpoint{-11.358676in}{0.773588in}}%
\pgfpathlineto{\pgfqpoint{-11.288273in}{0.773588in}}%
\pgfpathlineto{\pgfqpoint{-11.217631in}{0.773588in}}%
\pgfpathlineto{\pgfqpoint{-11.144960in}{0.773588in}}%
\pgfpathlineto{\pgfqpoint{-11.075276in}{0.773588in}}%
\pgfpathlineto{\pgfqpoint{-11.005065in}{0.773588in}}%
\pgfpathlineto{\pgfqpoint{-10.933808in}{0.773588in}}%
\pgfpathlineto{\pgfqpoint{-10.863257in}{0.773588in}}%
\pgfpathlineto{\pgfqpoint{-10.793422in}{0.773588in}}%
\pgfpathlineto{\pgfqpoint{-10.721651in}{0.773588in}}%
\pgfpathlineto{\pgfqpoint{-10.652649in}{0.773588in}}%
\pgfpathlineto{\pgfqpoint{-10.582864in}{0.773588in}}%
\pgfpathlineto{\pgfqpoint{-10.511372in}{0.773588in}}%
\pgfpathlineto{\pgfqpoint{-10.441581in}{0.773588in}}%
\pgfpathlineto{\pgfqpoint{-10.371004in}{0.773588in}}%
\pgfpathlineto{\pgfqpoint{-10.297591in}{0.773588in}}%
\pgfpathlineto{\pgfqpoint{-10.227739in}{0.773588in}}%
\pgfpathlineto{\pgfqpoint{-10.157398in}{0.773588in}}%
\pgfpathlineto{\pgfqpoint{-10.084496in}{0.773588in}}%
\pgfpathlineto{\pgfqpoint{-10.012635in}{0.773588in}}%
\pgfpathlineto{\pgfqpoint{-9.941359in}{0.773588in}}%
\pgfpathlineto{\pgfqpoint{-9.868243in}{0.773588in}}%
\pgfpathlineto{\pgfqpoint{-9.796010in}{0.773588in}}%
\pgfpathlineto{\pgfqpoint{-9.723682in}{0.773588in}}%
\pgfpathlineto{\pgfqpoint{-9.648369in}{0.773588in}}%
\pgfpathlineto{\pgfqpoint{-9.575400in}{0.773588in}}%
\pgfpathlineto{\pgfqpoint{-9.504196in}{0.773588in}}%
\pgfpathlineto{\pgfqpoint{-9.429774in}{0.773588in}}%
\pgfpathlineto{\pgfqpoint{-9.356529in}{0.773588in}}%
\pgfpathlineto{\pgfqpoint{-9.283412in}{0.773588in}}%
\pgfpathlineto{\pgfqpoint{-9.208906in}{0.773588in}}%
\pgfpathlineto{\pgfqpoint{-9.138788in}{0.773588in}}%
\pgfpathlineto{\pgfqpoint{-9.069721in}{0.773588in}}%
\pgfpathlineto{\pgfqpoint{-8.998701in}{0.773588in}}%
\pgfpathlineto{\pgfqpoint{-8.928123in}{0.773588in}}%
\pgfpathlineto{\pgfqpoint{-8.858221in}{0.773588in}}%
\pgfpathlineto{\pgfqpoint{-8.787586in}{0.773588in}}%
\pgfpathlineto{\pgfqpoint{-8.718609in}{0.773588in}}%
\pgfpathlineto{\pgfqpoint{-8.649341in}{0.773588in}}%
\pgfpathlineto{\pgfqpoint{-8.578369in}{0.773588in}}%
\pgfpathlineto{\pgfqpoint{-8.508839in}{0.773588in}}%
\pgfpathlineto{\pgfqpoint{-8.438970in}{0.773588in}}%
\pgfpathlineto{\pgfqpoint{-8.367967in}{0.773588in}}%
\pgfpathlineto{\pgfqpoint{-8.298753in}{0.773588in}}%
\pgfpathlineto{\pgfqpoint{-8.228241in}{0.773588in}}%
\pgfpathlineto{\pgfqpoint{-8.156111in}{0.773588in}}%
\pgfpathlineto{\pgfqpoint{-8.084913in}{0.773588in}}%
\pgfpathlineto{\pgfqpoint{-8.014387in}{0.773588in}}%
\pgfpathlineto{\pgfqpoint{-7.942200in}{0.773588in}}%
\pgfpathlineto{\pgfqpoint{-7.870788in}{0.773588in}}%
\pgfpathlineto{\pgfqpoint{-7.799875in}{0.773588in}}%
\pgfpathlineto{\pgfqpoint{-7.727286in}{0.773588in}}%
\pgfpathlineto{\pgfqpoint{-7.657024in}{0.773588in}}%
\pgfpathlineto{\pgfqpoint{-7.585515in}{0.773588in}}%
\pgfpathlineto{\pgfqpoint{-7.512981in}{0.773588in}}%
\pgfpathlineto{\pgfqpoint{-7.440452in}{0.773588in}}%
\pgfpathlineto{\pgfqpoint{-7.368338in}{0.773588in}}%
\pgfpathlineto{\pgfqpoint{-7.294929in}{0.773588in}}%
\pgfpathlineto{\pgfqpoint{-7.223147in}{0.773588in}}%
\pgfpathlineto{\pgfqpoint{-7.151909in}{0.773588in}}%
\pgfpathlineto{\pgfqpoint{-7.075897in}{0.773588in}}%
\pgfpathlineto{\pgfqpoint{-7.003598in}{0.773588in}}%
\pgfpathlineto{\pgfqpoint{-6.933558in}{0.773588in}}%
\pgfpathlineto{\pgfqpoint{-6.860576in}{0.773588in}}%
\pgfpathlineto{\pgfqpoint{-6.787495in}{0.773588in}}%
\pgfpathlineto{\pgfqpoint{-6.715245in}{0.773588in}}%
\pgfpathlineto{\pgfqpoint{-6.640496in}{0.773588in}}%
\pgfpathlineto{\pgfqpoint{-6.568712in}{0.773588in}}%
\pgfpathlineto{\pgfqpoint{-6.496669in}{0.773588in}}%
\pgfpathlineto{\pgfqpoint{-6.421127in}{0.773588in}}%
\pgfpathlineto{\pgfqpoint{-6.349020in}{0.773588in}}%
\pgfpathlineto{\pgfqpoint{-6.278887in}{0.773588in}}%
\pgfpathlineto{\pgfqpoint{-6.205113in}{0.773588in}}%
\pgfpathlineto{\pgfqpoint{-6.135189in}{0.773588in}}%
\pgfpathlineto{\pgfqpoint{-6.064664in}{0.773588in}}%
\pgfpathlineto{\pgfqpoint{-5.992127in}{0.773588in}}%
\pgfpathlineto{\pgfqpoint{-5.921215in}{0.773588in}}%
\pgfpathlineto{\pgfqpoint{-5.849895in}{0.773588in}}%
\pgfpathlineto{\pgfqpoint{-5.776542in}{0.773588in}}%
\pgfpathlineto{\pgfqpoint{-5.706570in}{0.773588in}}%
\pgfpathlineto{\pgfqpoint{-5.635685in}{0.773588in}}%
\pgfpathlineto{\pgfqpoint{-5.563648in}{0.773588in}}%
\pgfpathlineto{\pgfqpoint{-5.493615in}{0.773588in}}%
\pgfpathlineto{\pgfqpoint{-5.421890in}{0.773588in}}%
\pgfpathlineto{\pgfqpoint{-5.348665in}{0.773588in}}%
\pgfpathlineto{\pgfqpoint{-5.277970in}{0.773588in}}%
\pgfpathlineto{\pgfqpoint{-5.207317in}{0.773588in}}%
\pgfpathlineto{\pgfqpoint{-5.135005in}{0.773588in}}%
\pgfpathlineto{\pgfqpoint{-5.064817in}{0.773588in}}%
\pgfpathlineto{\pgfqpoint{-4.995002in}{0.773588in}}%
\pgfpathlineto{\pgfqpoint{-4.922664in}{0.773588in}}%
\pgfpathlineto{\pgfqpoint{-4.852652in}{0.773588in}}%
\pgfpathlineto{\pgfqpoint{-4.782965in}{0.773588in}}%
\pgfpathlineto{\pgfqpoint{-4.711002in}{0.773588in}}%
\pgfpathlineto{\pgfqpoint{-4.639570in}{0.773588in}}%
\pgfpathlineto{\pgfqpoint{-4.568595in}{0.773588in}}%
\pgfpathlineto{\pgfqpoint{-4.495401in}{0.773588in}}%
\pgfpathlineto{\pgfqpoint{-4.423289in}{0.773588in}}%
\pgfpathlineto{\pgfqpoint{-4.350759in}{0.773588in}}%
\pgfpathlineto{\pgfqpoint{-4.274748in}{0.773588in}}%
\pgfpathlineto{\pgfqpoint{-4.201246in}{0.773588in}}%
\pgfpathlineto{\pgfqpoint{-4.128352in}{0.773588in}}%
\pgfpathlineto{\pgfqpoint{-4.052173in}{0.773588in}}%
\pgfpathlineto{\pgfqpoint{-3.977308in}{0.773588in}}%
\pgfpathlineto{\pgfqpoint{-3.903164in}{0.773588in}}%
\pgfpathlineto{\pgfqpoint{-3.828230in}{0.773588in}}%
\pgfpathlineto{\pgfqpoint{-3.756893in}{0.773588in}}%
\pgfpathlineto{\pgfqpoint{-3.684119in}{0.773588in}}%
\pgfpathlineto{\pgfqpoint{-3.608325in}{0.773588in}}%
\pgfpathlineto{\pgfqpoint{-3.534002in}{0.773588in}}%
\pgfpathlineto{\pgfqpoint{-3.461876in}{0.773588in}}%
\pgfpathlineto{\pgfqpoint{-3.388885in}{0.773588in}}%
\pgfpathlineto{\pgfqpoint{-3.317677in}{0.773588in}}%
\pgfpathlineto{\pgfqpoint{-3.245632in}{0.773588in}}%
\pgfpathlineto{\pgfqpoint{-3.171735in}{0.773588in}}%
\pgfpathlineto{\pgfqpoint{-3.100293in}{0.773588in}}%
\pgfpathlineto{\pgfqpoint{-3.029702in}{0.773588in}}%
\pgfpathlineto{\pgfqpoint{-2.956196in}{0.773588in}}%
\pgfpathlineto{\pgfqpoint{-2.884033in}{0.773588in}}%
\pgfpathlineto{\pgfqpoint{-2.813631in}{0.773588in}}%
\pgfpathlineto{\pgfqpoint{-2.740325in}{0.773588in}}%
\pgfpathlineto{\pgfqpoint{-2.667020in}{0.773588in}}%
\pgfpathlineto{\pgfqpoint{-2.595405in}{0.773588in}}%
\pgfpathlineto{\pgfqpoint{-2.522349in}{0.773588in}}%
\pgfpathlineto{\pgfqpoint{-2.451489in}{0.773588in}}%
\pgfpathlineto{\pgfqpoint{-2.379035in}{0.773588in}}%
\pgfpathlineto{\pgfqpoint{-2.303263in}{0.773588in}}%
\pgfpathlineto{\pgfqpoint{-2.231477in}{0.773588in}}%
\pgfpathlineto{\pgfqpoint{-2.158577in}{0.773588in}}%
\pgfpathlineto{\pgfqpoint{-2.084338in}{0.773588in}}%
\pgfpathlineto{\pgfqpoint{-2.011817in}{0.773588in}}%
\pgfpathlineto{\pgfqpoint{-1.940536in}{0.773588in}}%
\pgfpathlineto{\pgfqpoint{-1.867809in}{0.773588in}}%
\pgfpathlineto{\pgfqpoint{-1.797123in}{0.773588in}}%
\pgfpathlineto{\pgfqpoint{-1.725105in}{0.773588in}}%
\pgfpathlineto{\pgfqpoint{-1.650581in}{0.773588in}}%
\pgfpathlineto{\pgfqpoint{-1.577590in}{0.773588in}}%
\pgfpathlineto{\pgfqpoint{-1.503984in}{0.773588in}}%
\pgfpathlineto{\pgfqpoint{-1.429441in}{0.773588in}}%
\pgfpathlineto{\pgfqpoint{-1.356794in}{0.773588in}}%
\pgfpathlineto{\pgfqpoint{-1.283200in}{0.773588in}}%
\pgfpathlineto{\pgfqpoint{-1.207908in}{0.773588in}}%
\pgfpathlineto{\pgfqpoint{-1.135138in}{0.773588in}}%
\pgfpathlineto{\pgfqpoint{-1.063245in}{0.773588in}}%
\pgfpathlineto{\pgfqpoint{-0.989522in}{0.773588in}}%
\pgfpathlineto{\pgfqpoint{-0.918138in}{0.773588in}}%
\pgfpathlineto{\pgfqpoint{-0.844408in}{0.773588in}}%
\pgfpathlineto{\pgfqpoint{-0.769987in}{0.773588in}}%
\pgfpathlineto{\pgfqpoint{-0.698122in}{0.773588in}}%
\pgfpathlineto{\pgfqpoint{-0.623932in}{0.773588in}}%
\pgfpathlineto{\pgfqpoint{-0.549722in}{0.773588in}}%
\pgfpathlineto{\pgfqpoint{-0.477784in}{0.773588in}}%
\pgfpathlineto{\pgfqpoint{-0.404497in}{0.773588in}}%
\pgfpathlineto{\pgfqpoint{-0.329323in}{0.773588in}}%
\pgfpathlineto{\pgfqpoint{-0.257459in}{0.773588in}}%
\pgfpathlineto{\pgfqpoint{-0.185623in}{0.773588in}}%
\pgfpathlineto{\pgfqpoint{-0.112821in}{0.773588in}}%
\pgfpathlineto{\pgfqpoint{-0.042097in}{0.773588in}}%
\pgfpathlineto{\pgfqpoint{0.030045in}{0.773588in}}%
\pgfpathlineto{\pgfqpoint{0.103944in}{0.773588in}}%
\pgfpathlineto{\pgfqpoint{0.176284in}{0.773588in}}%
\pgfpathlineto{\pgfqpoint{0.248641in}{0.773588in}}%
\pgfpathlineto{\pgfqpoint{0.321303in}{0.773588in}}%
\pgfpathlineto{\pgfqpoint{0.391498in}{0.773588in}}%
\pgfpathlineto{\pgfqpoint{0.462885in}{0.773588in}}%
\pgfpathlineto{\pgfqpoint{0.537447in}{0.773588in}}%
\pgfpathlineto{\pgfqpoint{0.609267in}{0.773588in}}%
\pgfpathlineto{\pgfqpoint{0.679536in}{0.773588in}}%
\pgfpathlineto{\pgfqpoint{0.753203in}{0.773588in}}%
\pgfpathlineto{\pgfqpoint{0.824780in}{0.773588in}}%
\pgfpathlineto{\pgfqpoint{0.895203in}{0.773588in}}%
\pgfpathlineto{\pgfqpoint{0.968165in}{0.773588in}}%
\pgfpathlineto{\pgfqpoint{1.038437in}{0.773588in}}%
\pgfpathlineto{\pgfqpoint{1.111472in}{0.773588in}}%
\pgfpathlineto{\pgfqpoint{1.188027in}{0.773588in}}%
\pgfpathlineto{\pgfqpoint{1.261309in}{0.773588in}}%
\pgfpathlineto{\pgfqpoint{1.335256in}{0.773588in}}%
\pgfpathlineto{\pgfqpoint{1.410931in}{0.773588in}}%
\pgfpathlineto{\pgfqpoint{1.485557in}{0.773588in}}%
\pgfpathlineto{\pgfqpoint{1.558242in}{0.773588in}}%
\pgfpathlineto{\pgfqpoint{1.633550in}{0.773588in}}%
\pgfpathlineto{\pgfqpoint{1.707271in}{0.773588in}}%
\pgfpathlineto{\pgfqpoint{1.781074in}{0.773588in}}%
\pgfpathlineto{\pgfqpoint{1.857049in}{0.773588in}}%
\pgfpathlineto{\pgfqpoint{1.931091in}{0.773588in}}%
\pgfpathlineto{\pgfqpoint{2.004776in}{0.773588in}}%
\pgfpathlineto{\pgfqpoint{2.079647in}{0.773588in}}%
\pgfpathlineto{\pgfqpoint{2.151611in}{0.773588in}}%
\pgfpathlineto{\pgfqpoint{2.224651in}{0.773588in}}%
\pgfpathlineto{\pgfqpoint{2.300156in}{0.773588in}}%
\pgfpathlineto{\pgfqpoint{2.371299in}{0.773588in}}%
\pgfpathlineto{\pgfqpoint{2.442412in}{0.773588in}}%
\pgfpathlineto{\pgfqpoint{2.515763in}{0.773588in}}%
\pgfpathlineto{\pgfqpoint{2.586109in}{0.773588in}}%
\pgfpathlineto{\pgfqpoint{2.658524in}{0.773588in}}%
\pgfpathlineto{\pgfqpoint{2.732102in}{0.773588in}}%
\pgfpathlineto{\pgfqpoint{2.802412in}{0.773588in}}%
\pgfpathlineto{\pgfqpoint{2.873367in}{0.773588in}}%
\pgfpathlineto{\pgfqpoint{2.946925in}{0.773588in}}%
\pgfpathlineto{\pgfqpoint{3.019212in}{0.773588in}}%
\pgfpathlineto{\pgfqpoint{3.091740in}{0.773588in}}%
\pgfpathlineto{\pgfqpoint{3.166494in}{0.773588in}}%
\pgfpathlineto{\pgfqpoint{3.237461in}{0.773588in}}%
\pgfpathlineto{\pgfqpoint{3.309976in}{0.773588in}}%
\pgfpathlineto{\pgfqpoint{3.384706in}{0.773588in}}%
\pgfpathlineto{\pgfqpoint{3.455601in}{0.773588in}}%
\pgfpathlineto{\pgfqpoint{3.527918in}{0.773588in}}%
\pgfpathlineto{\pgfqpoint{3.603230in}{0.773588in}}%
\pgfpathlineto{\pgfqpoint{3.674965in}{0.773588in}}%
\pgfpathlineto{\pgfqpoint{3.745938in}{0.773588in}}%
\pgfpathlineto{\pgfqpoint{3.820753in}{0.773588in}}%
\pgfpathlineto{\pgfqpoint{3.899537in}{0.773588in}}%
\pgfpathlineto{\pgfqpoint{4.022313in}{0.773588in}}%
\pgfpathlineto{\pgfqpoint{4.111286in}{0.773588in}}%
\pgfpathlineto{\pgfqpoint{4.189282in}{0.773588in}}%
\pgfpathlineto{\pgfqpoint{4.253332in}{1.591997in}}%
\pgfpathlineto{\pgfqpoint{4.318577in}{3.700065in}}%
\pgfpathlineto{\pgfqpoint{4.390596in}{3.811738in}}%
\pgfpathlineto{\pgfqpoint{4.460559in}{3.977733in}}%
\pgfpathlineto{\pgfqpoint{4.532858in}{3.810697in}}%
\pgfpathlineto{\pgfqpoint{4.602459in}{3.912811in}}%
\pgfpathlineto{\pgfqpoint{4.671388in}{3.948574in}}%
\pgfpathlineto{\pgfqpoint{4.741243in}{3.999837in}}%
\pgfpathlineto{\pgfqpoint{4.808397in}{4.001718in}}%
\pgfpathlineto{\pgfqpoint{4.875969in}{4.007261in}}%
\pgfpathlineto{\pgfqpoint{4.944850in}{3.951551in}}%
\pgfpathlineto{\pgfqpoint{5.010860in}{4.110835in}}%
\pgfpathlineto{\pgfqpoint{5.076713in}{4.085256in}}%
\pgfpathlineto{\pgfqpoint{5.144715in}{4.115543in}}%
\pgfpathlineto{\pgfqpoint{5.209656in}{4.151117in}}%
\pgfpathlineto{\pgfqpoint{5.275100in}{4.095111in}}%
\pgfpathlineto{\pgfqpoint{5.341477in}{4.215547in}}%
\pgfpathlineto{\pgfqpoint{5.405234in}{4.207643in}}%
\pgfpathlineto{\pgfqpoint{5.469981in}{4.115512in}}%
\pgfpathlineto{\pgfqpoint{5.535779in}{4.257509in}}%
\pgfpathlineto{\pgfqpoint{5.599590in}{4.172457in}}%
\pgfpathlineto{\pgfqpoint{5.599590in}{5.872359in}}%
\pgfpathlineto{\pgfqpoint{5.599590in}{5.872359in}}%
\pgfpathlineto{\pgfqpoint{5.535779in}{5.930845in}}%
\pgfpathlineto{\pgfqpoint{5.469981in}{5.814953in}}%
\pgfpathlineto{\pgfqpoint{5.405234in}{5.879596in}}%
\pgfpathlineto{\pgfqpoint{5.341477in}{5.875626in}}%
\pgfpathlineto{\pgfqpoint{5.275100in}{5.772285in}}%
\pgfpathlineto{\pgfqpoint{5.209656in}{5.793255in}}%
\pgfpathlineto{\pgfqpoint{5.144715in}{5.790749in}}%
\pgfpathlineto{\pgfqpoint{5.076713in}{5.732424in}}%
\pgfpathlineto{\pgfqpoint{5.010860in}{5.748599in}}%
\pgfpathlineto{\pgfqpoint{4.944850in}{5.625548in}}%
\pgfpathlineto{\pgfqpoint{4.875969in}{5.661488in}}%
\pgfpathlineto{\pgfqpoint{4.808397in}{5.619869in}}%
\pgfpathlineto{\pgfqpoint{4.741243in}{5.638191in}}%
\pgfpathlineto{\pgfqpoint{4.671388in}{5.539128in}}%
\pgfpathlineto{\pgfqpoint{4.602459in}{5.488048in}}%
\pgfpathlineto{\pgfqpoint{4.532858in}{5.399336in}}%
\pgfpathlineto{\pgfqpoint{4.460559in}{5.506716in}}%
\pgfpathlineto{\pgfqpoint{4.390596in}{5.319244in}}%
\pgfpathlineto{\pgfqpoint{4.318577in}{5.187514in}}%
\pgfpathlineto{\pgfqpoint{4.253332in}{1.724813in}}%
\pgfpathlineto{\pgfqpoint{4.189282in}{0.773588in}}%
\pgfpathlineto{\pgfqpoint{4.111286in}{0.773588in}}%
\pgfpathlineto{\pgfqpoint{4.022313in}{0.773588in}}%
\pgfpathlineto{\pgfqpoint{3.899537in}{0.773588in}}%
\pgfpathlineto{\pgfqpoint{3.820753in}{0.773588in}}%
\pgfpathlineto{\pgfqpoint{3.745938in}{0.773588in}}%
\pgfpathlineto{\pgfqpoint{3.674965in}{0.773588in}}%
\pgfpathlineto{\pgfqpoint{3.603230in}{0.773588in}}%
\pgfpathlineto{\pgfqpoint{3.527918in}{0.773588in}}%
\pgfpathlineto{\pgfqpoint{3.455601in}{0.773588in}}%
\pgfpathlineto{\pgfqpoint{3.384706in}{0.773588in}}%
\pgfpathlineto{\pgfqpoint{3.309976in}{0.773588in}}%
\pgfpathlineto{\pgfqpoint{3.237461in}{0.773588in}}%
\pgfpathlineto{\pgfqpoint{3.166494in}{0.773588in}}%
\pgfpathlineto{\pgfqpoint{3.091740in}{0.773588in}}%
\pgfpathlineto{\pgfqpoint{3.019212in}{0.773588in}}%
\pgfpathlineto{\pgfqpoint{2.946925in}{0.773588in}}%
\pgfpathlineto{\pgfqpoint{2.873367in}{0.773588in}}%
\pgfpathlineto{\pgfqpoint{2.802412in}{0.773588in}}%
\pgfpathlineto{\pgfqpoint{2.732102in}{0.773588in}}%
\pgfpathlineto{\pgfqpoint{2.658524in}{0.773588in}}%
\pgfpathlineto{\pgfqpoint{2.586109in}{0.773588in}}%
\pgfpathlineto{\pgfqpoint{2.515763in}{0.773588in}}%
\pgfpathlineto{\pgfqpoint{2.442412in}{0.773588in}}%
\pgfpathlineto{\pgfqpoint{2.371299in}{0.773588in}}%
\pgfpathlineto{\pgfqpoint{2.300156in}{0.773588in}}%
\pgfpathlineto{\pgfqpoint{2.224651in}{0.773588in}}%
\pgfpathlineto{\pgfqpoint{2.151611in}{0.773588in}}%
\pgfpathlineto{\pgfqpoint{2.079647in}{0.773588in}}%
\pgfpathlineto{\pgfqpoint{2.004776in}{0.773588in}}%
\pgfpathlineto{\pgfqpoint{1.931091in}{0.773588in}}%
\pgfpathlineto{\pgfqpoint{1.857049in}{0.773588in}}%
\pgfpathlineto{\pgfqpoint{1.781074in}{0.773588in}}%
\pgfpathlineto{\pgfqpoint{1.707271in}{0.773588in}}%
\pgfpathlineto{\pgfqpoint{1.633550in}{0.773588in}}%
\pgfpathlineto{\pgfqpoint{1.558242in}{0.773588in}}%
\pgfpathlineto{\pgfqpoint{1.485557in}{0.773588in}}%
\pgfpathlineto{\pgfqpoint{1.410931in}{0.773588in}}%
\pgfpathlineto{\pgfqpoint{1.335256in}{0.773588in}}%
\pgfpathlineto{\pgfqpoint{1.261309in}{0.773588in}}%
\pgfpathlineto{\pgfqpoint{1.188027in}{0.773588in}}%
\pgfpathlineto{\pgfqpoint{1.111472in}{0.773588in}}%
\pgfpathlineto{\pgfqpoint{1.038437in}{0.773588in}}%
\pgfpathlineto{\pgfqpoint{0.968165in}{0.773588in}}%
\pgfpathlineto{\pgfqpoint{0.895203in}{0.773588in}}%
\pgfpathlineto{\pgfqpoint{0.824780in}{0.773588in}}%
\pgfpathlineto{\pgfqpoint{0.753203in}{0.773588in}}%
\pgfpathlineto{\pgfqpoint{0.679536in}{0.773588in}}%
\pgfpathlineto{\pgfqpoint{0.609267in}{0.773588in}}%
\pgfpathlineto{\pgfqpoint{0.537447in}{0.773588in}}%
\pgfpathlineto{\pgfqpoint{0.462885in}{0.773588in}}%
\pgfpathlineto{\pgfqpoint{0.391498in}{0.773588in}}%
\pgfpathlineto{\pgfqpoint{0.321303in}{0.773588in}}%
\pgfpathlineto{\pgfqpoint{0.248641in}{0.773588in}}%
\pgfpathlineto{\pgfqpoint{0.176284in}{0.773588in}}%
\pgfpathlineto{\pgfqpoint{0.103944in}{0.773588in}}%
\pgfpathlineto{\pgfqpoint{0.030045in}{0.773588in}}%
\pgfpathlineto{\pgfqpoint{-0.042097in}{0.773588in}}%
\pgfpathlineto{\pgfqpoint{-0.112821in}{0.773588in}}%
\pgfpathlineto{\pgfqpoint{-0.185623in}{0.773588in}}%
\pgfpathlineto{\pgfqpoint{-0.257459in}{0.773588in}}%
\pgfpathlineto{\pgfqpoint{-0.329323in}{0.773588in}}%
\pgfpathlineto{\pgfqpoint{-0.404497in}{0.773588in}}%
\pgfpathlineto{\pgfqpoint{-0.477784in}{0.773588in}}%
\pgfpathlineto{\pgfqpoint{-0.549722in}{0.773588in}}%
\pgfpathlineto{\pgfqpoint{-0.623932in}{0.773588in}}%
\pgfpathlineto{\pgfqpoint{-0.698122in}{0.773588in}}%
\pgfpathlineto{\pgfqpoint{-0.769987in}{0.773588in}}%
\pgfpathlineto{\pgfqpoint{-0.844408in}{0.773588in}}%
\pgfpathlineto{\pgfqpoint{-0.918138in}{0.773588in}}%
\pgfpathlineto{\pgfqpoint{-0.989522in}{0.773588in}}%
\pgfpathlineto{\pgfqpoint{-1.063245in}{0.773588in}}%
\pgfpathlineto{\pgfqpoint{-1.135138in}{0.773588in}}%
\pgfpathlineto{\pgfqpoint{-1.207908in}{0.773588in}}%
\pgfpathlineto{\pgfqpoint{-1.283200in}{0.773588in}}%
\pgfpathlineto{\pgfqpoint{-1.356794in}{0.773588in}}%
\pgfpathlineto{\pgfqpoint{-1.429441in}{0.773588in}}%
\pgfpathlineto{\pgfqpoint{-1.503984in}{0.773588in}}%
\pgfpathlineto{\pgfqpoint{-1.577590in}{0.773588in}}%
\pgfpathlineto{\pgfqpoint{-1.650581in}{0.773588in}}%
\pgfpathlineto{\pgfqpoint{-1.725105in}{0.773588in}}%
\pgfpathlineto{\pgfqpoint{-1.797123in}{0.773588in}}%
\pgfpathlineto{\pgfqpoint{-1.867809in}{0.773588in}}%
\pgfpathlineto{\pgfqpoint{-1.940536in}{0.773588in}}%
\pgfpathlineto{\pgfqpoint{-2.011817in}{0.773588in}}%
\pgfpathlineto{\pgfqpoint{-2.084338in}{0.773588in}}%
\pgfpathlineto{\pgfqpoint{-2.158577in}{0.773588in}}%
\pgfpathlineto{\pgfqpoint{-2.231477in}{0.773588in}}%
\pgfpathlineto{\pgfqpoint{-2.303263in}{0.773588in}}%
\pgfpathlineto{\pgfqpoint{-2.379035in}{0.773588in}}%
\pgfpathlineto{\pgfqpoint{-2.451489in}{0.773588in}}%
\pgfpathlineto{\pgfqpoint{-2.522349in}{0.773588in}}%
\pgfpathlineto{\pgfqpoint{-2.595405in}{0.773588in}}%
\pgfpathlineto{\pgfqpoint{-2.667020in}{0.773588in}}%
\pgfpathlineto{\pgfqpoint{-2.740325in}{0.773588in}}%
\pgfpathlineto{\pgfqpoint{-2.813631in}{0.773588in}}%
\pgfpathlineto{\pgfqpoint{-2.884033in}{0.773588in}}%
\pgfpathlineto{\pgfqpoint{-2.956196in}{0.773588in}}%
\pgfpathlineto{\pgfqpoint{-3.029702in}{0.773588in}}%
\pgfpathlineto{\pgfqpoint{-3.100293in}{0.773588in}}%
\pgfpathlineto{\pgfqpoint{-3.171735in}{0.773588in}}%
\pgfpathlineto{\pgfqpoint{-3.245632in}{0.773588in}}%
\pgfpathlineto{\pgfqpoint{-3.317677in}{0.773588in}}%
\pgfpathlineto{\pgfqpoint{-3.388885in}{0.773588in}}%
\pgfpathlineto{\pgfqpoint{-3.461876in}{0.773588in}}%
\pgfpathlineto{\pgfqpoint{-3.534002in}{0.773588in}}%
\pgfpathlineto{\pgfqpoint{-3.608325in}{0.773588in}}%
\pgfpathlineto{\pgfqpoint{-3.684119in}{0.773588in}}%
\pgfpathlineto{\pgfqpoint{-3.756893in}{0.773588in}}%
\pgfpathlineto{\pgfqpoint{-3.828230in}{0.773588in}}%
\pgfpathlineto{\pgfqpoint{-3.903164in}{0.773588in}}%
\pgfpathlineto{\pgfqpoint{-3.977308in}{0.773588in}}%
\pgfpathlineto{\pgfqpoint{-4.052173in}{0.773588in}}%
\pgfpathlineto{\pgfqpoint{-4.128352in}{0.773588in}}%
\pgfpathlineto{\pgfqpoint{-4.201246in}{0.773588in}}%
\pgfpathlineto{\pgfqpoint{-4.274748in}{0.773588in}}%
\pgfpathlineto{\pgfqpoint{-4.350759in}{0.773588in}}%
\pgfpathlineto{\pgfqpoint{-4.423289in}{0.773588in}}%
\pgfpathlineto{\pgfqpoint{-4.495401in}{0.773588in}}%
\pgfpathlineto{\pgfqpoint{-4.568595in}{0.773588in}}%
\pgfpathlineto{\pgfqpoint{-4.639570in}{0.773588in}}%
\pgfpathlineto{\pgfqpoint{-4.711002in}{0.773588in}}%
\pgfpathlineto{\pgfqpoint{-4.782965in}{0.773588in}}%
\pgfpathlineto{\pgfqpoint{-4.852652in}{0.773588in}}%
\pgfpathlineto{\pgfqpoint{-4.922664in}{0.773588in}}%
\pgfpathlineto{\pgfqpoint{-4.995002in}{0.773588in}}%
\pgfpathlineto{\pgfqpoint{-5.064817in}{0.773588in}}%
\pgfpathlineto{\pgfqpoint{-5.135005in}{0.773588in}}%
\pgfpathlineto{\pgfqpoint{-5.207317in}{0.773588in}}%
\pgfpathlineto{\pgfqpoint{-5.277970in}{0.773588in}}%
\pgfpathlineto{\pgfqpoint{-5.348665in}{0.773588in}}%
\pgfpathlineto{\pgfqpoint{-5.421890in}{0.773588in}}%
\pgfpathlineto{\pgfqpoint{-5.493615in}{0.773588in}}%
\pgfpathlineto{\pgfqpoint{-5.563648in}{0.773588in}}%
\pgfpathlineto{\pgfqpoint{-5.635685in}{0.773588in}}%
\pgfpathlineto{\pgfqpoint{-5.706570in}{0.773588in}}%
\pgfpathlineto{\pgfqpoint{-5.776542in}{0.773588in}}%
\pgfpathlineto{\pgfqpoint{-5.849895in}{0.773588in}}%
\pgfpathlineto{\pgfqpoint{-5.921215in}{0.773588in}}%
\pgfpathlineto{\pgfqpoint{-5.992127in}{0.773588in}}%
\pgfpathlineto{\pgfqpoint{-6.064664in}{0.773588in}}%
\pgfpathlineto{\pgfqpoint{-6.135189in}{0.773588in}}%
\pgfpathlineto{\pgfqpoint{-6.205113in}{0.773588in}}%
\pgfpathlineto{\pgfqpoint{-6.278887in}{0.773588in}}%
\pgfpathlineto{\pgfqpoint{-6.349020in}{0.773588in}}%
\pgfpathlineto{\pgfqpoint{-6.421127in}{0.773588in}}%
\pgfpathlineto{\pgfqpoint{-6.496669in}{0.773588in}}%
\pgfpathlineto{\pgfqpoint{-6.568712in}{0.773588in}}%
\pgfpathlineto{\pgfqpoint{-6.640496in}{0.773588in}}%
\pgfpathlineto{\pgfqpoint{-6.715245in}{0.773588in}}%
\pgfpathlineto{\pgfqpoint{-6.787495in}{0.773588in}}%
\pgfpathlineto{\pgfqpoint{-6.860576in}{0.773588in}}%
\pgfpathlineto{\pgfqpoint{-6.933558in}{0.773588in}}%
\pgfpathlineto{\pgfqpoint{-7.003598in}{0.773588in}}%
\pgfpathlineto{\pgfqpoint{-7.075897in}{0.773588in}}%
\pgfpathlineto{\pgfqpoint{-7.151909in}{0.773588in}}%
\pgfpathlineto{\pgfqpoint{-7.223147in}{0.773588in}}%
\pgfpathlineto{\pgfqpoint{-7.294929in}{0.773588in}}%
\pgfpathlineto{\pgfqpoint{-7.368338in}{0.773588in}}%
\pgfpathlineto{\pgfqpoint{-7.440452in}{0.773588in}}%
\pgfpathlineto{\pgfqpoint{-7.512981in}{0.773588in}}%
\pgfpathlineto{\pgfqpoint{-7.585515in}{0.773588in}}%
\pgfpathlineto{\pgfqpoint{-7.657024in}{0.773588in}}%
\pgfpathlineto{\pgfqpoint{-7.727286in}{0.773588in}}%
\pgfpathlineto{\pgfqpoint{-7.799875in}{0.773588in}}%
\pgfpathlineto{\pgfqpoint{-7.870788in}{0.773588in}}%
\pgfpathlineto{\pgfqpoint{-7.942200in}{0.773588in}}%
\pgfpathlineto{\pgfqpoint{-8.014387in}{0.773588in}}%
\pgfpathlineto{\pgfqpoint{-8.084913in}{0.773588in}}%
\pgfpathlineto{\pgfqpoint{-8.156111in}{0.773588in}}%
\pgfpathlineto{\pgfqpoint{-8.228241in}{0.773588in}}%
\pgfpathlineto{\pgfqpoint{-8.298753in}{0.773588in}}%
\pgfpathlineto{\pgfqpoint{-8.367967in}{0.773588in}}%
\pgfpathlineto{\pgfqpoint{-8.438970in}{0.773588in}}%
\pgfpathlineto{\pgfqpoint{-8.508839in}{0.773588in}}%
\pgfpathlineto{\pgfqpoint{-8.578369in}{0.773588in}}%
\pgfpathlineto{\pgfqpoint{-8.649341in}{0.773588in}}%
\pgfpathlineto{\pgfqpoint{-8.718609in}{0.773588in}}%
\pgfpathlineto{\pgfqpoint{-8.787586in}{0.773588in}}%
\pgfpathlineto{\pgfqpoint{-8.858221in}{0.773588in}}%
\pgfpathlineto{\pgfqpoint{-8.928123in}{0.773588in}}%
\pgfpathlineto{\pgfqpoint{-8.998701in}{0.773588in}}%
\pgfpathlineto{\pgfqpoint{-9.069721in}{0.773588in}}%
\pgfpathlineto{\pgfqpoint{-9.138788in}{0.773588in}}%
\pgfpathlineto{\pgfqpoint{-9.208906in}{0.773588in}}%
\pgfpathlineto{\pgfqpoint{-9.283412in}{0.773588in}}%
\pgfpathlineto{\pgfqpoint{-9.356529in}{0.773588in}}%
\pgfpathlineto{\pgfqpoint{-9.429774in}{0.773588in}}%
\pgfpathlineto{\pgfqpoint{-9.504196in}{0.773588in}}%
\pgfpathlineto{\pgfqpoint{-9.575400in}{0.773588in}}%
\pgfpathlineto{\pgfqpoint{-9.648369in}{0.773588in}}%
\pgfpathlineto{\pgfqpoint{-9.723682in}{0.773588in}}%
\pgfpathlineto{\pgfqpoint{-9.796010in}{0.773588in}}%
\pgfpathlineto{\pgfqpoint{-9.868243in}{0.773588in}}%
\pgfpathlineto{\pgfqpoint{-9.941359in}{0.773588in}}%
\pgfpathlineto{\pgfqpoint{-10.012635in}{0.773588in}}%
\pgfpathlineto{\pgfqpoint{-10.084496in}{0.773588in}}%
\pgfpathlineto{\pgfqpoint{-10.157398in}{0.773588in}}%
\pgfpathlineto{\pgfqpoint{-10.227739in}{0.773588in}}%
\pgfpathlineto{\pgfqpoint{-10.297591in}{0.773588in}}%
\pgfpathlineto{\pgfqpoint{-10.371004in}{0.773588in}}%
\pgfpathlineto{\pgfqpoint{-10.441581in}{0.773588in}}%
\pgfpathlineto{\pgfqpoint{-10.511372in}{0.773588in}}%
\pgfpathlineto{\pgfqpoint{-10.582864in}{0.773588in}}%
\pgfpathlineto{\pgfqpoint{-10.652649in}{0.773588in}}%
\pgfpathlineto{\pgfqpoint{-10.721651in}{0.773588in}}%
\pgfpathlineto{\pgfqpoint{-10.793422in}{0.773588in}}%
\pgfpathlineto{\pgfqpoint{-10.863257in}{0.773588in}}%
\pgfpathlineto{\pgfqpoint{-10.933808in}{0.773588in}}%
\pgfpathlineto{\pgfqpoint{-11.005065in}{0.773588in}}%
\pgfpathlineto{\pgfqpoint{-11.075276in}{0.773588in}}%
\pgfpathlineto{\pgfqpoint{-11.144960in}{0.773588in}}%
\pgfpathlineto{\pgfqpoint{-11.217631in}{0.773588in}}%
\pgfpathlineto{\pgfqpoint{-11.288273in}{0.773588in}}%
\pgfpathlineto{\pgfqpoint{-11.358676in}{0.773588in}}%
\pgfpathlineto{\pgfqpoint{-11.430351in}{0.773588in}}%
\pgfpathlineto{\pgfqpoint{-11.500166in}{0.773588in}}%
\pgfpathlineto{\pgfqpoint{-11.571037in}{0.773588in}}%
\pgfpathlineto{\pgfqpoint{-11.645187in}{0.773588in}}%
\pgfpathlineto{\pgfqpoint{-11.716309in}{0.773588in}}%
\pgfpathlineto{\pgfqpoint{-11.786176in}{0.773588in}}%
\pgfpathlineto{\pgfqpoint{-11.858471in}{0.773588in}}%
\pgfpathlineto{\pgfqpoint{-11.928381in}{0.773588in}}%
\pgfpathlineto{\pgfqpoint{-11.998596in}{0.773588in}}%
\pgfpathlineto{\pgfqpoint{-12.069446in}{0.773588in}}%
\pgfpathlineto{\pgfqpoint{-12.139885in}{0.773588in}}%
\pgfpathlineto{\pgfqpoint{-12.211536in}{0.773588in}}%
\pgfpathlineto{\pgfqpoint{-12.285275in}{0.773588in}}%
\pgfpathlineto{\pgfqpoint{-12.357103in}{0.773588in}}%
\pgfpathlineto{\pgfqpoint{-12.428997in}{0.773588in}}%
\pgfpathlineto{\pgfqpoint{-12.503401in}{0.773588in}}%
\pgfpathlineto{\pgfqpoint{-12.574652in}{0.773588in}}%
\pgfpathlineto{\pgfqpoint{-12.645279in}{0.773588in}}%
\pgfpathlineto{\pgfqpoint{-12.718096in}{0.773588in}}%
\pgfpathlineto{\pgfqpoint{-12.788679in}{0.773588in}}%
\pgfpathlineto{\pgfqpoint{-12.859398in}{0.773588in}}%
\pgfpathlineto{\pgfqpoint{-12.932388in}{0.773588in}}%
\pgfpathlineto{\pgfqpoint{-13.002727in}{0.773588in}}%
\pgfpathlineto{\pgfqpoint{-13.073093in}{0.773588in}}%
\pgfpathlineto{\pgfqpoint{-13.145337in}{0.773588in}}%
\pgfpathlineto{\pgfqpoint{-13.216013in}{0.773588in}}%
\pgfpathlineto{\pgfqpoint{-13.285456in}{0.773588in}}%
\pgfpathlineto{\pgfqpoint{-13.356968in}{0.773588in}}%
\pgfpathlineto{\pgfqpoint{-13.426276in}{0.773588in}}%
\pgfpathlineto{\pgfqpoint{-13.496151in}{0.773588in}}%
\pgfpathlineto{\pgfqpoint{-13.568713in}{0.773588in}}%
\pgfpathlineto{\pgfqpoint{-13.639609in}{0.773588in}}%
\pgfpathlineto{\pgfqpoint{-13.708089in}{0.773588in}}%
\pgfpathlineto{\pgfqpoint{-13.779339in}{0.773588in}}%
\pgfpathlineto{\pgfqpoint{-13.848756in}{0.773588in}}%
\pgfpathlineto{\pgfqpoint{-13.917455in}{0.773588in}}%
\pgfpathlineto{\pgfqpoint{-13.988983in}{0.773588in}}%
\pgfpathlineto{\pgfqpoint{-14.057377in}{0.773588in}}%
\pgfpathlineto{\pgfqpoint{-14.125666in}{0.773588in}}%
\pgfpathlineto{\pgfqpoint{-14.195714in}{0.773588in}}%
\pgfpathlineto{\pgfqpoint{-14.263436in}{0.773588in}}%
\pgfpathlineto{\pgfqpoint{-14.331277in}{0.773588in}}%
\pgfpathlineto{\pgfqpoint{-14.401957in}{0.773588in}}%
\pgfpathlineto{\pgfqpoint{-14.471123in}{0.773588in}}%
\pgfpathlineto{\pgfqpoint{-14.538988in}{0.773588in}}%
\pgfpathlineto{\pgfqpoint{-14.609938in}{0.773588in}}%
\pgfpathlineto{\pgfqpoint{-14.679263in}{0.773588in}}%
\pgfpathlineto{\pgfqpoint{-14.748835in}{0.773588in}}%
\pgfpathlineto{\pgfqpoint{-14.820408in}{0.773588in}}%
\pgfpathlineto{\pgfqpoint{-14.889438in}{0.773588in}}%
\pgfpathlineto{\pgfqpoint{-14.959865in}{0.773588in}}%
\pgfpathlineto{\pgfqpoint{-15.032843in}{0.773588in}}%
\pgfpathlineto{\pgfqpoint{-15.106256in}{0.773588in}}%
\pgfpathlineto{\pgfqpoint{-15.179182in}{0.773588in}}%
\pgfpathlineto{\pgfqpoint{-15.253001in}{0.773588in}}%
\pgfpathlineto{\pgfqpoint{-15.323000in}{0.773588in}}%
\pgfpathlineto{\pgfqpoint{-15.392048in}{0.773588in}}%
\pgfpathlineto{\pgfqpoint{-15.463558in}{0.773588in}}%
\pgfpathlineto{\pgfqpoint{-15.532639in}{0.773588in}}%
\pgfpathlineto{\pgfqpoint{-15.602559in}{0.773588in}}%
\pgfpathlineto{\pgfqpoint{-15.673559in}{0.773588in}}%
\pgfpathlineto{\pgfqpoint{-15.744769in}{0.773588in}}%
\pgfpathlineto{\pgfqpoint{-15.814335in}{0.773588in}}%
\pgfpathlineto{\pgfqpoint{-15.886757in}{0.773588in}}%
\pgfpathlineto{\pgfqpoint{-15.956358in}{0.773588in}}%
\pgfpathlineto{\pgfqpoint{-16.024276in}{0.773588in}}%
\pgfpathlineto{\pgfqpoint{-16.094895in}{0.773588in}}%
\pgfpathlineto{\pgfqpoint{-16.163654in}{0.773588in}}%
\pgfpathlineto{\pgfqpoint{-16.232068in}{0.773588in}}%
\pgfpathlineto{\pgfqpoint{-16.302639in}{0.773588in}}%
\pgfpathlineto{\pgfqpoint{-16.369724in}{0.773588in}}%
\pgfpathlineto{\pgfqpoint{-16.437097in}{0.773588in}}%
\pgfpathlineto{\pgfqpoint{-16.507797in}{0.773588in}}%
\pgfpathlineto{\pgfqpoint{-16.575864in}{0.773588in}}%
\pgfpathlineto{\pgfqpoint{-16.644038in}{0.773588in}}%
\pgfpathlineto{\pgfqpoint{-16.715630in}{0.773588in}}%
\pgfpathlineto{\pgfqpoint{-16.784012in}{0.773588in}}%
\pgfpathlineto{\pgfqpoint{-16.852234in}{0.773588in}}%
\pgfpathlineto{\pgfqpoint{-16.921714in}{0.773588in}}%
\pgfpathlineto{\pgfqpoint{-16.989953in}{0.773588in}}%
\pgfpathlineto{\pgfqpoint{-17.058127in}{0.773588in}}%
\pgfpathlineto{\pgfqpoint{-17.128098in}{0.773588in}}%
\pgfpathlineto{\pgfqpoint{-17.196258in}{0.773588in}}%
\pgfpathlineto{\pgfqpoint{-17.265552in}{0.773588in}}%
\pgfpathlineto{\pgfqpoint{-17.336134in}{0.773588in}}%
\pgfpathlineto{\pgfqpoint{-17.402675in}{0.773588in}}%
\pgfpathlineto{\pgfqpoint{-17.470673in}{0.773588in}}%
\pgfpathlineto{\pgfqpoint{-17.539598in}{0.773588in}}%
\pgfpathlineto{\pgfqpoint{-17.607453in}{0.773588in}}%
\pgfpathlineto{\pgfqpoint{-17.675451in}{0.773588in}}%
\pgfpathlineto{\pgfqpoint{-17.748008in}{0.773588in}}%
\pgfpathlineto{\pgfqpoint{-17.816737in}{0.773588in}}%
\pgfpathlineto{\pgfqpoint{-17.885356in}{0.773588in}}%
\pgfpathlineto{\pgfqpoint{-17.957185in}{0.773588in}}%
\pgfpathlineto{\pgfqpoint{-18.025876in}{0.773588in}}%
\pgfpathlineto{\pgfqpoint{-18.094404in}{0.773588in}}%
\pgfpathlineto{\pgfqpoint{-18.165145in}{0.773588in}}%
\pgfpathlineto{\pgfqpoint{-18.232823in}{0.773588in}}%
\pgfpathlineto{\pgfqpoint{-18.301710in}{0.773588in}}%
\pgfpathlineto{\pgfqpoint{-18.373957in}{0.773588in}}%
\pgfpathlineto{\pgfqpoint{-18.442527in}{0.773588in}}%
\pgfpathlineto{\pgfqpoint{-18.511159in}{0.773588in}}%
\pgfpathlineto{\pgfqpoint{-18.583201in}{0.773588in}}%
\pgfpathlineto{\pgfqpoint{-18.652004in}{0.773588in}}%
\pgfpathlineto{\pgfqpoint{-18.721737in}{0.773588in}}%
\pgfpathlineto{\pgfqpoint{-18.792270in}{0.773588in}}%
\pgfpathlineto{\pgfqpoint{-18.859965in}{0.773588in}}%
\pgfpathlineto{\pgfqpoint{-18.928709in}{0.773588in}}%
\pgfpathlineto{\pgfqpoint{-18.999640in}{0.773588in}}%
\pgfpathlineto{\pgfqpoint{-19.068147in}{0.773588in}}%
\pgfpathlineto{\pgfqpoint{-19.135788in}{0.773588in}}%
\pgfpathlineto{\pgfqpoint{-19.205228in}{0.773588in}}%
\pgfpathlineto{\pgfqpoint{-19.271476in}{0.773588in}}%
\pgfpathlineto{\pgfqpoint{-19.338364in}{0.773588in}}%
\pgfpathlineto{\pgfqpoint{-19.407422in}{0.773588in}}%
\pgfpathlineto{\pgfqpoint{-19.475599in}{0.773588in}}%
\pgfpathlineto{\pgfqpoint{-19.543393in}{0.773588in}}%
\pgfpathlineto{\pgfqpoint{-19.613839in}{0.773588in}}%
\pgfpathlineto{\pgfqpoint{-19.680997in}{0.773588in}}%
\pgfpathlineto{\pgfqpoint{-19.747542in}{0.773588in}}%
\pgfpathlineto{\pgfqpoint{-19.815535in}{0.773588in}}%
\pgfpathlineto{\pgfqpoint{-19.882267in}{0.773588in}}%
\pgfpathlineto{\pgfqpoint{-19.949841in}{0.773588in}}%
\pgfpathlineto{\pgfqpoint{-20.019359in}{0.773588in}}%
\pgfpathlineto{\pgfqpoint{-20.086943in}{0.773588in}}%
\pgfpathlineto{\pgfqpoint{-20.154101in}{0.773588in}}%
\pgfpathlineto{\pgfqpoint{-20.223560in}{0.773588in}}%
\pgfpathlineto{\pgfqpoint{-20.290460in}{0.773588in}}%
\pgfpathlineto{\pgfqpoint{-20.357252in}{0.773588in}}%
\pgfpathlineto{\pgfqpoint{-20.427103in}{0.773588in}}%
\pgfpathlineto{\pgfqpoint{-20.496847in}{0.773588in}}%
\pgfpathlineto{\pgfqpoint{-20.565295in}{0.773588in}}%
\pgfpathlineto{\pgfqpoint{-20.636228in}{0.773588in}}%
\pgfpathlineto{\pgfqpoint{-20.705385in}{0.773588in}}%
\pgfpathlineto{\pgfqpoint{-20.774312in}{0.773588in}}%
\pgfpathlineto{\pgfqpoint{-20.844220in}{0.773588in}}%
\pgfpathlineto{\pgfqpoint{-20.913231in}{0.773588in}}%
\pgfpathlineto{\pgfqpoint{-20.982338in}{0.773588in}}%
\pgfpathlineto{\pgfqpoint{-21.053709in}{0.773588in}}%
\pgfpathlineto{\pgfqpoint{-21.123238in}{0.773588in}}%
\pgfpathlineto{\pgfqpoint{-21.191384in}{0.773588in}}%
\pgfpathlineto{\pgfqpoint{-21.261541in}{0.773588in}}%
\pgfpathlineto{\pgfqpoint{-21.330134in}{0.773588in}}%
\pgfpathlineto{\pgfqpoint{-21.398346in}{0.773588in}}%
\pgfpathlineto{\pgfqpoint{-21.470634in}{0.773588in}}%
\pgfpathlineto{\pgfqpoint{-21.539549in}{0.773588in}}%
\pgfpathlineto{\pgfqpoint{-21.607622in}{0.773588in}}%
\pgfpathlineto{\pgfqpoint{-21.676089in}{0.773588in}}%
\pgfpathlineto{\pgfqpoint{-21.743422in}{0.773588in}}%
\pgfpathlineto{\pgfqpoint{-21.811964in}{0.773588in}}%
\pgfpathlineto{\pgfqpoint{-21.881469in}{0.773588in}}%
\pgfpathlineto{\pgfqpoint{-21.948488in}{0.773588in}}%
\pgfpathlineto{\pgfqpoint{-22.016091in}{0.773588in}}%
\pgfpathlineto{\pgfqpoint{-22.084923in}{0.773588in}}%
\pgfpathlineto{\pgfqpoint{-22.150098in}{0.773588in}}%
\pgfpathlineto{\pgfqpoint{-22.217508in}{0.773588in}}%
\pgfpathlineto{\pgfqpoint{-22.286974in}{0.773588in}}%
\pgfpathlineto{\pgfqpoint{-22.353759in}{0.773588in}}%
\pgfpathlineto{\pgfqpoint{-22.421718in}{0.773588in}}%
\pgfpathlineto{\pgfqpoint{-22.492630in}{0.773588in}}%
\pgfpathlineto{\pgfqpoint{-22.560590in}{0.773588in}}%
\pgfpathlineto{\pgfqpoint{-22.627376in}{0.773588in}}%
\pgfpathlineto{\pgfqpoint{-22.696138in}{0.773588in}}%
\pgfpathlineto{\pgfqpoint{-22.764599in}{0.773588in}}%
\pgfpathlineto{\pgfqpoint{-22.831621in}{0.773588in}}%
\pgfpathlineto{\pgfqpoint{-22.900479in}{0.773588in}}%
\pgfpathlineto{\pgfqpoint{-22.968635in}{0.773588in}}%
\pgfpathlineto{\pgfqpoint{-23.037151in}{0.773588in}}%
\pgfpathlineto{\pgfqpoint{-23.107951in}{0.773588in}}%
\pgfpathlineto{\pgfqpoint{-23.175783in}{0.773588in}}%
\pgfpathlineto{\pgfqpoint{-23.243519in}{0.773588in}}%
\pgfpathlineto{\pgfqpoint{-23.314406in}{0.773588in}}%
\pgfpathlineto{\pgfqpoint{-23.383303in}{0.773588in}}%
\pgfpathlineto{\pgfqpoint{-23.451780in}{0.773588in}}%
\pgfpathlineto{\pgfqpoint{-23.523979in}{0.773588in}}%
\pgfpathlineto{\pgfqpoint{-23.594251in}{0.773588in}}%
\pgfpathlineto{\pgfqpoint{-23.664795in}{0.773588in}}%
\pgfpathlineto{\pgfqpoint{-23.739951in}{0.773588in}}%
\pgfpathlineto{\pgfqpoint{-23.810815in}{0.773588in}}%
\pgfpathlineto{\pgfqpoint{-23.881105in}{0.773588in}}%
\pgfpathlineto{\pgfqpoint{-23.953546in}{0.773588in}}%
\pgfpathlineto{\pgfqpoint{-24.025129in}{0.773588in}}%
\pgfpathlineto{\pgfqpoint{-24.096931in}{0.773588in}}%
\pgfpathlineto{\pgfqpoint{-24.172427in}{0.773588in}}%
\pgfpathlineto{\pgfqpoint{-24.243942in}{0.773588in}}%
\pgfpathlineto{\pgfqpoint{-24.313406in}{0.773588in}}%
\pgfpathlineto{\pgfqpoint{-24.381203in}{0.773588in}}%
\pgfpathlineto{\pgfqpoint{-24.445935in}{0.773588in}}%
\pgfpathlineto{\pgfqpoint{-24.511431in}{0.773588in}}%
\pgfpathlineto{\pgfqpoint{-24.578904in}{0.773588in}}%
\pgfpathlineto{\pgfqpoint{-24.645693in}{0.773588in}}%
\pgfpathlineto{\pgfqpoint{-24.712889in}{0.773588in}}%
\pgfpathlineto{\pgfqpoint{-24.781655in}{0.773588in}}%
\pgfpathlineto{\pgfqpoint{-24.847966in}{0.773588in}}%
\pgfpathlineto{\pgfqpoint{-24.916019in}{0.773588in}}%
\pgfpathlineto{\pgfqpoint{-24.984232in}{0.773588in}}%
\pgfpathlineto{\pgfqpoint{-25.049351in}{0.773588in}}%
\pgfpathlineto{\pgfqpoint{-25.114674in}{0.773588in}}%
\pgfpathlineto{\pgfqpoint{-25.182394in}{0.773588in}}%
\pgfpathlineto{\pgfqpoint{-25.248757in}{0.773588in}}%
\pgfpathlineto{\pgfqpoint{-25.315375in}{0.773588in}}%
\pgfpathlineto{\pgfqpoint{-25.382786in}{0.773588in}}%
\pgfpathlineto{\pgfqpoint{-25.448976in}{0.773588in}}%
\pgfpathlineto{\pgfqpoint{-25.515370in}{0.773588in}}%
\pgfpathlineto{\pgfqpoint{-25.583742in}{0.773588in}}%
\pgfpathlineto{\pgfqpoint{-25.650614in}{0.773588in}}%
\pgfpathlineto{\pgfqpoint{-25.717568in}{0.773588in}}%
\pgfpathlineto{\pgfqpoint{-25.785528in}{0.773588in}}%
\pgfpathlineto{\pgfqpoint{-25.851769in}{0.773588in}}%
\pgfpathlineto{\pgfqpoint{-25.919100in}{0.773588in}}%
\pgfpathlineto{\pgfqpoint{-25.990773in}{0.773588in}}%
\pgfpathlineto{\pgfqpoint{-26.059544in}{0.773588in}}%
\pgfpathlineto{\pgfqpoint{-26.128033in}{0.773588in}}%
\pgfpathlineto{\pgfqpoint{-26.198294in}{0.773588in}}%
\pgfpathlineto{\pgfqpoint{-26.268985in}{0.773588in}}%
\pgfpathlineto{\pgfqpoint{-26.338662in}{0.773588in}}%
\pgfpathlineto{\pgfqpoint{-26.410031in}{0.773588in}}%
\pgfpathlineto{\pgfqpoint{-26.479146in}{0.773588in}}%
\pgfpathlineto{\pgfqpoint{-26.547093in}{0.773588in}}%
\pgfpathlineto{\pgfqpoint{-26.616985in}{0.773588in}}%
\pgfpathlineto{\pgfqpoint{-26.685925in}{0.773588in}}%
\pgfpathlineto{\pgfqpoint{-26.755672in}{0.773588in}}%
\pgfpathlineto{\pgfqpoint{-26.825734in}{0.773588in}}%
\pgfpathlineto{\pgfqpoint{-26.893662in}{0.773588in}}%
\pgfpathlineto{\pgfqpoint{-26.961733in}{0.773588in}}%
\pgfpathlineto{\pgfqpoint{-27.031677in}{0.773588in}}%
\pgfpathlineto{\pgfqpoint{-27.097972in}{0.773588in}}%
\pgfpathlineto{\pgfqpoint{-27.165061in}{0.773588in}}%
\pgfpathlineto{\pgfqpoint{-27.232937in}{0.773588in}}%
\pgfpathlineto{\pgfqpoint{-27.299189in}{0.773588in}}%
\pgfpathlineto{\pgfqpoint{-27.366087in}{0.773588in}}%
\pgfpathlineto{\pgfqpoint{-27.435671in}{0.773588in}}%
\pgfpathlineto{\pgfqpoint{-27.503250in}{0.773588in}}%
\pgfpathlineto{\pgfqpoint{-27.570291in}{0.773588in}}%
\pgfpathlineto{\pgfqpoint{-27.637925in}{0.773588in}}%
\pgfpathlineto{\pgfqpoint{-27.703951in}{0.773588in}}%
\pgfpathlineto{\pgfqpoint{-27.771032in}{0.773588in}}%
\pgfpathlineto{\pgfqpoint{-27.839702in}{0.773588in}}%
\pgfpathlineto{\pgfqpoint{-27.908279in}{0.773588in}}%
\pgfpathlineto{\pgfqpoint{-27.976694in}{0.773588in}}%
\pgfpathlineto{\pgfqpoint{-28.049197in}{0.773588in}}%
\pgfpathlineto{\pgfqpoint{-28.121663in}{0.773588in}}%
\pgfpathlineto{\pgfqpoint{-28.195772in}{0.773588in}}%
\pgfpathlineto{\pgfqpoint{-28.274638in}{0.773588in}}%
\pgfpathlineto{\pgfqpoint{-28.356803in}{0.773588in}}%
\pgfpathlineto{\pgfqpoint{-28.429123in}{0.773588in}}%
\pgfpathlineto{\pgfqpoint{-28.504596in}{0.773588in}}%
\pgfpathlineto{\pgfqpoint{-28.577221in}{0.773588in}}%
\pgfpathlineto{\pgfqpoint{-28.649563in}{0.773588in}}%
\pgfpathlineto{\pgfqpoint{-28.725736in}{0.773588in}}%
\pgfpathlineto{\pgfqpoint{-28.798746in}{0.773588in}}%
\pgfpathlineto{\pgfqpoint{-28.870906in}{0.773588in}}%
\pgfpathlineto{\pgfqpoint{-28.946213in}{0.773588in}}%
\pgfpathlineto{\pgfqpoint{-29.018119in}{0.773588in}}%
\pgfpathlineto{\pgfqpoint{-29.089467in}{0.773588in}}%
\pgfpathlineto{\pgfqpoint{-29.164288in}{0.773588in}}%
\pgfpathlineto{\pgfqpoint{-29.235084in}{0.773588in}}%
\pgfpathlineto{\pgfqpoint{-29.304065in}{0.773588in}}%
\pgfpathlineto{\pgfqpoint{-29.374900in}{0.773588in}}%
\pgfpathlineto{\pgfqpoint{-29.443353in}{0.773588in}}%
\pgfpathlineto{\pgfqpoint{-29.511058in}{0.773588in}}%
\pgfpathlineto{\pgfqpoint{-29.581285in}{0.773588in}}%
\pgfpathlineto{\pgfqpoint{-29.647778in}{0.773588in}}%
\pgfpathlineto{\pgfqpoint{-29.716346in}{0.773588in}}%
\pgfpathlineto{\pgfqpoint{-29.785834in}{0.773588in}}%
\pgfpathlineto{\pgfqpoint{-29.853575in}{0.773588in}}%
\pgfpathlineto{\pgfqpoint{-29.923374in}{0.773588in}}%
\pgfpathlineto{\pgfqpoint{-29.994776in}{0.773588in}}%
\pgfpathlineto{\pgfqpoint{-30.062931in}{0.773588in}}%
\pgfpathlineto{\pgfqpoint{-30.131142in}{0.773588in}}%
\pgfpathlineto{\pgfqpoint{-30.202864in}{0.773588in}}%
\pgfpathlineto{\pgfqpoint{-30.269201in}{0.773588in}}%
\pgfpathlineto{\pgfqpoint{-30.335915in}{0.773588in}}%
\pgfpathlineto{\pgfqpoint{-30.404868in}{0.773588in}}%
\pgfpathlineto{\pgfqpoint{-30.472294in}{0.773588in}}%
\pgfpathlineto{\pgfqpoint{-30.539383in}{0.773588in}}%
\pgfpathlineto{\pgfqpoint{-30.610127in}{0.773588in}}%
\pgfpathlineto{\pgfqpoint{-30.676901in}{0.773588in}}%
\pgfpathlineto{\pgfqpoint{-30.745143in}{0.773588in}}%
\pgfpathclose%
\pgfusepath{fill}%
\end{pgfscope}%
\begin{pgfscope}%
\pgfpathrectangle{\pgfqpoint{3.332180in}{0.773588in}}{\pgfqpoint{2.293918in}{5.415119in}}%
\pgfusepath{clip}%
\pgfsetbuttcap%
\pgfsetroundjoin%
\definecolor{currentfill}{rgb}{0.839216,0.152941,0.156863}%
\pgfsetfillcolor{currentfill}%
\pgfsetlinewidth{0.000000pt}%
\definecolor{currentstroke}{rgb}{0.000000,0.000000,0.000000}%
\pgfsetstrokecolor{currentstroke}%
\pgfsetdash{}{0pt}%
\pgfpathmoveto{\pgfqpoint{-30.745143in}{1.307859in}}%
\pgfpathlineto{\pgfqpoint{-30.745143in}{0.773588in}}%
\pgfpathlineto{\pgfqpoint{-30.676901in}{0.773588in}}%
\pgfpathlineto{\pgfqpoint{-30.610127in}{0.773588in}}%
\pgfpathlineto{\pgfqpoint{-30.539383in}{0.773588in}}%
\pgfpathlineto{\pgfqpoint{-30.472294in}{0.773588in}}%
\pgfpathlineto{\pgfqpoint{-30.404868in}{0.773588in}}%
\pgfpathlineto{\pgfqpoint{-30.335915in}{0.773588in}}%
\pgfpathlineto{\pgfqpoint{-30.269201in}{0.773588in}}%
\pgfpathlineto{\pgfqpoint{-30.202864in}{0.773588in}}%
\pgfpathlineto{\pgfqpoint{-30.131142in}{0.773588in}}%
\pgfpathlineto{\pgfqpoint{-30.062931in}{0.773588in}}%
\pgfpathlineto{\pgfqpoint{-29.994776in}{0.773588in}}%
\pgfpathlineto{\pgfqpoint{-29.923374in}{0.773588in}}%
\pgfpathlineto{\pgfqpoint{-29.853575in}{0.773588in}}%
\pgfpathlineto{\pgfqpoint{-29.785834in}{0.773588in}}%
\pgfpathlineto{\pgfqpoint{-29.716346in}{0.773588in}}%
\pgfpathlineto{\pgfqpoint{-29.647778in}{0.773588in}}%
\pgfpathlineto{\pgfqpoint{-29.581285in}{0.773588in}}%
\pgfpathlineto{\pgfqpoint{-29.511058in}{0.773588in}}%
\pgfpathlineto{\pgfqpoint{-29.443353in}{0.773588in}}%
\pgfpathlineto{\pgfqpoint{-29.374900in}{0.773588in}}%
\pgfpathlineto{\pgfqpoint{-29.304065in}{0.773588in}}%
\pgfpathlineto{\pgfqpoint{-29.235084in}{0.773588in}}%
\pgfpathlineto{\pgfqpoint{-29.164288in}{0.773588in}}%
\pgfpathlineto{\pgfqpoint{-29.089467in}{0.773588in}}%
\pgfpathlineto{\pgfqpoint{-29.018119in}{0.773588in}}%
\pgfpathlineto{\pgfqpoint{-28.946213in}{0.773588in}}%
\pgfpathlineto{\pgfqpoint{-28.870906in}{0.773588in}}%
\pgfpathlineto{\pgfqpoint{-28.798746in}{0.773588in}}%
\pgfpathlineto{\pgfqpoint{-28.725736in}{0.773588in}}%
\pgfpathlineto{\pgfqpoint{-28.649563in}{0.773588in}}%
\pgfpathlineto{\pgfqpoint{-28.577221in}{0.773588in}}%
\pgfpathlineto{\pgfqpoint{-28.504596in}{0.773588in}}%
\pgfpathlineto{\pgfqpoint{-28.429123in}{0.773588in}}%
\pgfpathlineto{\pgfqpoint{-28.356803in}{0.773588in}}%
\pgfpathlineto{\pgfqpoint{-28.274638in}{0.773588in}}%
\pgfpathlineto{\pgfqpoint{-28.195772in}{0.773588in}}%
\pgfpathlineto{\pgfqpoint{-28.121663in}{0.773588in}}%
\pgfpathlineto{\pgfqpoint{-28.049197in}{0.773588in}}%
\pgfpathlineto{\pgfqpoint{-27.976694in}{0.773588in}}%
\pgfpathlineto{\pgfqpoint{-27.908279in}{0.773588in}}%
\pgfpathlineto{\pgfqpoint{-27.839702in}{0.773588in}}%
\pgfpathlineto{\pgfqpoint{-27.771032in}{0.773588in}}%
\pgfpathlineto{\pgfqpoint{-27.703951in}{0.773588in}}%
\pgfpathlineto{\pgfqpoint{-27.637925in}{0.773588in}}%
\pgfpathlineto{\pgfqpoint{-27.570291in}{0.773588in}}%
\pgfpathlineto{\pgfqpoint{-27.503250in}{0.773588in}}%
\pgfpathlineto{\pgfqpoint{-27.435671in}{0.773588in}}%
\pgfpathlineto{\pgfqpoint{-27.366087in}{0.773588in}}%
\pgfpathlineto{\pgfqpoint{-27.299189in}{0.773588in}}%
\pgfpathlineto{\pgfqpoint{-27.232937in}{0.773588in}}%
\pgfpathlineto{\pgfqpoint{-27.165061in}{0.773588in}}%
\pgfpathlineto{\pgfqpoint{-27.097972in}{0.773588in}}%
\pgfpathlineto{\pgfqpoint{-27.031677in}{0.773588in}}%
\pgfpathlineto{\pgfqpoint{-26.961733in}{0.773588in}}%
\pgfpathlineto{\pgfqpoint{-26.893662in}{0.773588in}}%
\pgfpathlineto{\pgfqpoint{-26.825734in}{0.773588in}}%
\pgfpathlineto{\pgfqpoint{-26.755672in}{0.773588in}}%
\pgfpathlineto{\pgfqpoint{-26.685925in}{0.773588in}}%
\pgfpathlineto{\pgfqpoint{-26.616985in}{0.773588in}}%
\pgfpathlineto{\pgfqpoint{-26.547093in}{0.773588in}}%
\pgfpathlineto{\pgfqpoint{-26.479146in}{0.773588in}}%
\pgfpathlineto{\pgfqpoint{-26.410031in}{0.773588in}}%
\pgfpathlineto{\pgfqpoint{-26.338662in}{0.773588in}}%
\pgfpathlineto{\pgfqpoint{-26.268985in}{0.773588in}}%
\pgfpathlineto{\pgfqpoint{-26.198294in}{0.773588in}}%
\pgfpathlineto{\pgfqpoint{-26.128033in}{0.773588in}}%
\pgfpathlineto{\pgfqpoint{-26.059544in}{0.773588in}}%
\pgfpathlineto{\pgfqpoint{-25.990773in}{0.773588in}}%
\pgfpathlineto{\pgfqpoint{-25.919100in}{0.773588in}}%
\pgfpathlineto{\pgfqpoint{-25.851769in}{0.773588in}}%
\pgfpathlineto{\pgfqpoint{-25.785528in}{0.773588in}}%
\pgfpathlineto{\pgfqpoint{-25.717568in}{0.773588in}}%
\pgfpathlineto{\pgfqpoint{-25.650614in}{0.773588in}}%
\pgfpathlineto{\pgfqpoint{-25.583742in}{0.773588in}}%
\pgfpathlineto{\pgfqpoint{-25.515370in}{0.773588in}}%
\pgfpathlineto{\pgfqpoint{-25.448976in}{0.773588in}}%
\pgfpathlineto{\pgfqpoint{-25.382786in}{0.773588in}}%
\pgfpathlineto{\pgfqpoint{-25.315375in}{0.773588in}}%
\pgfpathlineto{\pgfqpoint{-25.248757in}{0.773588in}}%
\pgfpathlineto{\pgfqpoint{-25.182394in}{0.773588in}}%
\pgfpathlineto{\pgfqpoint{-25.114674in}{0.773588in}}%
\pgfpathlineto{\pgfqpoint{-25.049351in}{0.773588in}}%
\pgfpathlineto{\pgfqpoint{-24.984232in}{0.773588in}}%
\pgfpathlineto{\pgfqpoint{-24.916019in}{0.773588in}}%
\pgfpathlineto{\pgfqpoint{-24.847966in}{0.773588in}}%
\pgfpathlineto{\pgfqpoint{-24.781655in}{0.773588in}}%
\pgfpathlineto{\pgfqpoint{-24.712889in}{0.773588in}}%
\pgfpathlineto{\pgfqpoint{-24.645693in}{0.773588in}}%
\pgfpathlineto{\pgfqpoint{-24.578904in}{0.773588in}}%
\pgfpathlineto{\pgfqpoint{-24.511431in}{0.773588in}}%
\pgfpathlineto{\pgfqpoint{-24.445935in}{0.773588in}}%
\pgfpathlineto{\pgfqpoint{-24.381203in}{0.773588in}}%
\pgfpathlineto{\pgfqpoint{-24.313406in}{0.773588in}}%
\pgfpathlineto{\pgfqpoint{-24.243942in}{0.773588in}}%
\pgfpathlineto{\pgfqpoint{-24.172427in}{0.773588in}}%
\pgfpathlineto{\pgfqpoint{-24.096931in}{0.773588in}}%
\pgfpathlineto{\pgfqpoint{-24.025129in}{0.773588in}}%
\pgfpathlineto{\pgfqpoint{-23.953546in}{0.773588in}}%
\pgfpathlineto{\pgfqpoint{-23.881105in}{0.773588in}}%
\pgfpathlineto{\pgfqpoint{-23.810815in}{0.773588in}}%
\pgfpathlineto{\pgfqpoint{-23.739951in}{0.773588in}}%
\pgfpathlineto{\pgfqpoint{-23.664795in}{0.773588in}}%
\pgfpathlineto{\pgfqpoint{-23.594251in}{0.773588in}}%
\pgfpathlineto{\pgfqpoint{-23.523979in}{0.773588in}}%
\pgfpathlineto{\pgfqpoint{-23.451780in}{0.773588in}}%
\pgfpathlineto{\pgfqpoint{-23.383303in}{0.773588in}}%
\pgfpathlineto{\pgfqpoint{-23.314406in}{0.773588in}}%
\pgfpathlineto{\pgfqpoint{-23.243519in}{0.773588in}}%
\pgfpathlineto{\pgfqpoint{-23.175783in}{0.773588in}}%
\pgfpathlineto{\pgfqpoint{-23.107951in}{0.773588in}}%
\pgfpathlineto{\pgfqpoint{-23.037151in}{0.773588in}}%
\pgfpathlineto{\pgfqpoint{-22.968635in}{0.773588in}}%
\pgfpathlineto{\pgfqpoint{-22.900479in}{0.773588in}}%
\pgfpathlineto{\pgfqpoint{-22.831621in}{0.773588in}}%
\pgfpathlineto{\pgfqpoint{-22.764599in}{0.773588in}}%
\pgfpathlineto{\pgfqpoint{-22.696138in}{0.773588in}}%
\pgfpathlineto{\pgfqpoint{-22.627376in}{0.773588in}}%
\pgfpathlineto{\pgfqpoint{-22.560590in}{0.773588in}}%
\pgfpathlineto{\pgfqpoint{-22.492630in}{0.773588in}}%
\pgfpathlineto{\pgfqpoint{-22.421718in}{0.773588in}}%
\pgfpathlineto{\pgfqpoint{-22.353759in}{0.773588in}}%
\pgfpathlineto{\pgfqpoint{-22.286974in}{0.773588in}}%
\pgfpathlineto{\pgfqpoint{-22.217508in}{0.773588in}}%
\pgfpathlineto{\pgfqpoint{-22.150098in}{0.773588in}}%
\pgfpathlineto{\pgfqpoint{-22.084923in}{0.773588in}}%
\pgfpathlineto{\pgfqpoint{-22.016091in}{0.773588in}}%
\pgfpathlineto{\pgfqpoint{-21.948488in}{0.773588in}}%
\pgfpathlineto{\pgfqpoint{-21.881469in}{0.773588in}}%
\pgfpathlineto{\pgfqpoint{-21.811964in}{0.773588in}}%
\pgfpathlineto{\pgfqpoint{-21.743422in}{0.773588in}}%
\pgfpathlineto{\pgfqpoint{-21.676089in}{0.773588in}}%
\pgfpathlineto{\pgfqpoint{-21.607622in}{0.773588in}}%
\pgfpathlineto{\pgfqpoint{-21.539549in}{0.773588in}}%
\pgfpathlineto{\pgfqpoint{-21.470634in}{0.773588in}}%
\pgfpathlineto{\pgfqpoint{-21.398346in}{0.773588in}}%
\pgfpathlineto{\pgfqpoint{-21.330134in}{0.773588in}}%
\pgfpathlineto{\pgfqpoint{-21.261541in}{0.773588in}}%
\pgfpathlineto{\pgfqpoint{-21.191384in}{0.773588in}}%
\pgfpathlineto{\pgfqpoint{-21.123238in}{0.773588in}}%
\pgfpathlineto{\pgfqpoint{-21.053709in}{0.773588in}}%
\pgfpathlineto{\pgfqpoint{-20.982338in}{0.773588in}}%
\pgfpathlineto{\pgfqpoint{-20.913231in}{0.773588in}}%
\pgfpathlineto{\pgfqpoint{-20.844220in}{0.773588in}}%
\pgfpathlineto{\pgfqpoint{-20.774312in}{0.773588in}}%
\pgfpathlineto{\pgfqpoint{-20.705385in}{0.773588in}}%
\pgfpathlineto{\pgfqpoint{-20.636228in}{0.773588in}}%
\pgfpathlineto{\pgfqpoint{-20.565295in}{0.773588in}}%
\pgfpathlineto{\pgfqpoint{-20.496847in}{0.773588in}}%
\pgfpathlineto{\pgfqpoint{-20.427103in}{0.773588in}}%
\pgfpathlineto{\pgfqpoint{-20.357252in}{0.773588in}}%
\pgfpathlineto{\pgfqpoint{-20.290460in}{0.773588in}}%
\pgfpathlineto{\pgfqpoint{-20.223560in}{0.773588in}}%
\pgfpathlineto{\pgfqpoint{-20.154101in}{0.773588in}}%
\pgfpathlineto{\pgfqpoint{-20.086943in}{0.773588in}}%
\pgfpathlineto{\pgfqpoint{-20.019359in}{0.773588in}}%
\pgfpathlineto{\pgfqpoint{-19.949841in}{0.773588in}}%
\pgfpathlineto{\pgfqpoint{-19.882267in}{0.773588in}}%
\pgfpathlineto{\pgfqpoint{-19.815535in}{0.773588in}}%
\pgfpathlineto{\pgfqpoint{-19.747542in}{0.773588in}}%
\pgfpathlineto{\pgfqpoint{-19.680997in}{0.773588in}}%
\pgfpathlineto{\pgfqpoint{-19.613839in}{0.773588in}}%
\pgfpathlineto{\pgfqpoint{-19.543393in}{0.773588in}}%
\pgfpathlineto{\pgfqpoint{-19.475599in}{0.773588in}}%
\pgfpathlineto{\pgfqpoint{-19.407422in}{0.773588in}}%
\pgfpathlineto{\pgfqpoint{-19.338364in}{0.773588in}}%
\pgfpathlineto{\pgfqpoint{-19.271476in}{0.773588in}}%
\pgfpathlineto{\pgfqpoint{-19.205228in}{0.773588in}}%
\pgfpathlineto{\pgfqpoint{-19.135788in}{0.773588in}}%
\pgfpathlineto{\pgfqpoint{-19.068147in}{0.773588in}}%
\pgfpathlineto{\pgfqpoint{-18.999640in}{0.773588in}}%
\pgfpathlineto{\pgfqpoint{-18.928709in}{0.773588in}}%
\pgfpathlineto{\pgfqpoint{-18.859965in}{0.773588in}}%
\pgfpathlineto{\pgfqpoint{-18.792270in}{0.773588in}}%
\pgfpathlineto{\pgfqpoint{-18.721737in}{0.773588in}}%
\pgfpathlineto{\pgfqpoint{-18.652004in}{0.773588in}}%
\pgfpathlineto{\pgfqpoint{-18.583201in}{0.773588in}}%
\pgfpathlineto{\pgfqpoint{-18.511159in}{0.773588in}}%
\pgfpathlineto{\pgfqpoint{-18.442527in}{0.773588in}}%
\pgfpathlineto{\pgfqpoint{-18.373957in}{0.773588in}}%
\pgfpathlineto{\pgfqpoint{-18.301710in}{0.773588in}}%
\pgfpathlineto{\pgfqpoint{-18.232823in}{0.773588in}}%
\pgfpathlineto{\pgfqpoint{-18.165145in}{0.773588in}}%
\pgfpathlineto{\pgfqpoint{-18.094404in}{0.773588in}}%
\pgfpathlineto{\pgfqpoint{-18.025876in}{0.773588in}}%
\pgfpathlineto{\pgfqpoint{-17.957185in}{0.773588in}}%
\pgfpathlineto{\pgfqpoint{-17.885356in}{0.773588in}}%
\pgfpathlineto{\pgfqpoint{-17.816737in}{0.773588in}}%
\pgfpathlineto{\pgfqpoint{-17.748008in}{0.773588in}}%
\pgfpathlineto{\pgfqpoint{-17.675451in}{0.773588in}}%
\pgfpathlineto{\pgfqpoint{-17.607453in}{0.773588in}}%
\pgfpathlineto{\pgfqpoint{-17.539598in}{0.773588in}}%
\pgfpathlineto{\pgfqpoint{-17.470673in}{0.773588in}}%
\pgfpathlineto{\pgfqpoint{-17.402675in}{0.773588in}}%
\pgfpathlineto{\pgfqpoint{-17.336134in}{0.773588in}}%
\pgfpathlineto{\pgfqpoint{-17.265552in}{0.773588in}}%
\pgfpathlineto{\pgfqpoint{-17.196258in}{0.773588in}}%
\pgfpathlineto{\pgfqpoint{-17.128098in}{0.773588in}}%
\pgfpathlineto{\pgfqpoint{-17.058127in}{0.773588in}}%
\pgfpathlineto{\pgfqpoint{-16.989953in}{0.773588in}}%
\pgfpathlineto{\pgfqpoint{-16.921714in}{0.773588in}}%
\pgfpathlineto{\pgfqpoint{-16.852234in}{0.773588in}}%
\pgfpathlineto{\pgfqpoint{-16.784012in}{0.773588in}}%
\pgfpathlineto{\pgfqpoint{-16.715630in}{0.773588in}}%
\pgfpathlineto{\pgfqpoint{-16.644038in}{0.773588in}}%
\pgfpathlineto{\pgfqpoint{-16.575864in}{0.773588in}}%
\pgfpathlineto{\pgfqpoint{-16.507797in}{0.773588in}}%
\pgfpathlineto{\pgfqpoint{-16.437097in}{0.773588in}}%
\pgfpathlineto{\pgfqpoint{-16.369724in}{0.773588in}}%
\pgfpathlineto{\pgfqpoint{-16.302639in}{0.773588in}}%
\pgfpathlineto{\pgfqpoint{-16.232068in}{0.773588in}}%
\pgfpathlineto{\pgfqpoint{-16.163654in}{0.773588in}}%
\pgfpathlineto{\pgfqpoint{-16.094895in}{0.773588in}}%
\pgfpathlineto{\pgfqpoint{-16.024276in}{0.773588in}}%
\pgfpathlineto{\pgfqpoint{-15.956358in}{0.773588in}}%
\pgfpathlineto{\pgfqpoint{-15.886757in}{0.773588in}}%
\pgfpathlineto{\pgfqpoint{-15.814335in}{0.773588in}}%
\pgfpathlineto{\pgfqpoint{-15.744769in}{0.773588in}}%
\pgfpathlineto{\pgfqpoint{-15.673559in}{0.773588in}}%
\pgfpathlineto{\pgfqpoint{-15.602559in}{0.773588in}}%
\pgfpathlineto{\pgfqpoint{-15.532639in}{0.773588in}}%
\pgfpathlineto{\pgfqpoint{-15.463558in}{0.773588in}}%
\pgfpathlineto{\pgfqpoint{-15.392048in}{0.773588in}}%
\pgfpathlineto{\pgfqpoint{-15.323000in}{0.773588in}}%
\pgfpathlineto{\pgfqpoint{-15.253001in}{0.773588in}}%
\pgfpathlineto{\pgfqpoint{-15.179182in}{0.773588in}}%
\pgfpathlineto{\pgfqpoint{-15.106256in}{0.773588in}}%
\pgfpathlineto{\pgfqpoint{-15.032843in}{0.773588in}}%
\pgfpathlineto{\pgfqpoint{-14.959865in}{0.773588in}}%
\pgfpathlineto{\pgfqpoint{-14.889438in}{0.773588in}}%
\pgfpathlineto{\pgfqpoint{-14.820408in}{0.773588in}}%
\pgfpathlineto{\pgfqpoint{-14.748835in}{0.773588in}}%
\pgfpathlineto{\pgfqpoint{-14.679263in}{0.773588in}}%
\pgfpathlineto{\pgfqpoint{-14.609938in}{0.773588in}}%
\pgfpathlineto{\pgfqpoint{-14.538988in}{0.773588in}}%
\pgfpathlineto{\pgfqpoint{-14.471123in}{0.773588in}}%
\pgfpathlineto{\pgfqpoint{-14.401957in}{0.773588in}}%
\pgfpathlineto{\pgfqpoint{-14.331277in}{0.773588in}}%
\pgfpathlineto{\pgfqpoint{-14.263436in}{0.773588in}}%
\pgfpathlineto{\pgfqpoint{-14.195714in}{0.773588in}}%
\pgfpathlineto{\pgfqpoint{-14.125666in}{0.773588in}}%
\pgfpathlineto{\pgfqpoint{-14.057377in}{0.773588in}}%
\pgfpathlineto{\pgfqpoint{-13.988983in}{0.773588in}}%
\pgfpathlineto{\pgfqpoint{-13.917455in}{0.773588in}}%
\pgfpathlineto{\pgfqpoint{-13.848756in}{0.773588in}}%
\pgfpathlineto{\pgfqpoint{-13.779339in}{0.773588in}}%
\pgfpathlineto{\pgfqpoint{-13.708089in}{0.773588in}}%
\pgfpathlineto{\pgfqpoint{-13.639609in}{0.773588in}}%
\pgfpathlineto{\pgfqpoint{-13.568713in}{0.773588in}}%
\pgfpathlineto{\pgfqpoint{-13.496151in}{0.773588in}}%
\pgfpathlineto{\pgfqpoint{-13.426276in}{0.773588in}}%
\pgfpathlineto{\pgfqpoint{-13.356968in}{0.773588in}}%
\pgfpathlineto{\pgfqpoint{-13.285456in}{0.773588in}}%
\pgfpathlineto{\pgfqpoint{-13.216013in}{0.773588in}}%
\pgfpathlineto{\pgfqpoint{-13.145337in}{0.773588in}}%
\pgfpathlineto{\pgfqpoint{-13.073093in}{0.773588in}}%
\pgfpathlineto{\pgfqpoint{-13.002727in}{0.773588in}}%
\pgfpathlineto{\pgfqpoint{-12.932388in}{0.773588in}}%
\pgfpathlineto{\pgfqpoint{-12.859398in}{0.773588in}}%
\pgfpathlineto{\pgfqpoint{-12.788679in}{0.773588in}}%
\pgfpathlineto{\pgfqpoint{-12.718096in}{0.773588in}}%
\pgfpathlineto{\pgfqpoint{-12.645279in}{0.773588in}}%
\pgfpathlineto{\pgfqpoint{-12.574652in}{0.773588in}}%
\pgfpathlineto{\pgfqpoint{-12.503401in}{0.773588in}}%
\pgfpathlineto{\pgfqpoint{-12.428997in}{0.773588in}}%
\pgfpathlineto{\pgfqpoint{-12.357103in}{0.773588in}}%
\pgfpathlineto{\pgfqpoint{-12.285275in}{0.773588in}}%
\pgfpathlineto{\pgfqpoint{-12.211536in}{0.773588in}}%
\pgfpathlineto{\pgfqpoint{-12.139885in}{0.773588in}}%
\pgfpathlineto{\pgfqpoint{-12.069446in}{0.773588in}}%
\pgfpathlineto{\pgfqpoint{-11.998596in}{0.773588in}}%
\pgfpathlineto{\pgfqpoint{-11.928381in}{0.773588in}}%
\pgfpathlineto{\pgfqpoint{-11.858471in}{0.773588in}}%
\pgfpathlineto{\pgfqpoint{-11.786176in}{0.773588in}}%
\pgfpathlineto{\pgfqpoint{-11.716309in}{0.773588in}}%
\pgfpathlineto{\pgfqpoint{-11.645187in}{0.773588in}}%
\pgfpathlineto{\pgfqpoint{-11.571037in}{0.773588in}}%
\pgfpathlineto{\pgfqpoint{-11.500166in}{0.773588in}}%
\pgfpathlineto{\pgfqpoint{-11.430351in}{0.773588in}}%
\pgfpathlineto{\pgfqpoint{-11.358676in}{0.773588in}}%
\pgfpathlineto{\pgfqpoint{-11.288273in}{0.773588in}}%
\pgfpathlineto{\pgfqpoint{-11.217631in}{0.773588in}}%
\pgfpathlineto{\pgfqpoint{-11.144960in}{0.773588in}}%
\pgfpathlineto{\pgfqpoint{-11.075276in}{0.773588in}}%
\pgfpathlineto{\pgfqpoint{-11.005065in}{0.773588in}}%
\pgfpathlineto{\pgfqpoint{-10.933808in}{0.773588in}}%
\pgfpathlineto{\pgfqpoint{-10.863257in}{0.773588in}}%
\pgfpathlineto{\pgfqpoint{-10.793422in}{0.773588in}}%
\pgfpathlineto{\pgfqpoint{-10.721651in}{0.773588in}}%
\pgfpathlineto{\pgfqpoint{-10.652649in}{0.773588in}}%
\pgfpathlineto{\pgfqpoint{-10.582864in}{0.773588in}}%
\pgfpathlineto{\pgfqpoint{-10.511372in}{0.773588in}}%
\pgfpathlineto{\pgfqpoint{-10.441581in}{0.773588in}}%
\pgfpathlineto{\pgfqpoint{-10.371004in}{0.773588in}}%
\pgfpathlineto{\pgfqpoint{-10.297591in}{0.773588in}}%
\pgfpathlineto{\pgfqpoint{-10.227739in}{0.773588in}}%
\pgfpathlineto{\pgfqpoint{-10.157398in}{0.773588in}}%
\pgfpathlineto{\pgfqpoint{-10.084496in}{0.773588in}}%
\pgfpathlineto{\pgfqpoint{-10.012635in}{0.773588in}}%
\pgfpathlineto{\pgfqpoint{-9.941359in}{0.773588in}}%
\pgfpathlineto{\pgfqpoint{-9.868243in}{0.773588in}}%
\pgfpathlineto{\pgfqpoint{-9.796010in}{0.773588in}}%
\pgfpathlineto{\pgfqpoint{-9.723682in}{0.773588in}}%
\pgfpathlineto{\pgfqpoint{-9.648369in}{0.773588in}}%
\pgfpathlineto{\pgfqpoint{-9.575400in}{0.773588in}}%
\pgfpathlineto{\pgfqpoint{-9.504196in}{0.773588in}}%
\pgfpathlineto{\pgfqpoint{-9.429774in}{0.773588in}}%
\pgfpathlineto{\pgfqpoint{-9.356529in}{0.773588in}}%
\pgfpathlineto{\pgfqpoint{-9.283412in}{0.773588in}}%
\pgfpathlineto{\pgfqpoint{-9.208906in}{0.773588in}}%
\pgfpathlineto{\pgfqpoint{-9.138788in}{0.773588in}}%
\pgfpathlineto{\pgfqpoint{-9.069721in}{0.773588in}}%
\pgfpathlineto{\pgfqpoint{-8.998701in}{0.773588in}}%
\pgfpathlineto{\pgfqpoint{-8.928123in}{0.773588in}}%
\pgfpathlineto{\pgfqpoint{-8.858221in}{0.773588in}}%
\pgfpathlineto{\pgfqpoint{-8.787586in}{0.773588in}}%
\pgfpathlineto{\pgfqpoint{-8.718609in}{0.773588in}}%
\pgfpathlineto{\pgfqpoint{-8.649341in}{0.773588in}}%
\pgfpathlineto{\pgfqpoint{-8.578369in}{0.773588in}}%
\pgfpathlineto{\pgfqpoint{-8.508839in}{0.773588in}}%
\pgfpathlineto{\pgfqpoint{-8.438970in}{0.773588in}}%
\pgfpathlineto{\pgfqpoint{-8.367967in}{0.773588in}}%
\pgfpathlineto{\pgfqpoint{-8.298753in}{0.773588in}}%
\pgfpathlineto{\pgfqpoint{-8.228241in}{0.773588in}}%
\pgfpathlineto{\pgfqpoint{-8.156111in}{0.773588in}}%
\pgfpathlineto{\pgfqpoint{-8.084913in}{0.773588in}}%
\pgfpathlineto{\pgfqpoint{-8.014387in}{0.773588in}}%
\pgfpathlineto{\pgfqpoint{-7.942200in}{0.773588in}}%
\pgfpathlineto{\pgfqpoint{-7.870788in}{0.773588in}}%
\pgfpathlineto{\pgfqpoint{-7.799875in}{0.773588in}}%
\pgfpathlineto{\pgfqpoint{-7.727286in}{0.773588in}}%
\pgfpathlineto{\pgfqpoint{-7.657024in}{0.773588in}}%
\pgfpathlineto{\pgfqpoint{-7.585515in}{0.773588in}}%
\pgfpathlineto{\pgfqpoint{-7.512981in}{0.773588in}}%
\pgfpathlineto{\pgfqpoint{-7.440452in}{0.773588in}}%
\pgfpathlineto{\pgfqpoint{-7.368338in}{0.773588in}}%
\pgfpathlineto{\pgfqpoint{-7.294929in}{0.773588in}}%
\pgfpathlineto{\pgfqpoint{-7.223147in}{0.773588in}}%
\pgfpathlineto{\pgfqpoint{-7.151909in}{0.773588in}}%
\pgfpathlineto{\pgfqpoint{-7.075897in}{0.773588in}}%
\pgfpathlineto{\pgfqpoint{-7.003598in}{0.773588in}}%
\pgfpathlineto{\pgfqpoint{-6.933558in}{0.773588in}}%
\pgfpathlineto{\pgfqpoint{-6.860576in}{0.773588in}}%
\pgfpathlineto{\pgfqpoint{-6.787495in}{0.773588in}}%
\pgfpathlineto{\pgfqpoint{-6.715245in}{0.773588in}}%
\pgfpathlineto{\pgfqpoint{-6.640496in}{0.773588in}}%
\pgfpathlineto{\pgfqpoint{-6.568712in}{0.773588in}}%
\pgfpathlineto{\pgfqpoint{-6.496669in}{0.773588in}}%
\pgfpathlineto{\pgfqpoint{-6.421127in}{0.773588in}}%
\pgfpathlineto{\pgfqpoint{-6.349020in}{0.773588in}}%
\pgfpathlineto{\pgfqpoint{-6.278887in}{0.773588in}}%
\pgfpathlineto{\pgfqpoint{-6.205113in}{0.773588in}}%
\pgfpathlineto{\pgfqpoint{-6.135189in}{0.773588in}}%
\pgfpathlineto{\pgfqpoint{-6.064664in}{0.773588in}}%
\pgfpathlineto{\pgfqpoint{-5.992127in}{0.773588in}}%
\pgfpathlineto{\pgfqpoint{-5.921215in}{0.773588in}}%
\pgfpathlineto{\pgfqpoint{-5.849895in}{0.773588in}}%
\pgfpathlineto{\pgfqpoint{-5.776542in}{0.773588in}}%
\pgfpathlineto{\pgfqpoint{-5.706570in}{0.773588in}}%
\pgfpathlineto{\pgfqpoint{-5.635685in}{0.773588in}}%
\pgfpathlineto{\pgfqpoint{-5.563648in}{0.773588in}}%
\pgfpathlineto{\pgfqpoint{-5.493615in}{0.773588in}}%
\pgfpathlineto{\pgfqpoint{-5.421890in}{0.773588in}}%
\pgfpathlineto{\pgfqpoint{-5.348665in}{0.773588in}}%
\pgfpathlineto{\pgfqpoint{-5.277970in}{0.773588in}}%
\pgfpathlineto{\pgfqpoint{-5.207317in}{0.773588in}}%
\pgfpathlineto{\pgfqpoint{-5.135005in}{0.773588in}}%
\pgfpathlineto{\pgfqpoint{-5.064817in}{0.773588in}}%
\pgfpathlineto{\pgfqpoint{-4.995002in}{0.773588in}}%
\pgfpathlineto{\pgfqpoint{-4.922664in}{0.773588in}}%
\pgfpathlineto{\pgfqpoint{-4.852652in}{0.773588in}}%
\pgfpathlineto{\pgfqpoint{-4.782965in}{0.773588in}}%
\pgfpathlineto{\pgfqpoint{-4.711002in}{0.773588in}}%
\pgfpathlineto{\pgfqpoint{-4.639570in}{0.773588in}}%
\pgfpathlineto{\pgfqpoint{-4.568595in}{0.773588in}}%
\pgfpathlineto{\pgfqpoint{-4.495401in}{0.773588in}}%
\pgfpathlineto{\pgfqpoint{-4.423289in}{0.773588in}}%
\pgfpathlineto{\pgfqpoint{-4.350759in}{0.773588in}}%
\pgfpathlineto{\pgfqpoint{-4.274748in}{0.773588in}}%
\pgfpathlineto{\pgfqpoint{-4.201246in}{0.773588in}}%
\pgfpathlineto{\pgfqpoint{-4.128352in}{0.773588in}}%
\pgfpathlineto{\pgfqpoint{-4.052173in}{0.773588in}}%
\pgfpathlineto{\pgfqpoint{-3.977308in}{0.773588in}}%
\pgfpathlineto{\pgfqpoint{-3.903164in}{0.773588in}}%
\pgfpathlineto{\pgfqpoint{-3.828230in}{0.773588in}}%
\pgfpathlineto{\pgfqpoint{-3.756893in}{0.773588in}}%
\pgfpathlineto{\pgfqpoint{-3.684119in}{0.773588in}}%
\pgfpathlineto{\pgfqpoint{-3.608325in}{0.773588in}}%
\pgfpathlineto{\pgfqpoint{-3.534002in}{0.773588in}}%
\pgfpathlineto{\pgfqpoint{-3.461876in}{0.773588in}}%
\pgfpathlineto{\pgfqpoint{-3.388885in}{0.773588in}}%
\pgfpathlineto{\pgfqpoint{-3.317677in}{0.773588in}}%
\pgfpathlineto{\pgfqpoint{-3.245632in}{0.773588in}}%
\pgfpathlineto{\pgfqpoint{-3.171735in}{0.773588in}}%
\pgfpathlineto{\pgfqpoint{-3.100293in}{0.773588in}}%
\pgfpathlineto{\pgfqpoint{-3.029702in}{0.773588in}}%
\pgfpathlineto{\pgfqpoint{-2.956196in}{0.773588in}}%
\pgfpathlineto{\pgfqpoint{-2.884033in}{0.773588in}}%
\pgfpathlineto{\pgfqpoint{-2.813631in}{0.773588in}}%
\pgfpathlineto{\pgfqpoint{-2.740325in}{0.773588in}}%
\pgfpathlineto{\pgfqpoint{-2.667020in}{0.773588in}}%
\pgfpathlineto{\pgfqpoint{-2.595405in}{0.773588in}}%
\pgfpathlineto{\pgfqpoint{-2.522349in}{0.773588in}}%
\pgfpathlineto{\pgfqpoint{-2.451489in}{0.773588in}}%
\pgfpathlineto{\pgfqpoint{-2.379035in}{0.773588in}}%
\pgfpathlineto{\pgfqpoint{-2.303263in}{0.773588in}}%
\pgfpathlineto{\pgfqpoint{-2.231477in}{0.773588in}}%
\pgfpathlineto{\pgfqpoint{-2.158577in}{0.773588in}}%
\pgfpathlineto{\pgfqpoint{-2.084338in}{0.773588in}}%
\pgfpathlineto{\pgfqpoint{-2.011817in}{0.773588in}}%
\pgfpathlineto{\pgfqpoint{-1.940536in}{0.773588in}}%
\pgfpathlineto{\pgfqpoint{-1.867809in}{0.773588in}}%
\pgfpathlineto{\pgfqpoint{-1.797123in}{0.773588in}}%
\pgfpathlineto{\pgfqpoint{-1.725105in}{0.773588in}}%
\pgfpathlineto{\pgfqpoint{-1.650581in}{0.773588in}}%
\pgfpathlineto{\pgfqpoint{-1.577590in}{0.773588in}}%
\pgfpathlineto{\pgfqpoint{-1.503984in}{0.773588in}}%
\pgfpathlineto{\pgfqpoint{-1.429441in}{0.773588in}}%
\pgfpathlineto{\pgfqpoint{-1.356794in}{0.773588in}}%
\pgfpathlineto{\pgfqpoint{-1.283200in}{0.773588in}}%
\pgfpathlineto{\pgfqpoint{-1.207908in}{0.773588in}}%
\pgfpathlineto{\pgfqpoint{-1.135138in}{0.773588in}}%
\pgfpathlineto{\pgfqpoint{-1.063245in}{0.773588in}}%
\pgfpathlineto{\pgfqpoint{-0.989522in}{0.773588in}}%
\pgfpathlineto{\pgfqpoint{-0.918138in}{0.773588in}}%
\pgfpathlineto{\pgfqpoint{-0.844408in}{0.773588in}}%
\pgfpathlineto{\pgfqpoint{-0.769987in}{0.773588in}}%
\pgfpathlineto{\pgfqpoint{-0.698122in}{0.773588in}}%
\pgfpathlineto{\pgfqpoint{-0.623932in}{0.773588in}}%
\pgfpathlineto{\pgfqpoint{-0.549722in}{0.773588in}}%
\pgfpathlineto{\pgfqpoint{-0.477784in}{0.773588in}}%
\pgfpathlineto{\pgfqpoint{-0.404497in}{0.773588in}}%
\pgfpathlineto{\pgfqpoint{-0.329323in}{0.773588in}}%
\pgfpathlineto{\pgfqpoint{-0.257459in}{0.773588in}}%
\pgfpathlineto{\pgfqpoint{-0.185623in}{0.773588in}}%
\pgfpathlineto{\pgfqpoint{-0.112821in}{0.773588in}}%
\pgfpathlineto{\pgfqpoint{-0.042097in}{0.773588in}}%
\pgfpathlineto{\pgfqpoint{0.030045in}{0.773588in}}%
\pgfpathlineto{\pgfqpoint{0.103944in}{0.773588in}}%
\pgfpathlineto{\pgfqpoint{0.176284in}{0.773588in}}%
\pgfpathlineto{\pgfqpoint{0.248641in}{0.773588in}}%
\pgfpathlineto{\pgfqpoint{0.321303in}{0.773588in}}%
\pgfpathlineto{\pgfqpoint{0.391498in}{0.773588in}}%
\pgfpathlineto{\pgfqpoint{0.462885in}{0.773588in}}%
\pgfpathlineto{\pgfqpoint{0.537447in}{0.773588in}}%
\pgfpathlineto{\pgfqpoint{0.609267in}{0.773588in}}%
\pgfpathlineto{\pgfqpoint{0.679536in}{0.773588in}}%
\pgfpathlineto{\pgfqpoint{0.753203in}{0.773588in}}%
\pgfpathlineto{\pgfqpoint{0.824780in}{0.773588in}}%
\pgfpathlineto{\pgfqpoint{0.895203in}{0.773588in}}%
\pgfpathlineto{\pgfqpoint{0.968165in}{0.773588in}}%
\pgfpathlineto{\pgfqpoint{1.038437in}{0.773588in}}%
\pgfpathlineto{\pgfqpoint{1.111472in}{0.773588in}}%
\pgfpathlineto{\pgfqpoint{1.188027in}{0.773588in}}%
\pgfpathlineto{\pgfqpoint{1.261309in}{0.773588in}}%
\pgfpathlineto{\pgfqpoint{1.335256in}{0.773588in}}%
\pgfpathlineto{\pgfqpoint{1.410931in}{0.773588in}}%
\pgfpathlineto{\pgfqpoint{1.485557in}{0.773588in}}%
\pgfpathlineto{\pgfqpoint{1.558242in}{0.773588in}}%
\pgfpathlineto{\pgfqpoint{1.633550in}{0.773588in}}%
\pgfpathlineto{\pgfqpoint{1.707271in}{0.773588in}}%
\pgfpathlineto{\pgfqpoint{1.781074in}{0.773588in}}%
\pgfpathlineto{\pgfqpoint{1.857049in}{0.773588in}}%
\pgfpathlineto{\pgfqpoint{1.931091in}{0.773588in}}%
\pgfpathlineto{\pgfqpoint{2.004776in}{0.773588in}}%
\pgfpathlineto{\pgfqpoint{2.079647in}{0.773588in}}%
\pgfpathlineto{\pgfqpoint{2.151611in}{0.773588in}}%
\pgfpathlineto{\pgfqpoint{2.224651in}{0.773588in}}%
\pgfpathlineto{\pgfqpoint{2.300156in}{0.773588in}}%
\pgfpathlineto{\pgfqpoint{2.371299in}{0.773588in}}%
\pgfpathlineto{\pgfqpoint{2.442412in}{0.773588in}}%
\pgfpathlineto{\pgfqpoint{2.515763in}{0.773588in}}%
\pgfpathlineto{\pgfqpoint{2.586109in}{0.773588in}}%
\pgfpathlineto{\pgfqpoint{2.658524in}{0.773588in}}%
\pgfpathlineto{\pgfqpoint{2.732102in}{0.773588in}}%
\pgfpathlineto{\pgfqpoint{2.802412in}{0.773588in}}%
\pgfpathlineto{\pgfqpoint{2.873367in}{0.773588in}}%
\pgfpathlineto{\pgfqpoint{2.946925in}{0.773588in}}%
\pgfpathlineto{\pgfqpoint{3.019212in}{0.773588in}}%
\pgfpathlineto{\pgfqpoint{3.091740in}{0.773588in}}%
\pgfpathlineto{\pgfqpoint{3.166494in}{0.773588in}}%
\pgfpathlineto{\pgfqpoint{3.237461in}{0.773588in}}%
\pgfpathlineto{\pgfqpoint{3.309976in}{0.773588in}}%
\pgfpathlineto{\pgfqpoint{3.384706in}{0.773588in}}%
\pgfpathlineto{\pgfqpoint{3.455601in}{0.773588in}}%
\pgfpathlineto{\pgfqpoint{3.527918in}{0.773588in}}%
\pgfpathlineto{\pgfqpoint{3.603230in}{0.773588in}}%
\pgfpathlineto{\pgfqpoint{3.674965in}{0.773588in}}%
\pgfpathlineto{\pgfqpoint{3.745938in}{0.773588in}}%
\pgfpathlineto{\pgfqpoint{3.820753in}{0.773588in}}%
\pgfpathlineto{\pgfqpoint{3.899537in}{0.773588in}}%
\pgfpathlineto{\pgfqpoint{4.022313in}{0.773588in}}%
\pgfpathlineto{\pgfqpoint{4.111286in}{0.773588in}}%
\pgfpathlineto{\pgfqpoint{4.189282in}{0.773588in}}%
\pgfpathlineto{\pgfqpoint{4.253332in}{1.724813in}}%
\pgfpathlineto{\pgfqpoint{4.318577in}{5.187514in}}%
\pgfpathlineto{\pgfqpoint{4.390596in}{5.319244in}}%
\pgfpathlineto{\pgfqpoint{4.460559in}{5.506716in}}%
\pgfpathlineto{\pgfqpoint{4.532858in}{5.399336in}}%
\pgfpathlineto{\pgfqpoint{4.602459in}{5.488048in}}%
\pgfpathlineto{\pgfqpoint{4.671388in}{5.539128in}}%
\pgfpathlineto{\pgfqpoint{4.741243in}{5.638191in}}%
\pgfpathlineto{\pgfqpoint{4.808397in}{5.619869in}}%
\pgfpathlineto{\pgfqpoint{4.875969in}{5.661488in}}%
\pgfpathlineto{\pgfqpoint{4.944850in}{5.625548in}}%
\pgfpathlineto{\pgfqpoint{5.010860in}{5.748599in}}%
\pgfpathlineto{\pgfqpoint{5.076713in}{5.732424in}}%
\pgfpathlineto{\pgfqpoint{5.144715in}{5.790749in}}%
\pgfpathlineto{\pgfqpoint{5.209656in}{5.793255in}}%
\pgfpathlineto{\pgfqpoint{5.275100in}{5.772285in}}%
\pgfpathlineto{\pgfqpoint{5.341477in}{5.875626in}}%
\pgfpathlineto{\pgfqpoint{5.405234in}{5.879596in}}%
\pgfpathlineto{\pgfqpoint{5.469981in}{5.814953in}}%
\pgfpathlineto{\pgfqpoint{5.535779in}{5.930845in}}%
\pgfpathlineto{\pgfqpoint{5.599590in}{5.872359in}}%
\pgfpathlineto{\pgfqpoint{5.599590in}{5.872359in}}%
\pgfpathlineto{\pgfqpoint{5.599590in}{5.872359in}}%
\pgfpathlineto{\pgfqpoint{5.535779in}{5.930845in}}%
\pgfpathlineto{\pgfqpoint{5.469981in}{5.814953in}}%
\pgfpathlineto{\pgfqpoint{5.405234in}{5.879596in}}%
\pgfpathlineto{\pgfqpoint{5.341477in}{5.875626in}}%
\pgfpathlineto{\pgfqpoint{5.275100in}{5.772285in}}%
\pgfpathlineto{\pgfqpoint{5.209656in}{5.793255in}}%
\pgfpathlineto{\pgfqpoint{5.144715in}{5.790749in}}%
\pgfpathlineto{\pgfqpoint{5.076713in}{5.732424in}}%
\pgfpathlineto{\pgfqpoint{5.010860in}{5.748599in}}%
\pgfpathlineto{\pgfqpoint{4.944850in}{5.625548in}}%
\pgfpathlineto{\pgfqpoint{4.875969in}{5.661488in}}%
\pgfpathlineto{\pgfqpoint{4.808397in}{5.619869in}}%
\pgfpathlineto{\pgfqpoint{4.741243in}{5.638191in}}%
\pgfpathlineto{\pgfqpoint{4.671388in}{5.539128in}}%
\pgfpathlineto{\pgfqpoint{4.602459in}{5.488048in}}%
\pgfpathlineto{\pgfqpoint{4.532858in}{5.399336in}}%
\pgfpathlineto{\pgfqpoint{4.460559in}{5.506716in}}%
\pgfpathlineto{\pgfqpoint{4.390596in}{5.319244in}}%
\pgfpathlineto{\pgfqpoint{4.318577in}{5.187514in}}%
\pgfpathlineto{\pgfqpoint{4.253332in}{1.724813in}}%
\pgfpathlineto{\pgfqpoint{4.189282in}{0.926398in}}%
\pgfpathlineto{\pgfqpoint{4.111286in}{1.497722in}}%
\pgfpathlineto{\pgfqpoint{4.022313in}{1.253992in}}%
\pgfpathlineto{\pgfqpoint{3.899537in}{1.457530in}}%
\pgfpathlineto{\pgfqpoint{3.820753in}{1.484551in}}%
\pgfpathlineto{\pgfqpoint{3.745938in}{1.575304in}}%
\pgfpathlineto{\pgfqpoint{3.674965in}{1.567696in}}%
\pgfpathlineto{\pgfqpoint{3.603230in}{1.498229in}}%
\pgfpathlineto{\pgfqpoint{3.527918in}{1.504964in}}%
\pgfpathlineto{\pgfqpoint{3.455601in}{1.565434in}}%
\pgfpathlineto{\pgfqpoint{3.384706in}{1.472512in}}%
\pgfpathlineto{\pgfqpoint{3.309976in}{1.539915in}}%
\pgfpathlineto{\pgfqpoint{3.237461in}{1.566490in}}%
\pgfpathlineto{\pgfqpoint{3.166494in}{1.510439in}}%
\pgfpathlineto{\pgfqpoint{3.091740in}{1.518304in}}%
\pgfpathlineto{\pgfqpoint{3.019212in}{1.509126in}}%
\pgfpathlineto{\pgfqpoint{2.946925in}{1.532760in}}%
\pgfpathlineto{\pgfqpoint{2.873367in}{1.536135in}}%
\pgfpathlineto{\pgfqpoint{2.802412in}{1.509098in}}%
\pgfpathlineto{\pgfqpoint{2.732102in}{1.501294in}}%
\pgfpathlineto{\pgfqpoint{2.658524in}{1.544412in}}%
\pgfpathlineto{\pgfqpoint{2.586109in}{1.579842in}}%
\pgfpathlineto{\pgfqpoint{2.515763in}{1.597485in}}%
\pgfpathlineto{\pgfqpoint{2.442412in}{1.506848in}}%
\pgfpathlineto{\pgfqpoint{2.371299in}{1.510628in}}%
\pgfpathlineto{\pgfqpoint{2.300156in}{1.523428in}}%
\pgfpathlineto{\pgfqpoint{2.224651in}{1.547229in}}%
\pgfpathlineto{\pgfqpoint{2.151611in}{1.541899in}}%
\pgfpathlineto{\pgfqpoint{2.079647in}{1.540189in}}%
\pgfpathlineto{\pgfqpoint{2.004776in}{1.493064in}}%
\pgfpathlineto{\pgfqpoint{1.931091in}{1.495765in}}%
\pgfpathlineto{\pgfqpoint{1.857049in}{1.496544in}}%
\pgfpathlineto{\pgfqpoint{1.781074in}{1.535470in}}%
\pgfpathlineto{\pgfqpoint{1.707271in}{1.540608in}}%
\pgfpathlineto{\pgfqpoint{1.633550in}{1.504590in}}%
\pgfpathlineto{\pgfqpoint{1.558242in}{1.573511in}}%
\pgfpathlineto{\pgfqpoint{1.485557in}{1.469339in}}%
\pgfpathlineto{\pgfqpoint{1.410931in}{1.535735in}}%
\pgfpathlineto{\pgfqpoint{1.335256in}{1.565013in}}%
\pgfpathlineto{\pgfqpoint{1.261309in}{1.550963in}}%
\pgfpathlineto{\pgfqpoint{1.188027in}{1.520863in}}%
\pgfpathlineto{\pgfqpoint{1.111472in}{1.514249in}}%
\pgfpathlineto{\pgfqpoint{1.038437in}{1.580135in}}%
\pgfpathlineto{\pgfqpoint{0.968165in}{1.609372in}}%
\pgfpathlineto{\pgfqpoint{0.895203in}{1.551309in}}%
\pgfpathlineto{\pgfqpoint{0.824780in}{1.548278in}}%
\pgfpathlineto{\pgfqpoint{0.753203in}{1.510067in}}%
\pgfpathlineto{\pgfqpoint{0.679536in}{1.534031in}}%
\pgfpathlineto{\pgfqpoint{0.609267in}{1.584371in}}%
\pgfpathlineto{\pgfqpoint{0.537447in}{1.475635in}}%
\pgfpathlineto{\pgfqpoint{0.462885in}{1.564938in}}%
\pgfpathlineto{\pgfqpoint{0.391498in}{1.539139in}}%
\pgfpathlineto{\pgfqpoint{0.321303in}{1.524414in}}%
\pgfpathlineto{\pgfqpoint{0.248641in}{1.526914in}}%
\pgfpathlineto{\pgfqpoint{0.176284in}{1.478907in}}%
\pgfpathlineto{\pgfqpoint{0.103944in}{1.507670in}}%
\pgfpathlineto{\pgfqpoint{0.030045in}{1.531156in}}%
\pgfpathlineto{\pgfqpoint{-0.042097in}{1.546845in}}%
\pgfpathlineto{\pgfqpoint{-0.112821in}{1.521427in}}%
\pgfpathlineto{\pgfqpoint{-0.185623in}{1.549389in}}%
\pgfpathlineto{\pgfqpoint{-0.257459in}{1.518785in}}%
\pgfpathlineto{\pgfqpoint{-0.329323in}{1.554710in}}%
\pgfpathlineto{\pgfqpoint{-0.404497in}{1.596386in}}%
\pgfpathlineto{\pgfqpoint{-0.477784in}{1.548263in}}%
\pgfpathlineto{\pgfqpoint{-0.549722in}{1.529259in}}%
\pgfpathlineto{\pgfqpoint{-0.623932in}{1.493309in}}%
\pgfpathlineto{\pgfqpoint{-0.698122in}{1.531777in}}%
\pgfpathlineto{\pgfqpoint{-0.769987in}{1.546146in}}%
\pgfpathlineto{\pgfqpoint{-0.844408in}{1.510561in}}%
\pgfpathlineto{\pgfqpoint{-0.918138in}{1.535898in}}%
\pgfpathlineto{\pgfqpoint{-0.989522in}{1.547390in}}%
\pgfpathlineto{\pgfqpoint{-1.063245in}{1.584407in}}%
\pgfpathlineto{\pgfqpoint{-1.135138in}{1.585140in}}%
\pgfpathlineto{\pgfqpoint{-1.207908in}{1.529684in}}%
\pgfpathlineto{\pgfqpoint{-1.283200in}{1.551759in}}%
\pgfpathlineto{\pgfqpoint{-1.356794in}{1.496379in}}%
\pgfpathlineto{\pgfqpoint{-1.429441in}{1.517802in}}%
\pgfpathlineto{\pgfqpoint{-1.503984in}{1.500767in}}%
\pgfpathlineto{\pgfqpoint{-1.577590in}{1.539047in}}%
\pgfpathlineto{\pgfqpoint{-1.650581in}{1.496104in}}%
\pgfpathlineto{\pgfqpoint{-1.725105in}{1.579406in}}%
\pgfpathlineto{\pgfqpoint{-1.797123in}{1.559196in}}%
\pgfpathlineto{\pgfqpoint{-1.867809in}{1.527152in}}%
\pgfpathlineto{\pgfqpoint{-1.940536in}{1.571508in}}%
\pgfpathlineto{\pgfqpoint{-2.011817in}{1.569026in}}%
\pgfpathlineto{\pgfqpoint{-2.084338in}{1.524046in}}%
\pgfpathlineto{\pgfqpoint{-2.158577in}{1.540183in}}%
\pgfpathlineto{\pgfqpoint{-2.231477in}{1.484943in}}%
\pgfpathlineto{\pgfqpoint{-2.303263in}{1.527257in}}%
\pgfpathlineto{\pgfqpoint{-2.379035in}{1.524477in}}%
\pgfpathlineto{\pgfqpoint{-2.451489in}{1.552146in}}%
\pgfpathlineto{\pgfqpoint{-2.522349in}{1.583015in}}%
\pgfpathlineto{\pgfqpoint{-2.595405in}{1.517961in}}%
\pgfpathlineto{\pgfqpoint{-2.667020in}{1.469280in}}%
\pgfpathlineto{\pgfqpoint{-2.740325in}{1.517836in}}%
\pgfpathlineto{\pgfqpoint{-2.813631in}{1.560384in}}%
\pgfpathlineto{\pgfqpoint{-2.884033in}{1.504888in}}%
\pgfpathlineto{\pgfqpoint{-2.956196in}{1.503924in}}%
\pgfpathlineto{\pgfqpoint{-3.029702in}{1.549251in}}%
\pgfpathlineto{\pgfqpoint{-3.100293in}{1.448544in}}%
\pgfpathlineto{\pgfqpoint{-3.171735in}{1.548452in}}%
\pgfpathlineto{\pgfqpoint{-3.245632in}{1.510336in}}%
\pgfpathlineto{\pgfqpoint{-3.317677in}{1.521556in}}%
\pgfpathlineto{\pgfqpoint{-3.388885in}{1.544437in}}%
\pgfpathlineto{\pgfqpoint{-3.461876in}{1.569511in}}%
\pgfpathlineto{\pgfqpoint{-3.534002in}{1.509844in}}%
\pgfpathlineto{\pgfqpoint{-3.608325in}{1.468393in}}%
\pgfpathlineto{\pgfqpoint{-3.684119in}{1.459063in}}%
\pgfpathlineto{\pgfqpoint{-3.756893in}{1.534409in}}%
\pgfpathlineto{\pgfqpoint{-3.828230in}{1.492647in}}%
\pgfpathlineto{\pgfqpoint{-3.903164in}{1.464448in}}%
\pgfpathlineto{\pgfqpoint{-3.977308in}{1.520586in}}%
\pgfpathlineto{\pgfqpoint{-4.052173in}{1.481939in}}%
\pgfpathlineto{\pgfqpoint{-4.128352in}{1.532911in}}%
\pgfpathlineto{\pgfqpoint{-4.201246in}{1.474921in}}%
\pgfpathlineto{\pgfqpoint{-4.274748in}{1.495423in}}%
\pgfpathlineto{\pgfqpoint{-4.350759in}{1.536899in}}%
\pgfpathlineto{\pgfqpoint{-4.423289in}{1.517289in}}%
\pgfpathlineto{\pgfqpoint{-4.495401in}{1.590285in}}%
\pgfpathlineto{\pgfqpoint{-4.568595in}{1.507448in}}%
\pgfpathlineto{\pgfqpoint{-4.639570in}{1.508803in}}%
\pgfpathlineto{\pgfqpoint{-4.711002in}{1.531143in}}%
\pgfpathlineto{\pgfqpoint{-4.782965in}{1.587070in}}%
\pgfpathlineto{\pgfqpoint{-4.852652in}{1.536318in}}%
\pgfpathlineto{\pgfqpoint{-4.922664in}{1.505684in}}%
\pgfpathlineto{\pgfqpoint{-4.995002in}{1.577246in}}%
\pgfpathlineto{\pgfqpoint{-5.064817in}{1.579015in}}%
\pgfpathlineto{\pgfqpoint{-5.135005in}{1.554414in}}%
\pgfpathlineto{\pgfqpoint{-5.207317in}{1.577262in}}%
\pgfpathlineto{\pgfqpoint{-5.277970in}{1.546213in}}%
\pgfpathlineto{\pgfqpoint{-5.348665in}{1.531879in}}%
\pgfpathlineto{\pgfqpoint{-5.421890in}{1.485416in}}%
\pgfpathlineto{\pgfqpoint{-5.493615in}{1.532726in}}%
\pgfpathlineto{\pgfqpoint{-5.563648in}{1.637698in}}%
\pgfpathlineto{\pgfqpoint{-5.635685in}{1.552118in}}%
\pgfpathlineto{\pgfqpoint{-5.706570in}{1.553032in}}%
\pgfpathlineto{\pgfqpoint{-5.776542in}{1.540403in}}%
\pgfpathlineto{\pgfqpoint{-5.849895in}{1.515995in}}%
\pgfpathlineto{\pgfqpoint{-5.921215in}{1.543851in}}%
\pgfpathlineto{\pgfqpoint{-5.992127in}{1.507083in}}%
\pgfpathlineto{\pgfqpoint{-6.064664in}{1.559183in}}%
\pgfpathlineto{\pgfqpoint{-6.135189in}{1.598818in}}%
\pgfpathlineto{\pgfqpoint{-6.205113in}{1.429075in}}%
\pgfpathlineto{\pgfqpoint{-6.278887in}{1.578390in}}%
\pgfpathlineto{\pgfqpoint{-6.349020in}{1.548758in}}%
\pgfpathlineto{\pgfqpoint{-6.421127in}{1.493167in}}%
\pgfpathlineto{\pgfqpoint{-6.496669in}{1.501851in}}%
\pgfpathlineto{\pgfqpoint{-6.568712in}{1.542142in}}%
\pgfpathlineto{\pgfqpoint{-6.640496in}{1.502131in}}%
\pgfpathlineto{\pgfqpoint{-6.715245in}{1.503712in}}%
\pgfpathlineto{\pgfqpoint{-6.787495in}{1.482019in}}%
\pgfpathlineto{\pgfqpoint{-6.860576in}{1.550760in}}%
\pgfpathlineto{\pgfqpoint{-6.933558in}{1.593510in}}%
\pgfpathlineto{\pgfqpoint{-7.003598in}{1.522700in}}%
\pgfpathlineto{\pgfqpoint{-7.075897in}{1.496645in}}%
\pgfpathlineto{\pgfqpoint{-7.151909in}{1.524874in}}%
\pgfpathlineto{\pgfqpoint{-7.223147in}{1.515672in}}%
\pgfpathlineto{\pgfqpoint{-7.294929in}{1.551779in}}%
\pgfpathlineto{\pgfqpoint{-7.368338in}{1.573078in}}%
\pgfpathlineto{\pgfqpoint{-7.440452in}{1.480083in}}%
\pgfpathlineto{\pgfqpoint{-7.512981in}{1.583860in}}%
\pgfpathlineto{\pgfqpoint{-7.585515in}{1.482405in}}%
\pgfpathlineto{\pgfqpoint{-7.657024in}{1.559116in}}%
\pgfpathlineto{\pgfqpoint{-7.727286in}{1.565118in}}%
\pgfpathlineto{\pgfqpoint{-7.799875in}{1.598700in}}%
\pgfpathlineto{\pgfqpoint{-7.870788in}{1.532899in}}%
\pgfpathlineto{\pgfqpoint{-7.942200in}{1.604120in}}%
\pgfpathlineto{\pgfqpoint{-8.014387in}{1.584190in}}%
\pgfpathlineto{\pgfqpoint{-8.084913in}{1.504832in}}%
\pgfpathlineto{\pgfqpoint{-8.156111in}{1.546275in}}%
\pgfpathlineto{\pgfqpoint{-8.228241in}{1.506883in}}%
\pgfpathlineto{\pgfqpoint{-8.298753in}{1.547590in}}%
\pgfpathlineto{\pgfqpoint{-8.367967in}{1.562385in}}%
\pgfpathlineto{\pgfqpoint{-8.438970in}{1.521624in}}%
\pgfpathlineto{\pgfqpoint{-8.508839in}{1.575295in}}%
\pgfpathlineto{\pgfqpoint{-8.578369in}{1.560943in}}%
\pgfpathlineto{\pgfqpoint{-8.649341in}{1.599819in}}%
\pgfpathlineto{\pgfqpoint{-8.718609in}{1.614311in}}%
\pgfpathlineto{\pgfqpoint{-8.787586in}{1.627833in}}%
\pgfpathlineto{\pgfqpoint{-8.858221in}{1.591324in}}%
\pgfpathlineto{\pgfqpoint{-8.928123in}{1.585803in}}%
\pgfpathlineto{\pgfqpoint{-8.998701in}{1.587263in}}%
\pgfpathlineto{\pgfqpoint{-9.069721in}{1.558749in}}%
\pgfpathlineto{\pgfqpoint{-9.138788in}{1.500385in}}%
\pgfpathlineto{\pgfqpoint{-9.208906in}{1.557458in}}%
\pgfpathlineto{\pgfqpoint{-9.283412in}{1.501061in}}%
\pgfpathlineto{\pgfqpoint{-9.356529in}{1.470709in}}%
\pgfpathlineto{\pgfqpoint{-9.429774in}{1.531987in}}%
\pgfpathlineto{\pgfqpoint{-9.504196in}{1.514715in}}%
\pgfpathlineto{\pgfqpoint{-9.575400in}{1.478797in}}%
\pgfpathlineto{\pgfqpoint{-9.648369in}{1.524106in}}%
\pgfpathlineto{\pgfqpoint{-9.723682in}{1.545626in}}%
\pgfpathlineto{\pgfqpoint{-9.796010in}{1.500466in}}%
\pgfpathlineto{\pgfqpoint{-9.868243in}{1.562266in}}%
\pgfpathlineto{\pgfqpoint{-9.941359in}{1.535035in}}%
\pgfpathlineto{\pgfqpoint{-10.012635in}{1.476574in}}%
\pgfpathlineto{\pgfqpoint{-10.084496in}{1.536204in}}%
\pgfpathlineto{\pgfqpoint{-10.157398in}{1.545125in}}%
\pgfpathlineto{\pgfqpoint{-10.227739in}{1.540504in}}%
\pgfpathlineto{\pgfqpoint{-10.297591in}{1.536177in}}%
\pgfpathlineto{\pgfqpoint{-10.371004in}{1.498560in}}%
\pgfpathlineto{\pgfqpoint{-10.441581in}{1.507756in}}%
\pgfpathlineto{\pgfqpoint{-10.511372in}{1.557813in}}%
\pgfpathlineto{\pgfqpoint{-10.582864in}{1.475010in}}%
\pgfpathlineto{\pgfqpoint{-10.652649in}{1.557287in}}%
\pgfpathlineto{\pgfqpoint{-10.721651in}{1.600105in}}%
\pgfpathlineto{\pgfqpoint{-10.793422in}{1.518855in}}%
\pgfpathlineto{\pgfqpoint{-10.863257in}{1.523452in}}%
\pgfpathlineto{\pgfqpoint{-10.933808in}{1.575582in}}%
\pgfpathlineto{\pgfqpoint{-11.005065in}{1.479274in}}%
\pgfpathlineto{\pgfqpoint{-11.075276in}{1.568820in}}%
\pgfpathlineto{\pgfqpoint{-11.144960in}{1.589136in}}%
\pgfpathlineto{\pgfqpoint{-11.217631in}{1.512026in}}%
\pgfpathlineto{\pgfqpoint{-11.288273in}{1.553784in}}%
\pgfpathlineto{\pgfqpoint{-11.358676in}{1.528517in}}%
\pgfpathlineto{\pgfqpoint{-11.430351in}{1.569264in}}%
\pgfpathlineto{\pgfqpoint{-11.500166in}{1.524114in}}%
\pgfpathlineto{\pgfqpoint{-11.571037in}{1.576325in}}%
\pgfpathlineto{\pgfqpoint{-11.645187in}{1.535340in}}%
\pgfpathlineto{\pgfqpoint{-11.716309in}{1.578664in}}%
\pgfpathlineto{\pgfqpoint{-11.786176in}{1.590515in}}%
\pgfpathlineto{\pgfqpoint{-11.858471in}{1.551767in}}%
\pgfpathlineto{\pgfqpoint{-11.928381in}{1.582011in}}%
\pgfpathlineto{\pgfqpoint{-11.998596in}{1.547421in}}%
\pgfpathlineto{\pgfqpoint{-12.069446in}{1.558572in}}%
\pgfpathlineto{\pgfqpoint{-12.139885in}{1.526749in}}%
\pgfpathlineto{\pgfqpoint{-12.211536in}{1.530595in}}%
\pgfpathlineto{\pgfqpoint{-12.285275in}{1.484903in}}%
\pgfpathlineto{\pgfqpoint{-12.357103in}{1.542659in}}%
\pgfpathlineto{\pgfqpoint{-12.428997in}{1.474807in}}%
\pgfpathlineto{\pgfqpoint{-12.503401in}{1.558845in}}%
\pgfpathlineto{\pgfqpoint{-12.574652in}{1.536842in}}%
\pgfpathlineto{\pgfqpoint{-12.645279in}{1.556952in}}%
\pgfpathlineto{\pgfqpoint{-12.718096in}{1.497654in}}%
\pgfpathlineto{\pgfqpoint{-12.788679in}{1.548196in}}%
\pgfpathlineto{\pgfqpoint{-12.859398in}{1.573012in}}%
\pgfpathlineto{\pgfqpoint{-12.932388in}{1.522647in}}%
\pgfpathlineto{\pgfqpoint{-13.002727in}{1.554154in}}%
\pgfpathlineto{\pgfqpoint{-13.073093in}{1.530036in}}%
\pgfpathlineto{\pgfqpoint{-13.145337in}{1.576170in}}%
\pgfpathlineto{\pgfqpoint{-13.216013in}{1.592736in}}%
\pgfpathlineto{\pgfqpoint{-13.285456in}{1.551346in}}%
\pgfpathlineto{\pgfqpoint{-13.356968in}{1.555759in}}%
\pgfpathlineto{\pgfqpoint{-13.426276in}{1.550138in}}%
\pgfpathlineto{\pgfqpoint{-13.496151in}{1.567174in}}%
\pgfpathlineto{\pgfqpoint{-13.568713in}{1.490930in}}%
\pgfpathlineto{\pgfqpoint{-13.639609in}{1.606337in}}%
\pgfpathlineto{\pgfqpoint{-13.708089in}{1.568296in}}%
\pgfpathlineto{\pgfqpoint{-13.779339in}{1.531912in}}%
\pgfpathlineto{\pgfqpoint{-13.848756in}{1.569518in}}%
\pgfpathlineto{\pgfqpoint{-13.917455in}{1.532540in}}%
\pgfpathlineto{\pgfqpoint{-13.988983in}{1.551407in}}%
\pgfpathlineto{\pgfqpoint{-14.057377in}{1.566279in}}%
\pgfpathlineto{\pgfqpoint{-14.125666in}{1.614741in}}%
\pgfpathlineto{\pgfqpoint{-14.195714in}{1.556708in}}%
\pgfpathlineto{\pgfqpoint{-14.263436in}{1.576164in}}%
\pgfpathlineto{\pgfqpoint{-14.331277in}{1.520709in}}%
\pgfpathlineto{\pgfqpoint{-14.401957in}{1.549893in}}%
\pgfpathlineto{\pgfqpoint{-14.471123in}{1.607636in}}%
\pgfpathlineto{\pgfqpoint{-14.538988in}{1.579836in}}%
\pgfpathlineto{\pgfqpoint{-14.609938in}{1.530455in}}%
\pgfpathlineto{\pgfqpoint{-14.679263in}{1.567216in}}%
\pgfpathlineto{\pgfqpoint{-14.748835in}{1.535742in}}%
\pgfpathlineto{\pgfqpoint{-14.820408in}{1.545614in}}%
\pgfpathlineto{\pgfqpoint{-14.889438in}{1.576556in}}%
\pgfpathlineto{\pgfqpoint{-14.959865in}{1.568175in}}%
\pgfpathlineto{\pgfqpoint{-15.032843in}{1.512666in}}%
\pgfpathlineto{\pgfqpoint{-15.106256in}{1.479184in}}%
\pgfpathlineto{\pgfqpoint{-15.179182in}{1.518566in}}%
\pgfpathlineto{\pgfqpoint{-15.253001in}{1.547781in}}%
\pgfpathlineto{\pgfqpoint{-15.323000in}{1.588895in}}%
\pgfpathlineto{\pgfqpoint{-15.392048in}{1.615478in}}%
\pgfpathlineto{\pgfqpoint{-15.463558in}{1.562144in}}%
\pgfpathlineto{\pgfqpoint{-15.532639in}{1.533442in}}%
\pgfpathlineto{\pgfqpoint{-15.602559in}{1.570939in}}%
\pgfpathlineto{\pgfqpoint{-15.673559in}{1.537539in}}%
\pgfpathlineto{\pgfqpoint{-15.744769in}{1.558776in}}%
\pgfpathlineto{\pgfqpoint{-15.814335in}{1.546544in}}%
\pgfpathlineto{\pgfqpoint{-15.886757in}{1.485292in}}%
\pgfpathlineto{\pgfqpoint{-15.956358in}{1.576527in}}%
\pgfpathlineto{\pgfqpoint{-16.024276in}{1.591577in}}%
\pgfpathlineto{\pgfqpoint{-16.094895in}{1.518712in}}%
\pgfpathlineto{\pgfqpoint{-16.163654in}{1.568504in}}%
\pgfpathlineto{\pgfqpoint{-16.232068in}{1.594538in}}%
\pgfpathlineto{\pgfqpoint{-16.302639in}{1.556145in}}%
\pgfpathlineto{\pgfqpoint{-16.369724in}{1.625603in}}%
\pgfpathlineto{\pgfqpoint{-16.437097in}{1.491634in}}%
\pgfpathlineto{\pgfqpoint{-16.507797in}{1.563018in}}%
\pgfpathlineto{\pgfqpoint{-16.575864in}{1.574084in}}%
\pgfpathlineto{\pgfqpoint{-16.644038in}{1.487275in}}%
\pgfpathlineto{\pgfqpoint{-16.715630in}{1.555255in}}%
\pgfpathlineto{\pgfqpoint{-16.784012in}{1.564815in}}%
\pgfpathlineto{\pgfqpoint{-16.852234in}{1.560771in}}%
\pgfpathlineto{\pgfqpoint{-16.921714in}{1.647021in}}%
\pgfpathlineto{\pgfqpoint{-16.989953in}{1.585704in}}%
\pgfpathlineto{\pgfqpoint{-17.058127in}{1.591320in}}%
\pgfpathlineto{\pgfqpoint{-17.128098in}{1.612428in}}%
\pgfpathlineto{\pgfqpoint{-17.196258in}{1.562231in}}%
\pgfpathlineto{\pgfqpoint{-17.265552in}{1.518898in}}%
\pgfpathlineto{\pgfqpoint{-17.336134in}{1.565625in}}%
\pgfpathlineto{\pgfqpoint{-17.402675in}{1.603040in}}%
\pgfpathlineto{\pgfqpoint{-17.470673in}{1.596280in}}%
\pgfpathlineto{\pgfqpoint{-17.539598in}{1.516863in}}%
\pgfpathlineto{\pgfqpoint{-17.607453in}{1.625153in}}%
\pgfpathlineto{\pgfqpoint{-17.675451in}{1.533353in}}%
\pgfpathlineto{\pgfqpoint{-17.748008in}{1.555707in}}%
\pgfpathlineto{\pgfqpoint{-17.816737in}{1.590527in}}%
\pgfpathlineto{\pgfqpoint{-17.885356in}{1.566668in}}%
\pgfpathlineto{\pgfqpoint{-17.957185in}{1.554583in}}%
\pgfpathlineto{\pgfqpoint{-18.025876in}{1.562836in}}%
\pgfpathlineto{\pgfqpoint{-18.094404in}{1.561703in}}%
\pgfpathlineto{\pgfqpoint{-18.165145in}{1.577859in}}%
\pgfpathlineto{\pgfqpoint{-18.232823in}{1.553188in}}%
\pgfpathlineto{\pgfqpoint{-18.301710in}{1.514976in}}%
\pgfpathlineto{\pgfqpoint{-18.373957in}{1.605883in}}%
\pgfpathlineto{\pgfqpoint{-18.442527in}{1.539729in}}%
\pgfpathlineto{\pgfqpoint{-18.511159in}{1.522245in}}%
\pgfpathlineto{\pgfqpoint{-18.583201in}{1.532065in}}%
\pgfpathlineto{\pgfqpoint{-18.652004in}{1.588449in}}%
\pgfpathlineto{\pgfqpoint{-18.721737in}{1.556532in}}%
\pgfpathlineto{\pgfqpoint{-18.792270in}{1.597326in}}%
\pgfpathlineto{\pgfqpoint{-18.859965in}{1.569592in}}%
\pgfpathlineto{\pgfqpoint{-18.928709in}{1.546145in}}%
\pgfpathlineto{\pgfqpoint{-18.999640in}{1.583495in}}%
\pgfpathlineto{\pgfqpoint{-19.068147in}{1.553558in}}%
\pgfpathlineto{\pgfqpoint{-19.135788in}{1.590858in}}%
\pgfpathlineto{\pgfqpoint{-19.205228in}{1.582946in}}%
\pgfpathlineto{\pgfqpoint{-19.271476in}{1.595843in}}%
\pgfpathlineto{\pgfqpoint{-19.338364in}{1.548130in}}%
\pgfpathlineto{\pgfqpoint{-19.407422in}{1.548198in}}%
\pgfpathlineto{\pgfqpoint{-19.475599in}{1.576044in}}%
\pgfpathlineto{\pgfqpoint{-19.543393in}{1.542631in}}%
\pgfpathlineto{\pgfqpoint{-19.613839in}{1.530226in}}%
\pgfpathlineto{\pgfqpoint{-19.680997in}{1.623511in}}%
\pgfpathlineto{\pgfqpoint{-19.747542in}{1.609101in}}%
\pgfpathlineto{\pgfqpoint{-19.815535in}{1.591494in}}%
\pgfpathlineto{\pgfqpoint{-19.882267in}{1.591651in}}%
\pgfpathlineto{\pgfqpoint{-19.949841in}{1.588063in}}%
\pgfpathlineto{\pgfqpoint{-20.019359in}{1.538342in}}%
\pgfpathlineto{\pgfqpoint{-20.086943in}{1.651022in}}%
\pgfpathlineto{\pgfqpoint{-20.154101in}{1.568638in}}%
\pgfpathlineto{\pgfqpoint{-20.223560in}{1.640298in}}%
\pgfpathlineto{\pgfqpoint{-20.290460in}{1.617115in}}%
\pgfpathlineto{\pgfqpoint{-20.357252in}{1.563569in}}%
\pgfpathlineto{\pgfqpoint{-20.427103in}{1.562163in}}%
\pgfpathlineto{\pgfqpoint{-20.496847in}{1.597329in}}%
\pgfpathlineto{\pgfqpoint{-20.565295in}{1.616858in}}%
\pgfpathlineto{\pgfqpoint{-20.636228in}{1.558010in}}%
\pgfpathlineto{\pgfqpoint{-20.705385in}{1.584449in}}%
\pgfpathlineto{\pgfqpoint{-20.774312in}{1.672266in}}%
\pgfpathlineto{\pgfqpoint{-20.844220in}{1.559245in}}%
\pgfpathlineto{\pgfqpoint{-20.913231in}{1.565727in}}%
\pgfpathlineto{\pgfqpoint{-20.982338in}{1.560845in}}%
\pgfpathlineto{\pgfqpoint{-21.053709in}{1.575255in}}%
\pgfpathlineto{\pgfqpoint{-21.123238in}{1.601026in}}%
\pgfpathlineto{\pgfqpoint{-21.191384in}{1.609879in}}%
\pgfpathlineto{\pgfqpoint{-21.261541in}{1.594108in}}%
\pgfpathlineto{\pgfqpoint{-21.330134in}{1.577171in}}%
\pgfpathlineto{\pgfqpoint{-21.398346in}{1.577726in}}%
\pgfpathlineto{\pgfqpoint{-21.470634in}{1.579021in}}%
\pgfpathlineto{\pgfqpoint{-21.539549in}{1.571830in}}%
\pgfpathlineto{\pgfqpoint{-21.607622in}{1.607414in}}%
\pgfpathlineto{\pgfqpoint{-21.676089in}{1.600879in}}%
\pgfpathlineto{\pgfqpoint{-21.743422in}{1.632972in}}%
\pgfpathlineto{\pgfqpoint{-21.811964in}{1.514664in}}%
\pgfpathlineto{\pgfqpoint{-21.881469in}{1.605175in}}%
\pgfpathlineto{\pgfqpoint{-21.948488in}{1.530400in}}%
\pgfpathlineto{\pgfqpoint{-22.016091in}{1.584048in}}%
\pgfpathlineto{\pgfqpoint{-22.084923in}{1.669498in}}%
\pgfpathlineto{\pgfqpoint{-22.150098in}{1.616851in}}%
\pgfpathlineto{\pgfqpoint{-22.217508in}{1.556183in}}%
\pgfpathlineto{\pgfqpoint{-22.286974in}{1.608840in}}%
\pgfpathlineto{\pgfqpoint{-22.353759in}{1.594816in}}%
\pgfpathlineto{\pgfqpoint{-22.421718in}{1.567871in}}%
\pgfpathlineto{\pgfqpoint{-22.492630in}{1.530704in}}%
\pgfpathlineto{\pgfqpoint{-22.560590in}{1.616904in}}%
\pgfpathlineto{\pgfqpoint{-22.627376in}{1.650890in}}%
\pgfpathlineto{\pgfqpoint{-22.696138in}{1.573738in}}%
\pgfpathlineto{\pgfqpoint{-22.764599in}{1.628773in}}%
\pgfpathlineto{\pgfqpoint{-22.831621in}{1.653656in}}%
\pgfpathlineto{\pgfqpoint{-22.900479in}{1.528788in}}%
\pgfpathlineto{\pgfqpoint{-22.968635in}{1.564719in}}%
\pgfpathlineto{\pgfqpoint{-23.037151in}{1.561282in}}%
\pgfpathlineto{\pgfqpoint{-23.107951in}{1.618288in}}%
\pgfpathlineto{\pgfqpoint{-23.175783in}{1.528210in}}%
\pgfpathlineto{\pgfqpoint{-23.243519in}{1.612249in}}%
\pgfpathlineto{\pgfqpoint{-23.314406in}{1.525261in}}%
\pgfpathlineto{\pgfqpoint{-23.383303in}{1.584183in}}%
\pgfpathlineto{\pgfqpoint{-23.451780in}{1.567280in}}%
\pgfpathlineto{\pgfqpoint{-23.523979in}{1.528063in}}%
\pgfpathlineto{\pgfqpoint{-23.594251in}{1.588459in}}%
\pgfpathlineto{\pgfqpoint{-23.664795in}{1.543653in}}%
\pgfpathlineto{\pgfqpoint{-23.739951in}{1.502950in}}%
\pgfpathlineto{\pgfqpoint{-23.810815in}{1.595614in}}%
\pgfpathlineto{\pgfqpoint{-23.881105in}{1.558132in}}%
\pgfpathlineto{\pgfqpoint{-23.953546in}{1.571863in}}%
\pgfpathlineto{\pgfqpoint{-24.025129in}{1.536944in}}%
\pgfpathlineto{\pgfqpoint{-24.096931in}{1.513495in}}%
\pgfpathlineto{\pgfqpoint{-24.172427in}{1.509290in}}%
\pgfpathlineto{\pgfqpoint{-24.243942in}{1.475877in}}%
\pgfpathlineto{\pgfqpoint{-24.313406in}{1.570075in}}%
\pgfpathlineto{\pgfqpoint{-24.381203in}{1.649474in}}%
\pgfpathlineto{\pgfqpoint{-24.445935in}{1.620571in}}%
\pgfpathlineto{\pgfqpoint{-24.511431in}{1.612247in}}%
\pgfpathlineto{\pgfqpoint{-24.578904in}{1.623196in}}%
\pgfpathlineto{\pgfqpoint{-24.645693in}{1.609310in}}%
\pgfpathlineto{\pgfqpoint{-24.712889in}{1.618981in}}%
\pgfpathlineto{\pgfqpoint{-24.781655in}{1.588416in}}%
\pgfpathlineto{\pgfqpoint{-24.847966in}{1.573417in}}%
\pgfpathlineto{\pgfqpoint{-24.916019in}{1.538703in}}%
\pgfpathlineto{\pgfqpoint{-24.984232in}{1.636712in}}%
\pgfpathlineto{\pgfqpoint{-25.049351in}{1.600212in}}%
\pgfpathlineto{\pgfqpoint{-25.114674in}{1.667753in}}%
\pgfpathlineto{\pgfqpoint{-25.182394in}{1.645747in}}%
\pgfpathlineto{\pgfqpoint{-25.248757in}{1.572865in}}%
\pgfpathlineto{\pgfqpoint{-25.315375in}{1.638834in}}%
\pgfpathlineto{\pgfqpoint{-25.382786in}{1.620713in}}%
\pgfpathlineto{\pgfqpoint{-25.448976in}{1.567223in}}%
\pgfpathlineto{\pgfqpoint{-25.515370in}{1.596194in}}%
\pgfpathlineto{\pgfqpoint{-25.583742in}{1.598652in}}%
\pgfpathlineto{\pgfqpoint{-25.650614in}{1.585460in}}%
\pgfpathlineto{\pgfqpoint{-25.717568in}{1.669465in}}%
\pgfpathlineto{\pgfqpoint{-25.785528in}{1.591992in}}%
\pgfpathlineto{\pgfqpoint{-25.851769in}{1.635448in}}%
\pgfpathlineto{\pgfqpoint{-25.919100in}{1.531670in}}%
\pgfpathlineto{\pgfqpoint{-25.990773in}{1.566624in}}%
\pgfpathlineto{\pgfqpoint{-26.059544in}{1.581814in}}%
\pgfpathlineto{\pgfqpoint{-26.128033in}{1.539101in}}%
\pgfpathlineto{\pgfqpoint{-26.198294in}{1.546209in}}%
\pgfpathlineto{\pgfqpoint{-26.268985in}{1.578277in}}%
\pgfpathlineto{\pgfqpoint{-26.338662in}{1.564353in}}%
\pgfpathlineto{\pgfqpoint{-26.410031in}{1.514728in}}%
\pgfpathlineto{\pgfqpoint{-26.479146in}{1.586937in}}%
\pgfpathlineto{\pgfqpoint{-26.547093in}{1.541610in}}%
\pgfpathlineto{\pgfqpoint{-26.616985in}{1.543441in}}%
\pgfpathlineto{\pgfqpoint{-26.685925in}{1.551953in}}%
\pgfpathlineto{\pgfqpoint{-26.755672in}{1.554831in}}%
\pgfpathlineto{\pgfqpoint{-26.825734in}{1.646353in}}%
\pgfpathlineto{\pgfqpoint{-26.893662in}{1.553886in}}%
\pgfpathlineto{\pgfqpoint{-26.961733in}{1.619428in}}%
\pgfpathlineto{\pgfqpoint{-27.031677in}{1.573513in}}%
\pgfpathlineto{\pgfqpoint{-27.097972in}{1.573094in}}%
\pgfpathlineto{\pgfqpoint{-27.165061in}{1.602589in}}%
\pgfpathlineto{\pgfqpoint{-27.232937in}{1.631713in}}%
\pgfpathlineto{\pgfqpoint{-27.299189in}{1.597809in}}%
\pgfpathlineto{\pgfqpoint{-27.366087in}{1.577993in}}%
\pgfpathlineto{\pgfqpoint{-27.435671in}{1.541750in}}%
\pgfpathlineto{\pgfqpoint{-27.503250in}{1.536941in}}%
\pgfpathlineto{\pgfqpoint{-27.570291in}{1.579845in}}%
\pgfpathlineto{\pgfqpoint{-27.637925in}{1.648297in}}%
\pgfpathlineto{\pgfqpoint{-27.703951in}{1.555661in}}%
\pgfpathlineto{\pgfqpoint{-27.771032in}{1.641367in}}%
\pgfpathlineto{\pgfqpoint{-27.839702in}{1.559422in}}%
\pgfpathlineto{\pgfqpoint{-27.908279in}{1.549270in}}%
\pgfpathlineto{\pgfqpoint{-27.976694in}{1.615504in}}%
\pgfpathlineto{\pgfqpoint{-28.049197in}{1.512115in}}%
\pgfpathlineto{\pgfqpoint{-28.121663in}{1.518470in}}%
\pgfpathlineto{\pgfqpoint{-28.195772in}{1.481625in}}%
\pgfpathlineto{\pgfqpoint{-28.274638in}{1.460074in}}%
\pgfpathlineto{\pgfqpoint{-28.356803in}{1.535981in}}%
\pgfpathlineto{\pgfqpoint{-28.429123in}{1.539362in}}%
\pgfpathlineto{\pgfqpoint{-28.504596in}{1.497556in}}%
\pgfpathlineto{\pgfqpoint{-28.577221in}{1.577858in}}%
\pgfpathlineto{\pgfqpoint{-28.649563in}{1.503915in}}%
\pgfpathlineto{\pgfqpoint{-28.725736in}{1.511302in}}%
\pgfpathlineto{\pgfqpoint{-28.798746in}{1.526723in}}%
\pgfpathlineto{\pgfqpoint{-28.870906in}{1.508459in}}%
\pgfpathlineto{\pgfqpoint{-28.946213in}{1.539689in}}%
\pgfpathlineto{\pgfqpoint{-29.018119in}{1.571750in}}%
\pgfpathlineto{\pgfqpoint{-29.089467in}{1.533021in}}%
\pgfpathlineto{\pgfqpoint{-29.164288in}{1.542608in}}%
\pgfpathlineto{\pgfqpoint{-29.235084in}{1.565343in}}%
\pgfpathlineto{\pgfqpoint{-29.304065in}{1.557431in}}%
\pgfpathlineto{\pgfqpoint{-29.374900in}{1.515214in}}%
\pgfpathlineto{\pgfqpoint{-29.443353in}{1.600714in}}%
\pgfpathlineto{\pgfqpoint{-29.511058in}{1.615811in}}%
\pgfpathlineto{\pgfqpoint{-29.581285in}{1.597714in}}%
\pgfpathlineto{\pgfqpoint{-29.647778in}{1.607168in}}%
\pgfpathlineto{\pgfqpoint{-29.716346in}{1.572583in}}%
\pgfpathlineto{\pgfqpoint{-29.785834in}{1.607225in}}%
\pgfpathlineto{\pgfqpoint{-29.853575in}{1.543801in}}%
\pgfpathlineto{\pgfqpoint{-29.923374in}{1.552264in}}%
\pgfpathlineto{\pgfqpoint{-29.994776in}{1.605381in}}%
\pgfpathlineto{\pgfqpoint{-30.062931in}{1.586306in}}%
\pgfpathlineto{\pgfqpoint{-30.131142in}{1.553948in}}%
\pgfpathlineto{\pgfqpoint{-30.202864in}{1.576150in}}%
\pgfpathlineto{\pgfqpoint{-30.269201in}{1.605899in}}%
\pgfpathlineto{\pgfqpoint{-30.335915in}{1.566196in}}%
\pgfpathlineto{\pgfqpoint{-30.404868in}{1.515255in}}%
\pgfpathlineto{\pgfqpoint{-30.472294in}{1.615813in}}%
\pgfpathlineto{\pgfqpoint{-30.539383in}{1.550858in}}%
\pgfpathlineto{\pgfqpoint{-30.610127in}{1.586095in}}%
\pgfpathlineto{\pgfqpoint{-30.676901in}{1.613006in}}%
\pgfpathlineto{\pgfqpoint{-30.745143in}{1.307859in}}%
\pgfpathclose%
\pgfusepath{fill}%
\end{pgfscope}%
\begin{pgfscope}%
\pgfpathrectangle{\pgfqpoint{3.332180in}{0.773588in}}{\pgfqpoint{2.293918in}{5.415119in}}%
\pgfusepath{clip}%
\pgfsetbuttcap%
\pgfsetroundjoin%
\definecolor{currentfill}{rgb}{0.580392,0.403922,0.741176}%
\pgfsetfillcolor{currentfill}%
\pgfsetlinewidth{0.000000pt}%
\definecolor{currentstroke}{rgb}{0.000000,0.000000,0.000000}%
\pgfsetstrokecolor{currentstroke}%
\pgfsetdash{}{0pt}%
\pgfpathmoveto{\pgfqpoint{-30.745143in}{1.768207in}}%
\pgfpathlineto{\pgfqpoint{-30.745143in}{1.307859in}}%
\pgfpathlineto{\pgfqpoint{-30.676901in}{1.613006in}}%
\pgfpathlineto{\pgfqpoint{-30.610127in}{1.586095in}}%
\pgfpathlineto{\pgfqpoint{-30.539383in}{1.550858in}}%
\pgfpathlineto{\pgfqpoint{-30.472294in}{1.615813in}}%
\pgfpathlineto{\pgfqpoint{-30.404868in}{1.515255in}}%
\pgfpathlineto{\pgfqpoint{-30.335915in}{1.566196in}}%
\pgfpathlineto{\pgfqpoint{-30.269201in}{1.605899in}}%
\pgfpathlineto{\pgfqpoint{-30.202864in}{1.576150in}}%
\pgfpathlineto{\pgfqpoint{-30.131142in}{1.553948in}}%
\pgfpathlineto{\pgfqpoint{-30.062931in}{1.586306in}}%
\pgfpathlineto{\pgfqpoint{-29.994776in}{1.605381in}}%
\pgfpathlineto{\pgfqpoint{-29.923374in}{1.552264in}}%
\pgfpathlineto{\pgfqpoint{-29.853575in}{1.543801in}}%
\pgfpathlineto{\pgfqpoint{-29.785834in}{1.607225in}}%
\pgfpathlineto{\pgfqpoint{-29.716346in}{1.572583in}}%
\pgfpathlineto{\pgfqpoint{-29.647778in}{1.607168in}}%
\pgfpathlineto{\pgfqpoint{-29.581285in}{1.597714in}}%
\pgfpathlineto{\pgfqpoint{-29.511058in}{1.615811in}}%
\pgfpathlineto{\pgfqpoint{-29.443353in}{1.600714in}}%
\pgfpathlineto{\pgfqpoint{-29.374900in}{1.515214in}}%
\pgfpathlineto{\pgfqpoint{-29.304065in}{1.557431in}}%
\pgfpathlineto{\pgfqpoint{-29.235084in}{1.565343in}}%
\pgfpathlineto{\pgfqpoint{-29.164288in}{1.542608in}}%
\pgfpathlineto{\pgfqpoint{-29.089467in}{1.533021in}}%
\pgfpathlineto{\pgfqpoint{-29.018119in}{1.571750in}}%
\pgfpathlineto{\pgfqpoint{-28.946213in}{1.539689in}}%
\pgfpathlineto{\pgfqpoint{-28.870906in}{1.508459in}}%
\pgfpathlineto{\pgfqpoint{-28.798746in}{1.526723in}}%
\pgfpathlineto{\pgfqpoint{-28.725736in}{1.511302in}}%
\pgfpathlineto{\pgfqpoint{-28.649563in}{1.503915in}}%
\pgfpathlineto{\pgfqpoint{-28.577221in}{1.577858in}}%
\pgfpathlineto{\pgfqpoint{-28.504596in}{1.497556in}}%
\pgfpathlineto{\pgfqpoint{-28.429123in}{1.539362in}}%
\pgfpathlineto{\pgfqpoint{-28.356803in}{1.535981in}}%
\pgfpathlineto{\pgfqpoint{-28.274638in}{1.460074in}}%
\pgfpathlineto{\pgfqpoint{-28.195772in}{1.481625in}}%
\pgfpathlineto{\pgfqpoint{-28.121663in}{1.518470in}}%
\pgfpathlineto{\pgfqpoint{-28.049197in}{1.512115in}}%
\pgfpathlineto{\pgfqpoint{-27.976694in}{1.615504in}}%
\pgfpathlineto{\pgfqpoint{-27.908279in}{1.549270in}}%
\pgfpathlineto{\pgfqpoint{-27.839702in}{1.559422in}}%
\pgfpathlineto{\pgfqpoint{-27.771032in}{1.641367in}}%
\pgfpathlineto{\pgfqpoint{-27.703951in}{1.555661in}}%
\pgfpathlineto{\pgfqpoint{-27.637925in}{1.648297in}}%
\pgfpathlineto{\pgfqpoint{-27.570291in}{1.579845in}}%
\pgfpathlineto{\pgfqpoint{-27.503250in}{1.536941in}}%
\pgfpathlineto{\pgfqpoint{-27.435671in}{1.541750in}}%
\pgfpathlineto{\pgfqpoint{-27.366087in}{1.577993in}}%
\pgfpathlineto{\pgfqpoint{-27.299189in}{1.597809in}}%
\pgfpathlineto{\pgfqpoint{-27.232937in}{1.631713in}}%
\pgfpathlineto{\pgfqpoint{-27.165061in}{1.602589in}}%
\pgfpathlineto{\pgfqpoint{-27.097972in}{1.573094in}}%
\pgfpathlineto{\pgfqpoint{-27.031677in}{1.573513in}}%
\pgfpathlineto{\pgfqpoint{-26.961733in}{1.619428in}}%
\pgfpathlineto{\pgfqpoint{-26.893662in}{1.553886in}}%
\pgfpathlineto{\pgfqpoint{-26.825734in}{1.646353in}}%
\pgfpathlineto{\pgfqpoint{-26.755672in}{1.554831in}}%
\pgfpathlineto{\pgfqpoint{-26.685925in}{1.551953in}}%
\pgfpathlineto{\pgfqpoint{-26.616985in}{1.543441in}}%
\pgfpathlineto{\pgfqpoint{-26.547093in}{1.541610in}}%
\pgfpathlineto{\pgfqpoint{-26.479146in}{1.586937in}}%
\pgfpathlineto{\pgfqpoint{-26.410031in}{1.514728in}}%
\pgfpathlineto{\pgfqpoint{-26.338662in}{1.564353in}}%
\pgfpathlineto{\pgfqpoint{-26.268985in}{1.578277in}}%
\pgfpathlineto{\pgfqpoint{-26.198294in}{1.546209in}}%
\pgfpathlineto{\pgfqpoint{-26.128033in}{1.539101in}}%
\pgfpathlineto{\pgfqpoint{-26.059544in}{1.581814in}}%
\pgfpathlineto{\pgfqpoint{-25.990773in}{1.566624in}}%
\pgfpathlineto{\pgfqpoint{-25.919100in}{1.531670in}}%
\pgfpathlineto{\pgfqpoint{-25.851769in}{1.635448in}}%
\pgfpathlineto{\pgfqpoint{-25.785528in}{1.591992in}}%
\pgfpathlineto{\pgfqpoint{-25.717568in}{1.669465in}}%
\pgfpathlineto{\pgfqpoint{-25.650614in}{1.585460in}}%
\pgfpathlineto{\pgfqpoint{-25.583742in}{1.598652in}}%
\pgfpathlineto{\pgfqpoint{-25.515370in}{1.596194in}}%
\pgfpathlineto{\pgfqpoint{-25.448976in}{1.567223in}}%
\pgfpathlineto{\pgfqpoint{-25.382786in}{1.620713in}}%
\pgfpathlineto{\pgfqpoint{-25.315375in}{1.638834in}}%
\pgfpathlineto{\pgfqpoint{-25.248757in}{1.572865in}}%
\pgfpathlineto{\pgfqpoint{-25.182394in}{1.645747in}}%
\pgfpathlineto{\pgfqpoint{-25.114674in}{1.667753in}}%
\pgfpathlineto{\pgfqpoint{-25.049351in}{1.600212in}}%
\pgfpathlineto{\pgfqpoint{-24.984232in}{1.636712in}}%
\pgfpathlineto{\pgfqpoint{-24.916019in}{1.538703in}}%
\pgfpathlineto{\pgfqpoint{-24.847966in}{1.573417in}}%
\pgfpathlineto{\pgfqpoint{-24.781655in}{1.588416in}}%
\pgfpathlineto{\pgfqpoint{-24.712889in}{1.618981in}}%
\pgfpathlineto{\pgfqpoint{-24.645693in}{1.609310in}}%
\pgfpathlineto{\pgfqpoint{-24.578904in}{1.623196in}}%
\pgfpathlineto{\pgfqpoint{-24.511431in}{1.612247in}}%
\pgfpathlineto{\pgfqpoint{-24.445935in}{1.620571in}}%
\pgfpathlineto{\pgfqpoint{-24.381203in}{1.649474in}}%
\pgfpathlineto{\pgfqpoint{-24.313406in}{1.570075in}}%
\pgfpathlineto{\pgfqpoint{-24.243942in}{1.475877in}}%
\pgfpathlineto{\pgfqpoint{-24.172427in}{1.509290in}}%
\pgfpathlineto{\pgfqpoint{-24.096931in}{1.513495in}}%
\pgfpathlineto{\pgfqpoint{-24.025129in}{1.536944in}}%
\pgfpathlineto{\pgfqpoint{-23.953546in}{1.571863in}}%
\pgfpathlineto{\pgfqpoint{-23.881105in}{1.558132in}}%
\pgfpathlineto{\pgfqpoint{-23.810815in}{1.595614in}}%
\pgfpathlineto{\pgfqpoint{-23.739951in}{1.502950in}}%
\pgfpathlineto{\pgfqpoint{-23.664795in}{1.543653in}}%
\pgfpathlineto{\pgfqpoint{-23.594251in}{1.588459in}}%
\pgfpathlineto{\pgfqpoint{-23.523979in}{1.528063in}}%
\pgfpathlineto{\pgfqpoint{-23.451780in}{1.567280in}}%
\pgfpathlineto{\pgfqpoint{-23.383303in}{1.584183in}}%
\pgfpathlineto{\pgfqpoint{-23.314406in}{1.525261in}}%
\pgfpathlineto{\pgfqpoint{-23.243519in}{1.612249in}}%
\pgfpathlineto{\pgfqpoint{-23.175783in}{1.528210in}}%
\pgfpathlineto{\pgfqpoint{-23.107951in}{1.618288in}}%
\pgfpathlineto{\pgfqpoint{-23.037151in}{1.561282in}}%
\pgfpathlineto{\pgfqpoint{-22.968635in}{1.564719in}}%
\pgfpathlineto{\pgfqpoint{-22.900479in}{1.528788in}}%
\pgfpathlineto{\pgfqpoint{-22.831621in}{1.653656in}}%
\pgfpathlineto{\pgfqpoint{-22.764599in}{1.628773in}}%
\pgfpathlineto{\pgfqpoint{-22.696138in}{1.573738in}}%
\pgfpathlineto{\pgfqpoint{-22.627376in}{1.650890in}}%
\pgfpathlineto{\pgfqpoint{-22.560590in}{1.616904in}}%
\pgfpathlineto{\pgfqpoint{-22.492630in}{1.530704in}}%
\pgfpathlineto{\pgfqpoint{-22.421718in}{1.567871in}}%
\pgfpathlineto{\pgfqpoint{-22.353759in}{1.594816in}}%
\pgfpathlineto{\pgfqpoint{-22.286974in}{1.608840in}}%
\pgfpathlineto{\pgfqpoint{-22.217508in}{1.556183in}}%
\pgfpathlineto{\pgfqpoint{-22.150098in}{1.616851in}}%
\pgfpathlineto{\pgfqpoint{-22.084923in}{1.669498in}}%
\pgfpathlineto{\pgfqpoint{-22.016091in}{1.584048in}}%
\pgfpathlineto{\pgfqpoint{-21.948488in}{1.530400in}}%
\pgfpathlineto{\pgfqpoint{-21.881469in}{1.605175in}}%
\pgfpathlineto{\pgfqpoint{-21.811964in}{1.514664in}}%
\pgfpathlineto{\pgfqpoint{-21.743422in}{1.632972in}}%
\pgfpathlineto{\pgfqpoint{-21.676089in}{1.600879in}}%
\pgfpathlineto{\pgfqpoint{-21.607622in}{1.607414in}}%
\pgfpathlineto{\pgfqpoint{-21.539549in}{1.571830in}}%
\pgfpathlineto{\pgfqpoint{-21.470634in}{1.579021in}}%
\pgfpathlineto{\pgfqpoint{-21.398346in}{1.577726in}}%
\pgfpathlineto{\pgfqpoint{-21.330134in}{1.577171in}}%
\pgfpathlineto{\pgfqpoint{-21.261541in}{1.594108in}}%
\pgfpathlineto{\pgfqpoint{-21.191384in}{1.609879in}}%
\pgfpathlineto{\pgfqpoint{-21.123238in}{1.601026in}}%
\pgfpathlineto{\pgfqpoint{-21.053709in}{1.575255in}}%
\pgfpathlineto{\pgfqpoint{-20.982338in}{1.560845in}}%
\pgfpathlineto{\pgfqpoint{-20.913231in}{1.565727in}}%
\pgfpathlineto{\pgfqpoint{-20.844220in}{1.559245in}}%
\pgfpathlineto{\pgfqpoint{-20.774312in}{1.672266in}}%
\pgfpathlineto{\pgfqpoint{-20.705385in}{1.584449in}}%
\pgfpathlineto{\pgfqpoint{-20.636228in}{1.558010in}}%
\pgfpathlineto{\pgfqpoint{-20.565295in}{1.616858in}}%
\pgfpathlineto{\pgfqpoint{-20.496847in}{1.597329in}}%
\pgfpathlineto{\pgfqpoint{-20.427103in}{1.562163in}}%
\pgfpathlineto{\pgfqpoint{-20.357252in}{1.563569in}}%
\pgfpathlineto{\pgfqpoint{-20.290460in}{1.617115in}}%
\pgfpathlineto{\pgfqpoint{-20.223560in}{1.640298in}}%
\pgfpathlineto{\pgfqpoint{-20.154101in}{1.568638in}}%
\pgfpathlineto{\pgfqpoint{-20.086943in}{1.651022in}}%
\pgfpathlineto{\pgfqpoint{-20.019359in}{1.538342in}}%
\pgfpathlineto{\pgfqpoint{-19.949841in}{1.588063in}}%
\pgfpathlineto{\pgfqpoint{-19.882267in}{1.591651in}}%
\pgfpathlineto{\pgfqpoint{-19.815535in}{1.591494in}}%
\pgfpathlineto{\pgfqpoint{-19.747542in}{1.609101in}}%
\pgfpathlineto{\pgfqpoint{-19.680997in}{1.623511in}}%
\pgfpathlineto{\pgfqpoint{-19.613839in}{1.530226in}}%
\pgfpathlineto{\pgfqpoint{-19.543393in}{1.542631in}}%
\pgfpathlineto{\pgfqpoint{-19.475599in}{1.576044in}}%
\pgfpathlineto{\pgfqpoint{-19.407422in}{1.548198in}}%
\pgfpathlineto{\pgfqpoint{-19.338364in}{1.548130in}}%
\pgfpathlineto{\pgfqpoint{-19.271476in}{1.595843in}}%
\pgfpathlineto{\pgfqpoint{-19.205228in}{1.582946in}}%
\pgfpathlineto{\pgfqpoint{-19.135788in}{1.590858in}}%
\pgfpathlineto{\pgfqpoint{-19.068147in}{1.553558in}}%
\pgfpathlineto{\pgfqpoint{-18.999640in}{1.583495in}}%
\pgfpathlineto{\pgfqpoint{-18.928709in}{1.546145in}}%
\pgfpathlineto{\pgfqpoint{-18.859965in}{1.569592in}}%
\pgfpathlineto{\pgfqpoint{-18.792270in}{1.597326in}}%
\pgfpathlineto{\pgfqpoint{-18.721737in}{1.556532in}}%
\pgfpathlineto{\pgfqpoint{-18.652004in}{1.588449in}}%
\pgfpathlineto{\pgfqpoint{-18.583201in}{1.532065in}}%
\pgfpathlineto{\pgfqpoint{-18.511159in}{1.522245in}}%
\pgfpathlineto{\pgfqpoint{-18.442527in}{1.539729in}}%
\pgfpathlineto{\pgfqpoint{-18.373957in}{1.605883in}}%
\pgfpathlineto{\pgfqpoint{-18.301710in}{1.514976in}}%
\pgfpathlineto{\pgfqpoint{-18.232823in}{1.553188in}}%
\pgfpathlineto{\pgfqpoint{-18.165145in}{1.577859in}}%
\pgfpathlineto{\pgfqpoint{-18.094404in}{1.561703in}}%
\pgfpathlineto{\pgfqpoint{-18.025876in}{1.562836in}}%
\pgfpathlineto{\pgfqpoint{-17.957185in}{1.554583in}}%
\pgfpathlineto{\pgfqpoint{-17.885356in}{1.566668in}}%
\pgfpathlineto{\pgfqpoint{-17.816737in}{1.590527in}}%
\pgfpathlineto{\pgfqpoint{-17.748008in}{1.555707in}}%
\pgfpathlineto{\pgfqpoint{-17.675451in}{1.533353in}}%
\pgfpathlineto{\pgfqpoint{-17.607453in}{1.625153in}}%
\pgfpathlineto{\pgfqpoint{-17.539598in}{1.516863in}}%
\pgfpathlineto{\pgfqpoint{-17.470673in}{1.596280in}}%
\pgfpathlineto{\pgfqpoint{-17.402675in}{1.603040in}}%
\pgfpathlineto{\pgfqpoint{-17.336134in}{1.565625in}}%
\pgfpathlineto{\pgfqpoint{-17.265552in}{1.518898in}}%
\pgfpathlineto{\pgfqpoint{-17.196258in}{1.562231in}}%
\pgfpathlineto{\pgfqpoint{-17.128098in}{1.612428in}}%
\pgfpathlineto{\pgfqpoint{-17.058127in}{1.591320in}}%
\pgfpathlineto{\pgfqpoint{-16.989953in}{1.585704in}}%
\pgfpathlineto{\pgfqpoint{-16.921714in}{1.647021in}}%
\pgfpathlineto{\pgfqpoint{-16.852234in}{1.560771in}}%
\pgfpathlineto{\pgfqpoint{-16.784012in}{1.564815in}}%
\pgfpathlineto{\pgfqpoint{-16.715630in}{1.555255in}}%
\pgfpathlineto{\pgfqpoint{-16.644038in}{1.487275in}}%
\pgfpathlineto{\pgfqpoint{-16.575864in}{1.574084in}}%
\pgfpathlineto{\pgfqpoint{-16.507797in}{1.563018in}}%
\pgfpathlineto{\pgfqpoint{-16.437097in}{1.491634in}}%
\pgfpathlineto{\pgfqpoint{-16.369724in}{1.625603in}}%
\pgfpathlineto{\pgfqpoint{-16.302639in}{1.556145in}}%
\pgfpathlineto{\pgfqpoint{-16.232068in}{1.594538in}}%
\pgfpathlineto{\pgfqpoint{-16.163654in}{1.568504in}}%
\pgfpathlineto{\pgfqpoint{-16.094895in}{1.518712in}}%
\pgfpathlineto{\pgfqpoint{-16.024276in}{1.591577in}}%
\pgfpathlineto{\pgfqpoint{-15.956358in}{1.576527in}}%
\pgfpathlineto{\pgfqpoint{-15.886757in}{1.485292in}}%
\pgfpathlineto{\pgfqpoint{-15.814335in}{1.546544in}}%
\pgfpathlineto{\pgfqpoint{-15.744769in}{1.558776in}}%
\pgfpathlineto{\pgfqpoint{-15.673559in}{1.537539in}}%
\pgfpathlineto{\pgfqpoint{-15.602559in}{1.570939in}}%
\pgfpathlineto{\pgfqpoint{-15.532639in}{1.533442in}}%
\pgfpathlineto{\pgfqpoint{-15.463558in}{1.562144in}}%
\pgfpathlineto{\pgfqpoint{-15.392048in}{1.615478in}}%
\pgfpathlineto{\pgfqpoint{-15.323000in}{1.588895in}}%
\pgfpathlineto{\pgfqpoint{-15.253001in}{1.547781in}}%
\pgfpathlineto{\pgfqpoint{-15.179182in}{1.518566in}}%
\pgfpathlineto{\pgfqpoint{-15.106256in}{1.479184in}}%
\pgfpathlineto{\pgfqpoint{-15.032843in}{1.512666in}}%
\pgfpathlineto{\pgfqpoint{-14.959865in}{1.568175in}}%
\pgfpathlineto{\pgfqpoint{-14.889438in}{1.576556in}}%
\pgfpathlineto{\pgfqpoint{-14.820408in}{1.545614in}}%
\pgfpathlineto{\pgfqpoint{-14.748835in}{1.535742in}}%
\pgfpathlineto{\pgfqpoint{-14.679263in}{1.567216in}}%
\pgfpathlineto{\pgfqpoint{-14.609938in}{1.530455in}}%
\pgfpathlineto{\pgfqpoint{-14.538988in}{1.579836in}}%
\pgfpathlineto{\pgfqpoint{-14.471123in}{1.607636in}}%
\pgfpathlineto{\pgfqpoint{-14.401957in}{1.549893in}}%
\pgfpathlineto{\pgfqpoint{-14.331277in}{1.520709in}}%
\pgfpathlineto{\pgfqpoint{-14.263436in}{1.576164in}}%
\pgfpathlineto{\pgfqpoint{-14.195714in}{1.556708in}}%
\pgfpathlineto{\pgfqpoint{-14.125666in}{1.614741in}}%
\pgfpathlineto{\pgfqpoint{-14.057377in}{1.566279in}}%
\pgfpathlineto{\pgfqpoint{-13.988983in}{1.551407in}}%
\pgfpathlineto{\pgfqpoint{-13.917455in}{1.532540in}}%
\pgfpathlineto{\pgfqpoint{-13.848756in}{1.569518in}}%
\pgfpathlineto{\pgfqpoint{-13.779339in}{1.531912in}}%
\pgfpathlineto{\pgfqpoint{-13.708089in}{1.568296in}}%
\pgfpathlineto{\pgfqpoint{-13.639609in}{1.606337in}}%
\pgfpathlineto{\pgfqpoint{-13.568713in}{1.490930in}}%
\pgfpathlineto{\pgfqpoint{-13.496151in}{1.567174in}}%
\pgfpathlineto{\pgfqpoint{-13.426276in}{1.550138in}}%
\pgfpathlineto{\pgfqpoint{-13.356968in}{1.555759in}}%
\pgfpathlineto{\pgfqpoint{-13.285456in}{1.551346in}}%
\pgfpathlineto{\pgfqpoint{-13.216013in}{1.592736in}}%
\pgfpathlineto{\pgfqpoint{-13.145337in}{1.576170in}}%
\pgfpathlineto{\pgfqpoint{-13.073093in}{1.530036in}}%
\pgfpathlineto{\pgfqpoint{-13.002727in}{1.554154in}}%
\pgfpathlineto{\pgfqpoint{-12.932388in}{1.522647in}}%
\pgfpathlineto{\pgfqpoint{-12.859398in}{1.573012in}}%
\pgfpathlineto{\pgfqpoint{-12.788679in}{1.548196in}}%
\pgfpathlineto{\pgfqpoint{-12.718096in}{1.497654in}}%
\pgfpathlineto{\pgfqpoint{-12.645279in}{1.556952in}}%
\pgfpathlineto{\pgfqpoint{-12.574652in}{1.536842in}}%
\pgfpathlineto{\pgfqpoint{-12.503401in}{1.558845in}}%
\pgfpathlineto{\pgfqpoint{-12.428997in}{1.474807in}}%
\pgfpathlineto{\pgfqpoint{-12.357103in}{1.542659in}}%
\pgfpathlineto{\pgfqpoint{-12.285275in}{1.484903in}}%
\pgfpathlineto{\pgfqpoint{-12.211536in}{1.530595in}}%
\pgfpathlineto{\pgfqpoint{-12.139885in}{1.526749in}}%
\pgfpathlineto{\pgfqpoint{-12.069446in}{1.558572in}}%
\pgfpathlineto{\pgfqpoint{-11.998596in}{1.547421in}}%
\pgfpathlineto{\pgfqpoint{-11.928381in}{1.582011in}}%
\pgfpathlineto{\pgfqpoint{-11.858471in}{1.551767in}}%
\pgfpathlineto{\pgfqpoint{-11.786176in}{1.590515in}}%
\pgfpathlineto{\pgfqpoint{-11.716309in}{1.578664in}}%
\pgfpathlineto{\pgfqpoint{-11.645187in}{1.535340in}}%
\pgfpathlineto{\pgfqpoint{-11.571037in}{1.576325in}}%
\pgfpathlineto{\pgfqpoint{-11.500166in}{1.524114in}}%
\pgfpathlineto{\pgfqpoint{-11.430351in}{1.569264in}}%
\pgfpathlineto{\pgfqpoint{-11.358676in}{1.528517in}}%
\pgfpathlineto{\pgfqpoint{-11.288273in}{1.553784in}}%
\pgfpathlineto{\pgfqpoint{-11.217631in}{1.512026in}}%
\pgfpathlineto{\pgfqpoint{-11.144960in}{1.589136in}}%
\pgfpathlineto{\pgfqpoint{-11.075276in}{1.568820in}}%
\pgfpathlineto{\pgfqpoint{-11.005065in}{1.479274in}}%
\pgfpathlineto{\pgfqpoint{-10.933808in}{1.575582in}}%
\pgfpathlineto{\pgfqpoint{-10.863257in}{1.523452in}}%
\pgfpathlineto{\pgfqpoint{-10.793422in}{1.518855in}}%
\pgfpathlineto{\pgfqpoint{-10.721651in}{1.600105in}}%
\pgfpathlineto{\pgfqpoint{-10.652649in}{1.557287in}}%
\pgfpathlineto{\pgfqpoint{-10.582864in}{1.475010in}}%
\pgfpathlineto{\pgfqpoint{-10.511372in}{1.557813in}}%
\pgfpathlineto{\pgfqpoint{-10.441581in}{1.507756in}}%
\pgfpathlineto{\pgfqpoint{-10.371004in}{1.498560in}}%
\pgfpathlineto{\pgfqpoint{-10.297591in}{1.536177in}}%
\pgfpathlineto{\pgfqpoint{-10.227739in}{1.540504in}}%
\pgfpathlineto{\pgfqpoint{-10.157398in}{1.545125in}}%
\pgfpathlineto{\pgfqpoint{-10.084496in}{1.536204in}}%
\pgfpathlineto{\pgfqpoint{-10.012635in}{1.476574in}}%
\pgfpathlineto{\pgfqpoint{-9.941359in}{1.535035in}}%
\pgfpathlineto{\pgfqpoint{-9.868243in}{1.562266in}}%
\pgfpathlineto{\pgfqpoint{-9.796010in}{1.500466in}}%
\pgfpathlineto{\pgfqpoint{-9.723682in}{1.545626in}}%
\pgfpathlineto{\pgfqpoint{-9.648369in}{1.524106in}}%
\pgfpathlineto{\pgfqpoint{-9.575400in}{1.478797in}}%
\pgfpathlineto{\pgfqpoint{-9.504196in}{1.514715in}}%
\pgfpathlineto{\pgfqpoint{-9.429774in}{1.531987in}}%
\pgfpathlineto{\pgfqpoint{-9.356529in}{1.470709in}}%
\pgfpathlineto{\pgfqpoint{-9.283412in}{1.501061in}}%
\pgfpathlineto{\pgfqpoint{-9.208906in}{1.557458in}}%
\pgfpathlineto{\pgfqpoint{-9.138788in}{1.500385in}}%
\pgfpathlineto{\pgfqpoint{-9.069721in}{1.558749in}}%
\pgfpathlineto{\pgfqpoint{-8.998701in}{1.587263in}}%
\pgfpathlineto{\pgfqpoint{-8.928123in}{1.585803in}}%
\pgfpathlineto{\pgfqpoint{-8.858221in}{1.591324in}}%
\pgfpathlineto{\pgfqpoint{-8.787586in}{1.627833in}}%
\pgfpathlineto{\pgfqpoint{-8.718609in}{1.614311in}}%
\pgfpathlineto{\pgfqpoint{-8.649341in}{1.599819in}}%
\pgfpathlineto{\pgfqpoint{-8.578369in}{1.560943in}}%
\pgfpathlineto{\pgfqpoint{-8.508839in}{1.575295in}}%
\pgfpathlineto{\pgfqpoint{-8.438970in}{1.521624in}}%
\pgfpathlineto{\pgfqpoint{-8.367967in}{1.562385in}}%
\pgfpathlineto{\pgfqpoint{-8.298753in}{1.547590in}}%
\pgfpathlineto{\pgfqpoint{-8.228241in}{1.506883in}}%
\pgfpathlineto{\pgfqpoint{-8.156111in}{1.546275in}}%
\pgfpathlineto{\pgfqpoint{-8.084913in}{1.504832in}}%
\pgfpathlineto{\pgfqpoint{-8.014387in}{1.584190in}}%
\pgfpathlineto{\pgfqpoint{-7.942200in}{1.604120in}}%
\pgfpathlineto{\pgfqpoint{-7.870788in}{1.532899in}}%
\pgfpathlineto{\pgfqpoint{-7.799875in}{1.598700in}}%
\pgfpathlineto{\pgfqpoint{-7.727286in}{1.565118in}}%
\pgfpathlineto{\pgfqpoint{-7.657024in}{1.559116in}}%
\pgfpathlineto{\pgfqpoint{-7.585515in}{1.482405in}}%
\pgfpathlineto{\pgfqpoint{-7.512981in}{1.583860in}}%
\pgfpathlineto{\pgfqpoint{-7.440452in}{1.480083in}}%
\pgfpathlineto{\pgfqpoint{-7.368338in}{1.573078in}}%
\pgfpathlineto{\pgfqpoint{-7.294929in}{1.551779in}}%
\pgfpathlineto{\pgfqpoint{-7.223147in}{1.515672in}}%
\pgfpathlineto{\pgfqpoint{-7.151909in}{1.524874in}}%
\pgfpathlineto{\pgfqpoint{-7.075897in}{1.496645in}}%
\pgfpathlineto{\pgfqpoint{-7.003598in}{1.522700in}}%
\pgfpathlineto{\pgfqpoint{-6.933558in}{1.593510in}}%
\pgfpathlineto{\pgfqpoint{-6.860576in}{1.550760in}}%
\pgfpathlineto{\pgfqpoint{-6.787495in}{1.482019in}}%
\pgfpathlineto{\pgfqpoint{-6.715245in}{1.503712in}}%
\pgfpathlineto{\pgfqpoint{-6.640496in}{1.502131in}}%
\pgfpathlineto{\pgfqpoint{-6.568712in}{1.542142in}}%
\pgfpathlineto{\pgfqpoint{-6.496669in}{1.501851in}}%
\pgfpathlineto{\pgfqpoint{-6.421127in}{1.493167in}}%
\pgfpathlineto{\pgfqpoint{-6.349020in}{1.548758in}}%
\pgfpathlineto{\pgfqpoint{-6.278887in}{1.578390in}}%
\pgfpathlineto{\pgfqpoint{-6.205113in}{1.429075in}}%
\pgfpathlineto{\pgfqpoint{-6.135189in}{1.598818in}}%
\pgfpathlineto{\pgfqpoint{-6.064664in}{1.559183in}}%
\pgfpathlineto{\pgfqpoint{-5.992127in}{1.507083in}}%
\pgfpathlineto{\pgfqpoint{-5.921215in}{1.543851in}}%
\pgfpathlineto{\pgfqpoint{-5.849895in}{1.515995in}}%
\pgfpathlineto{\pgfqpoint{-5.776542in}{1.540403in}}%
\pgfpathlineto{\pgfqpoint{-5.706570in}{1.553032in}}%
\pgfpathlineto{\pgfqpoint{-5.635685in}{1.552118in}}%
\pgfpathlineto{\pgfqpoint{-5.563648in}{1.637698in}}%
\pgfpathlineto{\pgfqpoint{-5.493615in}{1.532726in}}%
\pgfpathlineto{\pgfqpoint{-5.421890in}{1.485416in}}%
\pgfpathlineto{\pgfqpoint{-5.348665in}{1.531879in}}%
\pgfpathlineto{\pgfqpoint{-5.277970in}{1.546213in}}%
\pgfpathlineto{\pgfqpoint{-5.207317in}{1.577262in}}%
\pgfpathlineto{\pgfqpoint{-5.135005in}{1.554414in}}%
\pgfpathlineto{\pgfqpoint{-5.064817in}{1.579015in}}%
\pgfpathlineto{\pgfqpoint{-4.995002in}{1.577246in}}%
\pgfpathlineto{\pgfqpoint{-4.922664in}{1.505684in}}%
\pgfpathlineto{\pgfqpoint{-4.852652in}{1.536318in}}%
\pgfpathlineto{\pgfqpoint{-4.782965in}{1.587070in}}%
\pgfpathlineto{\pgfqpoint{-4.711002in}{1.531143in}}%
\pgfpathlineto{\pgfqpoint{-4.639570in}{1.508803in}}%
\pgfpathlineto{\pgfqpoint{-4.568595in}{1.507448in}}%
\pgfpathlineto{\pgfqpoint{-4.495401in}{1.590285in}}%
\pgfpathlineto{\pgfqpoint{-4.423289in}{1.517289in}}%
\pgfpathlineto{\pgfqpoint{-4.350759in}{1.536899in}}%
\pgfpathlineto{\pgfqpoint{-4.274748in}{1.495423in}}%
\pgfpathlineto{\pgfqpoint{-4.201246in}{1.474921in}}%
\pgfpathlineto{\pgfqpoint{-4.128352in}{1.532911in}}%
\pgfpathlineto{\pgfqpoint{-4.052173in}{1.481939in}}%
\pgfpathlineto{\pgfqpoint{-3.977308in}{1.520586in}}%
\pgfpathlineto{\pgfqpoint{-3.903164in}{1.464448in}}%
\pgfpathlineto{\pgfqpoint{-3.828230in}{1.492647in}}%
\pgfpathlineto{\pgfqpoint{-3.756893in}{1.534409in}}%
\pgfpathlineto{\pgfqpoint{-3.684119in}{1.459063in}}%
\pgfpathlineto{\pgfqpoint{-3.608325in}{1.468393in}}%
\pgfpathlineto{\pgfqpoint{-3.534002in}{1.509844in}}%
\pgfpathlineto{\pgfqpoint{-3.461876in}{1.569511in}}%
\pgfpathlineto{\pgfqpoint{-3.388885in}{1.544437in}}%
\pgfpathlineto{\pgfqpoint{-3.317677in}{1.521556in}}%
\pgfpathlineto{\pgfqpoint{-3.245632in}{1.510336in}}%
\pgfpathlineto{\pgfqpoint{-3.171735in}{1.548452in}}%
\pgfpathlineto{\pgfqpoint{-3.100293in}{1.448544in}}%
\pgfpathlineto{\pgfqpoint{-3.029702in}{1.549251in}}%
\pgfpathlineto{\pgfqpoint{-2.956196in}{1.503924in}}%
\pgfpathlineto{\pgfqpoint{-2.884033in}{1.504888in}}%
\pgfpathlineto{\pgfqpoint{-2.813631in}{1.560384in}}%
\pgfpathlineto{\pgfqpoint{-2.740325in}{1.517836in}}%
\pgfpathlineto{\pgfqpoint{-2.667020in}{1.469280in}}%
\pgfpathlineto{\pgfqpoint{-2.595405in}{1.517961in}}%
\pgfpathlineto{\pgfqpoint{-2.522349in}{1.583015in}}%
\pgfpathlineto{\pgfqpoint{-2.451489in}{1.552146in}}%
\pgfpathlineto{\pgfqpoint{-2.379035in}{1.524477in}}%
\pgfpathlineto{\pgfqpoint{-2.303263in}{1.527257in}}%
\pgfpathlineto{\pgfqpoint{-2.231477in}{1.484943in}}%
\pgfpathlineto{\pgfqpoint{-2.158577in}{1.540183in}}%
\pgfpathlineto{\pgfqpoint{-2.084338in}{1.524046in}}%
\pgfpathlineto{\pgfqpoint{-2.011817in}{1.569026in}}%
\pgfpathlineto{\pgfqpoint{-1.940536in}{1.571508in}}%
\pgfpathlineto{\pgfqpoint{-1.867809in}{1.527152in}}%
\pgfpathlineto{\pgfqpoint{-1.797123in}{1.559196in}}%
\pgfpathlineto{\pgfqpoint{-1.725105in}{1.579406in}}%
\pgfpathlineto{\pgfqpoint{-1.650581in}{1.496104in}}%
\pgfpathlineto{\pgfqpoint{-1.577590in}{1.539047in}}%
\pgfpathlineto{\pgfqpoint{-1.503984in}{1.500767in}}%
\pgfpathlineto{\pgfqpoint{-1.429441in}{1.517802in}}%
\pgfpathlineto{\pgfqpoint{-1.356794in}{1.496379in}}%
\pgfpathlineto{\pgfqpoint{-1.283200in}{1.551759in}}%
\pgfpathlineto{\pgfqpoint{-1.207908in}{1.529684in}}%
\pgfpathlineto{\pgfqpoint{-1.135138in}{1.585140in}}%
\pgfpathlineto{\pgfqpoint{-1.063245in}{1.584407in}}%
\pgfpathlineto{\pgfqpoint{-0.989522in}{1.547390in}}%
\pgfpathlineto{\pgfqpoint{-0.918138in}{1.535898in}}%
\pgfpathlineto{\pgfqpoint{-0.844408in}{1.510561in}}%
\pgfpathlineto{\pgfqpoint{-0.769987in}{1.546146in}}%
\pgfpathlineto{\pgfqpoint{-0.698122in}{1.531777in}}%
\pgfpathlineto{\pgfqpoint{-0.623932in}{1.493309in}}%
\pgfpathlineto{\pgfqpoint{-0.549722in}{1.529259in}}%
\pgfpathlineto{\pgfqpoint{-0.477784in}{1.548263in}}%
\pgfpathlineto{\pgfqpoint{-0.404497in}{1.596386in}}%
\pgfpathlineto{\pgfqpoint{-0.329323in}{1.554710in}}%
\pgfpathlineto{\pgfqpoint{-0.257459in}{1.518785in}}%
\pgfpathlineto{\pgfqpoint{-0.185623in}{1.549389in}}%
\pgfpathlineto{\pgfqpoint{-0.112821in}{1.521427in}}%
\pgfpathlineto{\pgfqpoint{-0.042097in}{1.546845in}}%
\pgfpathlineto{\pgfqpoint{0.030045in}{1.531156in}}%
\pgfpathlineto{\pgfqpoint{0.103944in}{1.507670in}}%
\pgfpathlineto{\pgfqpoint{0.176284in}{1.478907in}}%
\pgfpathlineto{\pgfqpoint{0.248641in}{1.526914in}}%
\pgfpathlineto{\pgfqpoint{0.321303in}{1.524414in}}%
\pgfpathlineto{\pgfqpoint{0.391498in}{1.539139in}}%
\pgfpathlineto{\pgfqpoint{0.462885in}{1.564938in}}%
\pgfpathlineto{\pgfqpoint{0.537447in}{1.475635in}}%
\pgfpathlineto{\pgfqpoint{0.609267in}{1.584371in}}%
\pgfpathlineto{\pgfqpoint{0.679536in}{1.534031in}}%
\pgfpathlineto{\pgfqpoint{0.753203in}{1.510067in}}%
\pgfpathlineto{\pgfqpoint{0.824780in}{1.548278in}}%
\pgfpathlineto{\pgfqpoint{0.895203in}{1.551309in}}%
\pgfpathlineto{\pgfqpoint{0.968165in}{1.609372in}}%
\pgfpathlineto{\pgfqpoint{1.038437in}{1.580135in}}%
\pgfpathlineto{\pgfqpoint{1.111472in}{1.514249in}}%
\pgfpathlineto{\pgfqpoint{1.188027in}{1.520863in}}%
\pgfpathlineto{\pgfqpoint{1.261309in}{1.550963in}}%
\pgfpathlineto{\pgfqpoint{1.335256in}{1.565013in}}%
\pgfpathlineto{\pgfqpoint{1.410931in}{1.535735in}}%
\pgfpathlineto{\pgfqpoint{1.485557in}{1.469339in}}%
\pgfpathlineto{\pgfqpoint{1.558242in}{1.573511in}}%
\pgfpathlineto{\pgfqpoint{1.633550in}{1.504590in}}%
\pgfpathlineto{\pgfqpoint{1.707271in}{1.540608in}}%
\pgfpathlineto{\pgfqpoint{1.781074in}{1.535470in}}%
\pgfpathlineto{\pgfqpoint{1.857049in}{1.496544in}}%
\pgfpathlineto{\pgfqpoint{1.931091in}{1.495765in}}%
\pgfpathlineto{\pgfqpoint{2.004776in}{1.493064in}}%
\pgfpathlineto{\pgfqpoint{2.079647in}{1.540189in}}%
\pgfpathlineto{\pgfqpoint{2.151611in}{1.541899in}}%
\pgfpathlineto{\pgfqpoint{2.224651in}{1.547229in}}%
\pgfpathlineto{\pgfqpoint{2.300156in}{1.523428in}}%
\pgfpathlineto{\pgfqpoint{2.371299in}{1.510628in}}%
\pgfpathlineto{\pgfqpoint{2.442412in}{1.506848in}}%
\pgfpathlineto{\pgfqpoint{2.515763in}{1.597485in}}%
\pgfpathlineto{\pgfqpoint{2.586109in}{1.579842in}}%
\pgfpathlineto{\pgfqpoint{2.658524in}{1.544412in}}%
\pgfpathlineto{\pgfqpoint{2.732102in}{1.501294in}}%
\pgfpathlineto{\pgfqpoint{2.802412in}{1.509098in}}%
\pgfpathlineto{\pgfqpoint{2.873367in}{1.536135in}}%
\pgfpathlineto{\pgfqpoint{2.946925in}{1.532760in}}%
\pgfpathlineto{\pgfqpoint{3.019212in}{1.509126in}}%
\pgfpathlineto{\pgfqpoint{3.091740in}{1.518304in}}%
\pgfpathlineto{\pgfqpoint{3.166494in}{1.510439in}}%
\pgfpathlineto{\pgfqpoint{3.237461in}{1.566490in}}%
\pgfpathlineto{\pgfqpoint{3.309976in}{1.539915in}}%
\pgfpathlineto{\pgfqpoint{3.384706in}{1.472512in}}%
\pgfpathlineto{\pgfqpoint{3.455601in}{1.565434in}}%
\pgfpathlineto{\pgfqpoint{3.527918in}{1.504964in}}%
\pgfpathlineto{\pgfqpoint{3.603230in}{1.498229in}}%
\pgfpathlineto{\pgfqpoint{3.674965in}{1.567696in}}%
\pgfpathlineto{\pgfqpoint{3.745938in}{1.575304in}}%
\pgfpathlineto{\pgfqpoint{3.820753in}{1.484551in}}%
\pgfpathlineto{\pgfqpoint{3.899537in}{1.457530in}}%
\pgfpathlineto{\pgfqpoint{4.022313in}{1.253992in}}%
\pgfpathlineto{\pgfqpoint{4.111286in}{1.497722in}}%
\pgfpathlineto{\pgfqpoint{4.189282in}{0.926398in}}%
\pgfpathlineto{\pgfqpoint{4.253332in}{1.724813in}}%
\pgfpathlineto{\pgfqpoint{4.318577in}{5.187514in}}%
\pgfpathlineto{\pgfqpoint{4.390596in}{5.319244in}}%
\pgfpathlineto{\pgfqpoint{4.460559in}{5.506716in}}%
\pgfpathlineto{\pgfqpoint{4.532858in}{5.399336in}}%
\pgfpathlineto{\pgfqpoint{4.602459in}{5.488048in}}%
\pgfpathlineto{\pgfqpoint{4.671388in}{5.539128in}}%
\pgfpathlineto{\pgfqpoint{4.741243in}{5.638191in}}%
\pgfpathlineto{\pgfqpoint{4.808397in}{5.619869in}}%
\pgfpathlineto{\pgfqpoint{4.875969in}{5.661488in}}%
\pgfpathlineto{\pgfqpoint{4.944850in}{5.625548in}}%
\pgfpathlineto{\pgfqpoint{5.010860in}{5.748599in}}%
\pgfpathlineto{\pgfqpoint{5.076713in}{5.732424in}}%
\pgfpathlineto{\pgfqpoint{5.144715in}{5.790749in}}%
\pgfpathlineto{\pgfqpoint{5.209656in}{5.793255in}}%
\pgfpathlineto{\pgfqpoint{5.275100in}{5.772285in}}%
\pgfpathlineto{\pgfqpoint{5.341477in}{5.875626in}}%
\pgfpathlineto{\pgfqpoint{5.405234in}{5.879596in}}%
\pgfpathlineto{\pgfqpoint{5.469981in}{5.814953in}}%
\pgfpathlineto{\pgfqpoint{5.535779in}{5.930845in}}%
\pgfpathlineto{\pgfqpoint{5.599590in}{5.872359in}}%
\pgfpathlineto{\pgfqpoint{5.599590in}{5.872359in}}%
\pgfpathlineto{\pgfqpoint{5.599590in}{5.872359in}}%
\pgfpathlineto{\pgfqpoint{5.535779in}{5.930845in}}%
\pgfpathlineto{\pgfqpoint{5.469981in}{5.814953in}}%
\pgfpathlineto{\pgfqpoint{5.405234in}{5.879596in}}%
\pgfpathlineto{\pgfqpoint{5.341477in}{5.875626in}}%
\pgfpathlineto{\pgfqpoint{5.275100in}{5.772285in}}%
\pgfpathlineto{\pgfqpoint{5.209656in}{5.793255in}}%
\pgfpathlineto{\pgfqpoint{5.144715in}{5.790749in}}%
\pgfpathlineto{\pgfqpoint{5.076713in}{5.732424in}}%
\pgfpathlineto{\pgfqpoint{5.010860in}{5.748599in}}%
\pgfpathlineto{\pgfqpoint{4.944850in}{5.625548in}}%
\pgfpathlineto{\pgfqpoint{4.875969in}{5.661488in}}%
\pgfpathlineto{\pgfqpoint{4.808397in}{5.619869in}}%
\pgfpathlineto{\pgfqpoint{4.741243in}{5.638191in}}%
\pgfpathlineto{\pgfqpoint{4.671388in}{5.539128in}}%
\pgfpathlineto{\pgfqpoint{4.602459in}{5.488048in}}%
\pgfpathlineto{\pgfqpoint{4.532858in}{5.399336in}}%
\pgfpathlineto{\pgfqpoint{4.460559in}{5.506716in}}%
\pgfpathlineto{\pgfqpoint{4.390596in}{5.319244in}}%
\pgfpathlineto{\pgfqpoint{4.318577in}{5.187514in}}%
\pgfpathlineto{\pgfqpoint{4.253332in}{1.724813in}}%
\pgfpathlineto{\pgfqpoint{4.189282in}{0.926398in}}%
\pgfpathlineto{\pgfqpoint{4.111286in}{1.559351in}}%
\pgfpathlineto{\pgfqpoint{4.022313in}{1.695897in}}%
\pgfpathlineto{\pgfqpoint{3.899537in}{2.152523in}}%
\pgfpathlineto{\pgfqpoint{3.820753in}{2.229596in}}%
\pgfpathlineto{\pgfqpoint{3.745938in}{2.365524in}}%
\pgfpathlineto{\pgfqpoint{3.674965in}{2.323501in}}%
\pgfpathlineto{\pgfqpoint{3.603230in}{2.252098in}}%
\pgfpathlineto{\pgfqpoint{3.527918in}{2.265781in}}%
\pgfpathlineto{\pgfqpoint{3.455601in}{2.329202in}}%
\pgfpathlineto{\pgfqpoint{3.384706in}{2.228499in}}%
\pgfpathlineto{\pgfqpoint{3.309976in}{2.316665in}}%
\pgfpathlineto{\pgfqpoint{3.237461in}{2.375130in}}%
\pgfpathlineto{\pgfqpoint{3.166494in}{2.268673in}}%
\pgfpathlineto{\pgfqpoint{3.091740in}{2.263431in}}%
\pgfpathlineto{\pgfqpoint{3.019212in}{2.215727in}}%
\pgfpathlineto{\pgfqpoint{2.946925in}{2.315701in}}%
\pgfpathlineto{\pgfqpoint{2.873367in}{2.343943in}}%
\pgfpathlineto{\pgfqpoint{2.802412in}{2.265161in}}%
\pgfpathlineto{\pgfqpoint{2.732102in}{2.239019in}}%
\pgfpathlineto{\pgfqpoint{2.658524in}{2.335533in}}%
\pgfpathlineto{\pgfqpoint{2.586109in}{2.364491in}}%
\pgfpathlineto{\pgfqpoint{2.515763in}{2.399372in}}%
\pgfpathlineto{\pgfqpoint{2.442412in}{2.292558in}}%
\pgfpathlineto{\pgfqpoint{2.371299in}{2.304522in}}%
\pgfpathlineto{\pgfqpoint{2.300156in}{2.319907in}}%
\pgfpathlineto{\pgfqpoint{2.224651in}{2.315839in}}%
\pgfpathlineto{\pgfqpoint{2.151611in}{2.340502in}}%
\pgfpathlineto{\pgfqpoint{2.079647in}{2.326432in}}%
\pgfpathlineto{\pgfqpoint{2.004776in}{2.262254in}}%
\pgfpathlineto{\pgfqpoint{1.931091in}{2.250884in}}%
\pgfpathlineto{\pgfqpoint{1.857049in}{2.213105in}}%
\pgfpathlineto{\pgfqpoint{1.781074in}{2.263560in}}%
\pgfpathlineto{\pgfqpoint{1.707271in}{2.258974in}}%
\pgfpathlineto{\pgfqpoint{1.633550in}{2.261273in}}%
\pgfpathlineto{\pgfqpoint{1.558242in}{2.347512in}}%
\pgfpathlineto{\pgfqpoint{1.485557in}{2.220336in}}%
\pgfpathlineto{\pgfqpoint{1.410931in}{2.321530in}}%
\pgfpathlineto{\pgfqpoint{1.335256in}{2.259385in}}%
\pgfpathlineto{\pgfqpoint{1.261309in}{2.302499in}}%
\pgfpathlineto{\pgfqpoint{1.188027in}{2.198868in}}%
\pgfpathlineto{\pgfqpoint{1.111472in}{2.224849in}}%
\pgfpathlineto{\pgfqpoint{1.038437in}{2.343567in}}%
\pgfpathlineto{\pgfqpoint{0.968165in}{2.391312in}}%
\pgfpathlineto{\pgfqpoint{0.895203in}{2.373974in}}%
\pgfpathlineto{\pgfqpoint{0.824780in}{2.315904in}}%
\pgfpathlineto{\pgfqpoint{0.753203in}{2.296487in}}%
\pgfpathlineto{\pgfqpoint{0.679536in}{2.288985in}}%
\pgfpathlineto{\pgfqpoint{0.609267in}{2.355976in}}%
\pgfpathlineto{\pgfqpoint{0.537447in}{2.225068in}}%
\pgfpathlineto{\pgfqpoint{0.462885in}{2.327995in}}%
\pgfpathlineto{\pgfqpoint{0.391498in}{2.369077in}}%
\pgfpathlineto{\pgfqpoint{0.321303in}{2.331917in}}%
\pgfpathlineto{\pgfqpoint{0.248641in}{2.280462in}}%
\pgfpathlineto{\pgfqpoint{0.176284in}{2.251796in}}%
\pgfpathlineto{\pgfqpoint{0.103944in}{2.336838in}}%
\pgfpathlineto{\pgfqpoint{0.030045in}{2.268170in}}%
\pgfpathlineto{\pgfqpoint{-0.042097in}{2.302230in}}%
\pgfpathlineto{\pgfqpoint{-0.112821in}{2.321634in}}%
\pgfpathlineto{\pgfqpoint{-0.185623in}{2.314323in}}%
\pgfpathlineto{\pgfqpoint{-0.257459in}{2.280819in}}%
\pgfpathlineto{\pgfqpoint{-0.329323in}{2.184436in}}%
\pgfpathlineto{\pgfqpoint{-0.404497in}{2.263642in}}%
\pgfpathlineto{\pgfqpoint{-0.477784in}{2.277852in}}%
\pgfpathlineto{\pgfqpoint{-0.549722in}{2.291781in}}%
\pgfpathlineto{\pgfqpoint{-0.623932in}{2.226503in}}%
\pgfpathlineto{\pgfqpoint{-0.698122in}{2.345668in}}%
\pgfpathlineto{\pgfqpoint{-0.769987in}{2.307196in}}%
\pgfpathlineto{\pgfqpoint{-0.844408in}{2.262804in}}%
\pgfpathlineto{\pgfqpoint{-0.918138in}{2.290693in}}%
\pgfpathlineto{\pgfqpoint{-0.989522in}{2.342473in}}%
\pgfpathlineto{\pgfqpoint{-1.063245in}{2.314096in}}%
\pgfpathlineto{\pgfqpoint{-1.135138in}{2.297310in}}%
\pgfpathlineto{\pgfqpoint{-1.207908in}{2.233525in}}%
\pgfpathlineto{\pgfqpoint{-1.283200in}{2.240572in}}%
\pgfpathlineto{\pgfqpoint{-1.356794in}{2.267018in}}%
\pgfpathlineto{\pgfqpoint{-1.429441in}{2.302823in}}%
\pgfpathlineto{\pgfqpoint{-1.503984in}{2.233145in}}%
\pgfpathlineto{\pgfqpoint{-1.577590in}{2.294793in}}%
\pgfpathlineto{\pgfqpoint{-1.650581in}{2.260473in}}%
\pgfpathlineto{\pgfqpoint{-1.725105in}{2.325966in}}%
\pgfpathlineto{\pgfqpoint{-1.797123in}{2.339174in}}%
\pgfpathlineto{\pgfqpoint{-1.867809in}{2.314700in}}%
\pgfpathlineto{\pgfqpoint{-1.940536in}{2.298618in}}%
\pgfpathlineto{\pgfqpoint{-2.011817in}{2.308092in}}%
\pgfpathlineto{\pgfqpoint{-2.084338in}{2.225859in}}%
\pgfpathlineto{\pgfqpoint{-2.158577in}{2.293684in}}%
\pgfpathlineto{\pgfqpoint{-2.231477in}{2.212543in}}%
\pgfpathlineto{\pgfqpoint{-2.303263in}{2.266146in}}%
\pgfpathlineto{\pgfqpoint{-2.379035in}{2.264222in}}%
\pgfpathlineto{\pgfqpoint{-2.451489in}{2.328953in}}%
\pgfpathlineto{\pgfqpoint{-2.522349in}{2.351888in}}%
\pgfpathlineto{\pgfqpoint{-2.595405in}{2.301102in}}%
\pgfpathlineto{\pgfqpoint{-2.667020in}{2.243646in}}%
\pgfpathlineto{\pgfqpoint{-2.740325in}{2.244114in}}%
\pgfpathlineto{\pgfqpoint{-2.813631in}{2.325439in}}%
\pgfpathlineto{\pgfqpoint{-2.884033in}{2.281621in}}%
\pgfpathlineto{\pgfqpoint{-2.956196in}{2.249022in}}%
\pgfpathlineto{\pgfqpoint{-3.029702in}{2.371326in}}%
\pgfpathlineto{\pgfqpoint{-3.100293in}{2.210001in}}%
\pgfpathlineto{\pgfqpoint{-3.171735in}{2.251101in}}%
\pgfpathlineto{\pgfqpoint{-3.245632in}{2.253161in}}%
\pgfpathlineto{\pgfqpoint{-3.317677in}{2.300486in}}%
\pgfpathlineto{\pgfqpoint{-3.388885in}{2.334604in}}%
\pgfpathlineto{\pgfqpoint{-3.461876in}{2.334452in}}%
\pgfpathlineto{\pgfqpoint{-3.534002in}{2.221174in}}%
\pgfpathlineto{\pgfqpoint{-3.608325in}{2.189532in}}%
\pgfpathlineto{\pgfqpoint{-3.684119in}{2.204306in}}%
\pgfpathlineto{\pgfqpoint{-3.756893in}{2.299883in}}%
\pgfpathlineto{\pgfqpoint{-3.828230in}{2.281051in}}%
\pgfpathlineto{\pgfqpoint{-3.903164in}{2.226320in}}%
\pgfpathlineto{\pgfqpoint{-3.977308in}{2.248437in}}%
\pgfpathlineto{\pgfqpoint{-4.052173in}{2.202434in}}%
\pgfpathlineto{\pgfqpoint{-4.128352in}{2.307017in}}%
\pgfpathlineto{\pgfqpoint{-4.201246in}{2.267292in}}%
\pgfpathlineto{\pgfqpoint{-4.274748in}{2.267224in}}%
\pgfpathlineto{\pgfqpoint{-4.350759in}{2.293648in}}%
\pgfpathlineto{\pgfqpoint{-4.423289in}{2.266617in}}%
\pgfpathlineto{\pgfqpoint{-4.495401in}{2.356686in}}%
\pgfpathlineto{\pgfqpoint{-4.568595in}{2.312856in}}%
\pgfpathlineto{\pgfqpoint{-4.639570in}{2.325656in}}%
\pgfpathlineto{\pgfqpoint{-4.711002in}{2.274754in}}%
\pgfpathlineto{\pgfqpoint{-4.782965in}{2.375722in}}%
\pgfpathlineto{\pgfqpoint{-4.852652in}{2.336737in}}%
\pgfpathlineto{\pgfqpoint{-4.922664in}{2.283473in}}%
\pgfpathlineto{\pgfqpoint{-4.995002in}{2.412854in}}%
\pgfpathlineto{\pgfqpoint{-5.064817in}{2.378132in}}%
\pgfpathlineto{\pgfqpoint{-5.135005in}{2.324241in}}%
\pgfpathlineto{\pgfqpoint{-5.207317in}{2.323634in}}%
\pgfpathlineto{\pgfqpoint{-5.277970in}{2.288707in}}%
\pgfpathlineto{\pgfqpoint{-5.348665in}{2.256552in}}%
\pgfpathlineto{\pgfqpoint{-5.421890in}{2.300808in}}%
\pgfpathlineto{\pgfqpoint{-5.493615in}{2.295589in}}%
\pgfpathlineto{\pgfqpoint{-5.563648in}{2.408263in}}%
\pgfpathlineto{\pgfqpoint{-5.635685in}{2.342600in}}%
\pgfpathlineto{\pgfqpoint{-5.706570in}{2.389794in}}%
\pgfpathlineto{\pgfqpoint{-5.776542in}{2.302233in}}%
\pgfpathlineto{\pgfqpoint{-5.849895in}{2.259177in}}%
\pgfpathlineto{\pgfqpoint{-5.921215in}{2.327916in}}%
\pgfpathlineto{\pgfqpoint{-5.992127in}{2.283593in}}%
\pgfpathlineto{\pgfqpoint{-6.064664in}{2.303464in}}%
\pgfpathlineto{\pgfqpoint{-6.135189in}{2.378551in}}%
\pgfpathlineto{\pgfqpoint{-6.205113in}{2.209151in}}%
\pgfpathlineto{\pgfqpoint{-6.278887in}{2.342925in}}%
\pgfpathlineto{\pgfqpoint{-6.349020in}{2.321418in}}%
\pgfpathlineto{\pgfqpoint{-6.421127in}{2.237827in}}%
\pgfpathlineto{\pgfqpoint{-6.496669in}{2.265861in}}%
\pgfpathlineto{\pgfqpoint{-6.568712in}{2.296342in}}%
\pgfpathlineto{\pgfqpoint{-6.640496in}{2.253604in}}%
\pgfpathlineto{\pgfqpoint{-6.715245in}{2.301860in}}%
\pgfpathlineto{\pgfqpoint{-6.787495in}{2.218280in}}%
\pgfpathlineto{\pgfqpoint{-6.860576in}{2.296715in}}%
\pgfpathlineto{\pgfqpoint{-6.933558in}{2.393918in}}%
\pgfpathlineto{\pgfqpoint{-7.003598in}{2.286684in}}%
\pgfpathlineto{\pgfqpoint{-7.075897in}{2.176033in}}%
\pgfpathlineto{\pgfqpoint{-7.151909in}{2.260240in}}%
\pgfpathlineto{\pgfqpoint{-7.223147in}{2.306604in}}%
\pgfpathlineto{\pgfqpoint{-7.294929in}{2.255320in}}%
\pgfpathlineto{\pgfqpoint{-7.368338in}{2.348892in}}%
\pgfpathlineto{\pgfqpoint{-7.440452in}{2.210574in}}%
\pgfpathlineto{\pgfqpoint{-7.512981in}{2.386143in}}%
\pgfpathlineto{\pgfqpoint{-7.585515in}{2.214818in}}%
\pgfpathlineto{\pgfqpoint{-7.657024in}{2.345307in}}%
\pgfpathlineto{\pgfqpoint{-7.727286in}{2.348982in}}%
\pgfpathlineto{\pgfqpoint{-7.799875in}{2.328198in}}%
\pgfpathlineto{\pgfqpoint{-7.870788in}{2.241376in}}%
\pgfpathlineto{\pgfqpoint{-7.942200in}{2.363105in}}%
\pgfpathlineto{\pgfqpoint{-8.014387in}{2.319420in}}%
\pgfpathlineto{\pgfqpoint{-8.084913in}{2.253040in}}%
\pgfpathlineto{\pgfqpoint{-8.156111in}{2.315437in}}%
\pgfpathlineto{\pgfqpoint{-8.228241in}{2.269897in}}%
\pgfpathlineto{\pgfqpoint{-8.298753in}{2.376160in}}%
\pgfpathlineto{\pgfqpoint{-8.367967in}{2.386374in}}%
\pgfpathlineto{\pgfqpoint{-8.438970in}{2.326259in}}%
\pgfpathlineto{\pgfqpoint{-8.508839in}{2.293230in}}%
\pgfpathlineto{\pgfqpoint{-8.578369in}{2.389573in}}%
\pgfpathlineto{\pgfqpoint{-8.649341in}{2.397494in}}%
\pgfpathlineto{\pgfqpoint{-8.718609in}{2.421459in}}%
\pgfpathlineto{\pgfqpoint{-8.787586in}{2.482801in}}%
\pgfpathlineto{\pgfqpoint{-8.858221in}{2.395373in}}%
\pgfpathlineto{\pgfqpoint{-8.928123in}{2.348659in}}%
\pgfpathlineto{\pgfqpoint{-8.998701in}{2.323391in}}%
\pgfpathlineto{\pgfqpoint{-9.069721in}{2.398083in}}%
\pgfpathlineto{\pgfqpoint{-9.138788in}{2.303340in}}%
\pgfpathlineto{\pgfqpoint{-9.208906in}{2.348390in}}%
\pgfpathlineto{\pgfqpoint{-9.283412in}{2.223836in}}%
\pgfpathlineto{\pgfqpoint{-9.356529in}{2.170512in}}%
\pgfpathlineto{\pgfqpoint{-9.429774in}{2.278881in}}%
\pgfpathlineto{\pgfqpoint{-9.504196in}{2.330770in}}%
\pgfpathlineto{\pgfqpoint{-9.575400in}{2.197014in}}%
\pgfpathlineto{\pgfqpoint{-9.648369in}{2.291199in}}%
\pgfpathlineto{\pgfqpoint{-9.723682in}{2.291422in}}%
\pgfpathlineto{\pgfqpoint{-9.796010in}{2.259136in}}%
\pgfpathlineto{\pgfqpoint{-9.868243in}{2.339697in}}%
\pgfpathlineto{\pgfqpoint{-9.941359in}{2.303436in}}%
\pgfpathlineto{\pgfqpoint{-10.012635in}{2.268173in}}%
\pgfpathlineto{\pgfqpoint{-10.084496in}{2.288976in}}%
\pgfpathlineto{\pgfqpoint{-10.157398in}{2.299263in}}%
\pgfpathlineto{\pgfqpoint{-10.227739in}{2.355684in}}%
\pgfpathlineto{\pgfqpoint{-10.297591in}{2.304203in}}%
\pgfpathlineto{\pgfqpoint{-10.371004in}{2.235383in}}%
\pgfpathlineto{\pgfqpoint{-10.441581in}{2.319899in}}%
\pgfpathlineto{\pgfqpoint{-10.511372in}{2.379865in}}%
\pgfpathlineto{\pgfqpoint{-10.582864in}{2.236644in}}%
\pgfpathlineto{\pgfqpoint{-10.652649in}{2.361138in}}%
\pgfpathlineto{\pgfqpoint{-10.721651in}{2.379466in}}%
\pgfpathlineto{\pgfqpoint{-10.793422in}{2.293630in}}%
\pgfpathlineto{\pgfqpoint{-10.863257in}{2.326817in}}%
\pgfpathlineto{\pgfqpoint{-10.933808in}{2.387675in}}%
\pgfpathlineto{\pgfqpoint{-11.005065in}{2.261538in}}%
\pgfpathlineto{\pgfqpoint{-11.075276in}{2.308828in}}%
\pgfpathlineto{\pgfqpoint{-11.144960in}{2.336694in}}%
\pgfpathlineto{\pgfqpoint{-11.217631in}{2.289116in}}%
\pgfpathlineto{\pgfqpoint{-11.288273in}{2.361737in}}%
\pgfpathlineto{\pgfqpoint{-11.358676in}{2.294331in}}%
\pgfpathlineto{\pgfqpoint{-11.430351in}{2.382111in}}%
\pgfpathlineto{\pgfqpoint{-11.500166in}{2.291057in}}%
\pgfpathlineto{\pgfqpoint{-11.571037in}{2.286178in}}%
\pgfpathlineto{\pgfqpoint{-11.645187in}{2.285374in}}%
\pgfpathlineto{\pgfqpoint{-11.716309in}{2.389725in}}%
\pgfpathlineto{\pgfqpoint{-11.786176in}{2.363164in}}%
\pgfpathlineto{\pgfqpoint{-11.858471in}{2.277444in}}%
\pgfpathlineto{\pgfqpoint{-11.928381in}{2.419361in}}%
\pgfpathlineto{\pgfqpoint{-11.998596in}{2.351876in}}%
\pgfpathlineto{\pgfqpoint{-12.069446in}{2.360552in}}%
\pgfpathlineto{\pgfqpoint{-12.139885in}{2.283026in}}%
\pgfpathlineto{\pgfqpoint{-12.211536in}{2.296885in}}%
\pgfpathlineto{\pgfqpoint{-12.285275in}{2.284268in}}%
\pgfpathlineto{\pgfqpoint{-12.357103in}{2.289248in}}%
\pgfpathlineto{\pgfqpoint{-12.428997in}{2.232984in}}%
\pgfpathlineto{\pgfqpoint{-12.503401in}{2.271599in}}%
\pgfpathlineto{\pgfqpoint{-12.574652in}{2.266246in}}%
\pgfpathlineto{\pgfqpoint{-12.645279in}{2.330027in}}%
\pgfpathlineto{\pgfqpoint{-12.718096in}{2.284510in}}%
\pgfpathlineto{\pgfqpoint{-12.788679in}{2.293835in}}%
\pgfpathlineto{\pgfqpoint{-12.859398in}{2.416355in}}%
\pgfpathlineto{\pgfqpoint{-12.932388in}{2.333533in}}%
\pgfpathlineto{\pgfqpoint{-13.002727in}{2.394626in}}%
\pgfpathlineto{\pgfqpoint{-13.073093in}{2.296630in}}%
\pgfpathlineto{\pgfqpoint{-13.145337in}{2.321433in}}%
\pgfpathlineto{\pgfqpoint{-13.216013in}{2.426745in}}%
\pgfpathlineto{\pgfqpoint{-13.285456in}{2.377914in}}%
\pgfpathlineto{\pgfqpoint{-13.356968in}{2.353926in}}%
\pgfpathlineto{\pgfqpoint{-13.426276in}{2.341081in}}%
\pgfpathlineto{\pgfqpoint{-13.496151in}{2.293209in}}%
\pgfpathlineto{\pgfqpoint{-13.568713in}{2.245730in}}%
\pgfpathlineto{\pgfqpoint{-13.639609in}{2.376504in}}%
\pgfpathlineto{\pgfqpoint{-13.708089in}{2.355139in}}%
\pgfpathlineto{\pgfqpoint{-13.779339in}{2.315892in}}%
\pgfpathlineto{\pgfqpoint{-13.848756in}{2.375946in}}%
\pgfpathlineto{\pgfqpoint{-13.917455in}{2.313263in}}%
\pgfpathlineto{\pgfqpoint{-13.988983in}{2.352403in}}%
\pgfpathlineto{\pgfqpoint{-14.057377in}{2.295254in}}%
\pgfpathlineto{\pgfqpoint{-14.125666in}{2.414408in}}%
\pgfpathlineto{\pgfqpoint{-14.195714in}{2.344504in}}%
\pgfpathlineto{\pgfqpoint{-14.263436in}{2.420723in}}%
\pgfpathlineto{\pgfqpoint{-14.331277in}{2.310083in}}%
\pgfpathlineto{\pgfqpoint{-14.401957in}{2.317372in}}%
\pgfpathlineto{\pgfqpoint{-14.471123in}{2.450638in}}%
\pgfpathlineto{\pgfqpoint{-14.538988in}{2.402088in}}%
\pgfpathlineto{\pgfqpoint{-14.609938in}{2.319482in}}%
\pgfpathlineto{\pgfqpoint{-14.679263in}{2.315337in}}%
\pgfpathlineto{\pgfqpoint{-14.748835in}{2.334003in}}%
\pgfpathlineto{\pgfqpoint{-14.820408in}{2.347150in}}%
\pgfpathlineto{\pgfqpoint{-14.889438in}{2.362100in}}%
\pgfpathlineto{\pgfqpoint{-14.959865in}{2.408326in}}%
\pgfpathlineto{\pgfqpoint{-15.032843in}{2.237324in}}%
\pgfpathlineto{\pgfqpoint{-15.106256in}{2.213369in}}%
\pgfpathlineto{\pgfqpoint{-15.179182in}{2.296087in}}%
\pgfpathlineto{\pgfqpoint{-15.253001in}{2.346079in}}%
\pgfpathlineto{\pgfqpoint{-15.323000in}{2.416211in}}%
\pgfpathlineto{\pgfqpoint{-15.392048in}{2.441511in}}%
\pgfpathlineto{\pgfqpoint{-15.463558in}{2.378177in}}%
\pgfpathlineto{\pgfqpoint{-15.532639in}{2.336319in}}%
\pgfpathlineto{\pgfqpoint{-15.602559in}{2.389314in}}%
\pgfpathlineto{\pgfqpoint{-15.673559in}{2.309932in}}%
\pgfpathlineto{\pgfqpoint{-15.744769in}{2.316651in}}%
\pgfpathlineto{\pgfqpoint{-15.814335in}{2.339944in}}%
\pgfpathlineto{\pgfqpoint{-15.886757in}{2.298817in}}%
\pgfpathlineto{\pgfqpoint{-15.956358in}{2.387626in}}%
\pgfpathlineto{\pgfqpoint{-16.024276in}{2.337079in}}%
\pgfpathlineto{\pgfqpoint{-16.094895in}{2.315581in}}%
\pgfpathlineto{\pgfqpoint{-16.163654in}{2.367244in}}%
\pgfpathlineto{\pgfqpoint{-16.232068in}{2.386687in}}%
\pgfpathlineto{\pgfqpoint{-16.302639in}{2.366103in}}%
\pgfpathlineto{\pgfqpoint{-16.369724in}{2.425014in}}%
\pgfpathlineto{\pgfqpoint{-16.437097in}{2.334152in}}%
\pgfpathlineto{\pgfqpoint{-16.507797in}{2.362664in}}%
\pgfpathlineto{\pgfqpoint{-16.575864in}{2.414390in}}%
\pgfpathlineto{\pgfqpoint{-16.644038in}{2.269316in}}%
\pgfpathlineto{\pgfqpoint{-16.715630in}{2.352508in}}%
\pgfpathlineto{\pgfqpoint{-16.784012in}{2.407033in}}%
\pgfpathlineto{\pgfqpoint{-16.852234in}{2.378815in}}%
\pgfpathlineto{\pgfqpoint{-16.921714in}{2.406563in}}%
\pgfpathlineto{\pgfqpoint{-16.989953in}{2.368556in}}%
\pgfpathlineto{\pgfqpoint{-17.058127in}{2.382148in}}%
\pgfpathlineto{\pgfqpoint{-17.128098in}{2.394395in}}%
\pgfpathlineto{\pgfqpoint{-17.196258in}{2.324319in}}%
\pgfpathlineto{\pgfqpoint{-17.265552in}{2.344426in}}%
\pgfpathlineto{\pgfqpoint{-17.336134in}{2.399801in}}%
\pgfpathlineto{\pgfqpoint{-17.402675in}{2.396819in}}%
\pgfpathlineto{\pgfqpoint{-17.470673in}{2.427453in}}%
\pgfpathlineto{\pgfqpoint{-17.539598in}{2.322965in}}%
\pgfpathlineto{\pgfqpoint{-17.607453in}{2.405673in}}%
\pgfpathlineto{\pgfqpoint{-17.675451in}{2.318313in}}%
\pgfpathlineto{\pgfqpoint{-17.748008in}{2.364455in}}%
\pgfpathlineto{\pgfqpoint{-17.816737in}{2.360084in}}%
\pgfpathlineto{\pgfqpoint{-17.885356in}{2.298388in}}%
\pgfpathlineto{\pgfqpoint{-17.957185in}{2.404581in}}%
\pgfpathlineto{\pgfqpoint{-18.025876in}{2.400693in}}%
\pgfpathlineto{\pgfqpoint{-18.094404in}{2.329708in}}%
\pgfpathlineto{\pgfqpoint{-18.165145in}{2.389023in}}%
\pgfpathlineto{\pgfqpoint{-18.232823in}{2.338118in}}%
\pgfpathlineto{\pgfqpoint{-18.301710in}{2.313702in}}%
\pgfpathlineto{\pgfqpoint{-18.373957in}{2.321556in}}%
\pgfpathlineto{\pgfqpoint{-18.442527in}{2.363697in}}%
\pgfpathlineto{\pgfqpoint{-18.511159in}{2.275523in}}%
\pgfpathlineto{\pgfqpoint{-18.583201in}{2.316575in}}%
\pgfpathlineto{\pgfqpoint{-18.652004in}{2.345595in}}%
\pgfpathlineto{\pgfqpoint{-18.721737in}{2.350302in}}%
\pgfpathlineto{\pgfqpoint{-18.792270in}{2.460769in}}%
\pgfpathlineto{\pgfqpoint{-18.859965in}{2.394316in}}%
\pgfpathlineto{\pgfqpoint{-18.928709in}{2.382185in}}%
\pgfpathlineto{\pgfqpoint{-18.999640in}{2.424792in}}%
\pgfpathlineto{\pgfqpoint{-19.068147in}{2.329892in}}%
\pgfpathlineto{\pgfqpoint{-19.135788in}{2.381820in}}%
\pgfpathlineto{\pgfqpoint{-19.205228in}{2.445924in}}%
\pgfpathlineto{\pgfqpoint{-19.271476in}{2.454963in}}%
\pgfpathlineto{\pgfqpoint{-19.338364in}{2.340878in}}%
\pgfpathlineto{\pgfqpoint{-19.407422in}{2.287899in}}%
\pgfpathlineto{\pgfqpoint{-19.475599in}{2.412443in}}%
\pgfpathlineto{\pgfqpoint{-19.543393in}{2.309380in}}%
\pgfpathlineto{\pgfqpoint{-19.613839in}{2.312124in}}%
\pgfpathlineto{\pgfqpoint{-19.680997in}{2.412976in}}%
\pgfpathlineto{\pgfqpoint{-19.747542in}{2.447359in}}%
\pgfpathlineto{\pgfqpoint{-19.815535in}{2.399387in}}%
\pgfpathlineto{\pgfqpoint{-19.882267in}{2.418078in}}%
\pgfpathlineto{\pgfqpoint{-19.949841in}{2.427688in}}%
\pgfpathlineto{\pgfqpoint{-20.019359in}{2.379878in}}%
\pgfpathlineto{\pgfqpoint{-20.086943in}{2.465238in}}%
\pgfpathlineto{\pgfqpoint{-20.154101in}{2.391542in}}%
\pgfpathlineto{\pgfqpoint{-20.223560in}{2.440319in}}%
\pgfpathlineto{\pgfqpoint{-20.290460in}{2.462632in}}%
\pgfpathlineto{\pgfqpoint{-20.357252in}{2.396114in}}%
\pgfpathlineto{\pgfqpoint{-20.427103in}{2.352777in}}%
\pgfpathlineto{\pgfqpoint{-20.496847in}{2.377512in}}%
\pgfpathlineto{\pgfqpoint{-20.565295in}{2.445029in}}%
\pgfpathlineto{\pgfqpoint{-20.636228in}{2.355493in}}%
\pgfpathlineto{\pgfqpoint{-20.705385in}{2.362275in}}%
\pgfpathlineto{\pgfqpoint{-20.774312in}{2.465290in}}%
\pgfpathlineto{\pgfqpoint{-20.844220in}{2.385366in}}%
\pgfpathlineto{\pgfqpoint{-20.913231in}{2.392576in}}%
\pgfpathlineto{\pgfqpoint{-20.982338in}{2.317130in}}%
\pgfpathlineto{\pgfqpoint{-21.053709in}{2.385212in}}%
\pgfpathlineto{\pgfqpoint{-21.123238in}{2.375232in}}%
\pgfpathlineto{\pgfqpoint{-21.191384in}{2.401431in}}%
\pgfpathlineto{\pgfqpoint{-21.261541in}{2.378244in}}%
\pgfpathlineto{\pgfqpoint{-21.330134in}{2.459106in}}%
\pgfpathlineto{\pgfqpoint{-21.398346in}{2.387850in}}%
\pgfpathlineto{\pgfqpoint{-21.470634in}{2.316230in}}%
\pgfpathlineto{\pgfqpoint{-21.539549in}{2.404361in}}%
\pgfpathlineto{\pgfqpoint{-21.607622in}{2.524428in}}%
\pgfpathlineto{\pgfqpoint{-21.676089in}{2.370141in}}%
\pgfpathlineto{\pgfqpoint{-21.743422in}{2.428756in}}%
\pgfpathlineto{\pgfqpoint{-21.811964in}{2.326536in}}%
\pgfpathlineto{\pgfqpoint{-21.881469in}{2.451919in}}%
\pgfpathlineto{\pgfqpoint{-21.948488in}{2.351056in}}%
\pgfpathlineto{\pgfqpoint{-22.016091in}{2.401471in}}%
\pgfpathlineto{\pgfqpoint{-22.084923in}{2.503416in}}%
\pgfpathlineto{\pgfqpoint{-22.150098in}{2.483410in}}%
\pgfpathlineto{\pgfqpoint{-22.217508in}{2.332717in}}%
\pgfpathlineto{\pgfqpoint{-22.286974in}{2.435613in}}%
\pgfpathlineto{\pgfqpoint{-22.353759in}{2.374259in}}%
\pgfpathlineto{\pgfqpoint{-22.421718in}{2.342457in}}%
\pgfpathlineto{\pgfqpoint{-22.492630in}{2.324022in}}%
\pgfpathlineto{\pgfqpoint{-22.560590in}{2.437245in}}%
\pgfpathlineto{\pgfqpoint{-22.627376in}{2.492165in}}%
\pgfpathlineto{\pgfqpoint{-22.696138in}{2.355079in}}%
\pgfpathlineto{\pgfqpoint{-22.764599in}{2.418926in}}%
\pgfpathlineto{\pgfqpoint{-22.831621in}{2.493017in}}%
\pgfpathlineto{\pgfqpoint{-22.900479in}{2.343993in}}%
\pgfpathlineto{\pgfqpoint{-22.968635in}{2.369451in}}%
\pgfpathlineto{\pgfqpoint{-23.037151in}{2.347538in}}%
\pgfpathlineto{\pgfqpoint{-23.107951in}{2.407523in}}%
\pgfpathlineto{\pgfqpoint{-23.175783in}{2.376527in}}%
\pgfpathlineto{\pgfqpoint{-23.243519in}{2.415966in}}%
\pgfpathlineto{\pgfqpoint{-23.314406in}{2.274930in}}%
\pgfpathlineto{\pgfqpoint{-23.383303in}{2.339596in}}%
\pgfpathlineto{\pgfqpoint{-23.451780in}{2.350531in}}%
\pgfpathlineto{\pgfqpoint{-23.523979in}{2.304644in}}%
\pgfpathlineto{\pgfqpoint{-23.594251in}{2.310678in}}%
\pgfpathlineto{\pgfqpoint{-23.664795in}{2.276659in}}%
\pgfpathlineto{\pgfqpoint{-23.739951in}{2.263590in}}%
\pgfpathlineto{\pgfqpoint{-23.810815in}{2.387356in}}%
\pgfpathlineto{\pgfqpoint{-23.881105in}{2.329237in}}%
\pgfpathlineto{\pgfqpoint{-23.953546in}{2.329323in}}%
\pgfpathlineto{\pgfqpoint{-24.025129in}{2.320950in}}%
\pgfpathlineto{\pgfqpoint{-24.096931in}{2.253215in}}%
\pgfpathlineto{\pgfqpoint{-24.172427in}{2.272963in}}%
\pgfpathlineto{\pgfqpoint{-24.243942in}{2.256817in}}%
\pgfpathlineto{\pgfqpoint{-24.313406in}{2.444538in}}%
\pgfpathlineto{\pgfqpoint{-24.381203in}{2.517744in}}%
\pgfpathlineto{\pgfqpoint{-24.445935in}{2.463778in}}%
\pgfpathlineto{\pgfqpoint{-24.511431in}{2.524475in}}%
\pgfpathlineto{\pgfqpoint{-24.578904in}{2.473005in}}%
\pgfpathlineto{\pgfqpoint{-24.645693in}{2.423239in}}%
\pgfpathlineto{\pgfqpoint{-24.712889in}{2.407795in}}%
\pgfpathlineto{\pgfqpoint{-24.781655in}{2.413310in}}%
\pgfpathlineto{\pgfqpoint{-24.847966in}{2.381557in}}%
\pgfpathlineto{\pgfqpoint{-24.916019in}{2.349013in}}%
\pgfpathlineto{\pgfqpoint{-24.984232in}{2.521477in}}%
\pgfpathlineto{\pgfqpoint{-25.049351in}{2.471125in}}%
\pgfpathlineto{\pgfqpoint{-25.114674in}{2.483194in}}%
\pgfpathlineto{\pgfqpoint{-25.182394in}{2.468254in}}%
\pgfpathlineto{\pgfqpoint{-25.248757in}{2.413452in}}%
\pgfpathlineto{\pgfqpoint{-25.315375in}{2.491128in}}%
\pgfpathlineto{\pgfqpoint{-25.382786in}{2.396602in}}%
\pgfpathlineto{\pgfqpoint{-25.448976in}{2.347344in}}%
\pgfpathlineto{\pgfqpoint{-25.515370in}{2.454770in}}%
\pgfpathlineto{\pgfqpoint{-25.583742in}{2.415040in}}%
\pgfpathlineto{\pgfqpoint{-25.650614in}{2.422676in}}%
\pgfpathlineto{\pgfqpoint{-25.717568in}{2.515321in}}%
\pgfpathlineto{\pgfqpoint{-25.785528in}{2.438440in}}%
\pgfpathlineto{\pgfqpoint{-25.851769in}{2.424659in}}%
\pgfpathlineto{\pgfqpoint{-25.919100in}{2.340892in}}%
\pgfpathlineto{\pgfqpoint{-25.990773in}{2.344049in}}%
\pgfpathlineto{\pgfqpoint{-26.059544in}{2.353124in}}%
\pgfpathlineto{\pgfqpoint{-26.128033in}{2.396953in}}%
\pgfpathlineto{\pgfqpoint{-26.198294in}{2.320139in}}%
\pgfpathlineto{\pgfqpoint{-26.268985in}{2.365402in}}%
\pgfpathlineto{\pgfqpoint{-26.338662in}{2.370042in}}%
\pgfpathlineto{\pgfqpoint{-26.410031in}{2.299670in}}%
\pgfpathlineto{\pgfqpoint{-26.479146in}{2.395244in}}%
\pgfpathlineto{\pgfqpoint{-26.547093in}{2.392115in}}%
\pgfpathlineto{\pgfqpoint{-26.616985in}{2.344606in}}%
\pgfpathlineto{\pgfqpoint{-26.685925in}{2.310785in}}%
\pgfpathlineto{\pgfqpoint{-26.755672in}{2.338156in}}%
\pgfpathlineto{\pgfqpoint{-26.825734in}{2.472786in}}%
\pgfpathlineto{\pgfqpoint{-26.893662in}{2.369430in}}%
\pgfpathlineto{\pgfqpoint{-26.961733in}{2.366629in}}%
\pgfpathlineto{\pgfqpoint{-27.031677in}{2.389366in}}%
\pgfpathlineto{\pgfqpoint{-27.097972in}{2.420274in}}%
\pgfpathlineto{\pgfqpoint{-27.165061in}{2.441472in}}%
\pgfpathlineto{\pgfqpoint{-27.232937in}{2.440327in}}%
\pgfpathlineto{\pgfqpoint{-27.299189in}{2.432371in}}%
\pgfpathlineto{\pgfqpoint{-27.366087in}{2.363470in}}%
\pgfpathlineto{\pgfqpoint{-27.435671in}{2.371017in}}%
\pgfpathlineto{\pgfqpoint{-27.503250in}{2.276365in}}%
\pgfpathlineto{\pgfqpoint{-27.570291in}{2.406194in}}%
\pgfpathlineto{\pgfqpoint{-27.637925in}{2.500852in}}%
\pgfpathlineto{\pgfqpoint{-27.703951in}{2.400018in}}%
\pgfpathlineto{\pgfqpoint{-27.771032in}{2.459129in}}%
\pgfpathlineto{\pgfqpoint{-27.839702in}{2.399915in}}%
\pgfpathlineto{\pgfqpoint{-27.908279in}{2.356364in}}%
\pgfpathlineto{\pgfqpoint{-27.976694in}{2.415234in}}%
\pgfpathlineto{\pgfqpoint{-28.049197in}{2.293216in}}%
\pgfpathlineto{\pgfqpoint{-28.121663in}{2.273638in}}%
\pgfpathlineto{\pgfqpoint{-28.195772in}{2.204051in}}%
\pgfpathlineto{\pgfqpoint{-28.274638in}{2.117445in}}%
\pgfpathlineto{\pgfqpoint{-28.356803in}{2.282530in}}%
\pgfpathlineto{\pgfqpoint{-28.429123in}{2.295769in}}%
\pgfpathlineto{\pgfqpoint{-28.504596in}{2.262722in}}%
\pgfpathlineto{\pgfqpoint{-28.577221in}{2.336790in}}%
\pgfpathlineto{\pgfqpoint{-28.649563in}{2.185241in}}%
\pgfpathlineto{\pgfqpoint{-28.725736in}{2.273706in}}%
\pgfpathlineto{\pgfqpoint{-28.798746in}{2.296750in}}%
\pgfpathlineto{\pgfqpoint{-28.870906in}{2.252813in}}%
\pgfpathlineto{\pgfqpoint{-28.946213in}{2.286411in}}%
\pgfpathlineto{\pgfqpoint{-29.018119in}{2.313522in}}%
\pgfpathlineto{\pgfqpoint{-29.089467in}{2.298530in}}%
\pgfpathlineto{\pgfqpoint{-29.164288in}{2.273118in}}%
\pgfpathlineto{\pgfqpoint{-29.235084in}{2.357001in}}%
\pgfpathlineto{\pgfqpoint{-29.304065in}{2.352023in}}%
\pgfpathlineto{\pgfqpoint{-29.374900in}{2.301583in}}%
\pgfpathlineto{\pgfqpoint{-29.443353in}{2.404848in}}%
\pgfpathlineto{\pgfqpoint{-29.511058in}{2.398977in}}%
\pgfpathlineto{\pgfqpoint{-29.581285in}{2.404193in}}%
\pgfpathlineto{\pgfqpoint{-29.647778in}{2.394984in}}%
\pgfpathlineto{\pgfqpoint{-29.716346in}{2.409824in}}%
\pgfpathlineto{\pgfqpoint{-29.785834in}{2.393356in}}%
\pgfpathlineto{\pgfqpoint{-29.853575in}{2.334921in}}%
\pgfpathlineto{\pgfqpoint{-29.923374in}{2.310003in}}%
\pgfpathlineto{\pgfqpoint{-29.994776in}{2.341186in}}%
\pgfpathlineto{\pgfqpoint{-30.062931in}{2.418612in}}%
\pgfpathlineto{\pgfqpoint{-30.131142in}{2.337585in}}%
\pgfpathlineto{\pgfqpoint{-30.202864in}{2.416360in}}%
\pgfpathlineto{\pgfqpoint{-30.269201in}{2.417344in}}%
\pgfpathlineto{\pgfqpoint{-30.335915in}{2.395361in}}%
\pgfpathlineto{\pgfqpoint{-30.404868in}{2.342344in}}%
\pgfpathlineto{\pgfqpoint{-30.472294in}{2.407828in}}%
\pgfpathlineto{\pgfqpoint{-30.539383in}{2.339506in}}%
\pgfpathlineto{\pgfqpoint{-30.610127in}{2.361029in}}%
\pgfpathlineto{\pgfqpoint{-30.676901in}{2.447378in}}%
\pgfpathlineto{\pgfqpoint{-30.745143in}{1.768207in}}%
\pgfpathclose%
\pgfusepath{fill}%
\end{pgfscope}%
\begin{pgfscope}%
\pgfpathrectangle{\pgfqpoint{3.332180in}{0.773588in}}{\pgfqpoint{2.293918in}{5.415119in}}%
\pgfusepath{clip}%
\pgfsetbuttcap%
\pgfsetroundjoin%
\definecolor{currentfill}{rgb}{0.549020,0.337255,0.294118}%
\pgfsetfillcolor{currentfill}%
\pgfsetlinewidth{0.000000pt}%
\definecolor{currentstroke}{rgb}{0.000000,0.000000,0.000000}%
\pgfsetstrokecolor{currentstroke}%
\pgfsetdash{}{0pt}%
\pgfpathmoveto{\pgfqpoint{-30.745143in}{2.186873in}}%
\pgfpathlineto{\pgfqpoint{-30.745143in}{1.768207in}}%
\pgfpathlineto{\pgfqpoint{-30.676901in}{2.447378in}}%
\pgfpathlineto{\pgfqpoint{-30.610127in}{2.361029in}}%
\pgfpathlineto{\pgfqpoint{-30.539383in}{2.339506in}}%
\pgfpathlineto{\pgfqpoint{-30.472294in}{2.407828in}}%
\pgfpathlineto{\pgfqpoint{-30.404868in}{2.342344in}}%
\pgfpathlineto{\pgfqpoint{-30.335915in}{2.395361in}}%
\pgfpathlineto{\pgfqpoint{-30.269201in}{2.417344in}}%
\pgfpathlineto{\pgfqpoint{-30.202864in}{2.416360in}}%
\pgfpathlineto{\pgfqpoint{-30.131142in}{2.337585in}}%
\pgfpathlineto{\pgfqpoint{-30.062931in}{2.418612in}}%
\pgfpathlineto{\pgfqpoint{-29.994776in}{2.341186in}}%
\pgfpathlineto{\pgfqpoint{-29.923374in}{2.310003in}}%
\pgfpathlineto{\pgfqpoint{-29.853575in}{2.334921in}}%
\pgfpathlineto{\pgfqpoint{-29.785834in}{2.393356in}}%
\pgfpathlineto{\pgfqpoint{-29.716346in}{2.409824in}}%
\pgfpathlineto{\pgfqpoint{-29.647778in}{2.394984in}}%
\pgfpathlineto{\pgfqpoint{-29.581285in}{2.404193in}}%
\pgfpathlineto{\pgfqpoint{-29.511058in}{2.398977in}}%
\pgfpathlineto{\pgfqpoint{-29.443353in}{2.404848in}}%
\pgfpathlineto{\pgfqpoint{-29.374900in}{2.301583in}}%
\pgfpathlineto{\pgfqpoint{-29.304065in}{2.352023in}}%
\pgfpathlineto{\pgfqpoint{-29.235084in}{2.357001in}}%
\pgfpathlineto{\pgfqpoint{-29.164288in}{2.273118in}}%
\pgfpathlineto{\pgfqpoint{-29.089467in}{2.298530in}}%
\pgfpathlineto{\pgfqpoint{-29.018119in}{2.313522in}}%
\pgfpathlineto{\pgfqpoint{-28.946213in}{2.286411in}}%
\pgfpathlineto{\pgfqpoint{-28.870906in}{2.252813in}}%
\pgfpathlineto{\pgfqpoint{-28.798746in}{2.296750in}}%
\pgfpathlineto{\pgfqpoint{-28.725736in}{2.273706in}}%
\pgfpathlineto{\pgfqpoint{-28.649563in}{2.185241in}}%
\pgfpathlineto{\pgfqpoint{-28.577221in}{2.336790in}}%
\pgfpathlineto{\pgfqpoint{-28.504596in}{2.262722in}}%
\pgfpathlineto{\pgfqpoint{-28.429123in}{2.295769in}}%
\pgfpathlineto{\pgfqpoint{-28.356803in}{2.282530in}}%
\pgfpathlineto{\pgfqpoint{-28.274638in}{2.117445in}}%
\pgfpathlineto{\pgfqpoint{-28.195772in}{2.204051in}}%
\pgfpathlineto{\pgfqpoint{-28.121663in}{2.273638in}}%
\pgfpathlineto{\pgfqpoint{-28.049197in}{2.293216in}}%
\pgfpathlineto{\pgfqpoint{-27.976694in}{2.415234in}}%
\pgfpathlineto{\pgfqpoint{-27.908279in}{2.356364in}}%
\pgfpathlineto{\pgfqpoint{-27.839702in}{2.399915in}}%
\pgfpathlineto{\pgfqpoint{-27.771032in}{2.459129in}}%
\pgfpathlineto{\pgfqpoint{-27.703951in}{2.400018in}}%
\pgfpathlineto{\pgfqpoint{-27.637925in}{2.500852in}}%
\pgfpathlineto{\pgfqpoint{-27.570291in}{2.406194in}}%
\pgfpathlineto{\pgfqpoint{-27.503250in}{2.276365in}}%
\pgfpathlineto{\pgfqpoint{-27.435671in}{2.371017in}}%
\pgfpathlineto{\pgfqpoint{-27.366087in}{2.363470in}}%
\pgfpathlineto{\pgfqpoint{-27.299189in}{2.432371in}}%
\pgfpathlineto{\pgfqpoint{-27.232937in}{2.440327in}}%
\pgfpathlineto{\pgfqpoint{-27.165061in}{2.441472in}}%
\pgfpathlineto{\pgfqpoint{-27.097972in}{2.420274in}}%
\pgfpathlineto{\pgfqpoint{-27.031677in}{2.389366in}}%
\pgfpathlineto{\pgfqpoint{-26.961733in}{2.366629in}}%
\pgfpathlineto{\pgfqpoint{-26.893662in}{2.369430in}}%
\pgfpathlineto{\pgfqpoint{-26.825734in}{2.472786in}}%
\pgfpathlineto{\pgfqpoint{-26.755672in}{2.338156in}}%
\pgfpathlineto{\pgfqpoint{-26.685925in}{2.310785in}}%
\pgfpathlineto{\pgfqpoint{-26.616985in}{2.344606in}}%
\pgfpathlineto{\pgfqpoint{-26.547093in}{2.392115in}}%
\pgfpathlineto{\pgfqpoint{-26.479146in}{2.395244in}}%
\pgfpathlineto{\pgfqpoint{-26.410031in}{2.299670in}}%
\pgfpathlineto{\pgfqpoint{-26.338662in}{2.370042in}}%
\pgfpathlineto{\pgfqpoint{-26.268985in}{2.365402in}}%
\pgfpathlineto{\pgfqpoint{-26.198294in}{2.320139in}}%
\pgfpathlineto{\pgfqpoint{-26.128033in}{2.396953in}}%
\pgfpathlineto{\pgfqpoint{-26.059544in}{2.353124in}}%
\pgfpathlineto{\pgfqpoint{-25.990773in}{2.344049in}}%
\pgfpathlineto{\pgfqpoint{-25.919100in}{2.340892in}}%
\pgfpathlineto{\pgfqpoint{-25.851769in}{2.424659in}}%
\pgfpathlineto{\pgfqpoint{-25.785528in}{2.438440in}}%
\pgfpathlineto{\pgfqpoint{-25.717568in}{2.515321in}}%
\pgfpathlineto{\pgfqpoint{-25.650614in}{2.422676in}}%
\pgfpathlineto{\pgfqpoint{-25.583742in}{2.415040in}}%
\pgfpathlineto{\pgfqpoint{-25.515370in}{2.454770in}}%
\pgfpathlineto{\pgfqpoint{-25.448976in}{2.347344in}}%
\pgfpathlineto{\pgfqpoint{-25.382786in}{2.396602in}}%
\pgfpathlineto{\pgfqpoint{-25.315375in}{2.491128in}}%
\pgfpathlineto{\pgfqpoint{-25.248757in}{2.413452in}}%
\pgfpathlineto{\pgfqpoint{-25.182394in}{2.468254in}}%
\pgfpathlineto{\pgfqpoint{-25.114674in}{2.483194in}}%
\pgfpathlineto{\pgfqpoint{-25.049351in}{2.471125in}}%
\pgfpathlineto{\pgfqpoint{-24.984232in}{2.521477in}}%
\pgfpathlineto{\pgfqpoint{-24.916019in}{2.349013in}}%
\pgfpathlineto{\pgfqpoint{-24.847966in}{2.381557in}}%
\pgfpathlineto{\pgfqpoint{-24.781655in}{2.413310in}}%
\pgfpathlineto{\pgfqpoint{-24.712889in}{2.407795in}}%
\pgfpathlineto{\pgfqpoint{-24.645693in}{2.423239in}}%
\pgfpathlineto{\pgfqpoint{-24.578904in}{2.473005in}}%
\pgfpathlineto{\pgfqpoint{-24.511431in}{2.524475in}}%
\pgfpathlineto{\pgfqpoint{-24.445935in}{2.463778in}}%
\pgfpathlineto{\pgfqpoint{-24.381203in}{2.517744in}}%
\pgfpathlineto{\pgfqpoint{-24.313406in}{2.444538in}}%
\pgfpathlineto{\pgfqpoint{-24.243942in}{2.256817in}}%
\pgfpathlineto{\pgfqpoint{-24.172427in}{2.272963in}}%
\pgfpathlineto{\pgfqpoint{-24.096931in}{2.253215in}}%
\pgfpathlineto{\pgfqpoint{-24.025129in}{2.320950in}}%
\pgfpathlineto{\pgfqpoint{-23.953546in}{2.329323in}}%
\pgfpathlineto{\pgfqpoint{-23.881105in}{2.329237in}}%
\pgfpathlineto{\pgfqpoint{-23.810815in}{2.387356in}}%
\pgfpathlineto{\pgfqpoint{-23.739951in}{2.263590in}}%
\pgfpathlineto{\pgfqpoint{-23.664795in}{2.276659in}}%
\pgfpathlineto{\pgfqpoint{-23.594251in}{2.310678in}}%
\pgfpathlineto{\pgfqpoint{-23.523979in}{2.304644in}}%
\pgfpathlineto{\pgfqpoint{-23.451780in}{2.350531in}}%
\pgfpathlineto{\pgfqpoint{-23.383303in}{2.339596in}}%
\pgfpathlineto{\pgfqpoint{-23.314406in}{2.274930in}}%
\pgfpathlineto{\pgfqpoint{-23.243519in}{2.415966in}}%
\pgfpathlineto{\pgfqpoint{-23.175783in}{2.376527in}}%
\pgfpathlineto{\pgfqpoint{-23.107951in}{2.407523in}}%
\pgfpathlineto{\pgfqpoint{-23.037151in}{2.347538in}}%
\pgfpathlineto{\pgfqpoint{-22.968635in}{2.369451in}}%
\pgfpathlineto{\pgfqpoint{-22.900479in}{2.343993in}}%
\pgfpathlineto{\pgfqpoint{-22.831621in}{2.493017in}}%
\pgfpathlineto{\pgfqpoint{-22.764599in}{2.418926in}}%
\pgfpathlineto{\pgfqpoint{-22.696138in}{2.355079in}}%
\pgfpathlineto{\pgfqpoint{-22.627376in}{2.492165in}}%
\pgfpathlineto{\pgfqpoint{-22.560590in}{2.437245in}}%
\pgfpathlineto{\pgfqpoint{-22.492630in}{2.324022in}}%
\pgfpathlineto{\pgfqpoint{-22.421718in}{2.342457in}}%
\pgfpathlineto{\pgfqpoint{-22.353759in}{2.374259in}}%
\pgfpathlineto{\pgfqpoint{-22.286974in}{2.435613in}}%
\pgfpathlineto{\pgfqpoint{-22.217508in}{2.332717in}}%
\pgfpathlineto{\pgfqpoint{-22.150098in}{2.483410in}}%
\pgfpathlineto{\pgfqpoint{-22.084923in}{2.503416in}}%
\pgfpathlineto{\pgfqpoint{-22.016091in}{2.401471in}}%
\pgfpathlineto{\pgfqpoint{-21.948488in}{2.351056in}}%
\pgfpathlineto{\pgfqpoint{-21.881469in}{2.451919in}}%
\pgfpathlineto{\pgfqpoint{-21.811964in}{2.326536in}}%
\pgfpathlineto{\pgfqpoint{-21.743422in}{2.428756in}}%
\pgfpathlineto{\pgfqpoint{-21.676089in}{2.370141in}}%
\pgfpathlineto{\pgfqpoint{-21.607622in}{2.524428in}}%
\pgfpathlineto{\pgfqpoint{-21.539549in}{2.404361in}}%
\pgfpathlineto{\pgfqpoint{-21.470634in}{2.316230in}}%
\pgfpathlineto{\pgfqpoint{-21.398346in}{2.387850in}}%
\pgfpathlineto{\pgfqpoint{-21.330134in}{2.459106in}}%
\pgfpathlineto{\pgfqpoint{-21.261541in}{2.378244in}}%
\pgfpathlineto{\pgfqpoint{-21.191384in}{2.401431in}}%
\pgfpathlineto{\pgfqpoint{-21.123238in}{2.375232in}}%
\pgfpathlineto{\pgfqpoint{-21.053709in}{2.385212in}}%
\pgfpathlineto{\pgfqpoint{-20.982338in}{2.317130in}}%
\pgfpathlineto{\pgfqpoint{-20.913231in}{2.392576in}}%
\pgfpathlineto{\pgfqpoint{-20.844220in}{2.385366in}}%
\pgfpathlineto{\pgfqpoint{-20.774312in}{2.465290in}}%
\pgfpathlineto{\pgfqpoint{-20.705385in}{2.362275in}}%
\pgfpathlineto{\pgfqpoint{-20.636228in}{2.355493in}}%
\pgfpathlineto{\pgfqpoint{-20.565295in}{2.445029in}}%
\pgfpathlineto{\pgfqpoint{-20.496847in}{2.377512in}}%
\pgfpathlineto{\pgfqpoint{-20.427103in}{2.352777in}}%
\pgfpathlineto{\pgfqpoint{-20.357252in}{2.396114in}}%
\pgfpathlineto{\pgfqpoint{-20.290460in}{2.462632in}}%
\pgfpathlineto{\pgfqpoint{-20.223560in}{2.440319in}}%
\pgfpathlineto{\pgfqpoint{-20.154101in}{2.391542in}}%
\pgfpathlineto{\pgfqpoint{-20.086943in}{2.465238in}}%
\pgfpathlineto{\pgfqpoint{-20.019359in}{2.379878in}}%
\pgfpathlineto{\pgfqpoint{-19.949841in}{2.427688in}}%
\pgfpathlineto{\pgfqpoint{-19.882267in}{2.418078in}}%
\pgfpathlineto{\pgfqpoint{-19.815535in}{2.399387in}}%
\pgfpathlineto{\pgfqpoint{-19.747542in}{2.447359in}}%
\pgfpathlineto{\pgfqpoint{-19.680997in}{2.412976in}}%
\pgfpathlineto{\pgfqpoint{-19.613839in}{2.312124in}}%
\pgfpathlineto{\pgfqpoint{-19.543393in}{2.309380in}}%
\pgfpathlineto{\pgfqpoint{-19.475599in}{2.412443in}}%
\pgfpathlineto{\pgfqpoint{-19.407422in}{2.287899in}}%
\pgfpathlineto{\pgfqpoint{-19.338364in}{2.340878in}}%
\pgfpathlineto{\pgfqpoint{-19.271476in}{2.454963in}}%
\pgfpathlineto{\pgfqpoint{-19.205228in}{2.445924in}}%
\pgfpathlineto{\pgfqpoint{-19.135788in}{2.381820in}}%
\pgfpathlineto{\pgfqpoint{-19.068147in}{2.329892in}}%
\pgfpathlineto{\pgfqpoint{-18.999640in}{2.424792in}}%
\pgfpathlineto{\pgfqpoint{-18.928709in}{2.382185in}}%
\pgfpathlineto{\pgfqpoint{-18.859965in}{2.394316in}}%
\pgfpathlineto{\pgfqpoint{-18.792270in}{2.460769in}}%
\pgfpathlineto{\pgfqpoint{-18.721737in}{2.350302in}}%
\pgfpathlineto{\pgfqpoint{-18.652004in}{2.345595in}}%
\pgfpathlineto{\pgfqpoint{-18.583201in}{2.316575in}}%
\pgfpathlineto{\pgfqpoint{-18.511159in}{2.275523in}}%
\pgfpathlineto{\pgfqpoint{-18.442527in}{2.363697in}}%
\pgfpathlineto{\pgfqpoint{-18.373957in}{2.321556in}}%
\pgfpathlineto{\pgfqpoint{-18.301710in}{2.313702in}}%
\pgfpathlineto{\pgfqpoint{-18.232823in}{2.338118in}}%
\pgfpathlineto{\pgfqpoint{-18.165145in}{2.389023in}}%
\pgfpathlineto{\pgfqpoint{-18.094404in}{2.329708in}}%
\pgfpathlineto{\pgfqpoint{-18.025876in}{2.400693in}}%
\pgfpathlineto{\pgfqpoint{-17.957185in}{2.404581in}}%
\pgfpathlineto{\pgfqpoint{-17.885356in}{2.298388in}}%
\pgfpathlineto{\pgfqpoint{-17.816737in}{2.360084in}}%
\pgfpathlineto{\pgfqpoint{-17.748008in}{2.364455in}}%
\pgfpathlineto{\pgfqpoint{-17.675451in}{2.318313in}}%
\pgfpathlineto{\pgfqpoint{-17.607453in}{2.405673in}}%
\pgfpathlineto{\pgfqpoint{-17.539598in}{2.322965in}}%
\pgfpathlineto{\pgfqpoint{-17.470673in}{2.427453in}}%
\pgfpathlineto{\pgfqpoint{-17.402675in}{2.396819in}}%
\pgfpathlineto{\pgfqpoint{-17.336134in}{2.399801in}}%
\pgfpathlineto{\pgfqpoint{-17.265552in}{2.344426in}}%
\pgfpathlineto{\pgfqpoint{-17.196258in}{2.324319in}}%
\pgfpathlineto{\pgfqpoint{-17.128098in}{2.394395in}}%
\pgfpathlineto{\pgfqpoint{-17.058127in}{2.382148in}}%
\pgfpathlineto{\pgfqpoint{-16.989953in}{2.368556in}}%
\pgfpathlineto{\pgfqpoint{-16.921714in}{2.406563in}}%
\pgfpathlineto{\pgfqpoint{-16.852234in}{2.378815in}}%
\pgfpathlineto{\pgfqpoint{-16.784012in}{2.407033in}}%
\pgfpathlineto{\pgfqpoint{-16.715630in}{2.352508in}}%
\pgfpathlineto{\pgfqpoint{-16.644038in}{2.269316in}}%
\pgfpathlineto{\pgfqpoint{-16.575864in}{2.414390in}}%
\pgfpathlineto{\pgfqpoint{-16.507797in}{2.362664in}}%
\pgfpathlineto{\pgfqpoint{-16.437097in}{2.334152in}}%
\pgfpathlineto{\pgfqpoint{-16.369724in}{2.425014in}}%
\pgfpathlineto{\pgfqpoint{-16.302639in}{2.366103in}}%
\pgfpathlineto{\pgfqpoint{-16.232068in}{2.386687in}}%
\pgfpathlineto{\pgfqpoint{-16.163654in}{2.367244in}}%
\pgfpathlineto{\pgfqpoint{-16.094895in}{2.315581in}}%
\pgfpathlineto{\pgfqpoint{-16.024276in}{2.337079in}}%
\pgfpathlineto{\pgfqpoint{-15.956358in}{2.387626in}}%
\pgfpathlineto{\pgfqpoint{-15.886757in}{2.298817in}}%
\pgfpathlineto{\pgfqpoint{-15.814335in}{2.339944in}}%
\pgfpathlineto{\pgfqpoint{-15.744769in}{2.316651in}}%
\pgfpathlineto{\pgfqpoint{-15.673559in}{2.309932in}}%
\pgfpathlineto{\pgfqpoint{-15.602559in}{2.389314in}}%
\pgfpathlineto{\pgfqpoint{-15.532639in}{2.336319in}}%
\pgfpathlineto{\pgfqpoint{-15.463558in}{2.378177in}}%
\pgfpathlineto{\pgfqpoint{-15.392048in}{2.441511in}}%
\pgfpathlineto{\pgfqpoint{-15.323000in}{2.416211in}}%
\pgfpathlineto{\pgfqpoint{-15.253001in}{2.346079in}}%
\pgfpathlineto{\pgfqpoint{-15.179182in}{2.296087in}}%
\pgfpathlineto{\pgfqpoint{-15.106256in}{2.213369in}}%
\pgfpathlineto{\pgfqpoint{-15.032843in}{2.237324in}}%
\pgfpathlineto{\pgfqpoint{-14.959865in}{2.408326in}}%
\pgfpathlineto{\pgfqpoint{-14.889438in}{2.362100in}}%
\pgfpathlineto{\pgfqpoint{-14.820408in}{2.347150in}}%
\pgfpathlineto{\pgfqpoint{-14.748835in}{2.334003in}}%
\pgfpathlineto{\pgfqpoint{-14.679263in}{2.315337in}}%
\pgfpathlineto{\pgfqpoint{-14.609938in}{2.319482in}}%
\pgfpathlineto{\pgfqpoint{-14.538988in}{2.402088in}}%
\pgfpathlineto{\pgfqpoint{-14.471123in}{2.450638in}}%
\pgfpathlineto{\pgfqpoint{-14.401957in}{2.317372in}}%
\pgfpathlineto{\pgfqpoint{-14.331277in}{2.310083in}}%
\pgfpathlineto{\pgfqpoint{-14.263436in}{2.420723in}}%
\pgfpathlineto{\pgfqpoint{-14.195714in}{2.344504in}}%
\pgfpathlineto{\pgfqpoint{-14.125666in}{2.414408in}}%
\pgfpathlineto{\pgfqpoint{-14.057377in}{2.295254in}}%
\pgfpathlineto{\pgfqpoint{-13.988983in}{2.352403in}}%
\pgfpathlineto{\pgfqpoint{-13.917455in}{2.313263in}}%
\pgfpathlineto{\pgfqpoint{-13.848756in}{2.375946in}}%
\pgfpathlineto{\pgfqpoint{-13.779339in}{2.315892in}}%
\pgfpathlineto{\pgfqpoint{-13.708089in}{2.355139in}}%
\pgfpathlineto{\pgfqpoint{-13.639609in}{2.376504in}}%
\pgfpathlineto{\pgfqpoint{-13.568713in}{2.245730in}}%
\pgfpathlineto{\pgfqpoint{-13.496151in}{2.293209in}}%
\pgfpathlineto{\pgfqpoint{-13.426276in}{2.341081in}}%
\pgfpathlineto{\pgfqpoint{-13.356968in}{2.353926in}}%
\pgfpathlineto{\pgfqpoint{-13.285456in}{2.377914in}}%
\pgfpathlineto{\pgfqpoint{-13.216013in}{2.426745in}}%
\pgfpathlineto{\pgfqpoint{-13.145337in}{2.321433in}}%
\pgfpathlineto{\pgfqpoint{-13.073093in}{2.296630in}}%
\pgfpathlineto{\pgfqpoint{-13.002727in}{2.394626in}}%
\pgfpathlineto{\pgfqpoint{-12.932388in}{2.333533in}}%
\pgfpathlineto{\pgfqpoint{-12.859398in}{2.416355in}}%
\pgfpathlineto{\pgfqpoint{-12.788679in}{2.293835in}}%
\pgfpathlineto{\pgfqpoint{-12.718096in}{2.284510in}}%
\pgfpathlineto{\pgfqpoint{-12.645279in}{2.330027in}}%
\pgfpathlineto{\pgfqpoint{-12.574652in}{2.266246in}}%
\pgfpathlineto{\pgfqpoint{-12.503401in}{2.271599in}}%
\pgfpathlineto{\pgfqpoint{-12.428997in}{2.232984in}}%
\pgfpathlineto{\pgfqpoint{-12.357103in}{2.289248in}}%
\pgfpathlineto{\pgfqpoint{-12.285275in}{2.284268in}}%
\pgfpathlineto{\pgfqpoint{-12.211536in}{2.296885in}}%
\pgfpathlineto{\pgfqpoint{-12.139885in}{2.283026in}}%
\pgfpathlineto{\pgfqpoint{-12.069446in}{2.360552in}}%
\pgfpathlineto{\pgfqpoint{-11.998596in}{2.351876in}}%
\pgfpathlineto{\pgfqpoint{-11.928381in}{2.419361in}}%
\pgfpathlineto{\pgfqpoint{-11.858471in}{2.277444in}}%
\pgfpathlineto{\pgfqpoint{-11.786176in}{2.363164in}}%
\pgfpathlineto{\pgfqpoint{-11.716309in}{2.389725in}}%
\pgfpathlineto{\pgfqpoint{-11.645187in}{2.285374in}}%
\pgfpathlineto{\pgfqpoint{-11.571037in}{2.286178in}}%
\pgfpathlineto{\pgfqpoint{-11.500166in}{2.291057in}}%
\pgfpathlineto{\pgfqpoint{-11.430351in}{2.382111in}}%
\pgfpathlineto{\pgfqpoint{-11.358676in}{2.294331in}}%
\pgfpathlineto{\pgfqpoint{-11.288273in}{2.361737in}}%
\pgfpathlineto{\pgfqpoint{-11.217631in}{2.289116in}}%
\pgfpathlineto{\pgfqpoint{-11.144960in}{2.336694in}}%
\pgfpathlineto{\pgfqpoint{-11.075276in}{2.308828in}}%
\pgfpathlineto{\pgfqpoint{-11.005065in}{2.261538in}}%
\pgfpathlineto{\pgfqpoint{-10.933808in}{2.387675in}}%
\pgfpathlineto{\pgfqpoint{-10.863257in}{2.326817in}}%
\pgfpathlineto{\pgfqpoint{-10.793422in}{2.293630in}}%
\pgfpathlineto{\pgfqpoint{-10.721651in}{2.379466in}}%
\pgfpathlineto{\pgfqpoint{-10.652649in}{2.361138in}}%
\pgfpathlineto{\pgfqpoint{-10.582864in}{2.236644in}}%
\pgfpathlineto{\pgfqpoint{-10.511372in}{2.379865in}}%
\pgfpathlineto{\pgfqpoint{-10.441581in}{2.319899in}}%
\pgfpathlineto{\pgfqpoint{-10.371004in}{2.235383in}}%
\pgfpathlineto{\pgfqpoint{-10.297591in}{2.304203in}}%
\pgfpathlineto{\pgfqpoint{-10.227739in}{2.355684in}}%
\pgfpathlineto{\pgfqpoint{-10.157398in}{2.299263in}}%
\pgfpathlineto{\pgfqpoint{-10.084496in}{2.288976in}}%
\pgfpathlineto{\pgfqpoint{-10.012635in}{2.268173in}}%
\pgfpathlineto{\pgfqpoint{-9.941359in}{2.303436in}}%
\pgfpathlineto{\pgfqpoint{-9.868243in}{2.339697in}}%
\pgfpathlineto{\pgfqpoint{-9.796010in}{2.259136in}}%
\pgfpathlineto{\pgfqpoint{-9.723682in}{2.291422in}}%
\pgfpathlineto{\pgfqpoint{-9.648369in}{2.291199in}}%
\pgfpathlineto{\pgfqpoint{-9.575400in}{2.197014in}}%
\pgfpathlineto{\pgfqpoint{-9.504196in}{2.330770in}}%
\pgfpathlineto{\pgfqpoint{-9.429774in}{2.278881in}}%
\pgfpathlineto{\pgfqpoint{-9.356529in}{2.170512in}}%
\pgfpathlineto{\pgfqpoint{-9.283412in}{2.223836in}}%
\pgfpathlineto{\pgfqpoint{-9.208906in}{2.348390in}}%
\pgfpathlineto{\pgfqpoint{-9.138788in}{2.303340in}}%
\pgfpathlineto{\pgfqpoint{-9.069721in}{2.398083in}}%
\pgfpathlineto{\pgfqpoint{-8.998701in}{2.323391in}}%
\pgfpathlineto{\pgfqpoint{-8.928123in}{2.348659in}}%
\pgfpathlineto{\pgfqpoint{-8.858221in}{2.395373in}}%
\pgfpathlineto{\pgfqpoint{-8.787586in}{2.482801in}}%
\pgfpathlineto{\pgfqpoint{-8.718609in}{2.421459in}}%
\pgfpathlineto{\pgfqpoint{-8.649341in}{2.397494in}}%
\pgfpathlineto{\pgfqpoint{-8.578369in}{2.389573in}}%
\pgfpathlineto{\pgfqpoint{-8.508839in}{2.293230in}}%
\pgfpathlineto{\pgfqpoint{-8.438970in}{2.326259in}}%
\pgfpathlineto{\pgfqpoint{-8.367967in}{2.386374in}}%
\pgfpathlineto{\pgfqpoint{-8.298753in}{2.376160in}}%
\pgfpathlineto{\pgfqpoint{-8.228241in}{2.269897in}}%
\pgfpathlineto{\pgfqpoint{-8.156111in}{2.315437in}}%
\pgfpathlineto{\pgfqpoint{-8.084913in}{2.253040in}}%
\pgfpathlineto{\pgfqpoint{-8.014387in}{2.319420in}}%
\pgfpathlineto{\pgfqpoint{-7.942200in}{2.363105in}}%
\pgfpathlineto{\pgfqpoint{-7.870788in}{2.241376in}}%
\pgfpathlineto{\pgfqpoint{-7.799875in}{2.328198in}}%
\pgfpathlineto{\pgfqpoint{-7.727286in}{2.348982in}}%
\pgfpathlineto{\pgfqpoint{-7.657024in}{2.345307in}}%
\pgfpathlineto{\pgfqpoint{-7.585515in}{2.214818in}}%
\pgfpathlineto{\pgfqpoint{-7.512981in}{2.386143in}}%
\pgfpathlineto{\pgfqpoint{-7.440452in}{2.210574in}}%
\pgfpathlineto{\pgfqpoint{-7.368338in}{2.348892in}}%
\pgfpathlineto{\pgfqpoint{-7.294929in}{2.255320in}}%
\pgfpathlineto{\pgfqpoint{-7.223147in}{2.306604in}}%
\pgfpathlineto{\pgfqpoint{-7.151909in}{2.260240in}}%
\pgfpathlineto{\pgfqpoint{-7.075897in}{2.176033in}}%
\pgfpathlineto{\pgfqpoint{-7.003598in}{2.286684in}}%
\pgfpathlineto{\pgfqpoint{-6.933558in}{2.393918in}}%
\pgfpathlineto{\pgfqpoint{-6.860576in}{2.296715in}}%
\pgfpathlineto{\pgfqpoint{-6.787495in}{2.218280in}}%
\pgfpathlineto{\pgfqpoint{-6.715245in}{2.301860in}}%
\pgfpathlineto{\pgfqpoint{-6.640496in}{2.253604in}}%
\pgfpathlineto{\pgfqpoint{-6.568712in}{2.296342in}}%
\pgfpathlineto{\pgfqpoint{-6.496669in}{2.265861in}}%
\pgfpathlineto{\pgfqpoint{-6.421127in}{2.237827in}}%
\pgfpathlineto{\pgfqpoint{-6.349020in}{2.321418in}}%
\pgfpathlineto{\pgfqpoint{-6.278887in}{2.342925in}}%
\pgfpathlineto{\pgfqpoint{-6.205113in}{2.209151in}}%
\pgfpathlineto{\pgfqpoint{-6.135189in}{2.378551in}}%
\pgfpathlineto{\pgfqpoint{-6.064664in}{2.303464in}}%
\pgfpathlineto{\pgfqpoint{-5.992127in}{2.283593in}}%
\pgfpathlineto{\pgfqpoint{-5.921215in}{2.327916in}}%
\pgfpathlineto{\pgfqpoint{-5.849895in}{2.259177in}}%
\pgfpathlineto{\pgfqpoint{-5.776542in}{2.302233in}}%
\pgfpathlineto{\pgfqpoint{-5.706570in}{2.389794in}}%
\pgfpathlineto{\pgfqpoint{-5.635685in}{2.342600in}}%
\pgfpathlineto{\pgfqpoint{-5.563648in}{2.408263in}}%
\pgfpathlineto{\pgfqpoint{-5.493615in}{2.295589in}}%
\pgfpathlineto{\pgfqpoint{-5.421890in}{2.300808in}}%
\pgfpathlineto{\pgfqpoint{-5.348665in}{2.256552in}}%
\pgfpathlineto{\pgfqpoint{-5.277970in}{2.288707in}}%
\pgfpathlineto{\pgfqpoint{-5.207317in}{2.323634in}}%
\pgfpathlineto{\pgfqpoint{-5.135005in}{2.324241in}}%
\pgfpathlineto{\pgfqpoint{-5.064817in}{2.378132in}}%
\pgfpathlineto{\pgfqpoint{-4.995002in}{2.412854in}}%
\pgfpathlineto{\pgfqpoint{-4.922664in}{2.283473in}}%
\pgfpathlineto{\pgfqpoint{-4.852652in}{2.336737in}}%
\pgfpathlineto{\pgfqpoint{-4.782965in}{2.375722in}}%
\pgfpathlineto{\pgfqpoint{-4.711002in}{2.274754in}}%
\pgfpathlineto{\pgfqpoint{-4.639570in}{2.325656in}}%
\pgfpathlineto{\pgfqpoint{-4.568595in}{2.312856in}}%
\pgfpathlineto{\pgfqpoint{-4.495401in}{2.356686in}}%
\pgfpathlineto{\pgfqpoint{-4.423289in}{2.266617in}}%
\pgfpathlineto{\pgfqpoint{-4.350759in}{2.293648in}}%
\pgfpathlineto{\pgfqpoint{-4.274748in}{2.267224in}}%
\pgfpathlineto{\pgfqpoint{-4.201246in}{2.267292in}}%
\pgfpathlineto{\pgfqpoint{-4.128352in}{2.307017in}}%
\pgfpathlineto{\pgfqpoint{-4.052173in}{2.202434in}}%
\pgfpathlineto{\pgfqpoint{-3.977308in}{2.248437in}}%
\pgfpathlineto{\pgfqpoint{-3.903164in}{2.226320in}}%
\pgfpathlineto{\pgfqpoint{-3.828230in}{2.281051in}}%
\pgfpathlineto{\pgfqpoint{-3.756893in}{2.299883in}}%
\pgfpathlineto{\pgfqpoint{-3.684119in}{2.204306in}}%
\pgfpathlineto{\pgfqpoint{-3.608325in}{2.189532in}}%
\pgfpathlineto{\pgfqpoint{-3.534002in}{2.221174in}}%
\pgfpathlineto{\pgfqpoint{-3.461876in}{2.334452in}}%
\pgfpathlineto{\pgfqpoint{-3.388885in}{2.334604in}}%
\pgfpathlineto{\pgfqpoint{-3.317677in}{2.300486in}}%
\pgfpathlineto{\pgfqpoint{-3.245632in}{2.253161in}}%
\pgfpathlineto{\pgfqpoint{-3.171735in}{2.251101in}}%
\pgfpathlineto{\pgfqpoint{-3.100293in}{2.210001in}}%
\pgfpathlineto{\pgfqpoint{-3.029702in}{2.371326in}}%
\pgfpathlineto{\pgfqpoint{-2.956196in}{2.249022in}}%
\pgfpathlineto{\pgfqpoint{-2.884033in}{2.281621in}}%
\pgfpathlineto{\pgfqpoint{-2.813631in}{2.325439in}}%
\pgfpathlineto{\pgfqpoint{-2.740325in}{2.244114in}}%
\pgfpathlineto{\pgfqpoint{-2.667020in}{2.243646in}}%
\pgfpathlineto{\pgfqpoint{-2.595405in}{2.301102in}}%
\pgfpathlineto{\pgfqpoint{-2.522349in}{2.351888in}}%
\pgfpathlineto{\pgfqpoint{-2.451489in}{2.328953in}}%
\pgfpathlineto{\pgfqpoint{-2.379035in}{2.264222in}}%
\pgfpathlineto{\pgfqpoint{-2.303263in}{2.266146in}}%
\pgfpathlineto{\pgfqpoint{-2.231477in}{2.212543in}}%
\pgfpathlineto{\pgfqpoint{-2.158577in}{2.293684in}}%
\pgfpathlineto{\pgfqpoint{-2.084338in}{2.225859in}}%
\pgfpathlineto{\pgfqpoint{-2.011817in}{2.308092in}}%
\pgfpathlineto{\pgfqpoint{-1.940536in}{2.298618in}}%
\pgfpathlineto{\pgfqpoint{-1.867809in}{2.314700in}}%
\pgfpathlineto{\pgfqpoint{-1.797123in}{2.339174in}}%
\pgfpathlineto{\pgfqpoint{-1.725105in}{2.325966in}}%
\pgfpathlineto{\pgfqpoint{-1.650581in}{2.260473in}}%
\pgfpathlineto{\pgfqpoint{-1.577590in}{2.294793in}}%
\pgfpathlineto{\pgfqpoint{-1.503984in}{2.233145in}}%
\pgfpathlineto{\pgfqpoint{-1.429441in}{2.302823in}}%
\pgfpathlineto{\pgfqpoint{-1.356794in}{2.267018in}}%
\pgfpathlineto{\pgfqpoint{-1.283200in}{2.240572in}}%
\pgfpathlineto{\pgfqpoint{-1.207908in}{2.233525in}}%
\pgfpathlineto{\pgfqpoint{-1.135138in}{2.297310in}}%
\pgfpathlineto{\pgfqpoint{-1.063245in}{2.314096in}}%
\pgfpathlineto{\pgfqpoint{-0.989522in}{2.342473in}}%
\pgfpathlineto{\pgfqpoint{-0.918138in}{2.290693in}}%
\pgfpathlineto{\pgfqpoint{-0.844408in}{2.262804in}}%
\pgfpathlineto{\pgfqpoint{-0.769987in}{2.307196in}}%
\pgfpathlineto{\pgfqpoint{-0.698122in}{2.345668in}}%
\pgfpathlineto{\pgfqpoint{-0.623932in}{2.226503in}}%
\pgfpathlineto{\pgfqpoint{-0.549722in}{2.291781in}}%
\pgfpathlineto{\pgfqpoint{-0.477784in}{2.277852in}}%
\pgfpathlineto{\pgfqpoint{-0.404497in}{2.263642in}}%
\pgfpathlineto{\pgfqpoint{-0.329323in}{2.184436in}}%
\pgfpathlineto{\pgfqpoint{-0.257459in}{2.280819in}}%
\pgfpathlineto{\pgfqpoint{-0.185623in}{2.314323in}}%
\pgfpathlineto{\pgfqpoint{-0.112821in}{2.321634in}}%
\pgfpathlineto{\pgfqpoint{-0.042097in}{2.302230in}}%
\pgfpathlineto{\pgfqpoint{0.030045in}{2.268170in}}%
\pgfpathlineto{\pgfqpoint{0.103944in}{2.336838in}}%
\pgfpathlineto{\pgfqpoint{0.176284in}{2.251796in}}%
\pgfpathlineto{\pgfqpoint{0.248641in}{2.280462in}}%
\pgfpathlineto{\pgfqpoint{0.321303in}{2.331917in}}%
\pgfpathlineto{\pgfqpoint{0.391498in}{2.369077in}}%
\pgfpathlineto{\pgfqpoint{0.462885in}{2.327995in}}%
\pgfpathlineto{\pgfqpoint{0.537447in}{2.225068in}}%
\pgfpathlineto{\pgfqpoint{0.609267in}{2.355976in}}%
\pgfpathlineto{\pgfqpoint{0.679536in}{2.288985in}}%
\pgfpathlineto{\pgfqpoint{0.753203in}{2.296487in}}%
\pgfpathlineto{\pgfqpoint{0.824780in}{2.315904in}}%
\pgfpathlineto{\pgfqpoint{0.895203in}{2.373974in}}%
\pgfpathlineto{\pgfqpoint{0.968165in}{2.391312in}}%
\pgfpathlineto{\pgfqpoint{1.038437in}{2.343567in}}%
\pgfpathlineto{\pgfqpoint{1.111472in}{2.224849in}}%
\pgfpathlineto{\pgfqpoint{1.188027in}{2.198868in}}%
\pgfpathlineto{\pgfqpoint{1.261309in}{2.302499in}}%
\pgfpathlineto{\pgfqpoint{1.335256in}{2.259385in}}%
\pgfpathlineto{\pgfqpoint{1.410931in}{2.321530in}}%
\pgfpathlineto{\pgfqpoint{1.485557in}{2.220336in}}%
\pgfpathlineto{\pgfqpoint{1.558242in}{2.347512in}}%
\pgfpathlineto{\pgfqpoint{1.633550in}{2.261273in}}%
\pgfpathlineto{\pgfqpoint{1.707271in}{2.258974in}}%
\pgfpathlineto{\pgfqpoint{1.781074in}{2.263560in}}%
\pgfpathlineto{\pgfqpoint{1.857049in}{2.213105in}}%
\pgfpathlineto{\pgfqpoint{1.931091in}{2.250884in}}%
\pgfpathlineto{\pgfqpoint{2.004776in}{2.262254in}}%
\pgfpathlineto{\pgfqpoint{2.079647in}{2.326432in}}%
\pgfpathlineto{\pgfqpoint{2.151611in}{2.340502in}}%
\pgfpathlineto{\pgfqpoint{2.224651in}{2.315839in}}%
\pgfpathlineto{\pgfqpoint{2.300156in}{2.319907in}}%
\pgfpathlineto{\pgfqpoint{2.371299in}{2.304522in}}%
\pgfpathlineto{\pgfqpoint{2.442412in}{2.292558in}}%
\pgfpathlineto{\pgfqpoint{2.515763in}{2.399372in}}%
\pgfpathlineto{\pgfqpoint{2.586109in}{2.364491in}}%
\pgfpathlineto{\pgfqpoint{2.658524in}{2.335533in}}%
\pgfpathlineto{\pgfqpoint{2.732102in}{2.239019in}}%
\pgfpathlineto{\pgfqpoint{2.802412in}{2.265161in}}%
\pgfpathlineto{\pgfqpoint{2.873367in}{2.343943in}}%
\pgfpathlineto{\pgfqpoint{2.946925in}{2.315701in}}%
\pgfpathlineto{\pgfqpoint{3.019212in}{2.215727in}}%
\pgfpathlineto{\pgfqpoint{3.091740in}{2.263431in}}%
\pgfpathlineto{\pgfqpoint{3.166494in}{2.268673in}}%
\pgfpathlineto{\pgfqpoint{3.237461in}{2.375130in}}%
\pgfpathlineto{\pgfqpoint{3.309976in}{2.316665in}}%
\pgfpathlineto{\pgfqpoint{3.384706in}{2.228499in}}%
\pgfpathlineto{\pgfqpoint{3.455601in}{2.329202in}}%
\pgfpathlineto{\pgfqpoint{3.527918in}{2.265781in}}%
\pgfpathlineto{\pgfqpoint{3.603230in}{2.252098in}}%
\pgfpathlineto{\pgfqpoint{3.674965in}{2.323501in}}%
\pgfpathlineto{\pgfqpoint{3.745938in}{2.365524in}}%
\pgfpathlineto{\pgfqpoint{3.820753in}{2.229596in}}%
\pgfpathlineto{\pgfqpoint{3.899537in}{2.152523in}}%
\pgfpathlineto{\pgfqpoint{4.022313in}{1.695897in}}%
\pgfpathlineto{\pgfqpoint{4.111286in}{1.559351in}}%
\pgfpathlineto{\pgfqpoint{4.189282in}{0.926398in}}%
\pgfpathlineto{\pgfqpoint{4.253332in}{1.724813in}}%
\pgfpathlineto{\pgfqpoint{4.318577in}{5.187514in}}%
\pgfpathlineto{\pgfqpoint{4.390596in}{5.319244in}}%
\pgfpathlineto{\pgfqpoint{4.460559in}{5.506716in}}%
\pgfpathlineto{\pgfqpoint{4.532858in}{5.399336in}}%
\pgfpathlineto{\pgfqpoint{4.602459in}{5.488048in}}%
\pgfpathlineto{\pgfqpoint{4.671388in}{5.539128in}}%
\pgfpathlineto{\pgfqpoint{4.741243in}{5.638191in}}%
\pgfpathlineto{\pgfqpoint{4.808397in}{5.619869in}}%
\pgfpathlineto{\pgfqpoint{4.875969in}{5.661488in}}%
\pgfpathlineto{\pgfqpoint{4.944850in}{5.625548in}}%
\pgfpathlineto{\pgfqpoint{5.010860in}{5.748599in}}%
\pgfpathlineto{\pgfqpoint{5.076713in}{5.732424in}}%
\pgfpathlineto{\pgfqpoint{5.144715in}{5.790749in}}%
\pgfpathlineto{\pgfqpoint{5.209656in}{5.793255in}}%
\pgfpathlineto{\pgfqpoint{5.275100in}{5.772285in}}%
\pgfpathlineto{\pgfqpoint{5.341477in}{5.875626in}}%
\pgfpathlineto{\pgfqpoint{5.405234in}{5.879596in}}%
\pgfpathlineto{\pgfqpoint{5.469981in}{5.814953in}}%
\pgfpathlineto{\pgfqpoint{5.535779in}{5.930845in}}%
\pgfpathlineto{\pgfqpoint{5.599590in}{5.872359in}}%
\pgfpathlineto{\pgfqpoint{5.599590in}{5.872359in}}%
\pgfpathlineto{\pgfqpoint{5.599590in}{5.872359in}}%
\pgfpathlineto{\pgfqpoint{5.535779in}{5.930845in}}%
\pgfpathlineto{\pgfqpoint{5.469981in}{5.814953in}}%
\pgfpathlineto{\pgfqpoint{5.405234in}{5.879596in}}%
\pgfpathlineto{\pgfqpoint{5.341477in}{5.875626in}}%
\pgfpathlineto{\pgfqpoint{5.275100in}{5.772285in}}%
\pgfpathlineto{\pgfqpoint{5.209656in}{5.793255in}}%
\pgfpathlineto{\pgfqpoint{5.144715in}{5.790749in}}%
\pgfpathlineto{\pgfqpoint{5.076713in}{5.732424in}}%
\pgfpathlineto{\pgfqpoint{5.010860in}{5.748599in}}%
\pgfpathlineto{\pgfqpoint{4.944850in}{5.625548in}}%
\pgfpathlineto{\pgfqpoint{4.875969in}{5.661488in}}%
\pgfpathlineto{\pgfqpoint{4.808397in}{5.619869in}}%
\pgfpathlineto{\pgfqpoint{4.741243in}{5.638191in}}%
\pgfpathlineto{\pgfqpoint{4.671388in}{5.539128in}}%
\pgfpathlineto{\pgfqpoint{4.602459in}{5.488048in}}%
\pgfpathlineto{\pgfqpoint{4.532858in}{5.399336in}}%
\pgfpathlineto{\pgfqpoint{4.460559in}{5.506716in}}%
\pgfpathlineto{\pgfqpoint{4.390596in}{5.319244in}}%
\pgfpathlineto{\pgfqpoint{4.318577in}{5.187514in}}%
\pgfpathlineto{\pgfqpoint{4.253332in}{1.724813in}}%
\pgfpathlineto{\pgfqpoint{4.189282in}{0.926398in}}%
\pgfpathlineto{\pgfqpoint{4.111286in}{1.559351in}}%
\pgfpathlineto{\pgfqpoint{4.022313in}{1.695897in}}%
\pgfpathlineto{\pgfqpoint{3.899537in}{2.152523in}}%
\pgfpathlineto{\pgfqpoint{3.820753in}{2.229596in}}%
\pgfpathlineto{\pgfqpoint{3.745938in}{2.365524in}}%
\pgfpathlineto{\pgfqpoint{3.674965in}{2.323501in}}%
\pgfpathlineto{\pgfqpoint{3.603230in}{2.252098in}}%
\pgfpathlineto{\pgfqpoint{3.527918in}{2.265781in}}%
\pgfpathlineto{\pgfqpoint{3.455601in}{2.329202in}}%
\pgfpathlineto{\pgfqpoint{3.384706in}{2.228499in}}%
\pgfpathlineto{\pgfqpoint{3.309976in}{2.316665in}}%
\pgfpathlineto{\pgfqpoint{3.237461in}{2.375130in}}%
\pgfpathlineto{\pgfqpoint{3.166494in}{2.268673in}}%
\pgfpathlineto{\pgfqpoint{3.091740in}{2.263431in}}%
\pgfpathlineto{\pgfqpoint{3.019212in}{2.215727in}}%
\pgfpathlineto{\pgfqpoint{2.946925in}{2.315701in}}%
\pgfpathlineto{\pgfqpoint{2.873367in}{2.343943in}}%
\pgfpathlineto{\pgfqpoint{2.802412in}{2.265161in}}%
\pgfpathlineto{\pgfqpoint{2.732102in}{2.239019in}}%
\pgfpathlineto{\pgfqpoint{2.658524in}{2.335533in}}%
\pgfpathlineto{\pgfqpoint{2.586109in}{2.364491in}}%
\pgfpathlineto{\pgfqpoint{2.515763in}{2.399372in}}%
\pgfpathlineto{\pgfqpoint{2.442412in}{2.292558in}}%
\pgfpathlineto{\pgfqpoint{2.371299in}{2.304522in}}%
\pgfpathlineto{\pgfqpoint{2.300156in}{2.319907in}}%
\pgfpathlineto{\pgfqpoint{2.224651in}{2.315839in}}%
\pgfpathlineto{\pgfqpoint{2.151611in}{2.340502in}}%
\pgfpathlineto{\pgfqpoint{2.079647in}{2.326432in}}%
\pgfpathlineto{\pgfqpoint{2.004776in}{2.262254in}}%
\pgfpathlineto{\pgfqpoint{1.931091in}{2.250884in}}%
\pgfpathlineto{\pgfqpoint{1.857049in}{2.213105in}}%
\pgfpathlineto{\pgfqpoint{1.781074in}{2.263560in}}%
\pgfpathlineto{\pgfqpoint{1.707271in}{2.258974in}}%
\pgfpathlineto{\pgfqpoint{1.633550in}{2.261273in}}%
\pgfpathlineto{\pgfqpoint{1.558242in}{2.347512in}}%
\pgfpathlineto{\pgfqpoint{1.485557in}{2.220336in}}%
\pgfpathlineto{\pgfqpoint{1.410931in}{2.321530in}}%
\pgfpathlineto{\pgfqpoint{1.335256in}{2.259385in}}%
\pgfpathlineto{\pgfqpoint{1.261309in}{2.302499in}}%
\pgfpathlineto{\pgfqpoint{1.188027in}{2.198868in}}%
\pgfpathlineto{\pgfqpoint{1.111472in}{2.224849in}}%
\pgfpathlineto{\pgfqpoint{1.038437in}{2.343567in}}%
\pgfpathlineto{\pgfqpoint{0.968165in}{2.391312in}}%
\pgfpathlineto{\pgfqpoint{0.895203in}{2.373974in}}%
\pgfpathlineto{\pgfqpoint{0.824780in}{2.315904in}}%
\pgfpathlineto{\pgfqpoint{0.753203in}{2.296487in}}%
\pgfpathlineto{\pgfqpoint{0.679536in}{2.288985in}}%
\pgfpathlineto{\pgfqpoint{0.609267in}{2.355976in}}%
\pgfpathlineto{\pgfqpoint{0.537447in}{2.225068in}}%
\pgfpathlineto{\pgfqpoint{0.462885in}{2.327995in}}%
\pgfpathlineto{\pgfqpoint{0.391498in}{2.369077in}}%
\pgfpathlineto{\pgfqpoint{0.321303in}{2.331917in}}%
\pgfpathlineto{\pgfqpoint{0.248641in}{2.280462in}}%
\pgfpathlineto{\pgfqpoint{0.176284in}{2.251796in}}%
\pgfpathlineto{\pgfqpoint{0.103944in}{2.336838in}}%
\pgfpathlineto{\pgfqpoint{0.030045in}{2.268170in}}%
\pgfpathlineto{\pgfqpoint{-0.042097in}{2.302230in}}%
\pgfpathlineto{\pgfqpoint{-0.112821in}{2.321634in}}%
\pgfpathlineto{\pgfqpoint{-0.185623in}{2.314323in}}%
\pgfpathlineto{\pgfqpoint{-0.257459in}{2.280819in}}%
\pgfpathlineto{\pgfqpoint{-0.329323in}{2.184436in}}%
\pgfpathlineto{\pgfqpoint{-0.404497in}{2.263642in}}%
\pgfpathlineto{\pgfqpoint{-0.477784in}{2.277852in}}%
\pgfpathlineto{\pgfqpoint{-0.549722in}{2.291781in}}%
\pgfpathlineto{\pgfqpoint{-0.623932in}{2.226503in}}%
\pgfpathlineto{\pgfqpoint{-0.698122in}{2.345668in}}%
\pgfpathlineto{\pgfqpoint{-0.769987in}{2.307196in}}%
\pgfpathlineto{\pgfqpoint{-0.844408in}{2.262804in}}%
\pgfpathlineto{\pgfqpoint{-0.918138in}{2.290693in}}%
\pgfpathlineto{\pgfqpoint{-0.989522in}{2.342473in}}%
\pgfpathlineto{\pgfqpoint{-1.063245in}{2.314096in}}%
\pgfpathlineto{\pgfqpoint{-1.135138in}{2.297310in}}%
\pgfpathlineto{\pgfqpoint{-1.207908in}{2.233525in}}%
\pgfpathlineto{\pgfqpoint{-1.283200in}{2.240572in}}%
\pgfpathlineto{\pgfqpoint{-1.356794in}{2.267018in}}%
\pgfpathlineto{\pgfqpoint{-1.429441in}{2.302823in}}%
\pgfpathlineto{\pgfqpoint{-1.503984in}{2.233145in}}%
\pgfpathlineto{\pgfqpoint{-1.577590in}{2.294793in}}%
\pgfpathlineto{\pgfqpoint{-1.650581in}{2.260473in}}%
\pgfpathlineto{\pgfqpoint{-1.725105in}{2.325966in}}%
\pgfpathlineto{\pgfqpoint{-1.797123in}{2.339174in}}%
\pgfpathlineto{\pgfqpoint{-1.867809in}{2.314700in}}%
\pgfpathlineto{\pgfqpoint{-1.940536in}{2.298618in}}%
\pgfpathlineto{\pgfqpoint{-2.011817in}{2.308092in}}%
\pgfpathlineto{\pgfqpoint{-2.084338in}{2.225859in}}%
\pgfpathlineto{\pgfqpoint{-2.158577in}{2.293684in}}%
\pgfpathlineto{\pgfqpoint{-2.231477in}{2.212543in}}%
\pgfpathlineto{\pgfqpoint{-2.303263in}{2.266146in}}%
\pgfpathlineto{\pgfqpoint{-2.379035in}{2.264222in}}%
\pgfpathlineto{\pgfqpoint{-2.451489in}{2.328953in}}%
\pgfpathlineto{\pgfqpoint{-2.522349in}{2.351888in}}%
\pgfpathlineto{\pgfqpoint{-2.595405in}{2.301102in}}%
\pgfpathlineto{\pgfqpoint{-2.667020in}{2.243646in}}%
\pgfpathlineto{\pgfqpoint{-2.740325in}{2.244114in}}%
\pgfpathlineto{\pgfqpoint{-2.813631in}{2.325439in}}%
\pgfpathlineto{\pgfqpoint{-2.884033in}{2.281621in}}%
\pgfpathlineto{\pgfqpoint{-2.956196in}{2.249022in}}%
\pgfpathlineto{\pgfqpoint{-3.029702in}{2.371326in}}%
\pgfpathlineto{\pgfqpoint{-3.100293in}{2.210001in}}%
\pgfpathlineto{\pgfqpoint{-3.171735in}{2.251101in}}%
\pgfpathlineto{\pgfqpoint{-3.245632in}{2.253161in}}%
\pgfpathlineto{\pgfqpoint{-3.317677in}{2.300486in}}%
\pgfpathlineto{\pgfqpoint{-3.388885in}{2.334604in}}%
\pgfpathlineto{\pgfqpoint{-3.461876in}{2.334452in}}%
\pgfpathlineto{\pgfqpoint{-3.534002in}{2.221174in}}%
\pgfpathlineto{\pgfqpoint{-3.608325in}{2.189532in}}%
\pgfpathlineto{\pgfqpoint{-3.684119in}{2.204306in}}%
\pgfpathlineto{\pgfqpoint{-3.756893in}{2.299883in}}%
\pgfpathlineto{\pgfqpoint{-3.828230in}{2.281051in}}%
\pgfpathlineto{\pgfqpoint{-3.903164in}{2.226320in}}%
\pgfpathlineto{\pgfqpoint{-3.977308in}{2.248437in}}%
\pgfpathlineto{\pgfqpoint{-4.052173in}{2.202434in}}%
\pgfpathlineto{\pgfqpoint{-4.128352in}{2.307017in}}%
\pgfpathlineto{\pgfqpoint{-4.201246in}{2.267292in}}%
\pgfpathlineto{\pgfqpoint{-4.274748in}{2.267224in}}%
\pgfpathlineto{\pgfqpoint{-4.350759in}{2.293648in}}%
\pgfpathlineto{\pgfqpoint{-4.423289in}{2.266617in}}%
\pgfpathlineto{\pgfqpoint{-4.495401in}{2.356686in}}%
\pgfpathlineto{\pgfqpoint{-4.568595in}{2.312856in}}%
\pgfpathlineto{\pgfqpoint{-4.639570in}{2.325656in}}%
\pgfpathlineto{\pgfqpoint{-4.711002in}{2.274754in}}%
\pgfpathlineto{\pgfqpoint{-4.782965in}{2.375722in}}%
\pgfpathlineto{\pgfqpoint{-4.852652in}{2.336737in}}%
\pgfpathlineto{\pgfqpoint{-4.922664in}{2.283473in}}%
\pgfpathlineto{\pgfqpoint{-4.995002in}{2.412854in}}%
\pgfpathlineto{\pgfqpoint{-5.064817in}{2.378132in}}%
\pgfpathlineto{\pgfqpoint{-5.135005in}{2.324241in}}%
\pgfpathlineto{\pgfqpoint{-5.207317in}{2.323634in}}%
\pgfpathlineto{\pgfqpoint{-5.277970in}{2.288707in}}%
\pgfpathlineto{\pgfqpoint{-5.348665in}{2.256552in}}%
\pgfpathlineto{\pgfqpoint{-5.421890in}{2.300808in}}%
\pgfpathlineto{\pgfqpoint{-5.493615in}{2.295589in}}%
\pgfpathlineto{\pgfqpoint{-5.563648in}{2.408263in}}%
\pgfpathlineto{\pgfqpoint{-5.635685in}{2.342600in}}%
\pgfpathlineto{\pgfqpoint{-5.706570in}{2.389794in}}%
\pgfpathlineto{\pgfqpoint{-5.776542in}{2.302233in}}%
\pgfpathlineto{\pgfqpoint{-5.849895in}{2.259177in}}%
\pgfpathlineto{\pgfqpoint{-5.921215in}{2.327916in}}%
\pgfpathlineto{\pgfqpoint{-5.992127in}{2.283593in}}%
\pgfpathlineto{\pgfqpoint{-6.064664in}{2.303464in}}%
\pgfpathlineto{\pgfqpoint{-6.135189in}{2.378551in}}%
\pgfpathlineto{\pgfqpoint{-6.205113in}{2.209151in}}%
\pgfpathlineto{\pgfqpoint{-6.278887in}{2.342925in}}%
\pgfpathlineto{\pgfqpoint{-6.349020in}{2.321418in}}%
\pgfpathlineto{\pgfqpoint{-6.421127in}{2.237827in}}%
\pgfpathlineto{\pgfqpoint{-6.496669in}{2.265861in}}%
\pgfpathlineto{\pgfqpoint{-6.568712in}{2.296342in}}%
\pgfpathlineto{\pgfqpoint{-6.640496in}{2.253604in}}%
\pgfpathlineto{\pgfqpoint{-6.715245in}{2.301860in}}%
\pgfpathlineto{\pgfqpoint{-6.787495in}{2.218280in}}%
\pgfpathlineto{\pgfqpoint{-6.860576in}{2.296715in}}%
\pgfpathlineto{\pgfqpoint{-6.933558in}{2.393918in}}%
\pgfpathlineto{\pgfqpoint{-7.003598in}{2.286684in}}%
\pgfpathlineto{\pgfqpoint{-7.075897in}{2.176033in}}%
\pgfpathlineto{\pgfqpoint{-7.151909in}{2.260240in}}%
\pgfpathlineto{\pgfqpoint{-7.223147in}{2.306604in}}%
\pgfpathlineto{\pgfqpoint{-7.294929in}{2.255320in}}%
\pgfpathlineto{\pgfqpoint{-7.368338in}{2.348892in}}%
\pgfpathlineto{\pgfqpoint{-7.440452in}{2.210574in}}%
\pgfpathlineto{\pgfqpoint{-7.512981in}{2.386143in}}%
\pgfpathlineto{\pgfqpoint{-7.585515in}{2.214818in}}%
\pgfpathlineto{\pgfqpoint{-7.657024in}{2.345307in}}%
\pgfpathlineto{\pgfqpoint{-7.727286in}{2.348982in}}%
\pgfpathlineto{\pgfqpoint{-7.799875in}{2.328198in}}%
\pgfpathlineto{\pgfqpoint{-7.870788in}{2.241376in}}%
\pgfpathlineto{\pgfqpoint{-7.942200in}{2.363105in}}%
\pgfpathlineto{\pgfqpoint{-8.014387in}{2.319420in}}%
\pgfpathlineto{\pgfqpoint{-8.084913in}{2.253040in}}%
\pgfpathlineto{\pgfqpoint{-8.156111in}{2.315437in}}%
\pgfpathlineto{\pgfqpoint{-8.228241in}{2.269897in}}%
\pgfpathlineto{\pgfqpoint{-8.298753in}{2.376160in}}%
\pgfpathlineto{\pgfqpoint{-8.367967in}{2.386374in}}%
\pgfpathlineto{\pgfqpoint{-8.438970in}{2.326259in}}%
\pgfpathlineto{\pgfqpoint{-8.508839in}{2.293230in}}%
\pgfpathlineto{\pgfqpoint{-8.578369in}{2.389573in}}%
\pgfpathlineto{\pgfqpoint{-8.649341in}{2.397494in}}%
\pgfpathlineto{\pgfqpoint{-8.718609in}{2.421459in}}%
\pgfpathlineto{\pgfqpoint{-8.787586in}{2.482801in}}%
\pgfpathlineto{\pgfqpoint{-8.858221in}{2.395373in}}%
\pgfpathlineto{\pgfqpoint{-8.928123in}{2.348659in}}%
\pgfpathlineto{\pgfqpoint{-8.998701in}{2.323391in}}%
\pgfpathlineto{\pgfqpoint{-9.069721in}{2.398083in}}%
\pgfpathlineto{\pgfqpoint{-9.138788in}{2.303340in}}%
\pgfpathlineto{\pgfqpoint{-9.208906in}{2.348390in}}%
\pgfpathlineto{\pgfqpoint{-9.283412in}{2.223836in}}%
\pgfpathlineto{\pgfqpoint{-9.356529in}{2.170512in}}%
\pgfpathlineto{\pgfqpoint{-9.429774in}{2.278881in}}%
\pgfpathlineto{\pgfqpoint{-9.504196in}{2.330770in}}%
\pgfpathlineto{\pgfqpoint{-9.575400in}{2.197014in}}%
\pgfpathlineto{\pgfqpoint{-9.648369in}{2.291199in}}%
\pgfpathlineto{\pgfqpoint{-9.723682in}{2.291422in}}%
\pgfpathlineto{\pgfqpoint{-9.796010in}{2.259136in}}%
\pgfpathlineto{\pgfqpoint{-9.868243in}{2.339697in}}%
\pgfpathlineto{\pgfqpoint{-9.941359in}{2.303436in}}%
\pgfpathlineto{\pgfqpoint{-10.012635in}{2.268173in}}%
\pgfpathlineto{\pgfqpoint{-10.084496in}{2.288976in}}%
\pgfpathlineto{\pgfqpoint{-10.157398in}{2.299263in}}%
\pgfpathlineto{\pgfqpoint{-10.227739in}{2.355684in}}%
\pgfpathlineto{\pgfqpoint{-10.297591in}{2.304203in}}%
\pgfpathlineto{\pgfqpoint{-10.371004in}{2.235383in}}%
\pgfpathlineto{\pgfqpoint{-10.441581in}{2.319899in}}%
\pgfpathlineto{\pgfqpoint{-10.511372in}{2.379865in}}%
\pgfpathlineto{\pgfqpoint{-10.582864in}{2.236644in}}%
\pgfpathlineto{\pgfqpoint{-10.652649in}{2.361138in}}%
\pgfpathlineto{\pgfqpoint{-10.721651in}{2.379466in}}%
\pgfpathlineto{\pgfqpoint{-10.793422in}{2.293630in}}%
\pgfpathlineto{\pgfqpoint{-10.863257in}{2.326817in}}%
\pgfpathlineto{\pgfqpoint{-10.933808in}{2.387675in}}%
\pgfpathlineto{\pgfqpoint{-11.005065in}{2.261538in}}%
\pgfpathlineto{\pgfqpoint{-11.075276in}{2.308828in}}%
\pgfpathlineto{\pgfqpoint{-11.144960in}{2.336694in}}%
\pgfpathlineto{\pgfqpoint{-11.217631in}{2.289116in}}%
\pgfpathlineto{\pgfqpoint{-11.288273in}{2.361737in}}%
\pgfpathlineto{\pgfqpoint{-11.358676in}{2.294331in}}%
\pgfpathlineto{\pgfqpoint{-11.430351in}{2.382111in}}%
\pgfpathlineto{\pgfqpoint{-11.500166in}{2.291057in}}%
\pgfpathlineto{\pgfqpoint{-11.571037in}{2.286178in}}%
\pgfpathlineto{\pgfqpoint{-11.645187in}{2.285374in}}%
\pgfpathlineto{\pgfqpoint{-11.716309in}{2.389725in}}%
\pgfpathlineto{\pgfqpoint{-11.786176in}{2.363164in}}%
\pgfpathlineto{\pgfqpoint{-11.858471in}{2.277444in}}%
\pgfpathlineto{\pgfqpoint{-11.928381in}{2.419361in}}%
\pgfpathlineto{\pgfqpoint{-11.998596in}{2.351876in}}%
\pgfpathlineto{\pgfqpoint{-12.069446in}{2.360552in}}%
\pgfpathlineto{\pgfqpoint{-12.139885in}{2.283026in}}%
\pgfpathlineto{\pgfqpoint{-12.211536in}{2.296885in}}%
\pgfpathlineto{\pgfqpoint{-12.285275in}{2.284268in}}%
\pgfpathlineto{\pgfqpoint{-12.357103in}{2.289248in}}%
\pgfpathlineto{\pgfqpoint{-12.428997in}{2.232984in}}%
\pgfpathlineto{\pgfqpoint{-12.503401in}{2.271599in}}%
\pgfpathlineto{\pgfqpoint{-12.574652in}{2.266246in}}%
\pgfpathlineto{\pgfqpoint{-12.645279in}{2.330027in}}%
\pgfpathlineto{\pgfqpoint{-12.718096in}{2.284510in}}%
\pgfpathlineto{\pgfqpoint{-12.788679in}{2.293835in}}%
\pgfpathlineto{\pgfqpoint{-12.859398in}{2.416355in}}%
\pgfpathlineto{\pgfqpoint{-12.932388in}{2.333533in}}%
\pgfpathlineto{\pgfqpoint{-13.002727in}{2.394626in}}%
\pgfpathlineto{\pgfqpoint{-13.073093in}{2.296630in}}%
\pgfpathlineto{\pgfqpoint{-13.145337in}{2.321433in}}%
\pgfpathlineto{\pgfqpoint{-13.216013in}{2.426745in}}%
\pgfpathlineto{\pgfqpoint{-13.285456in}{2.377914in}}%
\pgfpathlineto{\pgfqpoint{-13.356968in}{2.353926in}}%
\pgfpathlineto{\pgfqpoint{-13.426276in}{2.341081in}}%
\pgfpathlineto{\pgfqpoint{-13.496151in}{2.293209in}}%
\pgfpathlineto{\pgfqpoint{-13.568713in}{2.245730in}}%
\pgfpathlineto{\pgfqpoint{-13.639609in}{2.376504in}}%
\pgfpathlineto{\pgfqpoint{-13.708089in}{2.355139in}}%
\pgfpathlineto{\pgfqpoint{-13.779339in}{2.315892in}}%
\pgfpathlineto{\pgfqpoint{-13.848756in}{2.375946in}}%
\pgfpathlineto{\pgfqpoint{-13.917455in}{2.313263in}}%
\pgfpathlineto{\pgfqpoint{-13.988983in}{2.352403in}}%
\pgfpathlineto{\pgfqpoint{-14.057377in}{2.295254in}}%
\pgfpathlineto{\pgfqpoint{-14.125666in}{2.414408in}}%
\pgfpathlineto{\pgfqpoint{-14.195714in}{2.344504in}}%
\pgfpathlineto{\pgfqpoint{-14.263436in}{2.420723in}}%
\pgfpathlineto{\pgfqpoint{-14.331277in}{2.310083in}}%
\pgfpathlineto{\pgfqpoint{-14.401957in}{2.317372in}}%
\pgfpathlineto{\pgfqpoint{-14.471123in}{2.450638in}}%
\pgfpathlineto{\pgfqpoint{-14.538988in}{2.402088in}}%
\pgfpathlineto{\pgfqpoint{-14.609938in}{2.319482in}}%
\pgfpathlineto{\pgfqpoint{-14.679263in}{2.315337in}}%
\pgfpathlineto{\pgfqpoint{-14.748835in}{2.334003in}}%
\pgfpathlineto{\pgfqpoint{-14.820408in}{2.347150in}}%
\pgfpathlineto{\pgfqpoint{-14.889438in}{2.362100in}}%
\pgfpathlineto{\pgfqpoint{-14.959865in}{2.408326in}}%
\pgfpathlineto{\pgfqpoint{-15.032843in}{2.237324in}}%
\pgfpathlineto{\pgfqpoint{-15.106256in}{2.213369in}}%
\pgfpathlineto{\pgfqpoint{-15.179182in}{2.296087in}}%
\pgfpathlineto{\pgfqpoint{-15.253001in}{2.346079in}}%
\pgfpathlineto{\pgfqpoint{-15.323000in}{2.416211in}}%
\pgfpathlineto{\pgfqpoint{-15.392048in}{2.441511in}}%
\pgfpathlineto{\pgfqpoint{-15.463558in}{2.378177in}}%
\pgfpathlineto{\pgfqpoint{-15.532639in}{2.336319in}}%
\pgfpathlineto{\pgfqpoint{-15.602559in}{2.389314in}}%
\pgfpathlineto{\pgfqpoint{-15.673559in}{2.309932in}}%
\pgfpathlineto{\pgfqpoint{-15.744769in}{2.316651in}}%
\pgfpathlineto{\pgfqpoint{-15.814335in}{2.339944in}}%
\pgfpathlineto{\pgfqpoint{-15.886757in}{2.298817in}}%
\pgfpathlineto{\pgfqpoint{-15.956358in}{2.387626in}}%
\pgfpathlineto{\pgfqpoint{-16.024276in}{2.337079in}}%
\pgfpathlineto{\pgfqpoint{-16.094895in}{2.315581in}}%
\pgfpathlineto{\pgfqpoint{-16.163654in}{2.367244in}}%
\pgfpathlineto{\pgfqpoint{-16.232068in}{2.386687in}}%
\pgfpathlineto{\pgfqpoint{-16.302639in}{2.366103in}}%
\pgfpathlineto{\pgfqpoint{-16.369724in}{2.425014in}}%
\pgfpathlineto{\pgfqpoint{-16.437097in}{2.334152in}}%
\pgfpathlineto{\pgfqpoint{-16.507797in}{2.362664in}}%
\pgfpathlineto{\pgfqpoint{-16.575864in}{2.414390in}}%
\pgfpathlineto{\pgfqpoint{-16.644038in}{2.269316in}}%
\pgfpathlineto{\pgfqpoint{-16.715630in}{2.352508in}}%
\pgfpathlineto{\pgfqpoint{-16.784012in}{2.407033in}}%
\pgfpathlineto{\pgfqpoint{-16.852234in}{2.378815in}}%
\pgfpathlineto{\pgfqpoint{-16.921714in}{2.406563in}}%
\pgfpathlineto{\pgfqpoint{-16.989953in}{2.368556in}}%
\pgfpathlineto{\pgfqpoint{-17.058127in}{2.382148in}}%
\pgfpathlineto{\pgfqpoint{-17.128098in}{2.394395in}}%
\pgfpathlineto{\pgfqpoint{-17.196258in}{2.324319in}}%
\pgfpathlineto{\pgfqpoint{-17.265552in}{2.344426in}}%
\pgfpathlineto{\pgfqpoint{-17.336134in}{2.399801in}}%
\pgfpathlineto{\pgfqpoint{-17.402675in}{2.396819in}}%
\pgfpathlineto{\pgfqpoint{-17.470673in}{2.427453in}}%
\pgfpathlineto{\pgfqpoint{-17.539598in}{2.322965in}}%
\pgfpathlineto{\pgfqpoint{-17.607453in}{2.405673in}}%
\pgfpathlineto{\pgfqpoint{-17.675451in}{2.318313in}}%
\pgfpathlineto{\pgfqpoint{-17.748008in}{2.364455in}}%
\pgfpathlineto{\pgfqpoint{-17.816737in}{2.360084in}}%
\pgfpathlineto{\pgfqpoint{-17.885356in}{2.298388in}}%
\pgfpathlineto{\pgfqpoint{-17.957185in}{2.404581in}}%
\pgfpathlineto{\pgfqpoint{-18.025876in}{2.400693in}}%
\pgfpathlineto{\pgfqpoint{-18.094404in}{2.329708in}}%
\pgfpathlineto{\pgfqpoint{-18.165145in}{2.389023in}}%
\pgfpathlineto{\pgfqpoint{-18.232823in}{2.338118in}}%
\pgfpathlineto{\pgfqpoint{-18.301710in}{2.313702in}}%
\pgfpathlineto{\pgfqpoint{-18.373957in}{2.321556in}}%
\pgfpathlineto{\pgfqpoint{-18.442527in}{2.363697in}}%
\pgfpathlineto{\pgfqpoint{-18.511159in}{2.275523in}}%
\pgfpathlineto{\pgfqpoint{-18.583201in}{2.316575in}}%
\pgfpathlineto{\pgfqpoint{-18.652004in}{2.345595in}}%
\pgfpathlineto{\pgfqpoint{-18.721737in}{2.350302in}}%
\pgfpathlineto{\pgfqpoint{-18.792270in}{2.460769in}}%
\pgfpathlineto{\pgfqpoint{-18.859965in}{2.394316in}}%
\pgfpathlineto{\pgfqpoint{-18.928709in}{2.382185in}}%
\pgfpathlineto{\pgfqpoint{-18.999640in}{2.424792in}}%
\pgfpathlineto{\pgfqpoint{-19.068147in}{2.329892in}}%
\pgfpathlineto{\pgfqpoint{-19.135788in}{2.381820in}}%
\pgfpathlineto{\pgfqpoint{-19.205228in}{2.445924in}}%
\pgfpathlineto{\pgfqpoint{-19.271476in}{2.454963in}}%
\pgfpathlineto{\pgfqpoint{-19.338364in}{2.340878in}}%
\pgfpathlineto{\pgfqpoint{-19.407422in}{2.287899in}}%
\pgfpathlineto{\pgfqpoint{-19.475599in}{2.412443in}}%
\pgfpathlineto{\pgfqpoint{-19.543393in}{2.309380in}}%
\pgfpathlineto{\pgfqpoint{-19.613839in}{2.312124in}}%
\pgfpathlineto{\pgfqpoint{-19.680997in}{2.412976in}}%
\pgfpathlineto{\pgfqpoint{-19.747542in}{2.447359in}}%
\pgfpathlineto{\pgfqpoint{-19.815535in}{2.399387in}}%
\pgfpathlineto{\pgfqpoint{-19.882267in}{2.418078in}}%
\pgfpathlineto{\pgfqpoint{-19.949841in}{2.427688in}}%
\pgfpathlineto{\pgfqpoint{-20.019359in}{2.379878in}}%
\pgfpathlineto{\pgfqpoint{-20.086943in}{2.465238in}}%
\pgfpathlineto{\pgfqpoint{-20.154101in}{2.391542in}}%
\pgfpathlineto{\pgfqpoint{-20.223560in}{2.440319in}}%
\pgfpathlineto{\pgfqpoint{-20.290460in}{2.462632in}}%
\pgfpathlineto{\pgfqpoint{-20.357252in}{2.396114in}}%
\pgfpathlineto{\pgfqpoint{-20.427103in}{2.352777in}}%
\pgfpathlineto{\pgfqpoint{-20.496847in}{2.377512in}}%
\pgfpathlineto{\pgfqpoint{-20.565295in}{2.445029in}}%
\pgfpathlineto{\pgfqpoint{-20.636228in}{2.355493in}}%
\pgfpathlineto{\pgfqpoint{-20.705385in}{2.362275in}}%
\pgfpathlineto{\pgfqpoint{-20.774312in}{2.465290in}}%
\pgfpathlineto{\pgfqpoint{-20.844220in}{2.385366in}}%
\pgfpathlineto{\pgfqpoint{-20.913231in}{2.392576in}}%
\pgfpathlineto{\pgfqpoint{-20.982338in}{2.317130in}}%
\pgfpathlineto{\pgfqpoint{-21.053709in}{2.385212in}}%
\pgfpathlineto{\pgfqpoint{-21.123238in}{2.375232in}}%
\pgfpathlineto{\pgfqpoint{-21.191384in}{2.401431in}}%
\pgfpathlineto{\pgfqpoint{-21.261541in}{2.378244in}}%
\pgfpathlineto{\pgfqpoint{-21.330134in}{2.459106in}}%
\pgfpathlineto{\pgfqpoint{-21.398346in}{2.387850in}}%
\pgfpathlineto{\pgfqpoint{-21.470634in}{2.316230in}}%
\pgfpathlineto{\pgfqpoint{-21.539549in}{2.404361in}}%
\pgfpathlineto{\pgfqpoint{-21.607622in}{2.524428in}}%
\pgfpathlineto{\pgfqpoint{-21.676089in}{2.370141in}}%
\pgfpathlineto{\pgfqpoint{-21.743422in}{2.428756in}}%
\pgfpathlineto{\pgfqpoint{-21.811964in}{2.326536in}}%
\pgfpathlineto{\pgfqpoint{-21.881469in}{2.451919in}}%
\pgfpathlineto{\pgfqpoint{-21.948488in}{2.351056in}}%
\pgfpathlineto{\pgfqpoint{-22.016091in}{2.401471in}}%
\pgfpathlineto{\pgfqpoint{-22.084923in}{2.503416in}}%
\pgfpathlineto{\pgfqpoint{-22.150098in}{2.483410in}}%
\pgfpathlineto{\pgfqpoint{-22.217508in}{2.332717in}}%
\pgfpathlineto{\pgfqpoint{-22.286974in}{2.435613in}}%
\pgfpathlineto{\pgfqpoint{-22.353759in}{2.374259in}}%
\pgfpathlineto{\pgfqpoint{-22.421718in}{2.342457in}}%
\pgfpathlineto{\pgfqpoint{-22.492630in}{2.324022in}}%
\pgfpathlineto{\pgfqpoint{-22.560590in}{2.437245in}}%
\pgfpathlineto{\pgfqpoint{-22.627376in}{2.492165in}}%
\pgfpathlineto{\pgfqpoint{-22.696138in}{2.355079in}}%
\pgfpathlineto{\pgfqpoint{-22.764599in}{2.418926in}}%
\pgfpathlineto{\pgfqpoint{-22.831621in}{2.493017in}}%
\pgfpathlineto{\pgfqpoint{-22.900479in}{2.343993in}}%
\pgfpathlineto{\pgfqpoint{-22.968635in}{2.369451in}}%
\pgfpathlineto{\pgfqpoint{-23.037151in}{2.347538in}}%
\pgfpathlineto{\pgfqpoint{-23.107951in}{2.407523in}}%
\pgfpathlineto{\pgfqpoint{-23.175783in}{2.376527in}}%
\pgfpathlineto{\pgfqpoint{-23.243519in}{2.415966in}}%
\pgfpathlineto{\pgfqpoint{-23.314406in}{2.274930in}}%
\pgfpathlineto{\pgfqpoint{-23.383303in}{2.339596in}}%
\pgfpathlineto{\pgfqpoint{-23.451780in}{2.350531in}}%
\pgfpathlineto{\pgfqpoint{-23.523979in}{2.304644in}}%
\pgfpathlineto{\pgfqpoint{-23.594251in}{2.310678in}}%
\pgfpathlineto{\pgfqpoint{-23.664795in}{2.276659in}}%
\pgfpathlineto{\pgfqpoint{-23.739951in}{2.263590in}}%
\pgfpathlineto{\pgfqpoint{-23.810815in}{2.387356in}}%
\pgfpathlineto{\pgfqpoint{-23.881105in}{2.329237in}}%
\pgfpathlineto{\pgfqpoint{-23.953546in}{2.329323in}}%
\pgfpathlineto{\pgfqpoint{-24.025129in}{2.320950in}}%
\pgfpathlineto{\pgfqpoint{-24.096931in}{2.253215in}}%
\pgfpathlineto{\pgfqpoint{-24.172427in}{2.272963in}}%
\pgfpathlineto{\pgfqpoint{-24.243942in}{2.256817in}}%
\pgfpathlineto{\pgfqpoint{-24.313406in}{2.444538in}}%
\pgfpathlineto{\pgfqpoint{-24.381203in}{2.517744in}}%
\pgfpathlineto{\pgfqpoint{-24.445935in}{2.463778in}}%
\pgfpathlineto{\pgfqpoint{-24.511431in}{2.524475in}}%
\pgfpathlineto{\pgfqpoint{-24.578904in}{2.473005in}}%
\pgfpathlineto{\pgfqpoint{-24.645693in}{2.423239in}}%
\pgfpathlineto{\pgfqpoint{-24.712889in}{2.407795in}}%
\pgfpathlineto{\pgfqpoint{-24.781655in}{2.413310in}}%
\pgfpathlineto{\pgfqpoint{-24.847966in}{2.381557in}}%
\pgfpathlineto{\pgfqpoint{-24.916019in}{2.349013in}}%
\pgfpathlineto{\pgfqpoint{-24.984232in}{2.521477in}}%
\pgfpathlineto{\pgfqpoint{-25.049351in}{2.471125in}}%
\pgfpathlineto{\pgfqpoint{-25.114674in}{2.483194in}}%
\pgfpathlineto{\pgfqpoint{-25.182394in}{2.468254in}}%
\pgfpathlineto{\pgfqpoint{-25.248757in}{2.413452in}}%
\pgfpathlineto{\pgfqpoint{-25.315375in}{2.491128in}}%
\pgfpathlineto{\pgfqpoint{-25.382786in}{2.396602in}}%
\pgfpathlineto{\pgfqpoint{-25.448976in}{2.347344in}}%
\pgfpathlineto{\pgfqpoint{-25.515370in}{2.454770in}}%
\pgfpathlineto{\pgfqpoint{-25.583742in}{2.415040in}}%
\pgfpathlineto{\pgfqpoint{-25.650614in}{2.422676in}}%
\pgfpathlineto{\pgfqpoint{-25.717568in}{2.515321in}}%
\pgfpathlineto{\pgfqpoint{-25.785528in}{2.438440in}}%
\pgfpathlineto{\pgfqpoint{-25.851769in}{2.424659in}}%
\pgfpathlineto{\pgfqpoint{-25.919100in}{2.340892in}}%
\pgfpathlineto{\pgfqpoint{-25.990773in}{2.344049in}}%
\pgfpathlineto{\pgfqpoint{-26.059544in}{2.353124in}}%
\pgfpathlineto{\pgfqpoint{-26.128033in}{2.396953in}}%
\pgfpathlineto{\pgfqpoint{-26.198294in}{2.320139in}}%
\pgfpathlineto{\pgfqpoint{-26.268985in}{2.365402in}}%
\pgfpathlineto{\pgfqpoint{-26.338662in}{2.370042in}}%
\pgfpathlineto{\pgfqpoint{-26.410031in}{2.299670in}}%
\pgfpathlineto{\pgfqpoint{-26.479146in}{2.395244in}}%
\pgfpathlineto{\pgfqpoint{-26.547093in}{2.392115in}}%
\pgfpathlineto{\pgfqpoint{-26.616985in}{2.344606in}}%
\pgfpathlineto{\pgfqpoint{-26.685925in}{2.310785in}}%
\pgfpathlineto{\pgfqpoint{-26.755672in}{2.338156in}}%
\pgfpathlineto{\pgfqpoint{-26.825734in}{2.472786in}}%
\pgfpathlineto{\pgfqpoint{-26.893662in}{2.369430in}}%
\pgfpathlineto{\pgfqpoint{-26.961733in}{2.366629in}}%
\pgfpathlineto{\pgfqpoint{-27.031677in}{2.389366in}}%
\pgfpathlineto{\pgfqpoint{-27.097972in}{2.420274in}}%
\pgfpathlineto{\pgfqpoint{-27.165061in}{2.441472in}}%
\pgfpathlineto{\pgfqpoint{-27.232937in}{2.440327in}}%
\pgfpathlineto{\pgfqpoint{-27.299189in}{2.432371in}}%
\pgfpathlineto{\pgfqpoint{-27.366087in}{2.363470in}}%
\pgfpathlineto{\pgfqpoint{-27.435671in}{2.371017in}}%
\pgfpathlineto{\pgfqpoint{-27.503250in}{2.276365in}}%
\pgfpathlineto{\pgfqpoint{-27.570291in}{2.406194in}}%
\pgfpathlineto{\pgfqpoint{-27.637925in}{2.500852in}}%
\pgfpathlineto{\pgfqpoint{-27.703951in}{2.400018in}}%
\pgfpathlineto{\pgfqpoint{-27.771032in}{2.459129in}}%
\pgfpathlineto{\pgfqpoint{-27.839702in}{2.399915in}}%
\pgfpathlineto{\pgfqpoint{-27.908279in}{2.356364in}}%
\pgfpathlineto{\pgfqpoint{-27.976694in}{2.415234in}}%
\pgfpathlineto{\pgfqpoint{-28.049197in}{2.293216in}}%
\pgfpathlineto{\pgfqpoint{-28.121663in}{2.273638in}}%
\pgfpathlineto{\pgfqpoint{-28.195772in}{2.204051in}}%
\pgfpathlineto{\pgfqpoint{-28.274638in}{2.117445in}}%
\pgfpathlineto{\pgfqpoint{-28.356803in}{2.282530in}}%
\pgfpathlineto{\pgfqpoint{-28.429123in}{2.295769in}}%
\pgfpathlineto{\pgfqpoint{-28.504596in}{2.262722in}}%
\pgfpathlineto{\pgfqpoint{-28.577221in}{2.336790in}}%
\pgfpathlineto{\pgfqpoint{-28.649563in}{2.185241in}}%
\pgfpathlineto{\pgfqpoint{-28.725736in}{2.273706in}}%
\pgfpathlineto{\pgfqpoint{-28.798746in}{2.296750in}}%
\pgfpathlineto{\pgfqpoint{-28.870906in}{2.252813in}}%
\pgfpathlineto{\pgfqpoint{-28.946213in}{2.286411in}}%
\pgfpathlineto{\pgfqpoint{-29.018119in}{2.313522in}}%
\pgfpathlineto{\pgfqpoint{-29.089467in}{2.298530in}}%
\pgfpathlineto{\pgfqpoint{-29.164288in}{2.273118in}}%
\pgfpathlineto{\pgfqpoint{-29.235084in}{2.357001in}}%
\pgfpathlineto{\pgfqpoint{-29.304065in}{2.352023in}}%
\pgfpathlineto{\pgfqpoint{-29.374900in}{2.994942in}}%
\pgfpathlineto{\pgfqpoint{-29.443353in}{3.199197in}}%
\pgfpathlineto{\pgfqpoint{-29.511058in}{3.226152in}}%
\pgfpathlineto{\pgfqpoint{-29.581285in}{3.206483in}}%
\pgfpathlineto{\pgfqpoint{-29.647778in}{3.206165in}}%
\pgfpathlineto{\pgfqpoint{-29.716346in}{3.194228in}}%
\pgfpathlineto{\pgfqpoint{-29.785834in}{3.210954in}}%
\pgfpathlineto{\pgfqpoint{-29.853575in}{3.139991in}}%
\pgfpathlineto{\pgfqpoint{-29.923374in}{3.080656in}}%
\pgfpathlineto{\pgfqpoint{-29.994776in}{3.128937in}}%
\pgfpathlineto{\pgfqpoint{-30.062931in}{3.199521in}}%
\pgfpathlineto{\pgfqpoint{-30.131142in}{3.112266in}}%
\pgfpathlineto{\pgfqpoint{-30.202864in}{3.185522in}}%
\pgfpathlineto{\pgfqpoint{-30.269201in}{3.179317in}}%
\pgfpathlineto{\pgfqpoint{-30.335915in}{3.189818in}}%
\pgfpathlineto{\pgfqpoint{-30.404868in}{3.146493in}}%
\pgfpathlineto{\pgfqpoint{-30.472294in}{3.268734in}}%
\pgfpathlineto{\pgfqpoint{-30.539383in}{3.112906in}}%
\pgfpathlineto{\pgfqpoint{-30.610127in}{3.203568in}}%
\pgfpathlineto{\pgfqpoint{-30.676901in}{3.217388in}}%
\pgfpathlineto{\pgfqpoint{-30.745143in}{2.186873in}}%
\pgfpathclose%
\pgfusepath{fill}%
\end{pgfscope}%
\begin{pgfscope}%
\pgfsetbuttcap%
\pgfsetroundjoin%
\definecolor{currentfill}{rgb}{0.000000,0.000000,0.000000}%
\pgfsetfillcolor{currentfill}%
\pgfsetlinewidth{0.803000pt}%
\definecolor{currentstroke}{rgb}{0.000000,0.000000,0.000000}%
\pgfsetstrokecolor{currentstroke}%
\pgfsetdash{}{0pt}%
\pgfsys@defobject{currentmarker}{\pgfqpoint{0.000000in}{-0.048611in}}{\pgfqpoint{0.000000in}{0.000000in}}{%
\pgfpathmoveto{\pgfqpoint{0.000000in}{0.000000in}}%
\pgfpathlineto{\pgfqpoint{0.000000in}{-0.048611in}}%
\pgfusepath{stroke,fill}%
}%
\begin{pgfscope}%
\pgfsys@transformshift{3.575311in}{0.773588in}%
\pgfsys@useobject{currentmarker}{}%
\end{pgfscope}%
\end{pgfscope}%
\begin{pgfscope}%
\definecolor{textcolor}{rgb}{0.000000,0.000000,0.000000}%
\pgfsetstrokecolor{textcolor}%
\pgfsetfillcolor{textcolor}%
\pgftext[x=3.575311in,y=0.676366in,,top]{\color{textcolor}\rmfamily\fontsize{10.000000}{12.000000}\selectfont \(\displaystyle {2450}\)}%
\end{pgfscope}%
\begin{pgfscope}%
\pgfsetbuttcap%
\pgfsetroundjoin%
\definecolor{currentfill}{rgb}{0.000000,0.000000,0.000000}%
\pgfsetfillcolor{currentfill}%
\pgfsetlinewidth{0.803000pt}%
\definecolor{currentstroke}{rgb}{0.000000,0.000000,0.000000}%
\pgfsetstrokecolor{currentstroke}%
\pgfsetdash{}{0pt}%
\pgfsys@defobject{currentmarker}{\pgfqpoint{0.000000in}{-0.048611in}}{\pgfqpoint{0.000000in}{0.000000in}}{%
\pgfpathmoveto{\pgfqpoint{0.000000in}{0.000000in}}%
\pgfpathlineto{\pgfqpoint{0.000000in}{-0.048611in}}%
\pgfusepath{stroke,fill}%
}%
\begin{pgfscope}%
\pgfsys@transformshift{4.276252in}{0.773588in}%
\pgfsys@useobject{currentmarker}{}%
\end{pgfscope}%
\end{pgfscope}%
\begin{pgfscope}%
\definecolor{textcolor}{rgb}{0.000000,0.000000,0.000000}%
\pgfsetstrokecolor{textcolor}%
\pgfsetfillcolor{textcolor}%
\pgftext[x=4.276252in,y=0.676366in,,top]{\color{textcolor}\rmfamily\fontsize{10.000000}{12.000000}\selectfont \(\displaystyle {2500}\)}%
\end{pgfscope}%
\begin{pgfscope}%
\pgfsetbuttcap%
\pgfsetroundjoin%
\definecolor{currentfill}{rgb}{0.000000,0.000000,0.000000}%
\pgfsetfillcolor{currentfill}%
\pgfsetlinewidth{0.803000pt}%
\definecolor{currentstroke}{rgb}{0.000000,0.000000,0.000000}%
\pgfsetstrokecolor{currentstroke}%
\pgfsetdash{}{0pt}%
\pgfsys@defobject{currentmarker}{\pgfqpoint{0.000000in}{-0.048611in}}{\pgfqpoint{0.000000in}{0.000000in}}{%
\pgfpathmoveto{\pgfqpoint{0.000000in}{0.000000in}}%
\pgfpathlineto{\pgfqpoint{0.000000in}{-0.048611in}}%
\pgfusepath{stroke,fill}%
}%
\begin{pgfscope}%
\pgfsys@transformshift{4.977194in}{0.773588in}%
\pgfsys@useobject{currentmarker}{}%
\end{pgfscope}%
\end{pgfscope}%
\begin{pgfscope}%
\definecolor{textcolor}{rgb}{0.000000,0.000000,0.000000}%
\pgfsetstrokecolor{textcolor}%
\pgfsetfillcolor{textcolor}%
\pgftext[x=4.977194in,y=0.676366in,,top]{\color{textcolor}\rmfamily\fontsize{10.000000}{12.000000}\selectfont \(\displaystyle {2550}\)}%
\end{pgfscope}%
\begin{pgfscope}%
\pgfsetrectcap%
\pgfsetroundjoin%
\pgfsetlinewidth{0.803000pt}%
\definecolor{currentstroke}{rgb}{0.000000,0.000000,0.000000}%
\pgfsetstrokecolor{currentstroke}%
\pgfsetdash{}{0pt}%
\pgfpathmoveto{\pgfqpoint{3.265876in}{0.707284in}}%
\pgfpathlineto{\pgfqpoint{3.398484in}{0.839893in}}%
\pgfusepath{stroke}%
\end{pgfscope}%
\begin{pgfscope}%
\pgfpathrectangle{\pgfqpoint{3.332180in}{0.773588in}}{\pgfqpoint{2.293918in}{5.415119in}}%
\pgfusepath{clip}%
\pgfsetrectcap%
\pgfsetroundjoin%
\pgfsetlinewidth{1.505625pt}%
\definecolor{currentstroke}{rgb}{0.000000,0.000000,1.000000}%
\pgfsetstrokecolor{currentstroke}%
\pgfsetdash{}{0pt}%
\pgfusepath{stroke}%
\end{pgfscope}%
\begin{pgfscope}%
\pgfpathrectangle{\pgfqpoint{3.332180in}{0.773588in}}{\pgfqpoint{2.293918in}{5.415119in}}%
\pgfusepath{clip}%
\pgfsetrectcap%
\pgfsetroundjoin%
\pgfsetlinewidth{1.505625pt}%
\definecolor{currentstroke}{rgb}{0.750000,0.750000,0.000000}%
\pgfsetstrokecolor{currentstroke}%
\pgfsetdash{}{0pt}%
\pgfpathmoveto{\pgfqpoint{4.033122in}{0.773588in}}%
\pgfpathlineto{\pgfqpoint{4.033122in}{6.188708in}}%
\pgfusepath{stroke}%
\end{pgfscope}%
\begin{pgfscope}%
\pgfpathrectangle{\pgfqpoint{3.332180in}{0.773588in}}{\pgfqpoint{2.293918in}{5.415119in}}%
\pgfusepath{clip}%
\pgfsetrectcap%
\pgfsetroundjoin%
\pgfsetlinewidth{1.505625pt}%
\definecolor{currentstroke}{rgb}{0.750000,0.000000,0.750000}%
\pgfsetstrokecolor{currentstroke}%
\pgfsetdash{}{0pt}%
\pgfpathmoveto{\pgfqpoint{4.072622in}{0.773588in}}%
\pgfpathlineto{\pgfqpoint{4.072622in}{6.188708in}}%
\pgfusepath{stroke}%
\end{pgfscope}%
\begin{pgfscope}%
\pgfpathrectangle{\pgfqpoint{3.332180in}{0.773588in}}{\pgfqpoint{2.293918in}{5.415119in}}%
\pgfusepath{clip}%
\pgfsetrectcap%
\pgfsetroundjoin%
\pgfsetlinewidth{1.505625pt}%
\definecolor{currentstroke}{rgb}{1.000000,0.000000,0.000000}%
\pgfsetstrokecolor{currentstroke}%
\pgfsetdash{}{0pt}%
\pgfpathmoveto{\pgfqpoint{4.161237in}{0.773588in}}%
\pgfpathlineto{\pgfqpoint{4.161237in}{6.188708in}}%
\pgfusepath{stroke}%
\end{pgfscope}%
\begin{pgfscope}%
\pgfpathrectangle{\pgfqpoint{3.332180in}{0.773588in}}{\pgfqpoint{2.293918in}{5.415119in}}%
\pgfusepath{clip}%
\pgfsetrectcap%
\pgfsetroundjoin%
\pgfsetlinewidth{1.505625pt}%
\definecolor{currentstroke}{rgb}{0.000000,0.500000,0.000000}%
\pgfsetstrokecolor{currentstroke}%
\pgfsetdash{}{0pt}%
\pgfpathmoveto{\pgfqpoint{4.161315in}{0.773588in}}%
\pgfpathlineto{\pgfqpoint{4.161315in}{6.188708in}}%
\pgfusepath{stroke}%
\end{pgfscope}%
\begin{pgfscope}%
\pgfpathrectangle{\pgfqpoint{3.332180in}{0.773588in}}{\pgfqpoint{2.293918in}{5.415119in}}%
\pgfusepath{clip}%
\pgfsetrectcap%
\pgfsetroundjoin%
\pgfsetlinewidth{1.505625pt}%
\definecolor{currentstroke}{rgb}{0.000000,0.750000,0.750000}%
\pgfsetstrokecolor{currentstroke}%
\pgfsetdash{}{0pt}%
\pgfpathmoveto{\pgfqpoint{4.161855in}{0.773588in}}%
\pgfpathlineto{\pgfqpoint{4.161855in}{6.188708in}}%
\pgfusepath{stroke}%
\end{pgfscope}%
\begin{pgfscope}%
\pgfpathrectangle{\pgfqpoint{3.332180in}{0.773588in}}{\pgfqpoint{2.293918in}{5.415119in}}%
\pgfusepath{clip}%
\pgfsetrectcap%
\pgfsetroundjoin%
\pgfsetlinewidth{1.505625pt}%
\definecolor{currentstroke}{rgb}{0.750000,0.000000,0.750000}%
\pgfsetstrokecolor{currentstroke}%
\pgfsetdash{}{0pt}%
\pgfpathmoveto{\pgfqpoint{4.224215in}{0.773588in}}%
\pgfpathlineto{\pgfqpoint{4.224215in}{6.188708in}}%
\pgfusepath{stroke}%
\end{pgfscope}%
\begin{pgfscope}%
\pgfsetrectcap%
\pgfsetmiterjoin%
\pgfsetlinewidth{0.803000pt}%
\definecolor{currentstroke}{rgb}{0.000000,0.000000,0.000000}%
\pgfsetstrokecolor{currentstroke}%
\pgfsetdash{}{0pt}%
\pgfpathmoveto{\pgfqpoint{3.332180in}{0.773588in}}%
\pgfpathlineto{\pgfqpoint{5.626098in}{0.773588in}}%
\pgfusepath{stroke}%
\end{pgfscope}%
\begin{pgfscope}%
\pgfsetbuttcap%
\pgfsetmiterjoin%
\pgfsetlinewidth{0.000000pt}%
\definecolor{currentstroke}{rgb}{0.000000,0.000000,0.000000}%
\pgfsetstrokecolor{currentstroke}%
\pgfsetstrokeopacity{0.000000}%
\pgfsetdash{}{0pt}%
\pgfpathmoveto{\pgfqpoint{0.781402in}{0.773588in}}%
\pgfpathlineto{\pgfqpoint{5.626098in}{0.773588in}}%
\pgfpathlineto{\pgfqpoint{5.626098in}{6.188708in}}%
\pgfpathlineto{\pgfqpoint{0.781402in}{6.188708in}}%
\pgfpathclose%
\pgfusepath{}%
\end{pgfscope}%
\begin{pgfscope}%
\definecolor{textcolor}{rgb}{0.000000,0.000000,0.000000}%
\pgfsetstrokecolor{textcolor}%
\pgfsetfillcolor{textcolor}%
\pgftext[x=3.203750in,y=0.356922in,,top]{\color{textcolor}\rmfamily\fontsize{10.000000}{12.000000}\selectfont Time (milliseconds)}%
\end{pgfscope}%
\begin{pgfscope}%
\definecolor{textcolor}{rgb}{0.000000,0.000000,0.000000}%
\pgfsetstrokecolor{textcolor}%
\pgfsetfillcolor{textcolor}%
\pgftext[x=0.364736in,y=3.481148in,,bottom,rotate=90.000000]{\color{textcolor}\rmfamily\fontsize{10.000000}{12.000000}\selectfont Throughput (million operations/second)}%
\end{pgfscope}%
\begin{pgfscope}%
\pgfsetbuttcap%
\pgfsetmiterjoin%
\definecolor{currentfill}{rgb}{1.000000,1.000000,1.000000}%
\pgfsetfillcolor{currentfill}%
\pgfsetfillopacity{0.800000}%
\pgfsetlinewidth{1.003750pt}%
\definecolor{currentstroke}{rgb}{0.800000,0.800000,0.800000}%
\pgfsetstrokecolor{currentstroke}%
\pgfsetstrokeopacity{0.800000}%
\pgfsetdash{}{0pt}%
\pgfpathmoveto{\pgfqpoint{0.878625in}{3.559851in}}%
\pgfpathlineto{\pgfqpoint{3.799925in}{3.559851in}}%
\pgfpathquadraticcurveto{\pgfqpoint{3.827703in}{3.559851in}}{\pgfqpoint{3.827703in}{3.587628in}}%
\pgfpathlineto{\pgfqpoint{3.827703in}{6.091486in}}%
\pgfpathquadraticcurveto{\pgfqpoint{3.827703in}{6.119263in}}{\pgfqpoint{3.799925in}{6.119263in}}%
\pgfpathlineto{\pgfqpoint{0.878625in}{6.119263in}}%
\pgfpathquadraticcurveto{\pgfqpoint{0.850847in}{6.119263in}}{\pgfqpoint{0.850847in}{6.091486in}}%
\pgfpathlineto{\pgfqpoint{0.850847in}{3.587628in}}%
\pgfpathquadraticcurveto{\pgfqpoint{0.850847in}{3.559851in}}{\pgfqpoint{0.878625in}{3.559851in}}%
\pgfpathclose%
\pgfusepath{stroke,fill}%
\end{pgfscope}%
\begin{pgfscope}%
\pgfsetrectcap%
\pgfsetroundjoin%
\pgfsetlinewidth{1.505625pt}%
\definecolor{currentstroke}{rgb}{0.000000,0.000000,1.000000}%
\pgfsetstrokecolor{currentstroke}%
\pgfsetdash{}{0pt}%
\pgfpathmoveto{\pgfqpoint{0.906402in}{6.015097in}}%
\pgfpathlineto{\pgfqpoint{1.184180in}{6.015097in}}%
\pgfusepath{stroke}%
\end{pgfscope}%
\begin{pgfscope}%
\definecolor{textcolor}{rgb}{0.000000,0.000000,0.000000}%
\pgfsetstrokecolor{textcolor}%
\pgfsetfillcolor{textcolor}%
\pgftext[x=1.295291in,y=5.966486in,left,base]{\color{textcolor}\rmfamily\fontsize{10.000000}{12.000000}\selectfont Started migration}%
\end{pgfscope}%
\begin{pgfscope}%
\pgfsetrectcap%
\pgfsetroundjoin%
\pgfsetlinewidth{1.505625pt}%
\definecolor{currentstroke}{rgb}{0.750000,0.750000,0.000000}%
\pgfsetstrokecolor{currentstroke}%
\pgfsetdash{}{0pt}%
\pgfpathmoveto{\pgfqpoint{0.906402in}{5.821424in}}%
\pgfpathlineto{\pgfqpoint{1.184180in}{5.821424in}}%
\pgfusepath{stroke}%
\end{pgfscope}%
\begin{pgfscope}%
\definecolor{textcolor}{rgb}{0.000000,0.000000,0.000000}%
\pgfsetstrokecolor{textcolor}%
\pgfsetfillcolor{textcolor}%
\pgftext[x=1.295291in,y=5.772813in,left,base]{\color{textcolor}\rmfamily\fontsize{10.000000}{12.000000}\selectfont Finished prefill writes}%
\end{pgfscope}%
\begin{pgfscope}%
\pgfsetrectcap%
\pgfsetroundjoin%
\pgfsetlinewidth{1.505625pt}%
\definecolor{currentstroke}{rgb}{0.750000,0.000000,0.750000}%
\pgfsetstrokecolor{currentstroke}%
\pgfsetdash{}{0pt}%
\pgfpathmoveto{\pgfqpoint{0.906402in}{5.627751in}}%
\pgfpathlineto{\pgfqpoint{1.184180in}{5.627751in}}%
\pgfusepath{stroke}%
\end{pgfscope}%
\begin{pgfscope}%
\definecolor{textcolor}{rgb}{0.000000,0.000000,0.000000}%
\pgfsetstrokecolor{textcolor}%
\pgfsetfillcolor{textcolor}%
\pgftext[x=1.295291in,y=5.579140in,left,base]{\color{textcolor}\rmfamily\fontsize{10.000000}{12.000000}\selectfont finished writes}%
\end{pgfscope}%
\begin{pgfscope}%
\pgfsetrectcap%
\pgfsetroundjoin%
\pgfsetlinewidth{1.505625pt}%
\definecolor{currentstroke}{rgb}{1.000000,0.000000,0.000000}%
\pgfsetstrokecolor{currentstroke}%
\pgfsetdash{}{0pt}%
\pgfpathmoveto{\pgfqpoint{0.906402in}{5.434078in}}%
\pgfpathlineto{\pgfqpoint{1.184180in}{5.434078in}}%
\pgfusepath{stroke}%
\end{pgfscope}%
\begin{pgfscope}%
\definecolor{textcolor}{rgb}{0.000000,0.000000,0.000000}%
\pgfsetstrokecolor{textcolor}%
\pgfsetfillcolor{textcolor}%
\pgftext[x=1.295291in,y=5.385467in,left,base]{\color{textcolor}\rmfamily\fontsize{10.000000}{12.000000}\selectfont finished reads}%
\end{pgfscope}%
\begin{pgfscope}%
\pgfsetrectcap%
\pgfsetroundjoin%
\pgfsetlinewidth{1.505625pt}%
\definecolor{currentstroke}{rgb}{0.000000,0.500000,0.000000}%
\pgfsetstrokecolor{currentstroke}%
\pgfsetdash{}{0pt}%
\pgfpathmoveto{\pgfqpoint{0.906402in}{5.240406in}}%
\pgfpathlineto{\pgfqpoint{1.184180in}{5.240406in}}%
\pgfusepath{stroke}%
\end{pgfscope}%
\begin{pgfscope}%
\definecolor{textcolor}{rgb}{0.000000,0.000000,0.000000}%
\pgfsetstrokecolor{textcolor}%
\pgfsetfillcolor{textcolor}%
\pgftext[x=1.295291in,y=5.191794in,left,base]{\color{textcolor}\rmfamily\fontsize{10.000000}{12.000000}\selectfont Transferred ownership to the destination}%
\end{pgfscope}%
\begin{pgfscope}%
\pgfsetrectcap%
\pgfsetroundjoin%
\pgfsetlinewidth{1.505625pt}%
\definecolor{currentstroke}{rgb}{0.000000,0.750000,0.750000}%
\pgfsetstrokecolor{currentstroke}%
\pgfsetdash{}{0pt}%
\pgfpathmoveto{\pgfqpoint{0.906402in}{5.046733in}}%
\pgfpathlineto{\pgfqpoint{1.184180in}{5.046733in}}%
\pgfusepath{stroke}%
\end{pgfscope}%
\begin{pgfscope}%
\definecolor{textcolor}{rgb}{0.000000,0.000000,0.000000}%
\pgfsetstrokecolor{textcolor}%
\pgfsetfillcolor{textcolor}%
\pgftext[x=1.295291in,y=4.998122in,left,base]{\color{textcolor}\rmfamily\fontsize{10.000000}{12.000000}\selectfont Started reading dirty pages}%
\end{pgfscope}%
\begin{pgfscope}%
\pgfsetrectcap%
\pgfsetroundjoin%
\pgfsetlinewidth{1.505625pt}%
\definecolor{currentstroke}{rgb}{0.750000,0.000000,0.750000}%
\pgfsetstrokecolor{currentstroke}%
\pgfsetdash{}{0pt}%
\pgfpathmoveto{\pgfqpoint{0.906402in}{4.853060in}}%
\pgfpathlineto{\pgfqpoint{1.184180in}{4.853060in}}%
\pgfusepath{stroke}%
\end{pgfscope}%
\begin{pgfscope}%
\definecolor{textcolor}{rgb}{0.000000,0.000000,0.000000}%
\pgfsetstrokecolor{textcolor}%
\pgfsetfillcolor{textcolor}%
\pgftext[x=1.295291in,y=4.804449in,left,base]{\color{textcolor}\rmfamily\fontsize{10.000000}{12.000000}\selectfont Finished reading dirty pages}%
\end{pgfscope}%
\begin{pgfscope}%
\pgfsetbuttcap%
\pgfsetmiterjoin%
\definecolor{currentfill}{rgb}{0.121569,0.466667,0.705882}%
\pgfsetfillcolor{currentfill}%
\pgfsetlinewidth{0.000000pt}%
\definecolor{currentstroke}{rgb}{0.000000,0.000000,0.000000}%
\pgfsetstrokecolor{currentstroke}%
\pgfsetstrokeopacity{0.000000}%
\pgfsetdash{}{0pt}%
\pgfpathmoveto{\pgfqpoint{0.906402in}{4.610776in}}%
\pgfpathlineto{\pgfqpoint{1.184180in}{4.610776in}}%
\pgfpathlineto{\pgfqpoint{1.184180in}{4.707998in}}%
\pgfpathlineto{\pgfqpoint{0.906402in}{4.707998in}}%
\pgfpathclose%
\pgfusepath{fill}%
\end{pgfscope}%
\begin{pgfscope}%
\definecolor{textcolor}{rgb}{0.000000,0.000000,0.000000}%
\pgfsetstrokecolor{textcolor}%
\pgfsetfillcolor{textcolor}%
\pgftext[x=1.295291in,y=4.610776in,left,base]{\color{textcolor}\rmfamily\fontsize{10.000000}{12.000000}\selectfont MP read at destination}%
\end{pgfscope}%
\begin{pgfscope}%
\pgfsetbuttcap%
\pgfsetmiterjoin%
\definecolor{currentfill}{rgb}{1.000000,0.498039,0.054902}%
\pgfsetfillcolor{currentfill}%
\pgfsetlinewidth{0.000000pt}%
\definecolor{currentstroke}{rgb}{0.000000,0.000000,0.000000}%
\pgfsetstrokecolor{currentstroke}%
\pgfsetstrokeopacity{0.000000}%
\pgfsetdash{}{0pt}%
\pgfpathmoveto{\pgfqpoint{0.906402in}{4.417103in}}%
\pgfpathlineto{\pgfqpoint{1.184180in}{4.417103in}}%
\pgfpathlineto{\pgfqpoint{1.184180in}{4.514326in}}%
\pgfpathlineto{\pgfqpoint{0.906402in}{4.514326in}}%
\pgfpathclose%
\pgfusepath{fill}%
\end{pgfscope}%
\begin{pgfscope}%
\definecolor{textcolor}{rgb}{0.000000,0.000000,0.000000}%
\pgfsetstrokecolor{textcolor}%
\pgfsetfillcolor{textcolor}%
\pgftext[x=1.295291in,y=4.417103in,left,base]{\color{textcolor}\rmfamily\fontsize{10.000000}{12.000000}\selectfont MP update at destination}%
\end{pgfscope}%
\begin{pgfscope}%
\pgfsetbuttcap%
\pgfsetmiterjoin%
\definecolor{currentfill}{rgb}{0.172549,0.627451,0.172549}%
\pgfsetfillcolor{currentfill}%
\pgfsetlinewidth{0.000000pt}%
\definecolor{currentstroke}{rgb}{0.000000,0.000000,0.000000}%
\pgfsetstrokecolor{currentstroke}%
\pgfsetstrokeopacity{0.000000}%
\pgfsetdash{}{0pt}%
\pgfpathmoveto{\pgfqpoint{0.906402in}{4.223431in}}%
\pgfpathlineto{\pgfqpoint{1.184180in}{4.223431in}}%
\pgfpathlineto{\pgfqpoint{1.184180in}{4.320653in}}%
\pgfpathlineto{\pgfqpoint{0.906402in}{4.320653in}}%
\pgfpathclose%
\pgfusepath{fill}%
\end{pgfscope}%
\begin{pgfscope}%
\definecolor{textcolor}{rgb}{0.000000,0.000000,0.000000}%
\pgfsetstrokecolor{textcolor}%
\pgfsetfillcolor{textcolor}%
\pgftext[x=1.295291in,y=4.223431in,left,base]{\color{textcolor}\rmfamily\fontsize{10.000000}{12.000000}\selectfont MP insert at destination}%
\end{pgfscope}%
\begin{pgfscope}%
\pgfsetbuttcap%
\pgfsetmiterjoin%
\definecolor{currentfill}{rgb}{0.839216,0.152941,0.156863}%
\pgfsetfillcolor{currentfill}%
\pgfsetlinewidth{0.000000pt}%
\definecolor{currentstroke}{rgb}{0.000000,0.000000,0.000000}%
\pgfsetstrokecolor{currentstroke}%
\pgfsetstrokeopacity{0.000000}%
\pgfsetdash{}{0pt}%
\pgfpathmoveto{\pgfqpoint{0.906402in}{4.029758in}}%
\pgfpathlineto{\pgfqpoint{1.184180in}{4.029758in}}%
\pgfpathlineto{\pgfqpoint{1.184180in}{4.126980in}}%
\pgfpathlineto{\pgfqpoint{0.906402in}{4.126980in}}%
\pgfpathclose%
\pgfusepath{fill}%
\end{pgfscope}%
\begin{pgfscope}%
\definecolor{textcolor}{rgb}{0.000000,0.000000,0.000000}%
\pgfsetstrokecolor{textcolor}%
\pgfsetfillcolor{textcolor}%
\pgftext[x=1.295291in,y=4.029758in,left,base]{\color{textcolor}\rmfamily\fontsize{10.000000}{12.000000}\selectfont MP read at source}%
\end{pgfscope}%
\begin{pgfscope}%
\pgfsetbuttcap%
\pgfsetmiterjoin%
\definecolor{currentfill}{rgb}{0.580392,0.403922,0.741176}%
\pgfsetfillcolor{currentfill}%
\pgfsetlinewidth{0.000000pt}%
\definecolor{currentstroke}{rgb}{0.000000,0.000000,0.000000}%
\pgfsetstrokecolor{currentstroke}%
\pgfsetstrokeopacity{0.000000}%
\pgfsetdash{}{0pt}%
\pgfpathmoveto{\pgfqpoint{0.906402in}{3.836085in}}%
\pgfpathlineto{\pgfqpoint{1.184180in}{3.836085in}}%
\pgfpathlineto{\pgfqpoint{1.184180in}{3.933307in}}%
\pgfpathlineto{\pgfqpoint{0.906402in}{3.933307in}}%
\pgfpathclose%
\pgfusepath{fill}%
\end{pgfscope}%
\begin{pgfscope}%
\definecolor{textcolor}{rgb}{0.000000,0.000000,0.000000}%
\pgfsetstrokecolor{textcolor}%
\pgfsetfillcolor{textcolor}%
\pgftext[x=1.295291in,y=3.836085in,left,base]{\color{textcolor}\rmfamily\fontsize{10.000000}{12.000000}\selectfont MP update at source}%
\end{pgfscope}%
\begin{pgfscope}%
\pgfsetbuttcap%
\pgfsetmiterjoin%
\definecolor{currentfill}{rgb}{0.549020,0.337255,0.294118}%
\pgfsetfillcolor{currentfill}%
\pgfsetlinewidth{0.000000pt}%
\definecolor{currentstroke}{rgb}{0.000000,0.000000,0.000000}%
\pgfsetstrokecolor{currentstroke}%
\pgfsetstrokeopacity{0.000000}%
\pgfsetdash{}{0pt}%
\pgfpathmoveto{\pgfqpoint{0.906402in}{3.642412in}}%
\pgfpathlineto{\pgfqpoint{1.184180in}{3.642412in}}%
\pgfpathlineto{\pgfqpoint{1.184180in}{3.739634in}}%
\pgfpathlineto{\pgfqpoint{0.906402in}{3.739634in}}%
\pgfpathclose%
\pgfusepath{fill}%
\end{pgfscope}%
\begin{pgfscope}%
\definecolor{textcolor}{rgb}{0.000000,0.000000,0.000000}%
\pgfsetstrokecolor{textcolor}%
\pgfsetfillcolor{textcolor}%
\pgftext[x=1.295291in,y=3.642412in,left,base]{\color{textcolor}\rmfamily\fontsize{10.000000}{12.000000}\selectfont MP insert at source}%
\end{pgfscope}%
\end{pgfpicture}%
\makeatother%
\endgroup%

    \end{center}
    \caption{Migration timeline of a map (2MB huge pages)}
    \label{fig:maphp}
\end{figure}


Each machine shares its processing time among the hash table partitions that
it owns. Initially, the target partition (MP) is owned by the source machine, but half of
the processing time of the source machine is allocated to another partition that
it owns (throughput for this other partition is omitted for brevity). As a result
the source machine decides to migrate MP to the destination machine, where MP
can use the available CPU time to achieve better throughput.

\autoref{fig:map} and \autoref{fig:maphp} show the timeline and the throughput
over time of the MP object as it is being migrated with 4KB and 2MB pages
respectively. Insert operations add new elements to the map, updates change
the value of a key in-place, and reads query a key. After starting the migration
we are no longer allowed to issue insert operations as they would require
memory allocations.

At the start of the prefill phase starting at 100ms, MP has grown to around 40MB.
During the long prefill phases, read and write operations on MP continue and as
a result, half of the pages in the 4KB pages case, and 23 out of the 24 pages in the
2MB page case are dirtied. Even though the finer grain 4KB pages detected the
dirty memory locations more accurately, huge pages outperformed 4KB pages in all
metrics in all configurations (MP size, read/write window closure duration,
dirty page percentage, etc.) of this workload, thanks to the 100Gbps network
throughput.

The end to end latency in this scenario is large but is not an important
performance factor because the operations
on the map that were needed to dirty the underlying pages of MP dominate
the migration duration.



% \TODO{this model encourages fixed objects, where you don't need to allocate repeatedly: bloom filter, hash table with fixed size values + ds's where access is local}
%\TODO{build a function as a service framework on this}
%\TODO{can this be used for other systems such as parallel processing
% based on actor models and message queues with efficient support for
%e.g., fan out}
% supporting read-only operations and even write operations with carefully
% created static buffers that can contain unsupported (those that allocate/deallocate)
% operations that the object has received after a call to initiate migration has
% already been made.
% big table can be implemented easily using Slope
% as opposed to the original methods where we only think about the movement of
% objects, here we think about movement of servers

\chapter{Related Work}
\label{chap:related}
In this section we discuss how research on smililar topics relates to
Slope, what makes Slope different, and how ideas from prior work across
multiple categories can be incorporated into Slope in the future.

\section{RAMP}
Slope borrows the notion of shared address spaces from RAMP
\cite{memon2018ramp}. This closely relates RAMP and Slope and therefore
a rundown of the differences between the two systems can be helpful.

\subsection{Programmability}
Slope makes data structures or more generally, self-contained units of
work migratable in a black box fashion, without forcing modifications to
their implementations in existing applications.
This means that any C++ entity which is capable of using a custom
allocator, including STL containers, can be made migratable through Slope
with almost zero extra programming effort. Furthermore Slope benefits
from the composition friendliness of the objects that conform to C++ 
allocator named requirement, making it easy to create complex migratable
types from simple building blocks.

RAMP in contrast not only discusses the
migration of memory segments,
but also uses a stateful memory allocator which makes it
backwards incompatible with existing C++ software.

\subsection{Usability}
We present a programming model for migratable objects which makes them
usable in at least two broad families of applications as discussed in
\ref{sec:api}, showing how this model can be used for high performance
applications in the real world.

On the other hand, RAMP introduces a bare bones memory segment migration
platform which outside of a controlled experiment will be at most as
useful as an RDMA-based transport, if at all usable,
because of the reasons discussed in the \ref{subsec:sendrecmig}.

\subsection{Migration performance}
Based on the type of workload, Slope benefits from prefilling the
destination machine's memory, resulting in smaller hand-off times and
quicker convergence to the steady state throughput.


RAMP only starts transferring the contents of a memory segment after
the segment ownership has been transferred. This results in its
convergence period or its window of unresponsiveness to go up to hundreds
of milliseconds for a 128 MB segment, while based on the type of
workload, Slope can finish the migration with as low as 10us with
negligible throughput distortion.

In Slope the number of parallel migrations between \emph{each pair}
of machines at any given time is only bounded by the number of receives
pre-posted to their queue pairs, whereas in RAMP, at any point in time
there can be at most a single migration in the \emph{whole cluster}.


\subsection{Memory allocation performance}
Slope uses pooled memory allocation and lazy deallocations to handle
memory allocation in a peer-to-peer fashion.

RAMP however uses a Zookeeper cluster to keep track of memory allocations.
Not only relying on an external service during run-time, which possibly
operates on a slower network performs worse under heavy load, but also it
is susceptible to contention when multiple servers race to allocate memory segments.


\section{Shared memory and RDMA based systems}
Herd \cite{kalia2016designguidelines} discusses guidelines that can be
used by RDMA applications for improved performance. We used many of their
insights in our implementation.

Multiple systems have provided designs
for high performance RPC, transaction processing, or shared memory over
RDMA. These range from eRPC \cite{kalia2019datacenter}, a general purpose
RPC framework which communicates in raw packet format over unreliable
datagram, to FaRM \cite{Dragojevic2014FaRM} which uses one-sided RDMA
verbs in conjunction with busy polling to provide a remote shared memory
abstraction with support for transactions. Examples from other design
points in this spectrum include FaSST \cite{kalia2016fasst}, which uses
unreliable datagram and combines low level design techniques such as
request batching, coroutines, and QP sharing to achieve high throughput
in transaction processing, ScaleRPC \cite{ScaleRPC2019} which again uses unreliable datagram,
and Storm \cite{novakovic2019storm} which
focuses on in-memory data structures and uses
one-sided and two-sided RDMA verbs in conjunction in a hybrid fashion.

Some of these systems have partially overlapping problem statements with
Slope. They target high distributed transaction throughput while limiting
the programming model to transaction processing or RPCs.
This direction only aligns with the goals of Slope in the cases where we
need to migrate a large number of relatively small objects. This suggests
that an RPC communication model might be better suited for such an
application than the migration model. Nevertheless, these systems
can be plugged into Slope as its control plane to improve its performance
in the cases where there are relatively large number of objects being
migrated at certain points in time.

Given that the design of Slope focuses on in memory data structures, there
is a lot of potential in using specialized memory hardware. DrTM
\cite{drtm2017} builds on hardware transactional memory and RDMA
to achieve high throughput transaction processing.
Hotpot \cite{Shan2017distributed} and Octopus \cite{Lu2017rdmadistributed}
use persistent memory to build distributed shared memory.


\section{Migration and replication}
Derecho \cite{jha2019derecho} aims to provide state machine replication
for cloud applications and shares some core concepts with Slope, for
    example in how they define their system around a data flow model.
    The main idea behind their approach to handling failures can be
    incorporated in Slope to implement a form of snapshot isolation at
    migration boundaries.

ProRea \cite{ProRea2013} and Zephyr \cite{zephyr2011elmore} are live
database migration methods similar in terminology Slope which is a
data structure migration engine. However in these systems much effort goes
towards conflict resolution and handling the ``dual ownership''
time-window, whereas in Slope we avoid dual ownership of objects
altogether to support the notion of the objects being memory resident
rather than view them as entries in a storage system.

\chapter{Discussion}
\label{chap:discussionfuture}

\section{Review and possible future directions}
We point out the most favorable and biggest flaws of the design and
implementation of Slope. We also discuss possible directions for Slope
and the migratable model going forward.

\paragraph{Handling failures:}
Before dealing with failures, we need to specify what failures mean for the
    systems that will use Slope. Depending on the type of application,
    this could be a non-issue (e.g. in data pipelines, where tasks are
    usually stateless and can easily be restarted) or a deal-breaker.
    In both cases, the notion of migrating runnable objects discussed in
    \autoref{subsec:runnable} can be a starting point for approaching failures.
    For example migrating or rerunning objects can be though of as natural
    points of synchronization or taking snapshots.


\paragraph{Routing and discovery:}
It is unlikely that Slope can be used without an extra multiplexing
layer on top. In listen and serve applications, a router is required to deliver
a request to the correct partition and in the state machine model,
given the heterogeneity of the nodes, a service discovery layer is
required to help nodes find appropriate migration destinations.

Given the
performance requirements of Slope, it might make more sense to build such
a system as part of Slope than to use an external service. Not only the
performance of Slope will be bottlenecked by the external routing layer, but
also such an application may benefit from running on \emph{top} of Slope.
Unreliable broadcasts over Infiniband and an Arp-like protocol sounds suitable
if this is to be done inside Slope in a leaderless fashion.



\paragraph{Dynamic resizing of the cluster:}
This is easier to achieve than the other requirements, but is still a required
engineering effort before we are able use Slope in production environments.
The machines must coordinate to make the shared address space reusable, as we
might add or remove machines indefinitely. We must also make minor changes to
the discovery protocol and possibly extract some of it to point to point
communication between the nodes.



\paragraph{Handling updates to Slope and application code:} Another technical
debt in the design of Slope lies in defining a procedure for updating Slope and
the applications that run on top of it such that the successive versions of
application binaries can coexist in the same cluster during the update period.
At least the core
functionality of Slope has to be backwards compatible with its previous versions.
On The applications side, system designers need to come up with an internal versioning
scheme to prevent objects from migrating back to the machines running older
binaries.

A much more complicated scenario
is one in which we make updates not to the internal functionalities, but to the
data members of the migratable objects. For example one might try to start by
providing functions that can translate an object in the pre-update format to an
object
in the current format, but it is not clear how we can reference the members of the
pre-update object while we only have code defining the current version of that type.
We might be able to get around this by limiting how migratable object schemes can be
updated similar to how this is done in Google's Protocol Buffers \cite{google:protobuf}.


\paragraph{Partial migration:}
With a black box view of the objects, we miss the cases where dividing the
algorithm or data structure into self-contained parts is not natural.
Tree-based data structures where subtrees typically resemble the full data
structure are good examples. With the current model, it is not
clear how one might approach the need for distributing subtrees across the
cluster arbitrarily. This motivates a search for a more capable ownership model
from one side, and tweaking the designs of data structures to make them
migration-friendly from the other.

\paragraph{Fixed address space:}
The fixed address space requirement is currently fundamental. A possible next
step would be coming up with a framework for creating \emph{relocatable}
objects, ones that can be placed at any base address in memory. We need
to define how relocatable objects compose, because upon relocating an object,
we most likely need to recursively relocate its members too. In addition to
that, we need to enforce a limited programming model to prevent direct reliance
on virtual memory addresses, as discussed in \autoref{sec:fixedfundamental}.

Another way to mitigate this is by allowing only ``named'' accesses to
resources, similar to how objects are accessed in Python (except the code
boundary of calling into external C functions). This way, we are effectively
wrapping the application's
virtual address space in another layer of virtual addressing provided by the
language runtime.



\section{Conclusion}

We identified a family of distributed applications and hypothesized that
their limited resource access pattern might call for a specialized transport
that they can benefit from. We then came up with an object-based migration
scheme which brings locality, load balancing, and easy of programming to the
above applications all at the same time.

Based on the above requirements, we designed Slope, provided its low level
specifications, and implemented it in C++ over RDMA. We pinpointed metrics that
we deem central to the performance of such a system and designed benchmarks to
measure them in environments which resemble those of real world applications.

We pointed out strengths, weaknesses and various possible future directions for
Slope or other systems working on a similar problem. We think of Slope as one
step towards a ubiquitous solution for making high performance computing
environments where specialized computing hardware can be added and used easily.




%% This adds a line for the Bibliography in the Table of Contents.
\addcontentsline{toc}{chapter}{Bibliography}
%% *** Set the bibliography style. ***
%% (change according to your preference/requirements)
\bibliographystyle{plain}

%% *** Set the bibliography file. ***
%% ("thesis.bib" by default; change as needed)
\bibliography{thesis}

%% *** NOTE ***
%% If you don't use bibliography files, comment out the previous line
%% and use \begin{thebibliography}...\end{thebibliography}.  (In that
%% case, you should probably put the bibliography in a separate file and
%% `\include' or `\input' it here).

\end{document}

%%% Local Variables:
%%% mode: latex
%%% TeX-master: t
%%% End:
