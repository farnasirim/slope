\chapter{Evaluation}
\label{chap:evaluation}

We go over a few use cases of Slope in real-world systems and present both
micro-benchmarks and full benchmarks for select applications. We also discuss
the metrics that the applications using Slope can measure to get a sense of
how much Slope is impacting their performance.

To measure times and calculate performance metrics globally in the cluster,
we synchronize the start time from one machine to all other machines in the
cluster, by estimating the round trip time between them, which we do by
calculating the median among multiple round trip time calculations. Round
trip times are approximately $5 {\mu}{s}$, a pessimistic upper bound for the
error in time synchronization.

To make sure our graphs are accurate, we run each configuration at least 5 times
and average out the results, and in some cases, drop the min and max values to
eliminate the outlying points. That means each point in each of our graphs is
the resulting value from the above calculation on multiple runs with the same
configuration. Even without eliminating the outlying points, the standard errors
of our measurements are negligible (i.e. multiple orders of magnitude smaller)
compared to the reported values, unless otherwise stated.

Looking at each sub-system in Slope, one can define certain metrics that
reflect if that sub-system is working efficiently. For example we measure
the elapsed time until the prefill operation completes. We also measure
other metrics which more directly impact application performance, such as
end to end migration delay or time during which the object is unusable at
either end.

\subsection{Migration friendliness of data structures}
Based on how the migration process and specifically how the prefill operation
works, objects which use their internal memory in a ``fixed'' manner, that is
without doing much memory allocation/deallocation, are very good candidates
for migration, since they can function seamlessly throughout the prefill phase,
by only inducing dirty page overhead.

Examples of these objects include bloom filters, where we have a fixed
array of bits the size of which always stays the same.
Similarly hash tables which use open addressing techniques such as cuckoo
hashing scheme for their collision resolution are also good candidates for the
same reason.
Apart from these objects which make the best-case scenario for Slope, we also
discuss more generic objects whose allocation/deallocation patterns are not
ideal.

\section{Case study: core metrics and STL objects}
\subsection{Migrating a vector with clean pages}
\label{sec:cleanvec}

\begin{figure}[tp]
    \begin{center}
        %% Creator: Matplotlib, PGF backend
%%
%% To include the figure in your LaTeX document, write
%%   \input{<filename>.pgf}
%%
%% Make sure the required packages are loaded in your preamble
%%   \usepackage{pgf}
%%
%% and, on pdftex
%%   \usepackage[utf8]{inputenc}\DeclareUnicodeCharacter{2212}{-}
%%
%% or, on luatex and xetex
%%   \usepackage{unicode-math}
%%
%% Figures using additional raster images can only be included by \input if
%% they are in the same directory as the main LaTeX file. For loading figures
%% from other directories you can use the `import` package
%%   \usepackage{import}
%%
%% and then include the figures with
%%   \import{<path to file>}{<filename>.pgf}
%%
%% Matplotlib used the following preamble
%%
\begingroup%
\makeatletter%
\begin{pgfpicture}%
\pgfpathrectangle{\pgfpointorigin}{\pgfqpoint{6.251220in}{3.516311in}}%
\pgfusepath{use as bounding box, clip}%
\begin{pgfscope}%
\pgfsetbuttcap%
\pgfsetmiterjoin%
\definecolor{currentfill}{rgb}{1.000000,1.000000,1.000000}%
\pgfsetfillcolor{currentfill}%
\pgfsetlinewidth{0.000000pt}%
\definecolor{currentstroke}{rgb}{1.000000,1.000000,1.000000}%
\pgfsetstrokecolor{currentstroke}%
\pgfsetdash{}{0pt}%
\pgfpathmoveto{\pgfqpoint{0.000000in}{0.000000in}}%
\pgfpathlineto{\pgfqpoint{6.251220in}{0.000000in}}%
\pgfpathlineto{\pgfqpoint{6.251220in}{3.516311in}}%
\pgfpathlineto{\pgfqpoint{0.000000in}{3.516311in}}%
\pgfpathclose%
\pgfusepath{fill}%
\end{pgfscope}%
\begin{pgfscope}%
\pgfsetbuttcap%
\pgfsetmiterjoin%
\definecolor{currentfill}{rgb}{1.000000,1.000000,1.000000}%
\pgfsetfillcolor{currentfill}%
\pgfsetlinewidth{0.000000pt}%
\definecolor{currentstroke}{rgb}{0.000000,0.000000,0.000000}%
\pgfsetstrokecolor{currentstroke}%
\pgfsetstrokeopacity{0.000000}%
\pgfsetdash{}{0pt}%
\pgfpathmoveto{\pgfqpoint{0.781402in}{0.386794in}}%
\pgfpathlineto{\pgfqpoint{5.626098in}{0.386794in}}%
\pgfpathlineto{\pgfqpoint{5.626098in}{3.094354in}}%
\pgfpathlineto{\pgfqpoint{0.781402in}{3.094354in}}%
\pgfpathclose%
\pgfusepath{fill}%
\end{pgfscope}%
\begin{pgfscope}%
\pgfsetbuttcap%
\pgfsetroundjoin%
\definecolor{currentfill}{rgb}{0.000000,0.000000,0.000000}%
\pgfsetfillcolor{currentfill}%
\pgfsetlinewidth{0.803000pt}%
\definecolor{currentstroke}{rgb}{0.000000,0.000000,0.000000}%
\pgfsetstrokecolor{currentstroke}%
\pgfsetdash{}{0pt}%
\pgfsys@defobject{currentmarker}{\pgfqpoint{0.000000in}{-0.048611in}}{\pgfqpoint{0.000000in}{0.000000in}}{%
\pgfpathmoveto{\pgfqpoint{0.000000in}{0.000000in}}%
\pgfpathlineto{\pgfqpoint{0.000000in}{-0.048611in}}%
\pgfusepath{stroke,fill}%
}%
\begin{pgfscope}%
\pgfsys@transformshift{1.001616in}{0.386794in}%
\pgfsys@useobject{currentmarker}{}%
\end{pgfscope}%
\end{pgfscope}%
\begin{pgfscope}%
\definecolor{textcolor}{rgb}{0.000000,0.000000,0.000000}%
\pgfsetstrokecolor{textcolor}%
\pgfsetfillcolor{textcolor}%
\pgftext[x=1.001616in,y=0.289572in,,top]{\color{textcolor}\rmfamily\fontsize{10.000000}{12.000000}\selectfont \(\displaystyle {1}\)}%
\end{pgfscope}%
\begin{pgfscope}%
\pgfsetbuttcap%
\pgfsetroundjoin%
\definecolor{currentfill}{rgb}{0.000000,0.000000,0.000000}%
\pgfsetfillcolor{currentfill}%
\pgfsetlinewidth{0.803000pt}%
\definecolor{currentstroke}{rgb}{0.000000,0.000000,0.000000}%
\pgfsetstrokecolor{currentstroke}%
\pgfsetdash{}{0pt}%
\pgfsys@defobject{currentmarker}{\pgfqpoint{0.000000in}{-0.048611in}}{\pgfqpoint{0.000000in}{0.000000in}}{%
\pgfpathmoveto{\pgfqpoint{0.000000in}{0.000000in}}%
\pgfpathlineto{\pgfqpoint{0.000000in}{-0.048611in}}%
\pgfusepath{stroke,fill}%
}%
\begin{pgfscope}%
\pgfsys@transformshift{1.882434in}{0.386794in}%
\pgfsys@useobject{currentmarker}{}%
\end{pgfscope}%
\end{pgfscope}%
\begin{pgfscope}%
\definecolor{textcolor}{rgb}{0.000000,0.000000,0.000000}%
\pgfsetstrokecolor{textcolor}%
\pgfsetfillcolor{textcolor}%
\pgftext[x=1.882434in,y=0.289572in,,top]{\color{textcolor}\rmfamily\fontsize{10.000000}{12.000000}\selectfont \(\displaystyle {20000}\)}%
\end{pgfscope}%
\begin{pgfscope}%
\pgfsetbuttcap%
\pgfsetroundjoin%
\definecolor{currentfill}{rgb}{0.000000,0.000000,0.000000}%
\pgfsetfillcolor{currentfill}%
\pgfsetlinewidth{0.803000pt}%
\definecolor{currentstroke}{rgb}{0.000000,0.000000,0.000000}%
\pgfsetstrokecolor{currentstroke}%
\pgfsetdash{}{0pt}%
\pgfsys@defobject{currentmarker}{\pgfqpoint{0.000000in}{-0.048611in}}{\pgfqpoint{0.000000in}{0.000000in}}{%
\pgfpathmoveto{\pgfqpoint{0.000000in}{0.000000in}}%
\pgfpathlineto{\pgfqpoint{0.000000in}{-0.048611in}}%
\pgfusepath{stroke,fill}%
}%
\begin{pgfscope}%
\pgfsys@transformshift{2.763297in}{0.386794in}%
\pgfsys@useobject{currentmarker}{}%
\end{pgfscope}%
\end{pgfscope}%
\begin{pgfscope}%
\definecolor{textcolor}{rgb}{0.000000,0.000000,0.000000}%
\pgfsetstrokecolor{textcolor}%
\pgfsetfillcolor{textcolor}%
\pgftext[x=2.763297in,y=0.289572in,,top]{\color{textcolor}\rmfamily\fontsize{10.000000}{12.000000}\selectfont \(\displaystyle {40000}\)}%
\end{pgfscope}%
\begin{pgfscope}%
\pgfsetbuttcap%
\pgfsetroundjoin%
\definecolor{currentfill}{rgb}{0.000000,0.000000,0.000000}%
\pgfsetfillcolor{currentfill}%
\pgfsetlinewidth{0.803000pt}%
\definecolor{currentstroke}{rgb}{0.000000,0.000000,0.000000}%
\pgfsetstrokecolor{currentstroke}%
\pgfsetdash{}{0pt}%
\pgfsys@defobject{currentmarker}{\pgfqpoint{0.000000in}{-0.048611in}}{\pgfqpoint{0.000000in}{0.000000in}}{%
\pgfpathmoveto{\pgfqpoint{0.000000in}{0.000000in}}%
\pgfpathlineto{\pgfqpoint{0.000000in}{-0.048611in}}%
\pgfusepath{stroke,fill}%
}%
\begin{pgfscope}%
\pgfsys@transformshift{3.644159in}{0.386794in}%
\pgfsys@useobject{currentmarker}{}%
\end{pgfscope}%
\end{pgfscope}%
\begin{pgfscope}%
\definecolor{textcolor}{rgb}{0.000000,0.000000,0.000000}%
\pgfsetstrokecolor{textcolor}%
\pgfsetfillcolor{textcolor}%
\pgftext[x=3.644159in,y=0.289572in,,top]{\color{textcolor}\rmfamily\fontsize{10.000000}{12.000000}\selectfont \(\displaystyle {60000}\)}%
\end{pgfscope}%
\begin{pgfscope}%
\pgfsetbuttcap%
\pgfsetroundjoin%
\definecolor{currentfill}{rgb}{0.000000,0.000000,0.000000}%
\pgfsetfillcolor{currentfill}%
\pgfsetlinewidth{0.803000pt}%
\definecolor{currentstroke}{rgb}{0.000000,0.000000,0.000000}%
\pgfsetstrokecolor{currentstroke}%
\pgfsetdash{}{0pt}%
\pgfsys@defobject{currentmarker}{\pgfqpoint{0.000000in}{-0.048611in}}{\pgfqpoint{0.000000in}{0.000000in}}{%
\pgfpathmoveto{\pgfqpoint{0.000000in}{0.000000in}}%
\pgfpathlineto{\pgfqpoint{0.000000in}{-0.048611in}}%
\pgfusepath{stroke,fill}%
}%
\begin{pgfscope}%
\pgfsys@transformshift{4.525022in}{0.386794in}%
\pgfsys@useobject{currentmarker}{}%
\end{pgfscope}%
\end{pgfscope}%
\begin{pgfscope}%
\definecolor{textcolor}{rgb}{0.000000,0.000000,0.000000}%
\pgfsetstrokecolor{textcolor}%
\pgfsetfillcolor{textcolor}%
\pgftext[x=4.525022in,y=0.289572in,,top]{\color{textcolor}\rmfamily\fontsize{10.000000}{12.000000}\selectfont \(\displaystyle {80000}\)}%
\end{pgfscope}%
\begin{pgfscope}%
\pgfsetbuttcap%
\pgfsetroundjoin%
\definecolor{currentfill}{rgb}{0.000000,0.000000,0.000000}%
\pgfsetfillcolor{currentfill}%
\pgfsetlinewidth{0.803000pt}%
\definecolor{currentstroke}{rgb}{0.000000,0.000000,0.000000}%
\pgfsetstrokecolor{currentstroke}%
\pgfsetdash{}{0pt}%
\pgfsys@defobject{currentmarker}{\pgfqpoint{0.000000in}{-0.048611in}}{\pgfqpoint{0.000000in}{0.000000in}}{%
\pgfpathmoveto{\pgfqpoint{0.000000in}{0.000000in}}%
\pgfpathlineto{\pgfqpoint{0.000000in}{-0.048611in}}%
\pgfusepath{stroke,fill}%
}%
\begin{pgfscope}%
\pgfsys@transformshift{5.405885in}{0.386794in}%
\pgfsys@useobject{currentmarker}{}%
\end{pgfscope}%
\end{pgfscope}%
\begin{pgfscope}%
\definecolor{textcolor}{rgb}{0.000000,0.000000,0.000000}%
\pgfsetstrokecolor{textcolor}%
\pgfsetfillcolor{textcolor}%
\pgftext[x=5.405885in,y=0.289572in,,top]{\color{textcolor}\rmfamily\fontsize{10.000000}{12.000000}\selectfont \(\displaystyle {100000}\)}%
\end{pgfscope}%
\begin{pgfscope}%
\definecolor{textcolor}{rgb}{0.000000,0.000000,0.000000}%
\pgfsetstrokecolor{textcolor}%
\pgfsetfillcolor{textcolor}%
\pgftext[x=3.203750in,y=0.110560in,,top]{\color{textcolor}\rmfamily\fontsize{10.000000}{12.000000}\selectfont Number of 4KB pages}%
\end{pgfscope}%
\begin{pgfscope}%
\pgfsetbuttcap%
\pgfsetroundjoin%
\definecolor{currentfill}{rgb}{0.000000,0.000000,0.000000}%
\pgfsetfillcolor{currentfill}%
\pgfsetlinewidth{0.803000pt}%
\definecolor{currentstroke}{rgb}{0.000000,0.000000,0.000000}%
\pgfsetstrokecolor{currentstroke}%
\pgfsetdash{}{0pt}%
\pgfsys@defobject{currentmarker}{\pgfqpoint{-0.048611in}{0.000000in}}{\pgfqpoint{-0.000000in}{0.000000in}}{%
\pgfpathmoveto{\pgfqpoint{-0.000000in}{0.000000in}}%
\pgfpathlineto{\pgfqpoint{-0.048611in}{0.000000in}}%
\pgfusepath{stroke,fill}%
}%
\begin{pgfscope}%
\pgfsys@transformshift{0.781402in}{0.509865in}%
\pgfsys@useobject{currentmarker}{}%
\end{pgfscope}%
\end{pgfscope}%
\begin{pgfscope}%
\definecolor{textcolor}{rgb}{0.000000,0.000000,0.000000}%
\pgfsetstrokecolor{textcolor}%
\pgfsetfillcolor{textcolor}%
\pgftext[x=0.614736in, y=0.461639in, left, base]{\color{textcolor}\rmfamily\fontsize{10.000000}{12.000000}\selectfont \(\displaystyle {0}\)}%
\end{pgfscope}%
\begin{pgfscope}%
\pgfsetbuttcap%
\pgfsetroundjoin%
\definecolor{currentfill}{rgb}{0.000000,0.000000,0.000000}%
\pgfsetfillcolor{currentfill}%
\pgfsetlinewidth{0.803000pt}%
\definecolor{currentstroke}{rgb}{0.000000,0.000000,0.000000}%
\pgfsetstrokecolor{currentstroke}%
\pgfsetdash{}{0pt}%
\pgfsys@defobject{currentmarker}{\pgfqpoint{-0.048611in}{0.000000in}}{\pgfqpoint{-0.000000in}{0.000000in}}{%
\pgfpathmoveto{\pgfqpoint{-0.000000in}{0.000000in}}%
\pgfpathlineto{\pgfqpoint{-0.048611in}{0.000000in}}%
\pgfusepath{stroke,fill}%
}%
\begin{pgfscope}%
\pgfsys@transformshift{0.781402in}{0.831595in}%
\pgfsys@useobject{currentmarker}{}%
\end{pgfscope}%
\end{pgfscope}%
\begin{pgfscope}%
\definecolor{textcolor}{rgb}{0.000000,0.000000,0.000000}%
\pgfsetstrokecolor{textcolor}%
\pgfsetfillcolor{textcolor}%
\pgftext[x=0.475846in, y=0.783370in, left, base]{\color{textcolor}\rmfamily\fontsize{10.000000}{12.000000}\selectfont \(\displaystyle {500}\)}%
\end{pgfscope}%
\begin{pgfscope}%
\pgfsetbuttcap%
\pgfsetroundjoin%
\definecolor{currentfill}{rgb}{0.000000,0.000000,0.000000}%
\pgfsetfillcolor{currentfill}%
\pgfsetlinewidth{0.803000pt}%
\definecolor{currentstroke}{rgb}{0.000000,0.000000,0.000000}%
\pgfsetstrokecolor{currentstroke}%
\pgfsetdash{}{0pt}%
\pgfsys@defobject{currentmarker}{\pgfqpoint{-0.048611in}{0.000000in}}{\pgfqpoint{-0.000000in}{0.000000in}}{%
\pgfpathmoveto{\pgfqpoint{-0.000000in}{0.000000in}}%
\pgfpathlineto{\pgfqpoint{-0.048611in}{0.000000in}}%
\pgfusepath{stroke,fill}%
}%
\begin{pgfscope}%
\pgfsys@transformshift{0.781402in}{1.153326in}%
\pgfsys@useobject{currentmarker}{}%
\end{pgfscope}%
\end{pgfscope}%
\begin{pgfscope}%
\definecolor{textcolor}{rgb}{0.000000,0.000000,0.000000}%
\pgfsetstrokecolor{textcolor}%
\pgfsetfillcolor{textcolor}%
\pgftext[x=0.406402in, y=1.105100in, left, base]{\color{textcolor}\rmfamily\fontsize{10.000000}{12.000000}\selectfont \(\displaystyle {1000}\)}%
\end{pgfscope}%
\begin{pgfscope}%
\pgfsetbuttcap%
\pgfsetroundjoin%
\definecolor{currentfill}{rgb}{0.000000,0.000000,0.000000}%
\pgfsetfillcolor{currentfill}%
\pgfsetlinewidth{0.803000pt}%
\definecolor{currentstroke}{rgb}{0.000000,0.000000,0.000000}%
\pgfsetstrokecolor{currentstroke}%
\pgfsetdash{}{0pt}%
\pgfsys@defobject{currentmarker}{\pgfqpoint{-0.048611in}{0.000000in}}{\pgfqpoint{-0.000000in}{0.000000in}}{%
\pgfpathmoveto{\pgfqpoint{-0.000000in}{0.000000in}}%
\pgfpathlineto{\pgfqpoint{-0.048611in}{0.000000in}}%
\pgfusepath{stroke,fill}%
}%
\begin{pgfscope}%
\pgfsys@transformshift{0.781402in}{1.475056in}%
\pgfsys@useobject{currentmarker}{}%
\end{pgfscope}%
\end{pgfscope}%
\begin{pgfscope}%
\definecolor{textcolor}{rgb}{0.000000,0.000000,0.000000}%
\pgfsetstrokecolor{textcolor}%
\pgfsetfillcolor{textcolor}%
\pgftext[x=0.406402in, y=1.426831in, left, base]{\color{textcolor}\rmfamily\fontsize{10.000000}{12.000000}\selectfont \(\displaystyle {1500}\)}%
\end{pgfscope}%
\begin{pgfscope}%
\pgfsetbuttcap%
\pgfsetroundjoin%
\definecolor{currentfill}{rgb}{0.000000,0.000000,0.000000}%
\pgfsetfillcolor{currentfill}%
\pgfsetlinewidth{0.803000pt}%
\definecolor{currentstroke}{rgb}{0.000000,0.000000,0.000000}%
\pgfsetstrokecolor{currentstroke}%
\pgfsetdash{}{0pt}%
\pgfsys@defobject{currentmarker}{\pgfqpoint{-0.048611in}{0.000000in}}{\pgfqpoint{-0.000000in}{0.000000in}}{%
\pgfpathmoveto{\pgfqpoint{-0.000000in}{0.000000in}}%
\pgfpathlineto{\pgfqpoint{-0.048611in}{0.000000in}}%
\pgfusepath{stroke,fill}%
}%
\begin{pgfscope}%
\pgfsys@transformshift{0.781402in}{1.796787in}%
\pgfsys@useobject{currentmarker}{}%
\end{pgfscope}%
\end{pgfscope}%
\begin{pgfscope}%
\definecolor{textcolor}{rgb}{0.000000,0.000000,0.000000}%
\pgfsetstrokecolor{textcolor}%
\pgfsetfillcolor{textcolor}%
\pgftext[x=0.406402in, y=1.748562in, left, base]{\color{textcolor}\rmfamily\fontsize{10.000000}{12.000000}\selectfont \(\displaystyle {2000}\)}%
\end{pgfscope}%
\begin{pgfscope}%
\pgfsetbuttcap%
\pgfsetroundjoin%
\definecolor{currentfill}{rgb}{0.000000,0.000000,0.000000}%
\pgfsetfillcolor{currentfill}%
\pgfsetlinewidth{0.803000pt}%
\definecolor{currentstroke}{rgb}{0.000000,0.000000,0.000000}%
\pgfsetstrokecolor{currentstroke}%
\pgfsetdash{}{0pt}%
\pgfsys@defobject{currentmarker}{\pgfqpoint{-0.048611in}{0.000000in}}{\pgfqpoint{-0.000000in}{0.000000in}}{%
\pgfpathmoveto{\pgfqpoint{-0.000000in}{0.000000in}}%
\pgfpathlineto{\pgfqpoint{-0.048611in}{0.000000in}}%
\pgfusepath{stroke,fill}%
}%
\begin{pgfscope}%
\pgfsys@transformshift{0.781402in}{2.118517in}%
\pgfsys@useobject{currentmarker}{}%
\end{pgfscope}%
\end{pgfscope}%
\begin{pgfscope}%
\definecolor{textcolor}{rgb}{0.000000,0.000000,0.000000}%
\pgfsetstrokecolor{textcolor}%
\pgfsetfillcolor{textcolor}%
\pgftext[x=0.406402in, y=2.070292in, left, base]{\color{textcolor}\rmfamily\fontsize{10.000000}{12.000000}\selectfont \(\displaystyle {2500}\)}%
\end{pgfscope}%
\begin{pgfscope}%
\pgfsetbuttcap%
\pgfsetroundjoin%
\definecolor{currentfill}{rgb}{0.000000,0.000000,0.000000}%
\pgfsetfillcolor{currentfill}%
\pgfsetlinewidth{0.803000pt}%
\definecolor{currentstroke}{rgb}{0.000000,0.000000,0.000000}%
\pgfsetstrokecolor{currentstroke}%
\pgfsetdash{}{0pt}%
\pgfsys@defobject{currentmarker}{\pgfqpoint{-0.048611in}{0.000000in}}{\pgfqpoint{-0.000000in}{0.000000in}}{%
\pgfpathmoveto{\pgfqpoint{-0.000000in}{0.000000in}}%
\pgfpathlineto{\pgfqpoint{-0.048611in}{0.000000in}}%
\pgfusepath{stroke,fill}%
}%
\begin{pgfscope}%
\pgfsys@transformshift{0.781402in}{2.440248in}%
\pgfsys@useobject{currentmarker}{}%
\end{pgfscope}%
\end{pgfscope}%
\begin{pgfscope}%
\definecolor{textcolor}{rgb}{0.000000,0.000000,0.000000}%
\pgfsetstrokecolor{textcolor}%
\pgfsetfillcolor{textcolor}%
\pgftext[x=0.406402in, y=2.392023in, left, base]{\color{textcolor}\rmfamily\fontsize{10.000000}{12.000000}\selectfont \(\displaystyle {3000}\)}%
\end{pgfscope}%
\begin{pgfscope}%
\pgfsetbuttcap%
\pgfsetroundjoin%
\definecolor{currentfill}{rgb}{0.000000,0.000000,0.000000}%
\pgfsetfillcolor{currentfill}%
\pgfsetlinewidth{0.803000pt}%
\definecolor{currentstroke}{rgb}{0.000000,0.000000,0.000000}%
\pgfsetstrokecolor{currentstroke}%
\pgfsetdash{}{0pt}%
\pgfsys@defobject{currentmarker}{\pgfqpoint{-0.048611in}{0.000000in}}{\pgfqpoint{-0.000000in}{0.000000in}}{%
\pgfpathmoveto{\pgfqpoint{-0.000000in}{0.000000in}}%
\pgfpathlineto{\pgfqpoint{-0.048611in}{0.000000in}}%
\pgfusepath{stroke,fill}%
}%
\begin{pgfscope}%
\pgfsys@transformshift{0.781402in}{2.761979in}%
\pgfsys@useobject{currentmarker}{}%
\end{pgfscope}%
\end{pgfscope}%
\begin{pgfscope}%
\definecolor{textcolor}{rgb}{0.000000,0.000000,0.000000}%
\pgfsetstrokecolor{textcolor}%
\pgfsetfillcolor{textcolor}%
\pgftext[x=0.406402in, y=2.713753in, left, base]{\color{textcolor}\rmfamily\fontsize{10.000000}{12.000000}\selectfont \(\displaystyle {3500}\)}%
\end{pgfscope}%
\begin{pgfscope}%
\pgfsetbuttcap%
\pgfsetroundjoin%
\definecolor{currentfill}{rgb}{0.000000,0.000000,0.000000}%
\pgfsetfillcolor{currentfill}%
\pgfsetlinewidth{0.803000pt}%
\definecolor{currentstroke}{rgb}{0.000000,0.000000,0.000000}%
\pgfsetstrokecolor{currentstroke}%
\pgfsetdash{}{0pt}%
\pgfsys@defobject{currentmarker}{\pgfqpoint{-0.048611in}{0.000000in}}{\pgfqpoint{-0.000000in}{0.000000in}}{%
\pgfpathmoveto{\pgfqpoint{-0.000000in}{0.000000in}}%
\pgfpathlineto{\pgfqpoint{-0.048611in}{0.000000in}}%
\pgfusepath{stroke,fill}%
}%
\begin{pgfscope}%
\pgfsys@transformshift{0.781402in}{3.083709in}%
\pgfsys@useobject{currentmarker}{}%
\end{pgfscope}%
\end{pgfscope}%
\begin{pgfscope}%
\definecolor{textcolor}{rgb}{0.000000,0.000000,0.000000}%
\pgfsetstrokecolor{textcolor}%
\pgfsetfillcolor{textcolor}%
\pgftext[x=0.406402in, y=3.035484in, left, base]{\color{textcolor}\rmfamily\fontsize{10.000000}{12.000000}\selectfont \(\displaystyle {4000}\)}%
\end{pgfscope}%
\begin{pgfscope}%
\definecolor{textcolor}{rgb}{0.000000,0.000000,0.000000}%
\pgfsetstrokecolor{textcolor}%
\pgfsetfillcolor{textcolor}%
\pgftext[x=0.350846in,y=1.740574in,,bottom,rotate=90.000000]{\color{textcolor}\rmfamily\fontsize{10.000000}{12.000000}\selectfont Elapsed time (milliseconds)}%
\end{pgfscope}%
\begin{pgfscope}%
\pgfpathrectangle{\pgfqpoint{0.781402in}{0.386794in}}{\pgfqpoint{4.844695in}{2.707560in}}%
\pgfusepath{clip}%
\pgfsetrectcap%
\pgfsetroundjoin%
\pgfsetlinewidth{1.505625pt}%
\definecolor{currentstroke}{rgb}{0.121569,0.466667,0.705882}%
\pgfsetstrokecolor{currentstroke}%
\pgfsetdash{}{0pt}%
\pgfpathmoveto{\pgfqpoint{1.001616in}{0.511396in}}%
\pgfpathlineto{\pgfqpoint{1.442003in}{0.708084in}}%
\pgfpathlineto{\pgfqpoint{1.882434in}{0.908626in}}%
\pgfpathlineto{\pgfqpoint{2.322866in}{1.123131in}}%
\pgfpathlineto{\pgfqpoint{2.763297in}{1.320690in}}%
\pgfpathlineto{\pgfqpoint{3.203728in}{1.536318in}}%
\pgfpathlineto{\pgfqpoint{3.644159in}{1.748271in}}%
\pgfpathlineto{\pgfqpoint{4.084591in}{1.958908in}}%
\pgfpathlineto{\pgfqpoint{4.525022in}{2.163276in}}%
\pgfpathlineto{\pgfqpoint{4.965453in}{2.378237in}}%
\pgfpathlineto{\pgfqpoint{5.405885in}{2.549731in}}%
\pgfusepath{stroke}%
\end{pgfscope}%
\begin{pgfscope}%
\pgfpathrectangle{\pgfqpoint{0.781402in}{0.386794in}}{\pgfqpoint{4.844695in}{2.707560in}}%
\pgfusepath{clip}%
\pgfsetbuttcap%
\pgfsetroundjoin%
\definecolor{currentfill}{rgb}{0.121569,0.466667,0.705882}%
\pgfsetfillcolor{currentfill}%
\pgfsetlinewidth{1.003750pt}%
\definecolor{currentstroke}{rgb}{0.121569,0.466667,0.705882}%
\pgfsetstrokecolor{currentstroke}%
\pgfsetdash{}{0pt}%
\pgfsys@defobject{currentmarker}{\pgfqpoint{-0.041667in}{-0.041667in}}{\pgfqpoint{0.041667in}{0.041667in}}{%
\pgfpathmoveto{\pgfqpoint{0.000000in}{-0.041667in}}%
\pgfpathcurveto{\pgfqpoint{0.011050in}{-0.041667in}}{\pgfqpoint{0.021649in}{-0.037276in}}{\pgfqpoint{0.029463in}{-0.029463in}}%
\pgfpathcurveto{\pgfqpoint{0.037276in}{-0.021649in}}{\pgfqpoint{0.041667in}{-0.011050in}}{\pgfqpoint{0.041667in}{0.000000in}}%
\pgfpathcurveto{\pgfqpoint{0.041667in}{0.011050in}}{\pgfqpoint{0.037276in}{0.021649in}}{\pgfqpoint{0.029463in}{0.029463in}}%
\pgfpathcurveto{\pgfqpoint{0.021649in}{0.037276in}}{\pgfqpoint{0.011050in}{0.041667in}}{\pgfqpoint{0.000000in}{0.041667in}}%
\pgfpathcurveto{\pgfqpoint{-0.011050in}{0.041667in}}{\pgfqpoint{-0.021649in}{0.037276in}}{\pgfqpoint{-0.029463in}{0.029463in}}%
\pgfpathcurveto{\pgfqpoint{-0.037276in}{0.021649in}}{\pgfqpoint{-0.041667in}{0.011050in}}{\pgfqpoint{-0.041667in}{0.000000in}}%
\pgfpathcurveto{\pgfqpoint{-0.041667in}{-0.011050in}}{\pgfqpoint{-0.037276in}{-0.021649in}}{\pgfqpoint{-0.029463in}{-0.029463in}}%
\pgfpathcurveto{\pgfqpoint{-0.021649in}{-0.037276in}}{\pgfqpoint{-0.011050in}{-0.041667in}}{\pgfqpoint{0.000000in}{-0.041667in}}%
\pgfpathclose%
\pgfusepath{stroke,fill}%
}%
\begin{pgfscope}%
\pgfsys@transformshift{1.001616in}{0.511396in}%
\pgfsys@useobject{currentmarker}{}%
\end{pgfscope}%
\begin{pgfscope}%
\pgfsys@transformshift{1.442003in}{0.708084in}%
\pgfsys@useobject{currentmarker}{}%
\end{pgfscope}%
\begin{pgfscope}%
\pgfsys@transformshift{1.882434in}{0.908626in}%
\pgfsys@useobject{currentmarker}{}%
\end{pgfscope}%
\begin{pgfscope}%
\pgfsys@transformshift{2.322866in}{1.123131in}%
\pgfsys@useobject{currentmarker}{}%
\end{pgfscope}%
\begin{pgfscope}%
\pgfsys@transformshift{2.763297in}{1.320690in}%
\pgfsys@useobject{currentmarker}{}%
\end{pgfscope}%
\begin{pgfscope}%
\pgfsys@transformshift{3.203728in}{1.536318in}%
\pgfsys@useobject{currentmarker}{}%
\end{pgfscope}%
\begin{pgfscope}%
\pgfsys@transformshift{3.644159in}{1.748271in}%
\pgfsys@useobject{currentmarker}{}%
\end{pgfscope}%
\begin{pgfscope}%
\pgfsys@transformshift{4.084591in}{1.958908in}%
\pgfsys@useobject{currentmarker}{}%
\end{pgfscope}%
\begin{pgfscope}%
\pgfsys@transformshift{4.525022in}{2.163276in}%
\pgfsys@useobject{currentmarker}{}%
\end{pgfscope}%
\begin{pgfscope}%
\pgfsys@transformshift{4.965453in}{2.378237in}%
\pgfsys@useobject{currentmarker}{}%
\end{pgfscope}%
\begin{pgfscope}%
\pgfsys@transformshift{5.405885in}{2.549731in}%
\pgfsys@useobject{currentmarker}{}%
\end{pgfscope}%
\end{pgfscope}%
\begin{pgfscope}%
\pgfpathrectangle{\pgfqpoint{0.781402in}{0.386794in}}{\pgfqpoint{4.844695in}{2.707560in}}%
\pgfusepath{clip}%
\pgfsetrectcap%
\pgfsetroundjoin%
\pgfsetlinewidth{1.505625pt}%
\definecolor{currentstroke}{rgb}{1.000000,0.498039,0.054902}%
\pgfsetstrokecolor{currentstroke}%
\pgfsetdash{}{0pt}%
\pgfpathmoveto{\pgfqpoint{1.001616in}{0.509865in}}%
\pgfpathlineto{\pgfqpoint{1.442003in}{0.509866in}}%
\pgfpathlineto{\pgfqpoint{1.882434in}{0.509866in}}%
\pgfpathlineto{\pgfqpoint{2.322866in}{0.509944in}}%
\pgfpathlineto{\pgfqpoint{2.763297in}{0.509925in}}%
\pgfpathlineto{\pgfqpoint{3.203728in}{0.509935in}}%
\pgfpathlineto{\pgfqpoint{3.644159in}{0.509920in}}%
\pgfpathlineto{\pgfqpoint{4.084591in}{0.509966in}}%
\pgfpathlineto{\pgfqpoint{4.525022in}{0.510025in}}%
\pgfpathlineto{\pgfqpoint{4.965453in}{0.509958in}}%
\pgfpathlineto{\pgfqpoint{5.405885in}{0.510074in}}%
\pgfusepath{stroke}%
\end{pgfscope}%
\begin{pgfscope}%
\pgfpathrectangle{\pgfqpoint{0.781402in}{0.386794in}}{\pgfqpoint{4.844695in}{2.707560in}}%
\pgfusepath{clip}%
\pgfsetbuttcap%
\pgfsetroundjoin%
\definecolor{currentfill}{rgb}{1.000000,0.498039,0.054902}%
\pgfsetfillcolor{currentfill}%
\pgfsetlinewidth{1.003750pt}%
\definecolor{currentstroke}{rgb}{1.000000,0.498039,0.054902}%
\pgfsetstrokecolor{currentstroke}%
\pgfsetdash{}{0pt}%
\pgfsys@defobject{currentmarker}{\pgfqpoint{-0.041667in}{-0.041667in}}{\pgfqpoint{0.041667in}{0.041667in}}{%
\pgfpathmoveto{\pgfqpoint{0.000000in}{-0.041667in}}%
\pgfpathcurveto{\pgfqpoint{0.011050in}{-0.041667in}}{\pgfqpoint{0.021649in}{-0.037276in}}{\pgfqpoint{0.029463in}{-0.029463in}}%
\pgfpathcurveto{\pgfqpoint{0.037276in}{-0.021649in}}{\pgfqpoint{0.041667in}{-0.011050in}}{\pgfqpoint{0.041667in}{0.000000in}}%
\pgfpathcurveto{\pgfqpoint{0.041667in}{0.011050in}}{\pgfqpoint{0.037276in}{0.021649in}}{\pgfqpoint{0.029463in}{0.029463in}}%
\pgfpathcurveto{\pgfqpoint{0.021649in}{0.037276in}}{\pgfqpoint{0.011050in}{0.041667in}}{\pgfqpoint{0.000000in}{0.041667in}}%
\pgfpathcurveto{\pgfqpoint{-0.011050in}{0.041667in}}{\pgfqpoint{-0.021649in}{0.037276in}}{\pgfqpoint{-0.029463in}{0.029463in}}%
\pgfpathcurveto{\pgfqpoint{-0.037276in}{0.021649in}}{\pgfqpoint{-0.041667in}{0.011050in}}{\pgfqpoint{-0.041667in}{0.000000in}}%
\pgfpathcurveto{\pgfqpoint{-0.041667in}{-0.011050in}}{\pgfqpoint{-0.037276in}{-0.021649in}}{\pgfqpoint{-0.029463in}{-0.029463in}}%
\pgfpathcurveto{\pgfqpoint{-0.021649in}{-0.037276in}}{\pgfqpoint{-0.011050in}{-0.041667in}}{\pgfqpoint{0.000000in}{-0.041667in}}%
\pgfpathclose%
\pgfusepath{stroke,fill}%
}%
\begin{pgfscope}%
\pgfsys@transformshift{1.001616in}{0.509865in}%
\pgfsys@useobject{currentmarker}{}%
\end{pgfscope}%
\begin{pgfscope}%
\pgfsys@transformshift{1.442003in}{0.509866in}%
\pgfsys@useobject{currentmarker}{}%
\end{pgfscope}%
\begin{pgfscope}%
\pgfsys@transformshift{1.882434in}{0.509866in}%
\pgfsys@useobject{currentmarker}{}%
\end{pgfscope}%
\begin{pgfscope}%
\pgfsys@transformshift{2.322866in}{0.509944in}%
\pgfsys@useobject{currentmarker}{}%
\end{pgfscope}%
\begin{pgfscope}%
\pgfsys@transformshift{2.763297in}{0.509925in}%
\pgfsys@useobject{currentmarker}{}%
\end{pgfscope}%
\begin{pgfscope}%
\pgfsys@transformshift{3.203728in}{0.509935in}%
\pgfsys@useobject{currentmarker}{}%
\end{pgfscope}%
\begin{pgfscope}%
\pgfsys@transformshift{3.644159in}{0.509920in}%
\pgfsys@useobject{currentmarker}{}%
\end{pgfscope}%
\begin{pgfscope}%
\pgfsys@transformshift{4.084591in}{0.509966in}%
\pgfsys@useobject{currentmarker}{}%
\end{pgfscope}%
\begin{pgfscope}%
\pgfsys@transformshift{4.525022in}{0.510025in}%
\pgfsys@useobject{currentmarker}{}%
\end{pgfscope}%
\begin{pgfscope}%
\pgfsys@transformshift{4.965453in}{0.509958in}%
\pgfsys@useobject{currentmarker}{}%
\end{pgfscope}%
\begin{pgfscope}%
\pgfsys@transformshift{5.405885in}{0.510074in}%
\pgfsys@useobject{currentmarker}{}%
\end{pgfscope}%
\end{pgfscope}%
\begin{pgfscope}%
\pgfpathrectangle{\pgfqpoint{0.781402in}{0.386794in}}{\pgfqpoint{4.844695in}{2.707560in}}%
\pgfusepath{clip}%
\pgfsetrectcap%
\pgfsetroundjoin%
\pgfsetlinewidth{1.505625pt}%
\definecolor{currentstroke}{rgb}{0.172549,0.627451,0.172549}%
\pgfsetstrokecolor{currentstroke}%
\pgfsetdash{}{0pt}%
\pgfpathmoveto{\pgfqpoint{1.001616in}{0.511428in}}%
\pgfpathlineto{\pgfqpoint{1.442003in}{0.748177in}}%
\pgfpathlineto{\pgfqpoint{1.882434in}{0.989661in}}%
\pgfpathlineto{\pgfqpoint{2.322866in}{1.245169in}}%
\pgfpathlineto{\pgfqpoint{2.763297in}{1.482760in}}%
\pgfpathlineto{\pgfqpoint{3.203728in}{1.741947in}}%
\pgfpathlineto{\pgfqpoint{3.644159in}{1.998971in}}%
\pgfpathlineto{\pgfqpoint{4.084591in}{2.250919in}}%
\pgfpathlineto{\pgfqpoint{4.525022in}{2.501474in}}%
\pgfpathlineto{\pgfqpoint{4.965453in}{2.747795in}}%
\pgfpathlineto{\pgfqpoint{5.405885in}{2.971283in}}%
\pgfusepath{stroke}%
\end{pgfscope}%
\begin{pgfscope}%
\pgfpathrectangle{\pgfqpoint{0.781402in}{0.386794in}}{\pgfqpoint{4.844695in}{2.707560in}}%
\pgfusepath{clip}%
\pgfsetbuttcap%
\pgfsetroundjoin%
\definecolor{currentfill}{rgb}{0.172549,0.627451,0.172549}%
\pgfsetfillcolor{currentfill}%
\pgfsetlinewidth{1.003750pt}%
\definecolor{currentstroke}{rgb}{0.172549,0.627451,0.172549}%
\pgfsetstrokecolor{currentstroke}%
\pgfsetdash{}{0pt}%
\pgfsys@defobject{currentmarker}{\pgfqpoint{-0.041667in}{-0.041667in}}{\pgfqpoint{0.041667in}{0.041667in}}{%
\pgfpathmoveto{\pgfqpoint{0.000000in}{-0.041667in}}%
\pgfpathcurveto{\pgfqpoint{0.011050in}{-0.041667in}}{\pgfqpoint{0.021649in}{-0.037276in}}{\pgfqpoint{0.029463in}{-0.029463in}}%
\pgfpathcurveto{\pgfqpoint{0.037276in}{-0.021649in}}{\pgfqpoint{0.041667in}{-0.011050in}}{\pgfqpoint{0.041667in}{0.000000in}}%
\pgfpathcurveto{\pgfqpoint{0.041667in}{0.011050in}}{\pgfqpoint{0.037276in}{0.021649in}}{\pgfqpoint{0.029463in}{0.029463in}}%
\pgfpathcurveto{\pgfqpoint{0.021649in}{0.037276in}}{\pgfqpoint{0.011050in}{0.041667in}}{\pgfqpoint{0.000000in}{0.041667in}}%
\pgfpathcurveto{\pgfqpoint{-0.011050in}{0.041667in}}{\pgfqpoint{-0.021649in}{0.037276in}}{\pgfqpoint{-0.029463in}{0.029463in}}%
\pgfpathcurveto{\pgfqpoint{-0.037276in}{0.021649in}}{\pgfqpoint{-0.041667in}{0.011050in}}{\pgfqpoint{-0.041667in}{0.000000in}}%
\pgfpathcurveto{\pgfqpoint{-0.041667in}{-0.011050in}}{\pgfqpoint{-0.037276in}{-0.021649in}}{\pgfqpoint{-0.029463in}{-0.029463in}}%
\pgfpathcurveto{\pgfqpoint{-0.021649in}{-0.037276in}}{\pgfqpoint{-0.011050in}{-0.041667in}}{\pgfqpoint{0.000000in}{-0.041667in}}%
\pgfpathclose%
\pgfusepath{stroke,fill}%
}%
\begin{pgfscope}%
\pgfsys@transformshift{1.001616in}{0.511428in}%
\pgfsys@useobject{currentmarker}{}%
\end{pgfscope}%
\begin{pgfscope}%
\pgfsys@transformshift{1.442003in}{0.748177in}%
\pgfsys@useobject{currentmarker}{}%
\end{pgfscope}%
\begin{pgfscope}%
\pgfsys@transformshift{1.882434in}{0.989661in}%
\pgfsys@useobject{currentmarker}{}%
\end{pgfscope}%
\begin{pgfscope}%
\pgfsys@transformshift{2.322866in}{1.245169in}%
\pgfsys@useobject{currentmarker}{}%
\end{pgfscope}%
\begin{pgfscope}%
\pgfsys@transformshift{2.763297in}{1.482760in}%
\pgfsys@useobject{currentmarker}{}%
\end{pgfscope}%
\begin{pgfscope}%
\pgfsys@transformshift{3.203728in}{1.741947in}%
\pgfsys@useobject{currentmarker}{}%
\end{pgfscope}%
\begin{pgfscope}%
\pgfsys@transformshift{3.644159in}{1.998971in}%
\pgfsys@useobject{currentmarker}{}%
\end{pgfscope}%
\begin{pgfscope}%
\pgfsys@transformshift{4.084591in}{2.250919in}%
\pgfsys@useobject{currentmarker}{}%
\end{pgfscope}%
\begin{pgfscope}%
\pgfsys@transformshift{4.525022in}{2.501474in}%
\pgfsys@useobject{currentmarker}{}%
\end{pgfscope}%
\begin{pgfscope}%
\pgfsys@transformshift{4.965453in}{2.747795in}%
\pgfsys@useobject{currentmarker}{}%
\end{pgfscope}%
\begin{pgfscope}%
\pgfsys@transformshift{5.405885in}{2.971283in}%
\pgfsys@useobject{currentmarker}{}%
\end{pgfscope}%
\end{pgfscope}%
\begin{pgfscope}%
\pgfsetrectcap%
\pgfsetmiterjoin%
\pgfsetlinewidth{0.803000pt}%
\definecolor{currentstroke}{rgb}{0.000000,0.000000,0.000000}%
\pgfsetstrokecolor{currentstroke}%
\pgfsetdash{}{0pt}%
\pgfpathmoveto{\pgfqpoint{0.781402in}{0.386794in}}%
\pgfpathlineto{\pgfqpoint{0.781402in}{3.094354in}}%
\pgfusepath{stroke}%
\end{pgfscope}%
\begin{pgfscope}%
\pgfsetrectcap%
\pgfsetmiterjoin%
\pgfsetlinewidth{0.803000pt}%
\definecolor{currentstroke}{rgb}{0.000000,0.000000,0.000000}%
\pgfsetstrokecolor{currentstroke}%
\pgfsetdash{}{0pt}%
\pgfpathmoveto{\pgfqpoint{5.626098in}{0.386794in}}%
\pgfpathlineto{\pgfqpoint{5.626098in}{3.094354in}}%
\pgfusepath{stroke}%
\end{pgfscope}%
\begin{pgfscope}%
\pgfsetrectcap%
\pgfsetmiterjoin%
\pgfsetlinewidth{0.803000pt}%
\definecolor{currentstroke}{rgb}{0.000000,0.000000,0.000000}%
\pgfsetstrokecolor{currentstroke}%
\pgfsetdash{}{0pt}%
\pgfpathmoveto{\pgfqpoint{0.781402in}{0.386794in}}%
\pgfpathlineto{\pgfqpoint{5.626098in}{0.386794in}}%
\pgfusepath{stroke}%
\end{pgfscope}%
\begin{pgfscope}%
\pgfsetrectcap%
\pgfsetmiterjoin%
\pgfsetlinewidth{0.803000pt}%
\definecolor{currentstroke}{rgb}{0.000000,0.000000,0.000000}%
\pgfsetstrokecolor{currentstroke}%
\pgfsetdash{}{0pt}%
\pgfpathmoveto{\pgfqpoint{0.781402in}{3.094354in}}%
\pgfpathlineto{\pgfqpoint{5.626098in}{3.094354in}}%
\pgfusepath{stroke}%
\end{pgfscope}%
\begin{pgfscope}%
\pgfsetbuttcap%
\pgfsetmiterjoin%
\definecolor{currentfill}{rgb}{1.000000,1.000000,1.000000}%
\pgfsetfillcolor{currentfill}%
\pgfsetfillopacity{0.800000}%
\pgfsetlinewidth{1.003750pt}%
\definecolor{currentstroke}{rgb}{0.800000,0.800000,0.800000}%
\pgfsetstrokecolor{currentstroke}%
\pgfsetstrokeopacity{0.800000}%
\pgfsetdash{}{0pt}%
\pgfpathmoveto{\pgfqpoint{0.878625in}{2.402224in}}%
\pgfpathlineto{\pgfqpoint{2.791822in}{2.402224in}}%
\pgfpathquadraticcurveto{\pgfqpoint{2.819600in}{2.402224in}}{\pgfqpoint{2.819600in}{2.430002in}}%
\pgfpathlineto{\pgfqpoint{2.819600in}{2.997132in}}%
\pgfpathquadraticcurveto{\pgfqpoint{2.819600in}{3.024909in}}{\pgfqpoint{2.791822in}{3.024909in}}%
\pgfpathlineto{\pgfqpoint{0.878625in}{3.024909in}}%
\pgfpathquadraticcurveto{\pgfqpoint{0.850847in}{3.024909in}}{\pgfqpoint{0.850847in}{2.997132in}}%
\pgfpathlineto{\pgfqpoint{0.850847in}{2.430002in}}%
\pgfpathquadraticcurveto{\pgfqpoint{0.850847in}{2.402224in}}{\pgfqpoint{0.878625in}{2.402224in}}%
\pgfpathclose%
\pgfusepath{stroke,fill}%
\end{pgfscope}%
\begin{pgfscope}%
\pgfsetrectcap%
\pgfsetroundjoin%
\pgfsetlinewidth{1.505625pt}%
\definecolor{currentstroke}{rgb}{0.121569,0.466667,0.705882}%
\pgfsetstrokecolor{currentstroke}%
\pgfsetdash{}{0pt}%
\pgfpathmoveto{\pgfqpoint{0.906402in}{2.920743in}}%
\pgfpathlineto{\pgfqpoint{1.184180in}{2.920743in}}%
\pgfusepath{stroke}%
\end{pgfscope}%
\begin{pgfscope}%
\pgfsetbuttcap%
\pgfsetroundjoin%
\definecolor{currentfill}{rgb}{0.121569,0.466667,0.705882}%
\pgfsetfillcolor{currentfill}%
\pgfsetlinewidth{1.003750pt}%
\definecolor{currentstroke}{rgb}{0.121569,0.466667,0.705882}%
\pgfsetstrokecolor{currentstroke}%
\pgfsetdash{}{0pt}%
\pgfsys@defobject{currentmarker}{\pgfqpoint{-0.041667in}{-0.041667in}}{\pgfqpoint{0.041667in}{0.041667in}}{%
\pgfpathmoveto{\pgfqpoint{0.000000in}{-0.041667in}}%
\pgfpathcurveto{\pgfqpoint{0.011050in}{-0.041667in}}{\pgfqpoint{0.021649in}{-0.037276in}}{\pgfqpoint{0.029463in}{-0.029463in}}%
\pgfpathcurveto{\pgfqpoint{0.037276in}{-0.021649in}}{\pgfqpoint{0.041667in}{-0.011050in}}{\pgfqpoint{0.041667in}{0.000000in}}%
\pgfpathcurveto{\pgfqpoint{0.041667in}{0.011050in}}{\pgfqpoint{0.037276in}{0.021649in}}{\pgfqpoint{0.029463in}{0.029463in}}%
\pgfpathcurveto{\pgfqpoint{0.021649in}{0.037276in}}{\pgfqpoint{0.011050in}{0.041667in}}{\pgfqpoint{0.000000in}{0.041667in}}%
\pgfpathcurveto{\pgfqpoint{-0.011050in}{0.041667in}}{\pgfqpoint{-0.021649in}{0.037276in}}{\pgfqpoint{-0.029463in}{0.029463in}}%
\pgfpathcurveto{\pgfqpoint{-0.037276in}{0.021649in}}{\pgfqpoint{-0.041667in}{0.011050in}}{\pgfqpoint{-0.041667in}{0.000000in}}%
\pgfpathcurveto{\pgfqpoint{-0.041667in}{-0.011050in}}{\pgfqpoint{-0.037276in}{-0.021649in}}{\pgfqpoint{-0.029463in}{-0.029463in}}%
\pgfpathcurveto{\pgfqpoint{-0.021649in}{-0.037276in}}{\pgfqpoint{-0.011050in}{-0.041667in}}{\pgfqpoint{0.000000in}{-0.041667in}}%
\pgfpathclose%
\pgfusepath{stroke,fill}%
}%
\begin{pgfscope}%
\pgfsys@transformshift{1.045291in}{2.920743in}%
\pgfsys@useobject{currentmarker}{}%
\end{pgfscope}%
\end{pgfscope}%
\begin{pgfscope}%
\definecolor{textcolor}{rgb}{0.000000,0.000000,0.000000}%
\pgfsetstrokecolor{textcolor}%
\pgfsetfillcolor{textcolor}%
\pgftext[x=1.295291in,y=2.872132in,left,base]{\color{textcolor}\rmfamily\fontsize{10.000000}{12.000000}\selectfont Prefill duration}%
\end{pgfscope}%
\begin{pgfscope}%
\pgfsetrectcap%
\pgfsetroundjoin%
\pgfsetlinewidth{1.505625pt}%
\definecolor{currentstroke}{rgb}{1.000000,0.498039,0.054902}%
\pgfsetstrokecolor{currentstroke}%
\pgfsetdash{}{0pt}%
\pgfpathmoveto{\pgfqpoint{0.906402in}{2.727070in}}%
\pgfpathlineto{\pgfqpoint{1.184180in}{2.727070in}}%
\pgfusepath{stroke}%
\end{pgfscope}%
\begin{pgfscope}%
\pgfsetbuttcap%
\pgfsetroundjoin%
\definecolor{currentfill}{rgb}{1.000000,0.498039,0.054902}%
\pgfsetfillcolor{currentfill}%
\pgfsetlinewidth{1.003750pt}%
\definecolor{currentstroke}{rgb}{1.000000,0.498039,0.054902}%
\pgfsetstrokecolor{currentstroke}%
\pgfsetdash{}{0pt}%
\pgfsys@defobject{currentmarker}{\pgfqpoint{-0.041667in}{-0.041667in}}{\pgfqpoint{0.041667in}{0.041667in}}{%
\pgfpathmoveto{\pgfqpoint{0.000000in}{-0.041667in}}%
\pgfpathcurveto{\pgfqpoint{0.011050in}{-0.041667in}}{\pgfqpoint{0.021649in}{-0.037276in}}{\pgfqpoint{0.029463in}{-0.029463in}}%
\pgfpathcurveto{\pgfqpoint{0.037276in}{-0.021649in}}{\pgfqpoint{0.041667in}{-0.011050in}}{\pgfqpoint{0.041667in}{0.000000in}}%
\pgfpathcurveto{\pgfqpoint{0.041667in}{0.011050in}}{\pgfqpoint{0.037276in}{0.021649in}}{\pgfqpoint{0.029463in}{0.029463in}}%
\pgfpathcurveto{\pgfqpoint{0.021649in}{0.037276in}}{\pgfqpoint{0.011050in}{0.041667in}}{\pgfqpoint{0.000000in}{0.041667in}}%
\pgfpathcurveto{\pgfqpoint{-0.011050in}{0.041667in}}{\pgfqpoint{-0.021649in}{0.037276in}}{\pgfqpoint{-0.029463in}{0.029463in}}%
\pgfpathcurveto{\pgfqpoint{-0.037276in}{0.021649in}}{\pgfqpoint{-0.041667in}{0.011050in}}{\pgfqpoint{-0.041667in}{0.000000in}}%
\pgfpathcurveto{\pgfqpoint{-0.041667in}{-0.011050in}}{\pgfqpoint{-0.037276in}{-0.021649in}}{\pgfqpoint{-0.029463in}{-0.029463in}}%
\pgfpathcurveto{\pgfqpoint{-0.021649in}{-0.037276in}}{\pgfqpoint{-0.011050in}{-0.041667in}}{\pgfqpoint{0.000000in}{-0.041667in}}%
\pgfpathclose%
\pgfusepath{stroke,fill}%
}%
\begin{pgfscope}%
\pgfsys@transformshift{1.045291in}{2.727070in}%
\pgfsys@useobject{currentmarker}{}%
\end{pgfscope}%
\end{pgfscope}%
\begin{pgfscope}%
\definecolor{textcolor}{rgb}{0.000000,0.000000,0.000000}%
\pgfsetstrokecolor{textcolor}%
\pgfsetfillcolor{textcolor}%
\pgftext[x=1.295291in,y=2.678459in,left,base]{\color{textcolor}\rmfamily\fontsize{10.000000}{12.000000}\selectfont Duration without owner}%
\end{pgfscope}%
\begin{pgfscope}%
\pgfsetrectcap%
\pgfsetroundjoin%
\pgfsetlinewidth{1.505625pt}%
\definecolor{currentstroke}{rgb}{0.172549,0.627451,0.172549}%
\pgfsetstrokecolor{currentstroke}%
\pgfsetdash{}{0pt}%
\pgfpathmoveto{\pgfqpoint{0.906402in}{2.533397in}}%
\pgfpathlineto{\pgfqpoint{1.184180in}{2.533397in}}%
\pgfusepath{stroke}%
\end{pgfscope}%
\begin{pgfscope}%
\pgfsetbuttcap%
\pgfsetroundjoin%
\definecolor{currentfill}{rgb}{0.172549,0.627451,0.172549}%
\pgfsetfillcolor{currentfill}%
\pgfsetlinewidth{1.003750pt}%
\definecolor{currentstroke}{rgb}{0.172549,0.627451,0.172549}%
\pgfsetstrokecolor{currentstroke}%
\pgfsetdash{}{0pt}%
\pgfsys@defobject{currentmarker}{\pgfqpoint{-0.041667in}{-0.041667in}}{\pgfqpoint{0.041667in}{0.041667in}}{%
\pgfpathmoveto{\pgfqpoint{0.000000in}{-0.041667in}}%
\pgfpathcurveto{\pgfqpoint{0.011050in}{-0.041667in}}{\pgfqpoint{0.021649in}{-0.037276in}}{\pgfqpoint{0.029463in}{-0.029463in}}%
\pgfpathcurveto{\pgfqpoint{0.037276in}{-0.021649in}}{\pgfqpoint{0.041667in}{-0.011050in}}{\pgfqpoint{0.041667in}{0.000000in}}%
\pgfpathcurveto{\pgfqpoint{0.041667in}{0.011050in}}{\pgfqpoint{0.037276in}{0.021649in}}{\pgfqpoint{0.029463in}{0.029463in}}%
\pgfpathcurveto{\pgfqpoint{0.021649in}{0.037276in}}{\pgfqpoint{0.011050in}{0.041667in}}{\pgfqpoint{0.000000in}{0.041667in}}%
\pgfpathcurveto{\pgfqpoint{-0.011050in}{0.041667in}}{\pgfqpoint{-0.021649in}{0.037276in}}{\pgfqpoint{-0.029463in}{0.029463in}}%
\pgfpathcurveto{\pgfqpoint{-0.037276in}{0.021649in}}{\pgfqpoint{-0.041667in}{0.011050in}}{\pgfqpoint{-0.041667in}{0.000000in}}%
\pgfpathcurveto{\pgfqpoint{-0.041667in}{-0.011050in}}{\pgfqpoint{-0.037276in}{-0.021649in}}{\pgfqpoint{-0.029463in}{-0.029463in}}%
\pgfpathcurveto{\pgfqpoint{-0.021649in}{-0.037276in}}{\pgfqpoint{-0.011050in}{-0.041667in}}{\pgfqpoint{0.000000in}{-0.041667in}}%
\pgfpathclose%
\pgfusepath{stroke,fill}%
}%
\begin{pgfscope}%
\pgfsys@transformshift{1.045291in}{2.533397in}%
\pgfsys@useobject{currentmarker}{}%
\end{pgfscope}%
\end{pgfscope}%
\begin{pgfscope}%
\definecolor{textcolor}{rgb}{0.000000,0.000000,0.000000}%
\pgfsetstrokecolor{textcolor}%
\pgfsetfillcolor{textcolor}%
\pgftext[x=1.295291in,y=2.484786in,left,base]{\color{textcolor}\rmfamily\fontsize{10.000000}{12.000000}\selectfont End to end latency}%
\end{pgfscope}%
\end{pgfpicture}%
\makeatother%
\endgroup%

    \end{center}
    \caption{Migration statistics of a clean vector}
    \label{fig:vectorreadonly}
\end{figure}

In our simplest example, we create a \texttt{vector}, initialize it with the
pre-specified size and migrate it to the destination. Approximately $8$ lines
of code are required on each of the source and destination sides to reproduce
this operation, excluding lines that serve the purpose of gathering statistics.
\autoref{fig:vectorreadonly} depicts the results.

Naturally, the prefill phase takes up most of the transfer time, which grows
linearly by increasing the size of the object. The increasing gap between the
prefill duration and the end to end latency can be attributed to syscalls and
invocation of the signal handler, both of which increase linearly with the
number of pages that are transferred before or after the prefill phase. The
time it takes to turn over the ownership is not impacted by the
size of the object and oscillates between tens of microseconds and hundreds
of microseconds. This is expected as this step consists of a single RDMA SEND.

\subsection{Migrating a vector while dirtying all of its pages}
\begin{figure}[tp]
    \begin{center}
        %% Creator: Matplotlib, PGF backend
%%
%% To include the figure in your LaTeX document, write
%%   \input{<filename>.pgf}
%%
%% Make sure the required packages are loaded in your preamble
%%   \usepackage{pgf}
%%
%% and, on pdftex
%%   \usepackage[utf8]{inputenc}\DeclareUnicodeCharacter{2212}{-}
%%
%% or, on luatex and xetex
%%   \usepackage{unicode-math}
%%
%% Figures using additional raster images can only be included by \input if
%% they are in the same directory as the main LaTeX file. For loading figures
%% from other directories you can use the `import` package
%%   \usepackage{import}
%%
%% and then include the figures with
%%   \import{<path to file>}{<filename>.pgf}
%%
%% Matplotlib used the following preamble
%%
\begingroup%
\makeatletter%
\begin{pgfpicture}%
\pgfpathrectangle{\pgfpointorigin}{\pgfqpoint{6.251220in}{3.516311in}}%
\pgfusepath{use as bounding box, clip}%
\begin{pgfscope}%
\pgfsetbuttcap%
\pgfsetmiterjoin%
\definecolor{currentfill}{rgb}{1.000000,1.000000,1.000000}%
\pgfsetfillcolor{currentfill}%
\pgfsetlinewidth{0.000000pt}%
\definecolor{currentstroke}{rgb}{1.000000,1.000000,1.000000}%
\pgfsetstrokecolor{currentstroke}%
\pgfsetdash{}{0pt}%
\pgfpathmoveto{\pgfqpoint{0.000000in}{0.000000in}}%
\pgfpathlineto{\pgfqpoint{6.251220in}{0.000000in}}%
\pgfpathlineto{\pgfqpoint{6.251220in}{3.516311in}}%
\pgfpathlineto{\pgfqpoint{0.000000in}{3.516311in}}%
\pgfpathclose%
\pgfusepath{fill}%
\end{pgfscope}%
\begin{pgfscope}%
\pgfsetbuttcap%
\pgfsetmiterjoin%
\definecolor{currentfill}{rgb}{1.000000,1.000000,1.000000}%
\pgfsetfillcolor{currentfill}%
\pgfsetlinewidth{0.000000pt}%
\definecolor{currentstroke}{rgb}{0.000000,0.000000,0.000000}%
\pgfsetstrokecolor{currentstroke}%
\pgfsetstrokeopacity{0.000000}%
\pgfsetdash{}{0pt}%
\pgfpathmoveto{\pgfqpoint{0.781402in}{0.386794in}}%
\pgfpathlineto{\pgfqpoint{5.626098in}{0.386794in}}%
\pgfpathlineto{\pgfqpoint{5.626098in}{3.094354in}}%
\pgfpathlineto{\pgfqpoint{0.781402in}{3.094354in}}%
\pgfpathclose%
\pgfusepath{fill}%
\end{pgfscope}%
\begin{pgfscope}%
\pgfsetbuttcap%
\pgfsetroundjoin%
\definecolor{currentfill}{rgb}{0.000000,0.000000,0.000000}%
\pgfsetfillcolor{currentfill}%
\pgfsetlinewidth{0.803000pt}%
\definecolor{currentstroke}{rgb}{0.000000,0.000000,0.000000}%
\pgfsetstrokecolor{currentstroke}%
\pgfsetdash{}{0pt}%
\pgfsys@defobject{currentmarker}{\pgfqpoint{0.000000in}{-0.048611in}}{\pgfqpoint{0.000000in}{0.000000in}}{%
\pgfpathmoveto{\pgfqpoint{0.000000in}{0.000000in}}%
\pgfpathlineto{\pgfqpoint{0.000000in}{-0.048611in}}%
\pgfusepath{stroke,fill}%
}%
\begin{pgfscope}%
\pgfsys@transformshift{1.001616in}{0.386794in}%
\pgfsys@useobject{currentmarker}{}%
\end{pgfscope}%
\end{pgfscope}%
\begin{pgfscope}%
\definecolor{textcolor}{rgb}{0.000000,0.000000,0.000000}%
\pgfsetstrokecolor{textcolor}%
\pgfsetfillcolor{textcolor}%
\pgftext[x=1.001616in,y=0.289572in,,top]{\color{textcolor}\rmfamily\fontsize{10.000000}{12.000000}\selectfont \(\displaystyle {1}\)}%
\end{pgfscope}%
\begin{pgfscope}%
\pgfsetbuttcap%
\pgfsetroundjoin%
\definecolor{currentfill}{rgb}{0.000000,0.000000,0.000000}%
\pgfsetfillcolor{currentfill}%
\pgfsetlinewidth{0.803000pt}%
\definecolor{currentstroke}{rgb}{0.000000,0.000000,0.000000}%
\pgfsetstrokecolor{currentstroke}%
\pgfsetdash{}{0pt}%
\pgfsys@defobject{currentmarker}{\pgfqpoint{0.000000in}{-0.048611in}}{\pgfqpoint{0.000000in}{0.000000in}}{%
\pgfpathmoveto{\pgfqpoint{0.000000in}{0.000000in}}%
\pgfpathlineto{\pgfqpoint{0.000000in}{-0.048611in}}%
\pgfusepath{stroke,fill}%
}%
\begin{pgfscope}%
\pgfsys@transformshift{1.882434in}{0.386794in}%
\pgfsys@useobject{currentmarker}{}%
\end{pgfscope}%
\end{pgfscope}%
\begin{pgfscope}%
\definecolor{textcolor}{rgb}{0.000000,0.000000,0.000000}%
\pgfsetstrokecolor{textcolor}%
\pgfsetfillcolor{textcolor}%
\pgftext[x=1.882434in,y=0.289572in,,top]{\color{textcolor}\rmfamily\fontsize{10.000000}{12.000000}\selectfont \(\displaystyle {20000}\)}%
\end{pgfscope}%
\begin{pgfscope}%
\pgfsetbuttcap%
\pgfsetroundjoin%
\definecolor{currentfill}{rgb}{0.000000,0.000000,0.000000}%
\pgfsetfillcolor{currentfill}%
\pgfsetlinewidth{0.803000pt}%
\definecolor{currentstroke}{rgb}{0.000000,0.000000,0.000000}%
\pgfsetstrokecolor{currentstroke}%
\pgfsetdash{}{0pt}%
\pgfsys@defobject{currentmarker}{\pgfqpoint{0.000000in}{-0.048611in}}{\pgfqpoint{0.000000in}{0.000000in}}{%
\pgfpathmoveto{\pgfqpoint{0.000000in}{0.000000in}}%
\pgfpathlineto{\pgfqpoint{0.000000in}{-0.048611in}}%
\pgfusepath{stroke,fill}%
}%
\begin{pgfscope}%
\pgfsys@transformshift{2.763297in}{0.386794in}%
\pgfsys@useobject{currentmarker}{}%
\end{pgfscope}%
\end{pgfscope}%
\begin{pgfscope}%
\definecolor{textcolor}{rgb}{0.000000,0.000000,0.000000}%
\pgfsetstrokecolor{textcolor}%
\pgfsetfillcolor{textcolor}%
\pgftext[x=2.763297in,y=0.289572in,,top]{\color{textcolor}\rmfamily\fontsize{10.000000}{12.000000}\selectfont \(\displaystyle {40000}\)}%
\end{pgfscope}%
\begin{pgfscope}%
\pgfsetbuttcap%
\pgfsetroundjoin%
\definecolor{currentfill}{rgb}{0.000000,0.000000,0.000000}%
\pgfsetfillcolor{currentfill}%
\pgfsetlinewidth{0.803000pt}%
\definecolor{currentstroke}{rgb}{0.000000,0.000000,0.000000}%
\pgfsetstrokecolor{currentstroke}%
\pgfsetdash{}{0pt}%
\pgfsys@defobject{currentmarker}{\pgfqpoint{0.000000in}{-0.048611in}}{\pgfqpoint{0.000000in}{0.000000in}}{%
\pgfpathmoveto{\pgfqpoint{0.000000in}{0.000000in}}%
\pgfpathlineto{\pgfqpoint{0.000000in}{-0.048611in}}%
\pgfusepath{stroke,fill}%
}%
\begin{pgfscope}%
\pgfsys@transformshift{3.644159in}{0.386794in}%
\pgfsys@useobject{currentmarker}{}%
\end{pgfscope}%
\end{pgfscope}%
\begin{pgfscope}%
\definecolor{textcolor}{rgb}{0.000000,0.000000,0.000000}%
\pgfsetstrokecolor{textcolor}%
\pgfsetfillcolor{textcolor}%
\pgftext[x=3.644159in,y=0.289572in,,top]{\color{textcolor}\rmfamily\fontsize{10.000000}{12.000000}\selectfont \(\displaystyle {60000}\)}%
\end{pgfscope}%
\begin{pgfscope}%
\pgfsetbuttcap%
\pgfsetroundjoin%
\definecolor{currentfill}{rgb}{0.000000,0.000000,0.000000}%
\pgfsetfillcolor{currentfill}%
\pgfsetlinewidth{0.803000pt}%
\definecolor{currentstroke}{rgb}{0.000000,0.000000,0.000000}%
\pgfsetstrokecolor{currentstroke}%
\pgfsetdash{}{0pt}%
\pgfsys@defobject{currentmarker}{\pgfqpoint{0.000000in}{-0.048611in}}{\pgfqpoint{0.000000in}{0.000000in}}{%
\pgfpathmoveto{\pgfqpoint{0.000000in}{0.000000in}}%
\pgfpathlineto{\pgfqpoint{0.000000in}{-0.048611in}}%
\pgfusepath{stroke,fill}%
}%
\begin{pgfscope}%
\pgfsys@transformshift{4.525022in}{0.386794in}%
\pgfsys@useobject{currentmarker}{}%
\end{pgfscope}%
\end{pgfscope}%
\begin{pgfscope}%
\definecolor{textcolor}{rgb}{0.000000,0.000000,0.000000}%
\pgfsetstrokecolor{textcolor}%
\pgfsetfillcolor{textcolor}%
\pgftext[x=4.525022in,y=0.289572in,,top]{\color{textcolor}\rmfamily\fontsize{10.000000}{12.000000}\selectfont \(\displaystyle {80000}\)}%
\end{pgfscope}%
\begin{pgfscope}%
\pgfsetbuttcap%
\pgfsetroundjoin%
\definecolor{currentfill}{rgb}{0.000000,0.000000,0.000000}%
\pgfsetfillcolor{currentfill}%
\pgfsetlinewidth{0.803000pt}%
\definecolor{currentstroke}{rgb}{0.000000,0.000000,0.000000}%
\pgfsetstrokecolor{currentstroke}%
\pgfsetdash{}{0pt}%
\pgfsys@defobject{currentmarker}{\pgfqpoint{0.000000in}{-0.048611in}}{\pgfqpoint{0.000000in}{0.000000in}}{%
\pgfpathmoveto{\pgfqpoint{0.000000in}{0.000000in}}%
\pgfpathlineto{\pgfqpoint{0.000000in}{-0.048611in}}%
\pgfusepath{stroke,fill}%
}%
\begin{pgfscope}%
\pgfsys@transformshift{5.405885in}{0.386794in}%
\pgfsys@useobject{currentmarker}{}%
\end{pgfscope}%
\end{pgfscope}%
\begin{pgfscope}%
\definecolor{textcolor}{rgb}{0.000000,0.000000,0.000000}%
\pgfsetstrokecolor{textcolor}%
\pgfsetfillcolor{textcolor}%
\pgftext[x=5.405885in,y=0.289572in,,top]{\color{textcolor}\rmfamily\fontsize{10.000000}{12.000000}\selectfont \(\displaystyle {100000}\)}%
\end{pgfscope}%
\begin{pgfscope}%
\definecolor{textcolor}{rgb}{0.000000,0.000000,0.000000}%
\pgfsetstrokecolor{textcolor}%
\pgfsetfillcolor{textcolor}%
\pgftext[x=3.203750in,y=0.110560in,,top]{\color{textcolor}\rmfamily\fontsize{10.000000}{12.000000}\selectfont Number of 4KB pages}%
\end{pgfscope}%
\begin{pgfscope}%
\pgfsetbuttcap%
\pgfsetroundjoin%
\definecolor{currentfill}{rgb}{0.000000,0.000000,0.000000}%
\pgfsetfillcolor{currentfill}%
\pgfsetlinewidth{0.803000pt}%
\definecolor{currentstroke}{rgb}{0.000000,0.000000,0.000000}%
\pgfsetstrokecolor{currentstroke}%
\pgfsetdash{}{0pt}%
\pgfsys@defobject{currentmarker}{\pgfqpoint{-0.048611in}{0.000000in}}{\pgfqpoint{-0.000000in}{0.000000in}}{%
\pgfpathmoveto{\pgfqpoint{-0.000000in}{0.000000in}}%
\pgfpathlineto{\pgfqpoint{-0.048611in}{0.000000in}}%
\pgfusepath{stroke,fill}%
}%
\begin{pgfscope}%
\pgfsys@transformshift{0.781402in}{0.509865in}%
\pgfsys@useobject{currentmarker}{}%
\end{pgfscope}%
\end{pgfscope}%
\begin{pgfscope}%
\definecolor{textcolor}{rgb}{0.000000,0.000000,0.000000}%
\pgfsetstrokecolor{textcolor}%
\pgfsetfillcolor{textcolor}%
\pgftext[x=0.614736in, y=0.461639in, left, base]{\color{textcolor}\rmfamily\fontsize{10.000000}{12.000000}\selectfont \(\displaystyle {0}\)}%
\end{pgfscope}%
\begin{pgfscope}%
\pgfsetbuttcap%
\pgfsetroundjoin%
\definecolor{currentfill}{rgb}{0.000000,0.000000,0.000000}%
\pgfsetfillcolor{currentfill}%
\pgfsetlinewidth{0.803000pt}%
\definecolor{currentstroke}{rgb}{0.000000,0.000000,0.000000}%
\pgfsetstrokecolor{currentstroke}%
\pgfsetdash{}{0pt}%
\pgfsys@defobject{currentmarker}{\pgfqpoint{-0.048611in}{0.000000in}}{\pgfqpoint{-0.000000in}{0.000000in}}{%
\pgfpathmoveto{\pgfqpoint{-0.000000in}{0.000000in}}%
\pgfpathlineto{\pgfqpoint{-0.048611in}{0.000000in}}%
\pgfusepath{stroke,fill}%
}%
\begin{pgfscope}%
\pgfsys@transformshift{0.781402in}{0.940778in}%
\pgfsys@useobject{currentmarker}{}%
\end{pgfscope}%
\end{pgfscope}%
\begin{pgfscope}%
\definecolor{textcolor}{rgb}{0.000000,0.000000,0.000000}%
\pgfsetstrokecolor{textcolor}%
\pgfsetfillcolor{textcolor}%
\pgftext[x=0.406402in, y=0.892553in, left, base]{\color{textcolor}\rmfamily\fontsize{10.000000}{12.000000}\selectfont \(\displaystyle {1000}\)}%
\end{pgfscope}%
\begin{pgfscope}%
\pgfsetbuttcap%
\pgfsetroundjoin%
\definecolor{currentfill}{rgb}{0.000000,0.000000,0.000000}%
\pgfsetfillcolor{currentfill}%
\pgfsetlinewidth{0.803000pt}%
\definecolor{currentstroke}{rgb}{0.000000,0.000000,0.000000}%
\pgfsetstrokecolor{currentstroke}%
\pgfsetdash{}{0pt}%
\pgfsys@defobject{currentmarker}{\pgfqpoint{-0.048611in}{0.000000in}}{\pgfqpoint{-0.000000in}{0.000000in}}{%
\pgfpathmoveto{\pgfqpoint{-0.000000in}{0.000000in}}%
\pgfpathlineto{\pgfqpoint{-0.048611in}{0.000000in}}%
\pgfusepath{stroke,fill}%
}%
\begin{pgfscope}%
\pgfsys@transformshift{0.781402in}{1.371692in}%
\pgfsys@useobject{currentmarker}{}%
\end{pgfscope}%
\end{pgfscope}%
\begin{pgfscope}%
\definecolor{textcolor}{rgb}{0.000000,0.000000,0.000000}%
\pgfsetstrokecolor{textcolor}%
\pgfsetfillcolor{textcolor}%
\pgftext[x=0.406402in, y=1.323466in, left, base]{\color{textcolor}\rmfamily\fontsize{10.000000}{12.000000}\selectfont \(\displaystyle {2000}\)}%
\end{pgfscope}%
\begin{pgfscope}%
\pgfsetbuttcap%
\pgfsetroundjoin%
\definecolor{currentfill}{rgb}{0.000000,0.000000,0.000000}%
\pgfsetfillcolor{currentfill}%
\pgfsetlinewidth{0.803000pt}%
\definecolor{currentstroke}{rgb}{0.000000,0.000000,0.000000}%
\pgfsetstrokecolor{currentstroke}%
\pgfsetdash{}{0pt}%
\pgfsys@defobject{currentmarker}{\pgfqpoint{-0.048611in}{0.000000in}}{\pgfqpoint{-0.000000in}{0.000000in}}{%
\pgfpathmoveto{\pgfqpoint{-0.000000in}{0.000000in}}%
\pgfpathlineto{\pgfqpoint{-0.048611in}{0.000000in}}%
\pgfusepath{stroke,fill}%
}%
\begin{pgfscope}%
\pgfsys@transformshift{0.781402in}{1.802605in}%
\pgfsys@useobject{currentmarker}{}%
\end{pgfscope}%
\end{pgfscope}%
\begin{pgfscope}%
\definecolor{textcolor}{rgb}{0.000000,0.000000,0.000000}%
\pgfsetstrokecolor{textcolor}%
\pgfsetfillcolor{textcolor}%
\pgftext[x=0.406402in, y=1.754380in, left, base]{\color{textcolor}\rmfamily\fontsize{10.000000}{12.000000}\selectfont \(\displaystyle {3000}\)}%
\end{pgfscope}%
\begin{pgfscope}%
\pgfsetbuttcap%
\pgfsetroundjoin%
\definecolor{currentfill}{rgb}{0.000000,0.000000,0.000000}%
\pgfsetfillcolor{currentfill}%
\pgfsetlinewidth{0.803000pt}%
\definecolor{currentstroke}{rgb}{0.000000,0.000000,0.000000}%
\pgfsetstrokecolor{currentstroke}%
\pgfsetdash{}{0pt}%
\pgfsys@defobject{currentmarker}{\pgfqpoint{-0.048611in}{0.000000in}}{\pgfqpoint{-0.000000in}{0.000000in}}{%
\pgfpathmoveto{\pgfqpoint{-0.000000in}{0.000000in}}%
\pgfpathlineto{\pgfqpoint{-0.048611in}{0.000000in}}%
\pgfusepath{stroke,fill}%
}%
\begin{pgfscope}%
\pgfsys@transformshift{0.781402in}{2.233519in}%
\pgfsys@useobject{currentmarker}{}%
\end{pgfscope}%
\end{pgfscope}%
\begin{pgfscope}%
\definecolor{textcolor}{rgb}{0.000000,0.000000,0.000000}%
\pgfsetstrokecolor{textcolor}%
\pgfsetfillcolor{textcolor}%
\pgftext[x=0.406402in, y=2.185293in, left, base]{\color{textcolor}\rmfamily\fontsize{10.000000}{12.000000}\selectfont \(\displaystyle {4000}\)}%
\end{pgfscope}%
\begin{pgfscope}%
\pgfsetbuttcap%
\pgfsetroundjoin%
\definecolor{currentfill}{rgb}{0.000000,0.000000,0.000000}%
\pgfsetfillcolor{currentfill}%
\pgfsetlinewidth{0.803000pt}%
\definecolor{currentstroke}{rgb}{0.000000,0.000000,0.000000}%
\pgfsetstrokecolor{currentstroke}%
\pgfsetdash{}{0pt}%
\pgfsys@defobject{currentmarker}{\pgfqpoint{-0.048611in}{0.000000in}}{\pgfqpoint{-0.000000in}{0.000000in}}{%
\pgfpathmoveto{\pgfqpoint{-0.000000in}{0.000000in}}%
\pgfpathlineto{\pgfqpoint{-0.048611in}{0.000000in}}%
\pgfusepath{stroke,fill}%
}%
\begin{pgfscope}%
\pgfsys@transformshift{0.781402in}{2.664432in}%
\pgfsys@useobject{currentmarker}{}%
\end{pgfscope}%
\end{pgfscope}%
\begin{pgfscope}%
\definecolor{textcolor}{rgb}{0.000000,0.000000,0.000000}%
\pgfsetstrokecolor{textcolor}%
\pgfsetfillcolor{textcolor}%
\pgftext[x=0.406402in, y=2.616207in, left, base]{\color{textcolor}\rmfamily\fontsize{10.000000}{12.000000}\selectfont \(\displaystyle {5000}\)}%
\end{pgfscope}%
\begin{pgfscope}%
\definecolor{textcolor}{rgb}{0.000000,0.000000,0.000000}%
\pgfsetstrokecolor{textcolor}%
\pgfsetfillcolor{textcolor}%
\pgftext[x=0.350846in,y=1.740574in,,bottom,rotate=90.000000]{\color{textcolor}\rmfamily\fontsize{10.000000}{12.000000}\selectfont Elapsed time (milliseconds)}%
\end{pgfscope}%
\begin{pgfscope}%
\pgfpathrectangle{\pgfqpoint{0.781402in}{0.386794in}}{\pgfqpoint{4.844695in}{2.707560in}}%
\pgfusepath{clip}%
\pgfsetrectcap%
\pgfsetroundjoin%
\pgfsetlinewidth{1.505625pt}%
\definecolor{currentstroke}{rgb}{0.121569,0.466667,0.705882}%
\pgfsetstrokecolor{currentstroke}%
\pgfsetdash{}{0pt}%
\pgfpathmoveto{\pgfqpoint{1.001616in}{0.510989in}}%
\pgfpathlineto{\pgfqpoint{1.442003in}{0.636670in}}%
\pgfpathlineto{\pgfqpoint{1.882434in}{0.769873in}}%
\pgfpathlineto{\pgfqpoint{2.322866in}{0.897130in}}%
\pgfpathlineto{\pgfqpoint{2.763297in}{1.038113in}}%
\pgfpathlineto{\pgfqpoint{3.203728in}{1.174420in}}%
\pgfpathlineto{\pgfqpoint{3.644159in}{1.303314in}}%
\pgfpathlineto{\pgfqpoint{4.084591in}{1.443472in}}%
\pgfpathlineto{\pgfqpoint{4.525022in}{1.583567in}}%
\pgfpathlineto{\pgfqpoint{4.965453in}{1.693462in}}%
\pgfpathlineto{\pgfqpoint{5.405885in}{1.839996in}}%
\pgfusepath{stroke}%
\end{pgfscope}%
\begin{pgfscope}%
\pgfpathrectangle{\pgfqpoint{0.781402in}{0.386794in}}{\pgfqpoint{4.844695in}{2.707560in}}%
\pgfusepath{clip}%
\pgfsetbuttcap%
\pgfsetroundjoin%
\definecolor{currentfill}{rgb}{0.121569,0.466667,0.705882}%
\pgfsetfillcolor{currentfill}%
\pgfsetlinewidth{1.003750pt}%
\definecolor{currentstroke}{rgb}{0.121569,0.466667,0.705882}%
\pgfsetstrokecolor{currentstroke}%
\pgfsetdash{}{0pt}%
\pgfsys@defobject{currentmarker}{\pgfqpoint{-0.041667in}{-0.041667in}}{\pgfqpoint{0.041667in}{0.041667in}}{%
\pgfpathmoveto{\pgfqpoint{0.000000in}{-0.041667in}}%
\pgfpathcurveto{\pgfqpoint{0.011050in}{-0.041667in}}{\pgfqpoint{0.021649in}{-0.037276in}}{\pgfqpoint{0.029463in}{-0.029463in}}%
\pgfpathcurveto{\pgfqpoint{0.037276in}{-0.021649in}}{\pgfqpoint{0.041667in}{-0.011050in}}{\pgfqpoint{0.041667in}{0.000000in}}%
\pgfpathcurveto{\pgfqpoint{0.041667in}{0.011050in}}{\pgfqpoint{0.037276in}{0.021649in}}{\pgfqpoint{0.029463in}{0.029463in}}%
\pgfpathcurveto{\pgfqpoint{0.021649in}{0.037276in}}{\pgfqpoint{0.011050in}{0.041667in}}{\pgfqpoint{0.000000in}{0.041667in}}%
\pgfpathcurveto{\pgfqpoint{-0.011050in}{0.041667in}}{\pgfqpoint{-0.021649in}{0.037276in}}{\pgfqpoint{-0.029463in}{0.029463in}}%
\pgfpathcurveto{\pgfqpoint{-0.037276in}{0.021649in}}{\pgfqpoint{-0.041667in}{0.011050in}}{\pgfqpoint{-0.041667in}{0.000000in}}%
\pgfpathcurveto{\pgfqpoint{-0.041667in}{-0.011050in}}{\pgfqpoint{-0.037276in}{-0.021649in}}{\pgfqpoint{-0.029463in}{-0.029463in}}%
\pgfpathcurveto{\pgfqpoint{-0.021649in}{-0.037276in}}{\pgfqpoint{-0.011050in}{-0.041667in}}{\pgfqpoint{0.000000in}{-0.041667in}}%
\pgfpathclose%
\pgfusepath{stroke,fill}%
}%
\begin{pgfscope}%
\pgfsys@transformshift{1.001616in}{0.510989in}%
\pgfsys@useobject{currentmarker}{}%
\end{pgfscope}%
\begin{pgfscope}%
\pgfsys@transformshift{1.442003in}{0.636670in}%
\pgfsys@useobject{currentmarker}{}%
\end{pgfscope}%
\begin{pgfscope}%
\pgfsys@transformshift{1.882434in}{0.769873in}%
\pgfsys@useobject{currentmarker}{}%
\end{pgfscope}%
\begin{pgfscope}%
\pgfsys@transformshift{2.322866in}{0.897130in}%
\pgfsys@useobject{currentmarker}{}%
\end{pgfscope}%
\begin{pgfscope}%
\pgfsys@transformshift{2.763297in}{1.038113in}%
\pgfsys@useobject{currentmarker}{}%
\end{pgfscope}%
\begin{pgfscope}%
\pgfsys@transformshift{3.203728in}{1.174420in}%
\pgfsys@useobject{currentmarker}{}%
\end{pgfscope}%
\begin{pgfscope}%
\pgfsys@transformshift{3.644159in}{1.303314in}%
\pgfsys@useobject{currentmarker}{}%
\end{pgfscope}%
\begin{pgfscope}%
\pgfsys@transformshift{4.084591in}{1.443472in}%
\pgfsys@useobject{currentmarker}{}%
\end{pgfscope}%
\begin{pgfscope}%
\pgfsys@transformshift{4.525022in}{1.583567in}%
\pgfsys@useobject{currentmarker}{}%
\end{pgfscope}%
\begin{pgfscope}%
\pgfsys@transformshift{4.965453in}{1.693462in}%
\pgfsys@useobject{currentmarker}{}%
\end{pgfscope}%
\begin{pgfscope}%
\pgfsys@transformshift{5.405885in}{1.839996in}%
\pgfsys@useobject{currentmarker}{}%
\end{pgfscope}%
\end{pgfscope}%
\begin{pgfscope}%
\pgfpathrectangle{\pgfqpoint{0.781402in}{0.386794in}}{\pgfqpoint{4.844695in}{2.707560in}}%
\pgfusepath{clip}%
\pgfsetrectcap%
\pgfsetroundjoin%
\pgfsetlinewidth{1.505625pt}%
\definecolor{currentstroke}{rgb}{1.000000,0.498039,0.054902}%
\pgfsetstrokecolor{currentstroke}%
\pgfsetdash{}{0pt}%
\pgfpathmoveto{\pgfqpoint{1.001616in}{0.510146in}}%
\pgfpathlineto{\pgfqpoint{1.442003in}{0.509865in}}%
\pgfpathlineto{\pgfqpoint{1.882434in}{0.509865in}}%
\pgfpathlineto{\pgfqpoint{2.322866in}{0.509865in}}%
\pgfpathlineto{\pgfqpoint{2.763297in}{0.509865in}}%
\pgfpathlineto{\pgfqpoint{3.203728in}{0.509865in}}%
\pgfpathlineto{\pgfqpoint{3.644159in}{0.510113in}}%
\pgfpathlineto{\pgfqpoint{4.084591in}{0.509865in}}%
\pgfpathlineto{\pgfqpoint{4.525022in}{0.509865in}}%
\pgfpathlineto{\pgfqpoint{4.965453in}{0.509865in}}%
\pgfpathlineto{\pgfqpoint{5.405885in}{0.509865in}}%
\pgfusepath{stroke}%
\end{pgfscope}%
\begin{pgfscope}%
\pgfpathrectangle{\pgfqpoint{0.781402in}{0.386794in}}{\pgfqpoint{4.844695in}{2.707560in}}%
\pgfusepath{clip}%
\pgfsetbuttcap%
\pgfsetroundjoin%
\definecolor{currentfill}{rgb}{1.000000,0.498039,0.054902}%
\pgfsetfillcolor{currentfill}%
\pgfsetlinewidth{1.003750pt}%
\definecolor{currentstroke}{rgb}{1.000000,0.498039,0.054902}%
\pgfsetstrokecolor{currentstroke}%
\pgfsetdash{}{0pt}%
\pgfsys@defobject{currentmarker}{\pgfqpoint{-0.041667in}{-0.041667in}}{\pgfqpoint{0.041667in}{0.041667in}}{%
\pgfpathmoveto{\pgfqpoint{0.000000in}{-0.041667in}}%
\pgfpathcurveto{\pgfqpoint{0.011050in}{-0.041667in}}{\pgfqpoint{0.021649in}{-0.037276in}}{\pgfqpoint{0.029463in}{-0.029463in}}%
\pgfpathcurveto{\pgfqpoint{0.037276in}{-0.021649in}}{\pgfqpoint{0.041667in}{-0.011050in}}{\pgfqpoint{0.041667in}{0.000000in}}%
\pgfpathcurveto{\pgfqpoint{0.041667in}{0.011050in}}{\pgfqpoint{0.037276in}{0.021649in}}{\pgfqpoint{0.029463in}{0.029463in}}%
\pgfpathcurveto{\pgfqpoint{0.021649in}{0.037276in}}{\pgfqpoint{0.011050in}{0.041667in}}{\pgfqpoint{0.000000in}{0.041667in}}%
\pgfpathcurveto{\pgfqpoint{-0.011050in}{0.041667in}}{\pgfqpoint{-0.021649in}{0.037276in}}{\pgfqpoint{-0.029463in}{0.029463in}}%
\pgfpathcurveto{\pgfqpoint{-0.037276in}{0.021649in}}{\pgfqpoint{-0.041667in}{0.011050in}}{\pgfqpoint{-0.041667in}{0.000000in}}%
\pgfpathcurveto{\pgfqpoint{-0.041667in}{-0.011050in}}{\pgfqpoint{-0.037276in}{-0.021649in}}{\pgfqpoint{-0.029463in}{-0.029463in}}%
\pgfpathcurveto{\pgfqpoint{-0.021649in}{-0.037276in}}{\pgfqpoint{-0.011050in}{-0.041667in}}{\pgfqpoint{0.000000in}{-0.041667in}}%
\pgfpathclose%
\pgfusepath{stroke,fill}%
}%
\begin{pgfscope}%
\pgfsys@transformshift{1.001616in}{0.510146in}%
\pgfsys@useobject{currentmarker}{}%
\end{pgfscope}%
\begin{pgfscope}%
\pgfsys@transformshift{1.442003in}{0.509865in}%
\pgfsys@useobject{currentmarker}{}%
\end{pgfscope}%
\begin{pgfscope}%
\pgfsys@transformshift{1.882434in}{0.509865in}%
\pgfsys@useobject{currentmarker}{}%
\end{pgfscope}%
\begin{pgfscope}%
\pgfsys@transformshift{2.322866in}{0.509865in}%
\pgfsys@useobject{currentmarker}{}%
\end{pgfscope}%
\begin{pgfscope}%
\pgfsys@transformshift{2.763297in}{0.509865in}%
\pgfsys@useobject{currentmarker}{}%
\end{pgfscope}%
\begin{pgfscope}%
\pgfsys@transformshift{3.203728in}{0.509865in}%
\pgfsys@useobject{currentmarker}{}%
\end{pgfscope}%
\begin{pgfscope}%
\pgfsys@transformshift{3.644159in}{0.510113in}%
\pgfsys@useobject{currentmarker}{}%
\end{pgfscope}%
\begin{pgfscope}%
\pgfsys@transformshift{4.084591in}{0.509865in}%
\pgfsys@useobject{currentmarker}{}%
\end{pgfscope}%
\begin{pgfscope}%
\pgfsys@transformshift{4.525022in}{0.509865in}%
\pgfsys@useobject{currentmarker}{}%
\end{pgfscope}%
\begin{pgfscope}%
\pgfsys@transformshift{4.965453in}{0.509865in}%
\pgfsys@useobject{currentmarker}{}%
\end{pgfscope}%
\begin{pgfscope}%
\pgfsys@transformshift{5.405885in}{0.509865in}%
\pgfsys@useobject{currentmarker}{}%
\end{pgfscope}%
\end{pgfscope}%
\begin{pgfscope}%
\pgfpathrectangle{\pgfqpoint{0.781402in}{0.386794in}}{\pgfqpoint{4.844695in}{2.707560in}}%
\pgfusepath{clip}%
\pgfsetrectcap%
\pgfsetroundjoin%
\pgfsetlinewidth{1.505625pt}%
\definecolor{currentstroke}{rgb}{0.172549,0.627451,0.172549}%
\pgfsetstrokecolor{currentstroke}%
\pgfsetdash{}{0pt}%
\pgfpathmoveto{\pgfqpoint{1.001616in}{0.509883in}}%
\pgfpathlineto{\pgfqpoint{1.442003in}{0.545971in}}%
\pgfpathlineto{\pgfqpoint{1.882434in}{0.591338in}}%
\pgfpathlineto{\pgfqpoint{2.322866in}{0.636282in}}%
\pgfpathlineto{\pgfqpoint{2.763297in}{0.683650in}}%
\pgfpathlineto{\pgfqpoint{3.203728in}{0.728078in}}%
\pgfpathlineto{\pgfqpoint{3.644159in}{0.776211in}}%
\pgfpathlineto{\pgfqpoint{4.084591in}{0.825757in}}%
\pgfpathlineto{\pgfqpoint{4.525022in}{0.881494in}}%
\pgfpathlineto{\pgfqpoint{4.965453in}{0.924303in}}%
\pgfpathlineto{\pgfqpoint{5.405885in}{0.970340in}}%
\pgfusepath{stroke}%
\end{pgfscope}%
\begin{pgfscope}%
\pgfpathrectangle{\pgfqpoint{0.781402in}{0.386794in}}{\pgfqpoint{4.844695in}{2.707560in}}%
\pgfusepath{clip}%
\pgfsetbuttcap%
\pgfsetroundjoin%
\definecolor{currentfill}{rgb}{0.172549,0.627451,0.172549}%
\pgfsetfillcolor{currentfill}%
\pgfsetlinewidth{1.003750pt}%
\definecolor{currentstroke}{rgb}{0.172549,0.627451,0.172549}%
\pgfsetstrokecolor{currentstroke}%
\pgfsetdash{}{0pt}%
\pgfsys@defobject{currentmarker}{\pgfqpoint{-0.041667in}{-0.041667in}}{\pgfqpoint{0.041667in}{0.041667in}}{%
\pgfpathmoveto{\pgfqpoint{0.000000in}{-0.041667in}}%
\pgfpathcurveto{\pgfqpoint{0.011050in}{-0.041667in}}{\pgfqpoint{0.021649in}{-0.037276in}}{\pgfqpoint{0.029463in}{-0.029463in}}%
\pgfpathcurveto{\pgfqpoint{0.037276in}{-0.021649in}}{\pgfqpoint{0.041667in}{-0.011050in}}{\pgfqpoint{0.041667in}{0.000000in}}%
\pgfpathcurveto{\pgfqpoint{0.041667in}{0.011050in}}{\pgfqpoint{0.037276in}{0.021649in}}{\pgfqpoint{0.029463in}{0.029463in}}%
\pgfpathcurveto{\pgfqpoint{0.021649in}{0.037276in}}{\pgfqpoint{0.011050in}{0.041667in}}{\pgfqpoint{0.000000in}{0.041667in}}%
\pgfpathcurveto{\pgfqpoint{-0.011050in}{0.041667in}}{\pgfqpoint{-0.021649in}{0.037276in}}{\pgfqpoint{-0.029463in}{0.029463in}}%
\pgfpathcurveto{\pgfqpoint{-0.037276in}{0.021649in}}{\pgfqpoint{-0.041667in}{0.011050in}}{\pgfqpoint{-0.041667in}{0.000000in}}%
\pgfpathcurveto{\pgfqpoint{-0.041667in}{-0.011050in}}{\pgfqpoint{-0.037276in}{-0.021649in}}{\pgfqpoint{-0.029463in}{-0.029463in}}%
\pgfpathcurveto{\pgfqpoint{-0.021649in}{-0.037276in}}{\pgfqpoint{-0.011050in}{-0.041667in}}{\pgfqpoint{0.000000in}{-0.041667in}}%
\pgfpathclose%
\pgfusepath{stroke,fill}%
}%
\begin{pgfscope}%
\pgfsys@transformshift{1.001616in}{0.509883in}%
\pgfsys@useobject{currentmarker}{}%
\end{pgfscope}%
\begin{pgfscope}%
\pgfsys@transformshift{1.442003in}{0.545971in}%
\pgfsys@useobject{currentmarker}{}%
\end{pgfscope}%
\begin{pgfscope}%
\pgfsys@transformshift{1.882434in}{0.591338in}%
\pgfsys@useobject{currentmarker}{}%
\end{pgfscope}%
\begin{pgfscope}%
\pgfsys@transformshift{2.322866in}{0.636282in}%
\pgfsys@useobject{currentmarker}{}%
\end{pgfscope}%
\begin{pgfscope}%
\pgfsys@transformshift{2.763297in}{0.683650in}%
\pgfsys@useobject{currentmarker}{}%
\end{pgfscope}%
\begin{pgfscope}%
\pgfsys@transformshift{3.203728in}{0.728078in}%
\pgfsys@useobject{currentmarker}{}%
\end{pgfscope}%
\begin{pgfscope}%
\pgfsys@transformshift{3.644159in}{0.776211in}%
\pgfsys@useobject{currentmarker}{}%
\end{pgfscope}%
\begin{pgfscope}%
\pgfsys@transformshift{4.084591in}{0.825757in}%
\pgfsys@useobject{currentmarker}{}%
\end{pgfscope}%
\begin{pgfscope}%
\pgfsys@transformshift{4.525022in}{0.881494in}%
\pgfsys@useobject{currentmarker}{}%
\end{pgfscope}%
\begin{pgfscope}%
\pgfsys@transformshift{4.965453in}{0.924303in}%
\pgfsys@useobject{currentmarker}{}%
\end{pgfscope}%
\begin{pgfscope}%
\pgfsys@transformshift{5.405885in}{0.970340in}%
\pgfsys@useobject{currentmarker}{}%
\end{pgfscope}%
\end{pgfscope}%
\begin{pgfscope}%
\pgfpathrectangle{\pgfqpoint{0.781402in}{0.386794in}}{\pgfqpoint{4.844695in}{2.707560in}}%
\pgfusepath{clip}%
\pgfsetrectcap%
\pgfsetroundjoin%
\pgfsetlinewidth{1.505625pt}%
\definecolor{currentstroke}{rgb}{0.839216,0.152941,0.156863}%
\pgfsetstrokecolor{currentstroke}%
\pgfsetdash{}{0pt}%
\pgfpathmoveto{\pgfqpoint{1.001616in}{0.511362in}}%
\pgfpathlineto{\pgfqpoint{1.442003in}{0.738373in}}%
\pgfpathlineto{\pgfqpoint{1.882434in}{0.982789in}}%
\pgfpathlineto{\pgfqpoint{2.322866in}{1.213948in}}%
\pgfpathlineto{\pgfqpoint{2.763297in}{1.479409in}}%
\pgfpathlineto{\pgfqpoint{3.203728in}{1.722312in}}%
\pgfpathlineto{\pgfqpoint{3.644159in}{1.963728in}}%
\pgfpathlineto{\pgfqpoint{4.084591in}{2.228331in}}%
\pgfpathlineto{\pgfqpoint{4.525022in}{2.489922in}}%
\pgfpathlineto{\pgfqpoint{4.965453in}{2.712077in}}%
\pgfpathlineto{\pgfqpoint{5.405885in}{2.971283in}}%
\pgfusepath{stroke}%
\end{pgfscope}%
\begin{pgfscope}%
\pgfpathrectangle{\pgfqpoint{0.781402in}{0.386794in}}{\pgfqpoint{4.844695in}{2.707560in}}%
\pgfusepath{clip}%
\pgfsetbuttcap%
\pgfsetroundjoin%
\definecolor{currentfill}{rgb}{0.839216,0.152941,0.156863}%
\pgfsetfillcolor{currentfill}%
\pgfsetlinewidth{1.003750pt}%
\definecolor{currentstroke}{rgb}{0.839216,0.152941,0.156863}%
\pgfsetstrokecolor{currentstroke}%
\pgfsetdash{}{0pt}%
\pgfsys@defobject{currentmarker}{\pgfqpoint{-0.041667in}{-0.041667in}}{\pgfqpoint{0.041667in}{0.041667in}}{%
\pgfpathmoveto{\pgfqpoint{0.000000in}{-0.041667in}}%
\pgfpathcurveto{\pgfqpoint{0.011050in}{-0.041667in}}{\pgfqpoint{0.021649in}{-0.037276in}}{\pgfqpoint{0.029463in}{-0.029463in}}%
\pgfpathcurveto{\pgfqpoint{0.037276in}{-0.021649in}}{\pgfqpoint{0.041667in}{-0.011050in}}{\pgfqpoint{0.041667in}{0.000000in}}%
\pgfpathcurveto{\pgfqpoint{0.041667in}{0.011050in}}{\pgfqpoint{0.037276in}{0.021649in}}{\pgfqpoint{0.029463in}{0.029463in}}%
\pgfpathcurveto{\pgfqpoint{0.021649in}{0.037276in}}{\pgfqpoint{0.011050in}{0.041667in}}{\pgfqpoint{0.000000in}{0.041667in}}%
\pgfpathcurveto{\pgfqpoint{-0.011050in}{0.041667in}}{\pgfqpoint{-0.021649in}{0.037276in}}{\pgfqpoint{-0.029463in}{0.029463in}}%
\pgfpathcurveto{\pgfqpoint{-0.037276in}{0.021649in}}{\pgfqpoint{-0.041667in}{0.011050in}}{\pgfqpoint{-0.041667in}{0.000000in}}%
\pgfpathcurveto{\pgfqpoint{-0.041667in}{-0.011050in}}{\pgfqpoint{-0.037276in}{-0.021649in}}{\pgfqpoint{-0.029463in}{-0.029463in}}%
\pgfpathcurveto{\pgfqpoint{-0.021649in}{-0.037276in}}{\pgfqpoint{-0.011050in}{-0.041667in}}{\pgfqpoint{0.000000in}{-0.041667in}}%
\pgfpathclose%
\pgfusepath{stroke,fill}%
}%
\begin{pgfscope}%
\pgfsys@transformshift{1.001616in}{0.511362in}%
\pgfsys@useobject{currentmarker}{}%
\end{pgfscope}%
\begin{pgfscope}%
\pgfsys@transformshift{1.442003in}{0.738373in}%
\pgfsys@useobject{currentmarker}{}%
\end{pgfscope}%
\begin{pgfscope}%
\pgfsys@transformshift{1.882434in}{0.982789in}%
\pgfsys@useobject{currentmarker}{}%
\end{pgfscope}%
\begin{pgfscope}%
\pgfsys@transformshift{2.322866in}{1.213948in}%
\pgfsys@useobject{currentmarker}{}%
\end{pgfscope}%
\begin{pgfscope}%
\pgfsys@transformshift{2.763297in}{1.479409in}%
\pgfsys@useobject{currentmarker}{}%
\end{pgfscope}%
\begin{pgfscope}%
\pgfsys@transformshift{3.203728in}{1.722312in}%
\pgfsys@useobject{currentmarker}{}%
\end{pgfscope}%
\begin{pgfscope}%
\pgfsys@transformshift{3.644159in}{1.963728in}%
\pgfsys@useobject{currentmarker}{}%
\end{pgfscope}%
\begin{pgfscope}%
\pgfsys@transformshift{4.084591in}{2.228331in}%
\pgfsys@useobject{currentmarker}{}%
\end{pgfscope}%
\begin{pgfscope}%
\pgfsys@transformshift{4.525022in}{2.489922in}%
\pgfsys@useobject{currentmarker}{}%
\end{pgfscope}%
\begin{pgfscope}%
\pgfsys@transformshift{4.965453in}{2.712077in}%
\pgfsys@useobject{currentmarker}{}%
\end{pgfscope}%
\begin{pgfscope}%
\pgfsys@transformshift{5.405885in}{2.971283in}%
\pgfsys@useobject{currentmarker}{}%
\end{pgfscope}%
\end{pgfscope}%
\begin{pgfscope}%
\pgfsetrectcap%
\pgfsetmiterjoin%
\pgfsetlinewidth{0.803000pt}%
\definecolor{currentstroke}{rgb}{0.000000,0.000000,0.000000}%
\pgfsetstrokecolor{currentstroke}%
\pgfsetdash{}{0pt}%
\pgfpathmoveto{\pgfqpoint{0.781402in}{0.386794in}}%
\pgfpathlineto{\pgfqpoint{0.781402in}{3.094354in}}%
\pgfusepath{stroke}%
\end{pgfscope}%
\begin{pgfscope}%
\pgfsetrectcap%
\pgfsetmiterjoin%
\pgfsetlinewidth{0.803000pt}%
\definecolor{currentstroke}{rgb}{0.000000,0.000000,0.000000}%
\pgfsetstrokecolor{currentstroke}%
\pgfsetdash{}{0pt}%
\pgfpathmoveto{\pgfqpoint{5.626098in}{0.386794in}}%
\pgfpathlineto{\pgfqpoint{5.626098in}{3.094354in}}%
\pgfusepath{stroke}%
\end{pgfscope}%
\begin{pgfscope}%
\pgfsetrectcap%
\pgfsetmiterjoin%
\pgfsetlinewidth{0.803000pt}%
\definecolor{currentstroke}{rgb}{0.000000,0.000000,0.000000}%
\pgfsetstrokecolor{currentstroke}%
\pgfsetdash{}{0pt}%
\pgfpathmoveto{\pgfqpoint{0.781402in}{0.386794in}}%
\pgfpathlineto{\pgfqpoint{5.626098in}{0.386794in}}%
\pgfusepath{stroke}%
\end{pgfscope}%
\begin{pgfscope}%
\pgfsetrectcap%
\pgfsetmiterjoin%
\pgfsetlinewidth{0.803000pt}%
\definecolor{currentstroke}{rgb}{0.000000,0.000000,0.000000}%
\pgfsetstrokecolor{currentstroke}%
\pgfsetdash{}{0pt}%
\pgfpathmoveto{\pgfqpoint{0.781402in}{3.094354in}}%
\pgfpathlineto{\pgfqpoint{5.626098in}{3.094354in}}%
\pgfusepath{stroke}%
\end{pgfscope}%
\begin{pgfscope}%
\pgfsetbuttcap%
\pgfsetmiterjoin%
\definecolor{currentfill}{rgb}{1.000000,1.000000,1.000000}%
\pgfsetfillcolor{currentfill}%
\pgfsetfillopacity{0.800000}%
\pgfsetlinewidth{1.003750pt}%
\definecolor{currentstroke}{rgb}{0.800000,0.800000,0.800000}%
\pgfsetstrokecolor{currentstroke}%
\pgfsetstrokeopacity{0.800000}%
\pgfsetdash{}{0pt}%
\pgfpathmoveto{\pgfqpoint{0.878625in}{2.208552in}}%
\pgfpathlineto{\pgfqpoint{3.483567in}{2.208552in}}%
\pgfpathquadraticcurveto{\pgfqpoint{3.511345in}{2.208552in}}{\pgfqpoint{3.511345in}{2.236329in}}%
\pgfpathlineto{\pgfqpoint{3.511345in}{2.997132in}}%
\pgfpathquadraticcurveto{\pgfqpoint{3.511345in}{3.024909in}}{\pgfqpoint{3.483567in}{3.024909in}}%
\pgfpathlineto{\pgfqpoint{0.878625in}{3.024909in}}%
\pgfpathquadraticcurveto{\pgfqpoint{0.850847in}{3.024909in}}{\pgfqpoint{0.850847in}{2.997132in}}%
\pgfpathlineto{\pgfqpoint{0.850847in}{2.236329in}}%
\pgfpathquadraticcurveto{\pgfqpoint{0.850847in}{2.208552in}}{\pgfqpoint{0.878625in}{2.208552in}}%
\pgfpathclose%
\pgfusepath{stroke,fill}%
\end{pgfscope}%
\begin{pgfscope}%
\pgfsetrectcap%
\pgfsetroundjoin%
\pgfsetlinewidth{1.505625pt}%
\definecolor{currentstroke}{rgb}{0.121569,0.466667,0.705882}%
\pgfsetstrokecolor{currentstroke}%
\pgfsetdash{}{0pt}%
\pgfpathmoveto{\pgfqpoint{0.906402in}{2.920743in}}%
\pgfpathlineto{\pgfqpoint{1.184180in}{2.920743in}}%
\pgfusepath{stroke}%
\end{pgfscope}%
\begin{pgfscope}%
\pgfsetbuttcap%
\pgfsetroundjoin%
\definecolor{currentfill}{rgb}{0.121569,0.466667,0.705882}%
\pgfsetfillcolor{currentfill}%
\pgfsetlinewidth{1.003750pt}%
\definecolor{currentstroke}{rgb}{0.121569,0.466667,0.705882}%
\pgfsetstrokecolor{currentstroke}%
\pgfsetdash{}{0pt}%
\pgfsys@defobject{currentmarker}{\pgfqpoint{-0.041667in}{-0.041667in}}{\pgfqpoint{0.041667in}{0.041667in}}{%
\pgfpathmoveto{\pgfqpoint{0.000000in}{-0.041667in}}%
\pgfpathcurveto{\pgfqpoint{0.011050in}{-0.041667in}}{\pgfqpoint{0.021649in}{-0.037276in}}{\pgfqpoint{0.029463in}{-0.029463in}}%
\pgfpathcurveto{\pgfqpoint{0.037276in}{-0.021649in}}{\pgfqpoint{0.041667in}{-0.011050in}}{\pgfqpoint{0.041667in}{0.000000in}}%
\pgfpathcurveto{\pgfqpoint{0.041667in}{0.011050in}}{\pgfqpoint{0.037276in}{0.021649in}}{\pgfqpoint{0.029463in}{0.029463in}}%
\pgfpathcurveto{\pgfqpoint{0.021649in}{0.037276in}}{\pgfqpoint{0.011050in}{0.041667in}}{\pgfqpoint{0.000000in}{0.041667in}}%
\pgfpathcurveto{\pgfqpoint{-0.011050in}{0.041667in}}{\pgfqpoint{-0.021649in}{0.037276in}}{\pgfqpoint{-0.029463in}{0.029463in}}%
\pgfpathcurveto{\pgfqpoint{-0.037276in}{0.021649in}}{\pgfqpoint{-0.041667in}{0.011050in}}{\pgfqpoint{-0.041667in}{0.000000in}}%
\pgfpathcurveto{\pgfqpoint{-0.041667in}{-0.011050in}}{\pgfqpoint{-0.037276in}{-0.021649in}}{\pgfqpoint{-0.029463in}{-0.029463in}}%
\pgfpathcurveto{\pgfqpoint{-0.021649in}{-0.037276in}}{\pgfqpoint{-0.011050in}{-0.041667in}}{\pgfqpoint{0.000000in}{-0.041667in}}%
\pgfpathclose%
\pgfusepath{stroke,fill}%
}%
\begin{pgfscope}%
\pgfsys@transformshift{1.045291in}{2.920743in}%
\pgfsys@useobject{currentmarker}{}%
\end{pgfscope}%
\end{pgfscope}%
\begin{pgfscope}%
\definecolor{textcolor}{rgb}{0.000000,0.000000,0.000000}%
\pgfsetstrokecolor{textcolor}%
\pgfsetfillcolor{textcolor}%
\pgftext[x=1.295291in,y=2.872132in,left,base]{\color{textcolor}\rmfamily\fontsize{10.000000}{12.000000}\selectfont Prefill duration}%
\end{pgfscope}%
\begin{pgfscope}%
\pgfsetrectcap%
\pgfsetroundjoin%
\pgfsetlinewidth{1.505625pt}%
\definecolor{currentstroke}{rgb}{1.000000,0.498039,0.054902}%
\pgfsetstrokecolor{currentstroke}%
\pgfsetdash{}{0pt}%
\pgfpathmoveto{\pgfqpoint{0.906402in}{2.727070in}}%
\pgfpathlineto{\pgfqpoint{1.184180in}{2.727070in}}%
\pgfusepath{stroke}%
\end{pgfscope}%
\begin{pgfscope}%
\pgfsetbuttcap%
\pgfsetroundjoin%
\definecolor{currentfill}{rgb}{1.000000,0.498039,0.054902}%
\pgfsetfillcolor{currentfill}%
\pgfsetlinewidth{1.003750pt}%
\definecolor{currentstroke}{rgb}{1.000000,0.498039,0.054902}%
\pgfsetstrokecolor{currentstroke}%
\pgfsetdash{}{0pt}%
\pgfsys@defobject{currentmarker}{\pgfqpoint{-0.041667in}{-0.041667in}}{\pgfqpoint{0.041667in}{0.041667in}}{%
\pgfpathmoveto{\pgfqpoint{0.000000in}{-0.041667in}}%
\pgfpathcurveto{\pgfqpoint{0.011050in}{-0.041667in}}{\pgfqpoint{0.021649in}{-0.037276in}}{\pgfqpoint{0.029463in}{-0.029463in}}%
\pgfpathcurveto{\pgfqpoint{0.037276in}{-0.021649in}}{\pgfqpoint{0.041667in}{-0.011050in}}{\pgfqpoint{0.041667in}{0.000000in}}%
\pgfpathcurveto{\pgfqpoint{0.041667in}{0.011050in}}{\pgfqpoint{0.037276in}{0.021649in}}{\pgfqpoint{0.029463in}{0.029463in}}%
\pgfpathcurveto{\pgfqpoint{0.021649in}{0.037276in}}{\pgfqpoint{0.011050in}{0.041667in}}{\pgfqpoint{0.000000in}{0.041667in}}%
\pgfpathcurveto{\pgfqpoint{-0.011050in}{0.041667in}}{\pgfqpoint{-0.021649in}{0.037276in}}{\pgfqpoint{-0.029463in}{0.029463in}}%
\pgfpathcurveto{\pgfqpoint{-0.037276in}{0.021649in}}{\pgfqpoint{-0.041667in}{0.011050in}}{\pgfqpoint{-0.041667in}{0.000000in}}%
\pgfpathcurveto{\pgfqpoint{-0.041667in}{-0.011050in}}{\pgfqpoint{-0.037276in}{-0.021649in}}{\pgfqpoint{-0.029463in}{-0.029463in}}%
\pgfpathcurveto{\pgfqpoint{-0.021649in}{-0.037276in}}{\pgfqpoint{-0.011050in}{-0.041667in}}{\pgfqpoint{0.000000in}{-0.041667in}}%
\pgfpathclose%
\pgfusepath{stroke,fill}%
}%
\begin{pgfscope}%
\pgfsys@transformshift{1.045291in}{2.727070in}%
\pgfsys@useobject{currentmarker}{}%
\end{pgfscope}%
\end{pgfscope}%
\begin{pgfscope}%
\definecolor{textcolor}{rgb}{0.000000,0.000000,0.000000}%
\pgfsetstrokecolor{textcolor}%
\pgfsetfillcolor{textcolor}%
\pgftext[x=1.295291in,y=2.678459in,left,base]{\color{textcolor}\rmfamily\fontsize{10.000000}{12.000000}\selectfont Duration without owner}%
\end{pgfscope}%
\begin{pgfscope}%
\pgfsetrectcap%
\pgfsetroundjoin%
\pgfsetlinewidth{1.505625pt}%
\definecolor{currentstroke}{rgb}{0.172549,0.627451,0.172549}%
\pgfsetstrokecolor{currentstroke}%
\pgfsetdash{}{0pt}%
\pgfpathmoveto{\pgfqpoint{0.906402in}{2.533397in}}%
\pgfpathlineto{\pgfqpoint{1.184180in}{2.533397in}}%
\pgfusepath{stroke}%
\end{pgfscope}%
\begin{pgfscope}%
\pgfsetbuttcap%
\pgfsetroundjoin%
\definecolor{currentfill}{rgb}{0.172549,0.627451,0.172549}%
\pgfsetfillcolor{currentfill}%
\pgfsetlinewidth{1.003750pt}%
\definecolor{currentstroke}{rgb}{0.172549,0.627451,0.172549}%
\pgfsetstrokecolor{currentstroke}%
\pgfsetdash{}{0pt}%
\pgfsys@defobject{currentmarker}{\pgfqpoint{-0.041667in}{-0.041667in}}{\pgfqpoint{0.041667in}{0.041667in}}{%
\pgfpathmoveto{\pgfqpoint{0.000000in}{-0.041667in}}%
\pgfpathcurveto{\pgfqpoint{0.011050in}{-0.041667in}}{\pgfqpoint{0.021649in}{-0.037276in}}{\pgfqpoint{0.029463in}{-0.029463in}}%
\pgfpathcurveto{\pgfqpoint{0.037276in}{-0.021649in}}{\pgfqpoint{0.041667in}{-0.011050in}}{\pgfqpoint{0.041667in}{0.000000in}}%
\pgfpathcurveto{\pgfqpoint{0.041667in}{0.011050in}}{\pgfqpoint{0.037276in}{0.021649in}}{\pgfqpoint{0.029463in}{0.029463in}}%
\pgfpathcurveto{\pgfqpoint{0.021649in}{0.037276in}}{\pgfqpoint{0.011050in}{0.041667in}}{\pgfqpoint{0.000000in}{0.041667in}}%
\pgfpathcurveto{\pgfqpoint{-0.011050in}{0.041667in}}{\pgfqpoint{-0.021649in}{0.037276in}}{\pgfqpoint{-0.029463in}{0.029463in}}%
\pgfpathcurveto{\pgfqpoint{-0.037276in}{0.021649in}}{\pgfqpoint{-0.041667in}{0.011050in}}{\pgfqpoint{-0.041667in}{0.000000in}}%
\pgfpathcurveto{\pgfqpoint{-0.041667in}{-0.011050in}}{\pgfqpoint{-0.037276in}{-0.021649in}}{\pgfqpoint{-0.029463in}{-0.029463in}}%
\pgfpathcurveto{\pgfqpoint{-0.021649in}{-0.037276in}}{\pgfqpoint{-0.011050in}{-0.041667in}}{\pgfqpoint{0.000000in}{-0.041667in}}%
\pgfpathclose%
\pgfusepath{stroke,fill}%
}%
\begin{pgfscope}%
\pgfsys@transformshift{1.045291in}{2.533397in}%
\pgfsys@useobject{currentmarker}{}%
\end{pgfscope}%
\end{pgfscope}%
\begin{pgfscope}%
\definecolor{textcolor}{rgb}{0.000000,0.000000,0.000000}%
\pgfsetstrokecolor{textcolor}%
\pgfsetfillcolor{textcolor}%
\pgftext[x=1.295291in,y=2.484786in,left,base]{\color{textcolor}\rmfamily\fontsize{10.000000}{12.000000}\selectfont Time spent transferring dirty pages}%
\end{pgfscope}%
\begin{pgfscope}%
\pgfsetrectcap%
\pgfsetroundjoin%
\pgfsetlinewidth{1.505625pt}%
\definecolor{currentstroke}{rgb}{0.839216,0.152941,0.156863}%
\pgfsetstrokecolor{currentstroke}%
\pgfsetdash{}{0pt}%
\pgfpathmoveto{\pgfqpoint{0.906402in}{2.339724in}}%
\pgfpathlineto{\pgfqpoint{1.184180in}{2.339724in}}%
\pgfusepath{stroke}%
\end{pgfscope}%
\begin{pgfscope}%
\pgfsetbuttcap%
\pgfsetroundjoin%
\definecolor{currentfill}{rgb}{0.839216,0.152941,0.156863}%
\pgfsetfillcolor{currentfill}%
\pgfsetlinewidth{1.003750pt}%
\definecolor{currentstroke}{rgb}{0.839216,0.152941,0.156863}%
\pgfsetstrokecolor{currentstroke}%
\pgfsetdash{}{0pt}%
\pgfsys@defobject{currentmarker}{\pgfqpoint{-0.041667in}{-0.041667in}}{\pgfqpoint{0.041667in}{0.041667in}}{%
\pgfpathmoveto{\pgfqpoint{0.000000in}{-0.041667in}}%
\pgfpathcurveto{\pgfqpoint{0.011050in}{-0.041667in}}{\pgfqpoint{0.021649in}{-0.037276in}}{\pgfqpoint{0.029463in}{-0.029463in}}%
\pgfpathcurveto{\pgfqpoint{0.037276in}{-0.021649in}}{\pgfqpoint{0.041667in}{-0.011050in}}{\pgfqpoint{0.041667in}{0.000000in}}%
\pgfpathcurveto{\pgfqpoint{0.041667in}{0.011050in}}{\pgfqpoint{0.037276in}{0.021649in}}{\pgfqpoint{0.029463in}{0.029463in}}%
\pgfpathcurveto{\pgfqpoint{0.021649in}{0.037276in}}{\pgfqpoint{0.011050in}{0.041667in}}{\pgfqpoint{0.000000in}{0.041667in}}%
\pgfpathcurveto{\pgfqpoint{-0.011050in}{0.041667in}}{\pgfqpoint{-0.021649in}{0.037276in}}{\pgfqpoint{-0.029463in}{0.029463in}}%
\pgfpathcurveto{\pgfqpoint{-0.037276in}{0.021649in}}{\pgfqpoint{-0.041667in}{0.011050in}}{\pgfqpoint{-0.041667in}{0.000000in}}%
\pgfpathcurveto{\pgfqpoint{-0.041667in}{-0.011050in}}{\pgfqpoint{-0.037276in}{-0.021649in}}{\pgfqpoint{-0.029463in}{-0.029463in}}%
\pgfpathcurveto{\pgfqpoint{-0.021649in}{-0.037276in}}{\pgfqpoint{-0.011050in}{-0.041667in}}{\pgfqpoint{0.000000in}{-0.041667in}}%
\pgfpathclose%
\pgfusepath{stroke,fill}%
}%
\begin{pgfscope}%
\pgfsys@transformshift{1.045291in}{2.339724in}%
\pgfsys@useobject{currentmarker}{}%
\end{pgfscope}%
\end{pgfscope}%
\begin{pgfscope}%
\definecolor{textcolor}{rgb}{0.000000,0.000000,0.000000}%
\pgfsetstrokecolor{textcolor}%
\pgfsetfillcolor{textcolor}%
\pgftext[x=1.295291in,y=2.291113in,left,base]{\color{textcolor}\rmfamily\fontsize{10.000000}{12.000000}\selectfont End to end latency}%
\end{pgfscope}%
\end{pgfpicture}%
\makeatother%
\endgroup%

    \end{center}
    \caption{Migration statistics of a vector with all pages dirty}
    \label{fig:vectorwriteall}
\end{figure}

In this micro-benchmark, we create a vector, dirty all of its pages after the
prefill phase has finished, and then finalize the transfer. Compared to the
clean scenario in \autoref{sec:cleanvec}, we need to spend extra time to
retransfer the dirty pages. As we would expect, the time it takes to turn over
the object ownership is remain unchanged, however based on the usage of the
application, there will be a period during which the object is read-only on the
sender's side and the writes will be delayed on the receiver's side. We
explore the implications of this in \autoref{sec:evalmigfriendly} and
\autoref{sec:evalgenericobj}.


Although the same set of pages are sent/received during the prefill phase and
transferring of the dirty pages, the former takes considerably longer. This
happens because we do not need to pin the object memory to physical memory all
over again on each side.

\section{Case study: Bloom filter}
\label{sec:evalmigfriendly}

\begin{figure}[tp]
    \begin{center}
        %% Creator: Matplotlib, PGF backend
%%
%% To include the figure in your LaTeX document, write
%%   \input{<filename>.pgf}
%%
%% Make sure the required packages are loaded in your preamble
%%   \usepackage{pgf}
%%
%% and, on pdftex
%%   \usepackage[utf8]{inputenc}\DeclareUnicodeCharacter{2212}{-}
%%
%% or, on luatex and xetex
%%   \usepackage{unicode-math}
%%
%% Figures using additional raster images can only be included by \input if
%% they are in the same directory as the main LaTeX file. For loading figures
%% from other directories you can use the `import` package
%%   \usepackage{import}
%%
%% and then include the figures with
%%   \import{<path to file>}{<filename>.pgf}
%%
%% Matplotlib used the following preamble
%%
\begingroup%
\makeatletter%
\begin{pgfpicture}%
\pgfpathrectangle{\pgfpointorigin}{\pgfqpoint{6.251220in}{7.032623in}}%
\pgfusepath{use as bounding box, clip}%
\begin{pgfscope}%
\pgfsetbuttcap%
\pgfsetmiterjoin%
\definecolor{currentfill}{rgb}{1.000000,1.000000,1.000000}%
\pgfsetfillcolor{currentfill}%
\pgfsetlinewidth{0.000000pt}%
\definecolor{currentstroke}{rgb}{1.000000,1.000000,1.000000}%
\pgfsetstrokecolor{currentstroke}%
\pgfsetdash{}{0pt}%
\pgfpathmoveto{\pgfqpoint{0.000000in}{0.000000in}}%
\pgfpathlineto{\pgfqpoint{6.251220in}{0.000000in}}%
\pgfpathlineto{\pgfqpoint{6.251220in}{7.032623in}}%
\pgfpathlineto{\pgfqpoint{0.000000in}{7.032623in}}%
\pgfpathclose%
\pgfusepath{fill}%
\end{pgfscope}%
\begin{pgfscope}%
\pgfsetbuttcap%
\pgfsetmiterjoin%
\definecolor{currentfill}{rgb}{1.000000,1.000000,1.000000}%
\pgfsetfillcolor{currentfill}%
\pgfsetlinewidth{0.000000pt}%
\definecolor{currentstroke}{rgb}{0.000000,0.000000,0.000000}%
\pgfsetstrokecolor{currentstroke}%
\pgfsetstrokeopacity{0.000000}%
\pgfsetdash{}{0pt}%
\pgfpathmoveto{\pgfqpoint{0.781402in}{0.773588in}}%
\pgfpathlineto{\pgfqpoint{5.626098in}{0.773588in}}%
\pgfpathlineto{\pgfqpoint{5.626098in}{6.188708in}}%
\pgfpathlineto{\pgfqpoint{0.781402in}{6.188708in}}%
\pgfpathclose%
\pgfusepath{fill}%
\end{pgfscope}%
\begin{pgfscope}%
\pgfpathrectangle{\pgfqpoint{0.781402in}{0.773588in}}{\pgfqpoint{4.844695in}{5.415119in}}%
\pgfusepath{clip}%
\pgfsetbuttcap%
\pgfsetroundjoin%
\definecolor{currentfill}{rgb}{0.121569,0.466667,0.705882}%
\pgfsetfillcolor{currentfill}%
\pgfsetlinewidth{0.000000pt}%
\definecolor{currentstroke}{rgb}{0.000000,0.000000,0.000000}%
\pgfsetstrokecolor{currentstroke}%
\pgfsetdash{}{0pt}%
\pgfpathmoveto{\pgfqpoint{1.001616in}{0.773588in}}%
\pgfpathlineto{\pgfqpoint{1.001616in}{0.773588in}}%
\pgfpathlineto{\pgfqpoint{1.042993in}{0.773588in}}%
\pgfpathlineto{\pgfqpoint{1.084375in}{0.773588in}}%
\pgfpathlineto{\pgfqpoint{1.125752in}{0.773588in}}%
\pgfpathlineto{\pgfqpoint{1.167091in}{0.773588in}}%
\pgfpathlineto{\pgfqpoint{1.208458in}{0.773588in}}%
\pgfpathlineto{\pgfqpoint{1.249856in}{0.773588in}}%
\pgfpathlineto{\pgfqpoint{1.291280in}{0.773588in}}%
\pgfpathlineto{\pgfqpoint{1.332721in}{0.773588in}}%
\pgfpathlineto{\pgfqpoint{1.374186in}{0.773588in}}%
\pgfpathlineto{\pgfqpoint{1.415629in}{0.773588in}}%
\pgfpathlineto{\pgfqpoint{1.457070in}{0.773588in}}%
\pgfpathlineto{\pgfqpoint{1.498540in}{0.773588in}}%
\pgfpathlineto{\pgfqpoint{1.539972in}{0.773588in}}%
\pgfpathlineto{\pgfqpoint{1.581396in}{0.773588in}}%
\pgfpathlineto{\pgfqpoint{1.622829in}{0.773588in}}%
\pgfpathlineto{\pgfqpoint{1.664297in}{0.773588in}}%
\pgfpathlineto{\pgfqpoint{1.705741in}{0.773588in}}%
\pgfpathlineto{\pgfqpoint{1.747243in}{0.773588in}}%
\pgfpathlineto{\pgfqpoint{1.795378in}{0.773588in}}%
\pgfpathlineto{\pgfqpoint{1.849942in}{0.773588in}}%
\pgfpathlineto{\pgfqpoint{1.905453in}{0.773588in}}%
\pgfpathlineto{\pgfqpoint{1.963640in}{0.773588in}}%
\pgfpathlineto{\pgfqpoint{2.021509in}{0.773588in}}%
\pgfpathlineto{\pgfqpoint{2.078568in}{0.773588in}}%
\pgfpathlineto{\pgfqpoint{2.134427in}{0.773588in}}%
\pgfpathlineto{\pgfqpoint{2.188824in}{0.773588in}}%
\pgfpathlineto{\pgfqpoint{2.244502in}{0.773588in}}%
\pgfpathlineto{\pgfqpoint{2.300442in}{0.773588in}}%
\pgfpathlineto{\pgfqpoint{2.355457in}{0.773588in}}%
\pgfpathlineto{\pgfqpoint{2.411446in}{0.773588in}}%
\pgfpathlineto{\pgfqpoint{2.469564in}{0.773588in}}%
\pgfpathlineto{\pgfqpoint{2.529907in}{0.773588in}}%
\pgfpathlineto{\pgfqpoint{2.589001in}{0.773588in}}%
\pgfpathlineto{\pgfqpoint{2.646993in}{0.773588in}}%
\pgfpathlineto{\pgfqpoint{2.704011in}{0.773588in}}%
\pgfpathlineto{\pgfqpoint{2.760229in}{0.773588in}}%
\pgfpathlineto{\pgfqpoint{2.815997in}{0.773588in}}%
\pgfpathlineto{\pgfqpoint{2.869949in}{0.773588in}}%
\pgfpathlineto{\pgfqpoint{2.921979in}{0.773588in}}%
\pgfpathlineto{\pgfqpoint{2.972423in}{0.773588in}}%
\pgfpathlineto{\pgfqpoint{3.021960in}{0.773588in}}%
\pgfpathlineto{\pgfqpoint{3.070408in}{0.773588in}}%
\pgfpathlineto{\pgfqpoint{3.117833in}{0.773588in}}%
\pgfpathlineto{\pgfqpoint{3.164694in}{0.773588in}}%
\pgfpathlineto{\pgfqpoint{3.210786in}{0.773588in}}%
\pgfpathlineto{\pgfqpoint{3.255737in}{0.773588in}}%
\pgfpathlineto{\pgfqpoint{3.297077in}{0.773588in}}%
\pgfpathlineto{\pgfqpoint{3.337163in}{0.773588in}}%
\pgfpathlineto{\pgfqpoint{3.376505in}{0.773588in}}%
\pgfpathlineto{\pgfqpoint{3.415888in}{0.773588in}}%
\pgfpathlineto{\pgfqpoint{3.455225in}{0.773588in}}%
\pgfpathlineto{\pgfqpoint{3.494517in}{0.773588in}}%
\pgfpathlineto{\pgfqpoint{3.533833in}{0.773588in}}%
\pgfpathlineto{\pgfqpoint{3.573169in}{0.773588in}}%
\pgfpathlineto{\pgfqpoint{3.612456in}{0.773588in}}%
\pgfpathlineto{\pgfqpoint{3.651783in}{0.773588in}}%
\pgfpathlineto{\pgfqpoint{3.691109in}{0.773588in}}%
\pgfpathlineto{\pgfqpoint{3.730402in}{0.773588in}}%
\pgfpathlineto{\pgfqpoint{3.769397in}{0.773588in}}%
\pgfpathlineto{\pgfqpoint{3.806671in}{0.773588in}}%
\pgfpathlineto{\pgfqpoint{3.843637in}{0.773588in}}%
\pgfpathlineto{\pgfqpoint{3.880608in}{0.773588in}}%
\pgfpathlineto{\pgfqpoint{3.917591in}{0.773588in}}%
\pgfpathlineto{\pgfqpoint{3.954566in}{0.773588in}}%
\pgfpathlineto{\pgfqpoint{3.991588in}{0.773588in}}%
\pgfpathlineto{\pgfqpoint{4.028745in}{0.773588in}}%
\pgfpathlineto{\pgfqpoint{4.065634in}{0.773588in}}%
\pgfpathlineto{\pgfqpoint{4.101899in}{0.773588in}}%
\pgfpathlineto{\pgfqpoint{4.138055in}{0.773588in}}%
\pgfpathlineto{\pgfqpoint{4.173852in}{0.773588in}}%
\pgfpathlineto{\pgfqpoint{4.209218in}{0.773588in}}%
\pgfpathlineto{\pgfqpoint{4.244034in}{0.773588in}}%
\pgfpathlineto{\pgfqpoint{4.278250in}{0.773588in}}%
\pgfpathlineto{\pgfqpoint{4.311749in}{0.773588in}}%
\pgfpathlineto{\pgfqpoint{4.344452in}{0.773588in}}%
\pgfpathlineto{\pgfqpoint{4.376239in}{0.773588in}}%
\pgfpathlineto{\pgfqpoint{4.407030in}{0.773588in}}%
\pgfpathlineto{\pgfqpoint{4.436765in}{0.773588in}}%
\pgfpathlineto{\pgfqpoint{4.465281in}{0.773588in}}%
\pgfpathlineto{\pgfqpoint{4.492692in}{0.773588in}}%
\pgfpathlineto{\pgfqpoint{4.518427in}{0.773588in}}%
\pgfpathlineto{\pgfqpoint{4.543529in}{0.773588in}}%
\pgfpathlineto{\pgfqpoint{4.568473in}{0.773588in}}%
\pgfpathlineto{\pgfqpoint{4.593377in}{0.773588in}}%
\pgfpathlineto{\pgfqpoint{4.618263in}{0.773588in}}%
\pgfpathlineto{\pgfqpoint{4.643083in}{0.773588in}}%
\pgfpathlineto{\pgfqpoint{4.667884in}{0.773588in}}%
\pgfpathlineto{\pgfqpoint{4.692637in}{0.773588in}}%
\pgfpathlineto{\pgfqpoint{4.717447in}{0.773588in}}%
\pgfpathlineto{\pgfqpoint{4.742205in}{0.773588in}}%
\pgfpathlineto{\pgfqpoint{4.766979in}{0.773588in}}%
\pgfpathlineto{\pgfqpoint{4.791764in}{0.773588in}}%
\pgfpathlineto{\pgfqpoint{4.816503in}{0.773588in}}%
\pgfpathlineto{\pgfqpoint{4.846540in}{0.773588in}}%
\pgfpathlineto{\pgfqpoint{4.883926in}{0.773588in}}%
\pgfpathlineto{\pgfqpoint{4.921196in}{0.773588in}}%
\pgfpathlineto{\pgfqpoint{4.958462in}{0.773588in}}%
\pgfpathlineto{\pgfqpoint{4.995714in}{0.773588in}}%
\pgfpathlineto{\pgfqpoint{5.033005in}{0.773588in}}%
\pgfpathlineto{\pgfqpoint{5.070304in}{0.773588in}}%
\pgfpathlineto{\pgfqpoint{5.107610in}{0.773588in}}%
\pgfpathlineto{\pgfqpoint{5.144914in}{0.773588in}}%
\pgfpathlineto{\pgfqpoint{5.182186in}{0.773588in}}%
\pgfpathlineto{\pgfqpoint{5.219442in}{0.773588in}}%
\pgfpathlineto{\pgfqpoint{5.256693in}{0.773588in}}%
\pgfpathlineto{\pgfqpoint{5.293977in}{0.773588in}}%
\pgfpathlineto{\pgfqpoint{5.331273in}{0.773588in}}%
\pgfpathlineto{\pgfqpoint{5.368577in}{0.773588in}}%
\pgfpathlineto{\pgfqpoint{5.405885in}{0.773588in}}%
\pgfpathlineto{\pgfqpoint{5.405885in}{2.097942in}}%
\pgfpathlineto{\pgfqpoint{5.405885in}{2.097942in}}%
\pgfpathlineto{\pgfqpoint{5.368577in}{2.097942in}}%
\pgfpathlineto{\pgfqpoint{5.331273in}{2.097942in}}%
\pgfpathlineto{\pgfqpoint{5.293977in}{2.097942in}}%
\pgfpathlineto{\pgfqpoint{5.256693in}{2.097942in}}%
\pgfpathlineto{\pgfqpoint{5.219442in}{2.097942in}}%
\pgfpathlineto{\pgfqpoint{5.182186in}{2.097942in}}%
\pgfpathlineto{\pgfqpoint{5.144914in}{2.097942in}}%
\pgfpathlineto{\pgfqpoint{5.107610in}{2.097942in}}%
\pgfpathlineto{\pgfqpoint{5.070304in}{2.097942in}}%
\pgfpathlineto{\pgfqpoint{5.033005in}{2.097942in}}%
\pgfpathlineto{\pgfqpoint{4.995714in}{2.097942in}}%
\pgfpathlineto{\pgfqpoint{4.958462in}{2.097942in}}%
\pgfpathlineto{\pgfqpoint{4.921196in}{2.097942in}}%
\pgfpathlineto{\pgfqpoint{4.883926in}{2.097942in}}%
\pgfpathlineto{\pgfqpoint{4.846540in}{2.097641in}}%
\pgfpathlineto{\pgfqpoint{4.816503in}{2.096840in}}%
\pgfpathlineto{\pgfqpoint{4.791764in}{2.095655in}}%
\pgfpathlineto{\pgfqpoint{4.766979in}{2.094194in}}%
\pgfpathlineto{\pgfqpoint{4.742205in}{2.095755in}}%
\pgfpathlineto{\pgfqpoint{4.717447in}{2.097702in}}%
\pgfpathlineto{\pgfqpoint{4.692637in}{2.092538in}}%
\pgfpathlineto{\pgfqpoint{4.667884in}{2.093673in}}%
\pgfpathlineto{\pgfqpoint{4.643083in}{2.094865in}}%
\pgfpathlineto{\pgfqpoint{4.618263in}{2.096784in}}%
\pgfpathlineto{\pgfqpoint{4.593377in}{2.097212in}}%
\pgfpathlineto{\pgfqpoint{4.568473in}{2.095802in}}%
\pgfpathlineto{\pgfqpoint{4.543529in}{2.096725in}}%
\pgfpathlineto{\pgfqpoint{4.518427in}{2.079197in}}%
\pgfpathlineto{\pgfqpoint{4.492692in}{1.982241in}}%
\pgfpathlineto{\pgfqpoint{4.465281in}{1.939656in}}%
\pgfpathlineto{\pgfqpoint{4.436765in}{1.835427in}}%
\pgfpathlineto{\pgfqpoint{4.407030in}{1.664611in}}%
\pgfpathlineto{\pgfqpoint{4.376239in}{1.526622in}}%
\pgfpathlineto{\pgfqpoint{4.344452in}{1.375846in}}%
\pgfpathlineto{\pgfqpoint{4.311749in}{1.253505in}}%
\pgfpathlineto{\pgfqpoint{4.278250in}{1.148913in}}%
\pgfpathlineto{\pgfqpoint{4.244034in}{1.061085in}}%
\pgfpathlineto{\pgfqpoint{4.209218in}{0.963881in}}%
\pgfpathlineto{\pgfqpoint{4.173852in}{0.905273in}}%
\pgfpathlineto{\pgfqpoint{4.138055in}{0.865780in}}%
\pgfpathlineto{\pgfqpoint{4.101899in}{0.818043in}}%
\pgfpathlineto{\pgfqpoint{4.065634in}{0.773588in}}%
\pgfpathlineto{\pgfqpoint{4.028745in}{0.773588in}}%
\pgfpathlineto{\pgfqpoint{3.991588in}{0.773588in}}%
\pgfpathlineto{\pgfqpoint{3.954566in}{0.773588in}}%
\pgfpathlineto{\pgfqpoint{3.917591in}{0.773588in}}%
\pgfpathlineto{\pgfqpoint{3.880608in}{0.773588in}}%
\pgfpathlineto{\pgfqpoint{3.843637in}{0.773588in}}%
\pgfpathlineto{\pgfqpoint{3.806671in}{0.773588in}}%
\pgfpathlineto{\pgfqpoint{3.769397in}{0.773588in}}%
\pgfpathlineto{\pgfqpoint{3.730402in}{0.773588in}}%
\pgfpathlineto{\pgfqpoint{3.691109in}{0.773588in}}%
\pgfpathlineto{\pgfqpoint{3.651783in}{0.773588in}}%
\pgfpathlineto{\pgfqpoint{3.612456in}{0.773588in}}%
\pgfpathlineto{\pgfqpoint{3.573169in}{0.773588in}}%
\pgfpathlineto{\pgfqpoint{3.533833in}{0.773588in}}%
\pgfpathlineto{\pgfqpoint{3.494517in}{0.773588in}}%
\pgfpathlineto{\pgfqpoint{3.455225in}{0.773588in}}%
\pgfpathlineto{\pgfqpoint{3.415888in}{0.773588in}}%
\pgfpathlineto{\pgfqpoint{3.376505in}{0.773588in}}%
\pgfpathlineto{\pgfqpoint{3.337163in}{0.773588in}}%
\pgfpathlineto{\pgfqpoint{3.297077in}{0.773588in}}%
\pgfpathlineto{\pgfqpoint{3.255737in}{0.773588in}}%
\pgfpathlineto{\pgfqpoint{3.210786in}{0.773588in}}%
\pgfpathlineto{\pgfqpoint{3.164694in}{0.773588in}}%
\pgfpathlineto{\pgfqpoint{3.117833in}{0.773588in}}%
\pgfpathlineto{\pgfqpoint{3.070408in}{0.773588in}}%
\pgfpathlineto{\pgfqpoint{3.021960in}{0.773588in}}%
\pgfpathlineto{\pgfqpoint{2.972423in}{0.773588in}}%
\pgfpathlineto{\pgfqpoint{2.921979in}{0.773588in}}%
\pgfpathlineto{\pgfqpoint{2.869949in}{0.773588in}}%
\pgfpathlineto{\pgfqpoint{2.815997in}{0.773588in}}%
\pgfpathlineto{\pgfqpoint{2.760229in}{0.773588in}}%
\pgfpathlineto{\pgfqpoint{2.704011in}{0.773588in}}%
\pgfpathlineto{\pgfqpoint{2.646993in}{0.773588in}}%
\pgfpathlineto{\pgfqpoint{2.589001in}{0.773588in}}%
\pgfpathlineto{\pgfqpoint{2.529907in}{0.773588in}}%
\pgfpathlineto{\pgfqpoint{2.469564in}{0.773588in}}%
\pgfpathlineto{\pgfqpoint{2.411446in}{0.773588in}}%
\pgfpathlineto{\pgfqpoint{2.355457in}{0.773588in}}%
\pgfpathlineto{\pgfqpoint{2.300442in}{0.773588in}}%
\pgfpathlineto{\pgfqpoint{2.244502in}{0.773588in}}%
\pgfpathlineto{\pgfqpoint{2.188824in}{0.773588in}}%
\pgfpathlineto{\pgfqpoint{2.134427in}{0.773588in}}%
\pgfpathlineto{\pgfqpoint{2.078568in}{0.773588in}}%
\pgfpathlineto{\pgfqpoint{2.021509in}{0.773588in}}%
\pgfpathlineto{\pgfqpoint{1.963640in}{0.773588in}}%
\pgfpathlineto{\pgfqpoint{1.905453in}{0.773588in}}%
\pgfpathlineto{\pgfqpoint{1.849942in}{0.773588in}}%
\pgfpathlineto{\pgfqpoint{1.795378in}{0.773588in}}%
\pgfpathlineto{\pgfqpoint{1.747243in}{0.773588in}}%
\pgfpathlineto{\pgfqpoint{1.705741in}{0.773588in}}%
\pgfpathlineto{\pgfqpoint{1.664297in}{0.773588in}}%
\pgfpathlineto{\pgfqpoint{1.622829in}{0.773588in}}%
\pgfpathlineto{\pgfqpoint{1.581396in}{0.773588in}}%
\pgfpathlineto{\pgfqpoint{1.539972in}{0.773588in}}%
\pgfpathlineto{\pgfqpoint{1.498540in}{0.773588in}}%
\pgfpathlineto{\pgfqpoint{1.457070in}{0.773588in}}%
\pgfpathlineto{\pgfqpoint{1.415629in}{0.773588in}}%
\pgfpathlineto{\pgfqpoint{1.374186in}{0.773588in}}%
\pgfpathlineto{\pgfqpoint{1.332721in}{0.773588in}}%
\pgfpathlineto{\pgfqpoint{1.291280in}{0.773588in}}%
\pgfpathlineto{\pgfqpoint{1.249856in}{0.773588in}}%
\pgfpathlineto{\pgfqpoint{1.208458in}{0.773588in}}%
\pgfpathlineto{\pgfqpoint{1.167091in}{0.773588in}}%
\pgfpathlineto{\pgfqpoint{1.125752in}{0.773588in}}%
\pgfpathlineto{\pgfqpoint{1.084375in}{0.773588in}}%
\pgfpathlineto{\pgfqpoint{1.042993in}{0.773588in}}%
\pgfpathlineto{\pgfqpoint{1.001616in}{0.773588in}}%
\pgfpathclose%
\pgfusepath{fill}%
\end{pgfscope}%
\begin{pgfscope}%
\pgfpathrectangle{\pgfqpoint{0.781402in}{0.773588in}}{\pgfqpoint{4.844695in}{5.415119in}}%
\pgfusepath{clip}%
\pgfsetbuttcap%
\pgfsetroundjoin%
\definecolor{currentfill}{rgb}{1.000000,0.498039,0.054902}%
\pgfsetfillcolor{currentfill}%
\pgfsetlinewidth{0.000000pt}%
\definecolor{currentstroke}{rgb}{0.000000,0.000000,0.000000}%
\pgfsetstrokecolor{currentstroke}%
\pgfsetdash{}{0pt}%
\pgfpathmoveto{\pgfqpoint{1.001616in}{0.773588in}}%
\pgfpathlineto{\pgfqpoint{1.001616in}{0.773588in}}%
\pgfpathlineto{\pgfqpoint{1.042993in}{0.773588in}}%
\pgfpathlineto{\pgfqpoint{1.084375in}{0.773588in}}%
\pgfpathlineto{\pgfqpoint{1.125752in}{0.773588in}}%
\pgfpathlineto{\pgfqpoint{1.167091in}{0.773588in}}%
\pgfpathlineto{\pgfqpoint{1.208458in}{0.773588in}}%
\pgfpathlineto{\pgfqpoint{1.249856in}{0.773588in}}%
\pgfpathlineto{\pgfqpoint{1.291280in}{0.773588in}}%
\pgfpathlineto{\pgfqpoint{1.332721in}{0.773588in}}%
\pgfpathlineto{\pgfqpoint{1.374186in}{0.773588in}}%
\pgfpathlineto{\pgfqpoint{1.415629in}{0.773588in}}%
\pgfpathlineto{\pgfqpoint{1.457070in}{0.773588in}}%
\pgfpathlineto{\pgfqpoint{1.498540in}{0.773588in}}%
\pgfpathlineto{\pgfqpoint{1.539972in}{0.773588in}}%
\pgfpathlineto{\pgfqpoint{1.581396in}{0.773588in}}%
\pgfpathlineto{\pgfqpoint{1.622829in}{0.773588in}}%
\pgfpathlineto{\pgfqpoint{1.664297in}{0.773588in}}%
\pgfpathlineto{\pgfqpoint{1.705741in}{0.773588in}}%
\pgfpathlineto{\pgfqpoint{1.747243in}{0.773588in}}%
\pgfpathlineto{\pgfqpoint{1.795378in}{0.773588in}}%
\pgfpathlineto{\pgfqpoint{1.849942in}{0.773588in}}%
\pgfpathlineto{\pgfqpoint{1.905453in}{0.773588in}}%
\pgfpathlineto{\pgfqpoint{1.963640in}{0.773588in}}%
\pgfpathlineto{\pgfqpoint{2.021509in}{0.773588in}}%
\pgfpathlineto{\pgfqpoint{2.078568in}{0.773588in}}%
\pgfpathlineto{\pgfqpoint{2.134427in}{0.773588in}}%
\pgfpathlineto{\pgfqpoint{2.188824in}{0.773588in}}%
\pgfpathlineto{\pgfqpoint{2.244502in}{0.773588in}}%
\pgfpathlineto{\pgfqpoint{2.300442in}{0.773588in}}%
\pgfpathlineto{\pgfqpoint{2.355457in}{0.773588in}}%
\pgfpathlineto{\pgfqpoint{2.411446in}{0.773588in}}%
\pgfpathlineto{\pgfqpoint{2.469564in}{0.773588in}}%
\pgfpathlineto{\pgfqpoint{2.529907in}{0.773588in}}%
\pgfpathlineto{\pgfqpoint{2.589001in}{0.773588in}}%
\pgfpathlineto{\pgfqpoint{2.646993in}{0.773588in}}%
\pgfpathlineto{\pgfqpoint{2.704011in}{0.773588in}}%
\pgfpathlineto{\pgfqpoint{2.760229in}{0.773588in}}%
\pgfpathlineto{\pgfqpoint{2.815997in}{0.773588in}}%
\pgfpathlineto{\pgfqpoint{2.869949in}{0.773588in}}%
\pgfpathlineto{\pgfqpoint{2.921979in}{0.773588in}}%
\pgfpathlineto{\pgfqpoint{2.972423in}{0.773588in}}%
\pgfpathlineto{\pgfqpoint{3.021960in}{0.773588in}}%
\pgfpathlineto{\pgfqpoint{3.070408in}{0.773588in}}%
\pgfpathlineto{\pgfqpoint{3.117833in}{0.773588in}}%
\pgfpathlineto{\pgfqpoint{3.164694in}{0.773588in}}%
\pgfpathlineto{\pgfqpoint{3.210786in}{0.773588in}}%
\pgfpathlineto{\pgfqpoint{3.255737in}{0.773588in}}%
\pgfpathlineto{\pgfqpoint{3.297077in}{0.773588in}}%
\pgfpathlineto{\pgfqpoint{3.337163in}{0.773588in}}%
\pgfpathlineto{\pgfqpoint{3.376505in}{0.773588in}}%
\pgfpathlineto{\pgfqpoint{3.415888in}{0.773588in}}%
\pgfpathlineto{\pgfqpoint{3.455225in}{0.773588in}}%
\pgfpathlineto{\pgfqpoint{3.494517in}{0.773588in}}%
\pgfpathlineto{\pgfqpoint{3.533833in}{0.773588in}}%
\pgfpathlineto{\pgfqpoint{3.573169in}{0.773588in}}%
\pgfpathlineto{\pgfqpoint{3.612456in}{0.773588in}}%
\pgfpathlineto{\pgfqpoint{3.651783in}{0.773588in}}%
\pgfpathlineto{\pgfqpoint{3.691109in}{0.773588in}}%
\pgfpathlineto{\pgfqpoint{3.730402in}{0.773588in}}%
\pgfpathlineto{\pgfqpoint{3.769397in}{0.773588in}}%
\pgfpathlineto{\pgfqpoint{3.806671in}{0.773588in}}%
\pgfpathlineto{\pgfqpoint{3.843637in}{0.773588in}}%
\pgfpathlineto{\pgfqpoint{3.880608in}{0.773588in}}%
\pgfpathlineto{\pgfqpoint{3.917591in}{0.773588in}}%
\pgfpathlineto{\pgfqpoint{3.954566in}{0.773588in}}%
\pgfpathlineto{\pgfqpoint{3.991588in}{0.773588in}}%
\pgfpathlineto{\pgfqpoint{4.028745in}{0.773588in}}%
\pgfpathlineto{\pgfqpoint{4.065634in}{0.773588in}}%
\pgfpathlineto{\pgfqpoint{4.101899in}{0.818043in}}%
\pgfpathlineto{\pgfqpoint{4.138055in}{0.865780in}}%
\pgfpathlineto{\pgfqpoint{4.173852in}{0.905273in}}%
\pgfpathlineto{\pgfqpoint{4.209218in}{0.963881in}}%
\pgfpathlineto{\pgfqpoint{4.244034in}{1.061085in}}%
\pgfpathlineto{\pgfqpoint{4.278250in}{1.148913in}}%
\pgfpathlineto{\pgfqpoint{4.311749in}{1.253505in}}%
\pgfpathlineto{\pgfqpoint{4.344452in}{1.375846in}}%
\pgfpathlineto{\pgfqpoint{4.376239in}{1.526622in}}%
\pgfpathlineto{\pgfqpoint{4.407030in}{1.664611in}}%
\pgfpathlineto{\pgfqpoint{4.436765in}{1.835427in}}%
\pgfpathlineto{\pgfqpoint{4.465281in}{1.939656in}}%
\pgfpathlineto{\pgfqpoint{4.492692in}{1.982241in}}%
\pgfpathlineto{\pgfqpoint{4.518427in}{2.079197in}}%
\pgfpathlineto{\pgfqpoint{4.543529in}{2.096725in}}%
\pgfpathlineto{\pgfqpoint{4.568473in}{2.095802in}}%
\pgfpathlineto{\pgfqpoint{4.593377in}{2.097212in}}%
\pgfpathlineto{\pgfqpoint{4.618263in}{2.096784in}}%
\pgfpathlineto{\pgfqpoint{4.643083in}{2.094865in}}%
\pgfpathlineto{\pgfqpoint{4.667884in}{2.093673in}}%
\pgfpathlineto{\pgfqpoint{4.692637in}{2.092538in}}%
\pgfpathlineto{\pgfqpoint{4.717447in}{2.097702in}}%
\pgfpathlineto{\pgfqpoint{4.742205in}{2.095755in}}%
\pgfpathlineto{\pgfqpoint{4.766979in}{2.094194in}}%
\pgfpathlineto{\pgfqpoint{4.791764in}{2.095655in}}%
\pgfpathlineto{\pgfqpoint{4.816503in}{2.096840in}}%
\pgfpathlineto{\pgfqpoint{4.846540in}{2.097641in}}%
\pgfpathlineto{\pgfqpoint{4.883926in}{2.097942in}}%
\pgfpathlineto{\pgfqpoint{4.921196in}{2.097942in}}%
\pgfpathlineto{\pgfqpoint{4.958462in}{2.097942in}}%
\pgfpathlineto{\pgfqpoint{4.995714in}{2.097942in}}%
\pgfpathlineto{\pgfqpoint{5.033005in}{2.097942in}}%
\pgfpathlineto{\pgfqpoint{5.070304in}{2.097942in}}%
\pgfpathlineto{\pgfqpoint{5.107610in}{2.097942in}}%
\pgfpathlineto{\pgfqpoint{5.144914in}{2.097942in}}%
\pgfpathlineto{\pgfqpoint{5.182186in}{2.097942in}}%
\pgfpathlineto{\pgfqpoint{5.219442in}{2.097942in}}%
\pgfpathlineto{\pgfqpoint{5.256693in}{2.097942in}}%
\pgfpathlineto{\pgfqpoint{5.293977in}{2.097942in}}%
\pgfpathlineto{\pgfqpoint{5.331273in}{2.097942in}}%
\pgfpathlineto{\pgfqpoint{5.368577in}{2.097942in}}%
\pgfpathlineto{\pgfqpoint{5.405885in}{2.097942in}}%
\pgfpathlineto{\pgfqpoint{5.405885in}{3.363104in}}%
\pgfpathlineto{\pgfqpoint{5.405885in}{3.363104in}}%
\pgfpathlineto{\pgfqpoint{5.368577in}{3.363104in}}%
\pgfpathlineto{\pgfqpoint{5.331273in}{3.363104in}}%
\pgfpathlineto{\pgfqpoint{5.293977in}{3.363104in}}%
\pgfpathlineto{\pgfqpoint{5.256693in}{3.363104in}}%
\pgfpathlineto{\pgfqpoint{5.219442in}{3.363104in}}%
\pgfpathlineto{\pgfqpoint{5.182186in}{3.363104in}}%
\pgfpathlineto{\pgfqpoint{5.144914in}{3.363104in}}%
\pgfpathlineto{\pgfqpoint{5.107610in}{3.363104in}}%
\pgfpathlineto{\pgfqpoint{5.070304in}{3.363104in}}%
\pgfpathlineto{\pgfqpoint{5.033005in}{3.363104in}}%
\pgfpathlineto{\pgfqpoint{4.995714in}{3.363104in}}%
\pgfpathlineto{\pgfqpoint{4.958462in}{3.363104in}}%
\pgfpathlineto{\pgfqpoint{4.921196in}{3.363104in}}%
\pgfpathlineto{\pgfqpoint{4.883926in}{3.363104in}}%
\pgfpathlineto{\pgfqpoint{4.846540in}{3.362461in}}%
\pgfpathlineto{\pgfqpoint{4.816503in}{3.359250in}}%
\pgfpathlineto{\pgfqpoint{4.791764in}{3.355505in}}%
\pgfpathlineto{\pgfqpoint{4.766979in}{3.352900in}}%
\pgfpathlineto{\pgfqpoint{4.742205in}{3.354473in}}%
\pgfpathlineto{\pgfqpoint{4.717447in}{3.354720in}}%
\pgfpathlineto{\pgfqpoint{4.692637in}{3.346168in}}%
\pgfpathlineto{\pgfqpoint{4.667884in}{3.343932in}}%
\pgfpathlineto{\pgfqpoint{4.643083in}{3.337346in}}%
\pgfpathlineto{\pgfqpoint{4.618263in}{3.333835in}}%
\pgfpathlineto{\pgfqpoint{4.593377in}{3.325424in}}%
\pgfpathlineto{\pgfqpoint{4.568473in}{3.307655in}}%
\pgfpathlineto{\pgfqpoint{4.543529in}{3.285500in}}%
\pgfpathlineto{\pgfqpoint{4.518427in}{3.212436in}}%
\pgfpathlineto{\pgfqpoint{4.492692in}{2.759135in}}%
\pgfpathlineto{\pgfqpoint{4.465281in}{2.409996in}}%
\pgfpathlineto{\pgfqpoint{4.436765in}{2.147956in}}%
\pgfpathlineto{\pgfqpoint{4.407030in}{1.886287in}}%
\pgfpathlineto{\pgfqpoint{4.376239in}{1.696774in}}%
\pgfpathlineto{\pgfqpoint{4.344452in}{1.505100in}}%
\pgfpathlineto{\pgfqpoint{4.311749in}{1.354290in}}%
\pgfpathlineto{\pgfqpoint{4.278250in}{1.229140in}}%
\pgfpathlineto{\pgfqpoint{4.244034in}{1.123699in}}%
\pgfpathlineto{\pgfqpoint{4.209218in}{1.012308in}}%
\pgfpathlineto{\pgfqpoint{4.173852in}{0.943149in}}%
\pgfpathlineto{\pgfqpoint{4.138055in}{0.879063in}}%
\pgfpathlineto{\pgfqpoint{4.101899in}{0.818043in}}%
\pgfpathlineto{\pgfqpoint{4.065634in}{0.773588in}}%
\pgfpathlineto{\pgfqpoint{4.028745in}{0.773588in}}%
\pgfpathlineto{\pgfqpoint{3.991588in}{0.773588in}}%
\pgfpathlineto{\pgfqpoint{3.954566in}{0.773588in}}%
\pgfpathlineto{\pgfqpoint{3.917591in}{0.773588in}}%
\pgfpathlineto{\pgfqpoint{3.880608in}{0.773588in}}%
\pgfpathlineto{\pgfqpoint{3.843637in}{0.773588in}}%
\pgfpathlineto{\pgfqpoint{3.806671in}{0.773588in}}%
\pgfpathlineto{\pgfqpoint{3.769397in}{0.773588in}}%
\pgfpathlineto{\pgfqpoint{3.730402in}{0.773588in}}%
\pgfpathlineto{\pgfqpoint{3.691109in}{0.773588in}}%
\pgfpathlineto{\pgfqpoint{3.651783in}{0.773588in}}%
\pgfpathlineto{\pgfqpoint{3.612456in}{0.773588in}}%
\pgfpathlineto{\pgfqpoint{3.573169in}{0.773588in}}%
\pgfpathlineto{\pgfqpoint{3.533833in}{0.773588in}}%
\pgfpathlineto{\pgfqpoint{3.494517in}{0.773588in}}%
\pgfpathlineto{\pgfqpoint{3.455225in}{0.773588in}}%
\pgfpathlineto{\pgfqpoint{3.415888in}{0.773588in}}%
\pgfpathlineto{\pgfqpoint{3.376505in}{0.773588in}}%
\pgfpathlineto{\pgfqpoint{3.337163in}{0.773588in}}%
\pgfpathlineto{\pgfqpoint{3.297077in}{0.773588in}}%
\pgfpathlineto{\pgfqpoint{3.255737in}{0.773588in}}%
\pgfpathlineto{\pgfqpoint{3.210786in}{0.773588in}}%
\pgfpathlineto{\pgfqpoint{3.164694in}{0.773588in}}%
\pgfpathlineto{\pgfqpoint{3.117833in}{0.773588in}}%
\pgfpathlineto{\pgfqpoint{3.070408in}{0.773588in}}%
\pgfpathlineto{\pgfqpoint{3.021960in}{0.773588in}}%
\pgfpathlineto{\pgfqpoint{2.972423in}{0.773588in}}%
\pgfpathlineto{\pgfqpoint{2.921979in}{0.773588in}}%
\pgfpathlineto{\pgfqpoint{2.869949in}{0.773588in}}%
\pgfpathlineto{\pgfqpoint{2.815997in}{0.773588in}}%
\pgfpathlineto{\pgfqpoint{2.760229in}{0.773588in}}%
\pgfpathlineto{\pgfqpoint{2.704011in}{0.773588in}}%
\pgfpathlineto{\pgfqpoint{2.646993in}{0.773588in}}%
\pgfpathlineto{\pgfqpoint{2.589001in}{0.773588in}}%
\pgfpathlineto{\pgfqpoint{2.529907in}{0.773588in}}%
\pgfpathlineto{\pgfqpoint{2.469564in}{0.773588in}}%
\pgfpathlineto{\pgfqpoint{2.411446in}{0.773588in}}%
\pgfpathlineto{\pgfqpoint{2.355457in}{0.773588in}}%
\pgfpathlineto{\pgfqpoint{2.300442in}{0.773588in}}%
\pgfpathlineto{\pgfqpoint{2.244502in}{0.773588in}}%
\pgfpathlineto{\pgfqpoint{2.188824in}{0.773588in}}%
\pgfpathlineto{\pgfqpoint{2.134427in}{0.773588in}}%
\pgfpathlineto{\pgfqpoint{2.078568in}{0.773588in}}%
\pgfpathlineto{\pgfqpoint{2.021509in}{0.773588in}}%
\pgfpathlineto{\pgfqpoint{1.963640in}{0.773588in}}%
\pgfpathlineto{\pgfqpoint{1.905453in}{0.773588in}}%
\pgfpathlineto{\pgfqpoint{1.849942in}{0.773588in}}%
\pgfpathlineto{\pgfqpoint{1.795378in}{0.773588in}}%
\pgfpathlineto{\pgfqpoint{1.747243in}{0.773588in}}%
\pgfpathlineto{\pgfqpoint{1.705741in}{0.773588in}}%
\pgfpathlineto{\pgfqpoint{1.664297in}{0.773588in}}%
\pgfpathlineto{\pgfqpoint{1.622829in}{0.773588in}}%
\pgfpathlineto{\pgfqpoint{1.581396in}{0.773588in}}%
\pgfpathlineto{\pgfqpoint{1.539972in}{0.773588in}}%
\pgfpathlineto{\pgfqpoint{1.498540in}{0.773588in}}%
\pgfpathlineto{\pgfqpoint{1.457070in}{0.773588in}}%
\pgfpathlineto{\pgfqpoint{1.415629in}{0.773588in}}%
\pgfpathlineto{\pgfqpoint{1.374186in}{0.773588in}}%
\pgfpathlineto{\pgfqpoint{1.332721in}{0.773588in}}%
\pgfpathlineto{\pgfqpoint{1.291280in}{0.773588in}}%
\pgfpathlineto{\pgfqpoint{1.249856in}{0.773588in}}%
\pgfpathlineto{\pgfqpoint{1.208458in}{0.773588in}}%
\pgfpathlineto{\pgfqpoint{1.167091in}{0.773588in}}%
\pgfpathlineto{\pgfqpoint{1.125752in}{0.773588in}}%
\pgfpathlineto{\pgfqpoint{1.084375in}{0.773588in}}%
\pgfpathlineto{\pgfqpoint{1.042993in}{0.773588in}}%
\pgfpathlineto{\pgfqpoint{1.001616in}{0.773588in}}%
\pgfpathclose%
\pgfusepath{fill}%
\end{pgfscope}%
\begin{pgfscope}%
\pgfpathrectangle{\pgfqpoint{0.781402in}{0.773588in}}{\pgfqpoint{4.844695in}{5.415119in}}%
\pgfusepath{clip}%
\pgfsetbuttcap%
\pgfsetroundjoin%
\definecolor{currentfill}{rgb}{0.172549,0.627451,0.172549}%
\pgfsetfillcolor{currentfill}%
\pgfsetlinewidth{0.000000pt}%
\definecolor{currentstroke}{rgb}{0.000000,0.000000,0.000000}%
\pgfsetstrokecolor{currentstroke}%
\pgfsetdash{}{0pt}%
\pgfpathmoveto{\pgfqpoint{1.001616in}{1.341197in}}%
\pgfpathlineto{\pgfqpoint{1.001616in}{0.773588in}}%
\pgfpathlineto{\pgfqpoint{1.042993in}{0.773588in}}%
\pgfpathlineto{\pgfqpoint{1.084375in}{0.773588in}}%
\pgfpathlineto{\pgfqpoint{1.125752in}{0.773588in}}%
\pgfpathlineto{\pgfqpoint{1.167091in}{0.773588in}}%
\pgfpathlineto{\pgfqpoint{1.208458in}{0.773588in}}%
\pgfpathlineto{\pgfqpoint{1.249856in}{0.773588in}}%
\pgfpathlineto{\pgfqpoint{1.291280in}{0.773588in}}%
\pgfpathlineto{\pgfqpoint{1.332721in}{0.773588in}}%
\pgfpathlineto{\pgfqpoint{1.374186in}{0.773588in}}%
\pgfpathlineto{\pgfqpoint{1.415629in}{0.773588in}}%
\pgfpathlineto{\pgfqpoint{1.457070in}{0.773588in}}%
\pgfpathlineto{\pgfqpoint{1.498540in}{0.773588in}}%
\pgfpathlineto{\pgfqpoint{1.539972in}{0.773588in}}%
\pgfpathlineto{\pgfqpoint{1.581396in}{0.773588in}}%
\pgfpathlineto{\pgfqpoint{1.622829in}{0.773588in}}%
\pgfpathlineto{\pgfqpoint{1.664297in}{0.773588in}}%
\pgfpathlineto{\pgfqpoint{1.705741in}{0.773588in}}%
\pgfpathlineto{\pgfqpoint{1.747243in}{0.773588in}}%
\pgfpathlineto{\pgfqpoint{1.795378in}{0.773588in}}%
\pgfpathlineto{\pgfqpoint{1.849942in}{0.773588in}}%
\pgfpathlineto{\pgfqpoint{1.905453in}{0.773588in}}%
\pgfpathlineto{\pgfqpoint{1.963640in}{0.773588in}}%
\pgfpathlineto{\pgfqpoint{2.021509in}{0.773588in}}%
\pgfpathlineto{\pgfqpoint{2.078568in}{0.773588in}}%
\pgfpathlineto{\pgfqpoint{2.134427in}{0.773588in}}%
\pgfpathlineto{\pgfqpoint{2.188824in}{0.773588in}}%
\pgfpathlineto{\pgfqpoint{2.244502in}{0.773588in}}%
\pgfpathlineto{\pgfqpoint{2.300442in}{0.773588in}}%
\pgfpathlineto{\pgfqpoint{2.355457in}{0.773588in}}%
\pgfpathlineto{\pgfqpoint{2.411446in}{0.773588in}}%
\pgfpathlineto{\pgfqpoint{2.469564in}{0.773588in}}%
\pgfpathlineto{\pgfqpoint{2.529907in}{0.773588in}}%
\pgfpathlineto{\pgfqpoint{2.589001in}{0.773588in}}%
\pgfpathlineto{\pgfqpoint{2.646993in}{0.773588in}}%
\pgfpathlineto{\pgfqpoint{2.704011in}{0.773588in}}%
\pgfpathlineto{\pgfqpoint{2.760229in}{0.773588in}}%
\pgfpathlineto{\pgfqpoint{2.815997in}{0.773588in}}%
\pgfpathlineto{\pgfqpoint{2.869949in}{0.773588in}}%
\pgfpathlineto{\pgfqpoint{2.921979in}{0.773588in}}%
\pgfpathlineto{\pgfqpoint{2.972423in}{0.773588in}}%
\pgfpathlineto{\pgfqpoint{3.021960in}{0.773588in}}%
\pgfpathlineto{\pgfqpoint{3.070408in}{0.773588in}}%
\pgfpathlineto{\pgfqpoint{3.117833in}{0.773588in}}%
\pgfpathlineto{\pgfqpoint{3.164694in}{0.773588in}}%
\pgfpathlineto{\pgfqpoint{3.210786in}{0.773588in}}%
\pgfpathlineto{\pgfqpoint{3.255737in}{0.773588in}}%
\pgfpathlineto{\pgfqpoint{3.297077in}{0.773588in}}%
\pgfpathlineto{\pgfqpoint{3.337163in}{0.773588in}}%
\pgfpathlineto{\pgfqpoint{3.376505in}{0.773588in}}%
\pgfpathlineto{\pgfqpoint{3.415888in}{0.773588in}}%
\pgfpathlineto{\pgfqpoint{3.455225in}{0.773588in}}%
\pgfpathlineto{\pgfqpoint{3.494517in}{0.773588in}}%
\pgfpathlineto{\pgfqpoint{3.533833in}{0.773588in}}%
\pgfpathlineto{\pgfqpoint{3.573169in}{0.773588in}}%
\pgfpathlineto{\pgfqpoint{3.612456in}{0.773588in}}%
\pgfpathlineto{\pgfqpoint{3.651783in}{0.773588in}}%
\pgfpathlineto{\pgfqpoint{3.691109in}{0.773588in}}%
\pgfpathlineto{\pgfqpoint{3.730402in}{0.773588in}}%
\pgfpathlineto{\pgfqpoint{3.769397in}{0.773588in}}%
\pgfpathlineto{\pgfqpoint{3.806671in}{0.773588in}}%
\pgfpathlineto{\pgfqpoint{3.843637in}{0.773588in}}%
\pgfpathlineto{\pgfqpoint{3.880608in}{0.773588in}}%
\pgfpathlineto{\pgfqpoint{3.917591in}{0.773588in}}%
\pgfpathlineto{\pgfqpoint{3.954566in}{0.773588in}}%
\pgfpathlineto{\pgfqpoint{3.991588in}{0.773588in}}%
\pgfpathlineto{\pgfqpoint{4.028745in}{0.773588in}}%
\pgfpathlineto{\pgfqpoint{4.065634in}{0.773588in}}%
\pgfpathlineto{\pgfqpoint{4.101899in}{0.818043in}}%
\pgfpathlineto{\pgfqpoint{4.138055in}{0.879063in}}%
\pgfpathlineto{\pgfqpoint{4.173852in}{0.943149in}}%
\pgfpathlineto{\pgfqpoint{4.209218in}{1.012308in}}%
\pgfpathlineto{\pgfqpoint{4.244034in}{1.123699in}}%
\pgfpathlineto{\pgfqpoint{4.278250in}{1.229140in}}%
\pgfpathlineto{\pgfqpoint{4.311749in}{1.354290in}}%
\pgfpathlineto{\pgfqpoint{4.344452in}{1.505100in}}%
\pgfpathlineto{\pgfqpoint{4.376239in}{1.696774in}}%
\pgfpathlineto{\pgfqpoint{4.407030in}{1.886287in}}%
\pgfpathlineto{\pgfqpoint{4.436765in}{2.147956in}}%
\pgfpathlineto{\pgfqpoint{4.465281in}{2.409996in}}%
\pgfpathlineto{\pgfqpoint{4.492692in}{2.759135in}}%
\pgfpathlineto{\pgfqpoint{4.518427in}{3.212436in}}%
\pgfpathlineto{\pgfqpoint{4.543529in}{3.285500in}}%
\pgfpathlineto{\pgfqpoint{4.568473in}{3.307655in}}%
\pgfpathlineto{\pgfqpoint{4.593377in}{3.325424in}}%
\pgfpathlineto{\pgfqpoint{4.618263in}{3.333835in}}%
\pgfpathlineto{\pgfqpoint{4.643083in}{3.337346in}}%
\pgfpathlineto{\pgfqpoint{4.667884in}{3.343932in}}%
\pgfpathlineto{\pgfqpoint{4.692637in}{3.346168in}}%
\pgfpathlineto{\pgfqpoint{4.717447in}{3.354720in}}%
\pgfpathlineto{\pgfqpoint{4.742205in}{3.354473in}}%
\pgfpathlineto{\pgfqpoint{4.766979in}{3.352900in}}%
\pgfpathlineto{\pgfqpoint{4.791764in}{3.355505in}}%
\pgfpathlineto{\pgfqpoint{4.816503in}{3.359250in}}%
\pgfpathlineto{\pgfqpoint{4.846540in}{3.362461in}}%
\pgfpathlineto{\pgfqpoint{4.883926in}{3.363104in}}%
\pgfpathlineto{\pgfqpoint{4.921196in}{3.363104in}}%
\pgfpathlineto{\pgfqpoint{4.958462in}{3.363104in}}%
\pgfpathlineto{\pgfqpoint{4.995714in}{3.363104in}}%
\pgfpathlineto{\pgfqpoint{5.033005in}{3.363104in}}%
\pgfpathlineto{\pgfqpoint{5.070304in}{3.363104in}}%
\pgfpathlineto{\pgfqpoint{5.107610in}{3.363104in}}%
\pgfpathlineto{\pgfqpoint{5.144914in}{3.363104in}}%
\pgfpathlineto{\pgfqpoint{5.182186in}{3.363104in}}%
\pgfpathlineto{\pgfqpoint{5.219442in}{3.363104in}}%
\pgfpathlineto{\pgfqpoint{5.256693in}{3.363104in}}%
\pgfpathlineto{\pgfqpoint{5.293977in}{3.363104in}}%
\pgfpathlineto{\pgfqpoint{5.331273in}{3.363104in}}%
\pgfpathlineto{\pgfqpoint{5.368577in}{3.363104in}}%
\pgfpathlineto{\pgfqpoint{5.405885in}{3.363104in}}%
\pgfpathlineto{\pgfqpoint{5.405885in}{4.664590in}}%
\pgfpathlineto{\pgfqpoint{5.405885in}{4.664590in}}%
\pgfpathlineto{\pgfqpoint{5.368577in}{4.667255in}}%
\pgfpathlineto{\pgfqpoint{5.331273in}{4.671324in}}%
\pgfpathlineto{\pgfqpoint{5.293977in}{4.671707in}}%
\pgfpathlineto{\pgfqpoint{5.256693in}{4.667846in}}%
\pgfpathlineto{\pgfqpoint{5.219442in}{4.669695in}}%
\pgfpathlineto{\pgfqpoint{5.182186in}{4.670181in}}%
\pgfpathlineto{\pgfqpoint{5.144914in}{4.669325in}}%
\pgfpathlineto{\pgfqpoint{5.107610in}{4.669076in}}%
\pgfpathlineto{\pgfqpoint{5.070304in}{4.669476in}}%
\pgfpathlineto{\pgfqpoint{5.033005in}{4.670555in}}%
\pgfpathlineto{\pgfqpoint{4.995714in}{4.670339in}}%
\pgfpathlineto{\pgfqpoint{4.958462in}{4.670386in}}%
\pgfpathlineto{\pgfqpoint{4.921196in}{4.668093in}}%
\pgfpathlineto{\pgfqpoint{4.883926in}{4.668401in}}%
\pgfpathlineto{\pgfqpoint{4.846540in}{4.661995in}}%
\pgfpathlineto{\pgfqpoint{4.816503in}{4.673521in}}%
\pgfpathlineto{\pgfqpoint{4.791764in}{4.660915in}}%
\pgfpathlineto{\pgfqpoint{4.766979in}{4.665676in}}%
\pgfpathlineto{\pgfqpoint{4.742205in}{4.656381in}}%
\pgfpathlineto{\pgfqpoint{4.717447in}{4.664977in}}%
\pgfpathlineto{\pgfqpoint{4.692637in}{4.655583in}}%
\pgfpathlineto{\pgfqpoint{4.667884in}{4.653380in}}%
\pgfpathlineto{\pgfqpoint{4.643083in}{4.649944in}}%
\pgfpathlineto{\pgfqpoint{4.618263in}{4.636646in}}%
\pgfpathlineto{\pgfqpoint{4.593377in}{4.637062in}}%
\pgfpathlineto{\pgfqpoint{4.568473in}{4.609645in}}%
\pgfpathlineto{\pgfqpoint{4.543529in}{4.595634in}}%
\pgfpathlineto{\pgfqpoint{4.518427in}{4.520896in}}%
\pgfpathlineto{\pgfqpoint{4.492692in}{4.073421in}}%
\pgfpathlineto{\pgfqpoint{4.465281in}{3.731008in}}%
\pgfpathlineto{\pgfqpoint{4.436765in}{3.461934in}}%
\pgfpathlineto{\pgfqpoint{4.407030in}{3.201960in}}%
\pgfpathlineto{\pgfqpoint{4.376239in}{3.014489in}}%
\pgfpathlineto{\pgfqpoint{4.344452in}{2.818283in}}%
\pgfpathlineto{\pgfqpoint{4.311749in}{2.668023in}}%
\pgfpathlineto{\pgfqpoint{4.278250in}{2.544748in}}%
\pgfpathlineto{\pgfqpoint{4.244034in}{2.442066in}}%
\pgfpathlineto{\pgfqpoint{4.209218in}{2.328778in}}%
\pgfpathlineto{\pgfqpoint{4.173852in}{2.257032in}}%
\pgfpathlineto{\pgfqpoint{4.138055in}{2.191613in}}%
\pgfpathlineto{\pgfqpoint{4.101899in}{2.133556in}}%
\pgfpathlineto{\pgfqpoint{4.065634in}{2.091875in}}%
\pgfpathlineto{\pgfqpoint{4.028745in}{2.076258in}}%
\pgfpathlineto{\pgfqpoint{3.991588in}{2.089416in}}%
\pgfpathlineto{\pgfqpoint{3.954566in}{2.090672in}}%
\pgfpathlineto{\pgfqpoint{3.917591in}{2.091743in}}%
\pgfpathlineto{\pgfqpoint{3.880608in}{2.089291in}}%
\pgfpathlineto{\pgfqpoint{3.843637in}{2.091515in}}%
\pgfpathlineto{\pgfqpoint{3.806671in}{2.093222in}}%
\pgfpathlineto{\pgfqpoint{3.769397in}{1.762417in}}%
\pgfpathlineto{\pgfqpoint{3.730402in}{1.362032in}}%
\pgfpathlineto{\pgfqpoint{3.691109in}{1.361860in}}%
\pgfpathlineto{\pgfqpoint{3.651783in}{1.360511in}}%
\pgfpathlineto{\pgfqpoint{3.612456in}{1.360497in}}%
\pgfpathlineto{\pgfqpoint{3.573169in}{1.360712in}}%
\pgfpathlineto{\pgfqpoint{3.533833in}{1.359435in}}%
\pgfpathlineto{\pgfqpoint{3.494517in}{1.360712in}}%
\pgfpathlineto{\pgfqpoint{3.455225in}{1.359787in}}%
\pgfpathlineto{\pgfqpoint{3.415888in}{1.362416in}}%
\pgfpathlineto{\pgfqpoint{3.376505in}{1.356972in}}%
\pgfpathlineto{\pgfqpoint{3.337163in}{1.362396in}}%
\pgfpathlineto{\pgfqpoint{3.297077in}{1.339133in}}%
\pgfpathlineto{\pgfqpoint{3.255737in}{1.343240in}}%
\pgfpathlineto{\pgfqpoint{3.210786in}{1.333897in}}%
\pgfpathlineto{\pgfqpoint{3.164694in}{1.330200in}}%
\pgfpathlineto{\pgfqpoint{3.117833in}{1.328486in}}%
\pgfpathlineto{\pgfqpoint{3.070408in}{1.329105in}}%
\pgfpathlineto{\pgfqpoint{3.021960in}{1.329435in}}%
\pgfpathlineto{\pgfqpoint{2.972423in}{1.324644in}}%
\pgfpathlineto{\pgfqpoint{2.921979in}{1.325477in}}%
\pgfpathlineto{\pgfqpoint{2.869949in}{1.322957in}}%
\pgfpathlineto{\pgfqpoint{2.815997in}{1.314994in}}%
\pgfpathlineto{\pgfqpoint{2.760229in}{1.315429in}}%
\pgfpathlineto{\pgfqpoint{2.704011in}{1.318014in}}%
\pgfpathlineto{\pgfqpoint{2.646993in}{1.317405in}}%
\pgfpathlineto{\pgfqpoint{2.589001in}{1.315048in}}%
\pgfpathlineto{\pgfqpoint{2.529907in}{1.314789in}}%
\pgfpathlineto{\pgfqpoint{2.469564in}{1.313535in}}%
\pgfpathlineto{\pgfqpoint{2.411446in}{1.312654in}}%
\pgfpathlineto{\pgfqpoint{2.355457in}{1.314666in}}%
\pgfpathlineto{\pgfqpoint{2.300442in}{1.315084in}}%
\pgfpathlineto{\pgfqpoint{2.244502in}{1.315276in}}%
\pgfpathlineto{\pgfqpoint{2.188824in}{1.314691in}}%
\pgfpathlineto{\pgfqpoint{2.134427in}{1.316378in}}%
\pgfpathlineto{\pgfqpoint{2.078568in}{1.312392in}}%
\pgfpathlineto{\pgfqpoint{2.021509in}{1.313651in}}%
\pgfpathlineto{\pgfqpoint{1.963640in}{1.309627in}}%
\pgfpathlineto{\pgfqpoint{1.905453in}{1.314036in}}%
\pgfpathlineto{\pgfqpoint{1.849942in}{1.317586in}}%
\pgfpathlineto{\pgfqpoint{1.795378in}{1.311311in}}%
\pgfpathlineto{\pgfqpoint{1.747243in}{1.363597in}}%
\pgfpathlineto{\pgfqpoint{1.705741in}{1.363465in}}%
\pgfpathlineto{\pgfqpoint{1.664297in}{1.363974in}}%
\pgfpathlineto{\pgfqpoint{1.622829in}{1.364053in}}%
\pgfpathlineto{\pgfqpoint{1.581396in}{1.362069in}}%
\pgfpathlineto{\pgfqpoint{1.539972in}{1.364675in}}%
\pgfpathlineto{\pgfqpoint{1.498540in}{1.364751in}}%
\pgfpathlineto{\pgfqpoint{1.457070in}{1.362567in}}%
\pgfpathlineto{\pgfqpoint{1.415629in}{1.361473in}}%
\pgfpathlineto{\pgfqpoint{1.374186in}{1.360805in}}%
\pgfpathlineto{\pgfqpoint{1.332721in}{1.362715in}}%
\pgfpathlineto{\pgfqpoint{1.291280in}{1.364344in}}%
\pgfpathlineto{\pgfqpoint{1.249856in}{1.360154in}}%
\pgfpathlineto{\pgfqpoint{1.208458in}{1.362371in}}%
\pgfpathlineto{\pgfqpoint{1.167091in}{1.360322in}}%
\pgfpathlineto{\pgfqpoint{1.125752in}{1.364467in}}%
\pgfpathlineto{\pgfqpoint{1.084375in}{1.364769in}}%
\pgfpathlineto{\pgfqpoint{1.042993in}{1.366427in}}%
\pgfpathlineto{\pgfqpoint{1.001616in}{1.341197in}}%
\pgfpathclose%
\pgfusepath{fill}%
\end{pgfscope}%
\begin{pgfscope}%
\pgfpathrectangle{\pgfqpoint{0.781402in}{0.773588in}}{\pgfqpoint{4.844695in}{5.415119in}}%
\pgfusepath{clip}%
\pgfsetbuttcap%
\pgfsetroundjoin%
\definecolor{currentfill}{rgb}{0.839216,0.152941,0.156863}%
\pgfsetfillcolor{currentfill}%
\pgfsetlinewidth{0.000000pt}%
\definecolor{currentstroke}{rgb}{0.000000,0.000000,0.000000}%
\pgfsetstrokecolor{currentstroke}%
\pgfsetdash{}{0pt}%
\pgfpathmoveto{\pgfqpoint{1.001616in}{1.866345in}}%
\pgfpathlineto{\pgfqpoint{1.001616in}{1.341197in}}%
\pgfpathlineto{\pgfqpoint{1.042993in}{1.366427in}}%
\pgfpathlineto{\pgfqpoint{1.084375in}{1.364769in}}%
\pgfpathlineto{\pgfqpoint{1.125752in}{1.364467in}}%
\pgfpathlineto{\pgfqpoint{1.167091in}{1.360322in}}%
\pgfpathlineto{\pgfqpoint{1.208458in}{1.362371in}}%
\pgfpathlineto{\pgfqpoint{1.249856in}{1.360154in}}%
\pgfpathlineto{\pgfqpoint{1.291280in}{1.364344in}}%
\pgfpathlineto{\pgfqpoint{1.332721in}{1.362715in}}%
\pgfpathlineto{\pgfqpoint{1.374186in}{1.360805in}}%
\pgfpathlineto{\pgfqpoint{1.415629in}{1.361473in}}%
\pgfpathlineto{\pgfqpoint{1.457070in}{1.362567in}}%
\pgfpathlineto{\pgfqpoint{1.498540in}{1.364751in}}%
\pgfpathlineto{\pgfqpoint{1.539972in}{1.364675in}}%
\pgfpathlineto{\pgfqpoint{1.581396in}{1.362069in}}%
\pgfpathlineto{\pgfqpoint{1.622829in}{1.364053in}}%
\pgfpathlineto{\pgfqpoint{1.664297in}{1.363974in}}%
\pgfpathlineto{\pgfqpoint{1.705741in}{1.363465in}}%
\pgfpathlineto{\pgfqpoint{1.747243in}{1.363597in}}%
\pgfpathlineto{\pgfqpoint{1.795378in}{1.311311in}}%
\pgfpathlineto{\pgfqpoint{1.849942in}{1.317586in}}%
\pgfpathlineto{\pgfqpoint{1.905453in}{1.314036in}}%
\pgfpathlineto{\pgfqpoint{1.963640in}{1.309627in}}%
\pgfpathlineto{\pgfqpoint{2.021509in}{1.313651in}}%
\pgfpathlineto{\pgfqpoint{2.078568in}{1.312392in}}%
\pgfpathlineto{\pgfqpoint{2.134427in}{1.316378in}}%
\pgfpathlineto{\pgfqpoint{2.188824in}{1.314691in}}%
\pgfpathlineto{\pgfqpoint{2.244502in}{1.315276in}}%
\pgfpathlineto{\pgfqpoint{2.300442in}{1.315084in}}%
\pgfpathlineto{\pgfqpoint{2.355457in}{1.314666in}}%
\pgfpathlineto{\pgfqpoint{2.411446in}{1.312654in}}%
\pgfpathlineto{\pgfqpoint{2.469564in}{1.313535in}}%
\pgfpathlineto{\pgfqpoint{2.529907in}{1.314789in}}%
\pgfpathlineto{\pgfqpoint{2.589001in}{1.315048in}}%
\pgfpathlineto{\pgfqpoint{2.646993in}{1.317405in}}%
\pgfpathlineto{\pgfqpoint{2.704011in}{1.318014in}}%
\pgfpathlineto{\pgfqpoint{2.760229in}{1.315429in}}%
\pgfpathlineto{\pgfqpoint{2.815997in}{1.314994in}}%
\pgfpathlineto{\pgfqpoint{2.869949in}{1.322957in}}%
\pgfpathlineto{\pgfqpoint{2.921979in}{1.325477in}}%
\pgfpathlineto{\pgfqpoint{2.972423in}{1.324644in}}%
\pgfpathlineto{\pgfqpoint{3.021960in}{1.329435in}}%
\pgfpathlineto{\pgfqpoint{3.070408in}{1.329105in}}%
\pgfpathlineto{\pgfqpoint{3.117833in}{1.328486in}}%
\pgfpathlineto{\pgfqpoint{3.164694in}{1.330200in}}%
\pgfpathlineto{\pgfqpoint{3.210786in}{1.333897in}}%
\pgfpathlineto{\pgfqpoint{3.255737in}{1.343240in}}%
\pgfpathlineto{\pgfqpoint{3.297077in}{1.339133in}}%
\pgfpathlineto{\pgfqpoint{3.337163in}{1.362396in}}%
\pgfpathlineto{\pgfqpoint{3.376505in}{1.356972in}}%
\pgfpathlineto{\pgfqpoint{3.415888in}{1.362416in}}%
\pgfpathlineto{\pgfqpoint{3.455225in}{1.359787in}}%
\pgfpathlineto{\pgfqpoint{3.494517in}{1.360712in}}%
\pgfpathlineto{\pgfqpoint{3.533833in}{1.359435in}}%
\pgfpathlineto{\pgfqpoint{3.573169in}{1.360712in}}%
\pgfpathlineto{\pgfqpoint{3.612456in}{1.360497in}}%
\pgfpathlineto{\pgfqpoint{3.651783in}{1.360511in}}%
\pgfpathlineto{\pgfqpoint{3.691109in}{1.361860in}}%
\pgfpathlineto{\pgfqpoint{3.730402in}{1.362032in}}%
\pgfpathlineto{\pgfqpoint{3.769397in}{1.762417in}}%
\pgfpathlineto{\pgfqpoint{3.806671in}{2.093222in}}%
\pgfpathlineto{\pgfqpoint{3.843637in}{2.091515in}}%
\pgfpathlineto{\pgfqpoint{3.880608in}{2.089291in}}%
\pgfpathlineto{\pgfqpoint{3.917591in}{2.091743in}}%
\pgfpathlineto{\pgfqpoint{3.954566in}{2.090672in}}%
\pgfpathlineto{\pgfqpoint{3.991588in}{2.089416in}}%
\pgfpathlineto{\pgfqpoint{4.028745in}{2.076258in}}%
\pgfpathlineto{\pgfqpoint{4.065634in}{2.091875in}}%
\pgfpathlineto{\pgfqpoint{4.101899in}{2.133556in}}%
\pgfpathlineto{\pgfqpoint{4.138055in}{2.191613in}}%
\pgfpathlineto{\pgfqpoint{4.173852in}{2.257032in}}%
\pgfpathlineto{\pgfqpoint{4.209218in}{2.328778in}}%
\pgfpathlineto{\pgfqpoint{4.244034in}{2.442066in}}%
\pgfpathlineto{\pgfqpoint{4.278250in}{2.544748in}}%
\pgfpathlineto{\pgfqpoint{4.311749in}{2.668023in}}%
\pgfpathlineto{\pgfqpoint{4.344452in}{2.818283in}}%
\pgfpathlineto{\pgfqpoint{4.376239in}{3.014489in}}%
\pgfpathlineto{\pgfqpoint{4.407030in}{3.201960in}}%
\pgfpathlineto{\pgfqpoint{4.436765in}{3.461934in}}%
\pgfpathlineto{\pgfqpoint{4.465281in}{3.731008in}}%
\pgfpathlineto{\pgfqpoint{4.492692in}{4.073421in}}%
\pgfpathlineto{\pgfqpoint{4.518427in}{4.520896in}}%
\pgfpathlineto{\pgfqpoint{4.543529in}{4.595634in}}%
\pgfpathlineto{\pgfqpoint{4.568473in}{4.609645in}}%
\pgfpathlineto{\pgfqpoint{4.593377in}{4.637062in}}%
\pgfpathlineto{\pgfqpoint{4.618263in}{4.636646in}}%
\pgfpathlineto{\pgfqpoint{4.643083in}{4.649944in}}%
\pgfpathlineto{\pgfqpoint{4.667884in}{4.653380in}}%
\pgfpathlineto{\pgfqpoint{4.692637in}{4.655583in}}%
\pgfpathlineto{\pgfqpoint{4.717447in}{4.664977in}}%
\pgfpathlineto{\pgfqpoint{4.742205in}{4.656381in}}%
\pgfpathlineto{\pgfqpoint{4.766979in}{4.665676in}}%
\pgfpathlineto{\pgfqpoint{4.791764in}{4.660915in}}%
\pgfpathlineto{\pgfqpoint{4.816503in}{4.673521in}}%
\pgfpathlineto{\pgfqpoint{4.846540in}{4.661995in}}%
\pgfpathlineto{\pgfqpoint{4.883926in}{4.668401in}}%
\pgfpathlineto{\pgfqpoint{4.921196in}{4.668093in}}%
\pgfpathlineto{\pgfqpoint{4.958462in}{4.670386in}}%
\pgfpathlineto{\pgfqpoint{4.995714in}{4.670339in}}%
\pgfpathlineto{\pgfqpoint{5.033005in}{4.670555in}}%
\pgfpathlineto{\pgfqpoint{5.070304in}{4.669476in}}%
\pgfpathlineto{\pgfqpoint{5.107610in}{4.669076in}}%
\pgfpathlineto{\pgfqpoint{5.144914in}{4.669325in}}%
\pgfpathlineto{\pgfqpoint{5.182186in}{4.670181in}}%
\pgfpathlineto{\pgfqpoint{5.219442in}{4.669695in}}%
\pgfpathlineto{\pgfqpoint{5.256693in}{4.667846in}}%
\pgfpathlineto{\pgfqpoint{5.293977in}{4.671707in}}%
\pgfpathlineto{\pgfqpoint{5.331273in}{4.671324in}}%
\pgfpathlineto{\pgfqpoint{5.368577in}{4.667255in}}%
\pgfpathlineto{\pgfqpoint{5.405885in}{4.664590in}}%
\pgfpathlineto{\pgfqpoint{5.405885in}{5.907459in}}%
\pgfpathlineto{\pgfqpoint{5.405885in}{5.907459in}}%
\pgfpathlineto{\pgfqpoint{5.368577in}{5.919721in}}%
\pgfpathlineto{\pgfqpoint{5.331273in}{5.925136in}}%
\pgfpathlineto{\pgfqpoint{5.293977in}{5.924980in}}%
\pgfpathlineto{\pgfqpoint{5.256693in}{5.924291in}}%
\pgfpathlineto{\pgfqpoint{5.219442in}{5.923400in}}%
\pgfpathlineto{\pgfqpoint{5.182186in}{5.923092in}}%
\pgfpathlineto{\pgfqpoint{5.144914in}{5.918788in}}%
\pgfpathlineto{\pgfqpoint{5.107610in}{5.921473in}}%
\pgfpathlineto{\pgfqpoint{5.070304in}{5.923674in}}%
\pgfpathlineto{\pgfqpoint{5.033005in}{5.924518in}}%
\pgfpathlineto{\pgfqpoint{4.995714in}{5.925125in}}%
\pgfpathlineto{\pgfqpoint{4.958462in}{5.923366in}}%
\pgfpathlineto{\pgfqpoint{4.921196in}{5.920973in}}%
\pgfpathlineto{\pgfqpoint{4.883926in}{5.921248in}}%
\pgfpathlineto{\pgfqpoint{4.846540in}{5.906757in}}%
\pgfpathlineto{\pgfqpoint{4.816503in}{5.930845in}}%
\pgfpathlineto{\pgfqpoint{4.791764in}{5.909307in}}%
\pgfpathlineto{\pgfqpoint{4.766979in}{5.925260in}}%
\pgfpathlineto{\pgfqpoint{4.742205in}{5.908598in}}%
\pgfpathlineto{\pgfqpoint{4.717447in}{5.920974in}}%
\pgfpathlineto{\pgfqpoint{4.692637in}{5.908731in}}%
\pgfpathlineto{\pgfqpoint{4.667884in}{5.906386in}}%
\pgfpathlineto{\pgfqpoint{4.643083in}{5.908976in}}%
\pgfpathlineto{\pgfqpoint{4.618263in}{5.885322in}}%
\pgfpathlineto{\pgfqpoint{4.593377in}{5.892171in}}%
\pgfpathlineto{\pgfqpoint{4.568473in}{5.862978in}}%
\pgfpathlineto{\pgfqpoint{4.543529in}{5.851222in}}%
\pgfpathlineto{\pgfqpoint{4.518427in}{5.771935in}}%
\pgfpathlineto{\pgfqpoint{4.492692in}{5.331064in}}%
\pgfpathlineto{\pgfqpoint{4.465281in}{4.995191in}}%
\pgfpathlineto{\pgfqpoint{4.436765in}{4.721643in}}%
\pgfpathlineto{\pgfqpoint{4.407030in}{4.462792in}}%
\pgfpathlineto{\pgfqpoint{4.376239in}{4.281051in}}%
\pgfpathlineto{\pgfqpoint{4.344452in}{4.080060in}}%
\pgfpathlineto{\pgfqpoint{4.311749in}{3.929436in}}%
\pgfpathlineto{\pgfqpoint{4.278250in}{3.807579in}}%
\pgfpathlineto{\pgfqpoint{4.244034in}{3.705878in}}%
\pgfpathlineto{\pgfqpoint{4.209218in}{3.595131in}}%
\pgfpathlineto{\pgfqpoint{4.173852in}{3.521340in}}%
\pgfpathlineto{\pgfqpoint{4.138055in}{3.453135in}}%
\pgfpathlineto{\pgfqpoint{4.101899in}{3.395896in}}%
\pgfpathlineto{\pgfqpoint{4.065634in}{3.356544in}}%
\pgfpathlineto{\pgfqpoint{4.028745in}{3.337413in}}%
\pgfpathlineto{\pgfqpoint{3.991588in}{3.348610in}}%
\pgfpathlineto{\pgfqpoint{3.954566in}{3.352087in}}%
\pgfpathlineto{\pgfqpoint{3.917591in}{3.352081in}}%
\pgfpathlineto{\pgfqpoint{3.880608in}{3.352768in}}%
\pgfpathlineto{\pgfqpoint{3.843637in}{3.356373in}}%
\pgfpathlineto{\pgfqpoint{3.806671in}{3.359561in}}%
\pgfpathlineto{\pgfqpoint{3.769397in}{3.016708in}}%
\pgfpathlineto{\pgfqpoint{3.730402in}{2.616404in}}%
\pgfpathlineto{\pgfqpoint{3.691109in}{2.615119in}}%
\pgfpathlineto{\pgfqpoint{3.651783in}{2.612524in}}%
\pgfpathlineto{\pgfqpoint{3.612456in}{2.614373in}}%
\pgfpathlineto{\pgfqpoint{3.573169in}{2.614163in}}%
\pgfpathlineto{\pgfqpoint{3.533833in}{2.611885in}}%
\pgfpathlineto{\pgfqpoint{3.494517in}{2.616014in}}%
\pgfpathlineto{\pgfqpoint{3.455225in}{2.613496in}}%
\pgfpathlineto{\pgfqpoint{3.415888in}{2.615197in}}%
\pgfpathlineto{\pgfqpoint{3.376505in}{2.610689in}}%
\pgfpathlineto{\pgfqpoint{3.337163in}{2.615990in}}%
\pgfpathlineto{\pgfqpoint{3.297077in}{2.533652in}}%
\pgfpathlineto{\pgfqpoint{3.255737in}{2.179896in}}%
\pgfpathlineto{\pgfqpoint{3.210786in}{1.815662in}}%
\pgfpathlineto{\pgfqpoint{3.164694in}{1.802395in}}%
\pgfpathlineto{\pgfqpoint{3.117833in}{1.784407in}}%
\pgfpathlineto{\pgfqpoint{3.070408in}{1.768742in}}%
\pgfpathlineto{\pgfqpoint{3.021960in}{1.749076in}}%
\pgfpathlineto{\pgfqpoint{2.972423in}{1.725347in}}%
\pgfpathlineto{\pgfqpoint{2.921979in}{1.705542in}}%
\pgfpathlineto{\pgfqpoint{2.869949in}{1.674784in}}%
\pgfpathlineto{\pgfqpoint{2.815997in}{1.641840in}}%
\pgfpathlineto{\pgfqpoint{2.760229in}{1.618898in}}%
\pgfpathlineto{\pgfqpoint{2.704011in}{1.622280in}}%
\pgfpathlineto{\pgfqpoint{2.646993in}{1.600676in}}%
\pgfpathlineto{\pgfqpoint{2.589001in}{1.588521in}}%
\pgfpathlineto{\pgfqpoint{2.529907in}{1.567887in}}%
\pgfpathlineto{\pgfqpoint{2.469564in}{1.574774in}}%
\pgfpathlineto{\pgfqpoint{2.411446in}{1.613791in}}%
\pgfpathlineto{\pgfqpoint{2.355457in}{1.641934in}}%
\pgfpathlineto{\pgfqpoint{2.300442in}{1.633385in}}%
\pgfpathlineto{\pgfqpoint{2.244502in}{1.624399in}}%
\pgfpathlineto{\pgfqpoint{2.188824in}{1.638642in}}%
\pgfpathlineto{\pgfqpoint{2.134427in}{1.651335in}}%
\pgfpathlineto{\pgfqpoint{2.078568in}{1.603821in}}%
\pgfpathlineto{\pgfqpoint{2.021509in}{1.616476in}}%
\pgfpathlineto{\pgfqpoint{1.963640in}{1.585604in}}%
\pgfpathlineto{\pgfqpoint{1.905453in}{1.598895in}}%
\pgfpathlineto{\pgfqpoint{1.849942in}{1.687110in}}%
\pgfpathlineto{\pgfqpoint{1.795378in}{1.640816in}}%
\pgfpathlineto{\pgfqpoint{1.747243in}{1.926389in}}%
\pgfpathlineto{\pgfqpoint{1.705741in}{1.927812in}}%
\pgfpathlineto{\pgfqpoint{1.664297in}{1.925967in}}%
\pgfpathlineto{\pgfqpoint{1.622829in}{1.926168in}}%
\pgfpathlineto{\pgfqpoint{1.581396in}{1.922523in}}%
\pgfpathlineto{\pgfqpoint{1.539972in}{1.926730in}}%
\pgfpathlineto{\pgfqpoint{1.498540in}{1.927486in}}%
\pgfpathlineto{\pgfqpoint{1.457070in}{1.925064in}}%
\pgfpathlineto{\pgfqpoint{1.415629in}{1.924122in}}%
\pgfpathlineto{\pgfqpoint{1.374186in}{1.922103in}}%
\pgfpathlineto{\pgfqpoint{1.332721in}{1.925826in}}%
\pgfpathlineto{\pgfqpoint{1.291280in}{1.926536in}}%
\pgfpathlineto{\pgfqpoint{1.249856in}{1.923914in}}%
\pgfpathlineto{\pgfqpoint{1.208458in}{1.925928in}}%
\pgfpathlineto{\pgfqpoint{1.167091in}{1.924419in}}%
\pgfpathlineto{\pgfqpoint{1.125752in}{1.927518in}}%
\pgfpathlineto{\pgfqpoint{1.084375in}{1.924975in}}%
\pgfpathlineto{\pgfqpoint{1.042993in}{1.930978in}}%
\pgfpathlineto{\pgfqpoint{1.001616in}{1.866345in}}%
\pgfpathclose%
\pgfusepath{fill}%
\end{pgfscope}%
\begin{pgfscope}%
\pgfpathrectangle{\pgfqpoint{0.781402in}{0.773588in}}{\pgfqpoint{4.844695in}{5.415119in}}%
\pgfusepath{clip}%
\pgfsetbuttcap%
\pgfsetroundjoin%
\definecolor{currentfill}{rgb}{0.580392,0.403922,0.741176}%
\pgfsetfillcolor{currentfill}%
\pgfsetlinewidth{0.000000pt}%
\definecolor{currentstroke}{rgb}{0.000000,0.000000,0.000000}%
\pgfsetstrokecolor{currentstroke}%
\pgfsetdash{}{0pt}%
\pgfpathmoveto{\pgfqpoint{1.001616in}{2.423881in}}%
\pgfpathlineto{\pgfqpoint{1.001616in}{1.866345in}}%
\pgfpathlineto{\pgfqpoint{1.042993in}{1.930978in}}%
\pgfpathlineto{\pgfqpoint{1.084375in}{1.924975in}}%
\pgfpathlineto{\pgfqpoint{1.125752in}{1.927518in}}%
\pgfpathlineto{\pgfqpoint{1.167091in}{1.924419in}}%
\pgfpathlineto{\pgfqpoint{1.208458in}{1.925928in}}%
\pgfpathlineto{\pgfqpoint{1.249856in}{1.923914in}}%
\pgfpathlineto{\pgfqpoint{1.291280in}{1.926536in}}%
\pgfpathlineto{\pgfqpoint{1.332721in}{1.925826in}}%
\pgfpathlineto{\pgfqpoint{1.374186in}{1.922103in}}%
\pgfpathlineto{\pgfqpoint{1.415629in}{1.924122in}}%
\pgfpathlineto{\pgfqpoint{1.457070in}{1.925064in}}%
\pgfpathlineto{\pgfqpoint{1.498540in}{1.927486in}}%
\pgfpathlineto{\pgfqpoint{1.539972in}{1.926730in}}%
\pgfpathlineto{\pgfqpoint{1.581396in}{1.922523in}}%
\pgfpathlineto{\pgfqpoint{1.622829in}{1.926168in}}%
\pgfpathlineto{\pgfqpoint{1.664297in}{1.925967in}}%
\pgfpathlineto{\pgfqpoint{1.705741in}{1.927812in}}%
\pgfpathlineto{\pgfqpoint{1.747243in}{1.926389in}}%
\pgfpathlineto{\pgfqpoint{1.795378in}{1.640816in}}%
\pgfpathlineto{\pgfqpoint{1.849942in}{1.687110in}}%
\pgfpathlineto{\pgfqpoint{1.905453in}{1.598895in}}%
\pgfpathlineto{\pgfqpoint{1.963640in}{1.585604in}}%
\pgfpathlineto{\pgfqpoint{2.021509in}{1.616476in}}%
\pgfpathlineto{\pgfqpoint{2.078568in}{1.603821in}}%
\pgfpathlineto{\pgfqpoint{2.134427in}{1.651335in}}%
\pgfpathlineto{\pgfqpoint{2.188824in}{1.638642in}}%
\pgfpathlineto{\pgfqpoint{2.244502in}{1.624399in}}%
\pgfpathlineto{\pgfqpoint{2.300442in}{1.633385in}}%
\pgfpathlineto{\pgfqpoint{2.355457in}{1.641934in}}%
\pgfpathlineto{\pgfqpoint{2.411446in}{1.613791in}}%
\pgfpathlineto{\pgfqpoint{2.469564in}{1.574774in}}%
\pgfpathlineto{\pgfqpoint{2.529907in}{1.567887in}}%
\pgfpathlineto{\pgfqpoint{2.589001in}{1.588521in}}%
\pgfpathlineto{\pgfqpoint{2.646993in}{1.600676in}}%
\pgfpathlineto{\pgfqpoint{2.704011in}{1.622280in}}%
\pgfpathlineto{\pgfqpoint{2.760229in}{1.618898in}}%
\pgfpathlineto{\pgfqpoint{2.815997in}{1.641840in}}%
\pgfpathlineto{\pgfqpoint{2.869949in}{1.674784in}}%
\pgfpathlineto{\pgfqpoint{2.921979in}{1.705542in}}%
\pgfpathlineto{\pgfqpoint{2.972423in}{1.725347in}}%
\pgfpathlineto{\pgfqpoint{3.021960in}{1.749076in}}%
\pgfpathlineto{\pgfqpoint{3.070408in}{1.768742in}}%
\pgfpathlineto{\pgfqpoint{3.117833in}{1.784407in}}%
\pgfpathlineto{\pgfqpoint{3.164694in}{1.802395in}}%
\pgfpathlineto{\pgfqpoint{3.210786in}{1.815662in}}%
\pgfpathlineto{\pgfqpoint{3.255737in}{2.179896in}}%
\pgfpathlineto{\pgfqpoint{3.297077in}{2.533652in}}%
\pgfpathlineto{\pgfqpoint{3.337163in}{2.615990in}}%
\pgfpathlineto{\pgfqpoint{3.376505in}{2.610689in}}%
\pgfpathlineto{\pgfqpoint{3.415888in}{2.615197in}}%
\pgfpathlineto{\pgfqpoint{3.455225in}{2.613496in}}%
\pgfpathlineto{\pgfqpoint{3.494517in}{2.616014in}}%
\pgfpathlineto{\pgfqpoint{3.533833in}{2.611885in}}%
\pgfpathlineto{\pgfqpoint{3.573169in}{2.614163in}}%
\pgfpathlineto{\pgfqpoint{3.612456in}{2.614373in}}%
\pgfpathlineto{\pgfqpoint{3.651783in}{2.612524in}}%
\pgfpathlineto{\pgfqpoint{3.691109in}{2.615119in}}%
\pgfpathlineto{\pgfqpoint{3.730402in}{2.616404in}}%
\pgfpathlineto{\pgfqpoint{3.769397in}{3.016708in}}%
\pgfpathlineto{\pgfqpoint{3.806671in}{3.359561in}}%
\pgfpathlineto{\pgfqpoint{3.843637in}{3.356373in}}%
\pgfpathlineto{\pgfqpoint{3.880608in}{3.352768in}}%
\pgfpathlineto{\pgfqpoint{3.917591in}{3.352081in}}%
\pgfpathlineto{\pgfqpoint{3.954566in}{3.352087in}}%
\pgfpathlineto{\pgfqpoint{3.991588in}{3.348610in}}%
\pgfpathlineto{\pgfqpoint{4.028745in}{3.337413in}}%
\pgfpathlineto{\pgfqpoint{4.065634in}{3.356544in}}%
\pgfpathlineto{\pgfqpoint{4.101899in}{3.395896in}}%
\pgfpathlineto{\pgfqpoint{4.138055in}{3.453135in}}%
\pgfpathlineto{\pgfqpoint{4.173852in}{3.521340in}}%
\pgfpathlineto{\pgfqpoint{4.209218in}{3.595131in}}%
\pgfpathlineto{\pgfqpoint{4.244034in}{3.705878in}}%
\pgfpathlineto{\pgfqpoint{4.278250in}{3.807579in}}%
\pgfpathlineto{\pgfqpoint{4.311749in}{3.929436in}}%
\pgfpathlineto{\pgfqpoint{4.344452in}{4.080060in}}%
\pgfpathlineto{\pgfqpoint{4.376239in}{4.281051in}}%
\pgfpathlineto{\pgfqpoint{4.407030in}{4.462792in}}%
\pgfpathlineto{\pgfqpoint{4.436765in}{4.721643in}}%
\pgfpathlineto{\pgfqpoint{4.465281in}{4.995191in}}%
\pgfpathlineto{\pgfqpoint{4.492692in}{5.331064in}}%
\pgfpathlineto{\pgfqpoint{4.518427in}{5.771935in}}%
\pgfpathlineto{\pgfqpoint{4.543529in}{5.851222in}}%
\pgfpathlineto{\pgfqpoint{4.568473in}{5.862978in}}%
\pgfpathlineto{\pgfqpoint{4.593377in}{5.892171in}}%
\pgfpathlineto{\pgfqpoint{4.618263in}{5.885322in}}%
\pgfpathlineto{\pgfqpoint{4.643083in}{5.908976in}}%
\pgfpathlineto{\pgfqpoint{4.667884in}{5.906386in}}%
\pgfpathlineto{\pgfqpoint{4.692637in}{5.908731in}}%
\pgfpathlineto{\pgfqpoint{4.717447in}{5.920974in}}%
\pgfpathlineto{\pgfqpoint{4.742205in}{5.908598in}}%
\pgfpathlineto{\pgfqpoint{4.766979in}{5.925260in}}%
\pgfpathlineto{\pgfqpoint{4.791764in}{5.909307in}}%
\pgfpathlineto{\pgfqpoint{4.816503in}{5.930845in}}%
\pgfpathlineto{\pgfqpoint{4.846540in}{5.906757in}}%
\pgfpathlineto{\pgfqpoint{4.883926in}{5.921248in}}%
\pgfpathlineto{\pgfqpoint{4.921196in}{5.920973in}}%
\pgfpathlineto{\pgfqpoint{4.958462in}{5.923366in}}%
\pgfpathlineto{\pgfqpoint{4.995714in}{5.925125in}}%
\pgfpathlineto{\pgfqpoint{5.033005in}{5.924518in}}%
\pgfpathlineto{\pgfqpoint{5.070304in}{5.923674in}}%
\pgfpathlineto{\pgfqpoint{5.107610in}{5.921473in}}%
\pgfpathlineto{\pgfqpoint{5.144914in}{5.918788in}}%
\pgfpathlineto{\pgfqpoint{5.182186in}{5.923092in}}%
\pgfpathlineto{\pgfqpoint{5.219442in}{5.923400in}}%
\pgfpathlineto{\pgfqpoint{5.256693in}{5.924291in}}%
\pgfpathlineto{\pgfqpoint{5.293977in}{5.924980in}}%
\pgfpathlineto{\pgfqpoint{5.331273in}{5.925136in}}%
\pgfpathlineto{\pgfqpoint{5.368577in}{5.919721in}}%
\pgfpathlineto{\pgfqpoint{5.405885in}{5.907459in}}%
\pgfpathlineto{\pgfqpoint{5.405885in}{5.907459in}}%
\pgfpathlineto{\pgfqpoint{5.405885in}{5.907459in}}%
\pgfpathlineto{\pgfqpoint{5.368577in}{5.919721in}}%
\pgfpathlineto{\pgfqpoint{5.331273in}{5.925136in}}%
\pgfpathlineto{\pgfqpoint{5.293977in}{5.924980in}}%
\pgfpathlineto{\pgfqpoint{5.256693in}{5.924291in}}%
\pgfpathlineto{\pgfqpoint{5.219442in}{5.923400in}}%
\pgfpathlineto{\pgfqpoint{5.182186in}{5.923092in}}%
\pgfpathlineto{\pgfqpoint{5.144914in}{5.918788in}}%
\pgfpathlineto{\pgfqpoint{5.107610in}{5.921473in}}%
\pgfpathlineto{\pgfqpoint{5.070304in}{5.923674in}}%
\pgfpathlineto{\pgfqpoint{5.033005in}{5.924518in}}%
\pgfpathlineto{\pgfqpoint{4.995714in}{5.925125in}}%
\pgfpathlineto{\pgfqpoint{4.958462in}{5.923366in}}%
\pgfpathlineto{\pgfqpoint{4.921196in}{5.920973in}}%
\pgfpathlineto{\pgfqpoint{4.883926in}{5.921248in}}%
\pgfpathlineto{\pgfqpoint{4.846540in}{5.906757in}}%
\pgfpathlineto{\pgfqpoint{4.816503in}{5.930845in}}%
\pgfpathlineto{\pgfqpoint{4.791764in}{5.909307in}}%
\pgfpathlineto{\pgfqpoint{4.766979in}{5.925260in}}%
\pgfpathlineto{\pgfqpoint{4.742205in}{5.908598in}}%
\pgfpathlineto{\pgfqpoint{4.717447in}{5.920974in}}%
\pgfpathlineto{\pgfqpoint{4.692637in}{5.908731in}}%
\pgfpathlineto{\pgfqpoint{4.667884in}{5.906386in}}%
\pgfpathlineto{\pgfqpoint{4.643083in}{5.908976in}}%
\pgfpathlineto{\pgfqpoint{4.618263in}{5.885322in}}%
\pgfpathlineto{\pgfqpoint{4.593377in}{5.892171in}}%
\pgfpathlineto{\pgfqpoint{4.568473in}{5.862978in}}%
\pgfpathlineto{\pgfqpoint{4.543529in}{5.851222in}}%
\pgfpathlineto{\pgfqpoint{4.518427in}{5.771935in}}%
\pgfpathlineto{\pgfqpoint{4.492692in}{5.331064in}}%
\pgfpathlineto{\pgfqpoint{4.465281in}{4.995191in}}%
\pgfpathlineto{\pgfqpoint{4.436765in}{4.721643in}}%
\pgfpathlineto{\pgfqpoint{4.407030in}{4.462792in}}%
\pgfpathlineto{\pgfqpoint{4.376239in}{4.281051in}}%
\pgfpathlineto{\pgfqpoint{4.344452in}{4.080060in}}%
\pgfpathlineto{\pgfqpoint{4.311749in}{3.929436in}}%
\pgfpathlineto{\pgfqpoint{4.278250in}{3.807579in}}%
\pgfpathlineto{\pgfqpoint{4.244034in}{3.705878in}}%
\pgfpathlineto{\pgfqpoint{4.209218in}{3.595131in}}%
\pgfpathlineto{\pgfqpoint{4.173852in}{3.521340in}}%
\pgfpathlineto{\pgfqpoint{4.138055in}{3.453135in}}%
\pgfpathlineto{\pgfqpoint{4.101899in}{3.395896in}}%
\pgfpathlineto{\pgfqpoint{4.065634in}{3.356544in}}%
\pgfpathlineto{\pgfqpoint{4.028745in}{3.337413in}}%
\pgfpathlineto{\pgfqpoint{3.991588in}{3.348610in}}%
\pgfpathlineto{\pgfqpoint{3.954566in}{3.352087in}}%
\pgfpathlineto{\pgfqpoint{3.917591in}{3.352081in}}%
\pgfpathlineto{\pgfqpoint{3.880608in}{3.352768in}}%
\pgfpathlineto{\pgfqpoint{3.843637in}{3.356373in}}%
\pgfpathlineto{\pgfqpoint{3.806671in}{3.359561in}}%
\pgfpathlineto{\pgfqpoint{3.769397in}{3.279789in}}%
\pgfpathlineto{\pgfqpoint{3.730402in}{3.201512in}}%
\pgfpathlineto{\pgfqpoint{3.691109in}{3.200491in}}%
\pgfpathlineto{\pgfqpoint{3.651783in}{3.198256in}}%
\pgfpathlineto{\pgfqpoint{3.612456in}{3.201385in}}%
\pgfpathlineto{\pgfqpoint{3.573169in}{3.201452in}}%
\pgfpathlineto{\pgfqpoint{3.533833in}{3.198066in}}%
\pgfpathlineto{\pgfqpoint{3.494517in}{3.204856in}}%
\pgfpathlineto{\pgfqpoint{3.455225in}{3.202255in}}%
\pgfpathlineto{\pgfqpoint{3.415888in}{3.202131in}}%
\pgfpathlineto{\pgfqpoint{3.376505in}{3.194553in}}%
\pgfpathlineto{\pgfqpoint{3.337163in}{3.200877in}}%
\pgfpathlineto{\pgfqpoint{3.297077in}{3.099470in}}%
\pgfpathlineto{\pgfqpoint{3.255737in}{2.746544in}}%
\pgfpathlineto{\pgfqpoint{3.210786in}{2.373235in}}%
\pgfpathlineto{\pgfqpoint{3.164694in}{2.357668in}}%
\pgfpathlineto{\pgfqpoint{3.117833in}{2.339936in}}%
\pgfpathlineto{\pgfqpoint{3.070408in}{2.323635in}}%
\pgfpathlineto{\pgfqpoint{3.021960in}{2.301234in}}%
\pgfpathlineto{\pgfqpoint{2.972423in}{2.276296in}}%
\pgfpathlineto{\pgfqpoint{2.921979in}{2.259215in}}%
\pgfpathlineto{\pgfqpoint{2.869949in}{2.223093in}}%
\pgfpathlineto{\pgfqpoint{2.815997in}{2.184370in}}%
\pgfpathlineto{\pgfqpoint{2.760229in}{2.162658in}}%
\pgfpathlineto{\pgfqpoint{2.704011in}{2.165773in}}%
\pgfpathlineto{\pgfqpoint{2.646993in}{2.144592in}}%
\pgfpathlineto{\pgfqpoint{2.589001in}{2.132370in}}%
\pgfpathlineto{\pgfqpoint{2.529907in}{2.108492in}}%
\pgfpathlineto{\pgfqpoint{2.469564in}{2.116072in}}%
\pgfpathlineto{\pgfqpoint{2.411446in}{2.153662in}}%
\pgfpathlineto{\pgfqpoint{2.355457in}{2.185053in}}%
\pgfpathlineto{\pgfqpoint{2.300442in}{2.176320in}}%
\pgfpathlineto{\pgfqpoint{2.244502in}{2.166321in}}%
\pgfpathlineto{\pgfqpoint{2.188824in}{2.182296in}}%
\pgfpathlineto{\pgfqpoint{2.134427in}{2.196483in}}%
\pgfpathlineto{\pgfqpoint{2.078568in}{2.140756in}}%
\pgfpathlineto{\pgfqpoint{2.021509in}{2.156892in}}%
\pgfpathlineto{\pgfqpoint{1.963640in}{2.122721in}}%
\pgfpathlineto{\pgfqpoint{1.905453in}{2.136216in}}%
\pgfpathlineto{\pgfqpoint{1.849942in}{2.228684in}}%
\pgfpathlineto{\pgfqpoint{1.795378in}{2.179073in}}%
\pgfpathlineto{\pgfqpoint{1.747243in}{2.514716in}}%
\pgfpathlineto{\pgfqpoint{1.705741in}{2.516533in}}%
\pgfpathlineto{\pgfqpoint{1.664297in}{2.515981in}}%
\pgfpathlineto{\pgfqpoint{1.622829in}{2.516396in}}%
\pgfpathlineto{\pgfqpoint{1.581396in}{2.512910in}}%
\pgfpathlineto{\pgfqpoint{1.539972in}{2.514893in}}%
\pgfpathlineto{\pgfqpoint{1.498540in}{2.515524in}}%
\pgfpathlineto{\pgfqpoint{1.457070in}{2.515639in}}%
\pgfpathlineto{\pgfqpoint{1.415629in}{2.515241in}}%
\pgfpathlineto{\pgfqpoint{1.374186in}{2.513631in}}%
\pgfpathlineto{\pgfqpoint{1.332721in}{2.514045in}}%
\pgfpathlineto{\pgfqpoint{1.291280in}{2.515772in}}%
\pgfpathlineto{\pgfqpoint{1.249856in}{2.515562in}}%
\pgfpathlineto{\pgfqpoint{1.208458in}{2.517861in}}%
\pgfpathlineto{\pgfqpoint{1.167091in}{2.518390in}}%
\pgfpathlineto{\pgfqpoint{1.125752in}{2.518898in}}%
\pgfpathlineto{\pgfqpoint{1.084375in}{2.514891in}}%
\pgfpathlineto{\pgfqpoint{1.042993in}{2.519575in}}%
\pgfpathlineto{\pgfqpoint{1.001616in}{2.423881in}}%
\pgfpathclose%
\pgfusepath{fill}%
\end{pgfscope}%
\begin{pgfscope}%
\pgfpathrectangle{\pgfqpoint{0.781402in}{0.773588in}}{\pgfqpoint{4.844695in}{5.415119in}}%
\pgfusepath{clip}%
\pgfsetbuttcap%
\pgfsetroundjoin%
\definecolor{currentfill}{rgb}{0.549020,0.337255,0.294118}%
\pgfsetfillcolor{currentfill}%
\pgfsetlinewidth{0.000000pt}%
\definecolor{currentstroke}{rgb}{0.000000,0.000000,0.000000}%
\pgfsetstrokecolor{currentstroke}%
\pgfsetdash{}{0pt}%
\pgfpathmoveto{\pgfqpoint{1.001616in}{2.946160in}}%
\pgfpathlineto{\pgfqpoint{1.001616in}{2.423881in}}%
\pgfpathlineto{\pgfqpoint{1.042993in}{2.519575in}}%
\pgfpathlineto{\pgfqpoint{1.084375in}{2.514891in}}%
\pgfpathlineto{\pgfqpoint{1.125752in}{2.518898in}}%
\pgfpathlineto{\pgfqpoint{1.167091in}{2.518390in}}%
\pgfpathlineto{\pgfqpoint{1.208458in}{2.517861in}}%
\pgfpathlineto{\pgfqpoint{1.249856in}{2.515562in}}%
\pgfpathlineto{\pgfqpoint{1.291280in}{2.515772in}}%
\pgfpathlineto{\pgfqpoint{1.332721in}{2.514045in}}%
\pgfpathlineto{\pgfqpoint{1.374186in}{2.513631in}}%
\pgfpathlineto{\pgfqpoint{1.415629in}{2.515241in}}%
\pgfpathlineto{\pgfqpoint{1.457070in}{2.515639in}}%
\pgfpathlineto{\pgfqpoint{1.498540in}{2.515524in}}%
\pgfpathlineto{\pgfqpoint{1.539972in}{2.514893in}}%
\pgfpathlineto{\pgfqpoint{1.581396in}{2.512910in}}%
\pgfpathlineto{\pgfqpoint{1.622829in}{2.516396in}}%
\pgfpathlineto{\pgfqpoint{1.664297in}{2.515981in}}%
\pgfpathlineto{\pgfqpoint{1.705741in}{2.516533in}}%
\pgfpathlineto{\pgfqpoint{1.747243in}{2.514716in}}%
\pgfpathlineto{\pgfqpoint{1.795378in}{2.179073in}}%
\pgfpathlineto{\pgfqpoint{1.849942in}{2.228684in}}%
\pgfpathlineto{\pgfqpoint{1.905453in}{2.136216in}}%
\pgfpathlineto{\pgfqpoint{1.963640in}{2.122721in}}%
\pgfpathlineto{\pgfqpoint{2.021509in}{2.156892in}}%
\pgfpathlineto{\pgfqpoint{2.078568in}{2.140756in}}%
\pgfpathlineto{\pgfqpoint{2.134427in}{2.196483in}}%
\pgfpathlineto{\pgfqpoint{2.188824in}{2.182296in}}%
\pgfpathlineto{\pgfqpoint{2.244502in}{2.166321in}}%
\pgfpathlineto{\pgfqpoint{2.300442in}{2.176320in}}%
\pgfpathlineto{\pgfqpoint{2.355457in}{2.185053in}}%
\pgfpathlineto{\pgfqpoint{2.411446in}{2.153662in}}%
\pgfpathlineto{\pgfqpoint{2.469564in}{2.116072in}}%
\pgfpathlineto{\pgfqpoint{2.529907in}{2.108492in}}%
\pgfpathlineto{\pgfqpoint{2.589001in}{2.132370in}}%
\pgfpathlineto{\pgfqpoint{2.646993in}{2.144592in}}%
\pgfpathlineto{\pgfqpoint{2.704011in}{2.165773in}}%
\pgfpathlineto{\pgfqpoint{2.760229in}{2.162658in}}%
\pgfpathlineto{\pgfqpoint{2.815997in}{2.184370in}}%
\pgfpathlineto{\pgfqpoint{2.869949in}{2.223093in}}%
\pgfpathlineto{\pgfqpoint{2.921979in}{2.259215in}}%
\pgfpathlineto{\pgfqpoint{2.972423in}{2.276296in}}%
\pgfpathlineto{\pgfqpoint{3.021960in}{2.301234in}}%
\pgfpathlineto{\pgfqpoint{3.070408in}{2.323635in}}%
\pgfpathlineto{\pgfqpoint{3.117833in}{2.339936in}}%
\pgfpathlineto{\pgfqpoint{3.164694in}{2.357668in}}%
\pgfpathlineto{\pgfqpoint{3.210786in}{2.373235in}}%
\pgfpathlineto{\pgfqpoint{3.255737in}{2.746544in}}%
\pgfpathlineto{\pgfqpoint{3.297077in}{3.099470in}}%
\pgfpathlineto{\pgfqpoint{3.337163in}{3.200877in}}%
\pgfpathlineto{\pgfqpoint{3.376505in}{3.194553in}}%
\pgfpathlineto{\pgfqpoint{3.415888in}{3.202131in}}%
\pgfpathlineto{\pgfqpoint{3.455225in}{3.202255in}}%
\pgfpathlineto{\pgfqpoint{3.494517in}{3.204856in}}%
\pgfpathlineto{\pgfqpoint{3.533833in}{3.198066in}}%
\pgfpathlineto{\pgfqpoint{3.573169in}{3.201452in}}%
\pgfpathlineto{\pgfqpoint{3.612456in}{3.201385in}}%
\pgfpathlineto{\pgfqpoint{3.651783in}{3.198256in}}%
\pgfpathlineto{\pgfqpoint{3.691109in}{3.200491in}}%
\pgfpathlineto{\pgfqpoint{3.730402in}{3.201512in}}%
\pgfpathlineto{\pgfqpoint{3.769397in}{3.279789in}}%
\pgfpathlineto{\pgfqpoint{3.806671in}{3.359561in}}%
\pgfpathlineto{\pgfqpoint{3.843637in}{3.356373in}}%
\pgfpathlineto{\pgfqpoint{3.880608in}{3.352768in}}%
\pgfpathlineto{\pgfqpoint{3.917591in}{3.352081in}}%
\pgfpathlineto{\pgfqpoint{3.954566in}{3.352087in}}%
\pgfpathlineto{\pgfqpoint{3.991588in}{3.348610in}}%
\pgfpathlineto{\pgfqpoint{4.028745in}{3.337413in}}%
\pgfpathlineto{\pgfqpoint{4.065634in}{3.356544in}}%
\pgfpathlineto{\pgfqpoint{4.101899in}{3.395896in}}%
\pgfpathlineto{\pgfqpoint{4.138055in}{3.453135in}}%
\pgfpathlineto{\pgfqpoint{4.173852in}{3.521340in}}%
\pgfpathlineto{\pgfqpoint{4.209218in}{3.595131in}}%
\pgfpathlineto{\pgfqpoint{4.244034in}{3.705878in}}%
\pgfpathlineto{\pgfqpoint{4.278250in}{3.807579in}}%
\pgfpathlineto{\pgfqpoint{4.311749in}{3.929436in}}%
\pgfpathlineto{\pgfqpoint{4.344452in}{4.080060in}}%
\pgfpathlineto{\pgfqpoint{4.376239in}{4.281051in}}%
\pgfpathlineto{\pgfqpoint{4.407030in}{4.462792in}}%
\pgfpathlineto{\pgfqpoint{4.436765in}{4.721643in}}%
\pgfpathlineto{\pgfqpoint{4.465281in}{4.995191in}}%
\pgfpathlineto{\pgfqpoint{4.492692in}{5.331064in}}%
\pgfpathlineto{\pgfqpoint{4.518427in}{5.771935in}}%
\pgfpathlineto{\pgfqpoint{4.543529in}{5.851222in}}%
\pgfpathlineto{\pgfqpoint{4.568473in}{5.862978in}}%
\pgfpathlineto{\pgfqpoint{4.593377in}{5.892171in}}%
\pgfpathlineto{\pgfqpoint{4.618263in}{5.885322in}}%
\pgfpathlineto{\pgfqpoint{4.643083in}{5.908976in}}%
\pgfpathlineto{\pgfqpoint{4.667884in}{5.906386in}}%
\pgfpathlineto{\pgfqpoint{4.692637in}{5.908731in}}%
\pgfpathlineto{\pgfqpoint{4.717447in}{5.920974in}}%
\pgfpathlineto{\pgfqpoint{4.742205in}{5.908598in}}%
\pgfpathlineto{\pgfqpoint{4.766979in}{5.925260in}}%
\pgfpathlineto{\pgfqpoint{4.791764in}{5.909307in}}%
\pgfpathlineto{\pgfqpoint{4.816503in}{5.930845in}}%
\pgfpathlineto{\pgfqpoint{4.846540in}{5.906757in}}%
\pgfpathlineto{\pgfqpoint{4.883926in}{5.921248in}}%
\pgfpathlineto{\pgfqpoint{4.921196in}{5.920973in}}%
\pgfpathlineto{\pgfqpoint{4.958462in}{5.923366in}}%
\pgfpathlineto{\pgfqpoint{4.995714in}{5.925125in}}%
\pgfpathlineto{\pgfqpoint{5.033005in}{5.924518in}}%
\pgfpathlineto{\pgfqpoint{5.070304in}{5.923674in}}%
\pgfpathlineto{\pgfqpoint{5.107610in}{5.921473in}}%
\pgfpathlineto{\pgfqpoint{5.144914in}{5.918788in}}%
\pgfpathlineto{\pgfqpoint{5.182186in}{5.923092in}}%
\pgfpathlineto{\pgfqpoint{5.219442in}{5.923400in}}%
\pgfpathlineto{\pgfqpoint{5.256693in}{5.924291in}}%
\pgfpathlineto{\pgfqpoint{5.293977in}{5.924980in}}%
\pgfpathlineto{\pgfqpoint{5.331273in}{5.925136in}}%
\pgfpathlineto{\pgfqpoint{5.368577in}{5.919721in}}%
\pgfpathlineto{\pgfqpoint{5.405885in}{5.907459in}}%
\pgfpathlineto{\pgfqpoint{5.405885in}{5.907459in}}%
\pgfpathlineto{\pgfqpoint{5.405885in}{5.907459in}}%
\pgfpathlineto{\pgfqpoint{5.368577in}{5.919721in}}%
\pgfpathlineto{\pgfqpoint{5.331273in}{5.925136in}}%
\pgfpathlineto{\pgfqpoint{5.293977in}{5.924980in}}%
\pgfpathlineto{\pgfqpoint{5.256693in}{5.924291in}}%
\pgfpathlineto{\pgfqpoint{5.219442in}{5.923400in}}%
\pgfpathlineto{\pgfqpoint{5.182186in}{5.923092in}}%
\pgfpathlineto{\pgfqpoint{5.144914in}{5.918788in}}%
\pgfpathlineto{\pgfqpoint{5.107610in}{5.921473in}}%
\pgfpathlineto{\pgfqpoint{5.070304in}{5.923674in}}%
\pgfpathlineto{\pgfqpoint{5.033005in}{5.924518in}}%
\pgfpathlineto{\pgfqpoint{4.995714in}{5.925125in}}%
\pgfpathlineto{\pgfqpoint{4.958462in}{5.923366in}}%
\pgfpathlineto{\pgfqpoint{4.921196in}{5.920973in}}%
\pgfpathlineto{\pgfqpoint{4.883926in}{5.921248in}}%
\pgfpathlineto{\pgfqpoint{4.846540in}{5.906757in}}%
\pgfpathlineto{\pgfqpoint{4.816503in}{5.930845in}}%
\pgfpathlineto{\pgfqpoint{4.791764in}{5.909307in}}%
\pgfpathlineto{\pgfqpoint{4.766979in}{5.925260in}}%
\pgfpathlineto{\pgfqpoint{4.742205in}{5.908598in}}%
\pgfpathlineto{\pgfqpoint{4.717447in}{5.920974in}}%
\pgfpathlineto{\pgfqpoint{4.692637in}{5.908731in}}%
\pgfpathlineto{\pgfqpoint{4.667884in}{5.906386in}}%
\pgfpathlineto{\pgfqpoint{4.643083in}{5.908976in}}%
\pgfpathlineto{\pgfqpoint{4.618263in}{5.885322in}}%
\pgfpathlineto{\pgfqpoint{4.593377in}{5.892171in}}%
\pgfpathlineto{\pgfqpoint{4.568473in}{5.862978in}}%
\pgfpathlineto{\pgfqpoint{4.543529in}{5.851222in}}%
\pgfpathlineto{\pgfqpoint{4.518427in}{5.771935in}}%
\pgfpathlineto{\pgfqpoint{4.492692in}{5.331064in}}%
\pgfpathlineto{\pgfqpoint{4.465281in}{4.995191in}}%
\pgfpathlineto{\pgfqpoint{4.436765in}{4.721643in}}%
\pgfpathlineto{\pgfqpoint{4.407030in}{4.462792in}}%
\pgfpathlineto{\pgfqpoint{4.376239in}{4.281051in}}%
\pgfpathlineto{\pgfqpoint{4.344452in}{4.080060in}}%
\pgfpathlineto{\pgfqpoint{4.311749in}{3.929436in}}%
\pgfpathlineto{\pgfqpoint{4.278250in}{3.807579in}}%
\pgfpathlineto{\pgfqpoint{4.244034in}{3.705878in}}%
\pgfpathlineto{\pgfqpoint{4.209218in}{3.595131in}}%
\pgfpathlineto{\pgfqpoint{4.173852in}{3.521340in}}%
\pgfpathlineto{\pgfqpoint{4.138055in}{3.453135in}}%
\pgfpathlineto{\pgfqpoint{4.101899in}{3.395896in}}%
\pgfpathlineto{\pgfqpoint{4.065634in}{3.356544in}}%
\pgfpathlineto{\pgfqpoint{4.028745in}{3.337413in}}%
\pgfpathlineto{\pgfqpoint{3.991588in}{3.348610in}}%
\pgfpathlineto{\pgfqpoint{3.954566in}{3.352087in}}%
\pgfpathlineto{\pgfqpoint{3.917591in}{3.352081in}}%
\pgfpathlineto{\pgfqpoint{3.880608in}{3.352768in}}%
\pgfpathlineto{\pgfqpoint{3.843637in}{3.356373in}}%
\pgfpathlineto{\pgfqpoint{3.806671in}{3.359561in}}%
\pgfpathlineto{\pgfqpoint{3.769397in}{3.279789in}}%
\pgfpathlineto{\pgfqpoint{3.730402in}{3.201512in}}%
\pgfpathlineto{\pgfqpoint{3.691109in}{3.200491in}}%
\pgfpathlineto{\pgfqpoint{3.651783in}{3.198256in}}%
\pgfpathlineto{\pgfqpoint{3.612456in}{3.201385in}}%
\pgfpathlineto{\pgfqpoint{3.573169in}{3.201452in}}%
\pgfpathlineto{\pgfqpoint{3.533833in}{3.198066in}}%
\pgfpathlineto{\pgfqpoint{3.494517in}{3.204856in}}%
\pgfpathlineto{\pgfqpoint{3.455225in}{3.202255in}}%
\pgfpathlineto{\pgfqpoint{3.415888in}{3.202131in}}%
\pgfpathlineto{\pgfqpoint{3.376505in}{3.194553in}}%
\pgfpathlineto{\pgfqpoint{3.337163in}{3.200877in}}%
\pgfpathlineto{\pgfqpoint{3.297077in}{3.099470in}}%
\pgfpathlineto{\pgfqpoint{3.255737in}{3.006630in}}%
\pgfpathlineto{\pgfqpoint{3.210786in}{2.855593in}}%
\pgfpathlineto{\pgfqpoint{3.164694in}{2.827309in}}%
\pgfpathlineto{\pgfqpoint{3.117833in}{2.797983in}}%
\pgfpathlineto{\pgfqpoint{3.070408in}{2.763368in}}%
\pgfpathlineto{\pgfqpoint{3.021960in}{2.721433in}}%
\pgfpathlineto{\pgfqpoint{2.972423in}{2.679835in}}%
\pgfpathlineto{\pgfqpoint{2.921979in}{2.639209in}}%
\pgfpathlineto{\pgfqpoint{2.869949in}{2.574241in}}%
\pgfpathlineto{\pgfqpoint{2.815997in}{2.508103in}}%
\pgfpathlineto{\pgfqpoint{2.760229in}{2.466942in}}%
\pgfpathlineto{\pgfqpoint{2.704011in}{2.467193in}}%
\pgfpathlineto{\pgfqpoint{2.646993in}{2.430474in}}%
\pgfpathlineto{\pgfqpoint{2.589001in}{2.406564in}}%
\pgfpathlineto{\pgfqpoint{2.529907in}{2.359749in}}%
\pgfpathlineto{\pgfqpoint{2.469564in}{2.379650in}}%
\pgfpathlineto{\pgfqpoint{2.411446in}{2.452689in}}%
\pgfpathlineto{\pgfqpoint{2.355457in}{2.510297in}}%
\pgfpathlineto{\pgfqpoint{2.300442in}{2.494798in}}%
\pgfpathlineto{\pgfqpoint{2.244502in}{2.474823in}}%
\pgfpathlineto{\pgfqpoint{2.188824in}{2.508172in}}%
\pgfpathlineto{\pgfqpoint{2.134427in}{2.531703in}}%
\pgfpathlineto{\pgfqpoint{2.078568in}{2.431374in}}%
\pgfpathlineto{\pgfqpoint{2.021509in}{2.463064in}}%
\pgfpathlineto{\pgfqpoint{1.963640in}{2.398288in}}%
\pgfpathlineto{\pgfqpoint{1.905453in}{2.420577in}}%
\pgfpathlineto{\pgfqpoint{1.849942in}{2.593795in}}%
\pgfpathlineto{\pgfqpoint{1.795378in}{2.508748in}}%
\pgfpathlineto{\pgfqpoint{1.747243in}{3.074592in}}%
\pgfpathlineto{\pgfqpoint{1.705741in}{3.074584in}}%
\pgfpathlineto{\pgfqpoint{1.664297in}{3.076473in}}%
\pgfpathlineto{\pgfqpoint{1.622829in}{3.076781in}}%
\pgfpathlineto{\pgfqpoint{1.581396in}{3.077315in}}%
\pgfpathlineto{\pgfqpoint{1.539972in}{3.077723in}}%
\pgfpathlineto{\pgfqpoint{1.498540in}{3.076889in}}%
\pgfpathlineto{\pgfqpoint{1.457070in}{3.074954in}}%
\pgfpathlineto{\pgfqpoint{1.415629in}{3.077147in}}%
\pgfpathlineto{\pgfqpoint{1.374186in}{3.074872in}}%
\pgfpathlineto{\pgfqpoint{1.332721in}{3.075079in}}%
\pgfpathlineto{\pgfqpoint{1.291280in}{3.076669in}}%
\pgfpathlineto{\pgfqpoint{1.249856in}{3.077450in}}%
\pgfpathlineto{\pgfqpoint{1.208458in}{3.080244in}}%
\pgfpathlineto{\pgfqpoint{1.167091in}{3.080759in}}%
\pgfpathlineto{\pgfqpoint{1.125752in}{3.082011in}}%
\pgfpathlineto{\pgfqpoint{1.084375in}{3.077983in}}%
\pgfpathlineto{\pgfqpoint{1.042993in}{3.080930in}}%
\pgfpathlineto{\pgfqpoint{1.001616in}{2.946160in}}%
\pgfpathclose%
\pgfusepath{fill}%
\end{pgfscope}%
\begin{pgfscope}%
\pgfsetbuttcap%
\pgfsetroundjoin%
\definecolor{currentfill}{rgb}{0.000000,0.000000,0.000000}%
\pgfsetfillcolor{currentfill}%
\pgfsetlinewidth{0.803000pt}%
\definecolor{currentstroke}{rgb}{0.000000,0.000000,0.000000}%
\pgfsetstrokecolor{currentstroke}%
\pgfsetdash{}{0pt}%
\pgfsys@defobject{currentmarker}{\pgfqpoint{0.000000in}{-0.048611in}}{\pgfqpoint{0.000000in}{0.000000in}}{%
\pgfpathmoveto{\pgfqpoint{0.000000in}{0.000000in}}%
\pgfpathlineto{\pgfqpoint{0.000000in}{-0.048611in}}%
\pgfusepath{stroke,fill}%
}%
\begin{pgfscope}%
\pgfsys@transformshift{0.981539in}{0.773588in}%
\pgfsys@useobject{currentmarker}{}%
\end{pgfscope}%
\end{pgfscope}%
\begin{pgfscope}%
\definecolor{textcolor}{rgb}{0.000000,0.000000,0.000000}%
\pgfsetstrokecolor{textcolor}%
\pgfsetfillcolor{textcolor}%
\pgftext[x=0.981539in,y=0.676366in,,top]{\color{textcolor}\rmfamily\fontsize{10.000000}{12.000000}\selectfont \(\displaystyle {0}\)}%
\end{pgfscope}%
\begin{pgfscope}%
\pgfsetbuttcap%
\pgfsetroundjoin%
\definecolor{currentfill}{rgb}{0.000000,0.000000,0.000000}%
\pgfsetfillcolor{currentfill}%
\pgfsetlinewidth{0.803000pt}%
\definecolor{currentstroke}{rgb}{0.000000,0.000000,0.000000}%
\pgfsetstrokecolor{currentstroke}%
\pgfsetdash{}{0pt}%
\pgfsys@defobject{currentmarker}{\pgfqpoint{0.000000in}{-0.048611in}}{\pgfqpoint{0.000000in}{0.000000in}}{%
\pgfpathmoveto{\pgfqpoint{0.000000in}{0.000000in}}%
\pgfpathlineto{\pgfqpoint{0.000000in}{-0.048611in}}%
\pgfusepath{stroke,fill}%
}%
\begin{pgfscope}%
\pgfsys@transformshift{1.638927in}{0.773588in}%
\pgfsys@useobject{currentmarker}{}%
\end{pgfscope}%
\end{pgfscope}%
\begin{pgfscope}%
\definecolor{textcolor}{rgb}{0.000000,0.000000,0.000000}%
\pgfsetstrokecolor{textcolor}%
\pgfsetfillcolor{textcolor}%
\pgftext[x=1.638927in,y=0.676366in,,top]{\color{textcolor}\rmfamily\fontsize{10.000000}{12.000000}\selectfont \(\displaystyle {2000}\)}%
\end{pgfscope}%
\begin{pgfscope}%
\pgfsetbuttcap%
\pgfsetroundjoin%
\definecolor{currentfill}{rgb}{0.000000,0.000000,0.000000}%
\pgfsetfillcolor{currentfill}%
\pgfsetlinewidth{0.803000pt}%
\definecolor{currentstroke}{rgb}{0.000000,0.000000,0.000000}%
\pgfsetstrokecolor{currentstroke}%
\pgfsetdash{}{0pt}%
\pgfsys@defobject{currentmarker}{\pgfqpoint{0.000000in}{-0.048611in}}{\pgfqpoint{0.000000in}{0.000000in}}{%
\pgfpathmoveto{\pgfqpoint{0.000000in}{0.000000in}}%
\pgfpathlineto{\pgfqpoint{0.000000in}{-0.048611in}}%
\pgfusepath{stroke,fill}%
}%
\begin{pgfscope}%
\pgfsys@transformshift{2.296314in}{0.773588in}%
\pgfsys@useobject{currentmarker}{}%
\end{pgfscope}%
\end{pgfscope}%
\begin{pgfscope}%
\definecolor{textcolor}{rgb}{0.000000,0.000000,0.000000}%
\pgfsetstrokecolor{textcolor}%
\pgfsetfillcolor{textcolor}%
\pgftext[x=2.296314in,y=0.676366in,,top]{\color{textcolor}\rmfamily\fontsize{10.000000}{12.000000}\selectfont \(\displaystyle {4000}\)}%
\end{pgfscope}%
\begin{pgfscope}%
\pgfsetbuttcap%
\pgfsetroundjoin%
\definecolor{currentfill}{rgb}{0.000000,0.000000,0.000000}%
\pgfsetfillcolor{currentfill}%
\pgfsetlinewidth{0.803000pt}%
\definecolor{currentstroke}{rgb}{0.000000,0.000000,0.000000}%
\pgfsetstrokecolor{currentstroke}%
\pgfsetdash{}{0pt}%
\pgfsys@defobject{currentmarker}{\pgfqpoint{0.000000in}{-0.048611in}}{\pgfqpoint{0.000000in}{0.000000in}}{%
\pgfpathmoveto{\pgfqpoint{0.000000in}{0.000000in}}%
\pgfpathlineto{\pgfqpoint{0.000000in}{-0.048611in}}%
\pgfusepath{stroke,fill}%
}%
\begin{pgfscope}%
\pgfsys@transformshift{2.953702in}{0.773588in}%
\pgfsys@useobject{currentmarker}{}%
\end{pgfscope}%
\end{pgfscope}%
\begin{pgfscope}%
\definecolor{textcolor}{rgb}{0.000000,0.000000,0.000000}%
\pgfsetstrokecolor{textcolor}%
\pgfsetfillcolor{textcolor}%
\pgftext[x=2.953702in,y=0.676366in,,top]{\color{textcolor}\rmfamily\fontsize{10.000000}{12.000000}\selectfont \(\displaystyle {6000}\)}%
\end{pgfscope}%
\begin{pgfscope}%
\pgfsetbuttcap%
\pgfsetroundjoin%
\definecolor{currentfill}{rgb}{0.000000,0.000000,0.000000}%
\pgfsetfillcolor{currentfill}%
\pgfsetlinewidth{0.803000pt}%
\definecolor{currentstroke}{rgb}{0.000000,0.000000,0.000000}%
\pgfsetstrokecolor{currentstroke}%
\pgfsetdash{}{0pt}%
\pgfsys@defobject{currentmarker}{\pgfqpoint{0.000000in}{-0.048611in}}{\pgfqpoint{0.000000in}{0.000000in}}{%
\pgfpathmoveto{\pgfqpoint{0.000000in}{0.000000in}}%
\pgfpathlineto{\pgfqpoint{0.000000in}{-0.048611in}}%
\pgfusepath{stroke,fill}%
}%
\begin{pgfscope}%
\pgfsys@transformshift{3.611089in}{0.773588in}%
\pgfsys@useobject{currentmarker}{}%
\end{pgfscope}%
\end{pgfscope}%
\begin{pgfscope}%
\definecolor{textcolor}{rgb}{0.000000,0.000000,0.000000}%
\pgfsetstrokecolor{textcolor}%
\pgfsetfillcolor{textcolor}%
\pgftext[x=3.611089in,y=0.676366in,,top]{\color{textcolor}\rmfamily\fontsize{10.000000}{12.000000}\selectfont \(\displaystyle {8000}\)}%
\end{pgfscope}%
\begin{pgfscope}%
\pgfsetbuttcap%
\pgfsetroundjoin%
\definecolor{currentfill}{rgb}{0.000000,0.000000,0.000000}%
\pgfsetfillcolor{currentfill}%
\pgfsetlinewidth{0.803000pt}%
\definecolor{currentstroke}{rgb}{0.000000,0.000000,0.000000}%
\pgfsetstrokecolor{currentstroke}%
\pgfsetdash{}{0pt}%
\pgfsys@defobject{currentmarker}{\pgfqpoint{0.000000in}{-0.048611in}}{\pgfqpoint{0.000000in}{0.000000in}}{%
\pgfpathmoveto{\pgfqpoint{0.000000in}{0.000000in}}%
\pgfpathlineto{\pgfqpoint{0.000000in}{-0.048611in}}%
\pgfusepath{stroke,fill}%
}%
\begin{pgfscope}%
\pgfsys@transformshift{4.268476in}{0.773588in}%
\pgfsys@useobject{currentmarker}{}%
\end{pgfscope}%
\end{pgfscope}%
\begin{pgfscope}%
\definecolor{textcolor}{rgb}{0.000000,0.000000,0.000000}%
\pgfsetstrokecolor{textcolor}%
\pgfsetfillcolor{textcolor}%
\pgftext[x=4.268476in,y=0.676366in,,top]{\color{textcolor}\rmfamily\fontsize{10.000000}{12.000000}\selectfont \(\displaystyle {10000}\)}%
\end{pgfscope}%
\begin{pgfscope}%
\pgfsetbuttcap%
\pgfsetroundjoin%
\definecolor{currentfill}{rgb}{0.000000,0.000000,0.000000}%
\pgfsetfillcolor{currentfill}%
\pgfsetlinewidth{0.803000pt}%
\definecolor{currentstroke}{rgb}{0.000000,0.000000,0.000000}%
\pgfsetstrokecolor{currentstroke}%
\pgfsetdash{}{0pt}%
\pgfsys@defobject{currentmarker}{\pgfqpoint{0.000000in}{-0.048611in}}{\pgfqpoint{0.000000in}{0.000000in}}{%
\pgfpathmoveto{\pgfqpoint{0.000000in}{0.000000in}}%
\pgfpathlineto{\pgfqpoint{0.000000in}{-0.048611in}}%
\pgfusepath{stroke,fill}%
}%
\begin{pgfscope}%
\pgfsys@transformshift{4.925864in}{0.773588in}%
\pgfsys@useobject{currentmarker}{}%
\end{pgfscope}%
\end{pgfscope}%
\begin{pgfscope}%
\definecolor{textcolor}{rgb}{0.000000,0.000000,0.000000}%
\pgfsetstrokecolor{textcolor}%
\pgfsetfillcolor{textcolor}%
\pgftext[x=4.925864in,y=0.676366in,,top]{\color{textcolor}\rmfamily\fontsize{10.000000}{12.000000}\selectfont \(\displaystyle {12000}\)}%
\end{pgfscope}%
\begin{pgfscope}%
\pgfsetbuttcap%
\pgfsetroundjoin%
\definecolor{currentfill}{rgb}{0.000000,0.000000,0.000000}%
\pgfsetfillcolor{currentfill}%
\pgfsetlinewidth{0.803000pt}%
\definecolor{currentstroke}{rgb}{0.000000,0.000000,0.000000}%
\pgfsetstrokecolor{currentstroke}%
\pgfsetdash{}{0pt}%
\pgfsys@defobject{currentmarker}{\pgfqpoint{0.000000in}{-0.048611in}}{\pgfqpoint{0.000000in}{0.000000in}}{%
\pgfpathmoveto{\pgfqpoint{0.000000in}{0.000000in}}%
\pgfpathlineto{\pgfqpoint{0.000000in}{-0.048611in}}%
\pgfusepath{stroke,fill}%
}%
\begin{pgfscope}%
\pgfsys@transformshift{5.583251in}{0.773588in}%
\pgfsys@useobject{currentmarker}{}%
\end{pgfscope}%
\end{pgfscope}%
\begin{pgfscope}%
\definecolor{textcolor}{rgb}{0.000000,0.000000,0.000000}%
\pgfsetstrokecolor{textcolor}%
\pgfsetfillcolor{textcolor}%
\pgftext[x=5.583251in,y=0.676366in,,top]{\color{textcolor}\rmfamily\fontsize{10.000000}{12.000000}\selectfont \(\displaystyle {14000}\)}%
\end{pgfscope}%
\begin{pgfscope}%
\definecolor{textcolor}{rgb}{0.000000,0.000000,0.000000}%
\pgfsetstrokecolor{textcolor}%
\pgfsetfillcolor{textcolor}%
\pgftext[x=3.203750in,y=0.497354in,,top]{\color{textcolor}\rmfamily\fontsize{10.000000}{12.000000}\selectfont Time (milliseconds)}%
\end{pgfscope}%
\begin{pgfscope}%
\pgfsetbuttcap%
\pgfsetroundjoin%
\definecolor{currentfill}{rgb}{0.000000,0.000000,0.000000}%
\pgfsetfillcolor{currentfill}%
\pgfsetlinewidth{0.803000pt}%
\definecolor{currentstroke}{rgb}{0.000000,0.000000,0.000000}%
\pgfsetstrokecolor{currentstroke}%
\pgfsetdash{}{0pt}%
\pgfsys@defobject{currentmarker}{\pgfqpoint{-0.048611in}{0.000000in}}{\pgfqpoint{-0.000000in}{0.000000in}}{%
\pgfpathmoveto{\pgfqpoint{-0.000000in}{0.000000in}}%
\pgfpathlineto{\pgfqpoint{-0.048611in}{0.000000in}}%
\pgfusepath{stroke,fill}%
}%
\begin{pgfscope}%
\pgfsys@transformshift{0.781402in}{0.773588in}%
\pgfsys@useobject{currentmarker}{}%
\end{pgfscope}%
\end{pgfscope}%
\begin{pgfscope}%
\definecolor{textcolor}{rgb}{0.000000,0.000000,0.000000}%
\pgfsetstrokecolor{textcolor}%
\pgfsetfillcolor{textcolor}%
\pgftext[x=0.614736in, y=0.725363in, left, base]{\color{textcolor}\rmfamily\fontsize{10.000000}{12.000000}\selectfont \(\displaystyle {0}\)}%
\end{pgfscope}%
\begin{pgfscope}%
\pgfsetbuttcap%
\pgfsetroundjoin%
\definecolor{currentfill}{rgb}{0.000000,0.000000,0.000000}%
\pgfsetfillcolor{currentfill}%
\pgfsetlinewidth{0.803000pt}%
\definecolor{currentstroke}{rgb}{0.000000,0.000000,0.000000}%
\pgfsetstrokecolor{currentstroke}%
\pgfsetdash{}{0pt}%
\pgfsys@defobject{currentmarker}{\pgfqpoint{-0.048611in}{0.000000in}}{\pgfqpoint{-0.000000in}{0.000000in}}{%
\pgfpathmoveto{\pgfqpoint{-0.000000in}{0.000000in}}%
\pgfpathlineto{\pgfqpoint{-0.048611in}{0.000000in}}%
\pgfusepath{stroke,fill}%
}%
\begin{pgfscope}%
\pgfsys@transformshift{0.781402in}{1.934933in}%
\pgfsys@useobject{currentmarker}{}%
\end{pgfscope}%
\end{pgfscope}%
\begin{pgfscope}%
\definecolor{textcolor}{rgb}{0.000000,0.000000,0.000000}%
\pgfsetstrokecolor{textcolor}%
\pgfsetfillcolor{textcolor}%
\pgftext[x=0.614736in, y=1.886708in, left, base]{\color{textcolor}\rmfamily\fontsize{10.000000}{12.000000}\selectfont \(\displaystyle {2}\)}%
\end{pgfscope}%
\begin{pgfscope}%
\pgfsetbuttcap%
\pgfsetroundjoin%
\definecolor{currentfill}{rgb}{0.000000,0.000000,0.000000}%
\pgfsetfillcolor{currentfill}%
\pgfsetlinewidth{0.803000pt}%
\definecolor{currentstroke}{rgb}{0.000000,0.000000,0.000000}%
\pgfsetstrokecolor{currentstroke}%
\pgfsetdash{}{0pt}%
\pgfsys@defobject{currentmarker}{\pgfqpoint{-0.048611in}{0.000000in}}{\pgfqpoint{-0.000000in}{0.000000in}}{%
\pgfpathmoveto{\pgfqpoint{-0.000000in}{0.000000in}}%
\pgfpathlineto{\pgfqpoint{-0.048611in}{0.000000in}}%
\pgfusepath{stroke,fill}%
}%
\begin{pgfscope}%
\pgfsys@transformshift{0.781402in}{3.096277in}%
\pgfsys@useobject{currentmarker}{}%
\end{pgfscope}%
\end{pgfscope}%
\begin{pgfscope}%
\definecolor{textcolor}{rgb}{0.000000,0.000000,0.000000}%
\pgfsetstrokecolor{textcolor}%
\pgfsetfillcolor{textcolor}%
\pgftext[x=0.614736in, y=3.048052in, left, base]{\color{textcolor}\rmfamily\fontsize{10.000000}{12.000000}\selectfont \(\displaystyle {4}\)}%
\end{pgfscope}%
\begin{pgfscope}%
\pgfsetbuttcap%
\pgfsetroundjoin%
\definecolor{currentfill}{rgb}{0.000000,0.000000,0.000000}%
\pgfsetfillcolor{currentfill}%
\pgfsetlinewidth{0.803000pt}%
\definecolor{currentstroke}{rgb}{0.000000,0.000000,0.000000}%
\pgfsetstrokecolor{currentstroke}%
\pgfsetdash{}{0pt}%
\pgfsys@defobject{currentmarker}{\pgfqpoint{-0.048611in}{0.000000in}}{\pgfqpoint{-0.000000in}{0.000000in}}{%
\pgfpathmoveto{\pgfqpoint{-0.000000in}{0.000000in}}%
\pgfpathlineto{\pgfqpoint{-0.048611in}{0.000000in}}%
\pgfusepath{stroke,fill}%
}%
\begin{pgfscope}%
\pgfsys@transformshift{0.781402in}{4.257622in}%
\pgfsys@useobject{currentmarker}{}%
\end{pgfscope}%
\end{pgfscope}%
\begin{pgfscope}%
\definecolor{textcolor}{rgb}{0.000000,0.000000,0.000000}%
\pgfsetstrokecolor{textcolor}%
\pgfsetfillcolor{textcolor}%
\pgftext[x=0.614736in, y=4.209396in, left, base]{\color{textcolor}\rmfamily\fontsize{10.000000}{12.000000}\selectfont \(\displaystyle {6}\)}%
\end{pgfscope}%
\begin{pgfscope}%
\pgfsetbuttcap%
\pgfsetroundjoin%
\definecolor{currentfill}{rgb}{0.000000,0.000000,0.000000}%
\pgfsetfillcolor{currentfill}%
\pgfsetlinewidth{0.803000pt}%
\definecolor{currentstroke}{rgb}{0.000000,0.000000,0.000000}%
\pgfsetstrokecolor{currentstroke}%
\pgfsetdash{}{0pt}%
\pgfsys@defobject{currentmarker}{\pgfqpoint{-0.048611in}{0.000000in}}{\pgfqpoint{-0.000000in}{0.000000in}}{%
\pgfpathmoveto{\pgfqpoint{-0.000000in}{0.000000in}}%
\pgfpathlineto{\pgfqpoint{-0.048611in}{0.000000in}}%
\pgfusepath{stroke,fill}%
}%
\begin{pgfscope}%
\pgfsys@transformshift{0.781402in}{5.418966in}%
\pgfsys@useobject{currentmarker}{}%
\end{pgfscope}%
\end{pgfscope}%
\begin{pgfscope}%
\definecolor{textcolor}{rgb}{0.000000,0.000000,0.000000}%
\pgfsetstrokecolor{textcolor}%
\pgfsetfillcolor{textcolor}%
\pgftext[x=0.614736in, y=5.370741in, left, base]{\color{textcolor}\rmfamily\fontsize{10.000000}{12.000000}\selectfont \(\displaystyle {8}\)}%
\end{pgfscope}%
\begin{pgfscope}%
\definecolor{textcolor}{rgb}{0.000000,0.000000,0.000000}%
\pgfsetstrokecolor{textcolor}%
\pgfsetfillcolor{textcolor}%
\pgftext[x=0.559180in,y=3.481148in,,bottom,rotate=90.000000]{\color{textcolor}\rmfamily\fontsize{10.000000}{12.000000}\selectfont Throughput (million operations/second)}%
\end{pgfscope}%
\begin{pgfscope}%
\pgfpathrectangle{\pgfqpoint{0.781402in}{0.773588in}}{\pgfqpoint{4.844695in}{5.415119in}}%
\pgfusepath{clip}%
\pgfsetrectcap%
\pgfsetroundjoin%
\pgfsetlinewidth{1.505625pt}%
\definecolor{currentstroke}{rgb}{0.000000,0.000000,1.000000}%
\pgfsetstrokecolor{currentstroke}%
\pgfsetdash{}{0pt}%
\pgfpathmoveto{\pgfqpoint{0.983123in}{0.773588in}}%
\pgfpathlineto{\pgfqpoint{0.983123in}{6.188708in}}%
\pgfusepath{stroke}%
\end{pgfscope}%
\begin{pgfscope}%
\pgfpathrectangle{\pgfqpoint{0.781402in}{0.773588in}}{\pgfqpoint{4.844695in}{5.415119in}}%
\pgfusepath{clip}%
\pgfsetrectcap%
\pgfsetroundjoin%
\pgfsetlinewidth{1.505625pt}%
\definecolor{currentstroke}{rgb}{0.750000,0.750000,0.000000}%
\pgfsetstrokecolor{currentstroke}%
\pgfsetdash{}{0pt}%
\pgfpathmoveto{\pgfqpoint{1.766893in}{0.773588in}}%
\pgfpathlineto{\pgfqpoint{1.766893in}{6.188708in}}%
\pgfusepath{stroke}%
\end{pgfscope}%
\begin{pgfscope}%
\pgfpathrectangle{\pgfqpoint{0.781402in}{0.773588in}}{\pgfqpoint{4.844695in}{5.415119in}}%
\pgfusepath{clip}%
\pgfsetrectcap%
\pgfsetroundjoin%
\pgfsetlinewidth{1.505625pt}%
\definecolor{currentstroke}{rgb}{0.750000,0.000000,0.750000}%
\pgfsetstrokecolor{currentstroke}%
\pgfsetdash{}{0pt}%
\pgfpathmoveto{\pgfqpoint{3.253738in}{0.773588in}}%
\pgfpathlineto{\pgfqpoint{3.253738in}{6.188708in}}%
\pgfusepath{stroke}%
\end{pgfscope}%
\begin{pgfscope}%
\pgfpathrectangle{\pgfqpoint{0.781402in}{0.773588in}}{\pgfqpoint{4.844695in}{5.415119in}}%
\pgfusepath{clip}%
\pgfsetrectcap%
\pgfsetroundjoin%
\pgfsetlinewidth{1.505625pt}%
\definecolor{currentstroke}{rgb}{1.000000,0.000000,0.000000}%
\pgfsetstrokecolor{currentstroke}%
\pgfsetdash{}{0pt}%
\pgfpathmoveto{\pgfqpoint{3.766609in}{0.773588in}}%
\pgfpathlineto{\pgfqpoint{3.766609in}{6.188708in}}%
\pgfusepath{stroke}%
\end{pgfscope}%
\begin{pgfscope}%
\pgfpathrectangle{\pgfqpoint{0.781402in}{0.773588in}}{\pgfqpoint{4.844695in}{5.415119in}}%
\pgfusepath{clip}%
\pgfsetrectcap%
\pgfsetroundjoin%
\pgfsetlinewidth{1.505625pt}%
\definecolor{currentstroke}{rgb}{0.000000,0.500000,0.000000}%
\pgfsetstrokecolor{currentstroke}%
\pgfsetdash{}{0pt}%
\pgfpathmoveto{\pgfqpoint{4.001288in}{0.773588in}}%
\pgfpathlineto{\pgfqpoint{4.001288in}{6.188708in}}%
\pgfusepath{stroke}%
\end{pgfscope}%
\begin{pgfscope}%
\pgfpathrectangle{\pgfqpoint{0.781402in}{0.773588in}}{\pgfqpoint{4.844695in}{5.415119in}}%
\pgfusepath{clip}%
\pgfsetrectcap%
\pgfsetroundjoin%
\pgfsetlinewidth{1.505625pt}%
\definecolor{currentstroke}{rgb}{0.000000,0.750000,0.750000}%
\pgfsetstrokecolor{currentstroke}%
\pgfsetdash{}{0pt}%
\pgfpathmoveto{\pgfqpoint{4.500409in}{0.773588in}}%
\pgfpathlineto{\pgfqpoint{4.500409in}{6.188708in}}%
\pgfusepath{stroke}%
\end{pgfscope}%
\begin{pgfscope}%
\pgfpathrectangle{\pgfqpoint{0.781402in}{0.773588in}}{\pgfqpoint{4.844695in}{5.415119in}}%
\pgfusepath{clip}%
\pgfsetrectcap%
\pgfsetroundjoin%
\pgfsetlinewidth{1.505625pt}%
\definecolor{currentstroke}{rgb}{0.750000,0.000000,0.750000}%
\pgfsetstrokecolor{currentstroke}%
\pgfsetdash{}{0pt}%
\pgfpathmoveto{\pgfqpoint{4.500413in}{0.773588in}}%
\pgfpathlineto{\pgfqpoint{4.500413in}{6.188708in}}%
\pgfusepath{stroke}%
\end{pgfscope}%
\begin{pgfscope}%
\pgfsetrectcap%
\pgfsetmiterjoin%
\pgfsetlinewidth{0.803000pt}%
\definecolor{currentstroke}{rgb}{0.000000,0.000000,0.000000}%
\pgfsetstrokecolor{currentstroke}%
\pgfsetdash{}{0pt}%
\pgfpathmoveto{\pgfqpoint{0.781402in}{0.773588in}}%
\pgfpathlineto{\pgfqpoint{0.781402in}{6.188708in}}%
\pgfusepath{stroke}%
\end{pgfscope}%
\begin{pgfscope}%
\pgfsetrectcap%
\pgfsetmiterjoin%
\pgfsetlinewidth{0.803000pt}%
\definecolor{currentstroke}{rgb}{0.000000,0.000000,0.000000}%
\pgfsetstrokecolor{currentstroke}%
\pgfsetdash{}{0pt}%
\pgfpathmoveto{\pgfqpoint{5.626098in}{0.773588in}}%
\pgfpathlineto{\pgfqpoint{5.626098in}{6.188708in}}%
\pgfusepath{stroke}%
\end{pgfscope}%
\begin{pgfscope}%
\pgfsetrectcap%
\pgfsetmiterjoin%
\pgfsetlinewidth{0.803000pt}%
\definecolor{currentstroke}{rgb}{0.000000,0.000000,0.000000}%
\pgfsetstrokecolor{currentstroke}%
\pgfsetdash{}{0pt}%
\pgfpathmoveto{\pgfqpoint{0.781402in}{0.773588in}}%
\pgfpathlineto{\pgfqpoint{5.626098in}{0.773588in}}%
\pgfusepath{stroke}%
\end{pgfscope}%
\begin{pgfscope}%
\pgfsetrectcap%
\pgfsetmiterjoin%
\pgfsetlinewidth{0.803000pt}%
\definecolor{currentstroke}{rgb}{0.000000,0.000000,0.000000}%
\pgfsetstrokecolor{currentstroke}%
\pgfsetdash{}{0pt}%
\pgfpathmoveto{\pgfqpoint{0.781402in}{6.188708in}}%
\pgfpathlineto{\pgfqpoint{5.626098in}{6.188708in}}%
\pgfusepath{stroke}%
\end{pgfscope}%
\begin{pgfscope}%
\pgfsetbuttcap%
\pgfsetmiterjoin%
\definecolor{currentfill}{rgb}{1.000000,1.000000,1.000000}%
\pgfsetfillcolor{currentfill}%
\pgfsetfillopacity{0.800000}%
\pgfsetlinewidth{1.003750pt}%
\definecolor{currentstroke}{rgb}{0.800000,0.800000,0.800000}%
\pgfsetstrokecolor{currentstroke}%
\pgfsetstrokeopacity{0.800000}%
\pgfsetdash{}{0pt}%
\pgfpathmoveto{\pgfqpoint{0.878625in}{3.559851in}}%
\pgfpathlineto{\pgfqpoint{3.799925in}{3.559851in}}%
\pgfpathquadraticcurveto{\pgfqpoint{3.827703in}{3.559851in}}{\pgfqpoint{3.827703in}{3.587628in}}%
\pgfpathlineto{\pgfqpoint{3.827703in}{6.091486in}}%
\pgfpathquadraticcurveto{\pgfqpoint{3.827703in}{6.119263in}}{\pgfqpoint{3.799925in}{6.119263in}}%
\pgfpathlineto{\pgfqpoint{0.878625in}{6.119263in}}%
\pgfpathquadraticcurveto{\pgfqpoint{0.850847in}{6.119263in}}{\pgfqpoint{0.850847in}{6.091486in}}%
\pgfpathlineto{\pgfqpoint{0.850847in}{3.587628in}}%
\pgfpathquadraticcurveto{\pgfqpoint{0.850847in}{3.559851in}}{\pgfqpoint{0.878625in}{3.559851in}}%
\pgfpathclose%
\pgfusepath{stroke,fill}%
\end{pgfscope}%
\begin{pgfscope}%
\pgfsetrectcap%
\pgfsetroundjoin%
\pgfsetlinewidth{1.505625pt}%
\definecolor{currentstroke}{rgb}{0.000000,0.000000,1.000000}%
\pgfsetstrokecolor{currentstroke}%
\pgfsetdash{}{0pt}%
\pgfpathmoveto{\pgfqpoint{0.906402in}{6.015097in}}%
\pgfpathlineto{\pgfqpoint{1.184180in}{6.015097in}}%
\pgfusepath{stroke}%
\end{pgfscope}%
\begin{pgfscope}%
\definecolor{textcolor}{rgb}{0.000000,0.000000,0.000000}%
\pgfsetstrokecolor{textcolor}%
\pgfsetfillcolor{textcolor}%
\pgftext[x=1.295291in,y=5.966486in,left,base]{\color{textcolor}\rmfamily\fontsize{10.000000}{12.000000}\selectfont Started migration}%
\end{pgfscope}%
\begin{pgfscope}%
\pgfsetrectcap%
\pgfsetroundjoin%
\pgfsetlinewidth{1.505625pt}%
\definecolor{currentstroke}{rgb}{0.750000,0.750000,0.000000}%
\pgfsetstrokecolor{currentstroke}%
\pgfsetdash{}{0pt}%
\pgfpathmoveto{\pgfqpoint{0.906402in}{5.821424in}}%
\pgfpathlineto{\pgfqpoint{1.184180in}{5.821424in}}%
\pgfusepath{stroke}%
\end{pgfscope}%
\begin{pgfscope}%
\definecolor{textcolor}{rgb}{0.000000,0.000000,0.000000}%
\pgfsetstrokecolor{textcolor}%
\pgfsetfillcolor{textcolor}%
\pgftext[x=1.295291in,y=5.772813in,left,base]{\color{textcolor}\rmfamily\fontsize{10.000000}{12.000000}\selectfont Started prefill writes}%
\end{pgfscope}%
\begin{pgfscope}%
\pgfsetrectcap%
\pgfsetroundjoin%
\pgfsetlinewidth{1.505625pt}%
\definecolor{currentstroke}{rgb}{0.750000,0.000000,0.750000}%
\pgfsetstrokecolor{currentstroke}%
\pgfsetdash{}{0pt}%
\pgfpathmoveto{\pgfqpoint{0.906402in}{5.627751in}}%
\pgfpathlineto{\pgfqpoint{1.184180in}{5.627751in}}%
\pgfusepath{stroke}%
\end{pgfscope}%
\begin{pgfscope}%
\definecolor{textcolor}{rgb}{0.000000,0.000000,0.000000}%
\pgfsetstrokecolor{textcolor}%
\pgfsetfillcolor{textcolor}%
\pgftext[x=1.295291in,y=5.579140in,left,base]{\color{textcolor}\rmfamily\fontsize{10.000000}{12.000000}\selectfont Finished prefill writes}%
\end{pgfscope}%
\begin{pgfscope}%
\pgfsetrectcap%
\pgfsetroundjoin%
\pgfsetlinewidth{1.505625pt}%
\definecolor{currentstroke}{rgb}{1.000000,0.000000,0.000000}%
\pgfsetstrokecolor{currentstroke}%
\pgfsetdash{}{0pt}%
\pgfpathmoveto{\pgfqpoint{0.906402in}{5.434078in}}%
\pgfpathlineto{\pgfqpoint{1.184180in}{5.434078in}}%
\pgfusepath{stroke}%
\end{pgfscope}%
\begin{pgfscope}%
\definecolor{textcolor}{rgb}{0.000000,0.000000,0.000000}%
\pgfsetstrokecolor{textcolor}%
\pgfsetfillcolor{textcolor}%
\pgftext[x=1.295291in,y=5.385467in,left,base]{\color{textcolor}\rmfamily\fontsize{10.000000}{12.000000}\selectfont Transferred ownership to the destination}%
\end{pgfscope}%
\begin{pgfscope}%
\pgfsetrectcap%
\pgfsetroundjoin%
\pgfsetlinewidth{1.505625pt}%
\definecolor{currentstroke}{rgb}{0.000000,0.500000,0.000000}%
\pgfsetstrokecolor{currentstroke}%
\pgfsetdash{}{0pt}%
\pgfpathmoveto{\pgfqpoint{0.906402in}{5.240406in}}%
\pgfpathlineto{\pgfqpoint{1.184180in}{5.240406in}}%
\pgfusepath{stroke}%
\end{pgfscope}%
\begin{pgfscope}%
\definecolor{textcolor}{rgb}{0.000000,0.000000,0.000000}%
\pgfsetstrokecolor{textcolor}%
\pgfsetfillcolor{textcolor}%
\pgftext[x=1.295291in,y=5.191794in,left,base]{\color{textcolor}\rmfamily\fontsize{10.000000}{12.000000}\selectfont Started reading dirty pages}%
\end{pgfscope}%
\begin{pgfscope}%
\pgfsetrectcap%
\pgfsetroundjoin%
\pgfsetlinewidth{1.505625pt}%
\definecolor{currentstroke}{rgb}{0.000000,0.750000,0.750000}%
\pgfsetstrokecolor{currentstroke}%
\pgfsetdash{}{0pt}%
\pgfpathmoveto{\pgfqpoint{0.906402in}{5.046733in}}%
\pgfpathlineto{\pgfqpoint{1.184180in}{5.046733in}}%
\pgfusepath{stroke}%
\end{pgfscope}%
\begin{pgfscope}%
\definecolor{textcolor}{rgb}{0.000000,0.000000,0.000000}%
\pgfsetstrokecolor{textcolor}%
\pgfsetfillcolor{textcolor}%
\pgftext[x=1.295291in,y=4.998122in,left,base]{\color{textcolor}\rmfamily\fontsize{10.000000}{12.000000}\selectfont Finished reading dirty pages}%
\end{pgfscope}%
\begin{pgfscope}%
\pgfsetrectcap%
\pgfsetroundjoin%
\pgfsetlinewidth{1.505625pt}%
\definecolor{currentstroke}{rgb}{0.750000,0.000000,0.750000}%
\pgfsetstrokecolor{currentstroke}%
\pgfsetdash{}{0pt}%
\pgfpathmoveto{\pgfqpoint{0.906402in}{4.853060in}}%
\pgfpathlineto{\pgfqpoint{1.184180in}{4.853060in}}%
\pgfusepath{stroke}%
\end{pgfscope}%
\begin{pgfscope}%
\definecolor{textcolor}{rgb}{0.000000,0.000000,0.000000}%
\pgfsetstrokecolor{textcolor}%
\pgfsetfillcolor{textcolor}%
\pgftext[x=1.295291in,y=4.804449in,left,base]{\color{textcolor}\rmfamily\fontsize{10.000000}{12.000000}\selectfont Finished migration}%
\end{pgfscope}%
\begin{pgfscope}%
\pgfsetbuttcap%
\pgfsetmiterjoin%
\definecolor{currentfill}{rgb}{0.121569,0.466667,0.705882}%
\pgfsetfillcolor{currentfill}%
\pgfsetlinewidth{0.000000pt}%
\definecolor{currentstroke}{rgb}{0.000000,0.000000,0.000000}%
\pgfsetstrokecolor{currentstroke}%
\pgfsetstrokeopacity{0.000000}%
\pgfsetdash{}{0pt}%
\pgfpathmoveto{\pgfqpoint{0.906402in}{4.610776in}}%
\pgfpathlineto{\pgfqpoint{1.184180in}{4.610776in}}%
\pgfpathlineto{\pgfqpoint{1.184180in}{4.707998in}}%
\pgfpathlineto{\pgfqpoint{0.906402in}{4.707998in}}%
\pgfpathclose%
\pgfusepath{fill}%
\end{pgfscope}%
\begin{pgfscope}%
\definecolor{textcolor}{rgb}{0.000000,0.000000,0.000000}%
\pgfsetstrokecolor{textcolor}%
\pgfsetfillcolor{textcolor}%
\pgftext[x=1.295291in,y=4.610776in,left,base]{\color{textcolor}\rmfamily\fontsize{10.000000}{12.000000}\selectfont BF1 read at destination}%
\end{pgfscope}%
\begin{pgfscope}%
\pgfsetbuttcap%
\pgfsetmiterjoin%
\definecolor{currentfill}{rgb}{1.000000,0.498039,0.054902}%
\pgfsetfillcolor{currentfill}%
\pgfsetlinewidth{0.000000pt}%
\definecolor{currentstroke}{rgb}{0.000000,0.000000,0.000000}%
\pgfsetstrokecolor{currentstroke}%
\pgfsetstrokeopacity{0.000000}%
\pgfsetdash{}{0pt}%
\pgfpathmoveto{\pgfqpoint{0.906402in}{4.417103in}}%
\pgfpathlineto{\pgfqpoint{1.184180in}{4.417103in}}%
\pgfpathlineto{\pgfqpoint{1.184180in}{4.514326in}}%
\pgfpathlineto{\pgfqpoint{0.906402in}{4.514326in}}%
\pgfpathclose%
\pgfusepath{fill}%
\end{pgfscope}%
\begin{pgfscope}%
\definecolor{textcolor}{rgb}{0.000000,0.000000,0.000000}%
\pgfsetstrokecolor{textcolor}%
\pgfsetfillcolor{textcolor}%
\pgftext[x=1.295291in,y=4.417103in,left,base]{\color{textcolor}\rmfamily\fontsize{10.000000}{12.000000}\selectfont BF1 write at destination}%
\end{pgfscope}%
\begin{pgfscope}%
\pgfsetbuttcap%
\pgfsetmiterjoin%
\definecolor{currentfill}{rgb}{0.172549,0.627451,0.172549}%
\pgfsetfillcolor{currentfill}%
\pgfsetlinewidth{0.000000pt}%
\definecolor{currentstroke}{rgb}{0.000000,0.000000,0.000000}%
\pgfsetstrokecolor{currentstroke}%
\pgfsetstrokeopacity{0.000000}%
\pgfsetdash{}{0pt}%
\pgfpathmoveto{\pgfqpoint{0.906402in}{4.223431in}}%
\pgfpathlineto{\pgfqpoint{1.184180in}{4.223431in}}%
\pgfpathlineto{\pgfqpoint{1.184180in}{4.320653in}}%
\pgfpathlineto{\pgfqpoint{0.906402in}{4.320653in}}%
\pgfpathclose%
\pgfusepath{fill}%
\end{pgfscope}%
\begin{pgfscope}%
\definecolor{textcolor}{rgb}{0.000000,0.000000,0.000000}%
\pgfsetstrokecolor{textcolor}%
\pgfsetfillcolor{textcolor}%
\pgftext[x=1.295291in,y=4.223431in,left,base]{\color{textcolor}\rmfamily\fontsize{10.000000}{12.000000}\selectfont BF2 read at source}%
\end{pgfscope}%
\begin{pgfscope}%
\pgfsetbuttcap%
\pgfsetmiterjoin%
\definecolor{currentfill}{rgb}{0.839216,0.152941,0.156863}%
\pgfsetfillcolor{currentfill}%
\pgfsetlinewidth{0.000000pt}%
\definecolor{currentstroke}{rgb}{0.000000,0.000000,0.000000}%
\pgfsetstrokecolor{currentstroke}%
\pgfsetstrokeopacity{0.000000}%
\pgfsetdash{}{0pt}%
\pgfpathmoveto{\pgfqpoint{0.906402in}{4.029758in}}%
\pgfpathlineto{\pgfqpoint{1.184180in}{4.029758in}}%
\pgfpathlineto{\pgfqpoint{1.184180in}{4.126980in}}%
\pgfpathlineto{\pgfqpoint{0.906402in}{4.126980in}}%
\pgfpathclose%
\pgfusepath{fill}%
\end{pgfscope}%
\begin{pgfscope}%
\definecolor{textcolor}{rgb}{0.000000,0.000000,0.000000}%
\pgfsetstrokecolor{textcolor}%
\pgfsetfillcolor{textcolor}%
\pgftext[x=1.295291in,y=4.029758in,left,base]{\color{textcolor}\rmfamily\fontsize{10.000000}{12.000000}\selectfont BF2 write at source}%
\end{pgfscope}%
\begin{pgfscope}%
\pgfsetbuttcap%
\pgfsetmiterjoin%
\definecolor{currentfill}{rgb}{0.580392,0.403922,0.741176}%
\pgfsetfillcolor{currentfill}%
\pgfsetlinewidth{0.000000pt}%
\definecolor{currentstroke}{rgb}{0.000000,0.000000,0.000000}%
\pgfsetstrokecolor{currentstroke}%
\pgfsetstrokeopacity{0.000000}%
\pgfsetdash{}{0pt}%
\pgfpathmoveto{\pgfqpoint{0.906402in}{3.836085in}}%
\pgfpathlineto{\pgfqpoint{1.184180in}{3.836085in}}%
\pgfpathlineto{\pgfqpoint{1.184180in}{3.933307in}}%
\pgfpathlineto{\pgfqpoint{0.906402in}{3.933307in}}%
\pgfpathclose%
\pgfusepath{fill}%
\end{pgfscope}%
\begin{pgfscope}%
\definecolor{textcolor}{rgb}{0.000000,0.000000,0.000000}%
\pgfsetstrokecolor{textcolor}%
\pgfsetfillcolor{textcolor}%
\pgftext[x=1.295291in,y=3.836085in,left,base]{\color{textcolor}\rmfamily\fontsize{10.000000}{12.000000}\selectfont BF1 read at source}%
\end{pgfscope}%
\begin{pgfscope}%
\pgfsetbuttcap%
\pgfsetmiterjoin%
\definecolor{currentfill}{rgb}{0.549020,0.337255,0.294118}%
\pgfsetfillcolor{currentfill}%
\pgfsetlinewidth{0.000000pt}%
\definecolor{currentstroke}{rgb}{0.000000,0.000000,0.000000}%
\pgfsetstrokecolor{currentstroke}%
\pgfsetstrokeopacity{0.000000}%
\pgfsetdash{}{0pt}%
\pgfpathmoveto{\pgfqpoint{0.906402in}{3.642412in}}%
\pgfpathlineto{\pgfqpoint{1.184180in}{3.642412in}}%
\pgfpathlineto{\pgfqpoint{1.184180in}{3.739634in}}%
\pgfpathlineto{\pgfqpoint{0.906402in}{3.739634in}}%
\pgfpathclose%
\pgfusepath{fill}%
\end{pgfscope}%
\begin{pgfscope}%
\definecolor{textcolor}{rgb}{0.000000,0.000000,0.000000}%
\pgfsetstrokecolor{textcolor}%
\pgfsetfillcolor{textcolor}%
\pgftext[x=1.295291in,y=3.642412in,left,base]{\color{textcolor}\rmfamily\fontsize{10.000000}{12.000000}\selectfont BF1 write at source}%
\end{pgfscope}%
\end{pgfpicture}%
\makeatother%
\endgroup%

    \end{center}
    \caption{Migration timeline of a Bloom filter}
    \label{fig:bloomfilter}
\end{figure}

\section{Case study: generic objects}
\label{sec:evalgenericobj}

c++ map



% \TODO{this model encourages fixed objects, where you don't need to allocate repeatedly: bloom filter, hash table with fixed size values + ds's where access is local}
%\TODO{build a function as a service framework on this}
%\TODO{can this be used for other systems such as parallel processing
% based on actor models and message queues with efficient support for
%e.g. fan out}
% supporting read-only operations and even write operations with carefully
% created static buffers that can contain unsupported (those that allocate/deallocate)
% operations that the object has received after a call to initiate migration has
% already been made.
% big table can be implemented easily using Slope
% as opposed to the original methods where we only think about the movement of
% objects, here we think about movement of servers
